\documentclass[a4paper, 14pt]{extarticle}
\usepackage[utf8]{inputenc}
\usepackage[english, russian]{babel}
\usepackage[left=30mm, top=15mm, right=10mm, bottom=20mm, head=0mm, nofoot]{geometry}
\usepackage{amsmath, amssymb}
\usepackage{fontspec}
\usepackage{hyperref}
\usepackage{float}
\usepackage{titlesec}
\usepackage{tocloft} 

\usepackage{indentfirst}

%\geometry{showframe, showcrop, includeheadfoot=true}

\setmainfont{Times New Roman}
\setsansfont{Arial}
\setmonofont{Courier New}
\newfontfamily\cyrillicfont[Script=Cyrillic]{Times New Roman}
\newfontfamily\cyrillicfontsf[Script=Cyrillic]{Arial}
\newfontfamily\cyrillicfonttt[Script=Cyrillic]{Courier New}

\usepackage{polyglossia}
\setdefaultlanguage{russian}

\usepackage{minted}
\usepackage{caption}
\newenvironment{sourcecode}{\captionsetup{type=listing}}{}
\setminted[HTML]{linenos, fontsize=\small, breaklines=true, breakafter=>}
\setminted[CSS]{linenos, fontsize=\small, breaklines=true, breakafter=>}
\setminted[JS]{linenos, fontsize=\small, breaklines=true, breakafter=>}
\setminted[JSON]{linenos, fontsize=\small, breaklines=true, breakafter=>}

\renewcommand{\theFancyVerbLine}{\sffamily %стиль нумерации
\textcolor[rgb]{0.5,0.05,0.8}{\small
\arabic{FancyVerbLine}}}

\renewcommand{\cfttoctitlefont}{\hspace*{\fill}\large\bfseries}
\renewcommand{\cftaftertoctitle}{\hspace*{\fill}}

\cftsetpnumwidth{15pt}

\titleformat*{\section}{\Large\bfseries} %\large
\titleformat*{\subsection}{\normalsize\bfseries} %\normalsize

\makeatletter
\renewcommand{\l@section}{\@dottedtocline{1}{0em}{1.3em}}
\renewcommand{\l@subsection}{\@dottedtocline{1}{1em}{2.1em}}
\renewcommand{\l@subsubsection}{\@dottedtocline{1}{0em}{0em}}
\makeatother

\makeatletter
\renewcommand{\thesubsection}{\arabic{section}.\arabic{subsection}}
\renewcommand{\thesection}{\arabic{section}}
\makeatother

\newcommand{\lastsection}[1]{%
	\section*{\centering#1}%
	\addcontentsline{toc}{section}{#1}%
}

\addto\captionsrussian{
	\renewcommand{\listingscaption}{Исходный код}
}

\usepackage{graphicx}
\graphicspath{{img/}}
\DeclareGraphicsExtensions{.png}

\usepackage{pstricks}

\linespread{1.3}
\setlength{\parindent}{1.25cm}

\usepackage{fancyhdr}
\pagestyle{fancy}
\fancyhf{}
\fancyhead[C]{\thepage}
\renewcommand{\headrulewidth}{0pt}

\usepackage[
natbib		= true,
style		= gost-numeric,
sorting		= nyvt,
backend		= biber,
language	= autobib,
autolang	= other]{biblatex}
\addbibresource{bib/literature.bib}

\begin{document}
	\setcounter{page}{1}
\thispagestyle{empty}
\begin{center}
	МИНОБРНАУКИ РОССИИ 
		
	ФЕДЕРАЛЬНОЕ ГОСУДАРСТВЕННОЕ БЮДЖЕТНОЕ ОБРАЗОВАТЕЛЬНОЕ УЧРЕЖДЕНИЕ  
		
	ВЫСШЕГО ОБРАЗОВАНИЯ 
		
	«ВОРОНЕЖСКИЙ ГОСУДАРСТВЕННЫЙ УНИВЕРСИТЕТ»
		
	(ФГБОУ ВО «ВГУ») 
\end{center}
	
\begin{center}
	Математический факультет  
		
	Кафедра теории функций и геометрии
\end{center}
	
\hfill \break
	
\begin{center}
	\textbf{
		Использование современных веб-технологий при
		разработке удобного ПО для совместной работы с геометрическими объектами
	}
\end{center}
	
\hfill \break
	
\begin{center}
	Дипломная работа
		
	Специальность 01.05.01 Фундаментальные математика и механика 
		
	Специализация  Теория функций и приложения 
\end{center}
	
\hfill \break
	
\begin{flushleft}
	\begin{tabular}{l l l l}
		Зав. кафедрой &
		$\underset{\text{подпись}}{\underline{\hspace{0.15\textwidth}}}$ &
		д. ф.-м.н., проф. &
		Е.М. Семенов \quad 07.06.2023 \\
		&&& \\
		Обучающийся &
		$\underset{\text{подпись}}{\underline{\hspace{0.15\textwidth}}}$ &
		&
		А. Е. Зволинский \\
		&&& \\
		Руководитель &
		$\underset{\text{подпись}}{\underline{\hspace{0.15\textwidth}}}$ &
		д. ф.-м.н., проф. &
		Е.М. Семенов \\
	\end{tabular}
\end{flushleft}
	
\hfill \break
	
\begin{center}
	Воронеж 2023
\end{center}

	
	\newpage
	\setcounter{page}{2}
	\tableofcontents

	\newpage
	\section*{\centering Введение}
	\addcontentsline{toc}{section}{Введение}
	
	В условиях пандемии и дистанционного обучения появилась необходимость в выборе подходящего
программного обеспечения для проведения онлайн-занятий. Это вызвало определенные трудности:
не для каждой дисциплины можно было подобное подобрать. К примеру, в общем доступе есть большое
множество сервисов с встроенной интерактивной доской, чатом, голосовой связью, функцией демонстрации экрана.
Большинство из них достаточно удобны для проведения онлайн-занятий по некоторым дисциплинам,
особенно если они лекционные, где, по большей части, не требуется индивидуальное взаимодействие
преподавателя с учеником. Неудобства начинаются на практических занятиях, в частности
по математическим дисциплинам: алгебра, теория чисел, планиметрия, стереометрия и т.д. На таких
занятиях, помимо всего прочего, требуется индивидуальное взаимодействие с каждым учеником.
Трудности заключаются в том, что для этих целей нет специализированного сервиса, объединяющего
все необходимое. Под необходимым подразумеваются в первую очередь средства общения (чат и голосовая связь)
и интерактивные инструменты для работы с математическими и геометрическими объектами. Популярные
сервисы не обладают такой комбинацией. К примеру, известный от компании Google продукт Jamboard, который,
по сути своей, является исключительно интеративной доской без каких-либо дополнительных функций.

\begin{figure}[h]
	\begin{center}
		\includegraphics[width=\linewidth]{vvedenie/img1.png}
		\caption{Jamboard}
		\label{vvedenie_img1}
	\end{center}
\end{figure}

Кроме этого, можно рассмотреть сочетание нескольких программ: мессенджера и специализированного инструмента.
В качестве такого сочетания можно взять сервис Discord в качестве средства общения и инструмент для решения
математических задач Mathway. 

\begin{figure}[h]
	\begin{center}
		\includegraphics[width=\linewidth]{vvedenie/img2.png}
		\caption{Инструмент для решения математических задач Mathway}
		\label{vvedenie_img2}
	\end{center}
\end{figure}

Связка получается достаточно удобной, но только для проведения лекционных занятий. У нее есть существенный
минус: в среде Mathway невозможно работать группой людей. К примеру, преподаватель попросил ученика
построить график функции. Для выполнения этого задания ученику нужно включить в сервисе Discord
демонстрацию экрана, после этого переключиться на Mathway и соответственно построить требуемый график функции.
Ситуация станет намного сложнее, если ученику придется использовать наработки преподавателя, т.е. появятся
дополнительные действия по их копированию.

Таким образом, вопрос поиска специализированного программного обеспечения остается открытым. Именно поэтому
возникла идея создания многопользовательского интерактивного сервиса, который бы объединил средство общения и
инструмент для работы с математическими и геометрическими объектами. Таким сервисом стал проект «Geometry Room».

\begin{figure}[h]
	\begin{center}
		\includegraphics[width=\linewidth]{vvedenie/img3.png}
		\caption{Страница входа в «Geometry Room>>}
		\label{vvedenie_img2}
	\end{center}
\end{figure}

	
	\hfill \break
	
	\section*{\centering Цель и задачи}
	\addcontentsline{toc}{section}{Цель и задачи}
	
	

Цель дипломной работы: создание удобного веб-сервиса для многопользовательской работы с
геометрическими объектами.

Поставленная цель включает в себя выполнение определенного ряда задач. Сервис должен быть
удобным и быстрым для работы. Подходить как для проведения онлайн-занятий в учебных заведениях,
так и для простых онлайн-встреч, где могут на примере обсуждаться разные вопросы по геометрии.

Задачи дипломной работы:
\begin{enumerate} 
    \item Выбор и анализ инструмента для работы с геометрическими объектами. Он должен быть популярным и
    простым в использовании. Кроме того, он должен иметь открытый исходный код и свободную лицензию для
    возможности включить его в проект.
    \item Создание структуры проекта. Структура определяет тип приложения, его архитектуру и платформу.
    \item Определение схемы взаимодействия между пользователями. То есть собрать данные, как лучше организовать
    механизм многопользовательской работы и составить его схему.
    \item Выбор платформы проекта (сервера). Так как предполагается создание онлайн-сервиса,
    то он должен базироваться на клиент-серверной архитектуре. На этом основании подбирается платформа.
    \item Создание веб-интерфейса (клиента). Он должен быть функциональным и понятным пользователю.
    \item Настройка механизма непрерывного обмена данными между клиентом и сервером.
    \item Тестирование и отладка приложения.
\end{enumerate}

Представленные задачи требуют внимательного исследования и определенный подход к решению.

	
	\hfill \break
	
	\section{\centering Выбор и анализ инструмента для работы с геометрическими объектами}
	
	В первую очередь стоит поговорить об инструменте для работы
с геометрическими объектами. Выбор пал на программу, написанную
Маркусом Хохенвартером с названием «\emph{GeoGebra}».

«\emph{GeoGebra}» — это бесплатная кроссплатформенная динамическая математическая
программа для всех уровней образования, включающая в себя геометрию, алгебру,
таблицы, графы, статистику и арифметику, в одном пакете.

Программа предусматривает возможность работы с функциями (построение графиков,
вычисление корней, экстремумов, интегралов и т. д.) за счёт команд встроенного
языка (который также позволяет управлять и геометрическими построениями).

Данная программа, помимо своей большой популярности, является стандартом
в этой сфере.

К примеру: во время пандемии COVID-19 на математическом факультете ВГУ дистанционные занятия
по аналитической геометрии успешно проводились с использованием программы
«\emph{GeoGebra}», только транслировалась она через сторонний мессенджер,
используя демонстрацию экрана.

\begin{figure}[h]
	\begin{center}
		\includegraphics[width=\linewidth]{g1/img1.png}
		\caption{Пример работы программы «\emph{GeoGebra}»}
		\label{g1_img1}
	\end{center}
\end{figure}

Возможности «\emph{GeoGebra}»:
\begin{enumerate} 
    \item Построение кривых:
        \begin{itemize}
            \item Графиков функций $y=f(x)$.
            \item Кривых, заданных параметрически в декартовой системе координат: $x=f(t); \quad y=g(t)$.
            \item Конических сечений:
                \begin{itemize}
                    \item Коника произвольного вида — по пяти точкам.
                    \item Окружность:
                        \begin{itemize}
                            \item — по центру и точке на ней;
                            \item — по центру и радиусу;
                            \item — по трем точкам;
                        \end{itemize}
                    \item Эллипс — по двум фокусам и точке на кривой.
                    \item Парабола — по фокусу и директрисе.
                    \item Гипербола — по двум фокусам и точке на кривой.
                \end{itemize}
            \item Построение геометрического места точек, зависящих от положения некоторой другой точки,
                принадлежащей какой-либо кривой или многоугольнику.
        \end{itemize}
    \item Вычисления:
        \begin{itemize}
            \item Действия с матрицами:
                \begin{itemize}
                    \item Сложение, умножение;
                    \item Транспонирование, инвертирование;
                    \item Вычисление определителя;
                \end{itemize}
            \item Вычисления с комплексными числами;
            \item Нахождение точек пересечения кривых;
            \item Статистические функции:
                \begin{itemize}
                    \item Вычисление математического ожидания, дисперсии;
                    \item Вычисление коэффициента корреляции;
                \end{itemize}
            \item Аппроксимация множества точек кривой заданного вида:
                \begin{itemize}
                    \item полином;
                    \item экспонента;
                    \item логарифм;
                    \item синусоида;
                \end{itemize}
        \end{itemize}
\end{enumerate}

Помимо всего вышеизложенного, данная программа была включена в проект из-за того,
что имеет специальное и открытое API (\emph{GeoGebra Apps API}) для работы в браузере.
API предполагает создание объекта «\emph{GeoGebra}» — апплета, который представляет
собой отдельное окно программы, работающее на странице в браузере. На официальном сайте в документации
подробно описано API апплетов.

Именно функционал программы «\emph{GeoGebra}» и наличие открытого API для работы в браузере
выделяет ее на фоне всех остальных.

	
	\hfill \break
	
	\section{\centering Структура проекта}
	
	Первоочередная задача проекта - создание многопользовательского сервиса, включающего
в себя инструмент для работы с геометрическими объектами. Логичным решением поставленной
задачи будет приложения с клиент-серверной архитектурой.

<<Клиент -- сервер>> (англ. client -- server) — вычислительная или сетевая архитектура,
в которой задания или сетевая нагрузка распределены между поставщиками услуг, называемыми
серверами, и заказчиками услуг, называемыми клиентами.

\noindent
Фактически клиент и сервер — это программное обеспечение. Обычно программы расположены на
разных вычислительных машинах и взаимодействуют между собой через вычислительную сеть
посредством сетевых протоколов, но они могут быть расположены также и на одной машине.

\noindent
Программы-серверы ожидают от клиентских программ запросы и предоставляют им свои ресурсы
в виде данных (например, передача файлов посредством HTTP, FTP, BitTorrent, потоковое
мультимедиа или работа с базами данных) или в виде сервисных функций (например, работа
с электронной почтой, общение посредством систем мгновенного обмена сообщениями или
просмотр web-страниц во всемирной паутине).

\noindent
Поскольку одна программа-~сервер может
выполнять запросы от множества программ-клиентов, её размещают на специально выделенной
вычислительной машине, настроенной особым образом, как правило, совместно с другими
программами-серверами, поэтому производительность этой машины должна быть высокой.
Из-за особой роли такой машины в сети, специфики её оборудования и программного обеспечения,
её также называют сервером, а машины, выполняющие клиентские программы, соответственно, клиентами.

\newpage
\begin{figure}[h]
	\begin{center}
		\includegraphics[width=\linewidth]{g2/img1.png}
		\caption{<<клиент -- сервер>>}
		\label{g2_img1}
	\end{center}
\end{figure}

Клиенты и серверы обмениваются сообщениями в шаблоне запрос-ответ. Клиент отправляет запрос,
а сервер возвращает ответ. Этот обмен сообщениями является примером межпроцессного взаимодействия.
Для взаимодействия компьютеры должны иметь общий язык, и они должны следовать правилам,
чтобы и клиент, и сервер знали, чего ожидать.

\newpage
\begin{figure}[h]
	\begin{center}
		\includegraphics[width=\linewidth]{g2/img2.png}
		\caption{<<запрос-ответ>>}
		\label{g2_img2}
	\end{center}
\end{figure}

В нашем случае клиентом будет считаться страница в браузере пользователя, на которой будет располагаться
веб-интерфейс и инструмент <<GeoGebra>>. Сервер же будет средством обмена данными между клиентами.
Данная схема соответствует концепции <<сильный клиент>>, где большая часть обработки информации поручается
клиенту (пользователю). У такой концепции, в рамках проекта и поставленной задачи, есть существенный плюс: вычислительная
нагрузка распределяется на всю группу пользователей, что позволяет использовать менее производительный сервер и
обеспечить более высокую скорость обмена данными.

	
	\section{\centering Схема взаимодействия между пользователями}
	
	Проект «Geometry Room» должен открывать возможность
многопользовательской работы с инструментом «GeoGebra».
Проведя анализ работы современных мессенджеров и программ для организации видеоконференций,
можно выделить одну важную деталь — все они используют систему комнат.
То есть организатор онлайн-конференции создает комнату и приглашает в нее всех остальных участников.

Под комнатой тут подразумевается отдельное интерактивное пространство для проведения конференций.

Рассматривая наш случай, пусть у нас будет виртуальная комната с, условно, меловой доской. Возникает вопрос: «А как организовать
взаимодействие с этой доской?».

Это вопрос об одном меле и двух учениках: одновременно пользоваться
мелом не получится, только поочередно передавая его друг другу.
Остается решить, как этот мел будет передаваться между учениками у доски.
Появляются два варианта: ученики будут «отбирать» мел друг у друга,
или же спрашивать учителя: у кого в данный момент он должен быть.
Возникает очевидный вопрос: «почему мел один?». Предположим,
что два ученика попытаются на одной доске переместить одну и ту же
точку в разные места. В таком случае эта точка раздвоится, чего не должно случиться.

Исходя из вышесказанного, объединим пример и нашу задачу: нам нужна комната, в которой поочередно
будет передаваться право (аналог мела как в примере) на редактирование элементов в инструменте «GeoGebra».

Организуем два типа таких прав.

В первом случае, редактировать элементы может любой желающий, просто забрав на это право,
назовем такую систему свободной (free). Такой вариант хорошо подойдет для совместного обсуждения какой-либо задачи,
например, два студента пытаются построить кривую второго порядка, поочередно дополняя друг друга. 

Во втором случае, правом на редактирование элементов в инструменте «GeoGebra» будет управлять только
владелец комнаты. Назовем такую систему ограниченной (restricted). Приведем пример: онлайн-урок
по геометрии, учитель просит желающих «выйти к доске» и провести высоту треугольника. Один из учеников
«тянет руку», учитель передает ему право на редактирование элементов на доске. После того, как ученик выполнил задачу —
учитель забирает право на редактирование, кроме того, ученик может сам попросить право или вернуть его учителю.

Подведя итоги, получаем систему комнат с двумя типами прав:
свободные и ограниченные (free и restricted).

Свободная система прав [Рис.~\ref{fig:img1}] не предполагает наличие владельца комнаты,
каждый участник может забрать права на редактирование доски без
согласования с другими участниками.

\begin{figure}[H]
	\begin{center}
		\scalebox{0.6}{%LaTeX with PSTricks extensions
%%Creator: Inkscape 1.2 (dc2aedaf03, 2022-05-15)
%%Please note this file requires PSTricks extensions
\psset{xunit=.5pt,yunit=.5pt,runit=.5pt}
\begin{pspicture}(1024,768)
{
\newrgbcolor{curcolor}{0.50196081 0.50196081 0.50196081}
\pscustom[linestyle=none,fillstyle=solid,fillcolor=curcolor]
{
\newpath
\moveto(619.24689484,370.52987671)
\curveto(619.24689484,313.45365043)(572.9774714,267.18422699)(515.90124512,267.18422699)
\curveto(458.82501884,267.18422699)(412.5555954,313.45365043)(412.5555954,370.52987671)
\curveto(412.5555954,427.60610299)(458.82501884,473.87552643)(515.90124512,473.87552643)
\curveto(572.9774714,473.87552643)(619.24689484,427.60610299)(619.24689484,370.52987671)
\closepath
}
}
{
\newrgbcolor{curcolor}{0.50196081 0.50196081 0.50196081}
\pscustom[linestyle=none,fillstyle=solid,fillcolor=curcolor]
{
\newpath
\moveto(59.77836227,746.7229557)
\lineto(586.63853073,746.7229557)
\lineto(586.63853073,654.5224247)
\lineto(59.77836227,654.5224247)
\closepath
}
}
{
\newrgbcolor{curcolor}{0.50196081 0.50196081 0.50196081}
\pscustom[linestyle=none,fillstyle=solid,fillcolor=curcolor]
{
\newpath
\moveto(422.50131226,622.6068573)
\lineto(949.36148071,622.6068573)
\lineto(949.36148071,530.40632629)
\lineto(422.50131226,530.40632629)
\closepath
}
}
{
\newrgbcolor{curcolor}{0.50196081 0.50196081 0.50196081}
\pscustom[linestyle=none,fillstyle=solid,fillcolor=curcolor]
{
\newpath
\moveto(434.65960693,109.93139648)
\lineto(961.51977539,109.93139648)
\lineto(961.51977539,17.73086548)
\lineto(434.65960693,17.73086548)
\closepath
}
}
{
\newrgbcolor{curcolor}{0.50196081 0.50196081 0.50196081}
\pscustom[linestyle=none,fillstyle=solid,fillcolor=curcolor]
{
\newpath
\moveto(48.63323975,231.51452637)
\lineto(575.4934082,231.51452637)
\lineto(575.4934082,139.31399536)
\lineto(48.63323975,139.31399536)
\closepath
}
}
{
\newrgbcolor{curcolor}{0.50196081 0.50196081 0.50196081}
\pscustom[linestyle=none,fillstyle=solid,fillcolor=curcolor]
{
\newpath
\moveto(736.59100342,523.3139801)
\lineto(746.72292995,523.3139801)
\lineto(746.72292995,407.81002045)
\lineto(736.59100342,407.81002045)
\closepath
}
}
{
\newrgbcolor{curcolor}{0.50196081 0.50196081 0.50196081}
\pscustom[linestyle=none,fillstyle=solid,fillcolor=curcolor]
{
\newpath
\moveto(626.38939551,407.83844403)
\lineto(626.39160082,417.97037032)
\lineto(741.89555774,417.94522973)
\lineto(741.89335243,407.81330343)
\closepath
}
}
{
\newrgbcolor{curcolor}{0.50196081 0.50196081 0.50196081}
\pscustom[linestyle=none,fillstyle=solid,fillcolor=curcolor]
{
\newpath
\moveto(295.50967302,239.25205031)
\lineto(285.37774798,239.24654896)
\lineto(285.31503257,354.75049159)
\lineto(295.44695762,354.75599294)
\closepath
}
}
{
\newrgbcolor{curcolor}{0.50196081 0.50196081 0.50196081}
\pscustom[linestyle=none,fillstyle=solid,fillcolor=curcolor]
{
\newpath
\moveto(405.64860293,354.78742531)
\lineto(405.65189968,344.65549931)
\lineto(290.14794614,344.61791643)
\lineto(290.14464939,354.74984243)
\closepath
}
}
{
\newrgbcolor{curcolor}{0.50196081 0.50196081 0.50196081}
\pscustom[linestyle=none,fillstyle=solid,fillcolor=curcolor]
{
\newpath
\moveto(214.34463501,647.33656311)
\lineto(224.37473011,647.33656311)
\lineto(224.37473011,430.96183777)
\lineto(214.34463501,430.96183777)
\closepath
}
}
{
\newrgbcolor{curcolor}{0.50196081 0.50196081 0.50196081}
\pscustom[linestyle=none,fillstyle=solid,fillcolor=curcolor]
{
\newpath
\moveto(214.34977419,423.04143657)
\lineto(214.34515458,433.0715306)
\lineto(414.9470315,433.16392281)
\lineto(414.95165111,423.13382877)
\closepath
}
}
{
\newrgbcolor{curcolor}{0.50196081 0.50196081 0.50196081}
\pscustom[linestyle=none,fillstyle=solid,fillcolor=curcolor]
{
\newpath
\moveto(829.30119952,117.05295297)
\lineto(819.27117698,117.01480088)
\lineto(818.44813915,333.3879609)
\lineto(828.47816169,333.42611299)
\closepath
}
}
{
\newrgbcolor{curcolor}{0.50196081 0.50196081 0.50196081}
\pscustom[linestyle=none,fillstyle=solid,fillcolor=curcolor]
{
\newpath
\moveto(828.44294551,341.34642847)
\lineto(828.485717,331.31642457)
\lineto(627.88564272,330.46099486)
\lineto(627.84287123,340.49099876)
\closepath
}
}
{
\newrgbcolor{curcolor}{0 0 0}
\pscustom[linestyle=none,fillstyle=solid,fillcolor=curcolor]
{
\newpath
\moveto(449.84000286,356.17862991)
\lineto(449.84000286,386.64265261)
\lineto(473.94668749,386.64265261)
\lineto(473.94668749,356.17862991)
\lineto(467.50401602,356.17862991)
\lineto(467.50401602,381.2666486)
\lineto(456.28267433,381.2666486)
\lineto(456.28267433,356.17862991)
\closepath
}
}
{
\newrgbcolor{curcolor}{0 0 0}
\pscustom[linestyle=none,fillstyle=solid,fillcolor=curcolor]
{
\newpath
\moveto(494.08540481,379.90131425)
\curveto(496.70229565,379.90131425)(498.82140834,378.87731349)(500.44274288,376.82931196)
\curveto(502.06407742,374.8097549)(502.87474469,371.82308601)(502.87474469,367.86930529)
\curveto(502.87474469,363.8870801)(502.03563295,360.87196674)(500.35740948,358.82396522)
\curveto(498.67918601,356.77596369)(496.53162885,355.75196293)(493.91473801,355.75196293)
\curveto(492.23651454,355.75196293)(490.89962466,356.05062982)(489.90406836,356.64796359)
\curveto(488.90851206,357.27374184)(488.09784479,357.97063125)(487.47206655,358.73863182)
\lineto(487.13073296,358.73863182)
\curveto(487.35828868,357.54396426)(487.47206655,356.40618564)(487.47206655,355.32529594)
\lineto(487.47206655,345.93862228)
\lineto(481.11472848,345.93862228)
\lineto(481.11472848,379.47464727)
\lineto(486.27739899,379.47464727)
\lineto(487.17339966,376.44531168)
\lineto(487.47206655,376.44531168)
\curveto(488.09784479,377.38397904)(488.93695653,378.19464631)(489.98940176,378.87731349)
\curveto(491.04184698,379.55998066)(492.40718134,379.90131425)(494.08540481,379.90131425)
\closepath
\moveto(492.03740328,374.82397714)
\curveto(490.38762428,374.82397714)(489.22140118,374.29775452)(488.53873401,373.24530929)
\curveto(487.85606683,372.22130853)(487.50051101,370.67108515)(487.47206655,368.59463916)
\lineto(487.47206655,367.91197199)
\curveto(487.47206655,365.6648592)(487.7991779,363.9297468)(488.45340061,362.70663477)
\curveto(489.13606779,361.51196722)(490.35917981,360.91463344)(492.12273668,360.91463344)
\curveto(493.57340443,360.91463344)(494.64007189,361.51196722)(495.32273906,362.70663477)
\curveto(496.0338507,363.9297468)(496.38940652,365.67908143)(496.38940652,367.95463868)
\curveto(496.38940652,372.53419765)(494.93873878,374.82397714)(492.03740328,374.82397714)
\closepath
}
}
{
\newrgbcolor{curcolor}{0 0 0}
\pscustom[linestyle=none,fillstyle=solid,fillcolor=curcolor]
{
\newpath
\moveto(517.68006644,379.94398095)
\curveto(520.80895766,379.94398095)(523.19829278,379.26131378)(524.84807178,377.89597942)
\curveto(526.52629526,376.55908954)(527.36540699,374.49686578)(527.36540699,371.70930815)
\lineto(527.36540699,356.17862991)
\lineto(522.92807035,356.17862991)
\lineto(521.6907361,359.3359656)
\lineto(521.52006931,359.3359656)
\curveto(520.52451301,358.08440911)(519.47206778,357.17418621)(518.36273362,356.6052969)
\curveto(517.25339946,356.03640758)(515.73162055,355.75196293)(513.79739688,355.75196293)
\curveto(511.72095089,355.75196293)(510.00006072,356.3492967)(508.63472637,357.54396426)
\curveto(507.26939202,358.73863182)(506.58672485,360.60174432)(506.58672485,363.13330176)
\curveto(506.58672485,365.60797027)(507.45428105,367.42841607)(509.18939345,368.59463916)
\curveto(510.92450586,369.76086225)(513.52717446,370.41508496)(516.99739927,370.55730729)
\lineto(521.05073562,370.68530739)
\lineto(521.05073562,371.70930815)
\curveto(521.05073562,372.93242017)(520.72362427,373.82842084)(520.06940156,374.39731015)
\curveto(519.44362331,374.96619946)(518.56184488,375.25064412)(517.42406625,375.25064412)
\curveto(516.28628763,375.25064412)(515.17695347,375.07997733)(514.09606377,374.73864374)
\curveto(513.01517408,374.42575462)(511.93428439,374.0275321)(510.85339469,373.54397618)
\lineto(508.76272647,377.85331273)
\curveto(509.98583849,378.47909097)(511.36539507,378.97686912)(512.90139622,379.34664717)
\curveto(514.43739736,379.74486969)(516.03028744,379.94398095)(517.68006644,379.94398095)
\closepath
\moveto(521.05073562,366.97330462)
\lineto(518.57606711,366.88797122)
\curveto(516.52806559,366.83108229)(515.1058423,366.46130424)(514.30939727,365.77863706)
\curveto(513.51295223,365.09596989)(513.11472971,364.19996922)(513.11472971,363.09063506)
\curveto(513.11472971,362.12352323)(513.39917437,361.42663382)(513.96806368,360.99996684)
\curveto(514.53695299,360.60174432)(515.2765091,360.40263306)(516.186732,360.40263306)
\curveto(517.55206635,360.40263306)(518.70406721,360.80085558)(519.64273457,361.59730061)
\curveto(520.58140194,362.42219012)(521.05073562,363.57419098)(521.05073562,365.05330319)
\closepath
}
}
{
\newrgbcolor{curcolor}{0 0 0}
\pscustom[linestyle=none,fillstyle=solid,fillcolor=curcolor]
{
\newpath
\moveto(554.54409896,373.37330939)
\curveto(554.54409896,372.1217529)(554.14587644,371.05508544)(553.3494314,370.173307)
\curveto(552.58143083,369.29152857)(551.42942997,368.72263926)(549.89342883,368.46663907)
\lineto(549.89342883,368.29597227)
\curveto(551.51476337,368.09686101)(552.80898656,367.5279717)(553.77609839,366.58930433)
\curveto(554.77165469,365.67908143)(555.26943283,364.52708058)(555.26943283,363.13330176)
\curveto(555.26943283,361.79641187)(554.91387701,360.60174432)(554.20276537,359.54929909)
\curveto(553.5200982,358.49685386)(552.42498627,357.67196436)(550.91742959,357.07463058)
\curveto(549.40987291,356.4772968)(547.43298255,356.17862991)(544.98675851,356.17862991)
\lineto(533.89341691,356.17862991)
\lineto(533.89341691,379.47464727)
\lineto(544.98675851,379.47464727)
\curveto(546.80720431,379.47464727)(548.42853885,379.27553601)(549.85076213,378.87731349)
\curveto(551.30142988,378.50753544)(552.4392085,377.86753496)(553.26409801,376.95731206)
\curveto(554.11743198,376.07553362)(554.54409896,374.88086607)(554.54409896,373.37330939)
\closepath
\moveto(548.10142749,372.86130901)
\curveto(548.10142749,374.28353229)(546.9778711,374.99464393)(544.73075832,374.99464393)
\lineto(540.25075498,374.99464393)
\lineto(540.25075498,370.3866405)
\lineto(544.00542444,370.3866405)
\curveto(545.34231433,370.3866405)(546.35209286,370.57152952)(547.03476003,370.94130758)
\curveto(547.74587167,371.3395301)(548.10142749,371.97953057)(548.10142749,372.86130901)
\closepath
\moveto(548.69876127,363.47463535)
\curveto(548.69876127,364.38485825)(548.32898322,365.03908096)(547.58942711,365.43730348)
\curveto(546.87831547,365.86397046)(545.82587024,366.07730395)(544.43209143,366.07730395)
\lineto(540.25075498,366.07730395)
\lineto(540.25075498,360.57329985)
\lineto(544.56009152,360.57329985)
\curveto(545.75475908,360.57329985)(546.73609314,360.78663334)(547.50409372,361.21330033)
\curveto(548.30053875,361.66841178)(548.69876127,362.42219012)(548.69876127,363.47463535)
\closepath
}
}
{
\newrgbcolor{curcolor}{0 0 0}
\pscustom[linestyle=none,fillstyle=solid,fillcolor=curcolor]
{
\newpath
\moveto(570.28809867,379.94398095)
\curveto(573.41698989,379.94398095)(575.806325,379.26131378)(577.45610401,377.89597942)
\curveto(579.13432748,376.55908954)(579.97343922,374.49686578)(579.97343922,371.70930815)
\lineto(579.97343922,356.17862991)
\lineto(575.53610258,356.17862991)
\lineto(574.29876833,359.3359656)
\lineto(574.12810153,359.3359656)
\curveto(573.13254523,358.08440911)(572.08010001,357.17418621)(570.97076585,356.6052969)
\curveto(569.86143169,356.03640758)(568.33965277,355.75196293)(566.40542911,355.75196293)
\curveto(564.32898312,355.75196293)(562.60809295,356.3492967)(561.2427586,357.54396426)
\curveto(559.87742425,358.73863182)(559.19475707,360.60174432)(559.19475707,363.13330176)
\curveto(559.19475707,365.60797027)(560.06231327,367.42841607)(561.79742568,368.59463916)
\curveto(563.53253808,369.76086225)(566.13520669,370.41508496)(569.6054315,370.55730729)
\lineto(573.65876785,370.68530739)
\lineto(573.65876785,371.70930815)
\curveto(573.65876785,372.93242017)(573.33165649,373.82842084)(572.67743378,374.39731015)
\curveto(572.05165554,374.96619946)(571.16987711,375.25064412)(570.03209848,375.25064412)
\curveto(568.89431985,375.25064412)(567.78498569,375.07997733)(566.704096,374.73864374)
\curveto(565.62320631,374.42575462)(564.54231661,374.0275321)(563.46142692,373.54397618)
\lineto(561.37075869,377.85331273)
\curveto(562.59387072,378.47909097)(563.9734273,378.97686912)(565.50942844,379.34664717)
\curveto(567.04542959,379.74486969)(568.63831966,379.94398095)(570.28809867,379.94398095)
\closepath
\moveto(573.65876785,366.97330462)
\lineto(571.18409934,366.88797122)
\curveto(569.13609781,366.83108229)(567.71387453,366.46130424)(566.91742949,365.77863706)
\curveto(566.12098446,365.09596989)(565.72276194,364.19996922)(565.72276194,363.09063506)
\curveto(565.72276194,362.12352323)(566.00720659,361.42663382)(566.57609591,360.99996684)
\curveto(567.14498522,360.60174432)(567.88454132,360.40263306)(568.79476423,360.40263306)
\curveto(570.16009858,360.40263306)(571.31209943,360.80085558)(572.2507668,361.59730061)
\curveto(573.18943417,362.42219012)(573.65876785,363.57419098)(573.65876785,365.05330319)
\closepath
}
}
{
\newrgbcolor{curcolor}{0 0 0}
\pscustom[linestyle=none,fillstyle=solid,fillcolor=curcolor]
{
\newpath
\moveto(139.28802108,716.64265261)
\lineto(129.85868072,695.09596989)
\curveto(129.00534675,693.13330176)(128.09512385,691.45507829)(127.12801202,690.06129947)
\curveto(126.16090018,688.66752065)(124.93778816,687.60085319)(123.45867595,686.86129709)
\curveto(122.0080082,686.12174098)(120.08800677,685.75196293)(117.69867166,685.75196293)
\curveto(116.95911555,685.75196293)(116.14844828,685.80885186)(115.26666985,685.92262972)
\curveto(114.38489141,686.03640758)(113.57422414,686.19285214)(112.83466803,686.3919634)
\lineto(112.83466803,691.9386342)
\curveto(113.51733521,691.65418955)(114.25689132,691.45507829)(115.05333635,691.34130042)
\curveto(115.87822586,691.22752256)(116.66044866,691.17063363)(117.40000477,691.17063363)
\curveto(118.82222805,691.17063363)(119.84622881,691.51196722)(120.47200706,692.19463439)
\curveto(121.0977853,692.90574603)(121.59556345,693.75908)(121.9653415,694.7546363)
\lineto(111.42666698,716.64265261)
\lineto(118.25333874,716.64265261)
\lineto(123.92800963,703.45864279)
\curveto(124.12712089,703.0319758)(124.39734332,702.42041979)(124.7386769,701.62397475)
\curveto(125.08001049,700.85597418)(125.33601068,700.20175147)(125.50667747,699.66130662)
\lineto(125.72001097,699.66130662)
\curveto(125.89067776,700.173307)(126.13245572,700.84175195)(126.44534484,701.66664145)
\curveto(126.78667843,702.49153095)(127.08534532,703.21686483)(127.34134551,703.84264307)
\lineto(132.63201612,716.64265261)
\closepath
}
}
{
\newrgbcolor{curcolor}{0 0 0}
\pscustom[linestyle=none,fillstyle=solid,fillcolor=curcolor]
{
\newpath
\moveto(147.35201138,709.47464727)
\lineto(147.35201138,700.94130758)
\curveto(147.35201138,698.92175052)(148.29067875,697.91197199)(150.16801348,697.91197199)
\curveto(151.3911255,697.91197199)(152.52890413,698.03997208)(153.58134935,698.29597227)
\curveto(154.63379458,698.58041693)(155.68623981,698.95019498)(156.73868504,699.40530643)
\lineto(156.73868504,709.47464727)
\lineto(163.09602311,709.47464727)
\lineto(163.09602311,686.17862991)
\lineto(156.73868504,686.17862991)
\lineto(156.73868504,695.43730348)
\curveto(155.74312874,694.89685863)(154.60535012,694.39908048)(153.32534916,693.94396903)
\curveto(152.04534821,693.51730205)(150.59468046,693.30396855)(148.97334592,693.30396855)
\curveto(146.55556634,693.30396855)(144.62134268,693.91552456)(143.17067493,695.13863659)
\curveto(141.72000718,696.39019307)(140.99467331,698.28175004)(140.99467331,700.81330748)
\lineto(140.99467331,709.47464727)
\closepath
}
}
{
\newrgbcolor{curcolor}{0 0 0}
\pscustom[linestyle=none,fillstyle=solid,fillcolor=curcolor]
{
\newpath
\moveto(179.30937705,709.94398095)
\curveto(182.43826827,709.94398095)(184.82760339,709.26131378)(186.47738239,707.89597942)
\curveto(188.15560587,706.55908954)(188.9947176,704.49686578)(188.9947176,701.70930815)
\lineto(188.9947176,686.17862991)
\lineto(184.55738096,686.17862991)
\lineto(183.32004671,689.3359656)
\lineto(183.14937991,689.3359656)
\curveto(182.15382362,688.08440911)(181.10137839,687.17418621)(179.99204423,686.6052969)
\curveto(178.88271007,686.03640758)(177.36093116,685.75196293)(175.42670749,685.75196293)
\curveto(173.3502615,685.75196293)(171.62937133,686.3492967)(170.26403698,687.54396426)
\curveto(168.89870263,688.73863182)(168.21603545,690.60174432)(168.21603545,693.13330176)
\curveto(168.21603545,695.60797027)(169.08359166,697.42841607)(170.81870406,698.59463916)
\curveto(172.55381646,699.76086225)(175.15648507,700.41508496)(178.62670988,700.55730729)
\lineto(182.68004623,700.68530739)
\lineto(182.68004623,701.70930815)
\curveto(182.68004623,702.93242017)(182.35293488,703.82842084)(181.69871217,704.39731015)
\curveto(181.07293392,704.96619946)(180.19115549,705.25064412)(179.05337686,705.25064412)
\curveto(177.91559824,705.25064412)(176.80626408,705.07997733)(175.72537438,704.73864374)
\curveto(174.64448469,704.42575462)(173.56359499,704.0275321)(172.4827053,703.54397618)
\lineto(170.39203708,707.85331273)
\curveto(171.6151491,708.47909097)(172.99470568,708.97686912)(174.53070683,709.34664717)
\curveto(176.06670797,709.74486969)(177.65959805,709.94398095)(179.30937705,709.94398095)
\closepath
\moveto(182.68004623,696.97330462)
\lineto(180.20537772,696.88797122)
\curveto(178.15737619,696.83108229)(176.73515291,696.46130424)(175.93870787,695.77863706)
\curveto(175.14226284,695.09596989)(174.74404032,694.19996922)(174.74404032,693.09063506)
\curveto(174.74404032,692.12352323)(175.02848497,691.42663382)(175.59737429,690.99996684)
\curveto(176.1662636,690.60174432)(176.90581971,690.40263306)(177.81604261,690.40263306)
\curveto(179.18137696,690.40263306)(180.33337782,690.80085558)(181.27204518,691.59730061)
\curveto(182.21071255,692.42219012)(182.68004623,693.57419098)(182.68004623,695.05330319)
\closepath
}
}
{
\newrgbcolor{curcolor}{0 0 0}
\pscustom[linestyle=none,fillstyle=solid,fillcolor=curcolor]
{
\newpath
\moveto(204.99472694,685.75196293)
\curveto(201.52450214,685.75196293)(198.83650013,686.70485253)(196.93072094,688.61063172)
\curveto(195.0533862,690.51641092)(194.11471884,693.54574651)(194.11471884,697.69863849)
\curveto(194.11471884,700.54308506)(194.59827475,702.86130901)(195.56538659,704.65331034)
\curveto(196.53249842,706.44531168)(197.8693883,707.76797933)(199.57605624,708.6213133)
\curveto(201.31116864,709.47464727)(203.30228124,709.90131425)(205.54939402,709.90131425)
\curveto(207.1422841,709.90131425)(208.52184068,709.74486969)(209.68806377,709.43198057)
\curveto(210.88273133,709.11909145)(211.92095433,708.74931339)(212.80273276,708.32264641)
\lineto(210.92539803,703.41597609)
\curveto(209.92984173,703.81419861)(208.99117437,704.14130996)(208.10939593,704.39731015)
\curveto(207.25606196,704.65331034)(206.40272799,704.78131044)(205.54939402,704.78131044)
\curveto(202.24983601,704.78131044)(200.600057,702.43464202)(200.600057,697.74130519)
\curveto(200.600057,695.40885901)(201.02672399,693.68796884)(201.88005796,692.57863468)
\curveto(202.76183639,691.46930052)(203.98494841,690.91463344)(205.54939402,690.91463344)
\curveto(206.88628391,690.91463344)(208.06672923,691.08530023)(209.09073,691.42663382)
\curveto(210.11473076,691.79641187)(211.11028706,692.29419002)(212.07739889,692.91996827)
\lineto(212.07739889,687.50129756)
\curveto(211.11028706,686.87551932)(210.08628629,686.4346301)(209.0053966,686.17862991)
\curveto(207.95295137,685.89418525)(206.61606149,685.75196293)(204.99472694,685.75196293)
\closepath
}
}
{
\newrgbcolor{curcolor}{0 0 0}
\pscustom[linestyle=none,fillstyle=solid,fillcolor=curcolor]
{
\newpath
\moveto(236.73875454,704.69597704)
\lineto(229.10141552,704.69597704)
\lineto(229.10141552,686.17862991)
\lineto(222.74407745,686.17862991)
\lineto(222.74407745,704.69597704)
\lineto(215.10673842,704.69597704)
\lineto(215.10673842,709.47464727)
\lineto(236.73875454,709.47464727)
\closepath
}
}
{
\newrgbcolor{curcolor}{0 0 0}
\pscustom[linestyle=none,fillstyle=solid,fillcolor=curcolor]
{
\newpath
\moveto(247.40540281,709.47464727)
\lineto(247.40540281,700.51464059)
\lineto(256.28007608,700.51464059)
\lineto(256.28007608,709.47464727)
\lineto(262.63741415,709.47464727)
\lineto(262.63741415,686.17862991)
\lineto(256.28007608,686.17862991)
\lineto(256.28007608,695.77863706)
\lineto(247.40540281,695.77863706)
\lineto(247.40540281,686.17862991)
\lineto(241.04806474,686.17862991)
\lineto(241.04806474,709.47464727)
\closepath
}
}
{
\newrgbcolor{curcolor}{0 0 0}
\pscustom[linestyle=none,fillstyle=solid,fillcolor=curcolor]
{
\newpath
\moveto(275.43746116,709.47464727)
\lineto(275.43746116,700.2586404)
\curveto(275.43746116,699.77508449)(275.40901669,699.17775071)(275.35212776,698.46663907)
\curveto(275.3236833,697.75552743)(275.2810166,697.03019355)(275.22412767,696.29063744)
\curveto(275.1956832,695.55108134)(275.1530165,694.8826364)(275.09612757,694.28530262)
\curveto(275.03923864,693.7164133)(274.99657194,693.33241302)(274.96812748,693.13330176)
\lineto(285.72013549,709.47464727)
\lineto(293.35747451,709.47464727)
\lineto(293.35747451,686.17862991)
\lineto(287.21346993,686.17862991)
\lineto(287.21346993,695.47997017)
\curveto(287.21346993,696.21952628)(287.2419144,697.05863802)(287.29880333,697.99730538)
\curveto(287.35569226,698.93597275)(287.41258119,699.80352895)(287.46947012,700.59997399)
\curveto(287.55480352,701.42486349)(287.61169245,702.05064174)(287.64013692,702.47730872)
\lineto(276.9307956,686.17862991)
\lineto(269.29345658,686.17862991)
\lineto(269.29345658,709.47464727)
\closepath
}
}
{
\newrgbcolor{curcolor}{0 0 0}
\pscustom[linestyle=none,fillstyle=solid,fillcolor=curcolor]
{
\newpath
\moveto(315.28812059,709.47464727)
\lineto(322.28545914,709.47464727)
\lineto(313.06945227,698.29597227)
\lineto(323.09612641,686.17862991)
\lineto(315.88545437,686.17862991)
\lineto(306.37078061,697.99730538)
\lineto(306.37078061,686.17862991)
\lineto(300.01344254,686.17862991)
\lineto(300.01344254,709.47464727)
\lineto(306.37078061,709.47464727)
\lineto(306.37078061,698.16797218)
\closepath
}
}
{
\newrgbcolor{curcolor}{0 0 0}
\pscustom[linestyle=none,fillstyle=solid,fillcolor=curcolor]
{
\newpath
\moveto(352.79213365,709.47464727)
\lineto(359.7894722,709.47464727)
\lineto(350.57346533,698.29597227)
\lineto(360.60013947,686.17862991)
\lineto(353.38946743,686.17862991)
\lineto(343.87479367,697.99730538)
\lineto(343.87479367,686.17862991)
\lineto(337.5174556,686.17862991)
\lineto(337.5174556,709.47464727)
\lineto(343.87479367,709.47464727)
\lineto(343.87479367,698.16797218)
\closepath
}
}
{
\newrgbcolor{curcolor}{0 0 0}
\pscustom[linestyle=none,fillstyle=solid,fillcolor=curcolor]
{
\newpath
\moveto(384.23745005,697.86930529)
\curveto(384.23745005,694.00085796)(383.21344929,691.01418907)(381.16544776,688.90929861)
\curveto(379.1458907,686.80440815)(376.38677753,685.75196293)(372.88810826,685.75196293)
\curveto(370.72632887,685.75196293)(368.79210521,686.22129661)(367.08543727,687.15996398)
\curveto(365.4072138,688.09863134)(364.08454615,689.46396569)(363.11743431,691.25596703)
\curveto(362.15032248,693.07641283)(361.66676657,695.28085891)(361.66676657,697.86930529)
\curveto(361.66676657,701.73775261)(362.6765451,704.71019927)(364.69610216,706.78664527)
\curveto(366.71565922,708.86309126)(369.48899462,709.90131425)(373.01610836,709.90131425)
\curveto(375.20633221,709.90131425)(377.14055587,709.43198057)(378.81877935,708.4933132)
\curveto(380.49700282,707.55464584)(381.81967047,706.18931149)(382.7867823,704.39731015)
\curveto(383.75389413,702.60530882)(384.23745005,700.4293072)(384.23745005,697.86930529)
\closepath
\moveto(368.15210473,697.86930529)
\curveto(368.15210473,695.56530357)(368.52188279,693.81596893)(369.26143889,692.62130138)
\curveto(370.02943946,691.45507829)(371.26677372,690.87196674)(372.97344166,690.87196674)
\curveto(374.65166513,690.87196674)(375.86055492,691.45507829)(376.60011103,692.62130138)
\curveto(377.3681116,693.81596893)(377.75211189,695.56530357)(377.75211189,697.86930529)
\curveto(377.75211189,700.173307)(377.3681116,701.89419718)(376.60011103,703.0319758)
\curveto(375.86055492,704.19819889)(374.6374429,704.78131044)(372.93077496,704.78131044)
\curveto(371.25255149,704.78131044)(370.02943946,704.19819889)(369.26143889,703.0319758)
\curveto(368.52188279,701.89419718)(368.15210473,700.173307)(368.15210473,697.86930529)
\closepath
}
}
{
\newrgbcolor{curcolor}{0 0 0}
\pscustom[linestyle=none,fillstyle=solid,fillcolor=curcolor]
{
\newpath
\moveto(418.8401312,709.47464727)
\lineto(418.8401312,686.17862991)
\lineto(412.90946012,686.17862991)
\lineto(412.90946012,697.6133051)
\curveto(412.90946012,698.75108372)(412.92368235,699.86041788)(412.95212682,700.94130758)
\curveto(413.00901575,702.02219727)(413.08012691,703.01775357)(413.16546031,703.92797647)
\lineto(413.03746021,703.92797647)
\lineto(406.59478875,686.17862991)
\lineto(401.81611852,686.17862991)
\lineto(395.28811366,703.97064317)
\lineto(395.11744686,703.97064317)
\curveto(395.23122472,703.0319758)(395.30233589,702.02219727)(395.33078035,700.94130758)
\curveto(395.38766928,699.88886235)(395.41611375,698.72263926)(395.41611375,697.4426383)
\lineto(395.41611375,686.17862991)
\lineto(389.48544267,686.17862991)
\lineto(389.48544267,709.47464727)
\lineto(398.48811604,709.47464727)
\lineto(404.29078703,693.68796884)
\lineto(410.17879142,709.47464727)
\closepath
}
}
{
\newrgbcolor{curcolor}{0 0 0}
\pscustom[linestyle=none,fillstyle=solid,fillcolor=curcolor]
{
\newpath
\moveto(431.85346189,709.47464727)
\lineto(431.85346189,700.51464059)
\lineto(440.72813517,700.51464059)
\lineto(440.72813517,709.47464727)
\lineto(447.08547324,709.47464727)
\lineto(447.08547324,686.17862991)
\lineto(440.72813517,686.17862991)
\lineto(440.72813517,695.77863706)
\lineto(431.85346189,695.77863706)
\lineto(431.85346189,686.17862991)
\lineto(425.49612382,686.17862991)
\lineto(425.49612382,709.47464727)
\closepath
}
}
{
\newrgbcolor{curcolor}{0 0 0}
\pscustom[linestyle=none,fillstyle=solid,fillcolor=curcolor]
{
\newpath
\moveto(463.29885612,709.94398095)
\curveto(466.42774734,709.94398095)(468.81708245,709.26131378)(470.46686146,707.89597942)
\curveto(472.14508493,706.55908954)(472.98419667,704.49686578)(472.98419667,701.70930815)
\lineto(472.98419667,686.17862991)
\lineto(468.54686003,686.17862991)
\lineto(467.30952577,689.3359656)
\lineto(467.13885898,689.3359656)
\curveto(466.14330268,688.08440911)(465.09085745,687.17418621)(463.98152329,686.6052969)
\curveto(462.87218913,686.03640758)(461.35041022,685.75196293)(459.41618656,685.75196293)
\curveto(457.33974057,685.75196293)(455.6188504,686.3492967)(454.25351604,687.54396426)
\curveto(452.88818169,688.73863182)(452.20551452,690.60174432)(452.20551452,693.13330176)
\curveto(452.20551452,695.60797027)(453.07307072,697.42841607)(454.80818312,698.59463916)
\curveto(456.54329553,699.76086225)(459.14596413,700.41508496)(462.61618894,700.55730729)
\lineto(466.6695253,700.68530739)
\lineto(466.6695253,701.70930815)
\curveto(466.6695253,702.93242017)(466.34241394,703.82842084)(465.68819123,704.39731015)
\curveto(465.06241299,704.96619946)(464.18063455,705.25064412)(463.04285593,705.25064412)
\curveto(461.9050773,705.25064412)(460.79574314,705.07997733)(459.71485345,704.73864374)
\curveto(458.63396375,704.42575462)(457.55307406,704.0275321)(456.47218436,703.54397618)
\lineto(454.38151614,707.85331273)
\curveto(455.60462816,708.47909097)(456.98418475,708.97686912)(458.52018589,709.34664717)
\curveto(460.05618703,709.74486969)(461.64907711,709.94398095)(463.29885612,709.94398095)
\closepath
\moveto(466.6695253,696.97330462)
\lineto(464.19485678,696.88797122)
\curveto(462.14685526,696.83108229)(460.72463198,696.46130424)(459.92818694,695.77863706)
\curveto(459.1317419,695.09596989)(458.73351938,694.19996922)(458.73351938,693.09063506)
\curveto(458.73351938,692.12352323)(459.01796404,691.42663382)(459.58685335,690.99996684)
\curveto(460.15574266,690.60174432)(460.89529877,690.40263306)(461.80552167,690.40263306)
\curveto(463.17085602,690.40263306)(464.32285688,690.80085558)(465.26152425,691.59730061)
\curveto(466.20019161,692.42219012)(466.6695253,693.57419098)(466.6695253,695.05330319)
\closepath
}
}
{
\newrgbcolor{curcolor}{0 0 0}
\pscustom[linestyle=none,fillstyle=solid,fillcolor=curcolor]
{
\newpath
\moveto(498.79755428,704.69597704)
\lineto(491.16021526,704.69597704)
\lineto(491.16021526,686.17862991)
\lineto(484.80287719,686.17862991)
\lineto(484.80287719,704.69597704)
\lineto(477.16553817,704.69597704)
\lineto(477.16553817,709.47464727)
\lineto(498.79755428,709.47464727)
\closepath
}
}
{
\newrgbcolor{curcolor}{0 0 0}
\pscustom[linestyle=none,fillstyle=solid,fillcolor=curcolor]
{
\newpath
\moveto(503.10687211,686.17862991)
\lineto(503.10687211,709.47464727)
\lineto(509.46421018,709.47464727)
\lineto(509.46421018,700.47197389)
\lineto(512.53621247,700.47197389)
\curveto(516.09177067,700.47197389)(518.72288374,699.90308458)(520.42955168,698.76530596)
\curveto(522.13621962,697.62752733)(522.98955359,695.90663716)(522.98955359,693.60263544)
\curveto(522.98955359,691.32707819)(522.19310855,689.52085462)(520.60021848,688.18396474)
\curveto(519.0073284,686.84707485)(516.39043756,686.17862991)(512.74954596,686.17862991)
\closepath
\moveto(526.36022277,686.17862991)
\lineto(526.36022277,709.47464727)
\lineto(532.71756084,709.47464727)
\lineto(532.71756084,686.17862991)
\closepath
\moveto(509.46421018,690.57329985)
\lineto(512.40821237,690.57329985)
\curveto(513.65976886,690.57329985)(514.66954739,690.78663334)(515.43754796,691.21330033)
\curveto(516.233993,691.66841178)(516.63221552,692.43641235)(516.63221552,693.51730205)
\curveto(516.63221552,695.22396998)(515.19577,696.07730395)(512.32287897,696.07730395)
\lineto(509.46421018,696.07730395)
\closepath
}
}
{
\newrgbcolor{curcolor}{0 0 0}
\pscustom[linestyle=none,fillstyle=solid,fillcolor=curcolor]
{
\newpath
\moveto(512.21590108,590.44534261)
\lineto(502.78656072,568.89865989)
\curveto(501.93322675,566.93599176)(501.02300385,565.25776829)(500.05589202,563.86398947)
\curveto(499.08878018,562.47021065)(497.86566816,561.40354319)(496.38655595,560.66398709)
\curveto(494.9358882,559.92443098)(493.01588677,559.55465293)(490.62655166,559.55465293)
\curveto(489.88699555,559.55465293)(489.07632828,559.61154186)(488.19454985,559.72531972)
\curveto(487.31277141,559.83909758)(486.50210414,559.99554214)(485.76254803,560.1946534)
\lineto(485.76254803,565.7413242)
\curveto(486.44521521,565.45687955)(487.18477132,565.25776829)(487.98121635,565.14399042)
\curveto(488.80610586,565.03021256)(489.58832866,564.97332363)(490.32788477,564.97332363)
\curveto(491.75010805,564.97332363)(492.77410881,565.31465722)(493.39988706,565.99732439)
\curveto(494.0256653,566.70843603)(494.52344345,567.56177)(494.8932215,568.5573263)
\lineto(484.35454698,590.44534261)
\lineto(491.18121874,590.44534261)
\lineto(496.85588963,577.26133279)
\curveto(497.05500089,576.8346658)(497.32522332,576.22310979)(497.6665569,575.42666475)
\curveto(498.00789049,574.65866418)(498.26389068,574.00444147)(498.43455747,573.46399662)
\lineto(498.64789097,573.46399662)
\curveto(498.81855776,573.975997)(499.06033572,574.64444195)(499.37322484,575.46933145)
\curveto(499.71455843,576.29422095)(500.01322532,577.01955483)(500.26922551,577.64533307)
\lineto(505.55989612,590.44534261)
\closepath
}
}
{
\newrgbcolor{curcolor}{0 0 0}
\pscustom[linestyle=none,fillstyle=solid,fillcolor=curcolor]
{
\newpath
\moveto(520.27989138,583.27733727)
\lineto(520.27989138,574.74399758)
\curveto(520.27989138,572.72444052)(521.21855875,571.71466199)(523.09589348,571.71466199)
\curveto(524.3190055,571.71466199)(525.45678413,571.84266208)(526.50922935,572.09866227)
\curveto(527.56167458,572.38310693)(528.61411981,572.75288498)(529.66656504,573.20799643)
\lineto(529.66656504,583.27733727)
\lineto(536.02390311,583.27733727)
\lineto(536.02390311,559.98131991)
\lineto(529.66656504,559.98131991)
\lineto(529.66656504,569.23999348)
\curveto(528.67100874,568.69954863)(527.53323012,568.20177048)(526.25322916,567.74665903)
\curveto(524.97322821,567.31999205)(523.52256046,567.10665855)(521.90122592,567.10665855)
\curveto(519.48344634,567.10665855)(517.54922268,567.71821456)(516.09855493,568.94132659)
\curveto(514.64788718,570.19288307)(513.92255331,572.08444004)(513.92255331,574.61599748)
\lineto(513.92255331,583.27733727)
\closepath
}
}
{
\newrgbcolor{curcolor}{0 0 0}
\pscustom[linestyle=none,fillstyle=solid,fillcolor=curcolor]
{
\newpath
\moveto(552.23725705,583.74667095)
\curveto(555.36614827,583.74667095)(557.75548339,583.06400378)(559.40526239,581.69866942)
\curveto(561.08348587,580.36177954)(561.9225976,578.29955578)(561.9225976,575.51199815)
\lineto(561.9225976,559.98131991)
\lineto(557.48526096,559.98131991)
\lineto(556.24792671,563.1386556)
\lineto(556.07725991,563.1386556)
\curveto(555.08170362,561.88709911)(554.02925839,560.97687621)(552.91992423,560.4079869)
\curveto(551.81059007,559.83909758)(550.28881116,559.55465293)(548.35458749,559.55465293)
\curveto(546.2781415,559.55465293)(544.55725133,560.1519867)(543.19191698,561.34665426)
\curveto(541.82658263,562.54132182)(541.14391545,564.40443432)(541.14391545,566.93599176)
\curveto(541.14391545,569.41066027)(542.01147166,571.23110607)(543.74658406,572.39732916)
\curveto(545.48169646,573.56355225)(548.08436507,574.21777496)(551.55458988,574.35999729)
\lineto(555.60792623,574.48799739)
\lineto(555.60792623,575.51199815)
\curveto(555.60792623,576.73511017)(555.28081488,577.63111084)(554.62659217,578.20000015)
\curveto(554.00081392,578.76888946)(553.11903549,579.05333412)(551.98125686,579.05333412)
\curveto(550.84347824,579.05333412)(549.73414408,578.88266733)(548.65325438,578.54133374)
\curveto(547.57236469,578.22844462)(546.49147499,577.8302221)(545.4105853,577.34666618)
\lineto(543.31991708,581.65600273)
\curveto(544.5430291,582.28178097)(545.92258568,582.77955912)(547.45858683,583.14933717)
\curveto(548.99458797,583.54755969)(550.58747805,583.74667095)(552.23725705,583.74667095)
\closepath
\moveto(555.60792623,570.77599462)
\lineto(553.13325772,570.69066122)
\curveto(551.08525619,570.63377229)(549.66303291,570.26399424)(548.86658787,569.58132706)
\curveto(548.07014284,568.89865989)(547.67192032,568.00265922)(547.67192032,566.89332506)
\curveto(547.67192032,565.92621323)(547.95636497,565.22932382)(548.52525429,564.80265684)
\curveto(549.0941436,564.40443432)(549.83369971,564.20532306)(550.74392261,564.20532306)
\curveto(552.10925696,564.20532306)(553.26125782,564.60354558)(554.19992518,565.39999061)
\curveto(555.13859255,566.22488012)(555.60792623,567.37688098)(555.60792623,568.85599319)
\closepath
}
}
{
\newrgbcolor{curcolor}{0 0 0}
\pscustom[linestyle=none,fillstyle=solid,fillcolor=curcolor]
{
\newpath
\moveto(577.92260694,559.55465293)
\curveto(574.45238214,559.55465293)(571.76438013,560.50754253)(569.85860094,562.41332172)
\curveto(567.9812662,564.31910092)(567.04259884,567.34843651)(567.04259884,571.50132849)
\curveto(567.04259884,574.34577506)(567.52615475,576.66399901)(568.49326659,578.45600034)
\curveto(569.46037842,580.24800168)(570.7972683,581.57066933)(572.50393624,582.4240033)
\curveto(574.23904864,583.27733727)(576.23016124,583.70400425)(578.47727402,583.70400425)
\curveto(580.0701641,583.70400425)(581.44972068,583.54755969)(582.61594377,583.23467057)
\curveto(583.81061133,582.92178145)(584.84883433,582.55200339)(585.73061276,582.12533641)
\lineto(583.85327803,577.21866609)
\curveto(582.85772173,577.61688861)(581.91905437,577.94399996)(581.03727593,578.20000015)
\curveto(580.18394196,578.45600034)(579.33060799,578.58400044)(578.47727402,578.58400044)
\curveto(575.17771601,578.58400044)(573.527937,576.23733202)(573.527937,571.54399519)
\curveto(573.527937,569.21154901)(573.95460399,567.49065884)(574.80793796,566.38132468)
\curveto(575.68971639,565.27199052)(576.91282841,564.71732344)(578.47727402,564.71732344)
\curveto(579.81416391,564.71732344)(580.99460923,564.88799023)(582.01861,565.22932382)
\curveto(583.04261076,565.59910187)(584.03816706,566.09688002)(585.00527889,566.72265827)
\lineto(585.00527889,561.30398756)
\curveto(584.03816706,560.67820932)(583.01416629,560.2373201)(581.9332766,559.98131991)
\curveto(580.88083137,559.69687525)(579.54394149,559.55465293)(577.92260694,559.55465293)
\closepath
}
}
{
\newrgbcolor{curcolor}{0 0 0}
\pscustom[linestyle=none,fillstyle=solid,fillcolor=curcolor]
{
\newpath
\moveto(609.66663454,578.49866704)
\lineto(602.02929552,578.49866704)
\lineto(602.02929552,559.98131991)
\lineto(595.67195745,559.98131991)
\lineto(595.67195745,578.49866704)
\lineto(588.03461842,578.49866704)
\lineto(588.03461842,583.27733727)
\lineto(609.66663454,583.27733727)
\closepath
}
}
{
\newrgbcolor{curcolor}{0 0 0}
\pscustom[linestyle=none,fillstyle=solid,fillcolor=curcolor]
{
\newpath
\moveto(620.33328281,583.27733727)
\lineto(620.33328281,574.31733059)
\lineto(629.20795608,574.31733059)
\lineto(629.20795608,583.27733727)
\lineto(635.56529415,583.27733727)
\lineto(635.56529415,559.98131991)
\lineto(629.20795608,559.98131991)
\lineto(629.20795608,569.58132706)
\lineto(620.33328281,569.58132706)
\lineto(620.33328281,559.98131991)
\lineto(613.97594474,559.98131991)
\lineto(613.97594474,583.27733727)
\closepath
}
}
{
\newrgbcolor{curcolor}{0 0 0}
\pscustom[linestyle=none,fillstyle=solid,fillcolor=curcolor]
{
\newpath
\moveto(648.36534116,583.27733727)
\lineto(648.36534116,574.0613304)
\curveto(648.36534116,573.57777449)(648.33689669,572.98044071)(648.28000776,572.26932907)
\curveto(648.2515633,571.55821743)(648.2088966,570.83288355)(648.15200767,570.09332744)
\curveto(648.1235632,569.35377134)(648.0808965,568.6853264)(648.02400757,568.08799262)
\curveto(647.96711864,567.5191033)(647.92445194,567.13510302)(647.89600748,566.93599176)
\lineto(658.64801549,583.27733727)
\lineto(666.28535451,583.27733727)
\lineto(666.28535451,559.98131991)
\lineto(660.14134993,559.98131991)
\lineto(660.14134993,569.28266017)
\curveto(660.14134993,570.02221628)(660.1697944,570.86132802)(660.22668333,571.79999538)
\curveto(660.28357226,572.73866275)(660.34046119,573.60621895)(660.39735012,574.40266399)
\curveto(660.48268352,575.22755349)(660.53957245,575.85333174)(660.56801692,576.27999872)
\lineto(649.8586756,559.98131991)
\lineto(642.22133658,559.98131991)
\lineto(642.22133658,583.27733727)
\closepath
}
}
{
\newrgbcolor{curcolor}{0 0 0}
\pscustom[linestyle=none,fillstyle=solid,fillcolor=curcolor]
{
\newpath
\moveto(688.21600059,583.27733727)
\lineto(695.21333914,583.27733727)
\lineto(685.99733227,572.09866227)
\lineto(696.02400641,559.98131991)
\lineto(688.81333437,559.98131991)
\lineto(679.29866061,571.79999538)
\lineto(679.29866061,559.98131991)
\lineto(672.94132254,559.98131991)
\lineto(672.94132254,583.27733727)
\lineto(679.29866061,583.27733727)
\lineto(679.29866061,571.97066218)
\closepath
}
}
{
\newrgbcolor{curcolor}{0 0 0}
\pscustom[linestyle=none,fillstyle=solid,fillcolor=curcolor]
{
\newpath
\moveto(725.72001365,583.27733727)
\lineto(732.7173522,583.27733727)
\lineto(723.50134533,572.09866227)
\lineto(733.52801947,559.98131991)
\lineto(726.31734743,559.98131991)
\lineto(716.80267367,571.79999538)
\lineto(716.80267367,559.98131991)
\lineto(710.4453356,559.98131991)
\lineto(710.4453356,583.27733727)
\lineto(716.80267367,583.27733727)
\lineto(716.80267367,571.97066218)
\closepath
}
}
{
\newrgbcolor{curcolor}{0 0 0}
\pscustom[linestyle=none,fillstyle=solid,fillcolor=curcolor]
{
\newpath
\moveto(757.16533005,571.67199529)
\curveto(757.16533005,567.80354796)(756.14132929,564.81687907)(754.09332776,562.71198861)
\curveto(752.0737707,560.60709815)(749.31465753,559.55465293)(745.81598826,559.55465293)
\curveto(743.65420887,559.55465293)(741.71998521,560.02398661)(740.01331727,560.96265398)
\curveto(738.3350938,561.90132134)(737.01242615,563.26665569)(736.04531431,565.05865703)
\curveto(735.07820248,566.87910283)(734.59464657,569.08354891)(734.59464657,571.67199529)
\curveto(734.59464657,575.54044261)(735.6044251,578.51288927)(737.62398216,580.58933527)
\curveto(739.64353922,582.66578126)(742.41687462,583.70400425)(745.94398836,583.70400425)
\curveto(748.13421221,583.70400425)(750.06843587,583.23467057)(751.74665935,582.2960032)
\curveto(753.42488282,581.35733584)(754.74755047,579.99200149)(755.7146623,578.20000015)
\curveto(756.68177413,576.40799882)(757.16533005,574.2319972)(757.16533005,571.67199529)
\closepath
\moveto(741.07998473,571.67199529)
\curveto(741.07998473,569.36799357)(741.44976279,567.61865893)(742.18931889,566.42399138)
\curveto(742.95731946,565.25776829)(744.19465372,564.67465674)(745.90132166,564.67465674)
\curveto(747.57954513,564.67465674)(748.78843492,565.25776829)(749.52799103,566.42399138)
\curveto(750.2959916,567.61865893)(750.67999189,569.36799357)(750.67999189,571.67199529)
\curveto(750.67999189,573.975997)(750.2959916,575.69688718)(749.52799103,576.8346658)
\curveto(748.78843492,578.00088889)(747.5653229,578.58400044)(745.85865496,578.58400044)
\curveto(744.18043149,578.58400044)(742.95731946,578.00088889)(742.18931889,576.8346658)
\curveto(741.44976279,575.69688718)(741.07998473,573.975997)(741.07998473,571.67199529)
\closepath
}
}
{
\newrgbcolor{curcolor}{0 0 0}
\pscustom[linestyle=none,fillstyle=solid,fillcolor=curcolor]
{
\newpath
\moveto(791.7680112,583.27733727)
\lineto(791.7680112,559.98131991)
\lineto(785.83734012,559.98131991)
\lineto(785.83734012,571.4159951)
\curveto(785.83734012,572.55377372)(785.85156235,573.66310788)(785.88000682,574.74399758)
\curveto(785.93689575,575.82488727)(786.00800691,576.82044357)(786.09334031,577.73066647)
\lineto(785.96534021,577.73066647)
\lineto(779.52266875,559.98131991)
\lineto(774.74399852,559.98131991)
\lineto(768.21599366,577.77333317)
\lineto(768.04532686,577.77333317)
\curveto(768.15910472,576.8346658)(768.23021589,575.82488727)(768.25866035,574.74399758)
\curveto(768.31554928,573.69155235)(768.34399375,572.52532926)(768.34399375,571.2453283)
\lineto(768.34399375,559.98131991)
\lineto(762.41332267,559.98131991)
\lineto(762.41332267,583.27733727)
\lineto(771.41599604,583.27733727)
\lineto(777.21866703,567.49065884)
\lineto(783.10667142,583.27733727)
\closepath
}
}
{
\newrgbcolor{curcolor}{0 0 0}
\pscustom[linestyle=none,fillstyle=solid,fillcolor=curcolor]
{
\newpath
\moveto(804.78134189,583.27733727)
\lineto(804.78134189,574.31733059)
\lineto(813.65601517,574.31733059)
\lineto(813.65601517,583.27733727)
\lineto(820.01335324,583.27733727)
\lineto(820.01335324,559.98131991)
\lineto(813.65601517,559.98131991)
\lineto(813.65601517,569.58132706)
\lineto(804.78134189,569.58132706)
\lineto(804.78134189,559.98131991)
\lineto(798.42400382,559.98131991)
\lineto(798.42400382,583.27733727)
\closepath
}
}
{
\newrgbcolor{curcolor}{0 0 0}
\pscustom[linestyle=none,fillstyle=solid,fillcolor=curcolor]
{
\newpath
\moveto(836.22673612,583.74667095)
\curveto(839.35562734,583.74667095)(841.74496245,583.06400378)(843.39474146,581.69866942)
\curveto(845.07296493,580.36177954)(845.91207667,578.29955578)(845.91207667,575.51199815)
\lineto(845.91207667,559.98131991)
\lineto(841.47474003,559.98131991)
\lineto(840.23740577,563.1386556)
\lineto(840.06673898,563.1386556)
\curveto(839.07118268,561.88709911)(838.01873745,560.97687621)(836.90940329,560.4079869)
\curveto(835.80006913,559.83909758)(834.27829022,559.55465293)(832.34406656,559.55465293)
\curveto(830.26762057,559.55465293)(828.5467304,560.1519867)(827.18139604,561.34665426)
\curveto(825.81606169,562.54132182)(825.13339452,564.40443432)(825.13339452,566.93599176)
\curveto(825.13339452,569.41066027)(826.00095072,571.23110607)(827.73606312,572.39732916)
\curveto(829.47117553,573.56355225)(832.07384413,574.21777496)(835.54406894,574.35999729)
\lineto(839.5974053,574.48799739)
\lineto(839.5974053,575.51199815)
\curveto(839.5974053,576.73511017)(839.27029394,577.63111084)(838.61607123,578.20000015)
\curveto(837.99029299,578.76888946)(837.10851455,579.05333412)(835.97073593,579.05333412)
\curveto(834.8329573,579.05333412)(833.72362314,578.88266733)(832.64273345,578.54133374)
\curveto(831.56184375,578.22844462)(830.48095406,577.8302221)(829.40006436,577.34666618)
\lineto(827.30939614,581.65600273)
\curveto(828.53250816,582.28178097)(829.91206475,582.77955912)(831.44806589,583.14933717)
\curveto(832.98406703,583.54755969)(834.57695711,583.74667095)(836.22673612,583.74667095)
\closepath
\moveto(839.5974053,570.77599462)
\lineto(837.12273678,570.69066122)
\curveto(835.07473526,570.63377229)(833.65251198,570.26399424)(832.85606694,569.58132706)
\curveto(832.0596219,568.89865989)(831.66139938,568.00265922)(831.66139938,566.89332506)
\curveto(831.66139938,565.92621323)(831.94584404,565.22932382)(832.51473335,564.80265684)
\curveto(833.08362266,564.40443432)(833.82317877,564.20532306)(834.73340167,564.20532306)
\curveto(836.09873602,564.20532306)(837.25073688,564.60354558)(838.18940425,565.39999061)
\curveto(839.12807161,566.22488012)(839.5974053,567.37688098)(839.5974053,568.85599319)
\closepath
}
}
{
\newrgbcolor{curcolor}{0 0 0}
\pscustom[linestyle=none,fillstyle=solid,fillcolor=curcolor]
{
\newpath
\moveto(871.72543428,578.49866704)
\lineto(864.08809526,578.49866704)
\lineto(864.08809526,559.98131991)
\lineto(857.73075719,559.98131991)
\lineto(857.73075719,578.49866704)
\lineto(850.09341817,578.49866704)
\lineto(850.09341817,583.27733727)
\lineto(871.72543428,583.27733727)
\closepath
}
}
{
\newrgbcolor{curcolor}{0 0 0}
\pscustom[linestyle=none,fillstyle=solid,fillcolor=curcolor]
{
\newpath
\moveto(876.03475211,559.98131991)
\lineto(876.03475211,583.27733727)
\lineto(882.39209018,583.27733727)
\lineto(882.39209018,574.27466389)
\lineto(885.46409247,574.27466389)
\curveto(889.01965067,574.27466389)(891.65076374,573.70577458)(893.35743168,572.56799596)
\curveto(895.06409962,571.43021733)(895.91743359,569.70932716)(895.91743359,567.40532544)
\curveto(895.91743359,565.12976819)(895.12098855,563.32354462)(893.52809848,561.98665474)
\curveto(891.9352084,560.64976485)(889.31831756,559.98131991)(885.67742596,559.98131991)
\closepath
\moveto(899.28810277,559.98131991)
\lineto(899.28810277,583.27733727)
\lineto(905.64544084,583.27733727)
\lineto(905.64544084,559.98131991)
\closepath
\moveto(882.39209018,564.37598985)
\lineto(885.33609237,564.37598985)
\curveto(886.58764886,564.37598985)(887.59742739,564.58932334)(888.36542796,565.01599033)
\curveto(889.161873,565.47110178)(889.56009552,566.23910235)(889.56009552,567.31999205)
\curveto(889.56009552,569.02665998)(888.12365,569.87999395)(885.25075897,569.87999395)
\lineto(882.39209018,569.87999395)
\closepath
}
}
{
\newrgbcolor{curcolor}{0 0 0}
\pscustom[linestyle=none,fillstyle=solid,fillcolor=curcolor]
{
\newpath
\moveto(525.21590108,76.44535261)
\lineto(515.78656072,54.89866989)
\curveto(514.93322675,52.93600176)(514.02300385,51.25777829)(513.05589202,49.86399947)
\curveto(512.08878018,48.47022065)(510.86566816,47.40355319)(509.38655595,46.66399709)
\curveto(507.9358882,45.92444098)(506.01588677,45.55466293)(503.62655166,45.55466293)
\curveto(502.88699555,45.55466293)(502.07632828,45.61155186)(501.19454985,45.72532972)
\curveto(500.31277141,45.83910758)(499.50210414,45.99555214)(498.76254803,46.1946634)
\lineto(498.76254803,51.7413342)
\curveto(499.44521521,51.45688955)(500.18477132,51.25777829)(500.98121635,51.14400042)
\curveto(501.80610586,51.03022256)(502.58832866,50.97333363)(503.32788477,50.97333363)
\curveto(504.75010805,50.97333363)(505.77410881,51.31466722)(506.39988706,51.99733439)
\curveto(507.0256653,52.70844603)(507.52344345,53.56178)(507.8932215,54.5573363)
\lineto(497.35454698,76.44535261)
\lineto(504.18121874,76.44535261)
\lineto(509.85588963,63.26134279)
\curveto(510.05500089,62.8346758)(510.32522332,62.22311979)(510.6665569,61.42667475)
\curveto(511.00789049,60.65867418)(511.26389068,60.00445147)(511.43455747,59.46400662)
\lineto(511.64789097,59.46400662)
\curveto(511.81855776,59.976007)(512.06033572,60.64445195)(512.37322484,61.46934145)
\curveto(512.71455843,62.29423095)(513.01322532,63.01956483)(513.26922551,63.64534307)
\lineto(518.55989612,76.44535261)
\closepath
}
}
{
\newrgbcolor{curcolor}{0 0 0}
\pscustom[linestyle=none,fillstyle=solid,fillcolor=curcolor]
{
\newpath
\moveto(533.27989138,69.27734727)
\lineto(533.27989138,60.74400758)
\curveto(533.27989138,58.72445052)(534.21855875,57.71467199)(536.09589348,57.71467199)
\curveto(537.3190055,57.71467199)(538.45678413,57.84267208)(539.50922935,58.09867227)
\curveto(540.56167458,58.38311693)(541.61411981,58.75289498)(542.66656504,59.20800643)
\lineto(542.66656504,69.27734727)
\lineto(549.02390311,69.27734727)
\lineto(549.02390311,45.98132991)
\lineto(542.66656504,45.98132991)
\lineto(542.66656504,55.24000348)
\curveto(541.67100874,54.69955863)(540.53323012,54.20178048)(539.25322916,53.74666903)
\curveto(537.97322821,53.32000205)(536.52256046,53.10666855)(534.90122592,53.10666855)
\curveto(532.48344634,53.10666855)(530.54922268,53.71822456)(529.09855493,54.94133659)
\curveto(527.64788718,56.19289307)(526.92255331,58.08445004)(526.92255331,60.61600748)
\lineto(526.92255331,69.27734727)
\closepath
}
}
{
\newrgbcolor{curcolor}{0 0 0}
\pscustom[linestyle=none,fillstyle=solid,fillcolor=curcolor]
{
\newpath
\moveto(565.23725705,69.74668095)
\curveto(568.36614827,69.74668095)(570.75548339,69.06401378)(572.40526239,67.69867942)
\curveto(574.08348587,66.36178954)(574.9225976,64.29956578)(574.9225976,61.51200815)
\lineto(574.9225976,45.98132991)
\lineto(570.48526096,45.98132991)
\lineto(569.24792671,49.1386656)
\lineto(569.07725991,49.1386656)
\curveto(568.08170362,47.88710911)(567.02925839,46.97688621)(565.91992423,46.4079969)
\curveto(564.81059007,45.83910758)(563.28881116,45.55466293)(561.35458749,45.55466293)
\curveto(559.2781415,45.55466293)(557.55725133,46.1519967)(556.19191698,47.34666426)
\curveto(554.82658263,48.54133182)(554.14391545,50.40444432)(554.14391545,52.93600176)
\curveto(554.14391545,55.41067027)(555.01147166,57.23111607)(556.74658406,58.39733916)
\curveto(558.48169646,59.56356225)(561.08436507,60.21778496)(564.55458988,60.36000729)
\lineto(568.60792623,60.48800739)
\lineto(568.60792623,61.51200815)
\curveto(568.60792623,62.73512017)(568.28081488,63.63112084)(567.62659217,64.20001015)
\curveto(567.00081392,64.76889946)(566.11903549,65.05334412)(564.98125686,65.05334412)
\curveto(563.84347824,65.05334412)(562.73414408,64.88267733)(561.65325438,64.54134374)
\curveto(560.57236469,64.22845462)(559.49147499,63.8302321)(558.4105853,63.34667618)
\lineto(556.31991708,67.65601273)
\curveto(557.5430291,68.28179097)(558.92258568,68.77956912)(560.45858683,69.14934717)
\curveto(561.99458797,69.54756969)(563.58747805,69.74668095)(565.23725705,69.74668095)
\closepath
\moveto(568.60792623,56.77600462)
\lineto(566.13325772,56.69067122)
\curveto(564.08525619,56.63378229)(562.66303291,56.26400424)(561.86658787,55.58133706)
\curveto(561.07014284,54.89866989)(560.67192032,54.00266922)(560.67192032,52.89333506)
\curveto(560.67192032,51.92622323)(560.95636497,51.22933382)(561.52525429,50.80266684)
\curveto(562.0941436,50.40444432)(562.83369971,50.20533306)(563.74392261,50.20533306)
\curveto(565.10925696,50.20533306)(566.26125782,50.60355558)(567.19992518,51.40000061)
\curveto(568.13859255,52.22489012)(568.60792623,53.37689098)(568.60792623,54.85600319)
\closepath
}
}
{
\newrgbcolor{curcolor}{0 0 0}
\pscustom[linestyle=none,fillstyle=solid,fillcolor=curcolor]
{
\newpath
\moveto(590.92260694,45.55466293)
\curveto(587.45238214,45.55466293)(584.76438013,46.50755253)(582.85860094,48.41333172)
\curveto(580.9812662,50.31911092)(580.04259884,53.34844651)(580.04259884,57.50133849)
\curveto(580.04259884,60.34578506)(580.52615475,62.66400901)(581.49326659,64.45601034)
\curveto(582.46037842,66.24801168)(583.7972683,67.57067933)(585.50393624,68.4240133)
\curveto(587.23904864,69.27734727)(589.23016124,69.70401425)(591.47727402,69.70401425)
\curveto(593.0701641,69.70401425)(594.44972068,69.54756969)(595.61594377,69.23468057)
\curveto(596.81061133,68.92179145)(597.84883433,68.55201339)(598.73061276,68.12534641)
\lineto(596.85327803,63.21867609)
\curveto(595.85772173,63.61689861)(594.91905437,63.94400996)(594.03727593,64.20001015)
\curveto(593.18394196,64.45601034)(592.33060799,64.58401044)(591.47727402,64.58401044)
\curveto(588.17771601,64.58401044)(586.527937,62.23734202)(586.527937,57.54400519)
\curveto(586.527937,55.21155901)(586.95460399,53.49066884)(587.80793796,52.38133468)
\curveto(588.68971639,51.27200052)(589.91282841,50.71733344)(591.47727402,50.71733344)
\curveto(592.81416391,50.71733344)(593.99460923,50.88800023)(595.01861,51.22933382)
\curveto(596.04261076,51.59911187)(597.03816706,52.09689002)(598.00527889,52.72266827)
\lineto(598.00527889,47.30399756)
\curveto(597.03816706,46.67821932)(596.01416629,46.2373301)(594.9332766,45.98132991)
\curveto(593.88083137,45.69688525)(592.54394149,45.55466293)(590.92260694,45.55466293)
\closepath
}
}
{
\newrgbcolor{curcolor}{0 0 0}
\pscustom[linestyle=none,fillstyle=solid,fillcolor=curcolor]
{
\newpath
\moveto(622.66663454,64.49867704)
\lineto(615.02929552,64.49867704)
\lineto(615.02929552,45.98132991)
\lineto(608.67195745,45.98132991)
\lineto(608.67195745,64.49867704)
\lineto(601.03461842,64.49867704)
\lineto(601.03461842,69.27734727)
\lineto(622.66663454,69.27734727)
\closepath
}
}
{
\newrgbcolor{curcolor}{0 0 0}
\pscustom[linestyle=none,fillstyle=solid,fillcolor=curcolor]
{
\newpath
\moveto(633.33328281,69.27734727)
\lineto(633.33328281,60.31734059)
\lineto(642.20795608,60.31734059)
\lineto(642.20795608,69.27734727)
\lineto(648.56529415,69.27734727)
\lineto(648.56529415,45.98132991)
\lineto(642.20795608,45.98132991)
\lineto(642.20795608,55.58133706)
\lineto(633.33328281,55.58133706)
\lineto(633.33328281,45.98132991)
\lineto(626.97594474,45.98132991)
\lineto(626.97594474,69.27734727)
\closepath
}
}
{
\newrgbcolor{curcolor}{0 0 0}
\pscustom[linestyle=none,fillstyle=solid,fillcolor=curcolor]
{
\newpath
\moveto(661.36534116,69.27734727)
\lineto(661.36534116,60.0613404)
\curveto(661.36534116,59.57778449)(661.33689669,58.98045071)(661.28000776,58.26933907)
\curveto(661.2515633,57.55822743)(661.2088966,56.83289355)(661.15200767,56.09333744)
\curveto(661.1235632,55.35378134)(661.0808965,54.6853364)(661.02400757,54.08800262)
\curveto(660.96711864,53.5191133)(660.92445194,53.13511302)(660.89600748,52.93600176)
\lineto(671.64801549,69.27734727)
\lineto(679.28535451,69.27734727)
\lineto(679.28535451,45.98132991)
\lineto(673.14134993,45.98132991)
\lineto(673.14134993,55.28267017)
\curveto(673.14134993,56.02222628)(673.1697944,56.86133802)(673.22668333,57.80000538)
\curveto(673.28357226,58.73867275)(673.34046119,59.60622895)(673.39735012,60.40267399)
\curveto(673.48268352,61.22756349)(673.53957245,61.85334174)(673.56801692,62.28000872)
\lineto(662.8586756,45.98132991)
\lineto(655.22133658,45.98132991)
\lineto(655.22133658,69.27734727)
\closepath
}
}
{
\newrgbcolor{curcolor}{0 0 0}
\pscustom[linestyle=none,fillstyle=solid,fillcolor=curcolor]
{
\newpath
\moveto(701.21600059,69.27734727)
\lineto(708.21333914,69.27734727)
\lineto(698.99733227,58.09867227)
\lineto(709.02400641,45.98132991)
\lineto(701.81333437,45.98132991)
\lineto(692.29866061,57.80000538)
\lineto(692.29866061,45.98132991)
\lineto(685.94132254,45.98132991)
\lineto(685.94132254,69.27734727)
\lineto(692.29866061,69.27734727)
\lineto(692.29866061,57.97067218)
\closepath
}
}
{
\newrgbcolor{curcolor}{0 0 0}
\pscustom[linestyle=none,fillstyle=solid,fillcolor=curcolor]
{
\newpath
\moveto(738.72001365,69.27734727)
\lineto(745.7173522,69.27734727)
\lineto(736.50134533,58.09867227)
\lineto(746.52801947,45.98132991)
\lineto(739.31734743,45.98132991)
\lineto(729.80267367,57.80000538)
\lineto(729.80267367,45.98132991)
\lineto(723.4453356,45.98132991)
\lineto(723.4453356,69.27734727)
\lineto(729.80267367,69.27734727)
\lineto(729.80267367,57.97067218)
\closepath
}
}
{
\newrgbcolor{curcolor}{0 0 0}
\pscustom[linestyle=none,fillstyle=solid,fillcolor=curcolor]
{
\newpath
\moveto(770.16533005,57.67200529)
\curveto(770.16533005,53.80355796)(769.14132929,50.81688907)(767.09332776,48.71199861)
\curveto(765.0737707,46.60710815)(762.31465753,45.55466293)(758.81598826,45.55466293)
\curveto(756.65420887,45.55466293)(754.71998521,46.02399661)(753.01331727,46.96266398)
\curveto(751.3350938,47.90133134)(750.01242615,49.26666569)(749.04531431,51.05866703)
\curveto(748.07820248,52.87911283)(747.59464657,55.08355891)(747.59464657,57.67200529)
\curveto(747.59464657,61.54045261)(748.6044251,64.51289927)(750.62398216,66.58934527)
\curveto(752.64353922,68.66579126)(755.41687462,69.70401425)(758.94398836,69.70401425)
\curveto(761.13421221,69.70401425)(763.06843587,69.23468057)(764.74665935,68.2960132)
\curveto(766.42488282,67.35734584)(767.74755047,65.99201149)(768.7146623,64.20001015)
\curveto(769.68177413,62.40800882)(770.16533005,60.2320072)(770.16533005,57.67200529)
\closepath
\moveto(754.07998473,57.67200529)
\curveto(754.07998473,55.36800357)(754.44976279,53.61866893)(755.18931889,52.42400138)
\curveto(755.95731946,51.25777829)(757.19465372,50.67466674)(758.90132166,50.67466674)
\curveto(760.57954513,50.67466674)(761.78843492,51.25777829)(762.52799103,52.42400138)
\curveto(763.2959916,53.61866893)(763.67999189,55.36800357)(763.67999189,57.67200529)
\curveto(763.67999189,59.976007)(763.2959916,61.69689718)(762.52799103,62.8346758)
\curveto(761.78843492,64.00089889)(760.5653229,64.58401044)(758.85865496,64.58401044)
\curveto(757.18043149,64.58401044)(755.95731946,64.00089889)(755.18931889,62.8346758)
\curveto(754.44976279,61.69689718)(754.07998473,59.976007)(754.07998473,57.67200529)
\closepath
}
}
{
\newrgbcolor{curcolor}{0 0 0}
\pscustom[linestyle=none,fillstyle=solid,fillcolor=curcolor]
{
\newpath
\moveto(804.7680112,69.27734727)
\lineto(804.7680112,45.98132991)
\lineto(798.83734012,45.98132991)
\lineto(798.83734012,57.4160051)
\curveto(798.83734012,58.55378372)(798.85156235,59.66311788)(798.88000682,60.74400758)
\curveto(798.93689575,61.82489727)(799.00800691,62.82045357)(799.09334031,63.73067647)
\lineto(798.96534021,63.73067647)
\lineto(792.52266875,45.98132991)
\lineto(787.74399852,45.98132991)
\lineto(781.21599366,63.77334317)
\lineto(781.04532686,63.77334317)
\curveto(781.15910472,62.8346758)(781.23021589,61.82489727)(781.25866035,60.74400758)
\curveto(781.31554928,59.69156235)(781.34399375,58.52533926)(781.34399375,57.2453383)
\lineto(781.34399375,45.98132991)
\lineto(775.41332267,45.98132991)
\lineto(775.41332267,69.27734727)
\lineto(784.41599604,69.27734727)
\lineto(790.21866703,53.49066884)
\lineto(796.10667142,69.27734727)
\closepath
}
}
{
\newrgbcolor{curcolor}{0 0 0}
\pscustom[linestyle=none,fillstyle=solid,fillcolor=curcolor]
{
\newpath
\moveto(817.78134189,69.27734727)
\lineto(817.78134189,60.31734059)
\lineto(826.65601517,60.31734059)
\lineto(826.65601517,69.27734727)
\lineto(833.01335324,69.27734727)
\lineto(833.01335324,45.98132991)
\lineto(826.65601517,45.98132991)
\lineto(826.65601517,55.58133706)
\lineto(817.78134189,55.58133706)
\lineto(817.78134189,45.98132991)
\lineto(811.42400382,45.98132991)
\lineto(811.42400382,69.27734727)
\closepath
}
}
{
\newrgbcolor{curcolor}{0 0 0}
\pscustom[linestyle=none,fillstyle=solid,fillcolor=curcolor]
{
\newpath
\moveto(849.22673612,69.74668095)
\curveto(852.35562734,69.74668095)(854.74496245,69.06401378)(856.39474146,67.69867942)
\curveto(858.07296493,66.36178954)(858.91207667,64.29956578)(858.91207667,61.51200815)
\lineto(858.91207667,45.98132991)
\lineto(854.47474003,45.98132991)
\lineto(853.23740577,49.1386656)
\lineto(853.06673898,49.1386656)
\curveto(852.07118268,47.88710911)(851.01873745,46.97688621)(849.90940329,46.4079969)
\curveto(848.80006913,45.83910758)(847.27829022,45.55466293)(845.34406656,45.55466293)
\curveto(843.26762057,45.55466293)(841.5467304,46.1519967)(840.18139604,47.34666426)
\curveto(838.81606169,48.54133182)(838.13339452,50.40444432)(838.13339452,52.93600176)
\curveto(838.13339452,55.41067027)(839.00095072,57.23111607)(840.73606312,58.39733916)
\curveto(842.47117553,59.56356225)(845.07384413,60.21778496)(848.54406894,60.36000729)
\lineto(852.5974053,60.48800739)
\lineto(852.5974053,61.51200815)
\curveto(852.5974053,62.73512017)(852.27029394,63.63112084)(851.61607123,64.20001015)
\curveto(850.99029299,64.76889946)(850.10851455,65.05334412)(848.97073593,65.05334412)
\curveto(847.8329573,65.05334412)(846.72362314,64.88267733)(845.64273345,64.54134374)
\curveto(844.56184375,64.22845462)(843.48095406,63.8302321)(842.40006436,63.34667618)
\lineto(840.30939614,67.65601273)
\curveto(841.53250816,68.28179097)(842.91206475,68.77956912)(844.44806589,69.14934717)
\curveto(845.98406703,69.54756969)(847.57695711,69.74668095)(849.22673612,69.74668095)
\closepath
\moveto(852.5974053,56.77600462)
\lineto(850.12273678,56.69067122)
\curveto(848.07473526,56.63378229)(846.65251198,56.26400424)(845.85606694,55.58133706)
\curveto(845.0596219,54.89866989)(844.66139938,54.00266922)(844.66139938,52.89333506)
\curveto(844.66139938,51.92622323)(844.94584404,51.22933382)(845.51473335,50.80266684)
\curveto(846.08362266,50.40444432)(846.82317877,50.20533306)(847.73340167,50.20533306)
\curveto(849.09873602,50.20533306)(850.25073688,50.60355558)(851.18940425,51.40000061)
\curveto(852.12807161,52.22489012)(852.5974053,53.37689098)(852.5974053,54.85600319)
\closepath
}
}
{
\newrgbcolor{curcolor}{0 0 0}
\pscustom[linestyle=none,fillstyle=solid,fillcolor=curcolor]
{
\newpath
\moveto(884.72543428,64.49867704)
\lineto(877.08809526,64.49867704)
\lineto(877.08809526,45.98132991)
\lineto(870.73075719,45.98132991)
\lineto(870.73075719,64.49867704)
\lineto(863.09341817,64.49867704)
\lineto(863.09341817,69.27734727)
\lineto(884.72543428,69.27734727)
\closepath
}
}
{
\newrgbcolor{curcolor}{0 0 0}
\pscustom[linestyle=none,fillstyle=solid,fillcolor=curcolor]
{
\newpath
\moveto(889.03475211,45.98132991)
\lineto(889.03475211,69.27734727)
\lineto(895.39209018,69.27734727)
\lineto(895.39209018,60.27467389)
\lineto(898.46409247,60.27467389)
\curveto(902.01965067,60.27467389)(904.65076374,59.70578458)(906.35743168,58.56800596)
\curveto(908.06409962,57.43022733)(908.91743359,55.70933716)(908.91743359,53.40533544)
\curveto(908.91743359,51.12977819)(908.12098855,49.32355462)(906.52809848,47.98666474)
\curveto(904.9352084,46.64977485)(902.31831756,45.98132991)(898.67742596,45.98132991)
\closepath
\moveto(912.28810277,45.98132991)
\lineto(912.28810277,69.27734727)
\lineto(918.64544084,69.27734727)
\lineto(918.64544084,45.98132991)
\closepath
\moveto(895.39209018,50.37599985)
\lineto(898.33609237,50.37599985)
\curveto(899.58764886,50.37599985)(900.59742739,50.58933334)(901.36542796,51.01600033)
\curveto(902.161873,51.47111178)(902.56009552,52.23911235)(902.56009552,53.32000205)
\curveto(902.56009552,55.02666998)(901.12365,55.88000395)(898.25075897,55.88000395)
\lineto(895.39209018,55.88000395)
\closepath
}
}
{
\newrgbcolor{curcolor}{0 0 0}
\pscustom[linestyle=none,fillstyle=solid,fillcolor=curcolor]
{
\newpath
\moveto(129.21590608,197.44534261)
\lineto(119.78656572,175.89865989)
\curveto(118.93323175,173.93599176)(118.02300885,172.25776829)(117.05589702,170.86398947)
\curveto(116.08878518,169.47021065)(114.86567316,168.40354319)(113.38656095,167.66398709)
\curveto(111.9358932,166.92443098)(110.01589177,166.55465293)(107.62655666,166.55465293)
\curveto(106.88700055,166.55465293)(106.07633328,166.61154186)(105.19455485,166.72531972)
\curveto(104.31277641,166.83909758)(103.50210914,166.99554214)(102.76255303,167.1946534)
\lineto(102.76255303,172.7413242)
\curveto(103.44522021,172.45687955)(104.18477632,172.25776829)(104.98122135,172.14399042)
\curveto(105.80611086,172.03021256)(106.58833366,171.97332363)(107.32788977,171.97332363)
\curveto(108.75011305,171.97332363)(109.77411381,172.31465722)(110.39989206,172.99732439)
\curveto(111.0256703,173.70843603)(111.52344845,174.56177)(111.8932265,175.5573263)
\lineto(101.35455198,197.44534261)
\lineto(108.18122374,197.44534261)
\lineto(113.85589463,184.26133279)
\curveto(114.05500589,183.8346658)(114.32522832,183.22310979)(114.6665619,182.42666475)
\curveto(115.00789549,181.65866418)(115.26389568,181.00444147)(115.43456247,180.46399662)
\lineto(115.64789597,180.46399662)
\curveto(115.81856276,180.975997)(116.06034072,181.64444195)(116.37322984,182.46933145)
\curveto(116.71456343,183.29422095)(117.01323032,184.01955483)(117.26923051,184.64533307)
\lineto(122.55990112,197.44534261)
\closepath
}
}
{
\newrgbcolor{curcolor}{0 0 0}
\pscustom[linestyle=none,fillstyle=solid,fillcolor=curcolor]
{
\newpath
\moveto(137.27989638,190.27733727)
\lineto(137.27989638,181.74399758)
\curveto(137.27989638,179.72444052)(138.21856375,178.71466199)(140.09589848,178.71466199)
\curveto(141.3190105,178.71466199)(142.45678913,178.84266208)(143.50923435,179.09866227)
\curveto(144.56167958,179.38310693)(145.61412481,179.75288498)(146.66657004,180.20799643)
\lineto(146.66657004,190.27733727)
\lineto(153.02390811,190.27733727)
\lineto(153.02390811,166.98131991)
\lineto(146.66657004,166.98131991)
\lineto(146.66657004,176.23999348)
\curveto(145.67101374,175.69954863)(144.53323512,175.20177048)(143.25323416,174.74665903)
\curveto(141.97323321,174.31999205)(140.52256546,174.10665855)(138.90123092,174.10665855)
\curveto(136.48345134,174.10665855)(134.54922768,174.71821456)(133.09855993,175.94132659)
\curveto(131.64789218,177.19288307)(130.92255831,179.08444004)(130.92255831,181.61599748)
\lineto(130.92255831,190.27733727)
\closepath
}
}
{
\newrgbcolor{curcolor}{0 0 0}
\pscustom[linestyle=none,fillstyle=solid,fillcolor=curcolor]
{
\newpath
\moveto(169.23726205,190.74667095)
\curveto(172.36615327,190.74667095)(174.75548839,190.06400378)(176.40526739,188.69866942)
\curveto(178.08349087,187.36177954)(178.9226026,185.29955578)(178.9226026,182.51199815)
\lineto(178.9226026,166.98131991)
\lineto(174.48526596,166.98131991)
\lineto(173.24793171,170.1386556)
\lineto(173.07726491,170.1386556)
\curveto(172.08170862,168.88709911)(171.02926339,167.97687621)(169.91992923,167.4079869)
\curveto(168.81059507,166.83909758)(167.28881616,166.55465293)(165.35459249,166.55465293)
\curveto(163.2781465,166.55465293)(161.55725633,167.1519867)(160.19192198,168.34665426)
\curveto(158.82658763,169.54132182)(158.14392045,171.40443432)(158.14392045,173.93599176)
\curveto(158.14392045,176.41066027)(159.01147666,178.23110607)(160.74658906,179.39732916)
\curveto(162.48170146,180.56355225)(165.08437007,181.21777496)(168.55459488,181.35999729)
\lineto(172.60793123,181.48799739)
\lineto(172.60793123,182.51199815)
\curveto(172.60793123,183.73511017)(172.28081988,184.63111084)(171.62659717,185.20000015)
\curveto(171.00081892,185.76888946)(170.11904049,186.05333412)(168.98126186,186.05333412)
\curveto(167.84348324,186.05333412)(166.73414908,185.88266733)(165.65325938,185.54133374)
\curveto(164.57236969,185.22844462)(163.49147999,184.8302221)(162.4105903,184.34666618)
\lineto(160.31992208,188.65600273)
\curveto(161.5430341,189.28178097)(162.92259068,189.77955912)(164.45859183,190.14933717)
\curveto(165.99459297,190.54755969)(167.58748305,190.74667095)(169.23726205,190.74667095)
\closepath
\moveto(172.60793123,177.77599462)
\lineto(170.13326272,177.69066122)
\curveto(168.08526119,177.63377229)(166.66303791,177.26399424)(165.86659287,176.58132706)
\curveto(165.07014784,175.89865989)(164.67192532,175.00265922)(164.67192532,173.89332506)
\curveto(164.67192532,172.92621323)(164.95636997,172.22932382)(165.52525929,171.80265684)
\curveto(166.0941486,171.40443432)(166.83370471,171.20532306)(167.74392761,171.20532306)
\curveto(169.10926196,171.20532306)(170.26126282,171.60354558)(171.19993018,172.39999061)
\curveto(172.13859755,173.22488012)(172.60793123,174.37688098)(172.60793123,175.85599319)
\closepath
}
}
{
\newrgbcolor{curcolor}{0 0 0}
\pscustom[linestyle=none,fillstyle=solid,fillcolor=curcolor]
{
\newpath
\moveto(194.92261194,166.55465293)
\curveto(191.45238714,166.55465293)(188.76438513,167.50754253)(186.85860594,169.41332172)
\curveto(184.9812712,171.31910092)(184.04260384,174.34843651)(184.04260384,178.50132849)
\curveto(184.04260384,181.34577506)(184.52615975,183.66399901)(185.49327159,185.45600034)
\curveto(186.46038342,187.24800168)(187.7972733,188.57066933)(189.50394124,189.4240033)
\curveto(191.23905364,190.27733727)(193.23016624,190.70400425)(195.47727902,190.70400425)
\curveto(197.0701691,190.70400425)(198.44972568,190.54755969)(199.61594877,190.23467057)
\curveto(200.81061633,189.92178145)(201.84883933,189.55200339)(202.73061776,189.12533641)
\lineto(200.85328303,184.21866609)
\curveto(199.85772673,184.61688861)(198.91905937,184.94399996)(198.03728093,185.20000015)
\curveto(197.18394696,185.45600034)(196.33061299,185.58400044)(195.47727902,185.58400044)
\curveto(192.17772101,185.58400044)(190.527942,183.23733202)(190.527942,178.54399519)
\curveto(190.527942,176.21154901)(190.95460899,174.49065884)(191.80794296,173.38132468)
\curveto(192.68972139,172.27199052)(193.91283341,171.71732344)(195.47727902,171.71732344)
\curveto(196.81416891,171.71732344)(197.99461423,171.88799023)(199.018615,172.22932382)
\curveto(200.04261576,172.59910187)(201.03817206,173.09688002)(202.00528389,173.72265827)
\lineto(202.00528389,168.30398756)
\curveto(201.03817206,167.67820932)(200.01417129,167.2373201)(198.9332816,166.98131991)
\curveto(197.88083637,166.69687525)(196.54394649,166.55465293)(194.92261194,166.55465293)
\closepath
}
}
{
\newrgbcolor{curcolor}{0 0 0}
\pscustom[linestyle=none,fillstyle=solid,fillcolor=curcolor]
{
\newpath
\moveto(226.66663954,185.49866704)
\lineto(219.02930052,185.49866704)
\lineto(219.02930052,166.98131991)
\lineto(212.67196245,166.98131991)
\lineto(212.67196245,185.49866704)
\lineto(205.03462342,185.49866704)
\lineto(205.03462342,190.27733727)
\lineto(226.66663954,190.27733727)
\closepath
}
}
{
\newrgbcolor{curcolor}{0 0 0}
\pscustom[linestyle=none,fillstyle=solid,fillcolor=curcolor]
{
\newpath
\moveto(237.33328781,190.27733727)
\lineto(237.33328781,181.31733059)
\lineto(246.20796108,181.31733059)
\lineto(246.20796108,190.27733727)
\lineto(252.56529915,190.27733727)
\lineto(252.56529915,166.98131991)
\lineto(246.20796108,166.98131991)
\lineto(246.20796108,176.58132706)
\lineto(237.33328781,176.58132706)
\lineto(237.33328781,166.98131991)
\lineto(230.97594974,166.98131991)
\lineto(230.97594974,190.27733727)
\closepath
}
}
{
\newrgbcolor{curcolor}{0 0 0}
\pscustom[linestyle=none,fillstyle=solid,fillcolor=curcolor]
{
\newpath
\moveto(265.36534616,190.27733727)
\lineto(265.36534616,181.0613304)
\curveto(265.36534616,180.57777449)(265.33690169,179.98044071)(265.28001276,179.26932907)
\curveto(265.2515683,178.55821743)(265.2089016,177.83288355)(265.15201267,177.09332744)
\curveto(265.1235682,176.35377134)(265.0809015,175.6853264)(265.02401257,175.08799262)
\curveto(264.96712364,174.5191033)(264.92445694,174.13510302)(264.89601248,173.93599176)
\lineto(275.64802049,190.27733727)
\lineto(283.28535951,190.27733727)
\lineto(283.28535951,166.98131991)
\lineto(277.14135493,166.98131991)
\lineto(277.14135493,176.28266017)
\curveto(277.14135493,177.02221628)(277.1697994,177.86132802)(277.22668833,178.79999538)
\curveto(277.28357726,179.73866275)(277.34046619,180.60621895)(277.39735512,181.40266399)
\curveto(277.48268852,182.22755349)(277.53957745,182.85333174)(277.56802192,183.27999872)
\lineto(266.8586806,166.98131991)
\lineto(259.22134158,166.98131991)
\lineto(259.22134158,190.27733727)
\closepath
}
}
{
\newrgbcolor{curcolor}{0 0 0}
\pscustom[linestyle=none,fillstyle=solid,fillcolor=curcolor]
{
\newpath
\moveto(305.21600559,190.27733727)
\lineto(312.21334414,190.27733727)
\lineto(302.99733727,179.09866227)
\lineto(313.02401141,166.98131991)
\lineto(305.81333937,166.98131991)
\lineto(296.29866561,178.79999538)
\lineto(296.29866561,166.98131991)
\lineto(289.94132754,166.98131991)
\lineto(289.94132754,190.27733727)
\lineto(296.29866561,190.27733727)
\lineto(296.29866561,178.97066218)
\closepath
}
}
{
\newrgbcolor{curcolor}{0 0 0}
\pscustom[linestyle=none,fillstyle=solid,fillcolor=curcolor]
{
\newpath
\moveto(342.72001865,190.27733727)
\lineto(349.7173572,190.27733727)
\lineto(340.50135033,179.09866227)
\lineto(350.52802447,166.98131991)
\lineto(343.31735243,166.98131991)
\lineto(333.80267867,178.79999538)
\lineto(333.80267867,166.98131991)
\lineto(327.4453406,166.98131991)
\lineto(327.4453406,190.27733727)
\lineto(333.80267867,190.27733727)
\lineto(333.80267867,178.97066218)
\closepath
}
}
{
\newrgbcolor{curcolor}{0 0 0}
\pscustom[linestyle=none,fillstyle=solid,fillcolor=curcolor]
{
\newpath
\moveto(374.16533505,178.67199529)
\curveto(374.16533505,174.80354796)(373.14133429,171.81687907)(371.09333276,169.71198861)
\curveto(369.0737757,167.60709815)(366.31466253,166.55465293)(362.81599326,166.55465293)
\curveto(360.65421387,166.55465293)(358.71999021,167.02398661)(357.01332227,167.96265398)
\curveto(355.3350988,168.90132134)(354.01243115,170.26665569)(353.04531931,172.05865703)
\curveto(352.07820748,173.87910283)(351.59465157,176.08354891)(351.59465157,178.67199529)
\curveto(351.59465157,182.54044261)(352.6044301,185.51288927)(354.62398716,187.58933527)
\curveto(356.64354422,189.66578126)(359.41687962,190.70400425)(362.94399336,190.70400425)
\curveto(365.13421721,190.70400425)(367.06844087,190.23467057)(368.74666435,189.2960032)
\curveto(370.42488782,188.35733584)(371.74755547,186.99200149)(372.7146673,185.20000015)
\curveto(373.68177913,183.40799882)(374.16533505,181.2319972)(374.16533505,178.67199529)
\closepath
\moveto(358.07998973,178.67199529)
\curveto(358.07998973,176.36799357)(358.44976779,174.61865893)(359.18932389,173.42399138)
\curveto(359.95732446,172.25776829)(361.19465872,171.67465674)(362.90132666,171.67465674)
\curveto(364.57955013,171.67465674)(365.78843992,172.25776829)(366.52799603,173.42399138)
\curveto(367.2959966,174.61865893)(367.67999689,176.36799357)(367.67999689,178.67199529)
\curveto(367.67999689,180.975997)(367.2959966,182.69688718)(366.52799603,183.8346658)
\curveto(365.78843992,185.00088889)(364.5653279,185.58400044)(362.85865996,185.58400044)
\curveto(361.18043649,185.58400044)(359.95732446,185.00088889)(359.18932389,183.8346658)
\curveto(358.44976779,182.69688718)(358.07998973,180.975997)(358.07998973,178.67199529)
\closepath
}
}
{
\newrgbcolor{curcolor}{0 0 0}
\pscustom[linestyle=none,fillstyle=solid,fillcolor=curcolor]
{
\newpath
\moveto(408.7680162,190.27733727)
\lineto(408.7680162,166.98131991)
\lineto(402.83734512,166.98131991)
\lineto(402.83734512,178.4159951)
\curveto(402.83734512,179.55377372)(402.85156735,180.66310788)(402.88001182,181.74399758)
\curveto(402.93690075,182.82488727)(403.00801191,183.82044357)(403.09334531,184.73066647)
\lineto(402.96534521,184.73066647)
\lineto(396.52267375,166.98131991)
\lineto(391.74400352,166.98131991)
\lineto(385.21599866,184.77333317)
\lineto(385.04533186,184.77333317)
\curveto(385.15910972,183.8346658)(385.23022089,182.82488727)(385.25866535,181.74399758)
\curveto(385.31555428,180.69155235)(385.34399875,179.52532926)(385.34399875,178.2453283)
\lineto(385.34399875,166.98131991)
\lineto(379.41332767,166.98131991)
\lineto(379.41332767,190.27733727)
\lineto(388.41600104,190.27733727)
\lineto(394.21867203,174.49065884)
\lineto(400.10667642,190.27733727)
\closepath
}
}
{
\newrgbcolor{curcolor}{0 0 0}
\pscustom[linestyle=none,fillstyle=solid,fillcolor=curcolor]
{
\newpath
\moveto(421.78134689,190.27733727)
\lineto(421.78134689,181.31733059)
\lineto(430.65602017,181.31733059)
\lineto(430.65602017,190.27733727)
\lineto(437.01335824,190.27733727)
\lineto(437.01335824,166.98131991)
\lineto(430.65602017,166.98131991)
\lineto(430.65602017,176.58132706)
\lineto(421.78134689,176.58132706)
\lineto(421.78134689,166.98131991)
\lineto(415.42400882,166.98131991)
\lineto(415.42400882,190.27733727)
\closepath
}
}
{
\newrgbcolor{curcolor}{0 0 0}
\pscustom[linestyle=none,fillstyle=solid,fillcolor=curcolor]
{
\newpath
\moveto(453.22674112,190.74667095)
\curveto(456.35563234,190.74667095)(458.74496745,190.06400378)(460.39474646,188.69866942)
\curveto(462.07296993,187.36177954)(462.91208167,185.29955578)(462.91208167,182.51199815)
\lineto(462.91208167,166.98131991)
\lineto(458.47474503,166.98131991)
\lineto(457.23741077,170.1386556)
\lineto(457.06674398,170.1386556)
\curveto(456.07118768,168.88709911)(455.01874245,167.97687621)(453.90940829,167.4079869)
\curveto(452.80007413,166.83909758)(451.27829522,166.55465293)(449.34407156,166.55465293)
\curveto(447.26762557,166.55465293)(445.5467354,167.1519867)(444.18140104,168.34665426)
\curveto(442.81606669,169.54132182)(442.13339952,171.40443432)(442.13339952,173.93599176)
\curveto(442.13339952,176.41066027)(443.00095572,178.23110607)(444.73606812,179.39732916)
\curveto(446.47118053,180.56355225)(449.07384913,181.21777496)(452.54407394,181.35999729)
\lineto(456.5974103,181.48799739)
\lineto(456.5974103,182.51199815)
\curveto(456.5974103,183.73511017)(456.27029894,184.63111084)(455.61607623,185.20000015)
\curveto(454.99029799,185.76888946)(454.10851955,186.05333412)(452.97074093,186.05333412)
\curveto(451.8329623,186.05333412)(450.72362814,185.88266733)(449.64273845,185.54133374)
\curveto(448.56184875,185.22844462)(447.48095906,184.8302221)(446.40006936,184.34666618)
\lineto(444.30940114,188.65600273)
\curveto(445.53251316,189.28178097)(446.91206975,189.77955912)(448.44807089,190.14933717)
\curveto(449.98407203,190.54755969)(451.57696211,190.74667095)(453.22674112,190.74667095)
\closepath
\moveto(456.5974103,177.77599462)
\lineto(454.12274178,177.69066122)
\curveto(452.07474026,177.63377229)(450.65251698,177.26399424)(449.85607194,176.58132706)
\curveto(449.0596269,175.89865989)(448.66140438,175.00265922)(448.66140438,173.89332506)
\curveto(448.66140438,172.92621323)(448.94584904,172.22932382)(449.51473835,171.80265684)
\curveto(450.08362766,171.40443432)(450.82318377,171.20532306)(451.73340667,171.20532306)
\curveto(453.09874102,171.20532306)(454.25074188,171.60354558)(455.18940925,172.39999061)
\curveto(456.12807661,173.22488012)(456.5974103,174.37688098)(456.5974103,175.85599319)
\closepath
}
}
{
\newrgbcolor{curcolor}{0 0 0}
\pscustom[linestyle=none,fillstyle=solid,fillcolor=curcolor]
{
\newpath
\moveto(488.72543928,185.49866704)
\lineto(481.08810026,185.49866704)
\lineto(481.08810026,166.98131991)
\lineto(474.73076219,166.98131991)
\lineto(474.73076219,185.49866704)
\lineto(467.09342317,185.49866704)
\lineto(467.09342317,190.27733727)
\lineto(488.72543928,190.27733727)
\closepath
}
}
{
\newrgbcolor{curcolor}{0 0 0}
\pscustom[linestyle=none,fillstyle=solid,fillcolor=curcolor]
{
\newpath
\moveto(493.03475711,166.98131991)
\lineto(493.03475711,190.27733727)
\lineto(499.39209518,190.27733727)
\lineto(499.39209518,181.27466389)
\lineto(502.46409747,181.27466389)
\curveto(506.01965567,181.27466389)(508.65076874,180.70577458)(510.35743668,179.56799596)
\curveto(512.06410462,178.43021733)(512.91743859,176.70932716)(512.91743859,174.40532544)
\curveto(512.91743859,172.12976819)(512.12099355,170.32354462)(510.52810348,168.98665474)
\curveto(508.9352134,167.64976485)(506.31832256,166.98131991)(502.67743096,166.98131991)
\closepath
\moveto(516.28810777,166.98131991)
\lineto(516.28810777,190.27733727)
\lineto(522.64544584,190.27733727)
\lineto(522.64544584,166.98131991)
\closepath
\moveto(499.39209518,171.37598985)
\lineto(502.33609737,171.37598985)
\curveto(503.58765386,171.37598985)(504.59743239,171.58932334)(505.36543296,172.01599033)
\curveto(506.161878,172.47110178)(506.56010052,173.23910235)(506.56010052,174.31999205)
\curveto(506.56010052,176.02665998)(505.123655,176.87999395)(502.25076397,176.87999395)
\lineto(499.39209518,176.87999395)
\closepath
}
}
\end{pspicture}
}
		\caption{Свободная система прав}
		\label{fig:img1}
	\end{center}
\end{figure}

Ограниченная система прав [Рис.~\ref{fig:img2}] предполагает обязательное наличие владельца комнаты
и согласование с ним всех действий по передаче прав. В случае отсутствия владельца — возможность
редактирования доски «GeoGebra» приостанавливается.

\begin{figure}[H]
	\begin{center}
		\scalebox{0.6}{%LaTeX with PSTricks extensions
%%Creator: Inkscape 1.2 (dc2aedaf03, 2022-05-15)
%%Please note this file requires PSTricks extensions
\psset{xunit=.5pt,yunit=.5pt,runit=.5pt}
\begin{pspicture}(1024,768)
{
\newrgbcolor{curcolor}{0.50196081 0.50196081 0.50196081}
\pscustom[linestyle=none,fillstyle=solid,fillcolor=curcolor]
{
\newpath
\moveto(9.11873341,757.86807346)
\lineto(1015.21901417,757.86807346)
\lineto(1015.21901417,615.00792027)
\lineto(9.11873341,615.00792027)
\closepath
}
}
{
\newrgbcolor{curcolor}{0.50196081 0.50196081 0.50196081}
\pscustom[linestyle=none,fillstyle=solid,fillcolor=curcolor]
{
\newpath
\moveto(8.61212349,149.44592285)
\lineto(1014.71240425,149.44592285)
\lineto(1014.71240425,6.58576965)
\lineto(8.61212349,6.58576965)
\closepath
}
}
{
\newrgbcolor{curcolor}{0.50196081 0.50196081 0.50196081}
\pscustom[linestyle=none,fillstyle=solid,fillcolor=curcolor]
{
\newpath
\moveto(8.6121273,445.24536133)
\lineto(1014.71240807,445.24536133)
\lineto(1014.71240807,302.38520813)
\lineto(8.6121273,302.38520813)
\closepath
}
}
{
\newrgbcolor{curcolor}{0.50196081 0.50196081 0.50196081}
\pscustom[linestyle=none,fillstyle=solid,fillcolor=curcolor]
{
\newpath
\moveto(30.39577866,609.94195557)
\lineto(45.59366798,609.94195557)
\lineto(45.59366798,465.05540466)
\lineto(30.39577866,465.05540466)
\closepath
}
}
{
\newrgbcolor{curcolor}{0.50196081 0.50196081 0.50196081}
\pscustom[linestyle=none,fillstyle=solid,fillcolor=curcolor]
{
\newpath
\moveto(23.284149,467.99269)
\lineto(53.463989,468.26135)
\lineto(38.329292,449.81314)
\lineto(38.329292,449.81314)
\closepath
}
}
{
\newrgbcolor{curcolor}{0.50196081 0.50196081 0.50196081}
\pscustom[linestyle=none,fillstyle=solid,fillcolor=curcolor]
{
\newpath
\moveto(31.42565536,297.40866089)
\lineto(46.62354469,297.40866089)
\lineto(46.62354469,167.28443909)
\lineto(31.42565536,167.28443909)
\closepath
}
}
{
\newrgbcolor{curcolor}{0.50196081 0.50196081 0.50196081}
\pscustom[linestyle=none,fillstyle=solid,fillcolor=curcolor]
{
\newpath
\moveto(24.314026,169.92244)
\lineto(54.493866,170.16372)
\lineto(39.359169,153.59518)
\lineto(39.359169,153.59518)
\closepath
}
}
{
\newrgbcolor{curcolor}{0 0 0}
\pscustom[linestyle=none,fillstyle=solid,fillcolor=curcolor]
{
\newpath
\moveto(240.0053225,710.88383798)
\lineto(225.8613225,678.56383798)
\curveto(224.5813225,675.61983798)(223.21598917,673.10250464)(221.7653225,671.01183798)
\curveto(220.31465583,668.92117131)(218.47998917,667.32117131)(216.2613225,666.21183798)
\curveto(214.0853225,665.10250464)(211.2053225,664.54783798)(207.6213225,664.54783798)
\curveto(206.51198917,664.54783798)(205.29598917,664.63317131)(203.9733225,664.80383798)
\curveto(202.65065583,664.97450464)(201.43465583,665.20917131)(200.3253225,665.50783798)
\lineto(200.3253225,673.82783798)
\curveto(201.3493225,673.40117131)(202.45865583,673.10250464)(203.6533225,672.93183798)
\curveto(204.89065583,672.76117131)(206.06398917,672.67583798)(207.1733225,672.67583798)
\curveto(209.30665583,672.67583798)(210.84265583,673.18783798)(211.7813225,674.21183798)
\curveto(212.71998917,675.27850464)(213.46665583,676.55850464)(214.0213225,678.05183798)
\lineto(198.2133225,710.88383798)
\lineto(208.4533225,710.88383798)
\lineto(216.9653225,691.10783798)
\curveto(217.26398917,690.46783798)(217.6693225,689.55050464)(218.1813225,688.35583798)
\curveto(218.6933225,687.20383798)(219.0773225,686.22250464)(219.3333225,685.41183798)
\lineto(219.6533225,685.41183798)
\curveto(219.9093225,686.17983798)(220.27198917,687.18250464)(220.7413225,688.41983798)
\curveto(221.2533225,689.65717131)(221.7013225,690.74517131)(222.0853225,691.68383798)
\lineto(230.0213225,710.88383798)
\closepath
}
}
{
\newrgbcolor{curcolor}{0 0 0}
\pscustom[linestyle=none,fillstyle=solid,fillcolor=curcolor]
{
\newpath
\moveto(252.10130004,700.13183798)
\lineto(252.10130004,687.33183798)
\curveto(252.10130004,684.30250464)(253.50930004,682.78783798)(256.32530004,682.78783798)
\curveto(258.15996671,682.78783798)(259.86663337,682.97983798)(261.44530004,683.36383798)
\curveto(263.02396671,683.79050464)(264.60263337,684.34517131)(266.18130004,685.02783798)
\lineto(266.18130004,700.13183798)
\lineto(275.71730004,700.13183798)
\lineto(275.71730004,665.18783798)
\lineto(266.18130004,665.18783798)
\lineto(266.18130004,679.07583798)
\curveto(264.68796671,678.26517131)(262.98130004,677.51850464)(261.06130004,676.83583798)
\curveto(259.14130004,676.19583798)(256.96530004,675.87583798)(254.53330004,675.87583798)
\curveto(250.90663337,675.87583798)(248.00530004,676.79317131)(245.82930004,678.62783798)
\curveto(243.65330004,680.50517131)(242.56530004,683.34250464)(242.56530004,687.13983798)
\lineto(242.56530004,700.13183798)
\closepath
}
}
{
\newrgbcolor{curcolor}{0 0 0}
\pscustom[linestyle=none,fillstyle=solid,fillcolor=curcolor]
{
\newpath
\moveto(300.03730883,700.83583798)
\curveto(304.73064216,700.83583798)(308.31464216,699.81183798)(310.78930883,697.76383798)
\curveto(313.30664216,695.75850464)(314.56530883,692.66517131)(314.56530883,688.48383798)
\lineto(314.56530883,665.18783798)
\lineto(307.90930883,665.18783798)
\lineto(306.05330883,669.92383798)
\lineto(305.79730883,669.92383798)
\curveto(304.30397549,668.04650464)(302.72530883,666.68117131)(301.06130883,665.82783798)
\curveto(299.39730883,664.97450464)(297.11464216,664.54783798)(294.21330883,664.54783798)
\curveto(291.09864216,664.54783798)(288.51730883,665.44383798)(286.46930883,667.23583798)
\curveto(284.42130883,669.02783798)(283.39730883,671.82250464)(283.39730883,675.61983798)
\curveto(283.39730883,679.33183798)(284.69864216,682.06250464)(287.30130883,683.81183798)
\curveto(289.90397549,685.56117131)(293.80797549,686.54250464)(299.01330883,686.75583798)
\lineto(305.09330883,686.94783798)
\lineto(305.09330883,688.48383798)
\curveto(305.09330883,690.31850464)(304.60264216,691.66250464)(303.62130883,692.51583798)
\curveto(302.68264216,693.36917131)(301.35997549,693.79583798)(299.65330883,693.79583798)
\curveto(297.94664216,693.79583798)(296.28264216,693.53983798)(294.66130883,693.02783798)
\curveto(293.03997549,692.55850464)(291.41864216,691.96117131)(289.79730883,691.23583798)
\lineto(286.66130883,697.69983798)
\curveto(288.49597549,698.63850464)(290.56530883,699.38517131)(292.86930883,699.93983798)
\curveto(295.17330883,700.53717131)(297.56264216,700.83583798)(300.03730883,700.83583798)
\closepath
\moveto(305.09330883,681.37983798)
\lineto(301.38130883,681.25183798)
\curveto(298.30930883,681.16650464)(296.17597549,680.61183798)(294.98130883,679.58783798)
\curveto(293.78664216,678.56383798)(293.18930883,677.21983798)(293.18930883,675.55583798)
\curveto(293.18930883,674.10517131)(293.61597549,673.05983798)(294.46930883,672.41983798)
\curveto(295.32264216,671.82250464)(296.43197549,671.52383798)(297.79730883,671.52383798)
\curveto(299.84530883,671.52383798)(301.57330883,672.12117131)(302.98130883,673.31583798)
\curveto(304.38930883,674.55317131)(305.09330883,676.28117131)(305.09330883,678.49983798)
\closepath
}
}
{
\newrgbcolor{curcolor}{0 0 0}
\pscustom[linestyle=none,fillstyle=solid,fillcolor=curcolor]
{
\newpath
\moveto(338.56531469,664.54783798)
\curveto(333.35998135,664.54783798)(329.32798135,665.97717131)(326.46931469,668.83583798)
\curveto(323.65331469,671.69450464)(322.24531469,676.23850464)(322.24531469,682.46783798)
\curveto(322.24531469,686.73450464)(322.97064802,690.21183798)(324.42131469,692.89983798)
\curveto(325.87198135,695.58783798)(327.87731469,697.57183798)(330.43731469,698.85183798)
\curveto(333.03998135,700.13183798)(336.02664802,700.77183798)(339.39731469,700.77183798)
\curveto(341.78664802,700.77183798)(343.85598135,700.53717131)(345.60531469,700.06783798)
\curveto(347.39731469,699.59850464)(348.95464802,699.04383798)(350.27731469,698.40383798)
\lineto(347.46131469,691.04383798)
\curveto(345.96798135,691.64117131)(344.55998135,692.13183798)(343.23731469,692.51583798)
\curveto(341.95731469,692.89983798)(340.67731469,693.09183798)(339.39731469,693.09183798)
\curveto(334.44798135,693.09183798)(331.97331469,689.57183798)(331.97331469,682.53183798)
\curveto(331.97331469,679.03317131)(332.61331469,676.45183798)(333.89331469,674.78783798)
\curveto(335.21598135,673.12383798)(337.05064802,672.29183798)(339.39731469,672.29183798)
\curveto(341.40264802,672.29183798)(343.17331469,672.54783798)(344.70931469,673.05983798)
\curveto(346.24531469,673.61450464)(347.73864802,674.36117131)(349.18931469,675.29983798)
\lineto(349.18931469,667.17183798)
\curveto(347.73864802,666.23317131)(346.20264802,665.57183798)(344.58131469,665.18783798)
\curveto(343.00264802,664.76117131)(340.99731469,664.54783798)(338.56531469,664.54783798)
\closepath
}
}
{
\newrgbcolor{curcolor}{0 0 0}
\pscustom[linestyle=none,fillstyle=solid,fillcolor=curcolor]
{
\newpath
\moveto(386.18131078,692.96383798)
\lineto(374.72531078,692.96383798)
\lineto(374.72531078,665.18783798)
\lineto(365.18931078,665.18783798)
\lineto(365.18931078,692.96383798)
\lineto(353.73331078,692.96383798)
\lineto(353.73331078,700.13183798)
\lineto(386.18131078,700.13183798)
\closepath
}
}
{
\newrgbcolor{curcolor}{0 0 0}
\pscustom[linestyle=none,fillstyle=solid,fillcolor=curcolor]
{
\newpath
\moveto(402.18127855,700.13183798)
\lineto(402.18127855,686.69183798)
\lineto(415.49327855,686.69183798)
\lineto(415.49327855,700.13183798)
\lineto(425.02927855,700.13183798)
\lineto(425.02927855,665.18783798)
\lineto(415.49327855,665.18783798)
\lineto(415.49327855,679.58783798)
\lineto(402.18127855,679.58783798)
\lineto(402.18127855,665.18783798)
\lineto(392.64527855,665.18783798)
\lineto(392.64527855,700.13183798)
\closepath
}
}
{
\newrgbcolor{curcolor}{0 0 0}
\pscustom[linestyle=none,fillstyle=solid,fillcolor=curcolor]
{
\newpath
\moveto(444.22932055,700.13183798)
\lineto(444.22932055,686.30783798)
\curveto(444.22932055,685.58250464)(444.18665388,684.68650464)(444.10132055,683.61983798)
\curveto(444.05865388,682.55317131)(443.99465388,681.46517131)(443.90932055,680.35583798)
\curveto(443.86665388,679.24650464)(443.80265388,678.24383798)(443.71732055,677.34783798)
\curveto(443.63198721,676.49450464)(443.56798721,675.91850464)(443.52532055,675.61983798)
\lineto(459.65332055,700.13183798)
\lineto(471.10932055,700.13183798)
\lineto(471.10932055,665.18783798)
\lineto(461.89332055,665.18783798)
\lineto(461.89332055,679.13983798)
\curveto(461.89332055,680.24917131)(461.93598721,681.50783798)(462.02132055,682.91583798)
\curveto(462.10665388,684.32383798)(462.19198721,685.62517131)(462.27732055,686.81983798)
\curveto(462.40532055,688.05717131)(462.49065388,688.99583798)(462.53332055,689.63583798)
\lineto(446.46932055,665.18783798)
\lineto(435.01332055,665.18783798)
\lineto(435.01332055,700.13183798)
\closepath
}
}
{
\newrgbcolor{curcolor}{0 0 0}
\pscustom[linestyle=none,fillstyle=solid,fillcolor=curcolor]
{
\newpath
\moveto(504.0052766,700.13183798)
\lineto(514.5012766,700.13183798)
\lineto(500.6772766,683.36383798)
\lineto(515.7172766,665.18783798)
\lineto(504.9012766,665.18783798)
\lineto(490.6292766,682.91583798)
\lineto(490.6292766,665.18783798)
\lineto(481.0932766,665.18783798)
\lineto(481.0932766,700.13183798)
\lineto(490.6292766,700.13183798)
\lineto(490.6292766,683.17183798)
\closepath
}
}
{
\newrgbcolor{curcolor}{0 0 0}
\pscustom[linestyle=none,fillstyle=solid,fillcolor=curcolor]
{
\newpath
\moveto(560.26125805,700.13183798)
\lineto(570.75725805,700.13183798)
\lineto(556.93325805,683.36383798)
\lineto(571.97325805,665.18783798)
\lineto(561.15725805,665.18783798)
\lineto(546.88525805,682.91583798)
\lineto(546.88525805,665.18783798)
\lineto(537.34925805,665.18783798)
\lineto(537.34925805,700.13183798)
\lineto(546.88525805,700.13183798)
\lineto(546.88525805,683.17183798)
\closepath
}
}
{
\newrgbcolor{curcolor}{0 0 0}
\pscustom[linestyle=none,fillstyle=solid,fillcolor=curcolor]
{
\newpath
\moveto(607.42919555,682.72383798)
\curveto(607.42919555,676.92117131)(605.89319555,672.44117131)(602.82119555,669.28383798)
\curveto(599.79186221,666.12650464)(595.65319555,664.54783798)(590.40519555,664.54783798)
\curveto(587.16252888,664.54783798)(584.26119555,665.25183798)(581.70119555,666.65983798)
\curveto(579.18386221,668.06783798)(577.19986221,670.11583798)(575.74919555,672.80383798)
\curveto(574.29852888,675.53450464)(573.57319555,678.84117131)(573.57319555,682.72383798)
\curveto(573.57319555,688.52650464)(575.08786221,692.98517131)(578.11719555,696.09983798)
\curveto(581.14652888,699.21450464)(585.30652888,700.77183798)(590.59719555,700.77183798)
\curveto(593.88252888,700.77183798)(596.78386221,700.06783798)(599.30119555,698.65983798)
\curveto(601.81852888,697.25183798)(603.80252888,695.20383798)(605.25319555,692.51583798)
\curveto(606.70386221,689.82783798)(607.42919555,686.56383798)(607.42919555,682.72383798)
\closepath
\moveto(583.30119555,682.72383798)
\curveto(583.30119555,679.26783798)(583.85586221,676.64383798)(584.96519555,674.85183798)
\curveto(586.11719555,673.10250464)(587.97319555,672.22783798)(590.53319555,672.22783798)
\curveto(593.05052888,672.22783798)(594.86386221,673.10250464)(595.97319555,674.85183798)
\curveto(597.12519555,676.64383798)(597.70119555,679.26783798)(597.70119555,682.72383798)
\curveto(597.70119555,686.17983798)(597.12519555,688.76117131)(595.97319555,690.46783798)
\curveto(594.86386221,692.21717131)(593.02919555,693.09183798)(590.46919555,693.09183798)
\curveto(587.95186221,693.09183798)(586.11719555,692.21717131)(584.96519555,690.46783798)
\curveto(583.85586221,688.76117131)(583.30119555,686.17983798)(583.30119555,682.72383798)
\closepath
}
}
{
\newrgbcolor{curcolor}{0 0 0}
\pscustom[linestyle=none,fillstyle=solid,fillcolor=curcolor]
{
\newpath
\moveto(659.33316234,700.13183798)
\lineto(659.33316234,665.18783798)
\lineto(650.43716234,665.18783798)
\lineto(650.43716234,682.33983798)
\curveto(650.43716234,684.04650464)(650.45849568,685.71050464)(650.50116234,687.33183798)
\curveto(650.58649568,688.95317131)(650.69316234,690.44650464)(650.82116234,691.81183798)
\lineto(650.62916234,691.81183798)
\lineto(640.96516234,665.18783798)
\lineto(633.79716234,665.18783798)
\lineto(624.00516234,691.87583798)
\lineto(623.74916234,691.87583798)
\curveto(623.91982901,690.46783798)(624.02649568,688.95317131)(624.06916234,687.33183798)
\curveto(624.15449568,685.75317131)(624.19716234,684.00383798)(624.19716234,682.08383798)
\lineto(624.19716234,665.18783798)
\lineto(615.30116234,665.18783798)
\lineto(615.30116234,700.13183798)
\lineto(628.80516234,700.13183798)
\lineto(637.50916234,676.45183798)
\lineto(646.34116234,700.13183798)
\closepath
}
}
{
\newrgbcolor{curcolor}{0 0 0}
\pscustom[linestyle=none,fillstyle=solid,fillcolor=curcolor]
{
\newpath
\moveto(678.85315355,700.13183798)
\lineto(678.85315355,686.69183798)
\lineto(692.16515355,686.69183798)
\lineto(692.16515355,700.13183798)
\lineto(701.70115355,700.13183798)
\lineto(701.70115355,665.18783798)
\lineto(692.16515355,665.18783798)
\lineto(692.16515355,679.58783798)
\lineto(678.85315355,679.58783798)
\lineto(678.85315355,665.18783798)
\lineto(669.31715355,665.18783798)
\lineto(669.31715355,700.13183798)
\closepath
}
}
{
\newrgbcolor{curcolor}{0 0 0}
\pscustom[linestyle=none,fillstyle=solid,fillcolor=curcolor]
{
\newpath
\moveto(726.02119555,700.83583798)
\curveto(730.71452888,700.83583798)(734.29852888,699.81183798)(736.77319555,697.76383798)
\curveto(739.29052888,695.75850464)(740.54919555,692.66517131)(740.54919555,688.48383798)
\lineto(740.54919555,665.18783798)
\lineto(733.89319555,665.18783798)
\lineto(732.03719555,669.92383798)
\lineto(731.78119555,669.92383798)
\curveto(730.28786221,668.04650464)(728.70919555,666.68117131)(727.04519555,665.82783798)
\curveto(725.38119555,664.97450464)(723.09852888,664.54783798)(720.19719555,664.54783798)
\curveto(717.08252888,664.54783798)(714.50119555,665.44383798)(712.45319555,667.23583798)
\curveto(710.40519555,669.02783798)(709.38119555,671.82250464)(709.38119555,675.61983798)
\curveto(709.38119555,679.33183798)(710.68252888,682.06250464)(713.28519555,683.81183798)
\curveto(715.88786221,685.56117131)(719.79186221,686.54250464)(724.99719555,686.75583798)
\lineto(731.07719555,686.94783798)
\lineto(731.07719555,688.48383798)
\curveto(731.07719555,690.31850464)(730.58652888,691.66250464)(729.60519555,692.51583798)
\curveto(728.66652888,693.36917131)(727.34386221,693.79583798)(725.63719555,693.79583798)
\curveto(723.93052888,693.79583798)(722.26652888,693.53983798)(720.64519555,693.02783798)
\curveto(719.02386221,692.55850464)(717.40252888,691.96117131)(715.78119555,691.23583798)
\lineto(712.64519555,697.69983798)
\curveto(714.47986221,698.63850464)(716.54919555,699.38517131)(718.85319555,699.93983798)
\curveto(721.15719555,700.53717131)(723.54652888,700.83583798)(726.02119555,700.83583798)
\closepath
\moveto(731.07719555,681.37983798)
\lineto(727.36519555,681.25183798)
\curveto(724.29319555,681.16650464)(722.15986221,680.61183798)(720.96519555,679.58783798)
\curveto(719.77052888,678.56383798)(719.17319555,677.21983798)(719.17319555,675.55583798)
\curveto(719.17319555,674.10517131)(719.59986221,673.05983798)(720.45319555,672.41983798)
\curveto(721.30652888,671.82250464)(722.41586221,671.52383798)(723.78119555,671.52383798)
\curveto(725.82919555,671.52383798)(727.55719555,672.12117131)(728.96519555,673.31583798)
\curveto(730.37319555,674.55317131)(731.07719555,676.28117131)(731.07719555,678.49983798)
\closepath
}
}
{
\newrgbcolor{curcolor}{0 0 0}
\pscustom[linestyle=none,fillstyle=solid,fillcolor=curcolor]
{
\newpath
\moveto(779.26920141,692.96383798)
\lineto(767.81320141,692.96383798)
\lineto(767.81320141,665.18783798)
\lineto(758.27720141,665.18783798)
\lineto(758.27720141,692.96383798)
\lineto(746.82120141,692.96383798)
\lineto(746.82120141,700.13183798)
\lineto(779.26920141,700.13183798)
\closepath
}
}
{
\newrgbcolor{curcolor}{0 0 0}
\pscustom[linestyle=none,fillstyle=solid,fillcolor=curcolor]
{
\newpath
\moveto(785.73316918,665.18783798)
\lineto(785.73316918,700.13183798)
\lineto(795.26916918,700.13183798)
\lineto(795.26916918,686.62783798)
\lineto(799.87716918,686.62783798)
\curveto(805.21050251,686.62783798)(809.15716918,685.77450464)(811.71716918,684.06783798)
\curveto(814.27716918,682.36117131)(815.55716918,679.77983798)(815.55716918,676.32383798)
\curveto(815.55716918,672.91050464)(814.36250251,670.20117131)(811.97316918,668.19583798)
\curveto(809.58383585,666.19050464)(805.65850251,665.18783798)(800.19716918,665.18783798)
\closepath
\moveto(820.61316918,665.18783798)
\lineto(820.61316918,700.13183798)
\lineto(830.14916918,700.13183798)
\lineto(830.14916918,665.18783798)
\closepath
\moveto(795.26916918,671.77983798)
\lineto(799.68516918,671.77983798)
\curveto(801.56250251,671.77983798)(803.07716918,672.09983798)(804.22916918,672.73983798)
\curveto(805.42383585,673.42250464)(806.02116918,674.57450464)(806.02116918,676.19583798)
\curveto(806.02116918,678.75583798)(803.86650251,680.03583798)(799.55716918,680.03583798)
\lineto(795.26916918,680.03583798)
\closepath
}
}
{
\newrgbcolor{curcolor}{0 0 0}
\pscustom[linestyle=none,fillstyle=solid,fillcolor=curcolor]
{
\newpath
\moveto(246.6050795,95.10467888)
\lineto(232.4610795,62.78467888)
\curveto(231.1810795,59.84067888)(229.81574617,57.32334554)(228.3650795,55.23267888)
\curveto(226.91441283,53.14201221)(225.07974617,51.54201221)(222.8610795,50.43267888)
\curveto(220.6850795,49.32334554)(217.8050795,48.76867888)(214.2210795,48.76867888)
\curveto(213.11174617,48.76867888)(211.89574617,48.85401221)(210.5730795,49.02467888)
\curveto(209.25041283,49.19534554)(208.03441283,49.43001221)(206.9250795,49.72867888)
\lineto(206.9250795,58.04867888)
\curveto(207.9490795,57.62201221)(209.05841283,57.32334554)(210.2530795,57.15267888)
\curveto(211.49041283,56.98201221)(212.66374617,56.89667888)(213.7730795,56.89667888)
\curveto(215.90641283,56.89667888)(217.44241283,57.40867888)(218.3810795,58.43267888)
\curveto(219.31974617,59.49934554)(220.06641283,60.77934554)(220.6210795,62.27267888)
\lineto(204.8130795,95.10467888)
\lineto(215.0530795,95.10467888)
\lineto(223.5650795,75.32867888)
\curveto(223.86374617,74.68867888)(224.2690795,73.77134554)(224.7810795,72.57667888)
\curveto(225.2930795,71.42467888)(225.6770795,70.44334554)(225.9330795,69.63267888)
\lineto(226.2530795,69.63267888)
\curveto(226.5090795,70.40067888)(226.87174617,71.40334554)(227.3410795,72.64067888)
\curveto(227.8530795,73.87801221)(228.3010795,74.96601221)(228.6850795,75.90467888)
\lineto(236.6210795,95.10467888)
\closepath
}
}
{
\newrgbcolor{curcolor}{0 0 0}
\pscustom[linestyle=none,fillstyle=solid,fillcolor=curcolor]
{
\newpath
\moveto(258.70105704,84.35267888)
\lineto(258.70105704,71.55267888)
\curveto(258.70105704,68.52334554)(260.10905704,67.00867888)(262.92505704,67.00867888)
\curveto(264.75972371,67.00867888)(266.46639037,67.20067888)(268.04505704,67.58467888)
\curveto(269.62372371,68.01134554)(271.20239037,68.56601221)(272.78105704,69.24867888)
\lineto(272.78105704,84.35267888)
\lineto(282.31705704,84.35267888)
\lineto(282.31705704,49.40867888)
\lineto(272.78105704,49.40867888)
\lineto(272.78105704,63.29667888)
\curveto(271.28772371,62.48601221)(269.58105704,61.73934554)(267.66105704,61.05667888)
\curveto(265.74105704,60.41667888)(263.56505704,60.09667888)(261.13305704,60.09667888)
\curveto(257.50639037,60.09667888)(254.60505704,61.01401221)(252.42905704,62.84867888)
\curveto(250.25305704,64.72601221)(249.16505704,67.56334554)(249.16505704,71.36067888)
\lineto(249.16505704,84.35267888)
\closepath
}
}
{
\newrgbcolor{curcolor}{0 0 0}
\pscustom[linestyle=none,fillstyle=solid,fillcolor=curcolor]
{
\newpath
\moveto(306.63706583,85.05667888)
\curveto(311.33039916,85.05667888)(314.91439916,84.03267888)(317.38906583,81.98467888)
\curveto(319.90639916,79.97934554)(321.16506583,76.88601221)(321.16506583,72.70467888)
\lineto(321.16506583,49.40867888)
\lineto(314.50906583,49.40867888)
\lineto(312.65306583,54.14467888)
\lineto(312.39706583,54.14467888)
\curveto(310.90373249,52.26734554)(309.32506583,50.90201221)(307.66106583,50.04867888)
\curveto(305.99706583,49.19534554)(303.71439916,48.76867888)(300.81306583,48.76867888)
\curveto(297.69839916,48.76867888)(295.11706583,49.66467888)(293.06906583,51.45667888)
\curveto(291.02106583,53.24867888)(289.99706583,56.04334554)(289.99706583,59.84067888)
\curveto(289.99706583,63.55267888)(291.29839916,66.28334554)(293.90106583,68.03267888)
\curveto(296.50373249,69.78201221)(300.40773249,70.76334554)(305.61306583,70.97667888)
\lineto(311.69306583,71.16867888)
\lineto(311.69306583,72.70467888)
\curveto(311.69306583,74.53934554)(311.20239916,75.88334554)(310.22106583,76.73667888)
\curveto(309.28239916,77.59001221)(307.95973249,78.01667888)(306.25306583,78.01667888)
\curveto(304.54639916,78.01667888)(302.88239916,77.76067888)(301.26106583,77.24867888)
\curveto(299.63973249,76.77934554)(298.01839916,76.18201221)(296.39706583,75.45667888)
\lineto(293.26106583,81.92067888)
\curveto(295.09573249,82.85934554)(297.16506583,83.60601221)(299.46906583,84.16067888)
\curveto(301.77306583,84.75801221)(304.16239916,85.05667888)(306.63706583,85.05667888)
\closepath
\moveto(311.69306583,65.60067888)
\lineto(307.98106583,65.47267888)
\curveto(304.90906583,65.38734554)(302.77573249,64.83267888)(301.58106583,63.80867888)
\curveto(300.38639916,62.78467888)(299.78906583,61.44067888)(299.78906583,59.77667888)
\curveto(299.78906583,58.32601221)(300.21573249,57.28067888)(301.06906583,56.64067888)
\curveto(301.92239916,56.04334554)(303.03173249,55.74467888)(304.39706583,55.74467888)
\curveto(306.44506583,55.74467888)(308.17306583,56.34201221)(309.58106583,57.53667888)
\curveto(310.98906583,58.77401221)(311.69306583,60.50201221)(311.69306583,62.72067888)
\closepath
}
}
{
\newrgbcolor{curcolor}{0 0 0}
\pscustom[linestyle=none,fillstyle=solid,fillcolor=curcolor]
{
\newpath
\moveto(345.16507169,48.76867888)
\curveto(339.95973835,48.76867888)(335.92773835,50.19801221)(333.06907169,53.05667888)
\curveto(330.25307169,55.91534554)(328.84507169,60.45934554)(328.84507169,66.68867888)
\curveto(328.84507169,70.95534554)(329.57040502,74.43267888)(331.02107169,77.12067888)
\curveto(332.47173835,79.80867888)(334.47707169,81.79267888)(337.03707169,83.07267888)
\curveto(339.63973835,84.35267888)(342.62640502,84.99267888)(345.99707169,84.99267888)
\curveto(348.38640502,84.99267888)(350.45573835,84.75801221)(352.20507169,84.28867888)
\curveto(353.99707169,83.81934554)(355.55440502,83.26467888)(356.87707169,82.62467888)
\lineto(354.06107169,75.26467888)
\curveto(352.56773835,75.86201221)(351.15973835,76.35267888)(349.83707169,76.73667888)
\curveto(348.55707169,77.12067888)(347.27707169,77.31267888)(345.99707169,77.31267888)
\curveto(341.04773835,77.31267888)(338.57307169,73.79267888)(338.57307169,66.75267888)
\curveto(338.57307169,63.25401221)(339.21307169,60.67267888)(340.49307169,59.00867888)
\curveto(341.81573835,57.34467888)(343.65040502,56.51267888)(345.99707169,56.51267888)
\curveto(348.00240502,56.51267888)(349.77307169,56.76867888)(351.30907169,57.28067888)
\curveto(352.84507169,57.83534554)(354.33840502,58.58201221)(355.78907169,59.52067888)
\lineto(355.78907169,51.39267888)
\curveto(354.33840502,50.45401221)(352.80240502,49.79267888)(351.18107169,49.40867888)
\curveto(349.60240502,48.98201221)(347.59707169,48.76867888)(345.16507169,48.76867888)
\closepath
}
}
{
\newrgbcolor{curcolor}{0 0 0}
\pscustom[linestyle=none,fillstyle=solid,fillcolor=curcolor]
{
\newpath
\moveto(392.78106778,77.18467888)
\lineto(381.32506778,77.18467888)
\lineto(381.32506778,49.40867888)
\lineto(371.78906778,49.40867888)
\lineto(371.78906778,77.18467888)
\lineto(360.33306778,77.18467888)
\lineto(360.33306778,84.35267888)
\lineto(392.78106778,84.35267888)
\closepath
}
}
{
\newrgbcolor{curcolor}{0 0 0}
\pscustom[linestyle=none,fillstyle=solid,fillcolor=curcolor]
{
\newpath
\moveto(408.78103555,84.35267888)
\lineto(408.78103555,70.91267888)
\lineto(422.09303555,70.91267888)
\lineto(422.09303555,84.35267888)
\lineto(431.62903555,84.35267888)
\lineto(431.62903555,49.40867888)
\lineto(422.09303555,49.40867888)
\lineto(422.09303555,63.80867888)
\lineto(408.78103555,63.80867888)
\lineto(408.78103555,49.40867888)
\lineto(399.24503555,49.40867888)
\lineto(399.24503555,84.35267888)
\closepath
}
}
{
\newrgbcolor{curcolor}{0 0 0}
\pscustom[linestyle=none,fillstyle=solid,fillcolor=curcolor]
{
\newpath
\moveto(450.82907755,84.35267888)
\lineto(450.82907755,70.52867888)
\curveto(450.82907755,69.80334554)(450.78641088,68.90734554)(450.70107755,67.84067888)
\curveto(450.65841088,66.77401221)(450.59441088,65.68601221)(450.50907755,64.57667888)
\curveto(450.46641088,63.46734554)(450.40241088,62.46467888)(450.31707755,61.56867888)
\curveto(450.23174421,60.71534554)(450.16774421,60.13934554)(450.12507755,59.84067888)
\lineto(466.25307755,84.35267888)
\lineto(477.70907755,84.35267888)
\lineto(477.70907755,49.40867888)
\lineto(468.49307755,49.40867888)
\lineto(468.49307755,63.36067888)
\curveto(468.49307755,64.47001221)(468.53574421,65.72867888)(468.62107755,67.13667888)
\curveto(468.70641088,68.54467888)(468.79174421,69.84601221)(468.87707755,71.04067888)
\curveto(469.00507755,72.27801221)(469.09041088,73.21667888)(469.13307755,73.85667888)
\lineto(453.06907755,49.40867888)
\lineto(441.61307755,49.40867888)
\lineto(441.61307755,84.35267888)
\closepath
}
}
{
\newrgbcolor{curcolor}{0 0 0}
\pscustom[linestyle=none,fillstyle=solid,fillcolor=curcolor]
{
\newpath
\moveto(510.6050336,84.35267888)
\lineto(521.1010336,84.35267888)
\lineto(507.2770336,67.58467888)
\lineto(522.3170336,49.40867888)
\lineto(511.5010336,49.40867888)
\lineto(497.2290336,67.13667888)
\lineto(497.2290336,49.40867888)
\lineto(487.6930336,49.40867888)
\lineto(487.6930336,84.35267888)
\lineto(497.2290336,84.35267888)
\lineto(497.2290336,67.39267888)
\closepath
}
}
{
\newrgbcolor{curcolor}{0 0 0}
\pscustom[linestyle=none,fillstyle=solid,fillcolor=curcolor]
{
\newpath
\moveto(566.86101505,84.35267888)
\lineto(577.35701505,84.35267888)
\lineto(563.53301505,67.58467888)
\lineto(578.57301505,49.40867888)
\lineto(567.75701505,49.40867888)
\lineto(553.48501505,67.13667888)
\lineto(553.48501505,49.40867888)
\lineto(543.94901505,49.40867888)
\lineto(543.94901505,84.35267888)
\lineto(553.48501505,84.35267888)
\lineto(553.48501505,67.39267888)
\closepath
}
}
{
\newrgbcolor{curcolor}{0 0 0}
\pscustom[linestyle=none,fillstyle=solid,fillcolor=curcolor]
{
\newpath
\moveto(614.02895255,66.94467888)
\curveto(614.02895255,61.14201221)(612.49295255,56.66201221)(609.42095255,53.50467888)
\curveto(606.39161921,50.34734554)(602.25295255,48.76867888)(597.00495255,48.76867888)
\curveto(593.76228588,48.76867888)(590.86095255,49.47267888)(588.30095255,50.88067888)
\curveto(585.78361921,52.28867888)(583.79961921,54.33667888)(582.34895255,57.02467888)
\curveto(580.89828588,59.75534554)(580.17295255,63.06201221)(580.17295255,66.94467888)
\curveto(580.17295255,72.74734554)(581.68761921,77.20601221)(584.71695255,80.32067888)
\curveto(587.74628588,83.43534554)(591.90628588,84.99267888)(597.19695255,84.99267888)
\curveto(600.48228588,84.99267888)(603.38361921,84.28867888)(605.90095255,82.88067888)
\curveto(608.41828588,81.47267888)(610.40228588,79.42467888)(611.85295255,76.73667888)
\curveto(613.30361921,74.04867888)(614.02895255,70.78467888)(614.02895255,66.94467888)
\closepath
\moveto(589.90095255,66.94467888)
\curveto(589.90095255,63.48867888)(590.45561921,60.86467888)(591.56495255,59.07267888)
\curveto(592.71695255,57.32334554)(594.57295255,56.44867888)(597.13295255,56.44867888)
\curveto(599.65028588,56.44867888)(601.46361921,57.32334554)(602.57295255,59.07267888)
\curveto(603.72495255,60.86467888)(604.30095255,63.48867888)(604.30095255,66.94467888)
\curveto(604.30095255,70.40067888)(603.72495255,72.98201221)(602.57295255,74.68867888)
\curveto(601.46361921,76.43801221)(599.62895255,77.31267888)(597.06895255,77.31267888)
\curveto(594.55161921,77.31267888)(592.71695255,76.43801221)(591.56495255,74.68867888)
\curveto(590.45561921,72.98201221)(589.90095255,70.40067888)(589.90095255,66.94467888)
\closepath
}
}
{
\newrgbcolor{curcolor}{0 0 0}
\pscustom[linestyle=none,fillstyle=solid,fillcolor=curcolor]
{
\newpath
\moveto(665.93291934,84.35267888)
\lineto(665.93291934,49.40867888)
\lineto(657.03691934,49.40867888)
\lineto(657.03691934,66.56067888)
\curveto(657.03691934,68.26734554)(657.05825268,69.93134554)(657.10091934,71.55267888)
\curveto(657.18625268,73.17401221)(657.29291934,74.66734554)(657.42091934,76.03267888)
\lineto(657.22891934,76.03267888)
\lineto(647.56491934,49.40867888)
\lineto(640.39691934,49.40867888)
\lineto(630.60491934,76.09667888)
\lineto(630.34891934,76.09667888)
\curveto(630.51958601,74.68867888)(630.62625268,73.17401221)(630.66891934,71.55267888)
\curveto(630.75425268,69.97401221)(630.79691934,68.22467888)(630.79691934,66.30467888)
\lineto(630.79691934,49.40867888)
\lineto(621.90091934,49.40867888)
\lineto(621.90091934,84.35267888)
\lineto(635.40491934,84.35267888)
\lineto(644.10891934,60.67267888)
\lineto(652.94091934,84.35267888)
\closepath
}
}
{
\newrgbcolor{curcolor}{0 0 0}
\pscustom[linestyle=none,fillstyle=solid,fillcolor=curcolor]
{
\newpath
\moveto(685.45291055,84.35267888)
\lineto(685.45291055,70.91267888)
\lineto(698.76491055,70.91267888)
\lineto(698.76491055,84.35267888)
\lineto(708.30091055,84.35267888)
\lineto(708.30091055,49.40867888)
\lineto(698.76491055,49.40867888)
\lineto(698.76491055,63.80867888)
\lineto(685.45291055,63.80867888)
\lineto(685.45291055,49.40867888)
\lineto(675.91691055,49.40867888)
\lineto(675.91691055,84.35267888)
\closepath
}
}
{
\newrgbcolor{curcolor}{0 0 0}
\pscustom[linestyle=none,fillstyle=solid,fillcolor=curcolor]
{
\newpath
\moveto(732.62095255,85.05667888)
\curveto(737.31428588,85.05667888)(740.89828588,84.03267888)(743.37295255,81.98467888)
\curveto(745.89028588,79.97934554)(747.14895255,76.88601221)(747.14895255,72.70467888)
\lineto(747.14895255,49.40867888)
\lineto(740.49295255,49.40867888)
\lineto(738.63695255,54.14467888)
\lineto(738.38095255,54.14467888)
\curveto(736.88761921,52.26734554)(735.30895255,50.90201221)(733.64495255,50.04867888)
\curveto(731.98095255,49.19534554)(729.69828588,48.76867888)(726.79695255,48.76867888)
\curveto(723.68228588,48.76867888)(721.10095255,49.66467888)(719.05295255,51.45667888)
\curveto(717.00495255,53.24867888)(715.98095255,56.04334554)(715.98095255,59.84067888)
\curveto(715.98095255,63.55267888)(717.28228588,66.28334554)(719.88495255,68.03267888)
\curveto(722.48761921,69.78201221)(726.39161921,70.76334554)(731.59695255,70.97667888)
\lineto(737.67695255,71.16867888)
\lineto(737.67695255,72.70467888)
\curveto(737.67695255,74.53934554)(737.18628588,75.88334554)(736.20495255,76.73667888)
\curveto(735.26628588,77.59001221)(733.94361921,78.01667888)(732.23695255,78.01667888)
\curveto(730.53028588,78.01667888)(728.86628588,77.76067888)(727.24495255,77.24867888)
\curveto(725.62361921,76.77934554)(724.00228588,76.18201221)(722.38095255,75.45667888)
\lineto(719.24495255,81.92067888)
\curveto(721.07961921,82.85934554)(723.14895255,83.60601221)(725.45295255,84.16067888)
\curveto(727.75695255,84.75801221)(730.14628588,85.05667888)(732.62095255,85.05667888)
\closepath
\moveto(737.67695255,65.60067888)
\lineto(733.96495255,65.47267888)
\curveto(730.89295255,65.38734554)(728.75961921,64.83267888)(727.56495255,63.80867888)
\curveto(726.37028588,62.78467888)(725.77295255,61.44067888)(725.77295255,59.77667888)
\curveto(725.77295255,58.32601221)(726.19961921,57.28067888)(727.05295255,56.64067888)
\curveto(727.90628588,56.04334554)(729.01561921,55.74467888)(730.38095255,55.74467888)
\curveto(732.42895255,55.74467888)(734.15695255,56.34201221)(735.56495255,57.53667888)
\curveto(736.97295255,58.77401221)(737.67695255,60.50201221)(737.67695255,62.72067888)
\closepath
}
}
{
\newrgbcolor{curcolor}{0 0 0}
\pscustom[linestyle=none,fillstyle=solid,fillcolor=curcolor]
{
\newpath
\moveto(785.86895841,77.18467888)
\lineto(774.41295841,77.18467888)
\lineto(774.41295841,49.40867888)
\lineto(764.87695841,49.40867888)
\lineto(764.87695841,77.18467888)
\lineto(753.42095841,77.18467888)
\lineto(753.42095841,84.35267888)
\lineto(785.86895841,84.35267888)
\closepath
}
}
{
\newrgbcolor{curcolor}{0 0 0}
\pscustom[linestyle=none,fillstyle=solid,fillcolor=curcolor]
{
\newpath
\moveto(792.33292618,49.40867888)
\lineto(792.33292618,84.35267888)
\lineto(801.86892618,84.35267888)
\lineto(801.86892618,70.84867888)
\lineto(806.47692618,70.84867888)
\curveto(811.81025951,70.84867888)(815.75692618,69.99534554)(818.31692618,68.28867888)
\curveto(820.87692618,66.58201221)(822.15692618,64.00067888)(822.15692618,60.54467888)
\curveto(822.15692618,57.13134554)(820.96225951,54.42201221)(818.57292618,52.41667888)
\curveto(816.18359285,50.41134554)(812.25825951,49.40867888)(806.79692618,49.40867888)
\closepath
\moveto(827.21292618,49.40867888)
\lineto(827.21292618,84.35267888)
\lineto(836.74892618,84.35267888)
\lineto(836.74892618,49.40867888)
\closepath
\moveto(801.86892618,56.00067888)
\lineto(806.28492618,56.00067888)
\curveto(808.16225951,56.00067888)(809.67692618,56.32067888)(810.82892618,56.96067888)
\curveto(812.02359285,57.64334554)(812.62092618,58.79534554)(812.62092618,60.41667888)
\curveto(812.62092618,62.97667888)(810.46625951,64.25667888)(806.15692618,64.25667888)
\lineto(801.86892618,64.25667888)
\closepath
}
}
{
\newrgbcolor{curcolor}{0 0 0}
\pscustom[linestyle=none,fillstyle=solid,fillcolor=curcolor]
{
\newpath
\moveto(206.37147262,398.8204877)
\lineto(220.57947262,398.8204877)
\curveto(226.63813929,398.8204877)(231.22480596,397.96715437)(234.33947263,396.2604877)
\curveto(237.49680596,394.55382104)(239.07547263,391.54582104)(239.07547263,387.2364877)
\curveto(239.07547263,384.63382104)(238.45680596,382.4364877)(237.21947263,380.6444877)
\curveto(236.02480596,378.8524877)(234.29680596,377.7644877)(232.03547263,377.3804877)
\lineto(232.03547263,377.0604877)
\curveto(233.52880596,376.71915437)(234.89413929,376.1644877)(236.13147263,375.3964877)
\curveto(237.41147263,374.67115437)(238.41413929,373.58315437)(239.13947262,372.1324877)
\curveto(239.86480596,370.68182104)(240.22747263,368.76182104)(240.22747262,366.3724877)
\curveto(240.22747262,362.23382104)(238.71280596,358.99115437)(235.68347263,356.6444877)
\curveto(232.69680596,354.29782104)(228.62213929,353.1244877)(223.45947263,353.1244877)
\lineto(206.37147262,353.1244877)
\closepath
\moveto(216.03547263,380.7084877)
\lineto(221.66747263,380.7084877)
\curveto(224.48347263,380.7084877)(226.42480596,381.13515437)(227.49147263,381.9884877)
\curveto(228.60080596,382.8844877)(229.15547263,384.20715437)(229.15547263,385.9564877)
\curveto(229.15547263,387.70582104)(228.51547263,388.9644877)(227.23547262,389.7324877)
\curveto(225.95547262,390.5004877)(223.92880596,390.8844877)(221.15547263,390.8844877)
\lineto(216.03547263,390.8844877)
\closepath
\moveto(216.03547263,373.0284877)
\lineto(216.03547263,361.1244877)
\lineto(222.37147262,361.1244877)
\curveto(225.27280596,361.1244877)(227.29947263,361.67915437)(228.45147263,362.7884877)
\curveto(229.60347262,363.9404877)(230.17947263,365.45515437)(230.17947263,367.3324877)
\curveto(230.17947263,369.03915437)(229.58213929,370.4044877)(228.38747263,371.4284877)
\curveto(227.23547263,372.49515437)(225.12347263,373.0284877)(222.05147263,373.0284877)
\closepath
}
}
{
\newrgbcolor{curcolor}{0 0 0}
\pscustom[linestyle=none,fillstyle=solid,fillcolor=curcolor]
{
\newpath
\moveto(279.90752927,353.1244877)
\lineto(270.37152927,353.1244877)
\lineto(270.37152927,380.9004877)
\lineto(261.60352927,380.9004877)
\curveto(261.0488626,374.07382104)(260.30219593,368.56982104)(259.36352927,364.3884877)
\curveto(258.46752927,360.24982104)(257.18752927,357.2204877)(255.52352927,355.3004877)
\curveto(253.90219593,353.42315437)(251.74752927,352.4844877)(249.05952927,352.4844877)
\curveto(246.8408626,352.4844877)(245.02752927,352.82582104)(243.61952927,353.5084877)
\lineto(243.61952927,361.1244877)
\curveto(244.6008626,360.69782104)(245.6248626,360.4844877)(246.69152927,360.4844877)
\curveto(247.45952927,360.4844877)(248.16352927,360.8684877)(248.80352927,361.6364877)
\curveto(249.44352927,362.4044877)(250.0408626,363.79115437)(250.59552927,365.7964877)
\curveto(251.1928626,367.80182104)(251.72619593,370.5964877)(252.19552927,374.1804877)
\curveto(252.6648626,377.80715437)(253.09152927,382.4364877)(253.47552927,388.0684877)
\lineto(279.90752927,388.0684877)
\closepath
}
}
{
\newrgbcolor{curcolor}{0 0 0}
\pscustom[linestyle=none,fillstyle=solid,fillcolor=curcolor]
{
\newpath
\moveto(304.22755856,388.7724877)
\curveto(308.9208919,388.7724877)(312.5048919,387.7484877)(314.97955856,385.7004877)
\curveto(317.4968919,383.69515437)(318.75555856,380.60182104)(318.75555856,376.4204877)
\lineto(318.75555856,353.1244877)
\lineto(312.09955856,353.1244877)
\lineto(310.24355856,357.8604877)
\lineto(309.98755856,357.8604877)
\curveto(308.49422523,355.98315437)(306.91555856,354.61782104)(305.25155856,353.7644877)
\curveto(303.58755856,352.91115437)(301.3048919,352.4844877)(298.40355856,352.4844877)
\curveto(295.2888919,352.4844877)(292.70755856,353.3804877)(290.65955856,355.1724877)
\curveto(288.61155856,356.9644877)(287.58755856,359.75915437)(287.58755856,363.5564877)
\curveto(287.58755856,367.2684877)(288.8888919,369.99915437)(291.49155856,371.7484877)
\curveto(294.09422523,373.49782104)(297.99822523,374.47915437)(303.20355856,374.6924877)
\lineto(309.28355856,374.8844877)
\lineto(309.28355856,376.4204877)
\curveto(309.28355856,378.25515437)(308.7928919,379.59915437)(307.81155856,380.4524877)
\curveto(306.8728919,381.30582104)(305.55022523,381.7324877)(303.84355856,381.7324877)
\curveto(302.1368919,381.7324877)(300.4728919,381.4764877)(298.85155856,380.9644877)
\curveto(297.23022523,380.49515437)(295.6088919,379.89782104)(293.98755856,379.1724877)
\lineto(290.85155856,385.6364877)
\curveto(292.68622523,386.57515437)(294.75555856,387.32182104)(297.05955856,387.8764877)
\curveto(299.36355856,388.47382104)(301.7528919,388.7724877)(304.22755856,388.7724877)
\closepath
\moveto(309.28355856,369.3164877)
\lineto(305.57155856,369.1884877)
\curveto(302.49955856,369.10315437)(300.36622523,368.5484877)(299.17155856,367.5244877)
\curveto(297.9768919,366.5004877)(297.37955856,365.1564877)(297.37955856,363.4924877)
\curveto(297.37955856,362.04182104)(297.80622523,360.9964877)(298.65955856,360.3564877)
\curveto(299.5128919,359.75915437)(300.62222523,359.4604877)(301.98755856,359.4604877)
\curveto(304.03555856,359.4604877)(305.76355856,360.05782104)(307.17155856,361.2524877)
\curveto(308.57955856,362.48982104)(309.28355856,364.21782104)(309.28355856,366.4364877)
\closepath
}
}
{
\newrgbcolor{curcolor}{0 0 0}
\pscustom[linestyle=none,fillstyle=solid,fillcolor=curcolor]
{
\newpath
\moveto(359.97156442,388.0684877)
\lineto(359.97156442,360.1004877)
\lineto(365.09156442,360.1004877)
\lineto(365.09156442,340.5804877)
\lineto(356.51556442,340.5804877)
\lineto(356.51556442,353.1244877)
\lineto(333.02756442,353.1244877)
\lineto(333.02756442,340.5804877)
\lineto(324.45156442,340.5804877)
\lineto(324.45156442,360.1004877)
\lineto(327.39556442,360.1004877)
\curveto(328.93156442,362.44715437)(330.23289776,365.11382104)(331.29956442,368.1004877)
\curveto(332.36623109,371.12982104)(333.21956442,374.32982104)(333.85956442,377.7004877)
\curveto(334.49956442,381.11382104)(334.96889776,384.56982104)(335.26756442,388.0684877)
\closepath
\moveto(350.43556442,380.9004877)
\lineto(343.26756442,380.9004877)
\curveto(342.75556442,377.01782104)(342.05156442,373.32715437)(341.15556442,369.8284877)
\curveto(340.25956442,366.3724877)(339.00089776,363.12982104)(337.37956442,360.1004877)
\lineto(350.43556442,360.1004877)
\closepath
}
}
{
\newrgbcolor{curcolor}{0 0 0}
\pscustom[linestyle=none,fillstyle=solid,fillcolor=curcolor]
{
\newpath
\moveto(385.4435361,388.7084877)
\curveto(390.26486943,388.7084877)(394.0835361,387.32182104)(396.8995361,384.5484877)
\curveto(399.7155361,381.81782104)(401.1235361,377.91382104)(401.1235361,372.8364877)
\lineto(401.1235361,368.2284877)
\lineto(378.5955361,368.2284877)
\curveto(378.68086943,365.5404877)(379.47020277,363.4284877)(380.9635361,361.8924877)
\curveto(382.4995361,360.3564877)(384.6115361,359.5884877)(387.2995361,359.5884877)
\curveto(389.51820277,359.5884877)(391.54486943,359.80182104)(393.3795361,360.2284877)
\curveto(395.25686943,360.69782104)(397.17686943,361.40182104)(399.1395361,362.3404877)
\lineto(399.1395361,354.9804877)
\curveto(397.39020277,354.12715437)(395.57686943,353.5084877)(393.6995361,353.1244877)
\curveto(391.82220277,352.69782104)(389.5395361,352.4844877)(386.8515361,352.4844877)
\curveto(383.35286943,352.4844877)(380.2595361,353.1244877)(377.5715361,354.4044877)
\curveto(374.8835361,355.72715437)(372.7715361,357.68982104)(371.2355361,360.2924877)
\curveto(369.6995361,362.93782104)(368.9315361,366.28715437)(368.9315361,370.3404877)
\curveto(368.9315361,374.39382104)(369.61420277,377.78582104)(370.9795361,380.5164877)
\curveto(372.3875361,383.24715437)(374.32886943,385.29515437)(376.8035361,386.6604877)
\curveto(379.27820277,388.02582104)(382.15820277,388.7084877)(385.4435361,388.7084877)
\closepath
\moveto(385.5075361,381.9244877)
\curveto(383.63020277,381.9244877)(382.09420277,381.32715437)(380.8995361,380.1324877)
\curveto(379.70486943,378.93782104)(379.00086943,377.08182104)(378.7875361,374.5644877)
\lineto(392.1635361,374.5644877)
\curveto(392.12086943,376.65515437)(391.54486943,378.4044877)(390.4355361,379.8124877)
\curveto(389.36886943,381.2204877)(387.72620277,381.9244877)(385.5075361,381.9244877)
\closepath
}
}
{
\newrgbcolor{curcolor}{0 0 0}
\pscustom[linestyle=none,fillstyle=solid,fillcolor=curcolor]
{
\newpath
\moveto(440.16351071,353.1244877)
\lineto(430.62751071,353.1244877)
\lineto(430.62751071,380.9004877)
\lineto(421.85951071,380.9004877)
\curveto(421.30484404,374.07382104)(420.55817738,368.56982104)(419.61951071,364.3884877)
\curveto(418.72351071,360.24982104)(417.44351071,357.2204877)(415.77951071,355.3004877)
\curveto(414.15817738,353.42315437)(412.00351071,352.4844877)(409.31551071,352.4844877)
\curveto(407.09684404,352.4844877)(405.28351071,352.82582104)(403.87551071,353.5084877)
\lineto(403.87551071,361.1244877)
\curveto(404.85684404,360.69782104)(405.88084404,360.4844877)(406.94751071,360.4844877)
\curveto(407.71551071,360.4844877)(408.41951071,360.8684877)(409.05951071,361.6364877)
\curveto(409.69951071,362.4044877)(410.29684404,363.79115437)(410.85151071,365.7964877)
\curveto(411.44884404,367.80182104)(411.98217738,370.5964877)(412.45151071,374.1804877)
\curveto(412.92084404,377.80715437)(413.34751071,382.4364877)(413.73151071,388.0684877)
\lineto(440.16351071,388.0684877)
\closepath
}
}
{
\newrgbcolor{curcolor}{0 0 0}
\pscustom[linestyle=none,fillstyle=solid,fillcolor=curcolor]
{
\newpath
\moveto(464.54754001,388.7084877)
\curveto(469.36887334,388.7084877)(473.18754001,387.32182104)(476.00354001,384.5484877)
\curveto(478.81954001,381.81782104)(480.22754001,377.91382104)(480.22754001,372.8364877)
\lineto(480.22754001,368.2284877)
\lineto(457.69954001,368.2284877)
\curveto(457.78487334,365.5404877)(458.57420667,363.4284877)(460.06754001,361.8924877)
\curveto(461.60354001,360.3564877)(463.71554001,359.5884877)(466.40354001,359.5884877)
\curveto(468.62220667,359.5884877)(470.64887334,359.80182104)(472.48354001,360.2284877)
\curveto(474.36087334,360.69782104)(476.28087334,361.40182104)(478.24354001,362.3404877)
\lineto(478.24354001,354.9804877)
\curveto(476.49420667,354.12715437)(474.68087334,353.5084877)(472.80354001,353.1244877)
\curveto(470.92620667,352.69782104)(468.64354001,352.4844877)(465.95554001,352.4844877)
\curveto(462.45687334,352.4844877)(459.36354001,353.1244877)(456.67554001,354.4044877)
\curveto(453.98754001,355.72715437)(451.87554001,357.68982104)(450.33954001,360.2924877)
\curveto(448.80354001,362.93782104)(448.03554001,366.28715437)(448.03554001,370.3404877)
\curveto(448.03554001,374.39382104)(448.71820667,377.78582104)(450.08354001,380.5164877)
\curveto(451.49154001,383.24715437)(453.43287334,385.29515437)(455.90754001,386.6604877)
\curveto(458.38220667,388.02582104)(461.26220667,388.7084877)(464.54754001,388.7084877)
\closepath
\moveto(464.61154001,381.9244877)
\curveto(462.73420667,381.9244877)(461.19820667,381.32715437)(460.00354001,380.1324877)
\curveto(458.80887334,378.93782104)(458.10487334,377.08182104)(457.89154001,374.5644877)
\lineto(471.26754001,374.5644877)
\curveto(471.22487334,376.65515437)(470.64887334,378.4044877)(469.53954001,379.8124877)
\curveto(468.47287334,381.2204877)(466.83020667,381.9244877)(464.61154001,381.9244877)
\closepath
}
}
{
\newrgbcolor{curcolor}{0 0 0}
\pscustom[linestyle=none,fillstyle=solid,fillcolor=curcolor]
{
\newpath
\moveto(526.11551462,340.5804877)
\lineto(517.53951462,340.5804877)
\lineto(517.53951462,353.1244877)
\lineto(487.97151462,353.1244877)
\lineto(487.97151462,388.0684877)
\lineto(497.50751462,388.0684877)
\lineto(497.50751462,360.2924877)
\lineto(511.45951462,360.2924877)
\lineto(511.45951462,388.0684877)
\lineto(520.99551462,388.0684877)
\lineto(520.99551462,360.1004877)
\lineto(526.11551462,360.1004877)
\closepath
}
}
{
\newrgbcolor{curcolor}{0 0 0}
\pscustom[linestyle=none,fillstyle=solid,fillcolor=curcolor]
{
\newpath
\moveto(571.55551169,388.0684877)
\lineto(582.05151169,388.0684877)
\lineto(568.22751169,371.3004877)
\lineto(583.26751169,353.1244877)
\lineto(572.45151169,353.1244877)
\lineto(558.17951169,370.8524877)
\lineto(558.17951169,353.1244877)
\lineto(548.64351169,353.1244877)
\lineto(548.64351169,388.0684877)
\lineto(558.17951169,388.0684877)
\lineto(558.17951169,371.1084877)
\closepath
}
}
{
\newrgbcolor{curcolor}{0 0 0}
\pscustom[linestyle=none,fillstyle=solid,fillcolor=curcolor]
{
\newpath
\moveto(618.72344919,370.6604877)
\curveto(618.72344919,364.85782104)(617.18744919,360.37782104)(614.11544919,357.2204877)
\curveto(611.08611585,354.06315437)(606.94744919,352.4844877)(601.69944919,352.4844877)
\curveto(598.45678252,352.4844877)(595.55544919,353.1884877)(592.99544919,354.5964877)
\curveto(590.47811585,356.0044877)(588.49411585,358.0524877)(587.04344919,360.7404877)
\curveto(585.59278252,363.47115437)(584.86744919,366.77782104)(584.86744919,370.6604877)
\curveto(584.86744919,376.46315437)(586.38211585,380.92182104)(589.41144919,384.0364877)
\curveto(592.44078252,387.15115437)(596.60078252,388.7084877)(601.89144919,388.7084877)
\curveto(605.17678252,388.7084877)(608.07811585,388.0044877)(610.59544919,386.5964877)
\curveto(613.11278252,385.1884877)(615.09678252,383.1404877)(616.54744919,380.4524877)
\curveto(617.99811585,377.7644877)(618.72344919,374.5004877)(618.72344919,370.6604877)
\closepath
\moveto(594.59544919,370.6604877)
\curveto(594.59544919,367.2044877)(595.15011585,364.5804877)(596.25944919,362.7884877)
\curveto(597.41144919,361.03915437)(599.26744919,360.1644877)(601.82744919,360.1644877)
\curveto(604.34478252,360.1644877)(606.15811585,361.03915437)(607.26744919,362.7884877)
\curveto(608.41944919,364.5804877)(608.99544919,367.2044877)(608.99544919,370.6604877)
\curveto(608.99544919,374.1164877)(608.41944919,376.69782104)(607.26744919,378.4044877)
\curveto(606.15811585,380.15382104)(604.32344919,381.0284877)(601.76344919,381.0284877)
\curveto(599.24611585,381.0284877)(597.41144919,380.15382104)(596.25944919,378.4044877)
\curveto(595.15011585,376.69782104)(594.59544919,374.1164877)(594.59544919,370.6604877)
\closepath
}
}
{
\newrgbcolor{curcolor}{0 0 0}
\pscustom[linestyle=none,fillstyle=solid,fillcolor=curcolor]
{
\newpath
\moveto(670.62741598,388.0684877)
\lineto(670.62741598,353.1244877)
\lineto(661.73141598,353.1244877)
\lineto(661.73141598,370.2764877)
\curveto(661.73141598,371.98315437)(661.75274932,373.64715437)(661.79541598,375.2684877)
\curveto(661.88074932,376.88982104)(661.98741598,378.38315437)(662.11541598,379.7484877)
\lineto(661.92341598,379.7484877)
\lineto(652.25941598,353.1244877)
\lineto(645.09141598,353.1244877)
\lineto(635.29941598,379.8124877)
\lineto(635.04341598,379.8124877)
\curveto(635.21408265,378.4044877)(635.32074932,376.88982104)(635.36341598,375.2684877)
\curveto(635.44874932,373.68982104)(635.49141598,371.9404877)(635.49141598,370.0204877)
\lineto(635.49141598,353.1244877)
\lineto(626.59541598,353.1244877)
\lineto(626.59541598,388.0684877)
\lineto(640.09941598,388.0684877)
\lineto(648.80341598,364.3884877)
\lineto(657.63541598,388.0684877)
\closepath
}
}
{
\newrgbcolor{curcolor}{0 0 0}
\pscustom[linestyle=none,fillstyle=solid,fillcolor=curcolor]
{
\newpath
\moveto(690.1474072,388.0684877)
\lineto(690.1474072,374.6284877)
\lineto(703.4594072,374.6284877)
\lineto(703.4594072,388.0684877)
\lineto(712.9954072,388.0684877)
\lineto(712.9954072,353.1244877)
\lineto(703.4594072,353.1244877)
\lineto(703.4594072,367.5244877)
\lineto(690.1474072,367.5244877)
\lineto(690.1474072,353.1244877)
\lineto(680.6114072,353.1244877)
\lineto(680.6114072,388.0684877)
\closepath
}
}
{
\newrgbcolor{curcolor}{0 0 0}
\pscustom[linestyle=none,fillstyle=solid,fillcolor=curcolor]
{
\newpath
\moveto(737.31544919,388.7724877)
\curveto(742.00878252,388.7724877)(745.59278252,387.7484877)(748.06744919,385.7004877)
\curveto(750.58478252,383.69515437)(751.84344919,380.60182104)(751.84344919,376.4204877)
\lineto(751.84344919,353.1244877)
\lineto(745.18744919,353.1244877)
\lineto(743.33144919,357.8604877)
\lineto(743.07544919,357.8604877)
\curveto(741.58211585,355.98315437)(740.00344919,354.61782104)(738.33944919,353.7644877)
\curveto(736.67544919,352.91115437)(734.39278252,352.4844877)(731.49144919,352.4844877)
\curveto(728.37678252,352.4844877)(725.79544919,353.3804877)(723.74744919,355.1724877)
\curveto(721.69944919,356.9644877)(720.67544919,359.75915437)(720.67544919,363.5564877)
\curveto(720.67544919,367.2684877)(721.97678252,369.99915437)(724.57944919,371.7484877)
\curveto(727.18211585,373.49782104)(731.08611585,374.47915437)(736.29144919,374.6924877)
\lineto(742.37144919,374.8844877)
\lineto(742.37144919,376.4204877)
\curveto(742.37144919,378.25515437)(741.88078252,379.59915437)(740.89944919,380.4524877)
\curveto(739.96078252,381.30582104)(738.63811585,381.7324877)(736.93144919,381.7324877)
\curveto(735.22478252,381.7324877)(733.56078252,381.4764877)(731.93944919,380.9644877)
\curveto(730.31811585,380.49515437)(728.69678252,379.89782104)(727.07544919,379.1724877)
\lineto(723.93944919,385.6364877)
\curveto(725.77411585,386.57515437)(727.84344919,387.32182104)(730.14744919,387.8764877)
\curveto(732.45144919,388.47382104)(734.84078252,388.7724877)(737.31544919,388.7724877)
\closepath
\moveto(742.37144919,369.3164877)
\lineto(738.65944919,369.1884877)
\curveto(735.58744919,369.10315437)(733.45411585,368.5484877)(732.25944919,367.5244877)
\curveto(731.06478252,366.5004877)(730.46744919,365.1564877)(730.46744919,363.4924877)
\curveto(730.46744919,362.04182104)(730.89411585,360.9964877)(731.74744919,360.3564877)
\curveto(732.60078252,359.75915437)(733.71011585,359.4604877)(735.07544919,359.4604877)
\curveto(737.12344919,359.4604877)(738.85144919,360.05782104)(740.25944919,361.2524877)
\curveto(741.66744919,362.48982104)(742.37144919,364.21782104)(742.37144919,366.4364877)
\closepath
}
}
{
\newrgbcolor{curcolor}{0 0 0}
\pscustom[linestyle=none,fillstyle=solid,fillcolor=curcolor]
{
\newpath
\moveto(790.56345505,380.9004877)
\lineto(779.10745505,380.9004877)
\lineto(779.10745505,353.1244877)
\lineto(769.57145505,353.1244877)
\lineto(769.57145505,380.9004877)
\lineto(758.11545505,380.9004877)
\lineto(758.11545505,388.0684877)
\lineto(790.56345505,388.0684877)
\closepath
}
}
{
\newrgbcolor{curcolor}{0 0 0}
\pscustom[linestyle=none,fillstyle=solid,fillcolor=curcolor]
{
\newpath
\moveto(797.02742282,353.1244877)
\lineto(797.02742282,388.0684877)
\lineto(806.56342282,388.0684877)
\lineto(806.56342282,374.5644877)
\lineto(811.17142282,374.5644877)
\curveto(816.50475615,374.5644877)(820.45142282,373.71115437)(823.01142282,372.0044877)
\curveto(825.57142282,370.29782104)(826.85142282,367.7164877)(826.85142282,364.2604877)
\curveto(826.85142282,360.84715437)(825.65675615,358.13782104)(823.26742282,356.1324877)
\curveto(820.87808949,354.12715437)(816.95275615,353.1244877)(811.49142282,353.1244877)
\closepath
\moveto(831.90742282,353.1244877)
\lineto(831.90742282,388.0684877)
\lineto(841.44342282,388.0684877)
\lineto(841.44342282,353.1244877)
\closepath
\moveto(806.56342282,359.7164877)
\lineto(810.97942282,359.7164877)
\curveto(812.85675615,359.7164877)(814.37142282,360.0364877)(815.52342282,360.6764877)
\curveto(816.71808949,361.35915437)(817.31542282,362.51115437)(817.31542282,364.1324877)
\curveto(817.31542282,366.6924877)(815.16075615,367.9724877)(810.85142282,367.9724877)
\lineto(806.56342282,367.9724877)
\closepath
}
}
{
\newrgbcolor{curcolor}{0 0 0}
\pscustom[linestyle=none,fillstyle=solid,fillcolor=curcolor]
{
\newpath
\moveto(90.6615468,565.99789197)
\curveto(90.6615468,564.2058935)(90.02688067,562.76233917)(88.75754842,561.667229)
\curveto(87.48821617,560.57211882)(85.87043977,559.88767496)(83.90421922,559.61389741)
\lineto(83.90421922,559.50189751)
\curveto(86.34332825,559.25300883)(88.20999333,558.56856497)(89.50421445,557.44856592)
\curveto(90.82332444,556.35345574)(91.48287943,554.92234585)(91.48287943,553.15523624)
\curveto(91.48287943,550.81568268)(90.52465803,548.89923987)(88.60821522,547.40590781)
\curveto(86.71666127,545.93746461)(83.92910809,545.20324301)(80.24555567,545.20324301)
\curveto(78.22955739,545.20324301)(76.43755891,545.32768735)(74.86956025,545.57657603)
\curveto(73.32645045,545.82546471)(71.98245159,546.18635329)(70.83756368,546.65924177)
\lineto(70.83756368,551.40057107)
\curveto(71.60911858,551.02723806)(72.48022895,550.70368278)(73.45089479,550.42990523)
\curveto(74.42156063,550.18101655)(75.4046709,549.98190561)(76.40022561,549.83257241)
\curveto(77.39578032,549.70812807)(78.31666842,549.6459059)(79.16288992,549.6459059)
\curveto(81.52733236,549.6459059)(83.23221979,549.98190561)(84.27755224,550.65390504)
\curveto(85.34777355,551.35079334)(85.8828842,552.32145918)(85.8828842,553.56590256)
\curveto(85.8828842,554.83523481)(85.1113293,555.75612292)(83.56821951,556.32856687)
\curveto(82.02510971,556.9258997)(79.94688926,557.22456611)(77.33355815,557.22456611)
\lineto(74.83222695,557.22456611)
\lineto(74.83222695,561.62989569)
\lineto(77.07222504,561.62989569)
\curveto(79.18777879,561.62989569)(80.83044406,561.76678447)(82.00022084,562.04056201)
\curveto(83.16999762,562.31433956)(83.99133026,562.71256144)(84.46421874,563.23522766)
\curveto(84.93710723,563.75789388)(85.17355147,564.38011557)(85.17355147,565.10189274)
\curveto(85.17355147,566.02278084)(84.76288516,566.74455801)(83.94155252,567.26722423)
\curveto(83.12021989,567.81477932)(81.91310981,568.08855686)(80.32022227,568.08855686)
\curveto(78.92644568,568.08855686)(77.63222456,567.88944592)(76.43755891,567.49122404)
\curveto(75.24289326,567.09300215)(74.12289422,566.57033593)(73.07756177,565.92322537)
\lineto(70.61356387,569.69388883)
\curveto(71.98245159,570.58988806)(73.50067252,571.29922079)(75.16822666,571.82188702)
\curveto(76.8357808,572.34455324)(78.81444578,572.60588635)(81.1042216,572.60588635)
\curveto(84.1157746,572.60588635)(86.45532816,571.98366466)(88.1228823,570.73922127)
\curveto(89.8153253,569.49477789)(90.6615468,567.91433479)(90.6615468,565.99789197)
\closepath
}
}
{
\newrgbcolor{curcolor}{0 0 0}
\pscustom[linestyle=none,fillstyle=solid,fillcolor=curcolor]
{
\newpath
\moveto(104.36289613,566.37122499)
\curveto(107.10067158,566.37122499)(109.19133646,565.77389217)(110.63489079,564.57922652)
\curveto(112.10333398,563.40944973)(112.83755558,561.60500683)(112.83755558,559.16589779)
\lineto(112.83755558,545.57657603)
\lineto(108.95489222,545.57657603)
\lineto(107.87222648,548.33924034)
\lineto(107.72289327,548.33924034)
\curveto(106.8517829,547.24413017)(105.9308948,546.4476864)(104.96022896,545.94990905)
\curveto(103.98956312,545.45213169)(102.65800869,545.20324301)(100.96556569,545.20324301)
\curveto(99.14867835,545.20324301)(97.64290185,545.72590924)(96.4482362,546.77124168)
\curveto(95.25357055,547.81657412)(94.65623773,549.44679496)(94.65623773,551.66190418)
\curveto(94.65623773,553.82723567)(95.41534819,555.4201232)(96.93356912,556.44056678)
\curveto(98.45179005,557.46101035)(100.72912145,558.03345431)(103.76556331,558.15789865)
\lineto(107.31222695,558.26989855)
\lineto(107.31222695,559.16589779)
\curveto(107.31222695,560.2361191)(107.02600497,561.02011844)(106.45356102,561.51789579)
\curveto(105.90600593,562.01567314)(105.13445103,562.26456182)(104.13889632,562.26456182)
\curveto(103.14334161,562.26456182)(102.17267577,562.11522861)(101.2268988,561.8165622)
\curveto(100.28112183,561.54278466)(99.33534486,561.19434051)(98.38956788,560.77122976)
\lineto(96.56023611,564.54189321)
\curveto(97.63045742,565.0894483)(98.8375675,565.52500349)(100.18156636,565.84855877)
\curveto(101.52556521,566.19700292)(102.9193418,566.37122499)(104.36289613,566.37122499)
\closepath
\moveto(107.31222695,555.02190132)
\lineto(105.14689546,554.94723472)
\curveto(103.35489699,554.89745698)(102.1104536,554.5739017)(101.41356531,553.97656888)
\curveto(100.71667701,553.37923605)(100.36823287,552.59523672)(100.36823287,551.62457088)
\curveto(100.36823287,550.77834938)(100.61712154,550.16857212)(101.1148989,549.7952391)
\curveto(101.61267625,549.44679496)(102.25978681,549.27257288)(103.05623058,549.27257288)
\curveto(104.25089623,549.27257288)(105.25889537,549.62101703)(106.080228,550.31790533)
\curveto(106.90156064,551.03968249)(107.31222695,552.04768163)(107.31222695,553.34190275)
\closepath
}
}
{
\newrgbcolor{curcolor}{0 0 0}
\pscustom[linestyle=none,fillstyle=solid,fillcolor=curcolor]
{
\newpath
\moveto(137.06687246,565.96055867)
\lineto(137.06687246,545.57657603)
\lineto(131.50421053,545.57657603)
\lineto(131.50421053,561.7792289)
\lineto(124.11221683,561.7792289)
\lineto(124.11221683,545.57657603)
\lineto(118.5495549,545.57657603)
\lineto(118.5495549,565.96055867)
\closepath
}
}
{
\newrgbcolor{curcolor}{0 0 0}
\pscustom[linestyle=none,fillstyle=solid,fillcolor=curcolor]
{
\newpath
\moveto(154.24021005,566.33389169)
\curveto(156.52998588,566.33389169)(158.38420652,565.43789245)(159.80287198,563.64589398)
\curveto(161.22153744,561.87878437)(161.93087017,559.26545326)(161.93087017,555.80590065)
\curveto(161.93087017,552.32145918)(161.19664857,549.6832392)(159.72820538,547.89124073)
\curveto(158.25976218,546.09924225)(156.38065267,545.20324301)(154.09087684,545.20324301)
\curveto(152.62243365,545.20324301)(151.45265687,545.46457613)(150.5815465,545.98724235)
\curveto(149.71043613,546.53479744)(149.0011034,547.14457469)(148.45354831,547.81657412)
\lineto(148.1548819,547.81657412)
\curveto(148.35399284,546.77124168)(148.45354831,545.77568697)(148.45354831,544.82991)
\lineto(148.45354831,536.61658366)
\lineto(142.89088638,536.61658366)
\lineto(142.89088638,565.96055867)
\lineto(147.40821587,565.96055867)
\lineto(148.1922152,563.30989426)
\lineto(148.45354831,563.30989426)
\curveto(149.0011034,564.1312269)(149.735325,564.84055963)(150.6562131,565.43789245)
\curveto(151.57710121,566.03522528)(152.77176686,566.33389169)(154.24021005,566.33389169)
\closepath
\moveto(152.44821158,561.8912288)
\curveto(151.00465725,561.8912288)(149.98421367,561.43078475)(149.38688085,560.50989665)
\curveto(148.78954802,559.61389741)(148.47843718,558.25745412)(148.45354831,556.44056678)
\lineto(148.45354831,555.84323395)
\curveto(148.45354831,553.87701341)(148.73977029,552.35879248)(149.31221425,551.28857117)
\curveto(149.90954707,550.24323872)(150.97976838,549.7205725)(152.52287818,549.7205725)
\curveto(153.79221043,549.7205725)(154.72554297,550.24323872)(155.3228758,551.28857117)
\curveto(155.94509749,552.35879248)(156.25620833,553.88945784)(156.25620833,555.88056726)
\curveto(156.25620833,559.88767496)(154.98687608,561.8912288)(152.44821158,561.8912288)
\closepath
}
}
{
\newrgbcolor{curcolor}{0 0 0}
\pscustom[linestyle=none,fillstyle=solid,fillcolor=curcolor]
{
\newpath
\moveto(185.04016195,555.80590065)
\curveto(185.04016195,552.42101465)(184.14416272,549.80768354)(182.35216424,547.96590733)
\curveto(180.58505464,546.12413112)(178.17083447,545.20324301)(175.10950374,545.20324301)
\curveto(173.2179498,545.20324301)(171.5255068,545.61390933)(170.03217473,546.43524197)
\curveto(168.56373154,547.2565746)(167.40639919,548.45124025)(166.56017769,550.01923891)
\curveto(165.71395619,551.61212645)(165.29084544,553.54101369)(165.29084544,555.80590065)
\curveto(165.29084544,559.19078666)(166.17440024,561.79167333)(167.94150985,563.60856068)
\curveto(169.70861945,565.42544802)(172.13528405,566.33389169)(175.22150365,566.33389169)
\curveto(177.13794646,566.33389169)(178.83038946,565.92322537)(180.29883266,565.10189274)
\curveto(181.76727585,564.2805601)(182.9246082,563.08589445)(183.7708297,561.51789579)
\curveto(184.6170512,559.94989712)(185.04016195,558.04589875)(185.04016195,555.80590065)
\closepath
\moveto(170.96550727,555.80590065)
\curveto(170.96550727,553.78990237)(171.28906255,552.25923701)(171.93617311,551.21390456)
\curveto(172.60817254,550.19346099)(173.69083829,549.6832392)(175.18417035,549.6832392)
\curveto(176.65261354,549.6832392)(177.71039042,550.19346099)(178.35750098,551.21390456)
\curveto(179.02950041,552.25923701)(179.36550012,553.78990237)(179.36550012,555.80590065)
\curveto(179.36550012,557.82189894)(179.02950041,559.32767543)(178.35750098,560.32323014)
\curveto(177.71039042,561.34367372)(176.64016911,561.8538955)(175.14683705,561.8538955)
\curveto(173.67839385,561.8538955)(172.60817254,561.34367372)(171.93617311,560.32323014)
\curveto(171.28906255,559.32767543)(170.96550727,557.82189894)(170.96550727,555.80590065)
\closepath
}
}
{
\newrgbcolor{curcolor}{0 0 0}
\pscustom[linestyle=none,fillstyle=solid,fillcolor=curcolor]
{
\newpath
\moveto(197.92012841,545.20324301)
\curveto(194.88368655,545.20324301)(192.53168855,546.03702008)(190.86413442,547.70457422)
\curveto(189.22146915,549.37212835)(188.40013651,552.02279276)(188.40013651,555.65656745)
\curveto(188.40013651,558.14545422)(188.82324727,560.17389693)(189.66946877,561.7418956)
\curveto(190.51569027,563.30989426)(191.68546705,564.46722661)(193.17879911,565.21389264)
\curveto(194.69702004,565.96055867)(196.43924078,566.33389169)(198.40546133,566.33389169)
\curveto(199.79923792,566.33389169)(201.006348,566.19700292)(202.02679158,565.92322537)
\curveto(203.07212402,565.64944783)(203.98056769,565.32589255)(204.75212259,564.95255953)
\lineto(203.10945732,560.65922985)
\curveto(202.23834695,561.007674)(201.41701432,561.29389598)(200.64545942,561.51789579)
\curveto(199.89879339,561.7418956)(199.15212736,561.8538955)(198.40546133,561.8538955)
\curveto(195.51835268,561.8538955)(194.07479835,559.80056392)(194.07479835,555.69390075)
\curveto(194.07479835,553.6530136)(194.44813136,552.1472371)(195.1947974,551.17657126)
\curveto(195.96635229,550.20590542)(197.0365736,549.7205725)(198.40546133,549.7205725)
\curveto(199.57523811,549.7205725)(200.60812612,549.86990571)(201.50412536,550.16857212)
\curveto(202.40012459,550.4921274)(203.27123496,550.92768258)(204.11745646,551.47523767)
\lineto(204.11745646,546.73390838)
\curveto(203.27123496,546.18635329)(202.37523573,545.80057584)(201.42945875,545.57657603)
\curveto(200.50857065,545.32768735)(199.33879387,545.20324301)(197.92012841,545.20324301)
\closepath
}
}
{
\newrgbcolor{curcolor}{0 0 0}
\pscustom[linestyle=none,fillstyle=solid,fillcolor=curcolor]
{
\newpath
\moveto(237.04542624,565.96055867)
\lineto(237.04542624,545.57657603)
\lineto(231.48276431,545.57657603)
\lineto(231.48276431,561.7792289)
\lineto(224.0907706,561.7792289)
\lineto(224.0907706,545.57657603)
\lineto(218.52810867,545.57657603)
\lineto(218.52810867,565.96055867)
\closepath
}
}
{
\newrgbcolor{curcolor}{0 0 0}
\pscustom[linestyle=none,fillstyle=solid,fillcolor=curcolor]
{
\newpath
\moveto(254.21877145,566.33389169)
\curveto(256.50854728,566.33389169)(258.36276792,565.43789245)(259.78143338,563.64589398)
\curveto(261.20009884,561.87878437)(261.90943157,559.26545326)(261.90943157,555.80590065)
\curveto(261.90943157,552.32145918)(261.17520997,549.6832392)(259.70676678,547.89124073)
\curveto(258.23832358,546.09924225)(256.35921407,545.20324301)(254.06943825,545.20324301)
\curveto(252.60099505,545.20324301)(251.43121827,545.46457613)(250.5601079,545.98724235)
\curveto(249.68899753,546.53479744)(248.9796648,547.14457469)(248.43210971,547.81657412)
\lineto(248.1334433,547.81657412)
\curveto(248.33255424,546.77124168)(248.43210971,545.77568697)(248.43210971,544.82991)
\lineto(248.43210971,536.61658366)
\lineto(242.86944778,536.61658366)
\lineto(242.86944778,565.96055867)
\lineto(247.38677727,565.96055867)
\lineto(248.1707766,563.30989426)
\lineto(248.43210971,563.30989426)
\curveto(248.9796648,564.1312269)(249.7138864,564.84055963)(250.6347745,565.43789245)
\curveto(251.55566261,566.03522528)(252.75032826,566.33389169)(254.21877145,566.33389169)
\closepath
\moveto(252.42677298,561.8912288)
\curveto(250.98321865,561.8912288)(249.96277508,561.43078475)(249.36544225,560.50989665)
\curveto(248.76810943,559.61389741)(248.45699858,558.25745412)(248.43210971,556.44056678)
\lineto(248.43210971,555.84323395)
\curveto(248.43210971,553.87701341)(248.71833169,552.35879248)(249.29077565,551.28857117)
\curveto(249.88810847,550.24323872)(250.95832978,549.7205725)(252.50143958,549.7205725)
\curveto(253.77077183,549.7205725)(254.70410437,550.24323872)(255.3014372,551.28857117)
\curveto(255.92365889,552.35879248)(256.23476973,553.88945784)(256.23476973,555.88056726)
\curveto(256.23476973,559.88767496)(254.96543748,561.8912288)(252.42677298,561.8912288)
\closepath
}
}
{
\newrgbcolor{curcolor}{0 0 0}
\pscustom[linestyle=none,fillstyle=solid,fillcolor=curcolor]
{
\newpath
\moveto(274.86406534,566.37122499)
\curveto(277.60184078,566.37122499)(279.69250567,565.77389217)(281.13606,564.57922652)
\curveto(282.60450319,563.40944973)(283.33872479,561.60500683)(283.33872479,559.16589779)
\lineto(283.33872479,545.57657603)
\lineto(279.45606143,545.57657603)
\lineto(278.37339568,548.33924034)
\lineto(278.22406247,548.33924034)
\curveto(277.35295211,547.24413017)(276.432064,546.4476864)(275.46139816,545.94990905)
\curveto(274.49073232,545.45213169)(273.1591779,545.20324301)(271.4667349,545.20324301)
\curveto(269.64984755,545.20324301)(268.14407106,545.72590924)(266.94940541,546.77124168)
\curveto(265.75473976,547.81657412)(265.15740693,549.44679496)(265.15740693,551.66190418)
\curveto(265.15740693,553.82723567)(265.9165174,555.4201232)(267.43473833,556.44056678)
\curveto(268.95295926,557.46101035)(271.23029065,558.03345431)(274.26673251,558.15789865)
\lineto(277.81339616,558.26989855)
\lineto(277.81339616,559.16589779)
\curveto(277.81339616,560.2361191)(277.52717418,561.02011844)(276.95473022,561.51789579)
\curveto(276.40717513,562.01567314)(275.63562023,562.26456182)(274.64006553,562.26456182)
\curveto(273.64451082,562.26456182)(272.67384498,562.11522861)(271.72806801,561.8165622)
\curveto(270.78229103,561.54278466)(269.83651406,561.19434051)(268.89073709,560.77122976)
\lineto(267.06140531,564.54189321)
\curveto(268.13162662,565.0894483)(269.33873671,565.52500349)(270.68273556,565.84855877)
\curveto(272.02673442,566.19700292)(273.42051101,566.37122499)(274.86406534,566.37122499)
\closepath
\moveto(277.81339616,555.02190132)
\lineto(275.64806467,554.94723472)
\curveto(273.85606619,554.89745698)(272.61162281,554.5739017)(271.91473451,553.97656888)
\curveto(271.21784622,553.37923605)(270.86940207,552.59523672)(270.86940207,551.62457088)
\curveto(270.86940207,550.77834938)(271.11829075,550.16857212)(271.6160681,549.7952391)
\curveto(272.11384546,549.44679496)(272.76095602,549.27257288)(273.55739978,549.27257288)
\curveto(274.75206543,549.27257288)(275.76006457,549.62101703)(276.58139721,550.31790533)
\curveto(277.40272984,551.03968249)(277.81339616,552.04768163)(277.81339616,553.34190275)
\closepath
}
}
{
\newrgbcolor{curcolor}{0 0 0}
\pscustom[linestyle=none,fillstyle=solid,fillcolor=curcolor]
{
\newpath
\moveto(307.12004014,560.62189655)
\curveto(307.12004014,559.52678637)(306.771596,558.59345383)(306.0747077,557.82189894)
\curveto(305.40270827,557.05034404)(304.39470913,556.55256668)(303.05071028,556.32856687)
\lineto(303.05071028,556.17923367)
\curveto(304.46937573,556.00501159)(305.60181921,555.50723424)(306.44804072,554.68590161)
\curveto(307.31915109,553.88945784)(307.75470627,552.8814587)(307.75470627,551.66190418)
\curveto(307.75470627,550.4921274)(307.44359542,549.44679496)(306.82137373,548.52590685)
\curveto(306.22404091,547.60501875)(305.2658195,546.88324158)(303.94670951,546.36057536)
\curveto(302.62759952,545.83790914)(300.89782322,545.57657603)(298.7573806,545.57657603)
\lineto(289.0507222,545.57657603)
\lineto(289.0507222,565.96055867)
\lineto(298.7573806,565.96055867)
\curveto(300.35026813,565.96055867)(301.76893359,565.7863366)(303.01337697,565.43789245)
\curveto(304.28270923,565.11433717)(305.27826393,564.55433765)(306.0000411,563.75789388)
\curveto(306.74670713,562.98633898)(307.12004014,561.94100654)(307.12004014,560.62189655)
\closepath
\moveto(301.48271161,560.17389693)
\curveto(301.48271161,561.41834032)(300.49960134,562.04056201)(298.53338079,562.04056201)
\lineto(294.61338413,562.04056201)
\lineto(294.61338413,558.00856544)
\lineto(297.89871466,558.00856544)
\curveto(299.06849144,558.00856544)(299.95204625,558.17034308)(300.54937907,558.49389836)
\curveto(301.17160076,558.84234251)(301.48271161,559.40234204)(301.48271161,560.17389693)
\closepath
\moveto(302.00537783,551.96057059)
\curveto(302.00537783,552.75701436)(301.68182255,553.32945832)(301.03471199,553.67790246)
\curveto(300.4124903,554.05123548)(299.49160219,554.23790199)(298.27204768,554.23790199)
\lineto(294.61338413,554.23790199)
\lineto(294.61338413,549.42190609)
\lineto(298.38404758,549.42190609)
\curveto(299.42938003,549.42190609)(300.28804596,549.6085726)(300.96004539,549.98190561)
\curveto(301.65693368,550.3801275)(302.00537783,551.03968249)(302.00537783,551.96057059)
\closepath
}
}
{
\newrgbcolor{curcolor}{0 0 0}
\pscustom[linestyle=none,fillstyle=solid,fillcolor=curcolor]
{
\newpath
\moveto(327.80268466,565.96055867)
\lineto(327.80268466,558.12056535)
\lineto(335.56801138,558.12056535)
\lineto(335.56801138,565.96055867)
\lineto(341.13067331,565.96055867)
\lineto(341.13067331,545.57657603)
\lineto(335.56801138,545.57657603)
\lineto(335.56801138,553.97656888)
\lineto(327.80268466,553.97656888)
\lineto(327.80268466,545.57657603)
\lineto(322.24002273,545.57657603)
\lineto(322.24002273,565.96055867)
\closepath
}
}
{
\newrgbcolor{curcolor}{0 0 0}
\pscustom[linestyle=none,fillstyle=solid,fillcolor=curcolor]
{
\newpath
\moveto(355.31734292,566.37122499)
\curveto(358.05511837,566.37122499)(360.14578326,565.77389217)(361.58933758,564.57922652)
\curveto(363.05778078,563.40944973)(363.79200237,561.60500683)(363.79200237,559.16589779)
\lineto(363.79200237,545.57657603)
\lineto(359.90933901,545.57657603)
\lineto(358.82667327,548.33924034)
\lineto(358.67734006,548.33924034)
\curveto(357.80622969,547.24413017)(356.88534159,546.4476864)(355.91467575,545.94990905)
\curveto(354.94400991,545.45213169)(353.61245549,545.20324301)(351.92001248,545.20324301)
\curveto(350.10312514,545.20324301)(348.59734865,545.72590924)(347.402683,546.77124168)
\curveto(346.20801735,547.81657412)(345.61068452,549.44679496)(345.61068452,551.66190418)
\curveto(345.61068452,553.82723567)(346.36979499,555.4201232)(347.88801592,556.44056678)
\curveto(349.40623685,557.46101035)(351.68356824,558.03345431)(354.7200101,558.15789865)
\lineto(358.26667375,558.26989855)
\lineto(358.26667375,559.16589779)
\curveto(358.26667375,560.2361191)(357.98045177,561.02011844)(357.40800781,561.51789579)
\curveto(356.86045272,562.01567314)(356.08889782,562.26456182)(355.09334311,562.26456182)
\curveto(354.09778841,562.26456182)(353.12712257,562.11522861)(352.18134559,561.8165622)
\curveto(351.23556862,561.54278466)(350.28979165,561.19434051)(349.34401468,560.77122976)
\lineto(347.5146829,564.54189321)
\curveto(348.58490421,565.0894483)(349.7920143,565.52500349)(351.13601315,565.84855877)
\curveto(352.48001201,566.19700292)(353.8737886,566.37122499)(355.31734292,566.37122499)
\closepath
\moveto(358.26667375,555.02190132)
\lineto(356.10134226,554.94723472)
\curveto(354.30934378,554.89745698)(353.0649004,554.5739017)(352.3680121,553.97656888)
\curveto(351.67112381,553.37923605)(351.32267966,552.59523672)(351.32267966,551.62457088)
\curveto(351.32267966,550.77834938)(351.57156834,550.16857212)(352.06934569,549.7952391)
\curveto(352.56712304,549.44679496)(353.2142336,549.27257288)(354.01067737,549.27257288)
\curveto(355.20534302,549.27257288)(356.21334216,549.62101703)(357.03467479,550.31790533)
\curveto(357.85600743,551.03968249)(358.26667375,552.04768163)(358.26667375,553.34190275)
\closepath
}
}
{
\newrgbcolor{curcolor}{0 0 0}
\pscustom[linestyle=none,fillstyle=solid,fillcolor=curcolor]
{
\newpath
\moveto(390.55998849,566.33389169)
\curveto(392.84976432,566.33389169)(394.70398496,565.43789245)(396.12265042,563.64589398)
\curveto(397.54131588,561.87878437)(398.25064861,559.26545326)(398.25064861,555.80590065)
\curveto(398.25064861,552.32145918)(397.51642701,549.6832392)(396.04798382,547.89124073)
\curveto(394.57954063,546.09924225)(392.70043111,545.20324301)(390.41065529,545.20324301)
\curveto(388.94221209,545.20324301)(387.77243531,545.46457613)(386.90132494,545.98724235)
\curveto(386.03021457,546.53479744)(385.32088184,547.14457469)(384.77332675,547.81657412)
\lineto(384.47466034,547.81657412)
\curveto(384.67377128,546.77124168)(384.77332675,545.77568697)(384.77332675,544.82991)
\lineto(384.77332675,536.61658366)
\lineto(379.21066482,536.61658366)
\lineto(379.21066482,565.96055867)
\lineto(383.72799431,565.96055867)
\lineto(384.51199364,563.30989426)
\lineto(384.77332675,563.30989426)
\curveto(385.32088184,564.1312269)(386.05510344,564.84055963)(386.97599154,565.43789245)
\curveto(387.89687965,566.03522528)(389.0915453,566.33389169)(390.55998849,566.33389169)
\closepath
\moveto(388.76799002,561.8912288)
\curveto(387.32443569,561.8912288)(386.30399212,561.43078475)(385.70665929,560.50989665)
\curveto(385.10932647,559.61389741)(384.79821562,558.25745412)(384.77332675,556.44056678)
\lineto(384.77332675,555.84323395)
\curveto(384.77332675,553.87701341)(385.05954873,552.35879248)(385.63199269,551.28857117)
\curveto(386.22932551,550.24323872)(387.29954682,549.7205725)(388.84265662,549.7205725)
\curveto(390.11198887,549.7205725)(391.04532141,550.24323872)(391.64265424,551.28857117)
\curveto(392.26487593,552.35879248)(392.57598678,553.88945784)(392.57598678,555.88056726)
\curveto(392.57598678,559.88767496)(391.30665452,561.8912288)(388.76799002,561.8912288)
\closepath
}
}
{
\newrgbcolor{curcolor}{0 0 0}
\pscustom[linestyle=none,fillstyle=solid,fillcolor=curcolor]
{
\newpath
\moveto(411.24261568,566.33389169)
\curveto(414.05505773,566.33389169)(416.28261139,565.52500349)(417.92527665,563.90722709)
\curveto(419.56794192,562.31433956)(420.38927456,560.03700816)(420.38927456,557.07523291)
\lineto(420.38927456,554.38723519)
\lineto(407.24795241,554.38723519)
\curveto(407.29773015,552.81923653)(407.7581742,551.58723758)(408.62928457,550.69123834)
\curveto(409.52528381,549.7952391)(410.75728276,549.34723949)(412.32528142,549.34723949)
\curveto(413.61950254,549.34723949)(414.80172376,549.47168382)(415.87194507,549.7205725)
\curveto(416.96705525,549.99435005)(418.08705429,550.40501636)(419.23194221,550.95257145)
\lineto(419.23194221,546.65924177)
\curveto(418.21149863,546.16146442)(417.15372176,545.80057584)(416.05861158,545.57657603)
\curveto(414.9635014,545.32768735)(413.63194698,545.20324301)(412.06394831,545.20324301)
\curveto(410.02306116,545.20324301)(408.21861825,545.57657603)(406.65061959,546.32324206)
\curveto(405.08262092,547.09479696)(403.85062197,548.23968487)(402.95462274,549.7579058)
\curveto(402.0586235,551.3010156)(401.61062388,553.25479171)(401.61062388,555.61923415)
\curveto(401.61062388,557.98367658)(402.00884576,559.96234156)(402.80528953,561.55522909)
\curveto(403.62662216,563.14811662)(404.75906564,564.34278227)(406.20261997,565.13922604)
\curveto(407.6461743,565.93566981)(409.32617287,566.33389169)(411.24261568,566.33389169)
\closepath
\moveto(411.27994898,562.37656172)
\curveto(410.1848388,562.37656172)(409.28883956,562.02811758)(408.59195127,561.33122928)
\curveto(407.89506297,560.63434099)(407.48439666,559.55167524)(407.35995232,558.08323205)
\lineto(415.16261234,558.08323205)
\curveto(415.13772347,559.30278656)(414.80172376,560.32323014)(414.1546132,561.14456277)
\curveto(413.53239151,561.96589541)(412.5741701,562.37656172)(411.27994898,562.37656172)
\closepath
}
}
{
\newrgbcolor{curcolor}{0 0 0}
\pscustom[linestyle=none,fillstyle=solid,fillcolor=curcolor]
{
\newpath
\moveto(443.23726925,565.96055867)
\lineto(443.23726925,549.6459059)
\lineto(446.22393337,549.6459059)
\lineto(446.22393337,538.25924893)
\lineto(441.22127097,538.25924893)
\lineto(441.22127097,545.57657603)
\lineto(427.5199493,545.57657603)
\lineto(427.5199493,538.25924893)
\lineto(422.51728689,538.25924893)
\lineto(422.51728689,549.6459059)
\lineto(424.23461876,549.6459059)
\curveto(425.130618,551.01479362)(425.88972847,552.57034785)(426.51195016,554.31256859)
\curveto(427.13417185,556.0796782)(427.63194921,557.94634327)(428.00528222,559.91256382)
\curveto(428.37861524,561.90367324)(428.65239278,563.91967152)(428.82661485,565.96055867)
\closepath
\moveto(437.67460732,561.7792289)
\lineto(433.49327755,561.7792289)
\curveto(433.19461114,559.51434194)(432.78394482,557.36145488)(432.2612786,555.32056773)
\curveto(431.73861237,553.30456945)(431.00439078,551.4130155)(430.05861381,549.6459059)
\lineto(437.67460732,549.6459059)
\closepath
}
}
{
\newrgbcolor{curcolor}{0 0 0}
\pscustom[linestyle=none,fillstyle=solid,fillcolor=curcolor]
{
\newpath
\moveto(458.05855386,566.37122499)
\curveto(460.79632931,566.37122499)(462.88699419,565.77389217)(464.33054852,564.57922652)
\curveto(465.79899171,563.40944973)(466.53321331,561.60500683)(466.53321331,559.16589779)
\lineto(466.53321331,545.57657603)
\lineto(462.65054995,545.57657603)
\lineto(461.56788421,548.33924034)
\lineto(461.418551,548.33924034)
\curveto(460.54744063,547.24413017)(459.62655253,546.4476864)(458.65588669,545.94990905)
\curveto(457.68522085,545.45213169)(456.35366642,545.20324301)(454.66122342,545.20324301)
\curveto(452.84433608,545.20324301)(451.33855958,545.72590924)(450.14389393,546.77124168)
\curveto(448.94922828,547.81657412)(448.35189546,549.44679496)(448.35189546,551.66190418)
\curveto(448.35189546,553.82723567)(449.11100592,555.4201232)(450.62922685,556.44056678)
\curveto(452.14744778,557.46101035)(454.42477918,558.03345431)(457.46122104,558.15789865)
\lineto(461.00788468,558.26989855)
\lineto(461.00788468,559.16589779)
\curveto(461.00788468,560.2361191)(460.7216627,561.02011844)(460.14921875,561.51789579)
\curveto(459.60166366,562.01567314)(458.83010876,562.26456182)(457.83455405,562.26456182)
\curveto(456.83899934,562.26456182)(455.8683335,562.11522861)(454.92255653,561.8165622)
\curveto(453.97677956,561.54278466)(453.03100259,561.19434051)(452.08522561,560.77122976)
\lineto(450.25589384,564.54189321)
\curveto(451.32611515,565.0894483)(452.53322523,565.52500349)(453.87722409,565.84855877)
\curveto(455.22122294,566.19700292)(456.61499953,566.37122499)(458.05855386,566.37122499)
\closepath
\moveto(461.00788468,555.02190132)
\lineto(458.84255319,554.94723472)
\curveto(457.05055472,554.89745698)(455.80611133,554.5739017)(455.10922304,553.97656888)
\curveto(454.41233474,553.37923605)(454.0638906,552.59523672)(454.0638906,551.62457088)
\curveto(454.0638906,550.77834938)(454.31277927,550.16857212)(454.81055663,549.7952391)
\curveto(455.30833398,549.44679496)(455.95544454,549.27257288)(456.75188831,549.27257288)
\curveto(457.94655396,549.27257288)(458.9545531,549.62101703)(459.77588573,550.31790533)
\curveto(460.59721837,551.03968249)(461.00788468,552.04768163)(461.00788468,553.34190275)
\closepath
}
}
{
\newrgbcolor{curcolor}{0 0 0}
\pscustom[linestyle=none,fillstyle=solid,fillcolor=curcolor]
{
\newpath
\moveto(485.61053267,565.96055867)
\lineto(491.73319413,565.96055867)
\lineto(483.66920099,556.17923367)
\lineto(492.44252686,545.57657603)
\lineto(486.1331989,545.57657603)
\lineto(477.80787265,555.91790056)
\lineto(477.80787265,545.57657603)
\lineto(472.24521072,545.57657603)
\lineto(472.24521072,565.96055867)
\lineto(477.80787265,565.96055867)
\lineto(477.80787265,556.06723376)
\closepath
}
}
{
\newrgbcolor{curcolor}{0 0 0}
\pscustom[linestyle=none,fillstyle=solid,fillcolor=curcolor]
{
\newpath
\moveto(512.22917698,561.7792289)
\lineto(505.54651601,561.7792289)
\lineto(505.54651601,545.57657603)
\lineto(499.98385408,545.57657603)
\lineto(499.98385408,561.7792289)
\lineto(493.3011931,561.7792289)
\lineto(493.3011931,565.96055867)
\lineto(512.22917698,565.96055867)
\closepath
}
}
{
\newrgbcolor{curcolor}{0 0 0}
\pscustom[linestyle=none,fillstyle=solid,fillcolor=curcolor]
{
\newpath
\moveto(521.3758143,565.96055867)
\lineto(521.3758143,557.89656554)
\curveto(521.3758143,557.47345479)(521.35092543,556.95078857)(521.3011477,556.32856687)
\curveto(521.27625883,555.70634518)(521.23892553,555.07167906)(521.18914779,554.4245685)
\curveto(521.16425892,553.77745794)(521.12692562,553.19256954)(521.07714789,552.66990332)
\curveto(521.02737015,552.17212597)(520.99003685,551.83612626)(520.96514798,551.66190418)
\lineto(530.37313997,565.96055867)
\lineto(537.05580095,565.96055867)
\lineto(537.05580095,545.57657603)
\lineto(531.67980552,545.57657603)
\lineto(531.67980552,553.71523577)
\curveto(531.67980552,554.36234633)(531.70469439,555.09656792)(531.75447213,555.91790056)
\curveto(531.80424986,556.73923319)(531.8540276,557.49834366)(531.90380533,558.19523195)
\curveto(531.97847194,558.91700911)(532.02824967,559.4645642)(532.05313854,559.83789722)
\lineto(522.68247985,545.57657603)
\lineto(515.99981888,545.57657603)
\lineto(515.99981888,565.96055867)
\closepath
}
}
{
\newrgbcolor{curcolor}{0 0 0}
\pscustom[linestyle=none,fillstyle=solid,fillcolor=curcolor]
{
\newpath
\moveto(554.22908639,566.33389169)
\curveto(556.51886222,566.33389169)(558.37308286,565.43789245)(559.79174832,563.64589398)
\curveto(561.21041378,561.87878437)(561.91974651,559.26545326)(561.91974651,555.80590065)
\curveto(561.91974651,552.32145918)(561.18552491,549.6832392)(559.71708172,547.89124073)
\curveto(558.24863853,546.09924225)(556.36952901,545.20324301)(554.07975319,545.20324301)
\curveto(552.61130999,545.20324301)(551.44153321,545.46457613)(550.57042284,545.98724235)
\curveto(549.69931247,546.53479744)(548.98997974,547.14457469)(548.44242465,547.81657412)
\lineto(548.14375824,547.81657412)
\curveto(548.34286918,546.77124168)(548.44242465,545.77568697)(548.44242465,544.82991)
\lineto(548.44242465,536.61658366)
\lineto(542.87976272,536.61658366)
\lineto(542.87976272,565.96055867)
\lineto(547.39709221,565.96055867)
\lineto(548.18109154,563.30989426)
\lineto(548.44242465,563.30989426)
\curveto(548.98997974,564.1312269)(549.72420134,564.84055963)(550.64508944,565.43789245)
\curveto(551.56597755,566.03522528)(552.7606432,566.33389169)(554.22908639,566.33389169)
\closepath
\moveto(552.43708792,561.8912288)
\curveto(550.99353359,561.8912288)(549.97309002,561.43078475)(549.37575719,560.50989665)
\curveto(548.77842437,559.61389741)(548.46731352,558.25745412)(548.44242465,556.44056678)
\lineto(548.44242465,555.84323395)
\curveto(548.44242465,553.87701341)(548.72864663,552.35879248)(549.30109059,551.28857117)
\curveto(549.89842341,550.24323872)(550.96864472,549.7205725)(552.51175452,549.7205725)
\curveto(553.78108677,549.7205725)(554.71441931,550.24323872)(555.31175214,551.28857117)
\curveto(555.93397383,552.35879248)(556.24508468,553.88945784)(556.24508468,555.88056726)
\curveto(556.24508468,559.88767496)(554.97575242,561.8912288)(552.43708792,561.8912288)
\closepath
}
}
{
\newrgbcolor{curcolor}{0 0 0}
\pscustom[linestyle=none,fillstyle=solid,fillcolor=curcolor]
{
\newpath
\moveto(585.0290383,555.80590065)
\curveto(585.0290383,552.42101465)(584.13303906,549.80768354)(582.34104059,547.96590733)
\curveto(580.57393098,546.12413112)(578.15971081,545.20324301)(575.09838009,545.20324301)
\curveto(573.20682614,545.20324301)(571.51438314,545.61390933)(570.02105108,546.43524197)
\curveto(568.55260788,547.2565746)(567.39527553,548.45124025)(566.54905403,550.01923891)
\curveto(565.70283253,551.61212645)(565.27972178,553.54101369)(565.27972178,555.80590065)
\curveto(565.27972178,559.19078666)(566.16327658,561.79167333)(567.93038619,563.60856068)
\curveto(569.6974958,565.42544802)(572.1241604,566.33389169)(575.21037999,566.33389169)
\curveto(577.1268228,566.33389169)(578.81926581,565.92322537)(580.287709,565.10189274)
\curveto(581.7561522,564.2805601)(582.91348454,563.08589445)(583.75970604,561.51789579)
\curveto(584.60592755,559.94989712)(585.0290383,558.04589875)(585.0290383,555.80590065)
\closepath
\moveto(570.95438362,555.80590065)
\curveto(570.95438362,553.78990237)(571.2779389,552.25923701)(571.92504946,551.21390456)
\curveto(572.59704888,550.19346099)(573.67971463,549.6832392)(575.17304669,549.6832392)
\curveto(576.64148988,549.6832392)(577.69926676,550.19346099)(578.34637732,551.21390456)
\curveto(579.01837675,552.25923701)(579.35437646,553.78990237)(579.35437646,555.80590065)
\curveto(579.35437646,557.82189894)(579.01837675,559.32767543)(578.34637732,560.32323014)
\curveto(577.69926676,561.34367372)(576.62904545,561.8538955)(575.13571339,561.8538955)
\curveto(573.66727019,561.8538955)(572.59704888,561.34367372)(571.92504946,560.32323014)
\curveto(571.2779389,559.32767543)(570.95438362,557.82189894)(570.95438362,555.80590065)
\closepath
}
}
{
\newrgbcolor{curcolor}{0 0 0}
\pscustom[linestyle=none,fillstyle=solid,fillcolor=curcolor]
{
\newpath
\moveto(607.69035264,560.62189655)
\curveto(607.69035264,559.52678637)(607.3419085,558.59345383)(606.6450202,557.82189894)
\curveto(605.97302077,557.05034404)(604.96502163,556.55256668)(603.62102278,556.32856687)
\lineto(603.62102278,556.17923367)
\curveto(605.03968823,556.00501159)(606.17213171,555.50723424)(607.01835322,554.68590161)
\curveto(607.88946359,553.88945784)(608.32501877,552.8814587)(608.32501877,551.66190418)
\curveto(608.32501877,550.4921274)(608.01390792,549.44679496)(607.39168623,548.52590685)
\curveto(606.79435341,547.60501875)(605.836132,546.88324158)(604.51702201,546.36057536)
\curveto(603.19791202,545.83790914)(601.46813572,545.57657603)(599.3276931,545.57657603)
\lineto(589.6210347,545.57657603)
\lineto(589.6210347,565.96055867)
\lineto(599.3276931,565.96055867)
\curveto(600.92058063,565.96055867)(602.33924609,565.7863366)(603.58368947,565.43789245)
\curveto(604.85302173,565.11433717)(605.84857643,564.55433765)(606.5703536,563.75789388)
\curveto(607.31701963,562.98633898)(607.69035264,561.94100654)(607.69035264,560.62189655)
\closepath
\moveto(602.05302411,560.17389693)
\curveto(602.05302411,561.41834032)(601.06991384,562.04056201)(599.10369329,562.04056201)
\lineto(595.18369663,562.04056201)
\lineto(595.18369663,558.00856544)
\lineto(598.46902716,558.00856544)
\curveto(599.63880394,558.00856544)(600.52235875,558.17034308)(601.11969157,558.49389836)
\curveto(601.74191326,558.84234251)(602.05302411,559.40234204)(602.05302411,560.17389693)
\closepath
\moveto(602.57569033,551.96057059)
\curveto(602.57569033,552.75701436)(602.25213505,553.32945832)(601.60502449,553.67790246)
\curveto(600.9828028,554.05123548)(600.06191469,554.23790199)(598.84236018,554.23790199)
\lineto(595.18369663,554.23790199)
\lineto(595.18369663,549.42190609)
\lineto(598.95436008,549.42190609)
\curveto(599.99969253,549.42190609)(600.85835846,549.6085726)(601.53035789,549.98190561)
\curveto(602.22724618,550.3801275)(602.57569033,551.03968249)(602.57569033,551.96057059)
\closepath
}
}
{
\newrgbcolor{curcolor}{0 0 0}
\pscustom[linestyle=none,fillstyle=solid,fillcolor=curcolor]
{
\newpath
\moveto(621.46629922,566.37122499)
\curveto(624.20407467,566.37122499)(626.29473956,565.77389217)(627.73829388,564.57922652)
\curveto(629.20673708,563.40944973)(629.94095867,561.60500683)(629.94095867,559.16589779)
\lineto(629.94095867,545.57657603)
\lineto(626.05829531,545.57657603)
\lineto(624.97562957,548.33924034)
\lineto(624.82629636,548.33924034)
\curveto(623.95518599,547.24413017)(623.03429789,546.4476864)(622.06363205,545.94990905)
\curveto(621.09296621,545.45213169)(619.76141179,545.20324301)(618.06896878,545.20324301)
\curveto(616.25208144,545.20324301)(614.74630494,545.72590924)(613.5516393,546.77124168)
\curveto(612.35697365,547.81657412)(611.75964082,549.44679496)(611.75964082,551.66190418)
\curveto(611.75964082,553.82723567)(612.51875129,555.4201232)(614.03697222,556.44056678)
\curveto(615.55519314,557.46101035)(617.83252454,558.03345431)(620.8689664,558.15789865)
\lineto(624.41563004,558.26989855)
\lineto(624.41563004,559.16589779)
\curveto(624.41563004,560.2361191)(624.12940807,561.02011844)(623.55696411,561.51789579)
\curveto(623.00940902,562.01567314)(622.23785412,562.26456182)(621.24229941,562.26456182)
\curveto(620.24674471,562.26456182)(619.27607887,562.11522861)(618.33030189,561.8165622)
\curveto(617.38452492,561.54278466)(616.43874795,561.19434051)(615.49297098,560.77122976)
\lineto(613.6636392,564.54189321)
\curveto(614.73386051,565.0894483)(615.94097059,565.52500349)(617.28496945,565.84855877)
\curveto(618.62896831,566.19700292)(620.0227449,566.37122499)(621.46629922,566.37122499)
\closepath
\moveto(624.41563004,555.02190132)
\lineto(622.25029856,554.94723472)
\curveto(620.45830008,554.89745698)(619.2138567,554.5739017)(618.5169684,553.97656888)
\curveto(617.82008011,553.37923605)(617.47163596,552.59523672)(617.47163596,551.62457088)
\curveto(617.47163596,550.77834938)(617.72052463,550.16857212)(618.21830199,549.7952391)
\curveto(618.71607934,549.44679496)(619.3631899,549.27257288)(620.15963367,549.27257288)
\curveto(621.35429932,549.27257288)(622.36229846,549.62101703)(623.18363109,550.31790533)
\curveto(624.00496373,551.03968249)(624.41563004,552.04768163)(624.41563004,553.34190275)
\closepath
}
}
{
\newrgbcolor{curcolor}{0 0 0}
\pscustom[linestyle=none,fillstyle=solid,fillcolor=curcolor]
{
\newpath
\moveto(641.21561801,565.96055867)
\lineto(641.21561801,558.12056535)
\lineto(648.98094473,558.12056535)
\lineto(648.98094473,565.96055867)
\lineto(654.54360666,565.96055867)
\lineto(654.54360666,545.57657603)
\lineto(648.98094473,545.57657603)
\lineto(648.98094473,553.97656888)
\lineto(641.21561801,553.97656888)
\lineto(641.21561801,545.57657603)
\lineto(635.65295608,545.57657603)
\lineto(635.65295608,565.96055867)
\closepath
}
}
{
\newrgbcolor{curcolor}{0 0 0}
\pscustom[linestyle=none,fillstyle=solid,fillcolor=curcolor]
{
\newpath
\moveto(665.74361215,565.96055867)
\lineto(665.74361215,557.89656554)
\curveto(665.74361215,557.47345479)(665.71872328,556.95078857)(665.66894555,556.32856687)
\curveto(665.64405668,555.70634518)(665.60672338,555.07167906)(665.55694564,554.4245685)
\curveto(665.53205677,553.77745794)(665.49472347,553.19256954)(665.44494574,552.66990332)
\curveto(665.395168,552.17212597)(665.3578347,551.83612626)(665.33294583,551.66190418)
\lineto(674.74093782,565.96055867)
\lineto(681.4235988,565.96055867)
\lineto(681.4235988,545.57657603)
\lineto(676.04760338,545.57657603)
\lineto(676.04760338,553.71523577)
\curveto(676.04760338,554.36234633)(676.07249224,555.09656792)(676.12226998,555.91790056)
\curveto(676.17204771,556.73923319)(676.22182545,557.49834366)(676.27160319,558.19523195)
\curveto(676.34626979,558.91700911)(676.39604752,559.4645642)(676.42093639,559.83789722)
\lineto(667.0502777,545.57657603)
\lineto(660.36761673,545.57657603)
\lineto(660.36761673,565.96055867)
\closepath
}
}
{
\newrgbcolor{curcolor}{0 0 0}
\pscustom[linestyle=none,fillstyle=solid,fillcolor=curcolor]
{
\newpath
\moveto(695.64755342,566.33389169)
\curveto(698.45999547,566.33389169)(700.68754913,565.52500349)(702.3302144,563.90722709)
\curveto(703.97287967,562.31433956)(704.7942123,560.03700816)(704.7942123,557.07523291)
\lineto(704.7942123,554.38723519)
\lineto(691.65289016,554.38723519)
\curveto(691.70266789,552.81923653)(692.16311195,551.58723758)(693.03422231,550.69123834)
\curveto(693.93022155,549.7952391)(695.1622205,549.34723949)(696.73021917,549.34723949)
\curveto(698.02444029,549.34723949)(699.2066615,549.47168382)(700.27688281,549.7205725)
\curveto(701.37199299,549.99435005)(702.49199204,550.40501636)(703.63687995,550.95257145)
\lineto(703.63687995,546.65924177)
\curveto(702.61643638,546.16146442)(701.5586595,545.80057584)(700.46354932,545.57657603)
\curveto(699.36843914,545.32768735)(698.03688472,545.20324301)(696.46888606,545.20324301)
\curveto(694.42799891,545.20324301)(692.623556,545.57657603)(691.05555733,546.32324206)
\curveto(689.48755867,547.09479696)(688.25555972,548.23968487)(687.35956048,549.7579058)
\curveto(686.46356124,551.3010156)(686.01556162,553.25479171)(686.01556162,555.61923415)
\curveto(686.01556162,557.98367658)(686.41378351,559.96234156)(687.21022727,561.55522909)
\curveto(688.03155991,563.14811662)(689.16400339,564.34278227)(690.60755771,565.13922604)
\curveto(692.05111204,565.93566981)(693.73111061,566.33389169)(695.64755342,566.33389169)
\closepath
\moveto(695.68488672,562.37656172)
\curveto(694.58977655,562.37656172)(693.69377731,562.02811758)(692.99688901,561.33122928)
\curveto(692.30000072,560.63434099)(691.8893344,559.55167524)(691.76489006,558.08323205)
\lineto(699.56755008,558.08323205)
\curveto(699.54266122,559.30278656)(699.2066615,560.32323014)(698.55955094,561.14456277)
\curveto(697.93732925,561.96589541)(696.97910784,562.37656172)(695.68488672,562.37656172)
\closepath
}
}
{
\newrgbcolor{curcolor}{0 0 0}
\pscustom[linestyle=none,fillstyle=solid,fillcolor=curcolor]
{
\newpath
\moveto(737.34884152,565.96055867)
\lineto(737.34884152,549.6459059)
\lineto(740.33550564,549.6459059)
\lineto(740.33550564,538.25924893)
\lineto(735.33284323,538.25924893)
\lineto(735.33284323,545.57657603)
\lineto(721.63152157,545.57657603)
\lineto(721.63152157,538.25924893)
\lineto(716.62885916,538.25924893)
\lineto(716.62885916,549.6459059)
\lineto(718.34619103,549.6459059)
\curveto(719.24219027,551.01479362)(720.00130073,552.57034785)(720.62352242,554.31256859)
\curveto(721.24574412,556.0796782)(721.74352147,557.94634327)(722.11685449,559.91256382)
\curveto(722.4901875,561.90367324)(722.76396505,563.91967152)(722.93818712,565.96055867)
\closepath
\moveto(731.78617959,561.7792289)
\lineto(727.60484981,561.7792289)
\curveto(727.3061834,559.51434194)(726.89551708,557.36145488)(726.37285086,555.32056773)
\curveto(725.85018464,553.30456945)(725.11596304,551.4130155)(724.17018607,549.6459059)
\lineto(731.78617959,549.6459059)
\closepath
}
}
{
\newrgbcolor{curcolor}{0 0 0}
\pscustom[linestyle=none,fillstyle=solid,fillcolor=curcolor]
{
\newpath
\moveto(762.32481466,555.80590065)
\curveto(762.32481466,552.42101465)(761.42881543,549.80768354)(759.63681695,547.96590733)
\curveto(757.86970735,546.12413112)(755.45548718,545.20324301)(752.39415645,545.20324301)
\curveto(750.50260251,545.20324301)(748.81015951,545.61390933)(747.31682744,546.43524197)
\curveto(745.84838425,547.2565746)(744.6910519,548.45124025)(743.8448304,550.01923891)
\curveto(742.9986089,551.61212645)(742.57549815,553.54101369)(742.57549815,555.80590065)
\curveto(742.57549815,559.19078666)(743.45905295,561.79167333)(745.22616256,563.60856068)
\curveto(746.99327216,565.42544802)(749.41993676,566.33389169)(752.50615636,566.33389169)
\curveto(754.42259917,566.33389169)(756.11504217,565.92322537)(757.58348537,565.10189274)
\curveto(759.05192856,564.2805601)(760.20926091,563.08589445)(761.05548241,561.51789579)
\curveto(761.90170391,559.94989712)(762.32481466,558.04589875)(762.32481466,555.80590065)
\closepath
\moveto(748.25015998,555.80590065)
\curveto(748.25015998,553.78990237)(748.57371526,552.25923701)(749.22082582,551.21390456)
\curveto(749.89282525,550.19346099)(750.975491,549.6832392)(752.46882306,549.6832392)
\curveto(753.93726625,549.6832392)(754.99504313,550.19346099)(755.64215369,551.21390456)
\curveto(756.31415312,552.25923701)(756.65015283,553.78990237)(756.65015283,555.80590065)
\curveto(756.65015283,557.82189894)(756.31415312,559.32767543)(755.64215369,560.32323014)
\curveto(754.99504313,561.34367372)(753.92482182,561.8538955)(752.43148976,561.8538955)
\curveto(750.96304656,561.8538955)(749.89282525,561.34367372)(749.22082582,560.32323014)
\curveto(748.57371526,559.32767543)(748.25015998,557.82189894)(748.25015998,555.80590065)
\closepath
}
}
{
\newrgbcolor{curcolor}{0 0 0}
\pscustom[linestyle=none,fillstyle=solid,fillcolor=curcolor]
{
\newpath
\moveto(775.20480401,545.20324301)
\curveto(772.16836215,545.20324301)(769.81636415,546.03702008)(768.14881001,547.70457422)
\curveto(766.50614475,549.37212835)(765.68481211,552.02279276)(765.68481211,555.65656745)
\curveto(765.68481211,558.14545422)(766.10792286,560.17389693)(766.95414436,561.7418956)
\curveto(767.80036587,563.30989426)(768.97014265,564.46722661)(770.46347471,565.21389264)
\curveto(771.98169564,565.96055867)(773.72391638,566.33389169)(775.69013693,566.33389169)
\curveto(777.08391352,566.33389169)(778.2910236,566.19700292)(779.31146718,565.92322537)
\curveto(780.35679962,565.64944783)(781.26524329,565.32589255)(782.03679819,564.95255953)
\lineto(780.39413292,560.65922985)
\curveto(779.52302255,561.007674)(778.70168992,561.29389598)(777.93013502,561.51789579)
\curveto(777.18346899,561.7418956)(776.43680296,561.8538955)(775.69013693,561.8538955)
\curveto(772.80302827,561.8538955)(771.35947395,559.80056392)(771.35947395,555.69390075)
\curveto(771.35947395,553.6530136)(771.73280696,552.1472371)(772.47947299,551.17657126)
\curveto(773.25102789,550.20590542)(774.3212492,549.7205725)(775.69013693,549.7205725)
\curveto(776.85991371,549.7205725)(777.89280172,549.86990571)(778.78880095,550.16857212)
\curveto(779.68480019,550.4921274)(780.55591056,550.92768258)(781.40213206,551.47523767)
\lineto(781.40213206,546.73390838)
\curveto(780.55591056,546.18635329)(779.65991132,545.80057584)(778.71413435,545.57657603)
\curveto(777.79324625,545.32768735)(776.62346946,545.20324301)(775.20480401,545.20324301)
\closepath
}
}
{
\newrgbcolor{curcolor}{0 0 0}
\pscustom[linestyle=none,fillstyle=solid,fillcolor=curcolor]
{
\newpath
\moveto(799.47146407,565.96055867)
\lineto(805.59412552,565.96055867)
\lineto(797.53013239,556.17923367)
\lineto(806.30345825,545.57657603)
\lineto(799.99413029,545.57657603)
\lineto(791.66880405,555.91790056)
\lineto(791.66880405,545.57657603)
\lineto(786.10614212,545.57657603)
\lineto(786.10614212,565.96055867)
\lineto(791.66880405,565.96055867)
\lineto(791.66880405,556.06723376)
\closepath
}
}
{
\newrgbcolor{curcolor}{0 0 0}
\pscustom[linestyle=none,fillstyle=solid,fillcolor=curcolor]
{
\newpath
\moveto(814.59139047,565.96055867)
\lineto(814.59139047,557.89656554)
\curveto(814.59139047,557.47345479)(814.5665016,556.95078857)(814.51672387,556.32856687)
\curveto(814.491835,555.70634518)(814.4545017,555.07167906)(814.40472396,554.4245685)
\curveto(814.37983509,553.77745794)(814.34250179,553.19256954)(814.29272406,552.66990332)
\curveto(814.24294632,552.17212597)(814.20561302,551.83612626)(814.18072415,551.66190418)
\lineto(823.58871614,565.96055867)
\lineto(830.27137712,565.96055867)
\lineto(830.27137712,545.57657603)
\lineto(824.8953817,545.57657603)
\lineto(824.8953817,553.71523577)
\curveto(824.8953817,554.36234633)(824.92027056,555.09656792)(824.9700483,555.91790056)
\curveto(825.01982603,556.73923319)(825.06960377,557.49834366)(825.11938151,558.19523195)
\curveto(825.19404811,558.91700911)(825.24382584,559.4645642)(825.26871471,559.83789722)
\lineto(815.89805602,545.57657603)
\lineto(809.21539505,545.57657603)
\lineto(809.21539505,565.96055867)
\closepath
}
}
{
\newrgbcolor{curcolor}{0 0 0}
\pscustom[linestyle=none,fillstyle=solid,fillcolor=curcolor]
{
\newpath
\moveto(842.89006745,565.96055867)
\lineto(848.9753956,565.96055867)
\lineto(852.82072566,554.4992351)
\curveto(853.0198366,553.92679114)(853.16916981,553.35434718)(853.26872528,552.78190323)
\curveto(853.36828075,552.20945927)(853.44294735,551.59968201)(853.49272509,550.95257145)
\lineto(853.60472499,550.95257145)
\curveto(853.6793916,551.59968201)(853.77894707,552.20945927)(853.9033914,552.78190323)
\curveto(854.02783574,553.35434718)(854.18961338,553.92679114)(854.38872432,554.4992351)
\lineto(858.15938778,565.96055867)
\lineto(864.13271603,565.96055867)
\lineto(855.50872337,542.96324492)
\curveto(854.7122796,540.84769117)(853.57983612,539.26724807)(852.11139293,538.22191563)
\curveto(850.64294974,537.15169431)(848.9380623,536.61658366)(846.99673062,536.61658366)
\curveto(846.34962006,536.61658366)(845.80206497,536.65391696)(845.35406535,536.72858356)
\curveto(844.90606573,536.7783613)(844.50784385,536.84058347)(844.1593997,536.91525007)
\lineto(844.1593997,541.32057965)
\curveto(844.40828838,541.27080192)(844.73184366,541.22102418)(845.13006554,541.17124645)
\curveto(845.52828743,541.12146871)(845.93895374,541.09657984)(846.36206449,541.09657984)
\curveto(847.53184127,541.09657984)(848.45272938,541.45746843)(849.12472881,542.17924559)
\curveto(849.79672823,542.87613388)(850.30695002,543.72235539)(850.65539417,544.71791009)
\lineto(850.99139388,545.72590924)
\closepath
}
}
{
\newrgbcolor{curcolor}{0 0 0}
\pscustom[linestyle=none,fillstyle=solid,fillcolor=curcolor]
{
\newpath
\moveto(89.87754747,513.95527058)
\curveto(89.87754747,512.8601604)(89.52910332,511.92682786)(88.83221502,511.15527296)
\curveto(88.1602156,510.38371806)(87.15221646,509.88594071)(85.8082176,509.6619409)
\lineto(85.8082176,509.51260769)
\curveto(87.22688306,509.33838562)(88.35932654,508.84060826)(89.20554804,508.01927563)
\curveto(90.07665841,507.22283186)(90.51221359,506.21483272)(90.51221359,504.9952782)
\curveto(90.51221359,503.82550142)(90.20110275,502.78016898)(89.57888106,501.85928088)
\curveto(88.98154823,500.93839277)(88.02332682,500.21661561)(86.70421684,499.69394939)
\curveto(85.38510685,499.17128316)(83.65533054,498.90995005)(81.51488792,498.90995005)
\lineto(71.80822952,498.90995005)
\lineto(71.80822952,519.2939327)
\lineto(81.51488792,519.2939327)
\curveto(83.10777545,519.2939327)(84.52644091,519.11971062)(85.7708843,518.77126647)
\curveto(87.04021655,518.44771119)(88.03577126,517.88771167)(88.75754842,517.09126791)
\curveto(89.50421445,516.31971301)(89.87754747,515.27438056)(89.87754747,513.95527058)
\closepath
\moveto(84.24021893,513.50727096)
\curveto(84.24021893,514.75171434)(83.25710866,515.37393603)(81.29088811,515.37393603)
\lineto(77.37089145,515.37393603)
\lineto(77.37089145,511.34193947)
\lineto(80.65622199,511.34193947)
\curveto(81.82599877,511.34193947)(82.70955357,511.50371711)(83.3068864,511.82727239)
\curveto(83.92910809,512.17571654)(84.24021893,512.73571606)(84.24021893,513.50727096)
\closepath
\moveto(84.76288516,505.29394462)
\curveto(84.76288516,506.09038838)(84.43932988,506.66283234)(83.79221932,507.01127649)
\curveto(83.16999762,507.3846095)(82.24910952,507.57127601)(81.029555,507.57127601)
\lineto(77.37089145,507.57127601)
\lineto(77.37089145,502.75528011)
\lineto(81.14155491,502.75528011)
\curveto(82.18688735,502.75528011)(83.04555329,502.94194662)(83.71755271,503.31527964)
\curveto(84.41444101,503.71350152)(84.76288516,504.37305651)(84.76288516,505.29394462)
\closepath
}
}
{
\newrgbcolor{curcolor}{0 0 0}
\pscustom[linestyle=none,fillstyle=solid,fillcolor=curcolor]
{
\newpath
\moveto(113.54684947,498.90995005)
\lineto(107.98418754,498.90995005)
\lineto(107.98418754,515.11260292)
\lineto(102.86952523,515.11260292)
\curveto(102.54596995,511.13038409)(102.11041477,507.91972016)(101.56285968,505.48061113)
\curveto(101.04019346,503.06639096)(100.29352743,501.29928135)(99.32286159,500.17928231)
\curveto(98.37708461,499.08417213)(97.12019679,498.53661704)(95.55219813,498.53661704)
\curveto(94.25797701,498.53661704)(93.20020013,498.73572798)(92.3788675,499.13394986)
\lineto(92.3788675,503.57661275)
\curveto(92.95131146,503.32772407)(93.54864428,503.20327973)(94.17086597,503.20327973)
\curveto(94.61886559,503.20327973)(95.02953191,503.42727954)(95.40286492,503.87527916)
\curveto(95.77619794,504.32327878)(96.12464209,505.13216698)(96.44819737,506.30194376)
\curveto(96.79664151,507.47172054)(97.10775236,509.10194137)(97.3815299,511.19260626)
\curveto(97.65530745,513.30816002)(97.90419613,516.00860216)(98.12819594,519.2939327)
\lineto(113.54684947,519.2939327)
\closepath
}
}
{
\newrgbcolor{curcolor}{0 0 0}
\pscustom[linestyle=none,fillstyle=solid,fillcolor=curcolor]
{
\newpath
\moveto(127.73351877,519.70459901)
\curveto(130.47129421,519.70459901)(132.5619591,519.10726619)(134.00551343,517.91260054)
\curveto(135.47395662,516.74282376)(136.20817822,514.93838085)(136.20817822,512.49927182)
\lineto(136.20817822,498.90995005)
\lineto(132.32551486,498.90995005)
\lineto(131.24284911,501.67261437)
\lineto(131.09351591,501.67261437)
\curveto(130.22240554,500.57750419)(129.30151743,499.78106042)(128.33085159,499.28328307)
\curveto(127.36018575,498.78550571)(126.02863133,498.53661704)(124.33618833,498.53661704)
\curveto(122.51930098,498.53661704)(121.01352449,499.05928326)(119.81885884,500.1046157)
\curveto(118.62419319,501.14994815)(118.02686036,502.78016898)(118.02686036,504.9952782)
\curveto(118.02686036,507.16060969)(118.78597083,508.75349723)(120.30419176,509.7739408)
\curveto(121.82241269,510.79438438)(124.09974408,511.36682834)(127.13618594,511.49127267)
\lineto(130.68284959,511.60327258)
\lineto(130.68284959,512.49927182)
\curveto(130.68284959,513.56949313)(130.39662761,514.35349246)(129.82418365,514.85126981)
\curveto(129.27662856,515.34904717)(128.50507366,515.59793584)(127.50951896,515.59793584)
\curveto(126.51396425,515.59793584)(125.54329841,515.44860264)(124.59752144,515.14993622)
\curveto(123.65174446,514.87615868)(122.70596749,514.52771453)(121.76019052,514.10460378)
\lineto(119.93085874,517.87526724)
\curveto(121.00108005,518.42282233)(122.20819014,518.85837751)(123.55218899,519.18193279)
\curveto(124.89618785,519.53037694)(126.28996444,519.70459901)(127.73351877,519.70459901)
\closepath
\moveto(130.68284959,508.35527534)
\lineto(128.5175181,508.28060874)
\curveto(126.72551962,508.23083101)(125.48107624,507.90727573)(124.78418794,507.3099429)
\curveto(124.08729965,506.71261008)(123.7388555,505.92861074)(123.7388555,504.9579449)
\curveto(123.7388555,504.1117234)(123.98774418,503.50194614)(124.48552153,503.12861313)
\curveto(124.98329889,502.78016898)(125.63040945,502.60594691)(126.42685321,502.60594691)
\curveto(127.62151886,502.60594691)(128.629518,502.95439105)(129.45085064,503.65127935)
\curveto(130.27218327,504.37305651)(130.68284959,505.38105565)(130.68284959,506.67527677)
\closepath
}
}
{
\newrgbcolor{curcolor}{0 0 0}
\pscustom[linestyle=none,fillstyle=solid,fillcolor=curcolor]
{
\newpath
\moveto(160.25083431,519.2939327)
\lineto(160.25083431,502.97927992)
\lineto(163.23749844,502.97927992)
\lineto(163.23749844,491.59262295)
\lineto(158.23483603,491.59262295)
\lineto(158.23483603,498.90995005)
\lineto(144.53351436,498.90995005)
\lineto(144.53351436,491.59262295)
\lineto(139.53085196,491.59262295)
\lineto(139.53085196,502.97927992)
\lineto(141.24818383,502.97927992)
\curveto(142.14418307,504.34816764)(142.90329353,505.90372188)(143.52551522,507.64594261)
\curveto(144.14773691,509.41305222)(144.64551427,511.2797173)(145.01884728,513.24593785)
\curveto(145.3921803,515.23704726)(145.66595784,517.25304555)(145.84017992,519.2939327)
\closepath
\moveto(154.68817238,515.11260292)
\lineto(150.50684261,515.11260292)
\curveto(150.2081762,512.84771596)(149.79750988,510.69482891)(149.27484366,508.65394176)
\curveto(148.75217744,506.63794347)(148.01795584,504.74638953)(147.07217887,502.97927992)
\lineto(154.68817238,502.97927992)
\closepath
}
}
{
\newrgbcolor{curcolor}{0 0 0}
\pscustom[linestyle=none,fillstyle=solid,fillcolor=curcolor]
{
\newpath
\moveto(175.10946749,519.66726571)
\curveto(177.92190953,519.66726571)(180.14946319,518.85837751)(181.79212846,517.24060111)
\curveto(183.43479373,515.64771358)(184.25612636,513.37038218)(184.25612636,510.40860693)
\lineto(184.25612636,507.72060922)
\lineto(171.11480422,507.72060922)
\curveto(171.16458196,506.15261055)(171.62502601,504.9206116)(172.49613638,504.02461236)
\curveto(173.39213561,503.12861313)(174.62413457,502.68061351)(176.19213323,502.68061351)
\curveto(177.48635435,502.68061351)(178.66857557,502.80505785)(179.73879688,503.05394652)
\curveto(180.83390706,503.32772407)(181.9539061,503.73839039)(183.09879402,504.28594548)
\lineto(183.09879402,499.9926158)
\curveto(182.07835044,499.49483844)(181.02057356,499.13394986)(179.92546338,498.90995005)
\curveto(178.83035321,498.66106138)(177.49879878,498.53661704)(175.93080012,498.53661704)
\curveto(173.88991297,498.53661704)(172.08547006,498.90995005)(170.5174714,499.65661608)
\curveto(168.94947273,500.42817098)(167.71747378,501.5730589)(166.82147454,503.09127983)
\curveto(165.92547531,504.63438962)(165.47747569,506.58816574)(165.47747569,508.95260817)
\curveto(165.47747569,511.3170506)(165.87569757,513.29571558)(166.67214134,514.88860311)
\curveto(167.49347397,516.48149065)(168.62591745,517.6761563)(170.06947178,518.47260006)
\curveto(171.5130261,519.26904383)(173.19302467,519.66726571)(175.10946749,519.66726571)
\closepath
\moveto(175.14680079,515.70993575)
\curveto(174.05169061,515.70993575)(173.15569137,515.3614916)(172.45880308,514.6646033)
\curveto(171.76191478,513.96771501)(171.35124846,512.88504926)(171.22680412,511.41660607)
\lineto(179.02946415,511.41660607)
\curveto(179.00457528,512.63616059)(178.66857557,513.65660416)(178.02146501,514.4779368)
\curveto(177.39924331,515.29926943)(176.44102191,515.70993575)(175.14680079,515.70993575)
\closepath
}
}
{
\newrgbcolor{curcolor}{0 0 0}
\pscustom[linestyle=none,fillstyle=solid,fillcolor=curcolor]
{
\newpath
\moveto(207.02942394,498.90995005)
\lineto(201.46676201,498.90995005)
\lineto(201.46676201,515.11260292)
\lineto(196.35209969,515.11260292)
\curveto(196.02854441,511.13038409)(195.59298923,507.91972016)(195.04543414,505.48061113)
\curveto(194.52276792,503.06639096)(193.77610189,501.29928135)(192.80543605,500.17928231)
\curveto(191.85965908,499.08417213)(190.60277126,498.53661704)(189.03477259,498.53661704)
\curveto(187.74055147,498.53661704)(186.68277459,498.73572798)(185.86144196,499.13394986)
\lineto(185.86144196,503.57661275)
\curveto(186.43388592,503.32772407)(187.03121874,503.20327973)(187.65344044,503.20327973)
\curveto(188.10144005,503.20327973)(188.51210637,503.42727954)(188.88543939,503.87527916)
\curveto(189.2587724,504.32327878)(189.60721655,505.13216698)(189.93077183,506.30194376)
\curveto(190.27921598,507.47172054)(190.59032682,509.10194137)(190.86410437,511.19260626)
\curveto(191.13788191,513.30816002)(191.38677059,516.00860216)(191.6107704,519.2939327)
\lineto(207.02942394,519.2939327)
\closepath
}
}
{
\newrgbcolor{curcolor}{0 0 0}
\pscustom[linestyle=none,fillstyle=solid,fillcolor=curcolor]
{
\newpath
\moveto(218.41610324,511.41660607)
\lineto(222.3360999,511.41660607)
\curveto(225.47209723,511.41660607)(227.78676193,510.91882872)(229.28009399,509.92327401)
\curveto(230.79831492,508.9277193)(231.55742539,507.42194281)(231.55742539,505.40594452)
\curveto(231.55742539,503.41483511)(230.86053709,501.83439201)(229.4667605,500.66461523)
\curveto(228.07298391,499.49483844)(225.77076365,498.90995005)(222.56009971,498.90995005)
\lineto(212.85344131,498.90995005)
\lineto(212.85344131,519.2939327)
\lineto(218.41610324,519.2939327)
\closepath
\moveto(225.99476346,505.33127792)
\curveto(225.99476346,506.82460998)(224.73787564,507.57127601)(222.2241,507.57127601)
\lineto(218.41610324,507.57127601)
\lineto(218.41610324,502.75528011)
\lineto(222.2987666,502.75528011)
\curveto(223.36898791,502.75528011)(224.25254272,502.94194662)(224.94943101,503.31527964)
\curveto(225.64631931,503.71350152)(225.99476346,504.38550095)(225.99476346,505.33127792)
\closepath
}
}
{
\newrgbcolor{curcolor}{0 0 0}
\pscustom[linestyle=none,fillstyle=solid,fillcolor=curcolor]
{
\newpath
\moveto(258.02674223,491.59262295)
\lineto(253.02407982,491.59262295)
\lineto(253.02407982,498.90995005)
\lineto(235.77609451,498.90995005)
\lineto(235.77609451,519.2939327)
\lineto(241.33875644,519.2939327)
\lineto(241.33875644,503.09127983)
\lineto(249.47741618,503.09127983)
\lineto(249.47741618,519.2939327)
\lineto(255.04007811,519.2939327)
\lineto(255.04007811,502.97927992)
\lineto(258.02674223,502.97927992)
\closepath
}
}
{
\newrgbcolor{curcolor}{0 0 0}
\pscustom[linestyle=none,fillstyle=solid,fillcolor=curcolor]
{
\newpath
\moveto(269.82405679,519.70459901)
\curveto(272.56183224,519.70459901)(274.65249712,519.10726619)(276.09605145,517.91260054)
\curveto(277.56449464,516.74282376)(278.29871624,514.93838085)(278.29871624,512.49927182)
\lineto(278.29871624,498.90995005)
\lineto(274.41605288,498.90995005)
\lineto(273.33338714,501.67261437)
\lineto(273.18405393,501.67261437)
\curveto(272.31294356,500.57750419)(271.39205546,499.78106042)(270.42138962,499.28328307)
\curveto(269.45072378,498.78550571)(268.11916935,498.53661704)(266.42672635,498.53661704)
\curveto(264.60983901,498.53661704)(263.10406251,499.05928326)(261.90939686,500.1046157)
\curveto(260.71473121,501.14994815)(260.11739839,502.78016898)(260.11739839,504.9952782)
\curveto(260.11739839,507.16060969)(260.87650885,508.75349723)(262.39472978,509.7739408)
\curveto(263.91295071,510.79438438)(266.19028211,511.36682834)(269.22672397,511.49127267)
\lineto(272.77338761,511.60327258)
\lineto(272.77338761,512.49927182)
\curveto(272.77338761,513.56949313)(272.48716563,514.35349246)(271.91472168,514.85126981)
\curveto(271.36716659,515.34904717)(270.59561169,515.59793584)(269.60005698,515.59793584)
\curveto(268.60450227,515.59793584)(267.63383643,515.44860264)(266.68805946,515.14993622)
\curveto(265.74228249,514.87615868)(264.79650552,514.52771453)(263.85072854,514.10460378)
\lineto(262.02139677,517.87526724)
\curveto(263.09161808,518.42282233)(264.29872816,518.85837751)(265.64272702,519.18193279)
\curveto(266.98672587,519.53037694)(268.38050246,519.70459901)(269.82405679,519.70459901)
\closepath
\moveto(272.77338761,508.35527534)
\lineto(270.60805612,508.28060874)
\curveto(268.81605765,508.23083101)(267.57161426,507.90727573)(266.87472597,507.3099429)
\curveto(266.17783767,506.71261008)(265.82939353,505.92861074)(265.82939353,504.9579449)
\curveto(265.82939353,504.1117234)(266.0782822,503.50194614)(266.57605956,503.12861313)
\curveto(267.07383691,502.78016898)(267.72094747,502.60594691)(268.51739124,502.60594691)
\curveto(269.71205689,502.60594691)(270.72005603,502.95439105)(271.54138866,503.65127935)
\curveto(272.3627213,504.37305651)(272.77338761,505.38105565)(272.77338761,506.67527677)
\closepath
}
}
{
\newrgbcolor{curcolor}{0 0 0}
\pscustom[linestyle=none,fillstyle=solid,fillcolor=curcolor]
{
\newpath
\moveto(307.08270064,519.2939327)
\lineto(313.2053621,519.2939327)
\lineto(305.14136896,509.51260769)
\lineto(313.91469483,498.90995005)
\lineto(307.60536686,498.90995005)
\lineto(299.28004062,509.25127458)
\lineto(299.28004062,498.90995005)
\lineto(293.71737869,498.90995005)
\lineto(293.71737869,519.2939327)
\lineto(299.28004062,519.2939327)
\lineto(299.28004062,509.40060779)
\closepath
}
}
{
\newrgbcolor{curcolor}{0 0 0}
\pscustom[linestyle=none,fillstyle=solid,fillcolor=curcolor]
{
\newpath
\moveto(334.59730612,509.13927468)
\curveto(334.59730612,505.75438867)(333.70130688,503.14105756)(331.90930841,501.29928135)
\curveto(330.1421988,499.45750514)(327.72797864,498.53661704)(324.66664791,498.53661704)
\curveto(322.77509396,498.53661704)(321.08265096,498.94728335)(319.5893189,499.76861599)
\curveto(318.1208757,500.58994862)(316.96354336,501.78461427)(316.11732186,503.35261294)
\curveto(315.27110035,504.94550047)(314.8479896,506.87438772)(314.8479896,509.13927468)
\curveto(314.8479896,512.52416068)(315.73154441,515.12504736)(317.49865401,516.9419347)
\curveto(319.26576362,518.75882204)(321.69242822,519.66726571)(324.77864781,519.66726571)
\curveto(326.69509063,519.66726571)(328.38753363,519.25659939)(329.85597682,518.43526676)
\curveto(331.32442002,517.61393413)(332.48175237,516.41926848)(333.32797387,514.85126981)
\curveto(334.17419537,513.28327115)(334.59730612,511.37927277)(334.59730612,509.13927468)
\closepath
\moveto(320.52265144,509.13927468)
\curveto(320.52265144,507.12327639)(320.84620672,505.59261103)(321.49331728,504.54727859)
\curveto(322.16531671,503.52683501)(323.24798245,503.01661322)(324.74131451,503.01661322)
\curveto(326.20975771,503.01661322)(327.26753458,503.52683501)(327.91464514,504.54727859)
\curveto(328.58664457,505.59261103)(328.92264428,507.12327639)(328.92264428,509.13927468)
\curveto(328.92264428,511.15527296)(328.58664457,512.66104946)(327.91464514,513.65660416)
\curveto(327.26753458,514.67704774)(326.19731327,515.18726953)(324.70398121,515.18726953)
\curveto(323.23553802,515.18726953)(322.16531671,514.67704774)(321.49331728,513.65660416)
\curveto(320.84620672,512.66104946)(320.52265144,511.15527296)(320.52265144,509.13927468)
\closepath
}
}
{
\newrgbcolor{curcolor}{0 0 0}
\pscustom[linestyle=none,fillstyle=solid,fillcolor=curcolor]
{
\newpath
\moveto(364.87461398,519.2939327)
\lineto(364.87461398,498.90995005)
\lineto(359.68528507,498.90995005)
\lineto(359.68528507,508.91527487)
\curveto(359.68528507,509.91082957)(359.6977295,510.88149542)(359.72261837,511.82727239)
\curveto(359.7723961,512.77304936)(359.83461827,513.64415973)(359.90928488,514.4406035)
\lineto(359.79728497,514.4406035)
\lineto(354.15995644,498.90995005)
\lineto(349.97862666,498.90995005)
\lineto(344.26663153,514.4779368)
\lineto(344.11729832,514.4779368)
\curveto(344.21685379,513.65660416)(344.27907596,512.77304936)(344.30396483,511.82727239)
\curveto(344.35374257,510.90638428)(344.37863143,509.88594071)(344.37863143,508.76594166)
\lineto(344.37863143,498.90995005)
\lineto(339.18930252,498.90995005)
\lineto(339.18930252,519.2939327)
\lineto(347.06662914,519.2939327)
\lineto(352.14395815,505.48061113)
\lineto(357.29595377,519.2939327)
\closepath
}
}
{
\newrgbcolor{curcolor}{0 0 0}
\pscustom[linestyle=none,fillstyle=solid,fillcolor=curcolor]
{
\newpath
\moveto(376.26124179,519.2939327)
\lineto(376.26124179,511.45393937)
\lineto(384.02656851,511.45393937)
\lineto(384.02656851,519.2939327)
\lineto(389.58923044,519.2939327)
\lineto(389.58923044,498.90995005)
\lineto(384.02656851,498.90995005)
\lineto(384.02656851,507.3099429)
\lineto(376.26124179,507.3099429)
\lineto(376.26124179,498.90995005)
\lineto(370.69857986,498.90995005)
\lineto(370.69857986,519.2939327)
\closepath
}
}
{
\newrgbcolor{curcolor}{0 0 0}
\pscustom[linestyle=none,fillstyle=solid,fillcolor=curcolor]
{
\newpath
\moveto(403.77593057,519.70459901)
\curveto(406.51370602,519.70459901)(408.6043709,519.10726619)(410.04792523,517.91260054)
\curveto(411.51636842,516.74282376)(412.25059002,514.93838085)(412.25059002,512.49927182)
\lineto(412.25059002,498.90995005)
\lineto(408.36792666,498.90995005)
\lineto(407.28526092,501.67261437)
\lineto(407.13592771,501.67261437)
\curveto(406.26481734,500.57750419)(405.34392924,499.78106042)(404.37326339,499.28328307)
\curveto(403.40259755,498.78550571)(402.07104313,498.53661704)(400.37860013,498.53661704)
\curveto(398.56171279,498.53661704)(397.05593629,499.05928326)(395.86127064,500.1046157)
\curveto(394.66660499,501.14994815)(394.06927217,502.78016898)(394.06927217,504.9952782)
\curveto(394.06927217,507.16060969)(394.82838263,508.75349723)(396.34660356,509.7739408)
\curveto(397.86482449,510.79438438)(400.14215589,511.36682834)(403.17859775,511.49127267)
\lineto(406.72526139,511.60327258)
\lineto(406.72526139,512.49927182)
\curveto(406.72526139,513.56949313)(406.43903941,514.35349246)(405.86659546,514.85126981)
\curveto(405.31904037,515.34904717)(404.54748547,515.59793584)(403.55193076,515.59793584)
\curveto(402.55637605,515.59793584)(401.58571021,515.44860264)(400.63993324,515.14993622)
\curveto(399.69415627,514.87615868)(398.7483793,514.52771453)(397.80260232,514.10460378)
\lineto(395.97327055,517.87526724)
\curveto(397.04349186,518.42282233)(398.25060194,518.85837751)(399.5946008,519.18193279)
\curveto(400.93859965,519.53037694)(402.33237624,519.70459901)(403.77593057,519.70459901)
\closepath
\moveto(406.72526139,508.35527534)
\lineto(404.5599299,508.28060874)
\curveto(402.76793143,508.23083101)(401.52348804,507.90727573)(400.82659975,507.3099429)
\curveto(400.12971145,506.71261008)(399.78126731,505.92861074)(399.78126731,504.9579449)
\curveto(399.78126731,504.1117234)(400.03015598,503.50194614)(400.52793334,503.12861313)
\curveto(401.02571069,502.78016898)(401.67282125,502.60594691)(402.46926502,502.60594691)
\curveto(403.66393067,502.60594691)(404.67192981,502.95439105)(405.49326244,503.65127935)
\curveto(406.31459508,504.37305651)(406.72526139,505.38105565)(406.72526139,506.67527677)
\closepath
}
}
{
\newrgbcolor{curcolor}{0 0 0}
\pscustom[linestyle=none,fillstyle=solid,fillcolor=curcolor]
{
\newpath
\moveto(434.83723973,515.11260292)
\lineto(428.15457875,515.11260292)
\lineto(428.15457875,498.90995005)
\lineto(422.59191682,498.90995005)
\lineto(422.59191682,515.11260292)
\lineto(415.90925585,515.11260292)
\lineto(415.90925585,519.2939327)
\lineto(434.83723973,519.2939327)
\closepath
}
}
{
\newrgbcolor{curcolor}{0 0 0}
\pscustom[linestyle=none,fillstyle=solid,fillcolor=curcolor]
{
\newpath
\moveto(438.60788162,498.90995005)
\lineto(438.60788162,519.2939327)
\lineto(444.17054355,519.2939327)
\lineto(444.17054355,511.41660607)
\lineto(446.85854126,511.41660607)
\curveto(449.96964972,511.41660607)(452.27186999,510.91882872)(453.76520205,509.92327401)
\curveto(455.25853411,508.9277193)(456.00520014,507.42194281)(456.00520014,505.40594452)
\curveto(456.00520014,503.41483511)(455.30831184,501.83439201)(453.91453525,500.66461523)
\curveto(452.52075866,499.49483844)(450.23098283,498.90995005)(447.04520777,498.90995005)
\closepath
\moveto(458.95453096,498.90995005)
\lineto(458.95453096,519.2939327)
\lineto(464.51719289,519.2939327)
\lineto(464.51719289,498.90995005)
\closepath
\moveto(444.17054355,502.75528011)
\lineto(446.74654136,502.75528011)
\curveto(447.84165154,502.75528011)(448.72520634,502.94194662)(449.39720577,503.31527964)
\curveto(450.09409406,503.71350152)(450.44253821,504.38550095)(450.44253821,505.33127792)
\curveto(450.44253821,506.82460998)(449.18565039,507.57127601)(446.67187475,507.57127601)
\lineto(444.17054355,507.57127601)
\closepath
}
}
{
\newrgbcolor{curcolor}{0 0 0}
\pscustom[linestyle=none,fillstyle=solid,fillcolor=curcolor]
{
\newpath
\moveto(80.76851029,266.24972483)
\curveto(84.20317403,266.24972483)(86.70450524,265.5030588)(88.2725039,264.00972674)
\curveto(89.86539143,262.54128355)(90.6618352,260.51284083)(90.6618352,257.92439859)
\curveto(90.6618352,256.35639992)(90.33827992,254.90040116)(89.69116936,253.55640231)
\curveto(89.0440588,252.21240345)(87.96139306,251.12973771)(86.44317213,250.30840507)
\curveto(84.94984006,249.48707244)(82.90895291,249.07640612)(80.32051067,249.07640612)
\lineto(77.89384607,249.07640612)
\lineto(77.89384607,239.59374753)
\lineto(72.25651754,239.59374753)
\lineto(72.25651754,266.24972483)
\closepath
\moveto(80.46984388,261.62039544)
\lineto(77.89384607,261.62039544)
\lineto(77.89384607,253.70573551)
\lineto(79.76051115,253.70573551)
\curveto(81.35339868,253.70573551)(82.6102865,254.01684636)(83.53117461,254.63906805)
\curveto(84.47695158,255.28617861)(84.94984006,256.31906662)(84.94984006,257.73773208)
\curveto(84.94984006,260.32617432)(83.456508,261.62039544)(80.46984388,261.62039544)
\closepath
}
}
{
\newrgbcolor{curcolor}{0 0 0}
\pscustom[linestyle=none,fillstyle=solid,fillcolor=curcolor]
{
\newpath
\moveto(103.61650141,260.38839649)
\curveto(106.35427686,260.38839649)(108.44494174,259.79106367)(109.88849607,258.59639802)
\curveto(111.35693926,257.42662123)(112.09116086,255.62217833)(112.09116086,253.18306929)
\lineto(112.09116086,239.59374753)
\lineto(108.2084975,239.59374753)
\lineto(107.12583176,242.35641184)
\lineto(106.97649855,242.35641184)
\curveto(106.10538818,241.26130167)(105.18450008,240.4648579)(104.21383424,239.96708055)
\curveto(103.24316839,239.46930319)(101.91161397,239.22041451)(100.21917097,239.22041451)
\curveto(98.40228363,239.22041451)(96.89650713,239.74308074)(95.70184148,240.78841318)
\curveto(94.50717583,241.83374562)(93.90984301,243.46396646)(93.90984301,245.67907568)
\curveto(93.90984301,247.84440717)(94.66895347,249.4372947)(96.1871744,250.45773828)
\curveto(97.70539533,251.47818185)(99.98272673,252.05062581)(103.01916859,252.17507015)
\lineto(106.56583223,252.28707005)
\lineto(106.56583223,253.18306929)
\curveto(106.56583223,254.2532906)(106.27961025,255.03728994)(105.7071663,255.53506729)
\curveto(105.15961121,256.03284464)(104.38805631,256.28173332)(103.3925016,256.28173332)
\curveto(102.39694689,256.28173332)(101.42628105,256.13240011)(100.48050408,255.8337337)
\curveto(99.53472711,255.55995616)(98.58895014,255.21151201)(97.64317316,254.78840126)
\lineto(95.81384139,258.55906471)
\curveto(96.8840627,259.1066198)(98.09117278,259.54217499)(99.43517164,259.86573027)
\curveto(100.77917049,260.21417442)(102.17294708,260.38839649)(103.61650141,260.38839649)
\closepath
\moveto(106.56583223,249.03907282)
\lineto(104.40050074,248.96440622)
\curveto(102.60850227,248.91462848)(101.36405888,248.5910732)(100.66717059,247.99374038)
\curveto(99.97028229,247.39640755)(99.62183815,246.61240822)(99.62183815,245.64174238)
\curveto(99.62183815,244.79552088)(99.87072682,244.18574362)(100.36850418,243.8124106)
\curveto(100.86628153,243.46396646)(101.51339209,243.28974438)(102.30983586,243.28974438)
\curveto(103.50450151,243.28974438)(104.51250065,243.63818853)(105.33383328,244.33507683)
\curveto(106.15516592,245.05685399)(106.56583223,246.06485313)(106.56583223,247.35907425)
\closepath
}
}
{
\newrgbcolor{curcolor}{0 0 0}
\pscustom[linestyle=none,fillstyle=solid,fillcolor=curcolor]
{
\newpath
\moveto(125.23248909,260.35106319)
\curveto(126.70093229,260.35106319)(128.06982001,260.15195225)(129.33915226,259.75373036)
\curveto(130.63337338,259.38039735)(131.66626139,258.79550896)(132.43781629,257.99906519)
\curveto(133.23426006,257.20262142)(133.63248194,256.18217785)(133.63248194,254.93773446)
\curveto(133.63248194,253.71817995)(133.25914892,252.74751411)(132.51248289,252.02573694)
\curveto(131.76581686,251.30395978)(130.78270659,250.78129356)(129.56315207,250.45773828)
\lineto(129.56315207,250.27107177)
\curveto(130.43426244,250.07196083)(131.21826177,249.78573885)(131.91515007,249.41240584)
\curveto(132.61203836,249.06396169)(133.15959345,248.57862877)(133.55781534,247.95640708)
\curveto(133.98092609,247.35907425)(134.19248146,246.55018605)(134.19248146,245.52974248)
\curveto(134.19248146,244.40974343)(133.83159288,243.36441099)(133.10981572,242.39374515)
\curveto(132.41292742,241.44796817)(131.31781724,240.67641327)(129.82448518,240.07908045)
\curveto(128.33115312,239.50663649)(126.43959918,239.22041451)(124.14982335,239.22041451)
\curveto(120.76493734,239.22041451)(118.15160623,239.64352527)(116.30983002,240.48974677)
\lineto(116.30983002,245.08174286)
\curveto(117.15605153,244.68352097)(118.18893953,244.32263239)(119.40849405,243.99907711)
\curveto(120.65293744,243.67552183)(121.97204742,243.51374419)(123.36582402,243.51374419)
\curveto(124.88404495,243.51374419)(126.16582163,243.68796627)(127.21115407,244.03641041)
\curveto(128.25648652,244.38485456)(128.77915274,244.99463182)(128.77915274,245.86574219)
\curveto(128.77915274,247.48351859)(126.85026549,248.29240679)(122.992491,248.29240679)
\lineto(120.82715951,248.29240679)
\lineto(120.82715951,252.13773685)
\lineto(122.8804911,252.13773685)
\curveto(124.72226731,252.13773685)(126.14093276,252.28707005)(127.13648747,252.58573647)
\curveto(128.15693105,252.88440288)(128.66715283,253.4444024)(128.66715283,254.26573504)
\curveto(128.66715283,254.9128456)(128.34359755,255.39817852)(127.69648699,255.7217338)
\curveto(127.04937643,256.07017794)(125.99159956,256.24440002)(124.52315636,256.24440002)
\curveto(123.55249052,256.24440002)(122.48226921,256.13240011)(121.31249243,255.9084003)
\curveto(120.16760452,255.6844005)(119.09738321,255.36084522)(118.1018285,254.93773446)
\lineto(116.45916323,258.82039783)
\curveto(117.62894001,259.26839744)(118.89827226,259.62928603)(120.26715999,259.90306357)
\curveto(121.66093658,260.20172998)(123.31604628,260.35106319)(125.23248909,260.35106319)
\closepath
}
}
{
\newrgbcolor{curcolor}{0 0 0}
\pscustom[linestyle=none,fillstyle=solid,fillcolor=curcolor]
{
\newpath
\moveto(149.87247935,260.35106319)
\curveto(152.16225517,260.35106319)(154.01647582,259.45506395)(155.43514128,257.66306548)
\curveto(156.85380673,255.89595587)(157.56313946,253.28262476)(157.56313946,249.82307215)
\curveto(157.56313946,246.33863068)(156.82891787,243.7004107)(155.36047467,241.90841223)
\curveto(153.89203148,240.11641375)(152.01292197,239.22041451)(149.72314614,239.22041451)
\curveto(148.25470295,239.22041451)(147.08492616,239.48174763)(146.21381579,240.00441385)
\curveto(145.34270543,240.55196894)(144.6333727,241.16174619)(144.08581761,241.83374562)
\lineto(143.78715119,241.83374562)
\curveto(143.98626214,240.78841318)(144.08581761,239.79285847)(144.08581761,238.8470815)
\lineto(144.08581761,230.63375516)
\lineto(138.52315568,230.63375516)
\lineto(138.52315568,259.97773017)
\lineto(143.04048516,259.97773017)
\lineto(143.8244845,257.32706576)
\lineto(144.08581761,257.32706576)
\curveto(144.6333727,258.1483984)(145.36759429,258.85773113)(146.2884824,259.45506395)
\curveto(147.2093705,260.05239678)(148.40403615,260.35106319)(149.87247935,260.35106319)
\closepath
\moveto(148.08048087,255.9084003)
\curveto(146.63692655,255.9084003)(145.61648297,255.44795625)(145.01915015,254.52706815)
\curveto(144.42181732,253.63106891)(144.11070647,252.27462562)(144.08581761,250.45773828)
\lineto(144.08581761,249.86040545)
\curveto(144.08581761,247.89418491)(144.37203959,246.37596398)(144.94448354,245.30574267)
\curveto(145.54181637,244.26041022)(146.61203768,243.737744)(148.15514748,243.737744)
\curveto(149.42447973,243.737744)(150.35781227,244.26041022)(150.95514509,245.30574267)
\curveto(151.57736678,246.37596398)(151.88847763,247.90662934)(151.88847763,249.89773876)
\curveto(151.88847763,253.90484646)(150.61914538,255.9084003)(148.08048087,255.9084003)
\closepath
}
}
{
\newrgbcolor{curcolor}{0 0 0}
\pscustom[linestyle=none,fillstyle=solid,fillcolor=curcolor]
{
\newpath
\moveto(170.55510653,260.35106319)
\curveto(173.36754858,260.35106319)(175.59510224,259.54217499)(177.23776751,257.92439859)
\curveto(178.88043278,256.33151106)(179.70176541,254.05417966)(179.70176541,251.09240441)
\lineto(179.70176541,248.40440669)
\lineto(166.56044327,248.40440669)
\curveto(166.610221,246.83640803)(167.07066505,245.60440908)(167.94177542,244.70840984)
\curveto(168.83777466,243.8124106)(170.06977361,243.36441099)(171.63777228,243.36441099)
\curveto(172.9319934,243.36441099)(174.11421461,243.48885532)(175.18443592,243.737744)
\curveto(176.2795461,244.01152155)(177.39954515,244.42218786)(178.54443306,244.96974295)
\lineto(178.54443306,240.67641327)
\curveto(177.52398949,240.17863592)(176.46621261,239.81774734)(175.37110243,239.59374753)
\curveto(174.27599225,239.34485885)(172.94443783,239.22041451)(171.37643917,239.22041451)
\curveto(169.33555201,239.22041451)(167.53110911,239.59374753)(165.96311044,240.34041356)
\curveto(164.39511178,241.11196846)(163.16311283,242.25685637)(162.26711359,243.7750773)
\curveto(161.37111435,245.3181871)(160.92311473,247.27196321)(160.92311473,249.63640565)
\curveto(160.92311473,252.00084808)(161.32133662,253.97951306)(162.11778038,255.57240059)
\curveto(162.93911302,257.16528812)(164.0715565,258.35995377)(165.51511082,259.15639754)
\curveto(166.95866515,259.95284131)(168.63866372,260.35106319)(170.55510653,260.35106319)
\closepath
\moveto(170.59243983,256.39373322)
\curveto(169.49732965,256.39373322)(168.60133042,256.04528908)(167.90444212,255.34840078)
\curveto(167.20755383,254.65151249)(166.79688751,253.56884674)(166.67244317,252.10040355)
\lineto(174.47510319,252.10040355)
\curveto(174.45021433,253.31995806)(174.11421461,254.34040164)(173.46710405,255.16173427)
\curveto(172.84488236,255.98306691)(171.88666095,256.39373322)(170.59243983,256.39373322)
\closepath
}
}
{
\newrgbcolor{curcolor}{0 0 0}
\pscustom[linestyle=none,fillstyle=solid,fillcolor=curcolor]
{
\newpath
\moveto(214.57105268,259.97773017)
\lineto(214.57105268,239.59374753)
\lineto(184.21907853,239.59374753)
\lineto(184.21907853,259.97773017)
\lineto(189.78174046,259.97773017)
\lineto(189.78174046,243.7750773)
\lineto(196.61373464,243.7750773)
\lineto(196.61373464,259.97773017)
\lineto(202.17639657,259.97773017)
\lineto(202.17639657,243.7750773)
\lineto(209.00839075,243.7750773)
\lineto(209.00839075,259.97773017)
\closepath
}
}
{
\newrgbcolor{curcolor}{0 0 0}
\pscustom[linestyle=none,fillstyle=solid,fillcolor=curcolor]
{
\newpath
\moveto(228.7950281,260.35106319)
\curveto(231.60747015,260.35106319)(233.83502381,259.54217499)(235.47768908,257.92439859)
\curveto(237.12035435,256.33151106)(237.94168698,254.05417966)(237.94168698,251.09240441)
\lineto(237.94168698,248.40440669)
\lineto(224.80036484,248.40440669)
\curveto(224.85014257,246.83640803)(225.31058662,245.60440908)(226.18169699,244.70840984)
\curveto(227.07769623,243.8124106)(228.30969518,243.36441099)(229.87769385,243.36441099)
\curveto(231.17191497,243.36441099)(232.35413618,243.48885532)(233.42435749,243.737744)
\curveto(234.51946767,244.01152155)(235.63946672,244.42218786)(236.78435463,244.96974295)
\lineto(236.78435463,240.67641327)
\curveto(235.76391106,240.17863592)(234.70613418,239.81774734)(233.611024,239.59374753)
\curveto(232.51591382,239.34485885)(231.1843594,239.22041451)(229.61636074,239.22041451)
\curveto(227.57547358,239.22041451)(225.77103068,239.59374753)(224.20303201,240.34041356)
\curveto(222.63503335,241.11196846)(221.4030344,242.25685637)(220.50703516,243.7750773)
\curveto(219.61103592,245.3181871)(219.1630363,247.27196321)(219.1630363,249.63640565)
\curveto(219.1630363,252.00084808)(219.56125819,253.97951306)(220.35770195,255.57240059)
\curveto(221.17903459,257.16528812)(222.31147807,258.35995377)(223.75503239,259.15639754)
\curveto(225.19858672,259.95284131)(226.87858529,260.35106319)(228.7950281,260.35106319)
\closepath
\moveto(228.8323614,256.39373322)
\curveto(227.73725122,256.39373322)(226.84125199,256.04528908)(226.14436369,255.34840078)
\curveto(225.4474754,254.65151249)(225.03680908,253.56884674)(224.91236474,252.10040355)
\lineto(232.71502476,252.10040355)
\curveto(232.6901359,253.31995806)(232.35413618,254.34040164)(231.70702562,255.16173427)
\curveto(231.08480393,255.98306691)(230.12658252,256.39373322)(228.8323614,256.39373322)
\closepath
}
}
{
\newrgbcolor{curcolor}{0 0 0}
\pscustom[linestyle=none,fillstyle=solid,fillcolor=curcolor]
{
\newpath
\moveto(248.02167729,259.97773017)
\lineto(248.02167729,252.13773685)
\lineto(255.78700401,252.13773685)
\lineto(255.78700401,259.97773017)
\lineto(261.34966594,259.97773017)
\lineto(261.34966594,239.59374753)
\lineto(255.78700401,239.59374753)
\lineto(255.78700401,247.99374038)
\lineto(248.02167729,247.99374038)
\lineto(248.02167729,239.59374753)
\lineto(242.45901536,239.59374753)
\lineto(242.45901536,259.97773017)
\closepath
}
}
{
\newrgbcolor{curcolor}{0 0 0}
\pscustom[linestyle=none,fillstyle=solid,fillcolor=curcolor]
{
\newpath
\moveto(272.54967142,259.97773017)
\lineto(272.54967142,251.91373704)
\curveto(272.54967142,251.49062629)(272.52478256,250.96796007)(272.47500482,250.34573837)
\curveto(272.45011595,249.72351668)(272.41278265,249.08885056)(272.36300492,248.44174)
\curveto(272.33811605,247.79462944)(272.30078275,247.20974104)(272.25100501,246.68707482)
\curveto(272.20122728,246.18929747)(272.16389397,245.85329776)(272.13900511,245.67907568)
\lineto(281.5469971,259.97773017)
\lineto(288.22965807,259.97773017)
\lineto(288.22965807,239.59374753)
\lineto(282.85366265,239.59374753)
\lineto(282.85366265,247.73240727)
\curveto(282.85366265,248.37951783)(282.87855152,249.11373942)(282.92832925,249.93507206)
\curveto(282.97810699,250.75640469)(283.02788472,251.51551516)(283.07766246,252.21240345)
\curveto(283.15232906,252.93418061)(283.2021068,253.4817357)(283.22699567,253.85506872)
\lineto(273.85633698,239.59374753)
\lineto(267.173676,239.59374753)
\lineto(267.173676,259.97773017)
\closepath
}
}
{
\newrgbcolor{curcolor}{0 0 0}
\pscustom[linestyle=none,fillstyle=solid,fillcolor=curcolor]
{
\newpath
\moveto(302.45362796,260.35106319)
\curveto(305.26607001,260.35106319)(307.49362366,259.54217499)(309.13628893,257.92439859)
\curveto(310.7789542,256.33151106)(311.60028683,254.05417966)(311.60028683,251.09240441)
\lineto(311.60028683,248.40440669)
\lineto(298.45896469,248.40440669)
\curveto(298.50874243,246.83640803)(298.96918648,245.60440908)(299.84029685,244.70840984)
\curveto(300.73629608,243.8124106)(301.96829504,243.36441099)(303.5362937,243.36441099)
\curveto(304.83051482,243.36441099)(306.01273604,243.48885532)(307.08295735,243.737744)
\curveto(308.17806753,244.01152155)(309.29806657,244.42218786)(310.44295449,244.96974295)
\lineto(310.44295449,240.67641327)
\curveto(309.42251091,240.17863592)(308.36473403,239.81774734)(307.26962385,239.59374753)
\curveto(306.17451368,239.34485885)(304.84295925,239.22041451)(303.27496059,239.22041451)
\curveto(301.23407344,239.22041451)(299.42963053,239.59374753)(297.86163187,240.34041356)
\curveto(296.2936332,241.11196846)(295.06163425,242.25685637)(294.16563501,243.7750773)
\curveto(293.26963578,245.3181871)(292.82163616,247.27196321)(292.82163616,249.63640565)
\curveto(292.82163616,252.00084808)(293.21985804,253.97951306)(294.01630181,255.57240059)
\curveto(294.83763444,257.16528812)(295.97007792,258.35995377)(297.41363225,259.15639754)
\curveto(298.85718657,259.95284131)(300.53718514,260.35106319)(302.45362796,260.35106319)
\closepath
\moveto(302.49096126,256.39373322)
\curveto(301.39585108,256.39373322)(300.49985184,256.04528908)(299.80296355,255.34840078)
\curveto(299.10607525,254.65151249)(298.69540893,253.56884674)(298.57096459,252.10040355)
\lineto(306.37362462,252.10040355)
\curveto(306.34873575,253.31995806)(306.01273604,254.34040164)(305.36562548,255.16173427)
\curveto(304.74340378,255.98306691)(303.78518238,256.39373322)(302.49096126,256.39373322)
\closepath
}
}
{
\newrgbcolor{curcolor}{0 0 0}
\pscustom[linestyle=none,fillstyle=solid,fillcolor=curcolor]
{
\newpath
\moveto(331.20026041,259.97773017)
\lineto(331.20026041,251.91373704)
\curveto(331.20026041,251.49062629)(331.17537155,250.96796007)(331.12559381,250.34573837)
\curveto(331.10070494,249.72351668)(331.06337164,249.08885056)(331.01359391,248.44174)
\curveto(330.98870504,247.79462944)(330.95137174,247.20974104)(330.901594,246.68707482)
\curveto(330.85181627,246.18929747)(330.81448296,245.85329776)(330.7895941,245.67907568)
\lineto(340.19758609,259.97773017)
\lineto(346.88024706,259.97773017)
\lineto(346.88024706,239.59374753)
\lineto(341.50425164,239.59374753)
\lineto(341.50425164,247.73240727)
\curveto(341.50425164,248.37951783)(341.52914051,249.11373942)(341.57891824,249.93507206)
\curveto(341.62869598,250.75640469)(341.67847371,251.51551516)(341.72825145,252.21240345)
\curveto(341.80291805,252.93418061)(341.85269579,253.4817357)(341.87758465,253.85506872)
\lineto(332.50692597,239.59374753)
\lineto(325.82426499,239.59374753)
\lineto(325.82426499,259.97773017)
\closepath
}
}
{
\newrgbcolor{curcolor}{0 0 0}
\pscustom[linestyle=none,fillstyle=solid,fillcolor=curcolor]
{
\newpath
\moveto(370.96019329,239.59374753)
\lineto(365.39753136,239.59374753)
\lineto(365.39753136,255.7964004)
\lineto(360.28286905,255.7964004)
\curveto(359.95931377,251.81418157)(359.52375859,248.60351764)(358.9762035,246.1644086)
\curveto(358.45353728,243.75018844)(357.70687125,241.98307883)(356.73620541,240.86307978)
\curveto(355.79042843,239.7679696)(354.53354061,239.22041451)(352.96554195,239.22041451)
\curveto(351.67132083,239.22041451)(350.61354395,239.41952546)(349.79221132,239.81774734)
\lineto(349.79221132,244.26041022)
\curveto(350.36465527,244.01152155)(350.9619881,243.88707721)(351.58420979,243.88707721)
\curveto(352.03220941,243.88707721)(352.44287573,244.11107702)(352.81620874,244.55907664)
\curveto(353.18954176,245.00707625)(353.53798591,245.81596445)(353.86154119,246.98574124)
\curveto(354.20998533,248.15551802)(354.52109618,249.78573885)(354.79487372,251.87640374)
\curveto(355.06865127,253.99195749)(355.31753995,256.69239964)(355.54153976,259.97773017)
\lineto(370.96019329,259.97773017)
\closepath
}
}
{
\newrgbcolor{curcolor}{0 0 0}
\pscustom[linestyle=none,fillstyle=solid,fillcolor=curcolor]
{
\newpath
\moveto(382.16022135,259.97773017)
\lineto(382.16022135,251.91373704)
\curveto(382.16022135,251.49062629)(382.13533248,250.96796007)(382.08555475,250.34573837)
\curveto(382.06066588,249.72351668)(382.02333258,249.08885056)(381.97355484,248.44174)
\curveto(381.94866598,247.79462944)(381.91133267,247.20974104)(381.86155494,246.68707482)
\curveto(381.8117772,246.18929747)(381.7744439,245.85329776)(381.74955503,245.67907568)
\lineto(391.15754702,259.97773017)
\lineto(397.840208,259.97773017)
\lineto(397.840208,239.59374753)
\lineto(392.46421258,239.59374753)
\lineto(392.46421258,247.73240727)
\curveto(392.46421258,248.37951783)(392.48910144,249.11373942)(392.53887918,249.93507206)
\curveto(392.58865692,250.75640469)(392.63843465,251.51551516)(392.68821239,252.21240345)
\curveto(392.76287899,252.93418061)(392.81265672,253.4817357)(392.83754559,253.85506872)
\lineto(383.4668869,239.59374753)
\lineto(376.78422593,239.59374753)
\lineto(376.78422593,259.97773017)
\closepath
}
}
{
\newrgbcolor{curcolor}{0 0 0}
\pscustom[linestyle=none,fillstyle=solid,fillcolor=curcolor]
{
\newpath
\moveto(431.88815238,249.82307215)
\curveto(431.88815238,246.43818615)(430.99215314,243.82485504)(429.20015467,241.98307883)
\curveto(427.43304506,240.14130262)(425.0188249,239.22041451)(421.95749417,239.22041451)
\curveto(420.06594023,239.22041451)(418.37349722,239.63108083)(416.88016516,240.45241347)
\curveto(415.41172197,241.2737461)(414.25438962,242.46841175)(413.40816812,244.03641041)
\curveto(412.56194661,245.62929795)(412.13883586,247.55818519)(412.13883586,249.82307215)
\curveto(412.13883586,253.20795816)(413.02239067,255.80884483)(414.78950027,257.62573218)
\curveto(416.55660988,259.44261952)(418.98327448,260.35106319)(422.06949407,260.35106319)
\curveto(423.98593689,260.35106319)(425.67837989,259.94039687)(427.14682308,259.11906424)
\curveto(428.61526628,258.2977316)(429.77259863,257.10306595)(430.61882013,255.53506729)
\curveto(431.46504163,253.96706862)(431.88815238,252.06307025)(431.88815238,249.82307215)
\closepath
\moveto(417.8134977,249.82307215)
\curveto(417.8134977,247.80707387)(418.13705298,246.27640851)(418.78416354,245.23107606)
\curveto(419.45616297,244.21063249)(420.53882871,243.7004107)(422.03216077,243.7004107)
\curveto(423.50060397,243.7004107)(424.55838084,244.21063249)(425.2054914,245.23107606)
\curveto(425.87749083,246.27640851)(426.21349055,247.80707387)(426.21349055,249.82307215)
\curveto(426.21349055,251.83907044)(425.87749083,253.34484693)(425.2054914,254.34040164)
\curveto(424.55838084,255.36084522)(423.48815953,255.871067)(421.99482747,255.871067)
\curveto(420.52638428,255.871067)(419.45616297,255.36084522)(418.78416354,254.34040164)
\curveto(418.13705298,253.34484693)(417.8134977,251.83907044)(417.8134977,249.82307215)
\closepath
}
}
{
\newrgbcolor{curcolor}{0 0 0}
\pscustom[linestyle=none,fillstyle=solid,fillcolor=curcolor]
{
\newpath
\moveto(452.98144903,255.7964004)
\lineto(446.29878805,255.7964004)
\lineto(446.29878805,239.59374753)
\lineto(440.73612612,239.59374753)
\lineto(440.73612612,255.7964004)
\lineto(434.05346514,255.7964004)
\lineto(434.05346514,259.97773017)
\lineto(452.98144903,259.97773017)
\closepath
}
}
{
\newrgbcolor{curcolor}{0 0 0}
\pscustom[linestyle=none,fillstyle=solid,fillcolor=curcolor]
{
\newpath
\moveto(470.11741287,259.97773017)
\lineto(476.24007432,259.97773017)
\lineto(468.17608119,250.19640517)
\lineto(476.94940705,239.59374753)
\lineto(470.64007909,239.59374753)
\lineto(462.31475285,249.93507206)
\lineto(462.31475285,239.59374753)
\lineto(456.75209092,239.59374753)
\lineto(456.75209092,259.97773017)
\lineto(462.31475285,259.97773017)
\lineto(462.31475285,250.08440526)
\closepath
}
}
{
\newrgbcolor{curcolor}{0 0 0}
\pscustom[linestyle=none,fillstyle=solid,fillcolor=curcolor]
{
\newpath
\moveto(488.22403391,260.38839649)
\curveto(490.96180936,260.38839649)(493.05247424,259.79106367)(494.49602857,258.59639802)
\curveto(495.96447177,257.42662123)(496.69869336,255.62217833)(496.69869336,253.18306929)
\lineto(496.69869336,239.59374753)
\lineto(492.81603,239.59374753)
\lineto(491.73336426,242.35641184)
\lineto(491.58403105,242.35641184)
\curveto(490.71292068,241.26130167)(489.79203258,240.4648579)(488.82136674,239.96708055)
\curveto(487.8507009,239.46930319)(486.51914647,239.22041451)(484.82670347,239.22041451)
\curveto(483.00981613,239.22041451)(481.50403963,239.74308074)(480.30937398,240.78841318)
\curveto(479.11470833,241.83374562)(478.51737551,243.46396646)(478.51737551,245.67907568)
\curveto(478.51737551,247.84440717)(479.27648597,249.4372947)(480.7947069,250.45773828)
\curveto(482.31292783,251.47818185)(484.59025923,252.05062581)(487.62670109,252.17507015)
\lineto(491.17336473,252.28707005)
\lineto(491.17336473,253.18306929)
\curveto(491.17336473,254.2532906)(490.88714276,255.03728994)(490.3146988,255.53506729)
\curveto(489.76714371,256.03284464)(488.99558881,256.28173332)(488.0000341,256.28173332)
\curveto(487.00447939,256.28173332)(486.03381355,256.13240011)(485.08803658,255.8337337)
\curveto(484.14225961,255.55995616)(483.19648264,255.21151201)(482.25070566,254.78840126)
\lineto(480.42137389,258.55906471)
\curveto(481.4915952,259.1066198)(482.69870528,259.54217499)(484.04270414,259.86573027)
\curveto(485.38670299,260.21417442)(486.78047959,260.38839649)(488.22403391,260.38839649)
\closepath
\moveto(491.17336473,249.03907282)
\lineto(489.00803324,248.96440622)
\curveto(487.21603477,248.91462848)(485.97159139,248.5910732)(485.27470309,247.99374038)
\curveto(484.57781479,247.39640755)(484.22937065,246.61240822)(484.22937065,245.64174238)
\curveto(484.22937065,244.79552088)(484.47825932,244.18574362)(484.97603668,243.8124106)
\curveto(485.47381403,243.46396646)(486.12092459,243.28974438)(486.91736836,243.28974438)
\curveto(488.11203401,243.28974438)(489.12003315,243.63818853)(489.94136578,244.33507683)
\curveto(490.76269842,245.05685399)(491.17336473,246.06485313)(491.17336473,247.35907425)
\closepath
}
}
{
\newrgbcolor{curcolor}{0 0 0}
\pscustom[linestyle=none,fillstyle=solid,fillcolor=curcolor]
{
\newpath
\moveto(509.84001778,260.35106319)
\curveto(511.30846097,260.35106319)(512.6773487,260.15195225)(513.94668095,259.75373036)
\curveto(515.24090207,259.38039735)(516.27379008,258.79550896)(517.04534498,257.99906519)
\curveto(517.84178874,257.20262142)(518.24001063,256.18217785)(518.24001063,254.93773446)
\curveto(518.24001063,253.71817995)(517.86667761,252.74751411)(517.12001158,252.02573694)
\curveto(516.37334555,251.30395978)(515.39023528,250.78129356)(514.17068076,250.45773828)
\lineto(514.17068076,250.27107177)
\curveto(515.04179113,250.07196083)(515.82579046,249.78573885)(516.52267876,249.41240584)
\curveto(517.21956705,249.06396169)(517.76712214,248.57862877)(518.16534402,247.95640708)
\curveto(518.58845477,247.35907425)(518.80001015,246.55018605)(518.80001015,245.52974248)
\curveto(518.80001015,244.40974343)(518.43912157,243.36441099)(517.71734441,242.39374515)
\curveto(517.02045611,241.44796817)(515.92534593,240.67641327)(514.43201387,240.07908045)
\curveto(512.93868181,239.50663649)(511.04712786,239.22041451)(508.75735203,239.22041451)
\curveto(505.37246603,239.22041451)(502.75913492,239.64352527)(500.91735871,240.48974677)
\lineto(500.91735871,245.08174286)
\curveto(501.76358021,244.68352097)(502.79646822,244.32263239)(504.01602274,243.99907711)
\curveto(505.26046612,243.67552183)(506.57957611,243.51374419)(507.9733527,243.51374419)
\curveto(509.49157363,243.51374419)(510.77335032,243.68796627)(511.81868276,244.03641041)
\curveto(512.8640152,244.38485456)(513.38668143,244.99463182)(513.38668143,245.86574219)
\curveto(513.38668143,247.48351859)(511.45779418,248.29240679)(507.60001969,248.29240679)
\lineto(505.4346882,248.29240679)
\lineto(505.4346882,252.13773685)
\lineto(507.48801978,252.13773685)
\curveto(509.32979599,252.13773685)(510.74846145,252.28707005)(511.74401616,252.58573647)
\curveto(512.76445973,252.88440288)(513.27468152,253.4444024)(513.27468152,254.26573504)
\curveto(513.27468152,254.9128456)(512.95112624,255.39817852)(512.30401568,255.7217338)
\curveto(511.65690512,256.07017794)(510.59912824,256.24440002)(509.13068505,256.24440002)
\curveto(508.16001921,256.24440002)(507.0897979,256.13240011)(505.92002112,255.9084003)
\curveto(504.7751332,255.6844005)(503.70491189,255.36084522)(502.70935718,254.93773446)
\lineto(501.06669192,258.82039783)
\curveto(502.2364687,259.26839744)(503.50580095,259.62928603)(504.87468867,259.90306357)
\curveto(506.26846526,260.20172998)(507.92357497,260.35106319)(509.84001778,260.35106319)
\closepath
}
}
{
\newrgbcolor{curcolor}{0 0 0}
\pscustom[linestyle=none,fillstyle=solid,fillcolor=curcolor]
{
\newpath
\moveto(529.92535951,259.97773017)
\lineto(536.01068766,259.97773017)
\lineto(539.85601772,248.5164066)
\curveto(540.05512866,247.94396264)(540.20446187,247.37151868)(540.30401734,246.79907473)
\curveto(540.40357281,246.22663077)(540.47823941,245.61685351)(540.52801715,244.96974295)
\lineto(540.64001705,244.96974295)
\curveto(540.71468366,245.61685351)(540.81423913,246.22663077)(540.93868347,246.79907473)
\curveto(541.06312781,247.37151868)(541.22490545,247.94396264)(541.42401639,248.5164066)
\lineto(545.19467984,259.97773017)
\lineto(551.16800809,259.97773017)
\lineto(542.54401543,236.98041642)
\curveto(541.74757167,234.86486267)(540.61512819,233.28441957)(539.14668499,232.23908713)
\curveto(537.6782418,231.16886581)(535.97335436,230.63375516)(534.03202268,230.63375516)
\curveto(533.38491212,230.63375516)(532.83735703,230.67108846)(532.38935741,230.74575506)
\curveto(531.94135779,230.7955328)(531.54313591,230.85775497)(531.19469176,230.93242157)
\lineto(531.19469176,235.33775115)
\curveto(531.44358044,235.28797342)(531.76713572,235.23819568)(532.1653576,235.18841795)
\curveto(532.56357949,235.13864021)(532.9742458,235.11375134)(533.39735655,235.11375134)
\curveto(534.56713334,235.11375134)(535.48802144,235.47463993)(536.16002087,236.19641709)
\curveto(536.8320203,236.89330538)(537.34224208,237.73952689)(537.69068623,238.73508159)
\lineto(538.02668595,239.74308074)
\closepath
}
}
{
\newrgbcolor{curcolor}{0 0 0}
\pscustom[linestyle=none,fillstyle=solid,fillcolor=curcolor]
{
\newpath
\moveto(558.9706648,259.97773017)
\lineto(558.9706648,252.51106986)
\curveto(558.9706648,250.74396026)(559.79199743,249.86040545)(561.4346627,249.86040545)
\curveto(562.50488401,249.86040545)(563.50043872,249.97240536)(564.42132682,250.19640517)
\curveto(565.34221493,250.44529385)(566.26310303,250.76884913)(567.18399114,251.16707101)
\lineto(567.18399114,259.97773017)
\lineto(572.74665307,259.97773017)
\lineto(572.74665307,239.59374753)
\lineto(567.18399114,239.59374753)
\lineto(567.18399114,247.69507396)
\curveto(566.31288077,247.22218548)(565.31732606,246.78663029)(564.19732701,246.38840841)
\curveto(563.07732797,246.0150754)(561.80799572,245.82840889)(560.38933026,245.82840889)
\curveto(558.2737765,245.82840889)(556.5813335,246.36351954)(555.31200125,247.43374085)
\curveto(554.04266899,248.52885103)(553.40800287,250.18396073)(553.40800287,252.39906996)
\lineto(553.40800287,259.97773017)
\closepath
}
}
{
\newrgbcolor{curcolor}{0 0 0}
\pscustom[linestyle=none,fillstyle=solid,fillcolor=curcolor]
{
\newpath
\moveto(586.93332346,260.38839649)
\curveto(589.67109891,260.38839649)(591.7617638,259.79106367)(593.20531812,258.59639802)
\curveto(594.67376132,257.42662123)(595.40798291,255.62217833)(595.40798291,253.18306929)
\lineto(595.40798291,239.59374753)
\lineto(591.52531955,239.59374753)
\lineto(590.44265381,242.35641184)
\lineto(590.2933206,242.35641184)
\curveto(589.42221023,241.26130167)(588.50132213,240.4648579)(587.53065629,239.96708055)
\curveto(586.55999045,239.46930319)(585.22843603,239.22041451)(583.53599302,239.22041451)
\curveto(581.71910568,239.22041451)(580.21332918,239.74308074)(579.01866354,240.78841318)
\curveto(577.82399789,241.83374562)(577.22666506,243.46396646)(577.22666506,245.67907568)
\curveto(577.22666506,247.84440717)(577.98577553,249.4372947)(579.50399646,250.45773828)
\curveto(581.02221738,251.47818185)(583.29954878,252.05062581)(586.33599064,252.17507015)
\lineto(589.88265428,252.28707005)
\lineto(589.88265428,253.18306929)
\curveto(589.88265428,254.2532906)(589.59643231,255.03728994)(589.02398835,255.53506729)
\curveto(588.47643326,256.03284464)(587.70487836,256.28173332)(586.70932365,256.28173332)
\curveto(585.71376895,256.28173332)(584.74310311,256.13240011)(583.79732613,255.8337337)
\curveto(582.85154916,255.55995616)(581.90577219,255.21151201)(580.95999522,254.78840126)
\lineto(579.13066344,258.55906471)
\curveto(580.20088475,259.1066198)(581.40799483,259.54217499)(582.75199369,259.86573027)
\curveto(584.09599255,260.21417442)(585.48976914,260.38839649)(586.93332346,260.38839649)
\closepath
\moveto(589.88265428,249.03907282)
\lineto(587.71732279,248.96440622)
\curveto(585.92532432,248.91462848)(584.68088094,248.5910732)(583.98399264,247.99374038)
\curveto(583.28710434,247.39640755)(582.9386602,246.61240822)(582.9386602,245.64174238)
\curveto(582.9386602,244.79552088)(583.18754887,244.18574362)(583.68532623,243.8124106)
\curveto(584.18310358,243.46396646)(584.83021414,243.28974438)(585.62665791,243.28974438)
\curveto(586.82132356,243.28974438)(587.8293227,243.63818853)(588.65065533,244.33507683)
\curveto(589.47198797,245.05685399)(589.88265428,246.06485313)(589.88265428,247.35907425)
\closepath
}
}
{
\newrgbcolor{curcolor}{0 0 0}
\pscustom[linestyle=none,fillstyle=solid,fillcolor=curcolor]
{
\newpath
\moveto(609.40797327,239.22041451)
\curveto(606.37153141,239.22041451)(604.01953341,240.05419158)(602.35197927,241.72174572)
\curveto(600.70931401,243.38929985)(599.88798137,246.03996426)(599.88798137,249.67373895)
\curveto(599.88798137,252.16262572)(600.31109212,254.19106843)(601.15731362,255.7590671)
\curveto(602.00353513,257.32706576)(603.17331191,258.48439811)(604.66664397,259.23106414)
\curveto(606.1848649,259.97773017)(607.92708564,260.35106319)(609.89330619,260.35106319)
\curveto(611.28708278,260.35106319)(612.49419286,260.21417442)(613.51463644,259.94039687)
\curveto(614.55996888,259.66661933)(615.46841255,259.34306405)(616.23996745,258.96973103)
\lineto(614.59730218,254.67640135)
\curveto(613.72619181,255.0248455)(612.90485918,255.31106748)(612.13330428,255.53506729)
\curveto(611.38663825,255.7590671)(610.63997222,255.871067)(609.89330619,255.871067)
\curveto(607.00619753,255.871067)(605.56264321,253.81773542)(605.56264321,249.71107225)
\curveto(605.56264321,247.6701851)(605.93597622,246.1644086)(606.68264225,245.19374276)
\curveto(607.45419715,244.22307692)(608.52441846,243.737744)(609.89330619,243.737744)
\curveto(611.06308297,243.737744)(612.09597098,243.88707721)(612.99197021,244.18574362)
\curveto(613.88796945,244.5092989)(614.75907982,244.94485408)(615.60530132,245.49240917)
\lineto(615.60530132,240.75107988)
\curveto(614.75907982,240.20352479)(613.86308058,239.81774734)(612.91730361,239.59374753)
\curveto(611.99641551,239.34485885)(610.82663872,239.22041451)(609.40797327,239.22041451)
\closepath
}
}
{
\newrgbcolor{curcolor}{0 0 0}
\pscustom[linestyle=none,fillstyle=solid,fillcolor=curcolor]
{
\newpath
\moveto(637.18396368,255.7964004)
\lineto(630.5013027,255.7964004)
\lineto(630.5013027,239.59374753)
\lineto(624.93864077,239.59374753)
\lineto(624.93864077,255.7964004)
\lineto(618.25597979,255.7964004)
\lineto(618.25597979,259.97773017)
\lineto(637.18396368,259.97773017)
\closepath
}
}
{
\newrgbcolor{curcolor}{0 0 0}
\pscustom[linestyle=none,fillstyle=solid,fillcolor=curcolor]
{
\newpath
\moveto(646.51723698,259.97773017)
\lineto(646.51723698,252.13773685)
\lineto(654.2825637,252.13773685)
\lineto(654.2825637,259.97773017)
\lineto(659.84522563,259.97773017)
\lineto(659.84522563,239.59374753)
\lineto(654.2825637,239.59374753)
\lineto(654.2825637,247.99374038)
\lineto(646.51723698,247.99374038)
\lineto(646.51723698,239.59374753)
\lineto(640.95457505,239.59374753)
\lineto(640.95457505,259.97773017)
\closepath
}
}
{
\newrgbcolor{curcolor}{0 0 0}
\pscustom[linestyle=none,fillstyle=solid,fillcolor=curcolor]
{
\newpath
\moveto(671.04523112,259.97773017)
\lineto(671.04523112,251.91373704)
\curveto(671.04523112,251.49062629)(671.02034225,250.96796007)(670.97056451,250.34573837)
\curveto(670.94567565,249.72351668)(670.90834234,249.08885056)(670.85856461,248.44174)
\curveto(670.83367574,247.79462944)(670.79634244,247.20974104)(670.7465647,246.68707482)
\curveto(670.69678697,246.18929747)(670.65945367,245.85329776)(670.6345648,245.67907568)
\lineto(680.04255679,259.97773017)
\lineto(686.72521776,259.97773017)
\lineto(686.72521776,239.59374753)
\lineto(681.34922234,239.59374753)
\lineto(681.34922234,247.73240727)
\curveto(681.34922234,248.37951783)(681.37411121,249.11373942)(681.42388895,249.93507206)
\curveto(681.47366668,250.75640469)(681.52344442,251.51551516)(681.57322215,252.21240345)
\curveto(681.64788875,252.93418061)(681.69766649,253.4817357)(681.72255536,253.85506872)
\lineto(672.35189667,239.59374753)
\lineto(665.66923569,239.59374753)
\lineto(665.66923569,259.97773017)
\closepath
}
}
{
\newrgbcolor{curcolor}{0 0 0}
\pscustom[linestyle=none,fillstyle=solid,fillcolor=curcolor]
{
\newpath
\moveto(705.91456253,259.97773017)
\lineto(712.03722398,259.97773017)
\lineto(703.97323085,250.19640517)
\lineto(712.74655671,239.59374753)
\lineto(706.43722875,239.59374753)
\lineto(698.11190251,249.93507206)
\lineto(698.11190251,239.59374753)
\lineto(692.54924058,239.59374753)
\lineto(692.54924058,259.97773017)
\lineto(698.11190251,259.97773017)
\lineto(698.11190251,250.08440526)
\closepath
}
}
{
\newrgbcolor{curcolor}{0 0 0}
\pscustom[linestyle=none,fillstyle=solid,fillcolor=curcolor]
{
\newpath
\moveto(712.74649599,259.97773017)
\lineto(718.83182414,259.97773017)
\lineto(722.6771542,248.5164066)
\curveto(722.87626514,247.94396264)(723.02559834,247.37151868)(723.12515382,246.79907473)
\curveto(723.22470929,246.22663077)(723.29937589,245.61685351)(723.34915362,244.96974295)
\lineto(723.46115353,244.96974295)
\curveto(723.53582013,245.61685351)(723.6353756,246.22663077)(723.75981994,246.79907473)
\curveto(723.88426428,247.37151868)(724.04604192,247.94396264)(724.24515286,248.5164066)
\lineto(728.01581632,259.97773017)
\lineto(733.98914456,259.97773017)
\lineto(725.36515191,236.98041642)
\curveto(724.56870814,234.86486267)(723.43626466,233.28441957)(721.96782147,232.23908713)
\curveto(720.49937827,231.16886581)(718.79449084,230.63375516)(716.85315916,230.63375516)
\curveto(716.2060486,230.63375516)(715.65849351,230.67108846)(715.21049389,230.74575506)
\curveto(714.76249427,230.7955328)(714.36427239,230.85775497)(714.01582824,230.93242157)
\lineto(714.01582824,235.33775115)
\curveto(714.26471692,235.28797342)(714.5882722,235.23819568)(714.98649408,235.18841795)
\curveto(715.38471596,235.13864021)(715.79538228,235.11375134)(716.21849303,235.11375134)
\curveto(717.38826981,235.11375134)(718.30915792,235.47463993)(718.98115734,236.19641709)
\curveto(719.65315677,236.89330538)(720.16337856,237.73952689)(720.51182271,238.73508159)
\lineto(720.84782242,239.74308074)
\closepath
}
}
{
\newrgbcolor{curcolor}{0 0 0}
\pscustom[linestyle=none,fillstyle=solid,fillcolor=curcolor]
{
\newpath
\moveto(752.17049626,259.97773017)
\lineto(752.17049626,252.13773685)
\lineto(759.93582298,252.13773685)
\lineto(759.93582298,259.97773017)
\lineto(765.49848491,259.97773017)
\lineto(765.49848491,239.59374753)
\lineto(759.93582298,239.59374753)
\lineto(759.93582298,247.99374038)
\lineto(752.17049626,247.99374038)
\lineto(752.17049626,239.59374753)
\lineto(746.60783433,239.59374753)
\lineto(746.60783433,259.97773017)
\closepath
}
}
{
\newrgbcolor{curcolor}{0 0 0}
\pscustom[linestyle=none,fillstyle=solid,fillcolor=curcolor]
{
\newpath
\moveto(779.68515452,260.38839649)
\curveto(782.42292996,260.38839649)(784.51359485,259.79106367)(785.95714918,258.59639802)
\curveto(787.42559237,257.42662123)(788.15981397,255.62217833)(788.15981397,253.18306929)
\lineto(788.15981397,239.59374753)
\lineto(784.27715061,239.59374753)
\lineto(783.19448486,242.35641184)
\lineto(783.04515166,242.35641184)
\curveto(782.17404129,241.26130167)(781.25315318,240.4648579)(780.28248734,239.96708055)
\curveto(779.3118215,239.46930319)(777.98026708,239.22041451)(776.28782408,239.22041451)
\curveto(774.47093673,239.22041451)(772.96516024,239.74308074)(771.77049459,240.78841318)
\curveto(770.57582894,241.83374562)(769.97849612,243.46396646)(769.97849612,245.67907568)
\curveto(769.97849612,247.84440717)(770.73760658,249.4372947)(772.25582751,250.45773828)
\curveto(773.77404844,251.47818185)(776.05137983,252.05062581)(779.08782169,252.17507015)
\lineto(782.63448534,252.28707005)
\lineto(782.63448534,253.18306929)
\curveto(782.63448534,254.2532906)(782.34826336,255.03728994)(781.7758194,255.53506729)
\curveto(781.22826431,256.03284464)(780.45670942,256.28173332)(779.46115471,256.28173332)
\curveto(778.4656,256.28173332)(777.49493416,256.13240011)(776.54915719,255.8337337)
\curveto(775.60338021,255.55995616)(774.65760324,255.21151201)(773.71182627,254.78840126)
\lineto(771.88249449,258.55906471)
\curveto(772.95271581,259.1066198)(774.15982589,259.54217499)(775.50382474,259.86573027)
\curveto(776.8478236,260.21417442)(778.24160019,260.38839649)(779.68515452,260.38839649)
\closepath
\moveto(782.63448534,249.03907282)
\lineto(780.46915385,248.96440622)
\curveto(778.67715538,248.91462848)(777.43271199,248.5910732)(776.7358237,247.99374038)
\curveto(776.0389354,247.39640755)(775.69049125,246.61240822)(775.69049125,245.64174238)
\curveto(775.69049125,244.79552088)(775.93937993,244.18574362)(776.43715728,243.8124106)
\curveto(776.93493464,243.46396646)(777.5820452,243.28974438)(778.37848896,243.28974438)
\curveto(779.57315461,243.28974438)(780.58115375,243.63818853)(781.40248639,244.33507683)
\curveto(782.22381902,245.05685399)(782.63448534,246.06485313)(782.63448534,247.35907425)
\closepath
}
}
{
\newrgbcolor{curcolor}{0 0 0}
\pscustom[linestyle=none,fillstyle=solid,fillcolor=curcolor]
{
\newpath
\moveto(90.32583549,213.3111042)
\lineto(90.32583549,192.92712155)
\lineto(84.76317356,192.92712155)
\lineto(84.76317356,209.12977442)
\lineto(77.37117985,209.12977442)
\lineto(77.37117985,192.92712155)
\lineto(71.80851792,192.92712155)
\lineto(71.80851792,213.3111042)
\closepath
}
}
{
\newrgbcolor{curcolor}{0 0 0}
\pscustom[linestyle=none,fillstyle=solid,fillcolor=curcolor]
{
\newpath
\moveto(114.66716315,203.15644618)
\curveto(114.66716315,199.77156017)(113.77116392,197.15822906)(111.97916544,195.31645285)
\curveto(110.21205584,193.47467664)(107.79783567,192.55378854)(104.73650494,192.55378854)
\curveto(102.844951,192.55378854)(101.15250799,192.96445485)(99.65917593,193.78578749)
\curveto(98.19073274,194.60712012)(97.03340039,195.80178577)(96.18717889,197.36978444)
\curveto(95.34095739,198.96267197)(94.91784664,200.89155922)(94.91784664,203.15644618)
\curveto(94.91784664,206.54133218)(95.80140144,209.14221886)(97.56851105,210.9591062)
\curveto(99.33562065,212.77599354)(101.76228525,213.68443721)(104.84850485,213.68443721)
\curveto(106.76494766,213.68443721)(108.45739066,213.27377089)(109.92583386,212.45243826)
\curveto(111.39427705,211.63110563)(112.5516094,210.43643998)(113.3978309,208.86844131)
\curveto(114.2440524,207.30044265)(114.66716315,205.39644427)(114.66716315,203.15644618)
\closepath
\moveto(100.59250847,203.15644618)
\curveto(100.59250847,201.14044789)(100.91606375,199.60978253)(101.56317431,198.56445009)
\curveto(102.23517374,197.54400651)(103.31783948,197.03378472)(104.81117155,197.03378472)
\curveto(106.27961474,197.03378472)(107.33739162,197.54400651)(107.98450218,198.56445009)
\curveto(108.6565016,199.60978253)(108.99250132,201.14044789)(108.99250132,203.15644618)
\curveto(108.99250132,205.17244446)(108.6565016,206.67822096)(107.98450218,207.67377566)
\curveto(107.33739162,208.69421924)(106.26717031,209.20444103)(104.77383824,209.20444103)
\curveto(103.30539505,209.20444103)(102.23517374,208.69421924)(101.56317431,207.67377566)
\curveto(100.91606375,206.67822096)(100.59250847,205.17244446)(100.59250847,203.15644618)
\closepath
}
}
{
\newrgbcolor{curcolor}{0 0 0}
\pscustom[linestyle=none,fillstyle=solid,fillcolor=curcolor]
{
\newpath
\moveto(137.51512493,192.92712155)
\lineto(131.952463,192.92712155)
\lineto(131.952463,209.12977442)
\lineto(126.83780069,209.12977442)
\curveto(126.51424541,205.14755559)(126.07869023,201.93689166)(125.53113514,199.49778263)
\curveto(125.00846892,197.08356246)(124.26180289,195.31645285)(123.29113705,194.19645381)
\curveto(122.34536007,193.10134363)(121.08847225,192.55378854)(119.52047359,192.55378854)
\curveto(118.22625247,192.55378854)(117.16847559,192.75289948)(116.34714296,193.15112136)
\lineto(116.34714296,197.59378425)
\curveto(116.91958692,197.34489557)(117.51691974,197.22045123)(118.13914143,197.22045123)
\curveto(118.58714105,197.22045123)(118.99780737,197.44445104)(119.37114038,197.89245066)
\curveto(119.7444734,198.34045028)(120.09291755,199.14933848)(120.41647283,200.31911526)
\curveto(120.76491697,201.48889204)(121.07602782,203.11911287)(121.34980537,205.20977776)
\curveto(121.62358291,207.32533152)(121.87247159,210.02577366)(122.0964714,213.3111042)
\lineto(137.51512493,213.3111042)
\closepath
}
}
{
\newrgbcolor{curcolor}{0 0 0}
\pscustom[linestyle=none,fillstyle=solid,fillcolor=curcolor]
{
\newpath
\moveto(140.42713716,213.3111042)
\lineto(146.51246531,213.3111042)
\lineto(150.35779537,201.84978062)
\curveto(150.55690631,201.27733667)(150.70623952,200.70489271)(150.80579499,200.13244875)
\curveto(150.90535046,199.56000479)(150.98001706,198.95022754)(151.0297948,198.30311698)
\lineto(151.1417947,198.30311698)
\curveto(151.21646131,198.95022754)(151.31601678,199.56000479)(151.44046112,200.13244875)
\curveto(151.56490545,200.70489271)(151.72668309,201.27733667)(151.92579404,201.84978062)
\lineto(155.69645749,213.3111042)
\lineto(161.66978574,213.3111042)
\lineto(153.04579308,190.31379045)
\curveto(152.24934932,188.19823669)(151.11690584,186.61779359)(149.64846264,185.57246115)
\curveto(148.18001945,184.50223984)(146.47513201,183.96712918)(144.53380033,183.96712918)
\curveto(143.88668977,183.96712918)(143.33913468,184.00446248)(142.89113506,184.07912909)
\curveto(142.44313544,184.12890682)(142.04491356,184.19112899)(141.69646941,184.2657956)
\lineto(141.69646941,188.67112518)
\curveto(141.94535809,188.62134744)(142.26891337,188.57156971)(142.66713525,188.52179197)
\curveto(143.06535714,188.47201424)(143.47602345,188.44712537)(143.8991342,188.44712537)
\curveto(145.06891099,188.44712537)(145.98979909,188.80801395)(146.66179852,189.52979111)
\curveto(147.33379795,190.22667941)(147.84401973,191.07290091)(148.19246388,192.06845562)
\lineto(148.5284636,193.07645476)
\closepath
}
}
{
\newrgbcolor{curcolor}{0 0 0}
\pscustom[linestyle=none,fillstyle=solid,fillcolor=curcolor]
{
\newpath
\moveto(169.47245771,213.3111042)
\lineto(169.47245771,205.84444389)
\curveto(169.47245771,204.07733428)(170.29379034,203.19377948)(171.93645561,203.19377948)
\curveto(173.00667692,203.19377948)(174.00223163,203.30577938)(174.92311973,203.52977919)
\curveto(175.84400784,203.77866787)(176.76489594,204.10222315)(177.68578405,204.50044503)
\lineto(177.68578405,213.3111042)
\lineto(183.24844598,213.3111042)
\lineto(183.24844598,192.92712155)
\lineto(177.68578405,192.92712155)
\lineto(177.68578405,201.02844799)
\curveto(176.81467368,200.5555595)(175.81911897,200.12000432)(174.69911992,199.72178243)
\curveto(173.57912088,199.34844942)(172.30978862,199.16178291)(170.89112316,199.16178291)
\curveto(168.77556941,199.16178291)(167.08312641,199.69689357)(165.81379415,200.76711488)
\curveto(164.5444619,201.86222506)(163.90979578,203.51733476)(163.90979578,205.73244398)
\lineto(163.90979578,213.3111042)
\closepath
}
}
{
\newrgbcolor{curcolor}{0 0 0}
\pscustom[linestyle=none,fillstyle=solid,fillcolor=curcolor]
{
\newpath
\moveto(197.47244204,213.68443721)
\curveto(200.28488409,213.68443721)(202.51243775,212.87554901)(204.15510302,211.25777261)
\curveto(205.79776829,209.66488508)(206.61910092,207.38755368)(206.61910092,204.42577843)
\lineto(206.61910092,201.73778072)
\lineto(193.47777878,201.73778072)
\curveto(193.52755651,200.16978205)(193.98800056,198.9377831)(194.85911093,198.04178386)
\curveto(195.75511017,197.14578463)(196.98710912,196.69778501)(198.55510779,196.69778501)
\curveto(199.84932891,196.69778501)(201.03155012,196.82222935)(202.10177143,197.07111802)
\curveto(203.19688161,197.34489557)(204.31688066,197.75556189)(205.46176857,198.30311698)
\lineto(205.46176857,194.0097873)
\curveto(204.441325,193.51200994)(203.38354812,193.15112136)(202.28843794,192.92712155)
\curveto(201.19332776,192.67823288)(199.86177334,192.55378854)(198.29377468,192.55378854)
\curveto(196.25288753,192.55378854)(194.44844462,192.92712155)(192.88044595,193.67378758)
\curveto(191.31244729,194.44534248)(190.08044834,195.5902304)(189.1844491,197.10845133)
\curveto(188.28844986,198.65156112)(187.84045024,200.60533724)(187.84045024,202.96977967)
\curveto(187.84045024,205.3342221)(188.23867213,207.31288708)(189.03511589,208.90577461)
\curveto(189.85644853,210.49866215)(190.98889201,211.6933278)(192.43244633,212.48977156)
\curveto(193.87600066,213.28621533)(195.55599923,213.68443721)(197.47244204,213.68443721)
\closepath
\moveto(197.50977534,209.72710725)
\curveto(196.41466517,209.72710725)(195.51866593,209.3786631)(194.82177763,208.6817748)
\curveto(194.12488934,207.98488651)(193.71422302,206.90222076)(193.58977868,205.43377757)
\lineto(201.3924387,205.43377757)
\curveto(201.36754984,206.65333209)(201.03155012,207.67377566)(200.38443956,208.4951083)
\curveto(199.76221787,209.31644093)(198.80399646,209.72710725)(197.50977534,209.72710725)
\closepath
}
}
{
\newrgbcolor{curcolor}{0 0 0}
\pscustom[linestyle=none,fillstyle=solid,fillcolor=curcolor]
{
\newpath
\moveto(216.69907597,213.3111042)
\lineto(216.69907597,205.47111087)
\lineto(224.46440269,205.47111087)
\lineto(224.46440269,213.3111042)
\lineto(230.02706462,213.3111042)
\lineto(230.02706462,192.92712155)
\lineto(224.46440269,192.92712155)
\lineto(224.46440269,201.3271144)
\lineto(216.69907597,201.3271144)
\lineto(216.69907597,192.92712155)
\lineto(211.13641404,192.92712155)
\lineto(211.13641404,213.3111042)
\closepath
}
}
{
\newrgbcolor{curcolor}{0 0 0}
\pscustom[linestyle=none,fillstyle=solid,fillcolor=curcolor]
{
\newpath
\moveto(241.22708536,213.3111042)
\lineto(241.22708536,205.24711106)
\curveto(241.22708536,204.82400031)(241.2021965,204.30133409)(241.15241876,203.6791124)
\curveto(241.12752989,203.05689071)(241.09019659,202.42222458)(241.04041886,201.77511402)
\curveto(241.01552999,201.12800346)(240.97819669,200.54311507)(240.92841895,200.02044885)
\curveto(240.87864122,199.52267149)(240.84130792,199.18667178)(240.81641905,199.0124497)
\lineto(250.22441104,213.3111042)
\lineto(256.90707201,213.3111042)
\lineto(256.90707201,192.92712155)
\lineto(251.53107659,192.92712155)
\lineto(251.53107659,201.06578129)
\curveto(251.53107659,201.71289185)(251.55596546,202.44711345)(251.60574319,203.26844608)
\curveto(251.65552093,204.08977871)(251.70529866,204.84888918)(251.7550764,205.54577748)
\curveto(251.829743,206.26755464)(251.87952074,206.81510973)(251.90440961,207.18844274)
\lineto(242.53375092,192.92712155)
\lineto(235.85108994,192.92712155)
\lineto(235.85108994,213.3111042)
\closepath
}
}
{
\newrgbcolor{curcolor}{0 0 0}
\pscustom[linestyle=none,fillstyle=solid,fillcolor=curcolor]
{
\newpath
\moveto(271.13102664,213.68443721)
\curveto(273.94346869,213.68443721)(276.17102235,212.87554901)(277.81368761,211.25777261)
\curveto(279.45635288,209.66488508)(280.27768552,207.38755368)(280.27768552,204.42577843)
\lineto(280.27768552,201.73778072)
\lineto(267.13636337,201.73778072)
\curveto(267.18614111,200.16978205)(267.64658516,198.9377831)(268.51769553,198.04178386)
\curveto(269.41369477,197.14578463)(270.64569372,196.69778501)(272.21369238,196.69778501)
\curveto(273.5079135,196.69778501)(274.69013472,196.82222935)(275.76035603,197.07111802)
\curveto(276.85546621,197.34489557)(277.97546525,197.75556189)(279.12035317,198.30311698)
\lineto(279.12035317,194.0097873)
\curveto(278.09990959,193.51200994)(277.04213271,193.15112136)(275.94702254,192.92712155)
\curveto(274.85191236,192.67823288)(273.52035794,192.55378854)(271.95235927,192.55378854)
\curveto(269.91147212,192.55378854)(268.10702921,192.92712155)(266.53903055,193.67378758)
\curveto(264.97103188,194.44534248)(263.73903293,195.5902304)(262.84303369,197.10845133)
\curveto(261.94703446,198.65156112)(261.49903484,200.60533724)(261.49903484,202.96977967)
\curveto(261.49903484,205.3342221)(261.89725672,207.31288708)(262.69370049,208.90577461)
\curveto(263.51503312,210.49866215)(264.6474766,211.6933278)(266.09103093,212.48977156)
\curveto(267.53458525,213.28621533)(269.21458382,213.68443721)(271.13102664,213.68443721)
\closepath
\moveto(271.16835994,209.72710725)
\curveto(270.07324976,209.72710725)(269.17725052,209.3786631)(268.48036223,208.6817748)
\curveto(267.78347393,207.98488651)(267.37280761,206.90222076)(267.24836328,205.43377757)
\lineto(275.0510233,205.43377757)
\curveto(275.02613443,206.65333209)(274.69013472,207.67377566)(274.04302416,208.4951083)
\curveto(273.42080247,209.31644093)(272.46258106,209.72710725)(271.16835994,209.72710725)
\closepath
}
}
{
\newrgbcolor{curcolor}{0 0 0}
\pscustom[linestyle=none,fillstyle=solid,fillcolor=curcolor]
{
\newpath
\moveto(313.01898124,213.3111042)
\lineto(313.01898124,192.92712155)
\lineto(307.45631931,192.92712155)
\lineto(307.45631931,209.12977442)
\lineto(300.0643256,209.12977442)
\lineto(300.0643256,192.92712155)
\lineto(294.50166367,192.92712155)
\lineto(294.50166367,213.3111042)
\closepath
}
}
{
\newrgbcolor{curcolor}{0 0 0}
\pscustom[linestyle=none,fillstyle=solid,fillcolor=curcolor]
{
\newpath
\moveto(330.19232645,213.68443721)
\curveto(332.48210228,213.68443721)(334.33632292,212.78843797)(335.75498838,210.9964395)
\curveto(337.17365384,209.22932989)(337.88298657,206.61599879)(337.88298657,203.15644618)
\curveto(337.88298657,199.6720047)(337.14876497,197.03378472)(335.68032178,195.24178625)
\curveto(334.21187859,193.44978777)(332.33276908,192.55378854)(330.04299325,192.55378854)
\curveto(328.57455005,192.55378854)(327.40477327,192.81512165)(326.5336629,193.33778787)
\curveto(325.66255253,193.88534296)(324.9532198,194.49512022)(324.40566471,195.16711965)
\lineto(324.1069983,195.16711965)
\curveto(324.30610924,194.1217872)(324.40566471,193.12623249)(324.40566471,192.18045552)
\lineto(324.40566471,183.96712918)
\lineto(318.84300278,183.96712918)
\lineto(318.84300278,213.3111042)
\lineto(323.36033227,213.3111042)
\lineto(324.1443316,210.66043979)
\lineto(324.40566471,210.66043979)
\curveto(324.9532198,211.48177242)(325.6874414,212.19110515)(326.6083295,212.78843797)
\curveto(327.52921761,213.3857708)(328.72388326,213.68443721)(330.19232645,213.68443721)
\closepath
\moveto(328.40032798,209.24177433)
\curveto(326.95677365,209.24177433)(325.93633008,208.78133028)(325.33899725,207.86044217)
\curveto(324.74166443,206.96444293)(324.43055358,205.60799964)(324.40566471,203.7911123)
\lineto(324.40566471,203.19377948)
\curveto(324.40566471,201.22755893)(324.69188669,199.709338)(325.26433065,198.63911669)
\curveto(325.86166347,197.59378425)(326.93188478,197.07111802)(328.47499458,197.07111802)
\curveto(329.74432683,197.07111802)(330.67765937,197.59378425)(331.2749922,198.63911669)
\curveto(331.89721389,199.709338)(332.20832474,201.24000336)(332.20832474,203.23111278)
\curveto(332.20832474,207.23822048)(330.93899248,209.24177433)(328.40032798,209.24177433)
\closepath
}
}
{
\newrgbcolor{curcolor}{0 0 0}
\pscustom[linestyle=none,fillstyle=solid,fillcolor=curcolor]
{
\newpath
\moveto(350.83762034,213.72177051)
\curveto(353.57539578,213.72177051)(355.66606067,213.12443769)(357.109615,211.92977204)
\curveto(358.57805819,210.75999526)(359.31227979,208.95555235)(359.31227979,206.51644332)
\lineto(359.31227979,192.92712155)
\lineto(355.42961643,192.92712155)
\lineto(354.34695068,195.68978587)
\lineto(354.19761748,195.68978587)
\curveto(353.32650711,194.59467569)(352.405619,193.79823192)(351.43495316,193.30045457)
\curveto(350.46428732,192.80267721)(349.1327329,192.55378854)(347.4402899,192.55378854)
\curveto(345.62340256,192.55378854)(344.11762606,193.07645476)(342.92296041,194.1217872)
\curveto(341.72829476,195.16711965)(341.13096194,196.79734048)(341.13096194,199.0124497)
\curveto(341.13096194,201.17778119)(341.8900724,202.77066873)(343.40829333,203.7911123)
\curveto(344.92651426,204.81155588)(347.20384565,205.38399984)(350.24028751,205.50844417)
\lineto(353.78695116,205.62044408)
\lineto(353.78695116,206.51644332)
\curveto(353.78695116,207.58666463)(353.50072918,208.37066396)(352.92828522,208.86844131)
\curveto(352.38073013,209.36621867)(351.60917524,209.61510734)(350.61362053,209.61510734)
\curveto(349.61806582,209.61510734)(348.64739998,209.46577414)(347.70162301,209.16710772)
\curveto(346.75584604,208.89333018)(345.81006906,208.54488603)(344.86429209,208.12177528)
\lineto(343.03496031,211.89243874)
\curveto(344.10518163,212.43999383)(345.31229171,212.87554901)(346.65629056,213.19910429)
\curveto(348.00028942,213.54754844)(349.39406601,213.72177051)(350.83762034,213.72177051)
\closepath
\moveto(353.78695116,202.37244684)
\lineto(351.62161967,202.29778024)
\curveto(349.8296212,202.24800251)(348.58517781,201.92444723)(347.88828952,201.3271144)
\curveto(347.19140122,200.72978158)(346.84295707,199.94578224)(346.84295707,198.9751164)
\curveto(346.84295707,198.1288949)(347.09184575,197.51911764)(347.5896231,197.14578463)
\curveto(348.08740046,196.79734048)(348.73451102,196.62311841)(349.53095478,196.62311841)
\curveto(350.72562043,196.62311841)(351.73361957,196.97156255)(352.55495221,197.66845085)
\curveto(353.37628484,198.39022801)(353.78695116,199.39822715)(353.78695116,200.69244827)
\closepath
}
}
{
\newrgbcolor{curcolor}{0 0 0}
\pscustom[linestyle=none,fillstyle=solid,fillcolor=curcolor]
{
\newpath
\moveto(383.09359515,207.97244208)
\curveto(383.09359515,206.8773319)(382.745151,205.94399936)(382.0482627,205.17244446)
\curveto(381.37626327,204.40088956)(380.36826413,203.90311221)(379.02426528,203.6791124)
\lineto(379.02426528,203.52977919)
\curveto(380.44293074,203.35555712)(381.57537422,202.85777976)(382.42159572,202.03644713)
\curveto(383.29270609,201.24000336)(383.72826127,200.23200422)(383.72826127,199.0124497)
\curveto(383.72826127,197.84267292)(383.41715043,196.79734048)(382.79492873,195.87645238)
\curveto(382.19759591,194.95556427)(381.2393745,194.23378711)(379.92026451,193.71112089)
\curveto(378.60115453,193.18845466)(376.87137822,192.92712155)(374.7309356,192.92712155)
\lineto(365.0242772,192.92712155)
\lineto(365.0242772,213.3111042)
\lineto(374.7309356,213.3111042)
\curveto(376.32382313,213.3111042)(377.74248859,213.13688212)(378.98693198,212.78843797)
\curveto(380.25626423,212.46488269)(381.25181894,211.90488317)(381.9735961,211.10843941)
\curveto(382.72026213,210.33688451)(383.09359515,209.29155206)(383.09359515,207.97244208)
\closepath
\moveto(377.45626661,207.52444246)
\curveto(377.45626661,208.76888584)(376.47315634,209.39110753)(374.50693579,209.39110753)
\lineto(370.58693913,209.39110753)
\lineto(370.58693913,205.35911097)
\lineto(373.87226966,205.35911097)
\curveto(375.04204645,205.35911097)(375.92560125,205.52088861)(376.52293407,205.84444389)
\curveto(377.14515577,206.19288804)(377.45626661,206.75288756)(377.45626661,207.52444246)
\closepath
\moveto(377.97893283,199.31111612)
\curveto(377.97893283,200.10755988)(377.65537755,200.68000384)(377.00826699,201.02844799)
\curveto(376.3860453,201.401781)(375.4651572,201.58844751)(374.24560268,201.58844751)
\lineto(370.58693913,201.58844751)
\lineto(370.58693913,196.77245161)
\lineto(374.35760258,196.77245161)
\curveto(375.40293503,196.77245161)(376.26160096,196.95911812)(376.93360039,197.33245114)
\curveto(377.63048869,197.73067302)(377.97893283,198.39022801)(377.97893283,199.31111612)
\closepath
}
}
{
\newrgbcolor{curcolor}{0 0 0}
\pscustom[linestyle=none,fillstyle=solid,fillcolor=curcolor]
{
\newpath
\moveto(409.5629014,213.68443721)
\curveto(411.85267723,213.68443721)(413.70689788,212.78843797)(415.12556333,210.9964395)
\curveto(416.54422879,209.22932989)(417.25356152,206.61599879)(417.25356152,203.15644618)
\curveto(417.25356152,199.6720047)(416.51933993,197.03378472)(415.05089673,195.24178625)
\curveto(413.58245354,193.44978777)(411.70334403,192.55378854)(409.4135682,192.55378854)
\curveto(407.945125,192.55378854)(406.77534822,192.81512165)(405.90423785,193.33778787)
\curveto(405.03312748,193.88534296)(404.32379475,194.49512022)(403.77623966,195.16711965)
\lineto(403.47757325,195.16711965)
\curveto(403.67668419,194.1217872)(403.77623966,193.12623249)(403.77623966,192.18045552)
\lineto(403.77623966,183.96712918)
\lineto(398.21357773,183.96712918)
\lineto(398.21357773,213.3111042)
\lineto(402.73090722,213.3111042)
\lineto(403.51490655,210.66043979)
\lineto(403.77623966,210.66043979)
\curveto(404.32379475,211.48177242)(405.05801635,212.19110515)(405.97890446,212.78843797)
\curveto(406.89979256,213.3857708)(408.09445821,213.68443721)(409.5629014,213.68443721)
\closepath
\moveto(407.77090293,209.24177433)
\curveto(406.3273486,209.24177433)(405.30690503,208.78133028)(404.7095722,207.86044217)
\curveto(404.11223938,206.96444293)(403.80112853,205.60799964)(403.77623966,203.7911123)
\lineto(403.77623966,203.19377948)
\curveto(403.77623966,201.22755893)(404.06246164,199.709338)(404.6349056,198.63911669)
\curveto(405.23223843,197.59378425)(406.30245974,197.07111802)(407.84556953,197.07111802)
\curveto(409.11490179,197.07111802)(410.04823432,197.59378425)(410.64556715,198.63911669)
\curveto(411.26778884,199.709338)(411.57889969,201.24000336)(411.57889969,203.23111278)
\curveto(411.57889969,207.23822048)(410.30956744,209.24177433)(407.77090293,209.24177433)
\closepath
}
}
{
\newrgbcolor{curcolor}{0 0 0}
\pscustom[linestyle=none,fillstyle=solid,fillcolor=curcolor]
{
\newpath
\moveto(430.24552859,213.68443721)
\curveto(433.05797064,213.68443721)(435.2855243,212.87554901)(436.92818957,211.25777261)
\curveto(438.57085483,209.66488508)(439.39218747,207.38755368)(439.39218747,204.42577843)
\lineto(439.39218747,201.73778072)
\lineto(426.25086532,201.73778072)
\curveto(426.30064306,200.16978205)(426.76108711,198.9377831)(427.63219748,198.04178386)
\curveto(428.52819672,197.14578463)(429.76019567,196.69778501)(431.32819433,196.69778501)
\curveto(432.62241546,196.69778501)(433.80463667,196.82222935)(434.87485798,197.07111802)
\curveto(435.96996816,197.34489557)(437.08996721,197.75556189)(438.23485512,198.30311698)
\lineto(438.23485512,194.0097873)
\curveto(437.21441155,193.51200994)(436.15663467,193.15112136)(435.06152449,192.92712155)
\curveto(433.96641431,192.67823288)(432.63485989,192.55378854)(431.06686122,192.55378854)
\curveto(429.02597407,192.55378854)(427.22153117,192.92712155)(425.6535325,193.67378758)
\curveto(424.08553384,194.44534248)(422.85353488,195.5902304)(421.95753565,197.10845133)
\curveto(421.06153641,198.65156112)(420.61353679,200.60533724)(420.61353679,202.96977967)
\curveto(420.61353679,205.3342221)(421.01175867,207.31288708)(421.80820244,208.90577461)
\curveto(422.62953508,210.49866215)(423.76197856,211.6933278)(425.20553288,212.48977156)
\curveto(426.64908721,213.28621533)(428.32908578,213.68443721)(430.24552859,213.68443721)
\closepath
\moveto(430.28286189,209.72710725)
\curveto(429.18775171,209.72710725)(428.29175248,209.3786631)(427.59486418,208.6817748)
\curveto(426.89797589,207.98488651)(426.48730957,206.90222076)(426.36286523,205.43377757)
\lineto(434.16552525,205.43377757)
\curveto(434.14063638,206.65333209)(433.80463667,207.67377566)(433.15752611,208.4951083)
\curveto(432.53530442,209.31644093)(431.57708301,209.72710725)(430.28286189,209.72710725)
\closepath
}
}
{
\newrgbcolor{curcolor}{0 0 0}
\pscustom[linestyle=none,fillstyle=solid,fillcolor=curcolor]
{
\newpath
\moveto(462.24015164,213.3111042)
\lineto(462.24015164,196.99645142)
\lineto(465.22681577,196.99645142)
\lineto(465.22681577,185.60979445)
\lineto(460.22415336,185.60979445)
\lineto(460.22415336,192.92712155)
\lineto(446.52283169,192.92712155)
\lineto(446.52283169,185.60979445)
\lineto(441.52016929,185.60979445)
\lineto(441.52016929,196.99645142)
\lineto(443.23750116,196.99645142)
\curveto(444.1335004,198.36533914)(444.89261086,199.92089338)(445.51483255,201.66311411)
\curveto(446.13705425,203.43022372)(446.6348316,205.2968888)(447.00816461,207.26310935)
\curveto(447.38149763,209.25421876)(447.65527517,211.27021705)(447.82949725,213.3111042)
\closepath
\moveto(456.67748971,209.12977442)
\lineto(452.49615994,209.12977442)
\curveto(452.19749353,206.86488746)(451.78682721,204.71200041)(451.26416099,202.67111326)
\curveto(450.74149477,200.65511497)(450.00727317,198.76356103)(449.0614962,196.99645142)
\lineto(456.67748971,196.99645142)
\closepath
}
}
{
\newrgbcolor{curcolor}{0 0 0}
\pscustom[linestyle=none,fillstyle=solid,fillcolor=curcolor]
{
\newpath
\moveto(477.06146677,213.72177051)
\curveto(479.79924222,213.72177051)(481.88990711,213.12443769)(483.33346143,211.92977204)
\curveto(484.80190463,210.75999526)(485.53612622,208.95555235)(485.53612622,206.51644332)
\lineto(485.53612622,192.92712155)
\lineto(481.65346286,192.92712155)
\lineto(480.57079712,195.68978587)
\lineto(480.42146391,195.68978587)
\curveto(479.55035354,194.59467569)(478.62946544,193.79823192)(477.6587996,193.30045457)
\curveto(476.68813376,192.80267721)(475.35657934,192.55378854)(473.66413633,192.55378854)
\curveto(471.84724899,192.55378854)(470.3414725,193.07645476)(469.14680685,194.1217872)
\curveto(467.9521412,195.16711965)(467.35480837,196.79734048)(467.35480837,199.0124497)
\curveto(467.35480837,201.17778119)(468.11391884,202.77066873)(469.63213977,203.7911123)
\curveto(471.1503607,204.81155588)(473.42769209,205.38399984)(476.46413395,205.50844417)
\lineto(480.01079759,205.62044408)
\lineto(480.01079759,206.51644332)
\curveto(480.01079759,207.58666463)(479.72457562,208.37066396)(479.15213166,208.86844131)
\curveto(478.60457657,209.36621867)(477.83302167,209.61510734)(476.83746696,209.61510734)
\curveto(475.84191226,209.61510734)(474.87124642,209.46577414)(473.92546944,209.16710772)
\curveto(472.97969247,208.89333018)(472.0339155,208.54488603)(471.08813853,208.12177528)
\lineto(469.25880675,211.89243874)
\curveto(470.32902806,212.43999383)(471.53613814,212.87554901)(472.880137,213.19910429)
\curveto(474.22413586,213.54754844)(475.61791245,213.72177051)(477.06146677,213.72177051)
\closepath
\moveto(480.01079759,202.37244684)
\lineto(477.84546611,202.29778024)
\curveto(476.05346763,202.24800251)(474.80902425,201.92444723)(474.11213595,201.3271144)
\curveto(473.41524766,200.72978158)(473.06680351,199.94578224)(473.06680351,198.9751164)
\curveto(473.06680351,198.1288949)(473.31569218,197.51911764)(473.81346954,197.14578463)
\curveto(474.31124689,196.79734048)(474.95835745,196.62311841)(475.75480122,196.62311841)
\curveto(476.94946687,196.62311841)(477.95746601,196.97156255)(478.77879864,197.66845085)
\curveto(479.60013128,198.39022801)(480.01079759,199.39822715)(480.01079759,200.69244827)
\closepath
}
}
{
\newrgbcolor{curcolor}{0 0 0}
\pscustom[linestyle=none,fillstyle=solid,fillcolor=curcolor]
{
\newpath
\moveto(504.61344559,213.3111042)
\lineto(510.73610704,213.3111042)
\lineto(502.67211391,203.52977919)
\lineto(511.44543977,192.92712155)
\lineto(505.13611181,192.92712155)
\lineto(496.81078556,203.26844608)
\lineto(496.81078556,192.92712155)
\lineto(491.24812363,192.92712155)
\lineto(491.24812363,213.3111042)
\lineto(496.81078556,213.3111042)
\lineto(496.81078556,203.41777929)
\closepath
}
}
{
\newrgbcolor{curcolor}{0 0 0}
\pscustom[linestyle=none,fillstyle=solid,fillcolor=curcolor]
{
\newpath
\moveto(531.23205938,209.12977442)
\lineto(524.5493984,209.12977442)
\lineto(524.5493984,192.92712155)
\lineto(518.98673647,192.92712155)
\lineto(518.98673647,209.12977442)
\lineto(512.3040755,209.12977442)
\lineto(512.3040755,213.3111042)
\lineto(531.23205938,213.3111042)
\closepath
}
}
{
\newrgbcolor{curcolor}{0 0 0}
\pscustom[linestyle=none,fillstyle=solid,fillcolor=curcolor]
{
\newpath
\moveto(540.37869669,213.3111042)
\lineto(540.37869669,205.24711106)
\curveto(540.37869669,204.82400031)(540.35380782,204.30133409)(540.30403009,203.6791124)
\curveto(540.27914122,203.05689071)(540.24180792,202.42222458)(540.19203018,201.77511402)
\curveto(540.16714132,201.12800346)(540.12980802,200.54311507)(540.08003028,200.02044885)
\curveto(540.03025254,199.52267149)(539.99291924,199.18667178)(539.96803038,199.0124497)
\lineto(549.37602236,213.3111042)
\lineto(556.05868334,213.3111042)
\lineto(556.05868334,192.92712155)
\lineto(550.68268792,192.92712155)
\lineto(550.68268792,201.06578129)
\curveto(550.68268792,201.71289185)(550.70757679,202.44711345)(550.75735452,203.26844608)
\curveto(550.80713226,204.08977871)(550.85690999,204.84888918)(550.90668773,205.54577748)
\curveto(550.98135433,206.26755464)(551.03113207,206.81510973)(551.05602093,207.18844274)
\lineto(541.68536225,192.92712155)
\lineto(535.00270127,192.92712155)
\lineto(535.00270127,213.3111042)
\closepath
}
}
{
\newrgbcolor{curcolor}{0 0 0}
\pscustom[linestyle=none,fillstyle=solid,fillcolor=curcolor]
{
\newpath
\moveto(573.23196879,213.68443721)
\curveto(575.52174462,213.68443721)(577.37596526,212.78843797)(578.79463072,210.9964395)
\curveto(580.21329618,209.22932989)(580.92262891,206.61599879)(580.92262891,203.15644618)
\curveto(580.92262891,199.6720047)(580.18840731,197.03378472)(578.71996411,195.24178625)
\curveto(577.25152092,193.44978777)(575.37241141,192.55378854)(573.08263558,192.55378854)
\curveto(571.61419239,192.55378854)(570.44441561,192.81512165)(569.57330524,193.33778787)
\curveto(568.70219487,193.88534296)(567.99286214,194.49512022)(567.44530705,195.16711965)
\lineto(567.14664064,195.16711965)
\curveto(567.34575158,194.1217872)(567.44530705,193.12623249)(567.44530705,192.18045552)
\lineto(567.44530705,183.96712918)
\lineto(561.88264512,183.96712918)
\lineto(561.88264512,213.3111042)
\lineto(566.3999746,213.3111042)
\lineto(567.18397394,210.66043979)
\lineto(567.44530705,210.66043979)
\curveto(567.99286214,211.48177242)(568.72708373,212.19110515)(569.64797184,212.78843797)
\curveto(570.56885994,213.3857708)(571.76352559,213.68443721)(573.23196879,213.68443721)
\closepath
\moveto(571.43997031,209.24177433)
\curveto(569.99641599,209.24177433)(568.97597241,208.78133028)(568.37863959,207.86044217)
\curveto(567.78130676,206.96444293)(567.47019592,205.60799964)(567.44530705,203.7911123)
\lineto(567.44530705,203.19377948)
\curveto(567.44530705,201.22755893)(567.73152903,199.709338)(568.30397298,198.63911669)
\curveto(568.90130581,197.59378425)(569.97152712,197.07111802)(571.51463692,197.07111802)
\curveto(572.78396917,197.07111802)(573.71730171,197.59378425)(574.31463453,198.63911669)
\curveto(574.93685622,199.709338)(575.24796707,201.24000336)(575.24796707,203.23111278)
\curveto(575.24796707,207.23822048)(573.97863482,209.24177433)(571.43997031,209.24177433)
\closepath
}
}
{
\newrgbcolor{curcolor}{0 0 0}
\pscustom[linestyle=none,fillstyle=solid,fillcolor=curcolor]
{
\newpath
\moveto(604.03195121,203.15644618)
\curveto(604.03195121,199.77156017)(603.13595197,197.15822906)(601.3439535,195.31645285)
\curveto(599.57684389,193.47467664)(597.16262372,192.55378854)(594.101293,192.55378854)
\curveto(592.20973905,192.55378854)(590.51729605,192.96445485)(589.02396399,193.78578749)
\curveto(587.55552079,194.60712012)(586.39818845,195.80178577)(585.55196694,197.36978444)
\curveto(584.70574544,198.96267197)(584.28263469,200.89155922)(584.28263469,203.15644618)
\curveto(584.28263469,206.54133218)(585.1661895,209.14221886)(586.9332991,210.9591062)
\curveto(588.70040871,212.77599354)(591.12707331,213.68443721)(594.2132929,213.68443721)
\curveto(596.12973572,213.68443721)(597.82217872,213.27377089)(599.29062191,212.45243826)
\curveto(600.75906511,211.63110563)(601.91639745,210.43643998)(602.76261896,208.86844131)
\curveto(603.60884046,207.30044265)(604.03195121,205.39644427)(604.03195121,203.15644618)
\closepath
\moveto(589.95729653,203.15644618)
\curveto(589.95729653,201.14044789)(590.28085181,199.60978253)(590.92796237,198.56445009)
\curveto(591.59996179,197.54400651)(592.68262754,197.03378472)(594.1759596,197.03378472)
\curveto(595.6444028,197.03378472)(596.70217967,197.54400651)(597.34929023,198.56445009)
\curveto(598.02128966,199.60978253)(598.35728937,201.14044789)(598.35728937,203.15644618)
\curveto(598.35728937,205.17244446)(598.02128966,206.67822096)(597.34929023,207.67377566)
\curveto(596.70217967,208.69421924)(595.63195836,209.20444103)(594.1386263,209.20444103)
\curveto(592.67018311,209.20444103)(591.59996179,208.69421924)(590.92796237,207.67377566)
\curveto(590.28085181,206.67822096)(589.95729653,205.17244446)(589.95729653,203.15644618)
\closepath
}
}
{
\newrgbcolor{curcolor}{0 0 0}
\pscustom[linestyle=none,fillstyle=solid,fillcolor=curcolor]
{
\newpath
\moveto(626.69320452,207.97244208)
\curveto(626.69320452,206.8773319)(626.34476037,205.94399936)(625.64787208,205.17244446)
\curveto(624.97587265,204.40088956)(623.96787351,203.90311221)(622.62387465,203.6791124)
\lineto(622.62387465,203.52977919)
\curveto(624.04254011,203.35555712)(625.17498359,202.85777976)(626.02120509,202.03644713)
\curveto(626.89231546,201.24000336)(627.32787065,200.23200422)(627.32787065,199.0124497)
\curveto(627.32787065,197.84267292)(627.0167598,196.79734048)(626.39453811,195.87645238)
\curveto(625.79720528,194.95556427)(624.83898388,194.23378711)(623.51987389,193.71112089)
\curveto(622.2007639,193.18845466)(620.4709876,192.92712155)(618.33054497,192.92712155)
\lineto(608.62388657,192.92712155)
\lineto(608.62388657,213.3111042)
\lineto(618.33054497,213.3111042)
\curveto(619.92343251,213.3111042)(621.34209797,213.13688212)(622.58654135,212.78843797)
\curveto(623.8558736,212.46488269)(624.85142831,211.90488317)(625.57320547,211.10843941)
\curveto(626.3198715,210.33688451)(626.69320452,209.29155206)(626.69320452,207.97244208)
\closepath
\moveto(621.05587599,207.52444246)
\curveto(621.05587599,208.76888584)(620.07276571,209.39110753)(618.10654517,209.39110753)
\lineto(614.1865485,209.39110753)
\lineto(614.1865485,205.35911097)
\lineto(617.47187904,205.35911097)
\curveto(618.64165582,205.35911097)(619.52521062,205.52088861)(620.12254345,205.84444389)
\curveto(620.74476514,206.19288804)(621.05587599,206.75288756)(621.05587599,207.52444246)
\closepath
\moveto(621.57854221,199.31111612)
\curveto(621.57854221,200.10755988)(621.25498693,200.68000384)(620.60787637,201.02844799)
\curveto(619.98565468,201.401781)(619.06476657,201.58844751)(617.84521205,201.58844751)
\lineto(614.1865485,201.58844751)
\lineto(614.1865485,196.77245161)
\lineto(617.95721196,196.77245161)
\curveto(619.0025444,196.77245161)(619.86121034,196.95911812)(620.53320977,197.33245114)
\curveto(621.23009806,197.73067302)(621.57854221,198.39022801)(621.57854221,199.31111612)
\closepath
}
}
{
\newrgbcolor{curcolor}{0 0 0}
\pscustom[linestyle=none,fillstyle=solid,fillcolor=curcolor]
{
\newpath
\moveto(640.46921213,213.72177051)
\curveto(643.20698758,213.72177051)(645.29765247,213.12443769)(646.74120679,211.92977204)
\curveto(648.20964999,210.75999526)(648.94387158,208.95555235)(648.94387158,206.51644332)
\lineto(648.94387158,192.92712155)
\lineto(645.06120822,192.92712155)
\lineto(643.97854248,195.68978587)
\lineto(643.82920927,195.68978587)
\curveto(642.9580989,194.59467569)(642.0372108,193.79823192)(641.06654496,193.30045457)
\curveto(640.09587912,192.80267721)(638.7643247,192.55378854)(637.07188169,192.55378854)
\curveto(635.25499435,192.55378854)(633.74921786,193.07645476)(632.55455221,194.1217872)
\curveto(631.35988656,195.16711965)(630.76255373,196.79734048)(630.76255373,199.0124497)
\curveto(630.76255373,201.17778119)(631.5216642,202.77066873)(633.03988513,203.7911123)
\curveto(634.55810606,204.81155588)(636.83543745,205.38399984)(639.87187931,205.50844417)
\lineto(643.41854296,205.62044408)
\lineto(643.41854296,206.51644332)
\curveto(643.41854296,207.58666463)(643.13232098,208.37066396)(642.55987702,208.86844131)
\curveto(642.01232193,209.36621867)(641.24076703,209.61510734)(640.24521233,209.61510734)
\curveto(639.24965762,209.61510734)(638.27899178,209.46577414)(637.3332148,209.16710772)
\curveto(636.38743783,208.89333018)(635.44166086,208.54488603)(634.49588389,208.12177528)
\lineto(632.66655211,211.89243874)
\curveto(633.73677342,212.43999383)(634.94388351,212.87554901)(636.28788236,213.19910429)
\curveto(637.63188122,213.54754844)(639.02565781,213.72177051)(640.46921213,213.72177051)
\closepath
\moveto(643.41854296,202.37244684)
\lineto(641.25321147,202.29778024)
\curveto(639.46121299,202.24800251)(638.21676961,201.92444723)(637.51988131,201.3271144)
\curveto(636.82299302,200.72978158)(636.47454887,199.94578224)(636.47454887,198.9751164)
\curveto(636.47454887,198.1288949)(636.72343755,197.51911764)(637.2212149,197.14578463)
\curveto(637.71899225,196.79734048)(638.36610281,196.62311841)(639.16254658,196.62311841)
\curveto(640.35721223,196.62311841)(641.36521137,196.97156255)(642.18654401,197.66845085)
\curveto(643.00787664,198.39022801)(643.41854296,199.39822715)(643.41854296,200.69244827)
\closepath
}
}
{
\newrgbcolor{curcolor}{0 0 0}
\pscustom[linestyle=none,fillstyle=solid,fillcolor=curcolor]
{
\newpath
\moveto(660.21853092,213.3111042)
\lineto(660.21853092,205.47111087)
\lineto(667.98385765,205.47111087)
\lineto(667.98385765,213.3111042)
\lineto(673.54651958,213.3111042)
\lineto(673.54651958,192.92712155)
\lineto(667.98385765,192.92712155)
\lineto(667.98385765,201.3271144)
\lineto(660.21853092,201.3271144)
\lineto(660.21853092,192.92712155)
\lineto(654.65586899,192.92712155)
\lineto(654.65586899,213.3111042)
\closepath
}
}
{
\newrgbcolor{curcolor}{0 0 0}
\pscustom[linestyle=none,fillstyle=solid,fillcolor=curcolor]
{
\newpath
\moveto(684.74652506,213.3111042)
\lineto(684.74652506,205.24711106)
\curveto(684.74652506,204.82400031)(684.72163619,204.30133409)(684.67185846,203.6791124)
\curveto(684.64696959,203.05689071)(684.60963629,202.42222458)(684.55985855,201.77511402)
\curveto(684.53496969,201.12800346)(684.49763638,200.54311507)(684.44785865,200.02044885)
\curveto(684.39808091,199.52267149)(684.36074761,199.18667178)(684.33585874,199.0124497)
\lineto(693.74385073,213.3111042)
\lineto(700.42651171,213.3111042)
\lineto(700.42651171,192.92712155)
\lineto(695.05051629,192.92712155)
\lineto(695.05051629,201.06578129)
\curveto(695.05051629,201.71289185)(695.07540516,202.44711345)(695.12518289,203.26844608)
\curveto(695.17496063,204.08977871)(695.22473836,204.84888918)(695.2745161,205.54577748)
\curveto(695.3491827,206.26755464)(695.39896044,206.81510973)(695.4238493,207.18844274)
\lineto(686.05319062,192.92712155)
\lineto(679.37052964,192.92712155)
\lineto(679.37052964,213.3111042)
\closepath
}
}
{
\newrgbcolor{curcolor}{0 0 0}
\pscustom[linestyle=none,fillstyle=solid,fillcolor=curcolor]
{
\newpath
\moveto(709.34913752,192.92712155)
\lineto(703.33847597,192.92712155)
\lineto(708.82647129,200.99111469)
\curveto(707.78113885,201.41422544)(706.84780631,202.0986693)(706.02647368,203.04444627)
\curveto(705.23002991,204.01511211)(704.83180803,205.3342221)(704.83180803,207.00177624)
\curveto(704.83180803,209.04266339)(705.60336293,210.59821762)(707.14647272,211.66843893)
\curveto(708.68958252,212.76354911)(710.6682475,213.3111042)(713.08246767,213.3111042)
\lineto(722.56512626,213.3111042)
\lineto(722.56512626,192.92712155)
\lineto(717.00246433,192.92712155)
\lineto(717.00246433,200.50578177)
\lineto(713.9411336,200.50578177)
\closepath
\moveto(710.28247005,206.96444293)
\curveto(710.28247005,206.11822143)(710.61846977,205.446222)(711.2904692,204.94844465)
\curveto(711.96246862,204.47555616)(712.83357899,204.23911192)(713.9038003,204.23911192)
\lineto(717.00246433,204.23911192)
\lineto(717.00246433,209.39110753)
\lineto(713.19446757,209.39110753)
\curveto(712.19891287,209.39110753)(711.46469127,209.14221886)(710.99180278,208.6444415)
\curveto(710.5189143,208.17155302)(710.28247005,207.61155349)(710.28247005,206.96444293)
\closepath
}
}
{
\newrgbcolor{curcolor}{0 0 0}
\pscustom[linestyle=none,fillstyle=solid,fillcolor=curcolor]
{
\newpath
\moveto(756.42646146,213.3111042)
\lineto(756.42646146,196.99645142)
\lineto(759.41312558,196.99645142)
\lineto(759.41312558,185.60979445)
\lineto(754.41046318,185.60979445)
\lineto(754.41046318,192.92712155)
\lineto(740.70914151,192.92712155)
\lineto(740.70914151,185.60979445)
\lineto(735.7064791,185.60979445)
\lineto(735.7064791,196.99645142)
\lineto(737.42381097,196.99645142)
\curveto(738.31981021,198.36533914)(739.07892067,199.92089338)(739.70114237,201.66311411)
\curveto(740.32336406,203.43022372)(740.82114141,205.2968888)(741.19447443,207.26310935)
\curveto(741.56780744,209.25421876)(741.84158499,211.27021705)(742.01580706,213.3111042)
\closepath
\moveto(750.86379953,209.12977442)
\lineto(746.68246976,209.12977442)
\curveto(746.38380334,206.86488746)(745.97313703,204.71200041)(745.45047081,202.67111326)
\curveto(744.92780458,200.65511497)(744.19358299,198.76356103)(743.24780601,196.99645142)
\lineto(750.86379953,196.99645142)
\closepath
}
}
{
\newrgbcolor{curcolor}{0 0 0}
\pscustom[linestyle=none,fillstyle=solid,fillcolor=curcolor]
{
\newpath
\moveto(781.40243461,203.15644618)
\curveto(781.40243461,199.77156017)(780.50643537,197.15822906)(778.7144369,195.31645285)
\curveto(776.94732729,193.47467664)(774.53310712,192.55378854)(771.4717764,192.55378854)
\curveto(769.58022245,192.55378854)(767.88777945,192.96445485)(766.39444739,193.78578749)
\curveto(764.92600419,194.60712012)(763.76867184,195.80178577)(762.92245034,197.36978444)
\curveto(762.07622884,198.96267197)(761.65311809,200.89155922)(761.65311809,203.15644618)
\curveto(761.65311809,206.54133218)(762.53667289,209.14221886)(764.3037825,210.9591062)
\curveto(766.07089211,212.77599354)(768.49755671,213.68443721)(771.5837763,213.68443721)
\curveto(773.50021911,213.68443721)(775.19266212,213.27377089)(776.66110531,212.45243826)
\curveto(778.12954851,211.63110563)(779.28688085,210.43643998)(780.13310235,208.86844131)
\curveto(780.97932386,207.30044265)(781.40243461,205.39644427)(781.40243461,203.15644618)
\closepath
\moveto(767.32777993,203.15644618)
\curveto(767.32777993,201.14044789)(767.65133521,199.60978253)(768.29844577,198.56445009)
\curveto(768.97044519,197.54400651)(770.05311094,197.03378472)(771.546443,197.03378472)
\curveto(773.01488619,197.03378472)(774.07266307,197.54400651)(774.71977363,198.56445009)
\curveto(775.39177306,199.60978253)(775.72777277,201.14044789)(775.72777277,203.15644618)
\curveto(775.72777277,205.17244446)(775.39177306,206.67822096)(774.71977363,207.67377566)
\curveto(774.07266307,208.69421924)(773.00244176,209.20444103)(771.5091097,209.20444103)
\curveto(770.0406665,209.20444103)(768.97044519,208.69421924)(768.29844577,207.67377566)
\curveto(767.65133521,206.67822096)(767.32777993,205.17244446)(767.32777993,203.15644618)
\closepath
}
}
{
\newrgbcolor{curcolor}{0 0 0}
\pscustom[linestyle=none,fillstyle=solid,fillcolor=curcolor]
{
\newpath
\moveto(794.28242395,192.55378854)
\curveto(791.24598209,192.55378854)(788.89398409,193.38756561)(787.22642996,195.05511974)
\curveto(785.58376469,196.72267388)(784.76243206,199.37333829)(784.76243206,203.00711297)
\curveto(784.76243206,205.49599974)(785.18554281,207.52444246)(786.03176431,209.09244112)
\curveto(786.87798581,210.66043979)(788.04776259,211.81777213)(789.54109465,212.56443817)
\curveto(791.05931558,213.3111042)(792.80153632,213.68443721)(794.76775687,213.68443721)
\curveto(796.16153346,213.68443721)(797.36864354,213.54754844)(798.38908712,213.27377089)
\curveto(799.43441956,212.99999335)(800.34286323,212.67643807)(801.11441813,212.30310505)
\lineto(799.47175286,208.00977538)
\curveto(798.60064249,208.35821952)(797.77930986,208.6444415)(797.00775496,208.86844131)
\curveto(796.26108893,209.09244112)(795.5144229,209.20444103)(794.76775687,209.20444103)
\curveto(791.88064822,209.20444103)(790.43709389,207.15110944)(790.43709389,203.04444627)
\curveto(790.43709389,201.00355912)(790.81042691,199.49778263)(791.55709294,198.52711678)
\curveto(792.32864784,197.55645094)(793.39886915,197.07111802)(794.76775687,197.07111802)
\curveto(795.93753365,197.07111802)(796.97042166,197.22045123)(797.8664209,197.51911764)
\curveto(798.76242013,197.84267292)(799.6335305,198.27822811)(800.47975201,198.8257832)
\lineto(800.47975201,194.0844539)
\curveto(799.6335305,193.53689881)(798.73753127,193.15112136)(797.79175429,192.92712155)
\curveto(796.87086619,192.67823288)(795.70108941,192.55378854)(794.28242395,192.55378854)
\closepath
}
}
{
\newrgbcolor{curcolor}{0 0 0}
\pscustom[linestyle=none,fillstyle=solid,fillcolor=curcolor]
{
\newpath
\moveto(818.54902298,213.3111042)
\lineto(824.67168443,213.3111042)
\lineto(816.6076913,203.52977919)
\lineto(825.38101716,192.92712155)
\lineto(819.0716892,192.92712155)
\lineto(810.74636296,203.26844608)
\lineto(810.74636296,192.92712155)
\lineto(805.18370103,192.92712155)
\lineto(805.18370103,213.3111042)
\lineto(810.74636296,213.3111042)
\lineto(810.74636296,203.41777929)
\closepath
}
}
{
\newrgbcolor{curcolor}{0 0 0}
\pscustom[linestyle=none,fillstyle=solid,fillcolor=curcolor]
{
\newpath
\moveto(833.66901041,213.3111042)
\lineto(833.66901041,205.24711106)
\curveto(833.66901041,204.82400031)(833.64412155,204.30133409)(833.59434381,203.6791124)
\curveto(833.56945494,203.05689071)(833.53212164,202.42222458)(833.48234391,201.77511402)
\curveto(833.45745504,201.12800346)(833.42012174,200.54311507)(833.370344,200.02044885)
\curveto(833.32056627,199.52267149)(833.28323296,199.18667178)(833.2583441,199.0124497)
\lineto(842.66633609,213.3111042)
\lineto(849.34899706,213.3111042)
\lineto(849.34899706,192.92712155)
\lineto(843.97300164,192.92712155)
\lineto(843.97300164,201.06578129)
\curveto(843.97300164,201.71289185)(843.99789051,202.44711345)(844.04766824,203.26844608)
\curveto(844.09744598,204.08977871)(844.14722371,204.84888918)(844.19700145,205.54577748)
\curveto(844.27166805,206.26755464)(844.32144579,206.81510973)(844.34633465,207.18844274)
\lineto(834.97567597,192.92712155)
\lineto(828.29301499,192.92712155)
\lineto(828.29301499,213.3111042)
\closepath
}
}
\end{pspicture}
}
		\caption{Ограниченная система прав}
		\label{fig:img2}
	\end{center}
\end{figure}

Доступ к комнате со свободной системой прав будет осуществляться с помощью специального ключа участника.

\noindent
В случае комнаты с ограниченной системой прав — доступ предоставляется с помощью двух ключей: ключ владельца комнаты и общий ключ участника.
Если в комнату попытаются зайти два или более пользователей как владельцы комнаты, то сработает «право первого»: первый, кто зашел
как владелец, им и станет, остальные будут расценены как обычные участники без особых привилегий.

Исходя из проведенных исследований, данная схема взаимодействия пользователей будет оптимальным решением поставленной задачи.

	
	\newpage

	\section{\centering Платформа проекта: сервер}
	
	В качестве основы сервера была выбрана кроссплатформенная среда выполнения JavaScript
с открытым исходным кодом — \emph{Node.js}.

\emph{Node.js} — программная платформа, основанная на движке V8 (компилирующем
JavaScript в машинный код), превращающая JavaScript из узкоспециализированного языка
в язык общего назначения. Node.js добавляет возможность JavaScript взаимодействовать
с устройствами ввода-вывода через свой API, написанный на C++, подключать другие внешние
библиотеки, написанные на разных языках, обеспечивая вызовы к ним из JavaScript-кода.
Node.js применяется преимущественно на сервере, выполняя роль веб-сервера.
В основе Node.js лежит событийно-ориентированное и асинхронное (или реактивное)
программирование с неблокирующим вводом/выводом.

Событиийно-ориентиированное программирование (СОП) — парадигма программирования, в которой
выполнение программы определяется событиями — действиями пользователя (клавиатура, мышь, сенсорный экран),
сообщениями других программ и потоков, событиями операционной системы (например, поступлением сетевого пакета).

Асинхронное программирование — концепция программирования, которая заключается в том,
что результат выполнения функции доступен не сразу, а через некоторое время в виде
некоторого асинхронного (нарушающего обычный порядок выполнения) вызова.

\noindent
В отличие от синхронного программирования, где компьютер выполняет инструкции последовательно
и ожидает завершения системных операций (обращение к устройствам ввода-вывода, жесткому диску,
сетевой запрос) блокируя следующие операции в потоке выполнения, в асинхронном программировании
длительные операции запускаются без ожидания их завершения и не блокируя дальнейшее выполнение программы.

\noindent
Асинхронный ввод/вывод является формой неблокирующей обработки ввода/вывода, который позволяет
процессу продолжить выполнение, не дожидаясь окончания передачи данных.

\noindent
Входные и выходные (I/O) операции на компьютере могут быть весьма медленными, по сравнению с обработкой данных.
Устройство ввода/вывода может быть на несколько порядков медленнее, чем оперативная память.
Например, во время дисковой операции, которой требуется десять миллисекунд для выполнения, процессор,
который работает на частоте один гигагерц, может выполнить десять миллионов циклов команд обработки.

На платформе \emph{Node.js} будет базироваться HTTP-сервер — основа клиент-серверного приложения.

\noindent
HTTP (англ. HyperText Transfer Protocol) — протокол прикладного уровня передачи данных,
изначально — в виде гипертекстовых документов в формате HTML, в настоящее время используется
для передачи произвольных данных.

Так как стандартный функционал \emph{Node.js} «из коробки» не имеет удобных методов
маршрутизации сервера (процесс определения оптимального маршрута данных в сетях связи),
то было принято решение использовать фреймворк \emph{Express.js}.

\noindent
\emph{Express.js} — это минимальная и гибкая платформа веб-приложений Node.js,
которая предоставляет надежный и удобный в использовании набор функций для веб-приложений.

Связка \emph{Node.js} и \emph{Express.js} является очень популярной. Её используют многие
крупные IT-компании.

\newpage
\begin{figure}[H]
	\begin{center}
		\includegraphics[width=\linewidth]{g4/img1.png}
		\caption{Компании, использующие связку \emph{Node.js} и \emph{Express.js}}
		\label{g4_img1}
	\end{center}
\end{figure}

	
	\hfill \break

	\section{\centering Веб-интерфейс: клиент}
	
	Следующая задача: создать схему интерфейса (Рис.~\ref{g5_ink1}).

\begin{figure}[h]
	\begin{center}
		\scalebox{0.6}{%LaTeX with PSTricks extensions
%%Creator: Inkscape 1.2 (dc2aedaf03, 2022-05-15)
%%Please note this file requires PSTricks extensions
\psset{xunit=.5pt,yunit=.5pt,runit=.5pt}
\begin{pspicture}(1024,768)
{
\newrgbcolor{curcolor}{0.50196081 0.50196081 0.50196081}
\pscustom[linestyle=none,fillstyle=solid,fillcolor=curcolor]
{
\newpath
\moveto(600.12784576,374.23065186)
\curveto(600.12784576,326.38734633)(561.34318346,287.60268402)(513.49987793,287.60268402)
\curveto(465.6565724,287.60268402)(426.8719101,326.38734633)(426.8719101,374.23065186)
\curveto(426.8719101,422.07395738)(465.6565724,460.85861969)(513.49987793,460.85861969)
\curveto(561.34318346,460.85861969)(600.12784576,422.07395738)(600.12784576,374.23065186)
\closepath
}
}
{
\newrgbcolor{curcolor}{0.50196081 0.50196081 0.50196081}
\pscustom[linestyle=none,fillstyle=solid,fillcolor=curcolor]
{
\newpath
\moveto(67.8839035,734.56464386)
\lineto(320.16886139,734.56464386)
\lineto(320.16886139,570.42743683)
\lineto(67.8839035,570.42743683)
\closepath
}
}
{
\newrgbcolor{curcolor}{0.50196081 0.50196081 0.50196081}
\pscustom[linestyle=none,fillstyle=solid,fillcolor=curcolor]
{
\newpath
\moveto(39.00791931,469.10818481)
\lineto(291.2928772,469.10818481)
\lineto(291.2928772,304.97097778)
\lineto(39.00791931,304.97097778)
\closepath
}
}
{
\newrgbcolor{curcolor}{0.50196081 0.50196081 0.50196081}
\pscustom[linestyle=none,fillstyle=solid,fillcolor=curcolor]
{
\newpath
\moveto(95.7467041,209.73083496)
\lineto(348.03166199,209.73083496)
\lineto(348.03166199,45.59362793)
\lineto(95.7467041,45.59362793)
\closepath
}
}
{
\newrgbcolor{curcolor}{0.50196081 0.50196081 0.50196081}
\pscustom[linestyle=none,fillstyle=solid,fillcolor=curcolor]
{
\newpath
\moveto(444.28497314,186.42749023)
\lineto(696.56993103,186.42749023)
\lineto(696.56993103,22.2902832)
\lineto(444.28497314,22.2902832)
\closepath
}
}
{
\newrgbcolor{curcolor}{0.50196081 0.50196081 0.50196081}
\pscustom[linestyle=none,fillstyle=solid,fillcolor=curcolor]
{
\newpath
\moveto(429.08706665,746.7229557)
\lineto(681.37202454,746.7229557)
\lineto(681.37202454,582.58574867)
\lineto(429.08706665,582.58574867)
\closepath
}
}
{
\newrgbcolor{curcolor}{0.50196081 0.50196081 0.50196081}
\pscustom[linestyle=none,fillstyle=solid,fillcolor=curcolor]
{
\newpath
\moveto(735.07122803,597.78366089)
\lineto(987.35618591,597.78366089)
\lineto(987.35618591,433.64645386)
\lineto(735.07122803,433.64645386)
\closepath
}
}
{
\newrgbcolor{curcolor}{0.50196081 0.50196081 0.50196081}
\pscustom[linestyle=none,fillstyle=solid,fillcolor=curcolor]
{
\newpath
\moveto(754.32189941,322.19525146)
\lineto(1006.6068573,322.19525146)
\lineto(1006.6068573,158.05804443)
\lineto(754.32189941,158.05804443)
\closepath
}
}
{
\newrgbcolor{curcolor}{0.50196081 0.50196081 0.50196081}
\pscustom[linestyle=none,fillstyle=solid,fillcolor=curcolor]
{
\newpath
\moveto(313.79867554,594.62263489)
\lineto(394.03943634,594.62263489)
\lineto(394.03943634,583.87610435)
\lineto(313.79867554,583.87610435)
\closepath
}
}
{
\newrgbcolor{curcolor}{0.50196081 0.50196081 0.50196081}
\pscustom[linestyle=none,fillstyle=solid,fillcolor=curcolor]
{
\newpath
\moveto(382.67831421,444.97894287)
\lineto(470.97897339,444.97894287)
\lineto(470.97897339,434.23241234)
\lineto(382.67831421,434.23241234)
\closepath
}
}
{
\newrgbcolor{curcolor}{0.50196081 0.50196081 0.50196081}
\pscustom[linestyle=none,fillstyle=solid,fillcolor=curcolor]
{
\newpath
\moveto(285.64385986,387.85348511)
\lineto(446.89439392,387.85348511)
\lineto(446.89439392,377.10695457)
\lineto(285.64385986,377.10695457)
\closepath
}
}
{
\newrgbcolor{curcolor}{0.50196081 0.50196081 0.50196081}
\pscustom[linestyle=none,fillstyle=solid,fillcolor=curcolor]
{
\newpath
\moveto(330.60947553,205.08183262)
\lineto(330.39066842,326.80749602)
\lineto(341.13716587,326.84420616)
\lineto(341.35597297,205.11854276)
\closepath
}
}
{
\newrgbcolor{curcolor}{0.50196081 0.50196081 0.50196081}
\pscustom[linestyle=none,fillstyle=solid,fillcolor=curcolor]
{
\newpath
\moveto(330.92089844,326.840271)
\lineto(461.90229797,326.840271)
\lineto(461.90229797,316.09374046)
\lineto(330.92089844,316.09374046)
\closepath
}
}
{
\newrgbcolor{curcolor}{0.50196081 0.50196081 0.50196081}
\pscustom[linestyle=none,fillstyle=solid,fillcolor=curcolor]
{
\newpath
\moveto(382.57647705,594.00695801)
\lineto(394.03944302,594.00695801)
\lineto(394.03944302,434.24186707)
\lineto(382.57647705,434.24186707)
\closepath
}
}
{
\newrgbcolor{curcolor}{0.50196081 0.50196081 0.50196081}
\pscustom[linestyle=none,fillstyle=solid,fillcolor=curcolor]
{
\newpath
\moveto(514.24965441,177.18478123)
\lineto(513.8351335,296.27359812)
\lineto(524.58154595,296.34163792)
\lineto(524.99606686,177.25282103)
\closepath
}
}
{
\newrgbcolor{curcolor}{0.50196081 0.50196081 0.50196081}
\pscustom[linestyle=none,fillstyle=solid,fillcolor=curcolor]
{
\newpath
\moveto(673.71789551,264.7041626)
\lineto(769.1829071,264.7041626)
\lineto(769.1829071,253.95763206)
\lineto(673.71789551,253.95763206)
\closepath
}
}
{
\newrgbcolor{curcolor}{0.50196081 0.50196081 0.50196081}
\pscustom[linestyle=none,fillstyle=solid,fillcolor=curcolor]
{
\newpath
\moveto(683.04614003,342.22914301)
\lineto(682.43288177,253.93061343)
\lineto(671.68661041,254.00524932)
\lineto(672.29986868,342.3037789)
\closepath
}
}
{
\newrgbcolor{curcolor}{0.50196081 0.50196081 0.50196081}
\pscustom[linestyle=none,fillstyle=solid,fillcolor=curcolor]
{
\newpath
\moveto(559.59301758,342.27557373)
\lineto(682.81990051,342.27557373)
\lineto(682.81990051,331.5290432)
\lineto(559.59301758,331.5290432)
\closepath
}
}
{
\newrgbcolor{curcolor}{0.50196081 0.50196081 0.50196081}
\pscustom[linestyle=none,fillstyle=solid,fillcolor=curcolor]
{
\newpath
\moveto(527.68161008,589.24600695)
\lineto(527.66652697,451.80652469)
\lineto(516.91999687,451.8093819)
\lineto(516.93507998,589.24886417)
\closepath
}
}
{
\newrgbcolor{curcolor}{0.50196081 0.50196081 0.50196081}
\pscustom[linestyle=none,fillstyle=solid,fillcolor=curcolor]
{
\newpath
\moveto(659.51196289,500.82443237)
\lineto(747.81262207,500.82443237)
\lineto(747.81262207,490.07790184)
\lineto(659.51196289,490.07790184)
\closepath
}
}
{
\newrgbcolor{curcolor}{0.50196081 0.50196081 0.50196081}
\pscustom[linestyle=none,fillstyle=solid,fillcolor=curcolor]
{
\newpath
\moveto(659.94961687,420.78382899)
\lineto(659.62589085,500.13245011)
\lineto(670.37234872,500.16785472)
\lineto(670.69607474,420.8192336)
\closepath
}
}
{
\newrgbcolor{curcolor}{0.50196081 0.50196081 0.50196081}
\pscustom[linestyle=none,fillstyle=solid,fillcolor=curcolor]
{
\newpath
\moveto(576.84246826,431.52166748)
\lineto(665.14312744,431.52166748)
\lineto(665.14312744,420.77513695)
\lineto(576.84246826,420.77513695)
\closepath
}
}
{
\newrgbcolor{curcolor}{0 0 0}
\pscustom[linestyle=none,fillstyle=solid,fillcolor=curcolor]
{
\newpath
\moveto(131.86485746,685.78957373)
\lineto(131.86485746,666.97357373)
\lineto(134.90485746,666.97357373)
\lineto(134.90485746,656.28557373)
\lineto(130.23285746,656.28557373)
\lineto(130.23285746,662.94157373)
\lineto(115.92885746,662.94157373)
\lineto(115.92885746,656.28557373)
\lineto(111.25685746,656.28557373)
\lineto(111.25685746,666.97357373)
\lineto(113.01685746,666.97357373)
\curveto(113.82752413,668.55224039)(114.58485746,670.25890706)(115.28885746,672.09357373)
\curveto(115.99285746,673.92824039)(116.62219079,675.96557373)(117.17685746,678.20557373)
\curveto(117.73152413,680.44557373)(118.19019079,682.97357373)(118.55285746,685.78957373)
\closepath
\moveto(127.03285746,681.75757373)
\lineto(122.39285746,681.75757373)
\curveto(122.20085746,680.26424039)(121.89152413,678.66424039)(121.46485746,676.95757373)
\curveto(121.05952413,675.25090706)(120.56885746,673.53357373)(119.99285746,671.80557373)
\curveto(119.41685746,670.09890706)(118.78752413,668.48824039)(118.10485746,666.97357373)
\lineto(127.03285746,666.97357373)
\closepath
}
}
{
\newrgbcolor{curcolor}{0 0 0}
\pscustom[linestyle=none,fillstyle=solid,fillcolor=curcolor]
{
\newpath
\moveto(145.20885551,680.76557373)
\curveto(147.55552217,680.76557373)(149.34752217,680.25357373)(150.58485551,679.22957373)
\curveto(151.84352217,678.22690706)(152.47285551,676.68024039)(152.47285551,674.58957373)
\lineto(152.47285551,662.94157373)
\lineto(149.14485551,662.94157373)
\lineto(148.21685551,665.30957373)
\lineto(148.08885551,665.30957373)
\curveto(147.34218884,664.37090706)(146.55285551,663.68824039)(145.72085551,663.26157373)
\curveto(144.88885551,662.83490706)(143.74752217,662.62157373)(142.29685551,662.62157373)
\curveto(140.73952217,662.62157373)(139.44885551,663.06957373)(138.42485551,663.96557373)
\curveto(137.40085551,664.86157373)(136.88885551,666.25890706)(136.88885551,668.15757373)
\curveto(136.88885551,670.01357373)(137.53952217,671.37890706)(138.84085551,672.25357373)
\curveto(140.14218884,673.12824039)(142.09418884,673.61890706)(144.69685551,673.72557373)
\lineto(147.73685551,673.82157373)
\lineto(147.73685551,674.58957373)
\curveto(147.73685551,675.50690706)(147.49152217,676.17890706)(147.00085551,676.60557373)
\curveto(146.53152217,677.03224039)(145.87018884,677.24557373)(145.01685551,677.24557373)
\curveto(144.16352217,677.24557373)(143.33152217,677.11757373)(142.52085551,676.86157373)
\curveto(141.71018884,676.62690706)(140.89952217,676.32824039)(140.08885551,675.96557373)
\lineto(138.52085551,679.19757373)
\curveto(139.43818884,679.66690706)(140.47285551,680.04024039)(141.62485551,680.31757373)
\curveto(142.77685551,680.61624039)(143.97152217,680.76557373)(145.20885551,680.76557373)
\closepath
\moveto(147.73685551,671.03757373)
\lineto(145.88085551,670.97357373)
\curveto(144.34485551,670.93090706)(143.27818884,670.65357373)(142.68085551,670.14157373)
\curveto(142.08352217,669.62957373)(141.78485551,668.95757373)(141.78485551,668.12557373)
\curveto(141.78485551,667.40024039)(141.99818884,666.87757373)(142.42485551,666.55757373)
\curveto(142.85152217,666.25890706)(143.40618884,666.10957373)(144.08885551,666.10957373)
\curveto(145.11285551,666.10957373)(145.97685551,666.40824039)(146.68085551,667.00557373)
\curveto(147.38485551,667.62424039)(147.73685551,668.48824039)(147.73685551,669.59757373)
\closepath
}
}
{
\newrgbcolor{curcolor}{0 0 0}
\pscustom[linestyle=none,fillstyle=solid,fillcolor=curcolor]
{
\newpath
\moveto(162.13685844,680.41357373)
\lineto(162.13685844,673.69357373)
\lineto(168.79285844,673.69357373)
\lineto(168.79285844,680.41357373)
\lineto(173.56085844,680.41357373)
\lineto(173.56085844,662.94157373)
\lineto(168.79285844,662.94157373)
\lineto(168.79285844,670.14157373)
\lineto(162.13685844,670.14157373)
\lineto(162.13685844,662.94157373)
\lineto(157.36885844,662.94157373)
\lineto(157.36885844,680.41357373)
\closepath
}
}
{
\newrgbcolor{curcolor}{0 0 0}
\pscustom[linestyle=none,fillstyle=solid,fillcolor=curcolor]
{
\newpath
\moveto(183.32087943,680.41357373)
\lineto(183.32087943,673.69357373)
\lineto(189.97687943,673.69357373)
\lineto(189.97687943,680.41357373)
\lineto(194.74487943,680.41357373)
\lineto(194.74487943,662.94157373)
\lineto(189.97687943,662.94157373)
\lineto(189.97687943,670.14157373)
\lineto(183.32087943,670.14157373)
\lineto(183.32087943,662.94157373)
\lineto(178.55287943,662.94157373)
\lineto(178.55287943,680.41357373)
\closepath
}
}
{
\newrgbcolor{curcolor}{0 0 0}
\pscustom[linestyle=none,fillstyle=solid,fillcolor=curcolor]
{
\newpath
\moveto(199.73690043,662.94157373)
\lineto(199.73690043,680.41357373)
\lineto(204.50490043,680.41357373)
\lineto(204.50490043,673.66157373)
\lineto(206.80890043,673.66157373)
\curveto(209.4755671,673.66157373)(211.44890043,673.23490706)(212.72890043,672.38157373)
\curveto(214.00890043,671.52824039)(214.64890043,670.23757373)(214.64890043,668.50957373)
\curveto(214.64890043,666.80290706)(214.0515671,665.44824039)(212.85690043,664.44557373)
\curveto(211.66223376,663.44290706)(209.6995671,662.94157373)(206.96890043,662.94157373)
\closepath
\moveto(217.17690043,662.94157373)
\lineto(217.17690043,680.41357373)
\lineto(221.94490043,680.41357373)
\lineto(221.94490043,662.94157373)
\closepath
\moveto(204.50490043,666.23757373)
\lineto(206.71290043,666.23757373)
\curveto(207.6515671,666.23757373)(208.40890043,666.39757373)(208.98490043,666.71757373)
\curveto(209.58223376,667.05890706)(209.88090043,667.63490706)(209.88090043,668.44557373)
\curveto(209.88090043,669.72557373)(208.8035671,670.36557373)(206.64890043,670.36557373)
\lineto(204.50490043,670.36557373)
\closepath
}
}
{
\newrgbcolor{curcolor}{0 0 0}
\pscustom[linestyle=none,fillstyle=solid,fillcolor=curcolor]
{
\newpath
\moveto(234.13691264,680.73357373)
\curveto(236.5475793,680.73357373)(238.45691264,680.04024039)(239.86491264,678.65357373)
\curveto(241.27291264,677.28824039)(241.97691264,675.33624039)(241.97691264,672.79757373)
\lineto(241.97691264,670.49357373)
\lineto(230.71291264,670.49357373)
\curveto(230.7555793,669.14957373)(231.15024597,668.09357373)(231.89691264,667.32557373)
\curveto(232.66491264,666.55757373)(233.72091264,666.17357373)(235.06491264,666.17357373)
\curveto(236.17424597,666.17357373)(237.1875793,666.28024039)(238.10491264,666.49357373)
\curveto(239.0435793,666.72824039)(240.0035793,667.08024039)(240.98491264,667.54957373)
\lineto(240.98491264,663.86957373)
\curveto(240.11024597,663.44290706)(239.2035793,663.13357373)(238.26491264,662.94157373)
\curveto(237.32624597,662.72824039)(236.18491264,662.62157373)(234.84091264,662.62157373)
\curveto(233.0915793,662.62157373)(231.54491264,662.94157373)(230.20091264,663.58157373)
\curveto(228.85691264,664.24290706)(227.80091264,665.22424039)(227.03291264,666.52557373)
\curveto(226.26491264,667.84824039)(225.88091264,669.52290706)(225.88091264,671.54957373)
\curveto(225.88091264,673.57624039)(226.22224597,675.27224039)(226.90491264,676.63757373)
\curveto(227.60891264,678.00290706)(228.5795793,679.02690706)(229.81691264,679.70957373)
\curveto(231.05424597,680.39224039)(232.49424597,680.73357373)(234.13691264,680.73357373)
\closepath
\moveto(234.16891264,677.34157373)
\curveto(233.23024597,677.34157373)(232.46224597,677.04290706)(231.86491264,676.44557373)
\curveto(231.2675793,675.84824039)(230.9155793,674.92024039)(230.80891264,673.66157373)
\lineto(237.49691264,673.66157373)
\curveto(237.4755793,674.70690706)(237.1875793,675.58157373)(236.63291264,676.28557373)
\curveto(236.0995793,676.98957373)(235.27824597,677.34157373)(234.16891264,677.34157373)
\closepath
}
}
{
\newrgbcolor{curcolor}{0 0 0}
\pscustom[linestyle=none,fillstyle=solid,fillcolor=curcolor]
{
\newpath
\moveto(270.04090727,671.70957373)
\curveto(270.04090727,668.80824039)(269.27290727,666.56824039)(267.73690727,664.98957373)
\curveto(266.2222406,663.41090706)(264.15290727,662.62157373)(261.52890727,662.62157373)
\curveto(259.90757393,662.62157373)(258.45690727,662.97357373)(257.17690727,663.67757373)
\curveto(255.9182406,664.38157373)(254.9262406,665.40557373)(254.20090727,666.74957373)
\curveto(253.47557393,668.11490706)(253.11290727,669.76824039)(253.11290727,671.70957373)
\curveto(253.11290727,674.61090706)(253.8702406,676.84024039)(255.38490727,678.39757373)
\curveto(256.89957393,679.95490706)(258.97957393,680.73357373)(261.62490727,680.73357373)
\curveto(263.26757393,680.73357373)(264.7182406,680.38157373)(265.97690727,679.67757373)
\curveto(267.23557393,678.97357373)(268.22757393,677.94957373)(268.95290727,676.60557373)
\curveto(269.6782406,675.26157373)(270.04090727,673.62957373)(270.04090727,671.70957373)
\closepath
\moveto(257.97690727,671.70957373)
\curveto(257.97690727,669.98157373)(258.2542406,668.66957373)(258.80890727,667.77357373)
\curveto(259.38490727,666.89890706)(260.31290727,666.46157373)(261.59290727,666.46157373)
\curveto(262.85157393,666.46157373)(263.7582406,666.89890706)(264.31290727,667.77357373)
\curveto(264.88890727,668.66957373)(265.17690727,669.98157373)(265.17690727,671.70957373)
\curveto(265.17690727,673.43757373)(264.88890727,674.72824039)(264.31290727,675.58157373)
\curveto(263.7582406,676.45624039)(262.84090727,676.89357373)(261.56090727,676.89357373)
\curveto(260.3022406,676.89357373)(259.38490727,676.45624039)(258.80890727,675.58157373)
\curveto(258.2542406,674.72824039)(257.97690727,673.43757373)(257.97690727,671.70957373)
\closepath
}
}
{
\newrgbcolor{curcolor}{0 0 0}
\pscustom[linestyle=none,fillstyle=solid,fillcolor=curcolor]
{
\newpath
\moveto(133.84890629,640.41357373)
\lineto(139.09690629,640.41357373)
\lineto(132.18490629,632.02957373)
\lineto(139.70490629,622.94157373)
\lineto(134.29690629,622.94157373)
\lineto(127.16090629,631.80557373)
\lineto(127.16090629,622.94157373)
\lineto(122.39290629,622.94157373)
\lineto(122.39290629,640.41357373)
\lineto(127.16090629,640.41357373)
\lineto(127.16090629,631.93357373)
\closepath
}
}
{
\newrgbcolor{curcolor}{0 0 0}
\pscustom[linestyle=none,fillstyle=solid,fillcolor=curcolor]
{
\newpath
\moveto(157.43287504,631.70957373)
\curveto(157.43287504,628.80824039)(156.66487504,626.56824039)(155.12887504,624.98957373)
\curveto(153.61420837,623.41090706)(151.54487504,622.62157373)(148.92087504,622.62157373)
\curveto(147.29954171,622.62157373)(145.84887504,622.97357373)(144.56887504,623.67757373)
\curveto(143.31020837,624.38157373)(142.31820837,625.40557373)(141.59287504,626.74957373)
\curveto(140.86754171,628.11490706)(140.50487504,629.76824039)(140.50487504,631.70957373)
\curveto(140.50487504,634.61090706)(141.26220837,636.84024039)(142.77687504,638.39757373)
\curveto(144.29154171,639.95490706)(146.37154171,640.73357373)(149.01687504,640.73357373)
\curveto(150.65954171,640.73357373)(152.11020837,640.38157373)(153.36887504,639.67757373)
\curveto(154.62754171,638.97357373)(155.61954171,637.94957373)(156.34487504,636.60557373)
\curveto(157.07020837,635.26157373)(157.43287504,633.62957373)(157.43287504,631.70957373)
\closepath
\moveto(145.36887504,631.70957373)
\curveto(145.36887504,629.98157373)(145.64620837,628.66957373)(146.20087504,627.77357373)
\curveto(146.77687504,626.89890706)(147.70487504,626.46157373)(148.98487504,626.46157373)
\curveto(150.24354171,626.46157373)(151.15020837,626.89890706)(151.70487504,627.77357373)
\curveto(152.28087504,628.66957373)(152.56887504,629.98157373)(152.56887504,631.70957373)
\curveto(152.56887504,633.43757373)(152.28087504,634.72824039)(151.70487504,635.58157373)
\curveto(151.15020837,636.45624039)(150.23287504,636.89357373)(148.95287504,636.89357373)
\curveto(147.69420837,636.89357373)(146.77687504,636.45624039)(146.20087504,635.58157373)
\curveto(145.64620837,634.72824039)(145.36887504,633.43757373)(145.36887504,631.70957373)
\closepath
}
}
{
\newrgbcolor{curcolor}{0 0 0}
\pscustom[linestyle=none,fillstyle=solid,fillcolor=curcolor]
{
\newpath
\moveto(183.38485844,640.41357373)
\lineto(183.38485844,622.94157373)
\lineto(178.93685844,622.94157373)
\lineto(178.93685844,631.51757373)
\curveto(178.93685844,632.37090706)(178.9475251,633.20290706)(178.96885844,634.01357373)
\curveto(179.0115251,634.82424039)(179.06485844,635.57090706)(179.12885844,636.25357373)
\lineto(179.03285844,636.25357373)
\lineto(174.20085844,622.94157373)
\lineto(170.61685844,622.94157373)
\lineto(165.72085844,636.28557373)
\lineto(165.59285844,636.28557373)
\curveto(165.67819177,635.58157373)(165.7315251,634.82424039)(165.75285844,634.01357373)
\curveto(165.7955251,633.22424039)(165.81685844,632.34957373)(165.81685844,631.38957373)
\lineto(165.81685844,622.94157373)
\lineto(161.36885844,622.94157373)
\lineto(161.36885844,640.41357373)
\lineto(168.12085844,640.41357373)
\lineto(172.47285844,628.57357373)
\lineto(176.88885844,640.41357373)
\closepath
}
}
{
\newrgbcolor{curcolor}{0 0 0}
\pscustom[linestyle=none,fillstyle=solid,fillcolor=curcolor]
{
\newpath
\moveto(193.14485404,640.41357373)
\lineto(193.14485404,633.69357373)
\lineto(199.80085404,633.69357373)
\lineto(199.80085404,640.41357373)
\lineto(204.56885404,640.41357373)
\lineto(204.56885404,622.94157373)
\lineto(199.80085404,622.94157373)
\lineto(199.80085404,630.14157373)
\lineto(193.14485404,630.14157373)
\lineto(193.14485404,622.94157373)
\lineto(188.37685404,622.94157373)
\lineto(188.37685404,640.41357373)
\closepath
}
}
{
\newrgbcolor{curcolor}{0 0 0}
\pscustom[linestyle=none,fillstyle=solid,fillcolor=curcolor]
{
\newpath
\moveto(216.72887504,640.76557373)
\curveto(219.07554171,640.76557373)(220.86754171,640.25357373)(222.10487504,639.22957373)
\curveto(223.36354171,638.22690706)(223.99287504,636.68024039)(223.99287504,634.58957373)
\lineto(223.99287504,622.94157373)
\lineto(220.66487504,622.94157373)
\lineto(219.73687504,625.30957373)
\lineto(219.60887504,625.30957373)
\curveto(218.86220837,624.37090706)(218.07287504,623.68824039)(217.24087504,623.26157373)
\curveto(216.40887504,622.83490706)(215.26754171,622.62157373)(213.81687504,622.62157373)
\curveto(212.25954171,622.62157373)(210.96887504,623.06957373)(209.94487504,623.96557373)
\curveto(208.92087504,624.86157373)(208.40887504,626.25890706)(208.40887504,628.15757373)
\curveto(208.40887504,630.01357373)(209.05954171,631.37890706)(210.36087504,632.25357373)
\curveto(211.66220837,633.12824039)(213.61420837,633.61890706)(216.21687504,633.72557373)
\lineto(219.25687504,633.82157373)
\lineto(219.25687504,634.58957373)
\curveto(219.25687504,635.50690706)(219.01154171,636.17890706)(218.52087504,636.60557373)
\curveto(218.05154171,637.03224039)(217.39020837,637.24557373)(216.53687504,637.24557373)
\curveto(215.68354171,637.24557373)(214.85154171,637.11757373)(214.04087504,636.86157373)
\curveto(213.23020837,636.62690706)(212.41954171,636.32824039)(211.60887504,635.96557373)
\lineto(210.04087504,639.19757373)
\curveto(210.95820837,639.66690706)(211.99287504,640.04024039)(213.14487504,640.31757373)
\curveto(214.29687504,640.61624039)(215.49154171,640.76557373)(216.72887504,640.76557373)
\closepath
\moveto(219.25687504,631.03757373)
\lineto(217.40087504,630.97357373)
\curveto(215.86487504,630.93090706)(214.79820837,630.65357373)(214.20087504,630.14157373)
\curveto(213.60354171,629.62957373)(213.30487504,628.95757373)(213.30487504,628.12557373)
\curveto(213.30487504,627.40024039)(213.51820837,626.87757373)(213.94487504,626.55757373)
\curveto(214.37154171,626.25890706)(214.92620837,626.10957373)(215.60887504,626.10957373)
\curveto(216.63287504,626.10957373)(217.49687504,626.40824039)(218.20087504,627.00557373)
\curveto(218.90487504,627.62424039)(219.25687504,628.48824039)(219.25687504,629.59757373)
\closepath
}
}
{
\newrgbcolor{curcolor}{0 0 0}
\pscustom[linestyle=none,fillstyle=solid,fillcolor=curcolor]
{
\newpath
\moveto(243.35287797,636.82957373)
\lineto(237.62487797,636.82957373)
\lineto(237.62487797,622.94157373)
\lineto(232.85687797,622.94157373)
\lineto(232.85687797,636.82957373)
\lineto(227.12887797,636.82957373)
\lineto(227.12887797,640.41357373)
\lineto(243.35287797,640.41357373)
\closepath
}
}
{
\newrgbcolor{curcolor}{0 0 0}
\pscustom[linestyle=none,fillstyle=solid,fillcolor=curcolor]
{
\newpath
\moveto(253.46485453,640.73357373)
\curveto(255.8755212,640.73357373)(257.78485453,640.04024039)(259.19285453,638.65357373)
\curveto(260.60085453,637.28824039)(261.30485453,635.33624039)(261.30485453,632.79757373)
\lineto(261.30485453,630.49357373)
\lineto(250.04085453,630.49357373)
\curveto(250.0835212,629.14957373)(250.47818786,628.09357373)(251.22485453,627.32557373)
\curveto(251.99285453,626.55757373)(253.04885453,626.17357373)(254.39285453,626.17357373)
\curveto(255.50218786,626.17357373)(256.5155212,626.28024039)(257.43285453,626.49357373)
\curveto(258.3715212,626.72824039)(259.3315212,627.08024039)(260.31285453,627.54957373)
\lineto(260.31285453,623.86957373)
\curveto(259.43818786,623.44290706)(258.5315212,623.13357373)(257.59285453,622.94157373)
\curveto(256.65418786,622.72824039)(255.51285453,622.62157373)(254.16885453,622.62157373)
\curveto(252.4195212,622.62157373)(250.87285453,622.94157373)(249.52885453,623.58157373)
\curveto(248.18485453,624.24290706)(247.12885453,625.22424039)(246.36085453,626.52557373)
\curveto(245.59285453,627.84824039)(245.20885453,629.52290706)(245.20885453,631.54957373)
\curveto(245.20885453,633.57624039)(245.55018786,635.27224039)(246.23285453,636.63757373)
\curveto(246.93685453,638.00290706)(247.9075212,639.02690706)(249.14485453,639.70957373)
\curveto(250.38218786,640.39224039)(251.82218786,640.73357373)(253.46485453,640.73357373)
\closepath
\moveto(253.49685453,637.34157373)
\curveto(252.55818786,637.34157373)(251.79018786,637.04290706)(251.19285453,636.44557373)
\curveto(250.5955212,635.84824039)(250.2435212,634.92024039)(250.13685453,633.66157373)
\lineto(256.82485453,633.66157373)
\curveto(256.8035212,634.70690706)(256.5155212,635.58157373)(255.96085453,636.28557373)
\curveto(255.4275212,636.98957373)(254.60618786,637.34157373)(253.49685453,637.34157373)
\closepath
}
}
{
\newrgbcolor{curcolor}{0 0 0}
\pscustom[linestyle=none,fillstyle=solid,fillcolor=curcolor]
{
\newpath
\moveto(470.79520903,693.03228088)
\lineto(465.29120903,710.95228088)
\lineto(465.16320903,710.95228088)
\lineto(465.25920903,709.03228088)
\curveto(465.3018757,708.17894754)(465.34454236,707.26161421)(465.38720903,706.28028088)
\curveto(465.4298757,705.29894754)(465.45120903,704.42428088)(465.45120903,703.65628088)
\lineto(465.45120903,693.03228088)
\lineto(461.13120903,693.03228088)
\lineto(461.13120903,715.88028088)
\lineto(467.72320903,715.88028088)
\lineto(473.13120903,698.40828088)
\lineto(473.22720903,698.40828088)
\lineto(478.95520903,715.88028088)
\lineto(485.54720903,715.88028088)
\lineto(485.54720903,693.03228088)
\lineto(481.03520903,693.03228088)
\lineto(481.03520903,703.84828088)
\curveto(481.03520903,704.57361421)(481.0458757,705.40561421)(481.06720903,706.34428088)
\curveto(481.1098757,707.28294754)(481.1418757,708.16828088)(481.16320903,709.00028088)
\lineto(481.25920903,710.92028088)
\lineto(481.13120903,710.92028088)
\lineto(475.24320903,693.03228088)
\closepath
}
}
{
\newrgbcolor{curcolor}{0 0 0}
\pscustom[linestyle=none,fillstyle=solid,fillcolor=curcolor]
{
\newpath
\moveto(506.79523442,701.80028088)
\curveto(506.79523442,698.89894754)(506.02723442,696.65894754)(504.49123442,695.08028088)
\curveto(502.97656775,693.50161421)(500.90723442,692.71228088)(498.28323442,692.71228088)
\curveto(496.66190109,692.71228088)(495.21123442,693.06428088)(493.93123442,693.76828088)
\curveto(492.67256775,694.47228088)(491.68056775,695.49628088)(490.95523442,696.84028088)
\curveto(490.22990109,698.20561421)(489.86723442,699.85894754)(489.86723442,701.80028088)
\curveto(489.86723442,704.70161421)(490.62456775,706.93094754)(492.13923442,708.48828088)
\curveto(493.65390109,710.04561421)(495.73390109,710.82428088)(498.37923442,710.82428088)
\curveto(500.02190109,710.82428088)(501.47256775,710.47228088)(502.73123442,709.76828088)
\curveto(503.98990109,709.06428088)(504.98190109,708.04028088)(505.70723442,706.69628088)
\curveto(506.43256775,705.35228088)(506.79523442,703.72028088)(506.79523442,701.80028088)
\closepath
\moveto(494.73123442,701.80028088)
\curveto(494.73123442,700.07228088)(495.00856775,698.76028088)(495.56323442,697.86428088)
\curveto(496.13923442,696.98961421)(497.06723442,696.55228088)(498.34723442,696.55228088)
\curveto(499.60590109,696.55228088)(500.51256775,696.98961421)(501.06723442,697.86428088)
\curveto(501.64323442,698.76028088)(501.93123442,700.07228088)(501.93123442,701.80028088)
\curveto(501.93123442,703.52828088)(501.64323442,704.81894754)(501.06723442,705.67228088)
\curveto(500.51256775,706.54694754)(499.59523442,706.98428088)(498.31523442,706.98428088)
\curveto(497.05656775,706.98428088)(496.13923442,706.54694754)(495.56323442,705.67228088)
\curveto(495.00856775,704.81894754)(494.73123442,703.52828088)(494.73123442,701.80028088)
\closepath
}
}
{
\newrgbcolor{curcolor}{0 0 0}
\pscustom[linestyle=none,fillstyle=solid,fillcolor=curcolor]
{
\newpath
\moveto(526.44321782,710.50428088)
\lineto(526.44321782,696.52028088)
\lineto(529.00321782,696.52028088)
\lineto(529.00321782,686.76028088)
\lineto(524.71521782,686.76028088)
\lineto(524.71521782,693.03228088)
\lineto(512.97121782,693.03228088)
\lineto(512.97121782,686.76028088)
\lineto(508.68321782,686.76028088)
\lineto(508.68321782,696.52028088)
\lineto(510.15521782,696.52028088)
\curveto(510.92321782,697.69361421)(511.57388449,699.02694754)(512.10721782,700.52028088)
\curveto(512.64055115,702.03494754)(513.06721782,703.63494754)(513.38721782,705.32028088)
\curveto(513.70721782,707.02694754)(513.94188449,708.75494754)(514.09121782,710.50428088)
\closepath
\moveto(521.67521782,706.92028088)
\lineto(518.09121782,706.92028088)
\curveto(517.83521782,704.97894754)(517.48321782,703.13361421)(517.03521782,701.38428088)
\curveto(516.58721782,699.65628088)(515.95788449,698.03494754)(515.14721782,696.52028088)
\lineto(521.67521782,696.52028088)
\closepath
}
}
{
\newrgbcolor{curcolor}{0 0 0}
\pscustom[linestyle=none,fillstyle=solid,fillcolor=curcolor]
{
\newpath
\moveto(529.48320366,710.50428088)
\lineto(534.69920366,710.50428088)
\lineto(537.99520366,700.68028088)
\curveto(538.16587032,700.18961421)(538.29387032,699.69894754)(538.37920366,699.20828088)
\curveto(538.46453699,698.71761421)(538.52853699,698.19494754)(538.57120366,697.64028088)
\lineto(538.66720366,697.64028088)
\curveto(538.73120366,698.19494754)(538.81653699,698.71761421)(538.92320366,699.20828088)
\curveto(539.02987032,699.69894754)(539.16853699,700.18961421)(539.33920366,700.68028088)
\lineto(542.57120366,710.50428088)
\lineto(547.69120366,710.50428088)
\lineto(540.29920366,690.79228088)
\curveto(539.61653699,688.97894754)(538.64587032,687.62428088)(537.38720366,686.72828088)
\curveto(536.12853699,685.81094754)(534.66720366,685.35228088)(533.00320366,685.35228088)
\curveto(532.44853699,685.35228088)(531.97920366,685.38428088)(531.59520366,685.44828088)
\curveto(531.21120366,685.49094754)(530.86987032,685.54428088)(530.57120366,685.60828088)
\lineto(530.57120366,689.38428088)
\curveto(530.78453699,689.34161421)(531.06187032,689.29894754)(531.40320366,689.25628088)
\curveto(531.74453699,689.21361421)(532.09653699,689.19228088)(532.45920366,689.19228088)
\curveto(533.46187032,689.19228088)(534.25120366,689.50161421)(534.82720366,690.12028088)
\curveto(535.40320366,690.71761421)(535.84053699,691.44294754)(536.13920366,692.29628088)
\lineto(536.42720366,693.16028088)
\closepath
}
}
{
\newrgbcolor{curcolor}{0 0 0}
\pscustom[linestyle=none,fillstyle=solid,fillcolor=curcolor]
{
\newpath
\moveto(565.19519682,693.03228088)
\lineto(560.42719682,693.03228088)
\lineto(560.42719682,706.92028088)
\lineto(556.04319682,706.92028088)
\curveto(555.76586349,703.50694754)(555.39253016,700.75494754)(554.92319682,698.66428088)
\curveto(554.47519682,696.59494754)(553.83519682,695.08028088)(553.00319682,694.12028088)
\curveto(552.19253016,693.18161421)(551.11519682,692.71228088)(549.77119682,692.71228088)
\curveto(548.66186349,692.71228088)(547.75519682,692.88294754)(547.05119682,693.22428088)
\lineto(547.05119682,697.03228088)
\curveto(547.54186349,696.81894754)(548.05386349,696.71228088)(548.58719682,696.71228088)
\curveto(548.97119682,696.71228088)(549.32319682,696.90428088)(549.64319682,697.28828088)
\curveto(549.96319682,697.67228088)(550.26186349,698.36561421)(550.53919682,699.36828088)
\curveto(550.83786349,700.37094754)(551.10453016,701.76828088)(551.33919682,703.56028088)
\curveto(551.57386349,705.37361421)(551.78719682,707.68828088)(551.97919682,710.50428088)
\lineto(565.19519682,710.50428088)
\closepath
}
}
{
\newrgbcolor{curcolor}{0 0 0}
\pscustom[linestyle=none,fillstyle=solid,fillcolor=curcolor]
{
\newpath
\moveto(574.95521147,703.75228088)
\lineto(578.31521147,703.75228088)
\curveto(581.00321147,703.75228088)(582.98721147,703.32561421)(584.26721147,702.47228088)
\curveto(585.5685448,701.61894754)(586.21921147,700.32828088)(586.21921147,698.60028088)
\curveto(586.21921147,696.89361421)(585.62187814,695.53894754)(584.42721147,694.53628088)
\curveto(583.2325448,693.53361421)(581.25921147,693.03228088)(578.50721147,693.03228088)
\lineto(570.18721147,693.03228088)
\lineto(570.18721147,710.50428088)
\lineto(574.95521147,710.50428088)
\closepath
\moveto(581.45121147,698.53628088)
\curveto(581.45121147,699.81628088)(580.37387814,700.45628088)(578.21921147,700.45628088)
\lineto(574.95521147,700.45628088)
\lineto(574.95521147,696.32828088)
\lineto(578.28321147,696.32828088)
\curveto(579.2005448,696.32828088)(579.95787814,696.48828088)(580.55521147,696.80828088)
\curveto(581.1525448,697.14961421)(581.45121147,697.72561421)(581.45121147,698.53628088)
\closepath
}
}
{
\newrgbcolor{curcolor}{0 0 0}
\pscustom[linestyle=none,fillstyle=solid,fillcolor=curcolor]
{
\newpath
\moveto(613.86722905,710.50428088)
\lineto(613.86722905,696.52028088)
\lineto(616.42722905,696.52028088)
\lineto(616.42722905,686.76028088)
\lineto(612.13922905,686.76028088)
\lineto(612.13922905,693.03228088)
\lineto(600.39522905,693.03228088)
\lineto(600.39522905,686.76028088)
\lineto(596.10722905,686.76028088)
\lineto(596.10722905,696.52028088)
\lineto(597.57922905,696.52028088)
\curveto(598.34722905,697.69361421)(598.99789572,699.02694754)(599.53122905,700.52028088)
\curveto(600.06456238,702.03494754)(600.49122905,703.63494754)(600.81122905,705.32028088)
\curveto(601.13122905,707.02694754)(601.36589572,708.75494754)(601.51522905,710.50428088)
\closepath
\moveto(609.09922905,706.92028088)
\lineto(605.51522905,706.92028088)
\curveto(605.25922905,704.97894754)(604.90722905,703.13361421)(604.45922905,701.38428088)
\curveto(604.01122905,699.65628088)(603.38189572,698.03494754)(602.57122905,696.52028088)
\lineto(609.09922905,696.52028088)
\closepath
}
}
{
\newrgbcolor{curcolor}{0 0 0}
\pscustom[linestyle=none,fillstyle=solid,fillcolor=curcolor]
{
\newpath
\moveto(635.05121489,693.03228088)
\lineto(630.28321489,693.03228088)
\lineto(630.28321489,706.92028088)
\lineto(625.89921489,706.92028088)
\curveto(625.62188156,703.50694754)(625.24854822,700.75494754)(624.77921489,698.66428088)
\curveto(624.33121489,696.59494754)(623.69121489,695.08028088)(622.85921489,694.12028088)
\curveto(622.04854822,693.18161421)(620.97121489,692.71228088)(619.62721489,692.71228088)
\curveto(618.51788156,692.71228088)(617.61121489,692.88294754)(616.90721489,693.22428088)
\lineto(616.90721489,697.03228088)
\curveto(617.39788156,696.81894754)(617.90988156,696.71228088)(618.44321489,696.71228088)
\curveto(618.82721489,696.71228088)(619.17921489,696.90428088)(619.49921489,697.28828088)
\curveto(619.81921489,697.67228088)(620.11788156,698.36561421)(620.39521489,699.36828088)
\curveto(620.69388156,700.37094754)(620.96054822,701.76828088)(621.19521489,703.56028088)
\curveto(621.42988156,705.37361421)(621.64321489,707.68828088)(621.83521489,710.50428088)
\lineto(635.05121489,710.50428088)
\closepath
}
}
{
\newrgbcolor{curcolor}{0 0 0}
\pscustom[linestyle=none,fillstyle=solid,fillcolor=curcolor]
{
\newpath
\moveto(642.69922954,693.03228088)
\lineto(637.54722954,693.03228088)
\lineto(642.25122954,699.94428088)
\curveto(641.35522954,700.30694754)(640.55522954,700.89361421)(639.85122954,701.70428088)
\curveto(639.16856287,702.53628088)(638.82722954,703.66694754)(638.82722954,705.09628088)
\curveto(638.82722954,706.84561421)(639.48856287,708.17894754)(640.81122954,709.09628088)
\curveto(642.1338962,710.03494754)(643.8298962,710.50428088)(645.89922954,710.50428088)
\lineto(654.02722954,710.50428088)
\lineto(654.02722954,693.03228088)
\lineto(649.25922954,693.03228088)
\lineto(649.25922954,699.52828088)
\lineto(646.63522954,699.52828088)
\closepath
\moveto(643.49922954,705.06428088)
\curveto(643.49922954,704.33894754)(643.78722954,703.76294754)(644.36322954,703.33628088)
\curveto(644.93922954,702.93094754)(645.6858962,702.72828088)(646.60322954,702.72828088)
\lineto(649.25922954,702.72828088)
\lineto(649.25922954,707.14428088)
\lineto(645.99522954,707.14428088)
\curveto(645.1418962,707.14428088)(644.51256287,706.93094754)(644.10722954,706.50428088)
\curveto(643.7018962,706.09894754)(643.49922954,705.61894754)(643.49922954,705.06428088)
\closepath
}
}
{
\newrgbcolor{curcolor}{0 0 0}
\pscustom[linestyle=none,fillstyle=solid,fillcolor=curcolor]
{
\newpath
\moveto(495.29124907,670.82428088)
\curveto(497.25391574,670.82428088)(498.84324907,670.05628088)(500.05924907,668.52028088)
\curveto(501.27524907,667.00561421)(501.88324907,664.76561421)(501.88324907,661.80028088)
\curveto(501.88324907,658.81361421)(501.25391574,656.55228088)(499.99524907,655.01628088)
\curveto(498.7365824,653.48028088)(497.12591574,652.71228088)(495.16324907,652.71228088)
\curveto(493.9045824,652.71228088)(492.90191574,652.93628088)(492.15524907,653.38428088)
\curveto(491.4085824,653.85361421)(490.8005824,654.37628088)(490.33124907,654.95228088)
\lineto(490.07524907,654.95228088)
\curveto(490.24591574,654.05628088)(490.33124907,653.20294754)(490.33124907,652.39228088)
\lineto(490.33124907,645.35228088)
\lineto(485.56324907,645.35228088)
\lineto(485.56324907,670.50428088)
\lineto(489.43524907,670.50428088)
\lineto(490.10724907,668.23228088)
\lineto(490.33124907,668.23228088)
\curveto(490.8005824,668.93628088)(491.42991574,669.54428088)(492.21924907,670.05628088)
\curveto(493.0085824,670.56828088)(494.0325824,670.82428088)(495.29124907,670.82428088)
\closepath
\moveto(493.75524907,667.01628088)
\curveto(492.51791574,667.01628088)(491.64324907,666.62161421)(491.13124907,665.83228088)
\curveto(490.61924907,665.06428088)(490.3525824,663.90161421)(490.33124907,662.34428088)
\lineto(490.33124907,661.83228088)
\curveto(490.33124907,660.14694754)(490.5765824,658.84561421)(491.06724907,657.92828088)
\curveto(491.57924907,657.03228088)(492.4965824,656.58428088)(493.81924907,656.58428088)
\curveto(494.90724907,656.58428088)(495.70724907,657.03228088)(496.21924907,657.92828088)
\curveto(496.7525824,658.84561421)(497.01924907,660.15761421)(497.01924907,661.86428088)
\curveto(497.01924907,665.29894754)(495.93124907,667.01628088)(493.75524907,667.01628088)
\closepath
}
}
{
\newrgbcolor{curcolor}{0 0 0}
\pscustom[linestyle=none,fillstyle=solid,fillcolor=curcolor]
{
\newpath
\moveto(512.98723051,670.85628088)
\curveto(515.33389718,670.85628088)(517.12589718,670.34428088)(518.36323051,669.32028088)
\curveto(519.62189718,668.31761421)(520.25123051,666.77094754)(520.25123051,664.68028088)
\lineto(520.25123051,653.03228088)
\lineto(516.92323051,653.03228088)
\lineto(515.99523051,655.40028088)
\lineto(515.86723051,655.40028088)
\curveto(515.12056385,654.46161421)(514.33123051,653.77894754)(513.49923051,653.35228088)
\curveto(512.66723051,652.92561421)(511.52589718,652.71228088)(510.07523051,652.71228088)
\curveto(508.51789718,652.71228088)(507.22723051,653.16028088)(506.20323051,654.05628088)
\curveto(505.17923051,654.95228088)(504.66723051,656.34961421)(504.66723051,658.24828088)
\curveto(504.66723051,660.10428088)(505.31789718,661.46961421)(506.61923051,662.34428088)
\curveto(507.92056385,663.21894754)(509.87256385,663.70961421)(512.47523051,663.81628088)
\lineto(515.51523051,663.91228088)
\lineto(515.51523051,664.68028088)
\curveto(515.51523051,665.59761421)(515.26989718,666.26961421)(514.77923051,666.69628088)
\curveto(514.30989718,667.12294754)(513.64856385,667.33628088)(512.79523051,667.33628088)
\curveto(511.94189718,667.33628088)(511.10989718,667.20828088)(510.29923051,666.95228088)
\curveto(509.48856385,666.71761421)(508.67789718,666.41894754)(507.86723051,666.05628088)
\lineto(506.29923051,669.28828088)
\curveto(507.21656385,669.75761421)(508.25123051,670.13094754)(509.40323051,670.40828088)
\curveto(510.55523051,670.70694754)(511.74989718,670.85628088)(512.98723051,670.85628088)
\closepath
\moveto(515.51523051,661.12828088)
\lineto(513.65923051,661.06428088)
\curveto(512.12323051,661.02161421)(511.05656385,660.74428088)(510.45923051,660.23228088)
\curveto(509.86189718,659.72028088)(509.56323051,659.04828088)(509.56323051,658.21628088)
\curveto(509.56323051,657.49094754)(509.77656385,656.96828088)(510.20323051,656.64828088)
\curveto(510.62989718,656.34961421)(511.18456385,656.20028088)(511.86723051,656.20028088)
\curveto(512.89123051,656.20028088)(513.75523051,656.49894754)(514.45923051,657.09628088)
\curveto(515.16323051,657.71494754)(515.51523051,658.57894754)(515.51523051,659.68828088)
\closepath
}
}
{
\newrgbcolor{curcolor}{0 0 0}
\pscustom[linestyle=none,fillstyle=solid,fillcolor=curcolor]
{
\newpath
\moveto(524.09123344,663.49628088)
\curveto(524.09123344,667.35761421)(524.81656678,670.35494754)(526.26723344,672.48828088)
\curveto(527.73923344,674.62161421)(530.16056678,675.98694754)(533.53123344,676.58428088)
\curveto(534.64056678,676.77628088)(535.78190011,676.93628088)(536.95523344,677.06428088)
\curveto(538.12856678,677.21361421)(539.33390011,677.36294754)(540.57123344,677.51228088)
\lineto(541.11523344,673.35228088)
\curveto(540.38990011,673.26694754)(539.58990011,673.17094754)(538.71523344,673.06428088)
\lineto(536.15523344,672.74428088)
\curveto(535.30190011,672.65894754)(534.55523344,672.56294754)(533.91523344,672.45628088)
\curveto(532.84856678,672.28561421)(531.96323344,672.01894754)(531.25923344,671.65628088)
\curveto(530.55523344,671.31494754)(530.01123344,670.76028088)(529.62723344,669.99228088)
\curveto(529.24323344,669.22428088)(529.01923344,668.11494754)(528.95523344,666.66428088)
\lineto(529.17923344,666.66428088)
\curveto(529.43523344,667.04828088)(529.78723344,667.44294754)(530.23523344,667.84828088)
\curveto(530.70456678,668.27494754)(531.26990011,668.62694754)(531.93123344,668.90428088)
\curveto(532.61390011,669.18161421)(533.40323344,669.32028088)(534.29923344,669.32028088)
\curveto(536.38990011,669.32028088)(538.04323344,668.66961421)(539.25923344,667.36828088)
\curveto(540.49656678,666.08828088)(541.11523344,664.18961421)(541.11523344,661.67228088)
\curveto(541.11523344,659.68828088)(540.75256678,658.02428088)(540.02723344,656.68028088)
\curveto(539.30190011,655.35761421)(538.29923344,654.36561421)(537.01923344,653.70428088)
\curveto(535.73923344,653.04294754)(534.25656678,652.71228088)(532.57123344,652.71228088)
\curveto(529.98990011,652.71228088)(527.93123344,653.64028088)(526.39523344,655.49628088)
\curveto(524.85923344,657.35228088)(524.09123344,660.01894754)(524.09123344,663.49628088)
\closepath
\moveto(532.85923344,656.58428088)
\curveto(533.86190011,656.58428088)(534.67256678,656.92561421)(535.29123344,657.60828088)
\curveto(535.93123344,658.29094754)(536.25123344,659.50694754)(536.25123344,661.25628088)
\curveto(536.25123344,662.64294754)(536.01656678,663.74161421)(535.54723344,664.55228088)
\curveto(535.09923344,665.38428088)(534.30990011,665.80028088)(533.17923344,665.80028088)
\curveto(532.49656678,665.80028088)(531.85656678,665.62961421)(531.25923344,665.28828088)
\curveto(530.68323344,664.96828088)(530.19256678,664.59494754)(529.78723344,664.16828088)
\curveto(529.38190011,663.74161421)(529.10456678,663.38961421)(528.95523344,663.11228088)
\curveto(528.95523344,662.02428088)(529.07256678,660.97894754)(529.30723344,659.97628088)
\curveto(529.54190011,658.97361421)(529.93656678,658.15228088)(530.49123344,657.51228088)
\curveto(531.06723344,656.89361421)(531.85656678,656.58428088)(532.85923344,656.58428088)
\closepath
}
}
{
\newrgbcolor{curcolor}{0 0 0}
\pscustom[linestyle=none,fillstyle=solid,fillcolor=curcolor]
{
\newpath
\moveto(560.92322514,661.80028088)
\curveto(560.92322514,658.89894754)(560.15522514,656.65894754)(558.61922514,655.08028088)
\curveto(557.10455848,653.50161421)(555.03522514,652.71228088)(552.41122514,652.71228088)
\curveto(550.78989181,652.71228088)(549.33922514,653.06428088)(548.05922514,653.76828088)
\curveto(546.80055848,654.47228088)(545.80855848,655.49628088)(545.08322514,656.84028088)
\curveto(544.35789181,658.20561421)(543.99522514,659.85894754)(543.99522514,661.80028088)
\curveto(543.99522514,664.70161421)(544.75255848,666.93094754)(546.26722514,668.48828088)
\curveto(547.78189181,670.04561421)(549.86189181,670.82428088)(552.50722514,670.82428088)
\curveto(554.14989181,670.82428088)(555.60055848,670.47228088)(556.85922514,669.76828088)
\curveto(558.11789181,669.06428088)(559.10989181,668.04028088)(559.83522514,666.69628088)
\curveto(560.56055848,665.35228088)(560.92322514,663.72028088)(560.92322514,661.80028088)
\closepath
\moveto(548.85922514,661.80028088)
\curveto(548.85922514,660.07228088)(549.13655848,658.76028088)(549.69122514,657.86428088)
\curveto(550.26722514,656.98961421)(551.19522514,656.55228088)(552.47522514,656.55228088)
\curveto(553.73389181,656.55228088)(554.64055848,656.98961421)(555.19522514,657.86428088)
\curveto(555.77122514,658.76028088)(556.05922514,660.07228088)(556.05922514,661.80028088)
\curveto(556.05922514,663.52828088)(555.77122514,664.81894754)(555.19522514,665.67228088)
\curveto(554.64055848,666.54694754)(553.72322514,666.98428088)(552.44322514,666.98428088)
\curveto(551.18455848,666.98428088)(550.26722514,666.54694754)(549.69122514,665.67228088)
\curveto(549.13655848,664.81894754)(548.85922514,663.52828088)(548.85922514,661.80028088)
\closepath
}
}
{
\newrgbcolor{curcolor}{0 0 0}
\pscustom[linestyle=none,fillstyle=solid,fillcolor=curcolor]
{
\newpath
\moveto(579.00320122,666.92028088)
\lineto(573.27520122,666.92028088)
\lineto(573.27520122,653.03228088)
\lineto(568.50720122,653.03228088)
\lineto(568.50720122,666.92028088)
\lineto(562.77920122,666.92028088)
\lineto(562.77920122,670.50428088)
\lineto(579.00320122,670.50428088)
\closepath
}
}
{
\newrgbcolor{curcolor}{0 0 0}
\pscustom[linestyle=none,fillstyle=solid,fillcolor=curcolor]
{
\newpath
\moveto(582.2351851,653.03228088)
\lineto(582.2351851,670.50428088)
\lineto(587.0031851,670.50428088)
\lineto(587.0031851,663.75228088)
\lineto(589.3071851,663.75228088)
\curveto(591.97385177,663.75228088)(593.9471851,663.32561421)(595.2271851,662.47228088)
\curveto(596.5071851,661.61894754)(597.1471851,660.32828088)(597.1471851,658.60028088)
\curveto(597.1471851,656.89361421)(596.54985177,655.53894754)(595.3551851,654.53628088)
\curveto(594.16051844,653.53361421)(592.19785177,653.03228088)(589.4671851,653.03228088)
\closepath
\moveto(599.6751851,653.03228088)
\lineto(599.6751851,670.50428088)
\lineto(604.4431851,670.50428088)
\lineto(604.4431851,653.03228088)
\closepath
\moveto(587.0031851,656.32828088)
\lineto(589.2111851,656.32828088)
\curveto(590.14985177,656.32828088)(590.9071851,656.48828088)(591.4831851,656.80828088)
\curveto(592.08051844,657.14961421)(592.3791851,657.72561421)(592.3791851,658.53628088)
\curveto(592.3791851,659.81628088)(591.30185177,660.45628088)(589.1471851,660.45628088)
\lineto(587.0031851,660.45628088)
\closepath
}
}
{
\newrgbcolor{curcolor}{0 0 0}
\pscustom[linestyle=none,fillstyle=solid,fillcolor=curcolor]
{
\newpath
\moveto(624.85920463,652.71228088)
\curveto(622.25653797,652.71228088)(620.24053797,653.42694754)(618.81120463,654.85628088)
\curveto(617.40320463,656.28561421)(616.69920463,658.55761421)(616.69920463,661.67228088)
\curveto(616.69920463,663.80561421)(617.0618713,665.54428088)(617.78720463,666.88828088)
\curveto(618.51253797,668.23228088)(619.51520463,669.22428088)(620.79520463,669.86428088)
\curveto(622.09653797,670.50428088)(623.5898713,670.82428088)(625.27520463,670.82428088)
\curveto(626.4698713,670.82428088)(627.50453797,670.70694754)(628.37920463,670.47228088)
\curveto(629.27520463,670.23761421)(630.0538713,669.96028088)(630.71520463,669.64028088)
\lineto(629.30720463,665.96028088)
\curveto(628.56053797,666.25894754)(627.85653797,666.50428088)(627.19520463,666.69628088)
\curveto(626.55520463,666.88828088)(625.91520463,666.98428088)(625.27520463,666.98428088)
\curveto(622.80053797,666.98428088)(621.56320463,665.22428088)(621.56320463,661.70428088)
\curveto(621.56320463,659.95494754)(621.88320463,658.66428088)(622.52320463,657.83228088)
\curveto(623.18453797,657.00028088)(624.1018713,656.58428088)(625.27520463,656.58428088)
\curveto(626.2778713,656.58428088)(627.16320463,656.71228088)(627.93120463,656.96828088)
\curveto(628.69920463,657.24561421)(629.4458713,657.61894754)(630.17120463,658.08828088)
\lineto(630.17120463,654.02428088)
\curveto(629.4458713,653.55494754)(628.6778713,653.22428088)(627.86720463,653.03228088)
\curveto(627.0778713,652.81894754)(626.07520463,652.71228088)(624.85920463,652.71228088)
\closepath
}
}
{
\newrgbcolor{curcolor}{0 0 0}
\pscustom[linestyle=none,fillstyle=solid,fillcolor=curcolor]
{
\newpath
\moveto(486.2832769,612.71228088)
\curveto(483.68061023,612.71228088)(481.66461023,613.42694754)(480.2352769,614.85628088)
\curveto(478.8272769,616.28561421)(478.1232769,618.55761421)(478.1232769,621.67228088)
\curveto(478.1232769,623.80561421)(478.48594357,625.54428088)(479.2112769,626.88828088)
\curveto(479.93661023,628.23228088)(480.9392769,629.22428088)(482.2192769,629.86428088)
\curveto(483.52061023,630.50428088)(485.01394357,630.82428088)(486.6992769,630.82428088)
\curveto(487.89394357,630.82428088)(488.92861023,630.70694754)(489.8032769,630.47228088)
\curveto(490.6992769,630.23761421)(491.47794357,629.96028088)(492.1392769,629.64028088)
\lineto(490.7312769,625.96028088)
\curveto(489.98461023,626.25894754)(489.28061023,626.50428088)(488.6192769,626.69628088)
\curveto(487.9792769,626.88828088)(487.3392769,626.98428088)(486.6992769,626.98428088)
\curveto(484.22461023,626.98428088)(482.9872769,625.22428088)(482.9872769,621.70428088)
\curveto(482.9872769,619.95494754)(483.3072769,618.66428088)(483.9472769,617.83228088)
\curveto(484.60861023,617.00028088)(485.52594357,616.58428088)(486.6992769,616.58428088)
\curveto(487.70194357,616.58428088)(488.5872769,616.71228088)(489.3552769,616.96828088)
\curveto(490.1232769,617.24561421)(490.86994357,617.61894754)(491.5952769,618.08828088)
\lineto(491.5952769,614.02428088)
\curveto(490.86994357,613.55494754)(490.10194357,613.22428088)(489.2912769,613.03228088)
\curveto(488.50194357,612.81894754)(487.4992769,612.71228088)(486.2832769,612.71228088)
\closepath
}
}
{
\newrgbcolor{curcolor}{0 0 0}
\pscustom[linestyle=none,fillstyle=solid,fillcolor=curcolor]
{
\newpath
\moveto(511.49927495,621.80028088)
\curveto(511.49927495,618.89894754)(510.73127495,616.65894754)(509.19527495,615.08028088)
\curveto(507.68060828,613.50161421)(505.61127495,612.71228088)(502.98727495,612.71228088)
\curveto(501.36594161,612.71228088)(499.91527495,613.06428088)(498.63527495,613.76828088)
\curveto(497.37660828,614.47228088)(496.38460828,615.49628088)(495.65927495,616.84028088)
\curveto(494.93394161,618.20561421)(494.57127495,619.85894754)(494.57127495,621.80028088)
\curveto(494.57127495,624.70161421)(495.32860828,626.93094754)(496.84327495,628.48828088)
\curveto(498.35794161,630.04561421)(500.43794161,630.82428088)(503.08327495,630.82428088)
\curveto(504.72594161,630.82428088)(506.17660828,630.47228088)(507.43527495,629.76828088)
\curveto(508.69394161,629.06428088)(509.68594161,628.04028088)(510.41127495,626.69628088)
\curveto(511.13660828,625.35228088)(511.49927495,623.72028088)(511.49927495,621.80028088)
\closepath
\moveto(499.43527495,621.80028088)
\curveto(499.43527495,620.07228088)(499.71260828,618.76028088)(500.26727495,617.86428088)
\curveto(500.84327495,616.98961421)(501.77127495,616.55228088)(503.05127495,616.55228088)
\curveto(504.30994161,616.55228088)(505.21660828,616.98961421)(505.77127495,617.86428088)
\curveto(506.34727495,618.76028088)(506.63527495,620.07228088)(506.63527495,621.80028088)
\curveto(506.63527495,623.52828088)(506.34727495,624.81894754)(505.77127495,625.67228088)
\curveto(505.21660828,626.54694754)(504.29927495,626.98428088)(503.01927495,626.98428088)
\curveto(501.76060828,626.98428088)(500.84327495,626.54694754)(500.26727495,625.67228088)
\curveto(499.71260828,624.81894754)(499.43527495,623.52828088)(499.43527495,621.80028088)
\closepath
}
}
{
\newrgbcolor{curcolor}{0 0 0}
\pscustom[linestyle=none,fillstyle=solid,fillcolor=curcolor]
{
\newpath
\moveto(526.89125835,630.50428088)
\lineto(532.13925835,630.50428088)
\lineto(525.22725835,622.12028088)
\lineto(532.74725835,613.03228088)
\lineto(527.33925835,613.03228088)
\lineto(520.20325835,621.89628088)
\lineto(520.20325835,613.03228088)
\lineto(515.43525835,613.03228088)
\lineto(515.43525835,630.50428088)
\lineto(520.20325835,630.50428088)
\lineto(520.20325835,622.02428088)
\closepath
}
}
{
\newrgbcolor{curcolor}{0 0 0}
\pscustom[linestyle=none,fillstyle=solid,fillcolor=curcolor]
{
\newpath
\moveto(541.8032271,630.82428088)
\curveto(544.21389376,630.82428088)(546.1232271,630.13094754)(547.5312271,628.74428088)
\curveto(548.9392271,627.37894754)(549.6432271,625.42694754)(549.6432271,622.88828088)
\lineto(549.6432271,620.58428088)
\lineto(538.3792271,620.58428088)
\curveto(538.42189376,619.24028088)(538.81656043,618.18428088)(539.5632271,617.41628088)
\curveto(540.3312271,616.64828088)(541.3872271,616.26428088)(542.7312271,616.26428088)
\curveto(543.84056043,616.26428088)(544.85389376,616.37094754)(545.7712271,616.58428088)
\curveto(546.70989376,616.81894754)(547.66989376,617.17094754)(548.6512271,617.64028088)
\lineto(548.6512271,613.96028088)
\curveto(547.77656043,613.53361421)(546.86989376,613.22428088)(545.9312271,613.03228088)
\curveto(544.99256043,612.81894754)(543.8512271,612.71228088)(542.5072271,612.71228088)
\curveto(540.75789376,612.71228088)(539.2112271,613.03228088)(537.8672271,613.67228088)
\curveto(536.5232271,614.33361421)(535.4672271,615.31494754)(534.6992271,616.61628088)
\curveto(533.9312271,617.93894754)(533.5472271,619.61361421)(533.5472271,621.64028088)
\curveto(533.5472271,623.66694754)(533.88856043,625.36294754)(534.5712271,626.72828088)
\curveto(535.2752271,628.09361421)(536.24589376,629.11761421)(537.4832271,629.80028088)
\curveto(538.72056043,630.48294754)(540.16056043,630.82428088)(541.8032271,630.82428088)
\closepath
\moveto(541.8352271,627.43228088)
\curveto(540.89656043,627.43228088)(540.12856043,627.13361421)(539.5312271,626.53628088)
\curveto(538.93389376,625.93894754)(538.58189376,625.01094754)(538.4752271,623.75228088)
\lineto(545.1632271,623.75228088)
\curveto(545.14189376,624.79761421)(544.85389376,625.67228088)(544.2992271,626.37628088)
\curveto(543.76589376,627.08028088)(542.94456043,627.43228088)(541.8352271,627.43228088)
\closepath
}
}
{
\newrgbcolor{curcolor}{0 0 0}
\pscustom[linestyle=none,fillstyle=solid,fillcolor=curcolor]
{
\newpath
\moveto(567.9792144,626.92028088)
\lineto(562.2512144,626.92028088)
\lineto(562.2512144,613.03228088)
\lineto(557.4832144,613.03228088)
\lineto(557.4832144,626.92028088)
\lineto(551.7552144,626.92028088)
\lineto(551.7552144,630.50428088)
\lineto(567.9792144,630.50428088)
\closepath
}
}
{
\newrgbcolor{curcolor}{0 0 0}
\pscustom[linestyle=none,fillstyle=solid,fillcolor=curcolor]
{
\newpath
\moveto(578.37919829,630.85628088)
\curveto(580.72586495,630.85628088)(582.51786495,630.34428088)(583.75519829,629.32028088)
\curveto(585.01386495,628.31761421)(585.64319829,626.77094754)(585.64319829,624.68028088)
\lineto(585.64319829,613.03228088)
\lineto(582.31519829,613.03228088)
\lineto(581.38719829,615.40028088)
\lineto(581.25919829,615.40028088)
\curveto(580.51253162,614.46161421)(579.72319829,613.77894754)(578.89119829,613.35228088)
\curveto(578.05919829,612.92561421)(576.91786495,612.71228088)(575.46719829,612.71228088)
\curveto(573.90986495,612.71228088)(572.61919829,613.16028088)(571.59519829,614.05628088)
\curveto(570.57119829,614.95228088)(570.05919829,616.34961421)(570.05919829,618.24828088)
\curveto(570.05919829,620.10428088)(570.70986495,621.46961421)(572.01119829,622.34428088)
\curveto(573.31253162,623.21894754)(575.26453162,623.70961421)(577.86719829,623.81628088)
\lineto(580.90719829,623.91228088)
\lineto(580.90719829,624.68028088)
\curveto(580.90719829,625.59761421)(580.66186495,626.26961421)(580.17119829,626.69628088)
\curveto(579.70186495,627.12294754)(579.04053162,627.33628088)(578.18719829,627.33628088)
\curveto(577.33386495,627.33628088)(576.50186495,627.20828088)(575.69119829,626.95228088)
\curveto(574.88053162,626.71761421)(574.06986495,626.41894754)(573.25919829,626.05628088)
\lineto(571.69119829,629.28828088)
\curveto(572.60853162,629.75761421)(573.64319829,630.13094754)(574.79519829,630.40828088)
\curveto(575.94719829,630.70694754)(577.14186495,630.85628088)(578.37919829,630.85628088)
\closepath
\moveto(580.90719829,621.12828088)
\lineto(579.05119829,621.06428088)
\curveto(577.51519829,621.02161421)(576.44853162,620.74428088)(575.85119829,620.23228088)
\curveto(575.25386495,619.72028088)(574.95519829,619.04828088)(574.95519829,618.21628088)
\curveto(574.95519829,617.49094754)(575.16853162,616.96828088)(575.59519829,616.64828088)
\curveto(576.02186495,616.34961421)(576.57653162,616.20028088)(577.25919829,616.20028088)
\curveto(578.28319829,616.20028088)(579.14719829,616.49894754)(579.85119829,617.09628088)
\curveto(580.55519829,617.71494754)(580.90719829,618.57894754)(580.90719829,619.68828088)
\closepath
}
}
{
\newrgbcolor{curcolor}{0 0 0}
\pscustom[linestyle=none,fillstyle=solid,fillcolor=curcolor]
{
\newpath
\moveto(612.55520122,630.50428088)
\lineto(612.55520122,613.03228088)
\lineto(608.10720122,613.03228088)
\lineto(608.10720122,621.60828088)
\curveto(608.10720122,622.46161421)(608.11786788,623.29361421)(608.13920122,624.10428088)
\curveto(608.18186788,624.91494754)(608.23520122,625.66161421)(608.29920122,626.34428088)
\lineto(608.20320122,626.34428088)
\lineto(603.37120122,613.03228088)
\lineto(599.78720122,613.03228088)
\lineto(594.89120122,626.37628088)
\lineto(594.76320122,626.37628088)
\curveto(594.84853455,625.67228088)(594.90186788,624.91494754)(594.92320122,624.10428088)
\curveto(594.96586788,623.31494754)(594.98720122,622.44028088)(594.98720122,621.48028088)
\lineto(594.98720122,613.03228088)
\lineto(590.53920122,613.03228088)
\lineto(590.53920122,630.50428088)
\lineto(597.29120122,630.50428088)
\lineto(601.64320122,618.66428088)
\lineto(606.05920122,630.50428088)
\closepath
}
}
{
\newrgbcolor{curcolor}{0 0 0}
\pscustom[linestyle=none,fillstyle=solid,fillcolor=curcolor]
{
\newpath
\moveto(622.15519682,630.50428088)
\lineto(622.15519682,623.59228088)
\curveto(622.15519682,623.22961421)(622.13386349,622.78161421)(622.09119682,622.24828088)
\curveto(622.06986349,621.71494754)(622.03786349,621.17094754)(621.99519682,620.61628088)
\curveto(621.97386349,620.06161421)(621.94186349,619.56028088)(621.89919682,619.11228088)
\curveto(621.85653016,618.68561421)(621.82453016,618.39761421)(621.80319682,618.24828088)
\lineto(629.86719682,630.50428088)
\lineto(635.59519682,630.50428088)
\lineto(635.59519682,613.03228088)
\lineto(630.98719682,613.03228088)
\lineto(630.98719682,620.00828088)
\curveto(630.98719682,620.56294754)(631.00853016,621.19228088)(631.05119682,621.89628088)
\curveto(631.09386349,622.60028088)(631.13653016,623.25094754)(631.17919682,623.84828088)
\curveto(631.24319682,624.46694754)(631.28586349,624.93628088)(631.30719682,625.25628088)
\lineto(623.27519682,613.03228088)
\lineto(617.54719682,613.03228088)
\lineto(617.54719682,630.50428088)
\closepath
}
}
{
\newrgbcolor{curcolor}{0 0 0}
\pscustom[linestyle=none,fillstyle=solid,fillcolor=curcolor]
{
\newpath
\moveto(795.16591005,546.80179)
\lineto(795.16591005,542.80179)
\lineto(785.56591005,542.80179)
\lineto(785.56591005,523.95379)
\lineto(780.73391005,523.95379)
\lineto(780.73391005,546.80179)
\closepath
}
}
{
\newrgbcolor{curcolor}{0 0 0}
\pscustom[linestyle=none,fillstyle=solid,fillcolor=curcolor]
{
\newpath
\moveto(811.93394618,532.72179)
\curveto(811.93394618,529.82045667)(811.16594618,527.58045667)(809.62994618,526.00179)
\curveto(808.11527952,524.42312333)(806.04594618,523.63379)(803.42194618,523.63379)
\curveto(801.80061285,523.63379)(800.34994618,523.98579)(799.06994618,524.68979)
\curveto(797.81127952,525.39379)(796.81927952,526.41779)(796.09394618,527.76179)
\curveto(795.36861285,529.12712333)(795.00594618,530.78045667)(795.00594618,532.72179)
\curveto(795.00594618,535.62312333)(795.76327952,537.85245667)(797.27794618,539.40979)
\curveto(798.79261285,540.96712333)(800.87261285,541.74579)(803.51794618,541.74579)
\curveto(805.16061285,541.74579)(806.61127952,541.39379)(807.86994618,540.68979)
\curveto(809.12861285,539.98579)(810.12061285,538.96179)(810.84594618,537.61779)
\curveto(811.57127952,536.27379)(811.93394618,534.64179)(811.93394618,532.72179)
\closepath
\moveto(799.86994618,532.72179)
\curveto(799.86994618,530.99379)(800.14727952,529.68179)(800.70194618,528.78579)
\curveto(801.27794618,527.91112333)(802.20594618,527.47379)(803.48594618,527.47379)
\curveto(804.74461285,527.47379)(805.65127952,527.91112333)(806.20594618,528.78579)
\curveto(806.78194618,529.68179)(807.06994618,530.99379)(807.06994618,532.72179)
\curveto(807.06994618,534.44979)(806.78194618,535.74045667)(806.20594618,536.59379)
\curveto(805.65127952,537.46845667)(804.73394618,537.90579)(803.45394618,537.90579)
\curveto(802.19527952,537.90579)(801.27794618,537.46845667)(800.70194618,536.59379)
\curveto(800.14727952,535.74045667)(799.86994618,534.44979)(799.86994618,532.72179)
\closepath
}
}
{
\newrgbcolor{curcolor}{0 0 0}
\pscustom[linestyle=none,fillstyle=solid,fillcolor=curcolor]
{
\newpath
\moveto(831.51792958,523.95379)
\lineto(826.74992958,523.95379)
\lineto(826.74992958,537.84179)
\lineto(822.36592958,537.84179)
\curveto(822.08859625,534.42845667)(821.71526292,531.67645667)(821.24592958,529.58579)
\curveto(820.79792958,527.51645667)(820.15792958,526.00179)(819.32592958,525.04179)
\curveto(818.51526292,524.10312333)(817.43792958,523.63379)(816.09392958,523.63379)
\curveto(814.98459625,523.63379)(814.07792958,523.80445667)(813.37392958,524.14579)
\lineto(813.37392958,527.95379)
\curveto(813.86459625,527.74045667)(814.37659625,527.63379)(814.90992958,527.63379)
\curveto(815.29392958,527.63379)(815.64592958,527.82579)(815.96592958,528.20979)
\curveto(816.28592958,528.59379)(816.58459625,529.28712333)(816.86192958,530.28979)
\curveto(817.16059625,531.29245667)(817.42726292,532.68979)(817.66192958,534.48179)
\curveto(817.89659625,536.29512333)(818.10992958,538.60979)(818.30192958,541.42579)
\lineto(831.51792958,541.42579)
\closepath
}
}
{
\newrgbcolor{curcolor}{0 0 0}
\pscustom[linestyle=none,fillstyle=solid,fillcolor=curcolor]
{
\newpath
\moveto(852.38194423,532.72179)
\curveto(852.38194423,529.82045667)(851.61394423,527.58045667)(850.07794423,526.00179)
\curveto(848.56327756,524.42312333)(846.49394423,523.63379)(843.86994423,523.63379)
\curveto(842.2486109,523.63379)(840.79794423,523.98579)(839.51794423,524.68979)
\curveto(838.25927756,525.39379)(837.26727756,526.41779)(836.54194423,527.76179)
\curveto(835.8166109,529.12712333)(835.45394423,530.78045667)(835.45394423,532.72179)
\curveto(835.45394423,535.62312333)(836.21127756,537.85245667)(837.72594423,539.40979)
\curveto(839.2406109,540.96712333)(841.3206109,541.74579)(843.96594423,541.74579)
\curveto(845.6086109,541.74579)(847.05927756,541.39379)(848.31794423,540.68979)
\curveto(849.5766109,539.98579)(850.5686109,538.96179)(851.29394423,537.61779)
\curveto(852.01927756,536.27379)(852.38194423,534.64179)(852.38194423,532.72179)
\closepath
\moveto(840.31794423,532.72179)
\curveto(840.31794423,530.99379)(840.59527756,529.68179)(841.14994423,528.78579)
\curveto(841.72594423,527.91112333)(842.65394423,527.47379)(843.93394423,527.47379)
\curveto(845.1926109,527.47379)(846.09927756,527.91112333)(846.65394423,528.78579)
\curveto(847.22994423,529.68179)(847.51794423,530.99379)(847.51794423,532.72179)
\curveto(847.51794423,534.44979)(847.22994423,535.74045667)(846.65394423,536.59379)
\curveto(846.09927756,537.46845667)(845.18194423,537.90579)(843.90194423,537.90579)
\curveto(842.64327756,537.90579)(841.72594423,537.46845667)(841.14994423,536.59379)
\curveto(840.59527756,535.74045667)(840.31794423,534.44979)(840.31794423,532.72179)
\closepath
}
}
{
\newrgbcolor{curcolor}{0 0 0}
\pscustom[linestyle=none,fillstyle=solid,fillcolor=curcolor]
{
\newpath
\moveto(863.42192763,523.63379)
\curveto(860.81926096,523.63379)(858.80326096,524.34845667)(857.37392763,525.77779)
\curveto(855.96592763,527.20712333)(855.26192763,529.47912333)(855.26192763,532.59379)
\curveto(855.26192763,534.72712333)(855.6245943,536.46579)(856.34992763,537.80979)
\curveto(857.07526096,539.15379)(858.07792763,540.14579)(859.35792763,540.78579)
\curveto(860.65926096,541.42579)(862.1525943,541.74579)(863.83792763,541.74579)
\curveto(865.0325943,541.74579)(866.06726096,541.62845667)(866.94192763,541.39379)
\curveto(867.83792763,541.15912333)(868.6165943,540.88179)(869.27792763,540.56179)
\lineto(867.86992763,536.88179)
\curveto(867.12326096,537.18045667)(866.41926096,537.42579)(865.75792763,537.61779)
\curveto(865.11792763,537.80979)(864.47792763,537.90579)(863.83792763,537.90579)
\curveto(861.36326096,537.90579)(860.12592763,536.14579)(860.12592763,532.62579)
\curveto(860.12592763,530.87645667)(860.44592763,529.58579)(861.08592763,528.75379)
\curveto(861.74726096,527.92179)(862.6645943,527.50579)(863.83792763,527.50579)
\curveto(864.8405943,527.50579)(865.72592763,527.63379)(866.49392763,527.88979)
\curveto(867.26192763,528.16712333)(868.0085943,528.54045667)(868.73392763,529.00979)
\lineto(868.73392763,524.94579)
\curveto(868.0085943,524.47645667)(867.2405943,524.14579)(866.42992763,523.95379)
\curveto(865.6405943,523.74045667)(864.63792763,523.63379)(863.42192763,523.63379)
\closepath
}
}
{
\newrgbcolor{curcolor}{0 0 0}
\pscustom[linestyle=none,fillstyle=solid,fillcolor=curcolor]
{
\newpath
\moveto(888.63792568,532.72179)
\curveto(888.63792568,529.82045667)(887.86992568,527.58045667)(886.33392568,526.00179)
\curveto(884.81925901,524.42312333)(882.74992568,523.63379)(880.12592568,523.63379)
\curveto(878.50459234,523.63379)(877.05392568,523.98579)(875.77392568,524.68979)
\curveto(874.51525901,525.39379)(873.52325901,526.41779)(872.79792568,527.76179)
\curveto(872.07259234,529.12712333)(871.70992568,530.78045667)(871.70992568,532.72179)
\curveto(871.70992568,535.62312333)(872.46725901,537.85245667)(873.98192568,539.40979)
\curveto(875.49659234,540.96712333)(877.57659234,541.74579)(880.22192568,541.74579)
\curveto(881.86459234,541.74579)(883.31525901,541.39379)(884.57392568,540.68979)
\curveto(885.83259234,539.98579)(886.82459234,538.96179)(887.54992568,537.61779)
\curveto(888.27525901,536.27379)(888.63792568,534.64179)(888.63792568,532.72179)
\closepath
\moveto(876.57392568,532.72179)
\curveto(876.57392568,530.99379)(876.85125901,529.68179)(877.40592568,528.78579)
\curveto(877.98192568,527.91112333)(878.90992568,527.47379)(880.18992568,527.47379)
\curveto(881.44859234,527.47379)(882.35525901,527.91112333)(882.90992568,528.78579)
\curveto(883.48592568,529.68179)(883.77392568,530.99379)(883.77392568,532.72179)
\curveto(883.77392568,534.44979)(883.48592568,535.74045667)(882.90992568,536.59379)
\curveto(882.35525901,537.46845667)(881.43792568,537.90579)(880.15792568,537.90579)
\curveto(878.89925901,537.90579)(877.98192568,537.46845667)(877.40592568,536.59379)
\curveto(876.85125901,535.74045667)(876.57392568,534.44979)(876.57392568,532.72179)
\closepath
}
}
{
\newrgbcolor{curcolor}{0 0 0}
\pscustom[linestyle=none,fillstyle=solid,fillcolor=curcolor]
{
\newpath
\moveto(908.06190908,536.84979)
\curveto(908.06190908,535.91112333)(907.76324241,535.11112333)(907.16590907,534.44979)
\curveto(906.58990907,533.78845667)(905.72590908,533.36179)(904.57390907,533.16979)
\lineto(904.57390907,533.04179)
\curveto(905.78990908,532.89245667)(906.76057574,532.46579)(907.48590907,531.76179)
\curveto(908.23257574,531.07912333)(908.60590907,530.21512333)(908.60590907,529.16979)
\curveto(908.60590907,528.16712333)(908.33924241,527.27112333)(907.80590907,526.48179)
\curveto(907.29390907,525.69245667)(906.47257574,525.07379)(905.34190907,524.62579)
\curveto(904.21124241,524.17779)(902.72857574,523.95379)(900.89390908,523.95379)
\lineto(892.57390907,523.95379)
\lineto(892.57390907,541.42579)
\lineto(900.89390908,541.42579)
\curveto(902.25924241,541.42579)(903.47524241,541.27645667)(904.54190908,540.97779)
\curveto(905.62990907,540.70045667)(906.48324241,540.22045667)(907.10190907,539.53779)
\curveto(907.74190907,538.87645667)(908.06190908,537.98045667)(908.06190908,536.84979)
\closepath
\moveto(903.22990908,536.46579)
\curveto(903.22990907,537.53245667)(902.38724241,538.06579)(900.70190908,538.06579)
\lineto(897.34190907,538.06579)
\lineto(897.34190907,534.60979)
\lineto(900.15790908,534.60979)
\curveto(901.16057574,534.60979)(901.91790907,534.74845667)(902.42990907,535.02579)
\curveto(902.96324241,535.32445667)(903.22990908,535.80445667)(903.22990908,536.46579)
\closepath
\moveto(903.67790908,529.42579)
\curveto(903.67790908,530.10845667)(903.40057574,530.59912333)(902.84590908,530.89779)
\curveto(902.31257574,531.21779)(901.52324241,531.37779)(900.47790907,531.37779)
\lineto(897.34190907,531.37779)
\lineto(897.34190907,527.24979)
\lineto(900.57390907,527.24979)
\curveto(901.46990908,527.24979)(902.20590907,527.40979)(902.78190907,527.72979)
\curveto(903.37924241,528.07112333)(903.67790908,528.63645667)(903.67790908,529.42579)
\closepath
}
}
{
\newrgbcolor{curcolor}{0 0 0}
\pscustom[linestyle=none,fillstyle=solid,fillcolor=curcolor]
{
\newpath
\moveto(928.5738998,532.72179)
\curveto(928.5738998,529.82045667)(927.8058998,527.58045667)(926.2698998,526.00179)
\curveto(924.75523313,524.42312333)(922.6858998,523.63379)(920.0618998,523.63379)
\curveto(918.44056646,523.63379)(916.9898998,523.98579)(915.7098998,524.68979)
\curveto(914.45123313,525.39379)(913.45923313,526.41779)(912.7338998,527.76179)
\curveto(912.00856646,529.12712333)(911.6458998,530.78045667)(911.6458998,532.72179)
\curveto(911.6458998,535.62312333)(912.40323313,537.85245667)(913.9178998,539.40979)
\curveto(915.43256646,540.96712333)(917.51256646,541.74579)(920.1578998,541.74579)
\curveto(921.80056646,541.74579)(923.25123313,541.39379)(924.5098998,540.68979)
\curveto(925.76856646,539.98579)(926.76056646,538.96179)(927.4858998,537.61779)
\curveto(928.21123313,536.27379)(928.5738998,534.64179)(928.5738998,532.72179)
\closepath
\moveto(916.5098998,532.72179)
\curveto(916.5098998,530.99379)(916.78723313,529.68179)(917.3418998,528.78579)
\curveto(917.9178998,527.91112333)(918.8458998,527.47379)(920.1258998,527.47379)
\curveto(921.38456646,527.47379)(922.29123313,527.91112333)(922.8458998,528.78579)
\curveto(923.4218998,529.68179)(923.7098998,530.99379)(923.7098998,532.72179)
\curveto(923.7098998,534.44979)(923.4218998,535.74045667)(922.8458998,536.59379)
\curveto(922.29123313,537.46845667)(921.3738998,537.90579)(920.0938998,537.90579)
\curveto(918.83523313,537.90579)(917.9178998,537.46845667)(917.3418998,536.59379)
\curveto(916.78723313,535.74045667)(916.5098998,534.44979)(916.5098998,532.72179)
\closepath
}
}
{
\newrgbcolor{curcolor}{0 0 0}
\pscustom[linestyle=none,fillstyle=solid,fillcolor=curcolor]
{
\newpath
\moveto(937.1178832,541.42579)
\lineto(937.1178832,534.51379)
\curveto(937.1178832,534.15112333)(937.09654986,533.70312333)(937.0538832,533.16979)
\curveto(937.03254986,532.63645667)(937.00054986,532.09245667)(936.9578832,531.53779)
\curveto(936.93654986,530.98312333)(936.90454986,530.48179)(936.8618832,530.03379)
\curveto(936.81921653,529.60712333)(936.78721653,529.31912333)(936.7658832,529.16979)
\lineto(944.8298832,541.42579)
\lineto(950.5578832,541.42579)
\lineto(950.5578832,523.95379)
\lineto(945.9498832,523.95379)
\lineto(945.9498832,530.92979)
\curveto(945.9498832,531.48445667)(945.97121653,532.11379)(946.0138832,532.81779)
\curveto(946.05654986,533.52179)(946.09921653,534.17245667)(946.1418832,534.76979)
\curveto(946.2058832,535.38845667)(946.24854986,535.85779)(946.2698832,536.17779)
\lineto(938.2378832,523.95379)
\lineto(932.5098832,523.95379)
\lineto(932.5098832,541.42579)
\closepath
\moveto(949.2458832,548.94579)
\curveto(949.13921653,547.83645667)(948.8298832,546.85512333)(948.3178832,546.00179)
\curveto(947.8058832,545.16979)(947.0058832,544.51912333)(945.9178832,544.04979)
\curveto(944.8298832,543.58045667)(943.36854986,543.34579)(941.5338832,543.34579)
\curveto(939.65654986,543.34579)(938.18454986,543.56979)(937.1178832,544.01779)
\curveto(936.07254986,544.46579)(935.3258832,545.10579)(934.8778832,545.93779)
\curveto(934.4298832,546.79112333)(934.16321653,547.79379)(934.0778832,548.94579)
\lineto(938.3338832,548.94579)
\curveto(938.41921653,547.77245667)(938.70721653,546.99379)(939.1978832,546.60979)
\curveto(939.7098832,546.22579)(940.52054986,546.03379)(941.6298832,546.03379)
\curveto(942.54721653,546.03379)(943.2938832,546.23645667)(943.8698832,546.64179)
\curveto(944.46721653,547.06845667)(944.81921653,547.83645667)(944.9258832,548.94579)
\closepath
}
}
{
\newrgbcolor{curcolor}{0 0 0}
\pscustom[linestyle=none,fillstyle=solid,fillcolor=curcolor]
{
\newpath
\moveto(829.26187319,501.42579)
\lineto(834.50987319,501.42579)
\lineto(827.59787319,493.04179)
\lineto(835.11787319,483.95379)
\lineto(829.70987319,483.95379)
\lineto(822.57387319,492.81779)
\lineto(822.57387319,483.95379)
\lineto(817.80587319,483.95379)
\lineto(817.80587319,501.42579)
\lineto(822.57387319,501.42579)
\lineto(822.57387319,492.94579)
\closepath
}
}
{
\newrgbcolor{curcolor}{0 0 0}
\pscustom[linestyle=none,fillstyle=solid,fillcolor=curcolor]
{
\newpath
\moveto(844.78185658,501.77779)
\curveto(847.12852325,501.77779)(848.92052325,501.26579)(850.15785658,500.24179)
\curveto(851.41652325,499.23912333)(852.04585658,497.69245667)(852.04585658,495.60179)
\lineto(852.04585658,483.95379)
\lineto(848.71785658,483.95379)
\lineto(847.78985658,486.32179)
\lineto(847.66185658,486.32179)
\curveto(846.91518992,485.38312333)(846.12585658,484.70045667)(845.29385658,484.27379)
\curveto(844.46185658,483.84712333)(843.32052325,483.63379)(841.86985658,483.63379)
\curveto(840.31252325,483.63379)(839.02185658,484.08179)(837.99785658,484.97779)
\curveto(836.97385658,485.87379)(836.46185658,487.27112333)(836.46185658,489.16979)
\curveto(836.46185658,491.02579)(837.11252325,492.39112333)(838.41385658,493.26579)
\curveto(839.71518992,494.14045667)(841.66718992,494.63112333)(844.26985658,494.73779)
\lineto(847.30985658,494.83379)
\lineto(847.30985658,495.60179)
\curveto(847.30985658,496.51912333)(847.06452325,497.19112333)(846.57385658,497.61779)
\curveto(846.10452325,498.04445667)(845.44318992,498.25779)(844.58985658,498.25779)
\curveto(843.73652325,498.25779)(842.90452325,498.12979)(842.09385658,497.87379)
\curveto(841.28318992,497.63912333)(840.47252325,497.34045667)(839.66185658,496.97779)
\lineto(838.09385658,500.20979)
\curveto(839.01118992,500.67912333)(840.04585658,501.05245667)(841.19785658,501.32979)
\curveto(842.34985658,501.62845667)(843.54452325,501.77779)(844.78185658,501.77779)
\closepath
\moveto(847.30985658,492.04979)
\lineto(845.45385658,491.98579)
\curveto(843.91785658,491.94312333)(842.85118992,491.66579)(842.25385658,491.15379)
\curveto(841.65652325,490.64179)(841.35785658,489.96979)(841.35785658,489.13779)
\curveto(841.35785658,488.41245667)(841.57118992,487.88979)(841.99785658,487.56979)
\curveto(842.42452325,487.27112333)(842.97918992,487.12179)(843.66185658,487.12179)
\curveto(844.68585658,487.12179)(845.54985658,487.42045667)(846.25385658,488.01779)
\curveto(846.95785658,488.63645667)(847.30985658,489.50045667)(847.30985658,490.60979)
\closepath
}
}
{
\newrgbcolor{curcolor}{0 0 0}
\pscustom[linestyle=none,fillstyle=solid,fillcolor=curcolor]
{
\newpath
\moveto(861.70985951,501.42579)
\lineto(861.70985951,494.70579)
\lineto(868.36585951,494.70579)
\lineto(868.36585951,501.42579)
\lineto(873.13385951,501.42579)
\lineto(873.13385951,483.95379)
\lineto(868.36585951,483.95379)
\lineto(868.36585951,491.15379)
\lineto(861.70985951,491.15379)
\lineto(861.70985951,483.95379)
\lineto(856.94185951,483.95379)
\lineto(856.94185951,501.42579)
\closepath
}
}
{
\newrgbcolor{curcolor}{0 0 0}
\pscustom[linestyle=none,fillstyle=solid,fillcolor=curcolor]
{
\newpath
\moveto(885.29388051,501.77779)
\curveto(887.64054718,501.77779)(889.43254718,501.26579)(890.66988051,500.24179)
\curveto(891.92854718,499.23912333)(892.55788051,497.69245667)(892.55788051,495.60179)
\lineto(892.55788051,483.95379)
\lineto(889.22988051,483.95379)
\lineto(888.30188051,486.32179)
\lineto(888.17388051,486.32179)
\curveto(887.42721384,485.38312333)(886.63788051,484.70045667)(885.80588051,484.27379)
\curveto(884.97388051,483.84712333)(883.83254718,483.63379)(882.38188051,483.63379)
\curveto(880.82454718,483.63379)(879.53388051,484.08179)(878.50988051,484.97779)
\curveto(877.48588051,485.87379)(876.97388051,487.27112333)(876.97388051,489.16979)
\curveto(876.97388051,491.02579)(877.62454718,492.39112333)(878.92588051,493.26579)
\curveto(880.22721384,494.14045667)(882.17921384,494.63112333)(884.78188051,494.73779)
\lineto(887.82188051,494.83379)
\lineto(887.82188051,495.60179)
\curveto(887.82188051,496.51912333)(887.57654718,497.19112333)(887.08588051,497.61779)
\curveto(886.61654718,498.04445667)(885.95521384,498.25779)(885.10188051,498.25779)
\curveto(884.24854718,498.25779)(883.41654718,498.12979)(882.60588051,497.87379)
\curveto(881.79521384,497.63912333)(880.98454718,497.34045667)(880.17388051,496.97779)
\lineto(878.60588051,500.20979)
\curveto(879.52321384,500.67912333)(880.55788051,501.05245667)(881.70988051,501.32979)
\curveto(882.86188051,501.62845667)(884.05654718,501.77779)(885.29388051,501.77779)
\closepath
\moveto(887.82188051,492.04979)
\lineto(885.96588051,491.98579)
\curveto(884.42988051,491.94312333)(883.36321384,491.66579)(882.76588051,491.15379)
\curveto(882.16854718,490.64179)(881.86988051,489.96979)(881.86988051,489.13779)
\curveto(881.86988051,488.41245667)(882.08321384,487.88979)(882.50988051,487.56979)
\curveto(882.93654718,487.27112333)(883.49121384,487.12179)(884.17388051,487.12179)
\curveto(885.19788051,487.12179)(886.06188051,487.42045667)(886.76588051,488.01779)
\curveto(887.46988051,488.63645667)(887.82188051,489.50045667)(887.82188051,490.60979)
\closepath
}
}
{
\newrgbcolor{curcolor}{0 0 0}
\pscustom[linestyle=none,fillstyle=solid,fillcolor=curcolor]
{
\newpath
\moveto(913.10188344,483.95379)
\lineto(908.33388344,483.95379)
\lineto(908.33388344,497.84179)
\lineto(903.94988344,497.84179)
\curveto(903.67255011,494.42845667)(903.29921677,491.67645667)(902.82988344,489.58579)
\curveto(902.38188344,487.51645667)(901.74188344,486.00179)(900.90988344,485.04179)
\curveto(900.09921677,484.10312333)(899.02188344,483.63379)(897.67788344,483.63379)
\curveto(896.56855011,483.63379)(895.66188344,483.80445667)(894.95788344,484.14579)
\lineto(894.95788344,487.95379)
\curveto(895.44855011,487.74045667)(895.96055011,487.63379)(896.49388344,487.63379)
\curveto(896.87788344,487.63379)(897.22988344,487.82579)(897.54988344,488.20979)
\curveto(897.86988344,488.59379)(898.16855011,489.28712333)(898.44588344,490.28979)
\curveto(898.74455011,491.29245667)(899.01121677,492.68979)(899.24588344,494.48179)
\curveto(899.48055011,496.29512333)(899.69388344,498.60979)(899.88588344,501.42579)
\lineto(913.10188344,501.42579)
\closepath
}
}
{
\newrgbcolor{curcolor}{0 0 0}
\pscustom[linestyle=none,fillstyle=solid,fillcolor=curcolor]
{
\newpath
\moveto(870.75501978,228.78227088)
\lineto(865.92301978,228.78227088)
\lineto(865.92301978,237.61427088)
\curveto(864.57901978,237.14493754)(863.34168644,236.79293754)(862.21101978,236.55827088)
\curveto(861.10168644,236.32360421)(859.98168644,236.20627088)(858.85101978,236.20627088)
\curveto(856.71768644,236.20627088)(855.04301978,236.71827088)(853.82701978,237.74227088)
\curveto(852.63235311,238.78760421)(852.03501978,240.27027088)(852.03501978,242.19027088)
\lineto(852.03501978,251.63027088)
\lineto(856.86701978,251.63027088)
\lineto(856.86701978,243.56627088)
\curveto(856.86701978,242.45693754)(857.12301978,241.62493754)(857.63501978,241.07027088)
\curveto(858.14701978,240.51560421)(859.01101978,240.23827088)(860.22701978,240.23827088)
\curveto(861.12301978,240.23827088)(862.01901978,240.33427088)(862.91501978,240.52627088)
\curveto(863.81101978,240.71827088)(864.81368644,241.00627088)(865.92301978,241.39027088)
\lineto(865.92301978,251.63027088)
\lineto(870.75501978,251.63027088)
\closepath
}
}
{
\newrgbcolor{curcolor}{0 0 0}
\pscustom[linestyle=none,fillstyle=solid,fillcolor=curcolor]
{
\newpath
\moveto(883.29904028,246.60627088)
\curveto(885.64570695,246.60627088)(887.43770695,246.09427088)(888.67504028,245.07027088)
\curveto(889.93370695,244.06760421)(890.56304028,242.52093754)(890.56304028,240.43027088)
\lineto(890.56304028,228.78227088)
\lineto(887.23504028,228.78227088)
\lineto(886.30704028,231.15027088)
\lineto(886.17904028,231.15027088)
\curveto(885.43237362,230.21160421)(884.64304028,229.52893754)(883.81104028,229.10227088)
\curveto(882.97904028,228.67560421)(881.83770695,228.46227088)(880.38704028,228.46227088)
\curveto(878.82970695,228.46227088)(877.53904028,228.91027088)(876.51504028,229.80627088)
\curveto(875.49104028,230.70227088)(874.97904028,232.09960421)(874.97904028,233.99827088)
\curveto(874.97904028,235.85427088)(875.62970695,237.21960421)(876.93104028,238.09427088)
\curveto(878.23237362,238.96893754)(880.18437362,239.45960421)(882.78704028,239.56627088)
\lineto(885.82704028,239.66227088)
\lineto(885.82704028,240.43027088)
\curveto(885.82704028,241.34760421)(885.58170695,242.01960421)(885.09104028,242.44627088)
\curveto(884.62170695,242.87293754)(883.96037362,243.08627088)(883.10704028,243.08627088)
\curveto(882.25370695,243.08627088)(881.42170695,242.95827088)(880.61104028,242.70227088)
\curveto(879.80037362,242.46760421)(878.98970695,242.16893754)(878.17904028,241.80627088)
\lineto(876.61104028,245.03827088)
\curveto(877.52837362,245.50760421)(878.56304028,245.88093754)(879.71504028,246.15827088)
\curveto(880.86704028,246.45693754)(882.06170695,246.60627088)(883.29904028,246.60627088)
\closepath
\moveto(885.82704028,236.87827088)
\lineto(883.97104028,236.81427088)
\curveto(882.43504028,236.77160421)(881.36837362,236.49427088)(880.77104028,235.98227088)
\curveto(880.17370695,235.47027088)(879.87504028,234.79827088)(879.87504028,233.96627088)
\curveto(879.87504028,233.24093754)(880.08837362,232.71827088)(880.51504028,232.39827088)
\curveto(880.94170695,232.09960421)(881.49637362,231.95027088)(882.17904028,231.95027088)
\curveto(883.20304028,231.95027088)(884.06704028,232.24893754)(884.77104028,232.84627088)
\curveto(885.47504028,233.46493754)(885.82704028,234.32893754)(885.82704028,235.43827088)
\closepath
}
}
{
\newrgbcolor{curcolor}{0 0 0}
\pscustom[linestyle=none,fillstyle=solid,fillcolor=curcolor]
{
\newpath
\moveto(909.92304321,242.67027088)
\lineto(904.19504321,242.67027088)
\lineto(904.19504321,228.78227088)
\lineto(899.42704321,228.78227088)
\lineto(899.42704321,242.67027088)
\lineto(893.69904321,242.67027088)
\lineto(893.69904321,246.25427088)
\lineto(909.92304321,246.25427088)
\closepath
}
}
{
\newrgbcolor{curcolor}{0 0 0}
\pscustom[linestyle=none,fillstyle=solid,fillcolor=curcolor]
{
\newpath
\moveto(470.85849448,157.06069213)
\lineto(475.21049448,157.06069213)
\lineto(475.21049448,146.02069213)
\curveto(475.21049448,145.48735879)(475.19982781,144.89002546)(475.17849448,144.22869213)
\curveto(475.15716115,143.56735879)(475.13582781,142.91669213)(475.11449448,142.27669213)
\curveto(475.09316115,141.65802546)(475.07182781,141.11402546)(475.05049448,140.64469213)
\curveto(475.02916115,140.19669213)(475.00782781,139.88735879)(474.98649448,139.71669213)
\lineto(475.08249448,139.71669213)
\lineto(485.64249448,157.06069213)
\lineto(491.43449448,157.06069213)
\lineto(491.43449448,134.21269213)
\lineto(487.11449448,134.21269213)
\lineto(487.11449448,145.18869213)
\curveto(487.11449448,145.76469213)(487.12516115,146.39402546)(487.14649448,147.07669213)
\curveto(487.16782781,147.78069213)(487.18916115,148.45269213)(487.21049448,149.09269213)
\curveto(487.25316115,149.73269213)(487.28516115,150.28735879)(487.30649448,150.75669213)
\curveto(487.34916115,151.24735879)(487.38116115,151.56735879)(487.40249448,151.71669213)
\lineto(487.27449448,151.71669213)
\lineto(476.68249448,134.21269213)
\lineto(470.85849448,134.21269213)
\closepath
}
}
{
\newrgbcolor{curcolor}{0 0 0}
\pscustom[linestyle=none,fillstyle=solid,fillcolor=curcolor]
{
\newpath
\moveto(501.57849301,151.68469213)
\lineto(501.57849301,144.96469213)
\lineto(508.23449301,144.96469213)
\lineto(508.23449301,151.68469213)
\lineto(513.00249301,151.68469213)
\lineto(513.00249301,134.21269213)
\lineto(508.23449301,134.21269213)
\lineto(508.23449301,141.41269213)
\lineto(501.57849301,141.41269213)
\lineto(501.57849301,134.21269213)
\lineto(496.81049301,134.21269213)
\lineto(496.81049301,151.68469213)
\closepath
}
}
{
\newrgbcolor{curcolor}{0 0 0}
\pscustom[linestyle=none,fillstyle=solid,fillcolor=curcolor]
{
\newpath
\moveto(525.09851401,133.89269213)
\curveto(522.49584734,133.89269213)(520.47984734,134.60735879)(519.05051401,136.03669213)
\curveto(517.64251401,137.46602546)(516.93851401,139.73802546)(516.93851401,142.85269213)
\curveto(516.93851401,144.98602546)(517.30118068,146.72469213)(518.02651401,148.06869213)
\curveto(518.75184734,149.41269213)(519.75451401,150.40469213)(521.03451401,151.04469213)
\curveto(522.33584734,151.68469213)(523.82918068,152.00469213)(525.51451401,152.00469213)
\curveto(526.70918068,152.00469213)(527.74384734,151.88735879)(528.61851401,151.65269213)
\curveto(529.51451401,151.41802546)(530.29318068,151.14069213)(530.95451401,150.82069213)
\lineto(529.54651401,147.14069213)
\curveto(528.79984734,147.43935879)(528.09584734,147.68469213)(527.43451401,147.87669213)
\curveto(526.79451401,148.06869213)(526.15451401,148.16469213)(525.51451401,148.16469213)
\curveto(523.03984734,148.16469213)(521.80251401,146.40469213)(521.80251401,142.88469213)
\curveto(521.80251401,141.13535879)(522.12251401,139.84469213)(522.76251401,139.01269213)
\curveto(523.42384734,138.18069213)(524.34118068,137.76469213)(525.51451401,137.76469213)
\curveto(526.51718068,137.76469213)(527.40251401,137.89269213)(528.17051401,138.14869213)
\curveto(528.93851401,138.42602546)(529.68518068,138.79935879)(530.41051401,139.26869213)
\lineto(530.41051401,135.20469213)
\curveto(529.68518068,134.73535879)(528.91718068,134.40469213)(528.10651401,134.21269213)
\curveto(527.31718068,133.99935879)(526.31451401,133.89269213)(525.09851401,133.89269213)
\closepath
}
}
{
\newrgbcolor{curcolor}{0 0 0}
\pscustom[linestyle=none,fillstyle=solid,fillcolor=curcolor]
{
\newpath
\moveto(548.90651206,148.10069213)
\lineto(543.17851206,148.10069213)
\lineto(543.17851206,134.21269213)
\lineto(538.41051206,134.21269213)
\lineto(538.41051206,148.10069213)
\lineto(532.68251206,148.10069213)
\lineto(532.68251206,151.68469213)
\lineto(548.90651206,151.68469213)
\closepath
}
}
{
\newrgbcolor{curcolor}{0 0 0}
\pscustom[linestyle=none,fillstyle=solid,fillcolor=curcolor]
{
\newpath
\moveto(561.86649594,152.00469213)
\curveto(563.82916261,152.00469213)(565.41849594,151.23669213)(566.63449594,149.70069213)
\curveto(567.85049594,148.18602546)(568.45849594,145.94602546)(568.45849594,142.98069213)
\curveto(568.45849594,139.99402546)(567.82916261,137.73269213)(566.57049594,136.19669213)
\curveto(565.31182928,134.66069213)(563.70116261,133.89269213)(561.73849594,133.89269213)
\curveto(560.47982928,133.89269213)(559.47716261,134.11669213)(558.73049594,134.56469213)
\curveto(557.98382928,135.03402546)(557.37582928,135.55669213)(556.90649594,136.13269213)
\lineto(556.65049594,136.13269213)
\curveto(556.82116261,135.23669213)(556.90649594,134.38335879)(556.90649594,133.57269213)
\lineto(556.90649594,126.53269213)
\lineto(552.13849594,126.53269213)
\lineto(552.13849594,151.68469213)
\lineto(556.01049594,151.68469213)
\lineto(556.68249594,149.41269213)
\lineto(556.90649594,149.41269213)
\curveto(557.37582928,150.11669213)(558.00516261,150.72469213)(558.79449594,151.23669213)
\curveto(559.58382928,151.74869213)(560.60782928,152.00469213)(561.86649594,152.00469213)
\closepath
\moveto(560.33049594,148.19669213)
\curveto(559.09316261,148.19669213)(558.21849594,147.80202546)(557.70649594,147.01269213)
\curveto(557.19449594,146.24469213)(556.92782928,145.08202546)(556.90649594,143.52469213)
\lineto(556.90649594,143.01269213)
\curveto(556.90649594,141.32735879)(557.15182928,140.02602546)(557.64249594,139.10869213)
\curveto(558.15449594,138.21269213)(559.07182928,137.76469213)(560.39449594,137.76469213)
\curveto(561.48249594,137.76469213)(562.28249594,138.21269213)(562.79449594,139.10869213)
\curveto(563.32782928,140.02602546)(563.59449594,141.33802546)(563.59449594,143.04469213)
\curveto(563.59449594,146.47935879)(562.50649594,148.19669213)(560.33049594,148.19669213)
\closepath
}
}
{
\newrgbcolor{curcolor}{0 0 0}
\pscustom[linestyle=none,fillstyle=solid,fillcolor=curcolor]
{
\newpath
\moveto(569.57847007,151.68469213)
\lineto(574.79447007,151.68469213)
\lineto(578.09047007,141.86069213)
\curveto(578.26113673,141.37002546)(578.38913673,140.87935879)(578.47447007,140.38869213)
\curveto(578.5598034,139.89802546)(578.6238034,139.37535879)(578.66647007,138.82069213)
\lineto(578.76247007,138.82069213)
\curveto(578.82647007,139.37535879)(578.9118034,139.89802546)(579.01847007,140.38869213)
\curveto(579.12513673,140.87935879)(579.2638034,141.37002546)(579.43447007,141.86069213)
\lineto(582.66647007,151.68469213)
\lineto(587.78647007,151.68469213)
\lineto(580.39447007,131.97269213)
\curveto(579.7118034,130.15935879)(578.74113673,128.80469213)(577.48247007,127.90869213)
\curveto(576.2238034,126.99135879)(574.76247007,126.53269213)(573.09847007,126.53269213)
\curveto(572.5438034,126.53269213)(572.07447007,126.56469213)(571.69047007,126.62869213)
\curveto(571.30647007,126.67135879)(570.96513673,126.72469213)(570.66647007,126.78869213)
\lineto(570.66647007,130.56469213)
\curveto(570.8798034,130.52202546)(571.15713673,130.47935879)(571.49847007,130.43669213)
\curveto(571.8398034,130.39402546)(572.1918034,130.37269213)(572.55447007,130.37269213)
\curveto(573.55713673,130.37269213)(574.34647007,130.68202546)(574.92247007,131.30069213)
\curveto(575.49847007,131.89802546)(575.9358034,132.62335879)(576.23447007,133.47669213)
\lineto(576.52247007,134.34069213)
\closepath
}
}
{
\newrgbcolor{curcolor}{0 0 0}
\pscustom[linestyle=none,fillstyle=solid,fillcolor=curcolor]
{
\newpath
\moveto(612.29847788,151.68469213)
\lineto(612.29847788,134.21269213)
\lineto(607.85047788,134.21269213)
\lineto(607.85047788,142.78869213)
\curveto(607.85047788,143.64202546)(607.86114454,144.47402546)(607.88247788,145.28469213)
\curveto(607.92514454,146.09535879)(607.97847788,146.84202546)(608.04247788,147.52469213)
\lineto(607.94647788,147.52469213)
\lineto(603.11447788,134.21269213)
\lineto(599.53047788,134.21269213)
\lineto(594.63447788,147.55669213)
\lineto(594.50647788,147.55669213)
\curveto(594.59181121,146.85269213)(594.64514454,146.09535879)(594.66647788,145.28469213)
\curveto(594.70914454,144.49535879)(594.73047788,143.62069213)(594.73047788,142.66069213)
\lineto(594.73047788,134.21269213)
\lineto(590.28247788,134.21269213)
\lineto(590.28247788,151.68469213)
\lineto(597.03447788,151.68469213)
\lineto(601.38647788,139.84469213)
\lineto(605.80247788,151.68469213)
\closepath
}
}
{
\newrgbcolor{curcolor}{0 0 0}
\pscustom[linestyle=none,fillstyle=solid,fillcolor=curcolor]
{
\newpath
\moveto(624.49047348,152.00469213)
\curveto(626.90114015,152.00469213)(628.81047348,151.31135879)(630.21847348,149.92469213)
\curveto(631.62647348,148.55935879)(632.33047348,146.60735879)(632.33047348,144.06869213)
\lineto(632.33047348,141.76469213)
\lineto(621.06647348,141.76469213)
\curveto(621.10914015,140.42069213)(621.50380682,139.36469213)(622.25047348,138.59669213)
\curveto(623.01847348,137.82869213)(624.07447348,137.44469213)(625.41847348,137.44469213)
\curveto(626.52780682,137.44469213)(627.54114015,137.55135879)(628.45847348,137.76469213)
\curveto(629.39714015,137.99935879)(630.35714015,138.35135879)(631.33847348,138.82069213)
\lineto(631.33847348,135.14069213)
\curveto(630.46380682,134.71402546)(629.55714015,134.40469213)(628.61847348,134.21269213)
\curveto(627.67980682,133.99935879)(626.53847348,133.89269213)(625.19447348,133.89269213)
\curveto(623.44514015,133.89269213)(621.89847348,134.21269213)(620.55447348,134.85269213)
\curveto(619.21047348,135.51402546)(618.15447348,136.49535879)(617.38647348,137.79669213)
\curveto(616.61847348,139.11935879)(616.23447348,140.79402546)(616.23447348,142.82069213)
\curveto(616.23447348,144.84735879)(616.57580682,146.54335879)(617.25847348,147.90869213)
\curveto(617.96247348,149.27402546)(618.93314015,150.29802546)(620.17047348,150.98069213)
\curveto(621.40780682,151.66335879)(622.84780682,152.00469213)(624.49047348,152.00469213)
\closepath
\moveto(624.52247348,148.61269213)
\curveto(623.58380682,148.61269213)(622.81580682,148.31402546)(622.21847348,147.71669213)
\curveto(621.62114015,147.11935879)(621.26914015,146.19135879)(621.16247348,144.93269213)
\lineto(627.85047348,144.93269213)
\curveto(627.82914015,145.97802546)(627.54114015,146.85269213)(626.98647348,147.55669213)
\curveto(626.45314015,148.26069213)(625.63180682,148.61269213)(624.52247348,148.61269213)
\closepath
}
}
{
\newrgbcolor{curcolor}{0 0 0}
\pscustom[linestyle=none,fillstyle=solid,fillcolor=curcolor]
{
\newpath
\moveto(640.97046079,151.68469213)
\lineto(640.97046079,144.96469213)
\lineto(647.62646079,144.96469213)
\lineto(647.62646079,151.68469213)
\lineto(652.39446079,151.68469213)
\lineto(652.39446079,134.21269213)
\lineto(647.62646079,134.21269213)
\lineto(647.62646079,141.41269213)
\lineto(640.97046079,141.41269213)
\lineto(640.97046079,134.21269213)
\lineto(636.20246079,134.21269213)
\lineto(636.20246079,151.68469213)
\closepath
}
}
{
\newrgbcolor{curcolor}{0 0 0}
\pscustom[linestyle=none,fillstyle=solid,fillcolor=curcolor]
{
\newpath
\moveto(671.85048178,148.10069213)
\lineto(666.12248178,148.10069213)
\lineto(666.12248178,134.21269213)
\lineto(661.35448178,134.21269213)
\lineto(661.35448178,148.10069213)
\lineto(655.62648178,148.10069213)
\lineto(655.62648178,151.68469213)
\lineto(671.85048178,151.68469213)
\closepath
}
}
{
\newrgbcolor{curcolor}{0 0 0}
\pscustom[linestyle=none,fillstyle=solid,fillcolor=curcolor]
{
\newpath
\moveto(479.57849277,111.68469213)
\lineto(479.57849277,97.70069213)
\lineto(482.13849277,97.70069213)
\lineto(482.13849277,87.94069213)
\lineto(477.85049277,87.94069213)
\lineto(477.85049277,94.21269213)
\lineto(466.10649277,94.21269213)
\lineto(466.10649277,87.94069213)
\lineto(461.81849277,87.94069213)
\lineto(461.81849277,97.70069213)
\lineto(463.29049277,97.70069213)
\curveto(464.05849277,98.87402546)(464.70915944,100.20735879)(465.24249277,101.70069213)
\curveto(465.7758261,103.21535879)(466.20249277,104.81535879)(466.52249277,106.50069213)
\curveto(466.84249277,108.20735879)(467.07715944,109.93535879)(467.22649277,111.68469213)
\closepath
\moveto(474.81049277,108.10069213)
\lineto(471.22649277,108.10069213)
\curveto(470.97049277,106.15935879)(470.61849277,104.31402546)(470.17049277,102.56469213)
\curveto(469.72249277,100.83669213)(469.09315944,99.21535879)(468.28249277,97.70069213)
\lineto(474.81049277,97.70069213)
\closepath
}
}
{
\newrgbcolor{curcolor}{0 0 0}
\pscustom[linestyle=none,fillstyle=solid,fillcolor=curcolor]
{
\newpath
\moveto(500.76247861,94.21269213)
\lineto(495.99447861,94.21269213)
\lineto(495.99447861,108.10069213)
\lineto(491.61047861,108.10069213)
\curveto(491.33314528,104.68735879)(490.95981194,101.93535879)(490.49047861,99.84469213)
\curveto(490.04247861,97.77535879)(489.40247861,96.26069213)(488.57047861,95.30069213)
\curveto(487.75981194,94.36202546)(486.68247861,93.89269213)(485.33847861,93.89269213)
\curveto(484.22914528,93.89269213)(483.32247861,94.06335879)(482.61847861,94.40469213)
\lineto(482.61847861,98.21269213)
\curveto(483.10914528,97.99935879)(483.62114528,97.89269213)(484.15447861,97.89269213)
\curveto(484.53847861,97.89269213)(484.89047861,98.08469213)(485.21047861,98.46869213)
\curveto(485.53047861,98.85269213)(485.82914528,99.54602546)(486.10647861,100.54869213)
\curveto(486.40514528,101.55135879)(486.67181194,102.94869213)(486.90647861,104.74069213)
\curveto(487.14114528,106.55402546)(487.35447861,108.86869213)(487.54647861,111.68469213)
\lineto(500.76247861,111.68469213)
\closepath
}
}
{
\newrgbcolor{curcolor}{0 0 0}
\pscustom[linestyle=none,fillstyle=solid,fillcolor=curcolor]
{
\newpath
\moveto(508.41049326,94.21269213)
\lineto(503.25849326,94.21269213)
\lineto(507.96249326,101.12469213)
\curveto(507.06649326,101.48735879)(506.26649326,102.07402546)(505.56249326,102.88469213)
\curveto(504.87982659,103.71669213)(504.53849326,104.84735879)(504.53849326,106.27669213)
\curveto(504.53849326,108.02602546)(505.19982659,109.35935879)(506.52249326,110.27669213)
\curveto(507.84515993,111.21535879)(509.54115993,111.68469213)(511.61049326,111.68469213)
\lineto(519.73849326,111.68469213)
\lineto(519.73849326,94.21269213)
\lineto(514.97049326,94.21269213)
\lineto(514.97049326,100.70869213)
\lineto(512.34649326,100.70869213)
\closepath
\moveto(509.21049326,106.24469213)
\curveto(509.21049326,105.51935879)(509.49849326,104.94335879)(510.07449326,104.51669213)
\curveto(510.65049326,104.11135879)(511.39715993,103.90869213)(512.31449326,103.90869213)
\lineto(514.97049326,103.90869213)
\lineto(514.97049326,108.32469213)
\lineto(511.70649326,108.32469213)
\curveto(510.85315993,108.32469213)(510.22382659,108.11135879)(509.81849326,107.68469213)
\curveto(509.41315993,107.27935879)(509.21049326,106.79935879)(509.21049326,106.24469213)
\closepath
}
}
{
\newrgbcolor{curcolor}{0 0 0}
\pscustom[linestyle=none,fillstyle=solid,fillcolor=curcolor]
{
\newpath
\moveto(542.77851377,112.00469213)
\curveto(544.74118043,112.00469213)(546.33051377,111.23669213)(547.54651377,109.70069213)
\curveto(548.76251377,108.18602546)(549.37051377,105.94602546)(549.37051377,102.98069213)
\curveto(549.37051377,99.99402546)(548.74118043,97.73269213)(547.48251377,96.19669213)
\curveto(546.2238471,94.66069213)(544.61318043,93.89269213)(542.65051377,93.89269213)
\curveto(541.3918471,93.89269213)(540.38918043,94.11669213)(539.64251377,94.56469213)
\curveto(538.8958471,95.03402546)(538.2878471,95.55669213)(537.81851377,96.13269213)
\lineto(537.56251377,96.13269213)
\curveto(537.73318043,95.23669213)(537.81851377,94.38335879)(537.81851377,93.57269213)
\lineto(537.81851377,86.53269213)
\lineto(533.05051377,86.53269213)
\lineto(533.05051377,111.68469213)
\lineto(536.92251377,111.68469213)
\lineto(537.59451377,109.41269213)
\lineto(537.81851377,109.41269213)
\curveto(538.2878471,110.11669213)(538.91718043,110.72469213)(539.70651377,111.23669213)
\curveto(540.4958471,111.74869213)(541.5198471,112.00469213)(542.77851377,112.00469213)
\closepath
\moveto(541.24251377,108.19669213)
\curveto(540.00518043,108.19669213)(539.13051377,107.80202546)(538.61851377,107.01269213)
\curveto(538.10651377,106.24469213)(537.8398471,105.08202546)(537.81851377,103.52469213)
\lineto(537.81851377,103.01269213)
\curveto(537.81851377,101.32735879)(538.0638471,100.02602546)(538.55451377,99.10869213)
\curveto(539.06651377,98.21269213)(539.9838471,97.76469213)(541.30651377,97.76469213)
\curveto(542.39451377,97.76469213)(543.19451377,98.21269213)(543.70651377,99.10869213)
\curveto(544.2398471,100.02602546)(544.50651377,101.33802546)(544.50651377,103.04469213)
\curveto(544.50651377,106.47935879)(543.41851377,108.19669213)(541.24251377,108.19669213)
\closepath
}
}
{
\newrgbcolor{curcolor}{0 0 0}
\pscustom[linestyle=none,fillstyle=solid,fillcolor=curcolor]
{
\newpath
\moveto(560.47449521,112.03669213)
\curveto(562.82116188,112.03669213)(564.61316188,111.52469213)(565.85049521,110.50069213)
\curveto(567.10916188,109.49802546)(567.73849521,107.95135879)(567.73849521,105.86069213)
\lineto(567.73849521,94.21269213)
\lineto(564.41049521,94.21269213)
\lineto(563.48249521,96.58069213)
\lineto(563.35449521,96.58069213)
\curveto(562.60782855,95.64202546)(561.81849521,94.95935879)(560.98649521,94.53269213)
\curveto(560.15449521,94.10602546)(559.01316188,93.89269213)(557.56249521,93.89269213)
\curveto(556.00516188,93.89269213)(554.71449521,94.34069213)(553.69049521,95.23669213)
\curveto(552.66649521,96.13269213)(552.15449521,97.53002546)(552.15449521,99.42869213)
\curveto(552.15449521,101.28469213)(552.80516188,102.65002546)(554.10649521,103.52469213)
\curveto(555.40782855,104.39935879)(557.35982855,104.89002546)(559.96249521,104.99669213)
\lineto(563.00249521,105.09269213)
\lineto(563.00249521,105.86069213)
\curveto(563.00249521,106.77802546)(562.75716188,107.45002546)(562.26649521,107.87669213)
\curveto(561.79716188,108.30335879)(561.13582855,108.51669213)(560.28249521,108.51669213)
\curveto(559.42916188,108.51669213)(558.59716188,108.38869213)(557.78649521,108.13269213)
\curveto(556.97582855,107.89802546)(556.16516188,107.59935879)(555.35449521,107.23669213)
\lineto(553.78649521,110.46869213)
\curveto(554.70382855,110.93802546)(555.73849521,111.31135879)(556.89049521,111.58869213)
\curveto(558.04249521,111.88735879)(559.23716188,112.03669213)(560.47449521,112.03669213)
\closepath
\moveto(563.00249521,102.30869213)
\lineto(561.14649521,102.24469213)
\curveto(559.61049521,102.20202546)(558.54382855,101.92469213)(557.94649521,101.41269213)
\curveto(557.34916188,100.90069213)(557.05049521,100.22869213)(557.05049521,99.39669213)
\curveto(557.05049521,98.67135879)(557.26382855,98.14869213)(557.69049521,97.82869213)
\curveto(558.11716188,97.53002546)(558.67182855,97.38069213)(559.35449521,97.38069213)
\curveto(560.37849521,97.38069213)(561.24249521,97.67935879)(561.94649521,98.27669213)
\curveto(562.65049521,98.89535879)(563.00249521,99.75935879)(563.00249521,100.86869213)
\closepath
}
}
{
\newrgbcolor{curcolor}{0 0 0}
\pscustom[linestyle=none,fillstyle=solid,fillcolor=curcolor]
{
\newpath
\moveto(571.57849814,104.67669213)
\curveto(571.57849814,108.53802546)(572.30383147,111.53535879)(573.75449814,113.66869213)
\curveto(575.22649814,115.80202546)(577.64783147,117.16735879)(581.01849814,117.76469213)
\curveto(582.12783147,117.95669213)(583.26916481,118.11669213)(584.44249814,118.24469213)
\curveto(585.61583147,118.39402546)(586.82116481,118.54335879)(588.05849814,118.69269213)
\lineto(588.60249814,114.53269213)
\curveto(587.87716481,114.44735879)(587.07716481,114.35135879)(586.20249814,114.24469213)
\lineto(583.64249814,113.92469213)
\curveto(582.78916481,113.83935879)(582.04249814,113.74335879)(581.40249814,113.63669213)
\curveto(580.33583147,113.46602546)(579.45049814,113.19935879)(578.74649814,112.83669213)
\curveto(578.04249814,112.49535879)(577.49849814,111.94069213)(577.11449814,111.17269213)
\curveto(576.73049814,110.40469213)(576.50649814,109.29535879)(576.44249814,107.84469213)
\lineto(576.66649814,107.84469213)
\curveto(576.92249814,108.22869213)(577.27449814,108.62335879)(577.72249814,109.02869213)
\curveto(578.19183147,109.45535879)(578.75716481,109.80735879)(579.41849814,110.08469213)
\curveto(580.10116481,110.36202546)(580.89049814,110.50069213)(581.78649814,110.50069213)
\curveto(583.87716481,110.50069213)(585.53049814,109.85002546)(586.74649814,108.54869213)
\curveto(587.98383147,107.26869213)(588.60249814,105.37002546)(588.60249814,102.85269213)
\curveto(588.60249814,100.86869213)(588.23983147,99.20469213)(587.51449814,97.86069213)
\curveto(586.78916481,96.53802546)(585.78649814,95.54602546)(584.50649814,94.88469213)
\curveto(583.22649814,94.22335879)(581.74383147,93.89269213)(580.05849814,93.89269213)
\curveto(577.47716481,93.89269213)(575.41849814,94.82069213)(573.88249814,96.67669213)
\curveto(572.34649814,98.53269213)(571.57849814,101.19935879)(571.57849814,104.67669213)
\closepath
\moveto(580.34649814,97.76469213)
\curveto(581.34916481,97.76469213)(582.15983147,98.10602546)(582.77849814,98.78869213)
\curveto(583.41849814,99.47135879)(583.73849814,100.68735879)(583.73849814,102.43669213)
\curveto(583.73849814,103.82335879)(583.50383147,104.92202546)(583.03449814,105.73269213)
\curveto(582.58649814,106.56469213)(581.79716481,106.98069213)(580.66649814,106.98069213)
\curveto(579.98383147,106.98069213)(579.34383147,106.81002546)(578.74649814,106.46869213)
\curveto(578.17049814,106.14869213)(577.67983147,105.77535879)(577.27449814,105.34869213)
\curveto(576.86916481,104.92202546)(576.59183147,104.57002546)(576.44249814,104.29269213)
\curveto(576.44249814,103.20469213)(576.55983147,102.15935879)(576.79449814,101.15669213)
\curveto(577.02916481,100.15402546)(577.42383147,99.33269213)(577.97849814,98.69269213)
\curveto(578.55449814,98.07402546)(579.34383147,97.76469213)(580.34649814,97.76469213)
\closepath
}
}
{
\newrgbcolor{curcolor}{0 0 0}
\pscustom[linestyle=none,fillstyle=solid,fillcolor=curcolor]
{
\newpath
\moveto(608.41048984,102.98069213)
\curveto(608.41048984,100.07935879)(607.64248984,97.83935879)(606.10648984,96.26069213)
\curveto(604.59182317,94.68202546)(602.52248984,93.89269213)(599.89848984,93.89269213)
\curveto(598.27715651,93.89269213)(596.82648984,94.24469213)(595.54648984,94.94869213)
\curveto(594.28782317,95.65269213)(593.29582317,96.67669213)(592.57048984,98.02069213)
\curveto(591.84515651,99.38602546)(591.48248984,101.03935879)(591.48248984,102.98069213)
\curveto(591.48248984,105.88202546)(592.23982317,108.11135879)(593.75448984,109.66869213)
\curveto(595.26915651,111.22602546)(597.34915651,112.00469213)(599.99448984,112.00469213)
\curveto(601.63715651,112.00469213)(603.08782317,111.65269213)(604.34648984,110.94869213)
\curveto(605.60515651,110.24469213)(606.59715651,109.22069213)(607.32248984,107.87669213)
\curveto(608.04782317,106.53269213)(608.41048984,104.90069213)(608.41048984,102.98069213)
\closepath
\moveto(596.34648984,102.98069213)
\curveto(596.34648984,101.25269213)(596.62382317,99.94069213)(597.17848984,99.04469213)
\curveto(597.75448984,98.17002546)(598.68248984,97.73269213)(599.96248984,97.73269213)
\curveto(601.22115651,97.73269213)(602.12782317,98.17002546)(602.68248984,99.04469213)
\curveto(603.25848984,99.94069213)(603.54648984,101.25269213)(603.54648984,102.98069213)
\curveto(603.54648984,104.70869213)(603.25848984,105.99935879)(602.68248984,106.85269213)
\curveto(602.12782317,107.72735879)(601.21048984,108.16469213)(599.93048984,108.16469213)
\curveto(598.67182317,108.16469213)(597.75448984,107.72735879)(597.17848984,106.85269213)
\curveto(596.62382317,105.99935879)(596.34648984,104.70869213)(596.34648984,102.98069213)
\closepath
}
}
{
\newrgbcolor{curcolor}{0 0 0}
\pscustom[linestyle=none,fillstyle=solid,fillcolor=curcolor]
{
\newpath
\moveto(626.49046591,108.10069213)
\lineto(620.76246591,108.10069213)
\lineto(620.76246591,94.21269213)
\lineto(615.99446591,94.21269213)
\lineto(615.99446591,108.10069213)
\lineto(610.26646591,108.10069213)
\lineto(610.26646591,111.68469213)
\lineto(626.49046591,111.68469213)
\closepath
}
}
{
\newrgbcolor{curcolor}{0 0 0}
\pscustom[linestyle=none,fillstyle=solid,fillcolor=curcolor]
{
\newpath
\moveto(629.7224498,94.21269213)
\lineto(629.7224498,111.68469213)
\lineto(634.4904498,111.68469213)
\lineto(634.4904498,104.93269213)
\lineto(636.7944498,104.93269213)
\curveto(639.46111647,104.93269213)(641.4344498,104.50602546)(642.7144498,103.65269213)
\curveto(643.9944498,102.79935879)(644.6344498,101.50869213)(644.6344498,99.78069213)
\curveto(644.6344498,98.07402546)(644.03711647,96.71935879)(642.8424498,95.71669213)
\curveto(641.64778313,94.71402546)(639.68511647,94.21269213)(636.9544498,94.21269213)
\closepath
\moveto(647.1624498,94.21269213)
\lineto(647.1624498,111.68469213)
\lineto(651.9304498,111.68469213)
\lineto(651.9304498,94.21269213)
\closepath
\moveto(634.4904498,97.50869213)
\lineto(636.6984498,97.50869213)
\curveto(637.63711647,97.50869213)(638.3944498,97.66869213)(638.9704498,97.98869213)
\curveto(639.56778313,98.33002546)(639.8664498,98.90602546)(639.8664498,99.71669213)
\curveto(639.8664498,100.99669213)(638.78911647,101.63669213)(636.6344498,101.63669213)
\lineto(634.4904498,101.63669213)
\closepath
}
}
{
\newrgbcolor{curcolor}{0 0 0}
\pscustom[linestyle=none,fillstyle=solid,fillcolor=curcolor]
{
\newpath
\moveto(672.34646933,93.89269213)
\curveto(669.74380267,93.89269213)(667.72780267,94.60735879)(666.29846933,96.03669213)
\curveto(664.89046933,97.46602546)(664.18646933,99.73802546)(664.18646933,102.85269213)
\curveto(664.18646933,104.98602546)(664.549136,106.72469213)(665.27446933,108.06869213)
\curveto(665.99980267,109.41269213)(667.00246933,110.40469213)(668.28246933,111.04469213)
\curveto(669.58380267,111.68469213)(671.077136,112.00469213)(672.76246933,112.00469213)
\curveto(673.957136,112.00469213)(674.99180267,111.88735879)(675.86646933,111.65269213)
\curveto(676.76246933,111.41802546)(677.541136,111.14069213)(678.20246933,110.82069213)
\lineto(676.79446933,107.14069213)
\curveto(676.04780267,107.43935879)(675.34380267,107.68469213)(674.68246933,107.87669213)
\curveto(674.04246933,108.06869213)(673.40246933,108.16469213)(672.76246933,108.16469213)
\curveto(670.28780267,108.16469213)(669.05046933,106.40469213)(669.05046933,102.88469213)
\curveto(669.05046933,101.13535879)(669.37046933,99.84469213)(670.01046933,99.01269213)
\curveto(670.67180267,98.18069213)(671.589136,97.76469213)(672.76246933,97.76469213)
\curveto(673.765136,97.76469213)(674.65046933,97.89269213)(675.41846933,98.14869213)
\curveto(676.18646933,98.42602546)(676.933136,98.79935879)(677.65846933,99.26869213)
\lineto(677.65846933,95.20469213)
\curveto(676.933136,94.73535879)(676.165136,94.40469213)(675.35446933,94.21269213)
\curveto(674.565136,93.99935879)(673.56246933,93.89269213)(672.34646933,93.89269213)
\closepath
}
}
{
\newrgbcolor{curcolor}{0 0 0}
\pscustom[linestyle=none,fillstyle=solid,fillcolor=curcolor]
{
\newpath
\moveto(483.56254233,71.68469213)
\lineto(483.56254233,68.10069213)
\lineto(476.26654233,68.10069213)
\lineto(476.26654233,54.21269213)
\lineto(471.49854233,54.21269213)
\lineto(471.49854233,71.68469213)
\closepath
}
}
{
\newrgbcolor{curcolor}{0 0 0}
\pscustom[linestyle=none,fillstyle=solid,fillcolor=curcolor]
{
\newpath
\moveto(493.67455551,72.00469213)
\curveto(496.08522218,72.00469213)(497.99455551,71.31135879)(499.40255551,69.92469213)
\curveto(500.81055551,68.55935879)(501.51455551,66.60735879)(501.51455551,64.06869213)
\lineto(501.51455551,61.76469213)
\lineto(490.25055551,61.76469213)
\curveto(490.29322218,60.42069213)(490.68788885,59.36469213)(491.43455551,58.59669213)
\curveto(492.20255551,57.82869213)(493.25855551,57.44469213)(494.60255551,57.44469213)
\curveto(495.71188885,57.44469213)(496.72522218,57.55135879)(497.64255551,57.76469213)
\curveto(498.58122218,57.99935879)(499.54122218,58.35135879)(500.52255551,58.82069213)
\lineto(500.52255551,55.14069213)
\curveto(499.64788885,54.71402546)(498.74122218,54.40469213)(497.80255551,54.21269213)
\curveto(496.86388885,53.99935879)(495.72255551,53.89269213)(494.37855551,53.89269213)
\curveto(492.62922218,53.89269213)(491.08255551,54.21269213)(489.73855551,54.85269213)
\curveto(488.39455551,55.51402546)(487.33855551,56.49535879)(486.57055551,57.79669213)
\curveto(485.80255551,59.11935879)(485.41855551,60.79402546)(485.41855551,62.82069213)
\curveto(485.41855551,64.84735879)(485.75988885,66.54335879)(486.44255551,67.90869213)
\curveto(487.14655551,69.27402546)(488.11722218,70.29802546)(489.35455551,70.98069213)
\curveto(490.59188885,71.66335879)(492.03188885,72.00469213)(493.67455551,72.00469213)
\closepath
\moveto(493.70655551,68.61269213)
\curveto(492.76788885,68.61269213)(491.99988885,68.31402546)(491.40255551,67.71669213)
\curveto(490.80522218,67.11935879)(490.45322218,66.19135879)(490.34655551,64.93269213)
\lineto(497.03455551,64.93269213)
\curveto(497.01322218,65.97802546)(496.72522218,66.85269213)(496.17055551,67.55669213)
\curveto(495.63722218,68.26069213)(494.81588885,68.61269213)(493.70655551,68.61269213)
\closepath
}
}
{
\newrgbcolor{curcolor}{0 0 0}
\pscustom[linestyle=none,fillstyle=solid,fillcolor=curcolor]
{
\newpath
\moveto(521.25854282,62.98069213)
\curveto(521.25854282,60.07935879)(520.49054282,57.83935879)(518.95454282,56.26069213)
\curveto(517.43987615,54.68202546)(515.37054282,53.89269213)(512.74654282,53.89269213)
\curveto(511.12520949,53.89269213)(509.67454282,54.24469213)(508.39454282,54.94869213)
\curveto(507.13587615,55.65269213)(506.14387615,56.67669213)(505.41854282,58.02069213)
\curveto(504.69320949,59.38602546)(504.33054282,61.03935879)(504.33054282,62.98069213)
\curveto(504.33054282,65.88202546)(505.08787615,68.11135879)(506.60254282,69.66869213)
\curveto(508.11720949,71.22602546)(510.19720949,72.00469213)(512.84254282,72.00469213)
\curveto(514.48520949,72.00469213)(515.93587615,71.65269213)(517.19454282,70.94869213)
\curveto(518.45320949,70.24469213)(519.44520949,69.22069213)(520.17054282,67.87669213)
\curveto(520.89587615,66.53269213)(521.25854282,64.90069213)(521.25854282,62.98069213)
\closepath
\moveto(509.19454282,62.98069213)
\curveto(509.19454282,61.25269213)(509.47187615,59.94069213)(510.02654282,59.04469213)
\curveto(510.60254282,58.17002546)(511.53054282,57.73269213)(512.81054282,57.73269213)
\curveto(514.06920949,57.73269213)(514.97587615,58.17002546)(515.53054282,59.04469213)
\curveto(516.10654282,59.94069213)(516.39454282,61.25269213)(516.39454282,62.98069213)
\curveto(516.39454282,64.70869213)(516.10654282,65.99935879)(515.53054282,66.85269213)
\curveto(514.97587615,67.72735879)(514.05854282,68.16469213)(512.77854282,68.16469213)
\curveto(511.51987615,68.16469213)(510.60254282,67.72735879)(510.02654282,66.85269213)
\curveto(509.47187615,65.99935879)(509.19454282,64.70869213)(509.19454282,62.98069213)
\closepath
}
}
{
\newrgbcolor{curcolor}{0 0 0}
\pscustom[linestyle=none,fillstyle=solid,fillcolor=curcolor]
{
\newpath
\moveto(547.21052622,71.68469213)
\lineto(547.21052622,54.21269213)
\lineto(542.76252622,54.21269213)
\lineto(542.76252622,62.78869213)
\curveto(542.76252622,63.64202546)(542.77319288,64.47402546)(542.79452622,65.28469213)
\curveto(542.83719288,66.09535879)(542.89052622,66.84202546)(542.95452622,67.52469213)
\lineto(542.85852622,67.52469213)
\lineto(538.02652622,54.21269213)
\lineto(534.44252622,54.21269213)
\lineto(529.54652622,67.55669213)
\lineto(529.41852622,67.55669213)
\curveto(529.50385955,66.85269213)(529.55719288,66.09535879)(529.57852622,65.28469213)
\curveto(529.62119288,64.49535879)(529.64252622,63.62069213)(529.64252622,62.66069213)
\lineto(529.64252622,54.21269213)
\lineto(525.19452622,54.21269213)
\lineto(525.19452622,71.68469213)
\lineto(531.94652622,71.68469213)
\lineto(536.29852622,59.84469213)
\lineto(540.71452622,71.68469213)
\closepath
}
}
{
\newrgbcolor{curcolor}{0 0 0}
\pscustom[linestyle=none,fillstyle=solid,fillcolor=curcolor]
{
\newpath
\moveto(559.40252182,72.00469213)
\curveto(561.81318849,72.00469213)(563.72252182,71.31135879)(565.13052182,69.92469213)
\curveto(566.53852182,68.55935879)(567.24252182,66.60735879)(567.24252182,64.06869213)
\lineto(567.24252182,61.76469213)
\lineto(555.97852182,61.76469213)
\curveto(556.02118849,60.42069213)(556.41585516,59.36469213)(557.16252182,58.59669213)
\curveto(557.93052182,57.82869213)(558.98652182,57.44469213)(560.33052182,57.44469213)
\curveto(561.43985516,57.44469213)(562.45318849,57.55135879)(563.37052182,57.76469213)
\curveto(564.30918849,57.99935879)(565.26918849,58.35135879)(566.25052182,58.82069213)
\lineto(566.25052182,55.14069213)
\curveto(565.37585516,54.71402546)(564.46918849,54.40469213)(563.53052182,54.21269213)
\curveto(562.59185516,53.99935879)(561.45052182,53.89269213)(560.10652182,53.89269213)
\curveto(558.35718849,53.89269213)(556.81052182,54.21269213)(555.46652182,54.85269213)
\curveto(554.12252182,55.51402546)(553.06652182,56.49535879)(552.29852182,57.79669213)
\curveto(551.53052182,59.11935879)(551.14652182,60.79402546)(551.14652182,62.82069213)
\curveto(551.14652182,64.84735879)(551.48785516,66.54335879)(552.17052182,67.90869213)
\curveto(552.87452182,69.27402546)(553.84518849,70.29802546)(555.08252182,70.98069213)
\curveto(556.31985516,71.66335879)(557.75985516,72.00469213)(559.40252182,72.00469213)
\closepath
\moveto(559.43452182,68.61269213)
\curveto(558.49585516,68.61269213)(557.72785516,68.31402546)(557.13052182,67.71669213)
\curveto(556.53318849,67.11935879)(556.18118849,66.19135879)(556.07452182,64.93269213)
\lineto(562.76252182,64.93269213)
\curveto(562.74118849,65.97802546)(562.45318849,66.85269213)(561.89852182,67.55669213)
\curveto(561.36518849,68.26069213)(560.54385516,68.61269213)(559.43452182,68.61269213)
\closepath
}
}
{
\newrgbcolor{curcolor}{0 0 0}
\pscustom[linestyle=none,fillstyle=solid,fillcolor=curcolor]
{
\newpath
\moveto(585.57850913,68.10069213)
\lineto(579.85050913,68.10069213)
\lineto(579.85050913,54.21269213)
\lineto(575.08250913,54.21269213)
\lineto(575.08250913,68.10069213)
\lineto(569.35450913,68.10069213)
\lineto(569.35450913,71.68469213)
\lineto(585.57850913,71.68469213)
\closepath
}
}
{
\newrgbcolor{curcolor}{0 0 0}
\pscustom[linestyle=none,fillstyle=solid,fillcolor=curcolor]
{
\newpath
\moveto(598.53849301,72.00469213)
\curveto(600.50115968,72.00469213)(602.09049301,71.23669213)(603.30649301,69.70069213)
\curveto(604.52249301,68.18602546)(605.13049301,65.94602546)(605.13049301,62.98069213)
\curveto(605.13049301,59.99402546)(604.50115968,57.73269213)(603.24249301,56.19669213)
\curveto(601.98382635,54.66069213)(600.37315968,53.89269213)(598.41049301,53.89269213)
\curveto(597.15182635,53.89269213)(596.14915968,54.11669213)(595.40249301,54.56469213)
\curveto(594.65582635,55.03402546)(594.04782635,55.55669213)(593.57849301,56.13269213)
\lineto(593.32249301,56.13269213)
\curveto(593.49315968,55.23669213)(593.57849301,54.38335879)(593.57849301,53.57269213)
\lineto(593.57849301,46.53269213)
\lineto(588.81049301,46.53269213)
\lineto(588.81049301,71.68469213)
\lineto(592.68249301,71.68469213)
\lineto(593.35449301,69.41269213)
\lineto(593.57849301,69.41269213)
\curveto(594.04782635,70.11669213)(594.67715968,70.72469213)(595.46649301,71.23669213)
\curveto(596.25582635,71.74869213)(597.27982635,72.00469213)(598.53849301,72.00469213)
\closepath
\moveto(597.00249301,68.19669213)
\curveto(595.76515968,68.19669213)(594.89049301,67.80202546)(594.37849301,67.01269213)
\curveto(593.86649301,66.24469213)(593.59982635,65.08202546)(593.57849301,63.52469213)
\lineto(593.57849301,63.01269213)
\curveto(593.57849301,61.32735879)(593.82382635,60.02602546)(594.31449301,59.10869213)
\curveto(594.82649301,58.21269213)(595.74382635,57.76469213)(597.06649301,57.76469213)
\curveto(598.15449301,57.76469213)(598.95449301,58.21269213)(599.46649301,59.10869213)
\curveto(599.99982635,60.02602546)(600.26649301,61.33802546)(600.26649301,63.04469213)
\curveto(600.26649301,66.47935879)(599.17849301,68.19669213)(597.00249301,68.19669213)
\closepath
}
}
{
\newrgbcolor{curcolor}{0 0 0}
\pscustom[linestyle=none,fillstyle=solid,fillcolor=curcolor]
{
\newpath
\moveto(613.67447446,71.68469213)
\lineto(613.67447446,64.77269213)
\curveto(613.67447446,64.41002546)(613.65314113,63.96202546)(613.61047446,63.42869213)
\curveto(613.58914113,62.89535879)(613.55714113,62.35135879)(613.51447446,61.79669213)
\curveto(613.49314113,61.24202546)(613.46114113,60.74069213)(613.41847446,60.29269213)
\curveto(613.37580779,59.86602546)(613.34380779,59.57802546)(613.32247446,59.42869213)
\lineto(621.38647446,71.68469213)
\lineto(627.11447446,71.68469213)
\lineto(627.11447446,54.21269213)
\lineto(622.50647446,54.21269213)
\lineto(622.50647446,61.18869213)
\curveto(622.50647446,61.74335879)(622.52780779,62.37269213)(622.57047446,63.07669213)
\curveto(622.61314113,63.78069213)(622.65580779,64.43135879)(622.69847446,65.02869213)
\curveto(622.76247446,65.64735879)(622.80514113,66.11669213)(622.82647446,66.43669213)
\lineto(614.79447446,54.21269213)
\lineto(609.06647446,54.21269213)
\lineto(609.06647446,71.68469213)
\closepath
}
}
{
\newrgbcolor{curcolor}{0 0 0}
\pscustom[linestyle=none,fillstyle=solid,fillcolor=curcolor]
{
\newpath
\moveto(639.30645249,72.00469213)
\curveto(641.71711915,72.00469213)(643.62645249,71.31135879)(645.03445249,69.92469213)
\curveto(646.44245249,68.55935879)(647.14645249,66.60735879)(647.14645249,64.06869213)
\lineto(647.14645249,61.76469213)
\lineto(635.88245249,61.76469213)
\curveto(635.92511915,60.42069213)(636.31978582,59.36469213)(637.06645249,58.59669213)
\curveto(637.83445249,57.82869213)(638.89045249,57.44469213)(640.23445249,57.44469213)
\curveto(641.34378582,57.44469213)(642.35711915,57.55135879)(643.27445249,57.76469213)
\curveto(644.21311915,57.99935879)(645.17311915,58.35135879)(646.15445249,58.82069213)
\lineto(646.15445249,55.14069213)
\curveto(645.27978582,54.71402546)(644.37311915,54.40469213)(643.43445249,54.21269213)
\curveto(642.49578582,53.99935879)(641.35445249,53.89269213)(640.01045249,53.89269213)
\curveto(638.26111915,53.89269213)(636.71445249,54.21269213)(635.37045249,54.85269213)
\curveto(634.02645249,55.51402546)(632.97045249,56.49535879)(632.20245249,57.79669213)
\curveto(631.43445249,59.11935879)(631.05045249,60.79402546)(631.05045249,62.82069213)
\curveto(631.05045249,64.84735879)(631.39178582,66.54335879)(632.07445249,67.90869213)
\curveto(632.77845249,69.27402546)(633.74911915,70.29802546)(634.98645249,70.98069213)
\curveto(636.22378582,71.66335879)(637.66378582,72.00469213)(639.30645249,72.00469213)
\closepath
\moveto(639.33845249,68.61269213)
\curveto(638.39978582,68.61269213)(637.63178582,68.31402546)(637.03445249,67.71669213)
\curveto(636.43711915,67.11935879)(636.08511915,66.19135879)(635.97845249,64.93269213)
\lineto(642.66645249,64.93269213)
\curveto(642.64511915,65.97802546)(642.35711915,66.85269213)(641.80245249,67.55669213)
\curveto(641.26911915,68.26069213)(640.44778582,68.61269213)(639.33845249,68.61269213)
\closepath
}
}
{
\newrgbcolor{curcolor}{0 0 0}
\pscustom[linestyle=none,fillstyle=solid,fillcolor=curcolor]
{
\newpath
\moveto(655.62643979,71.68469213)
\lineto(655.62643979,64.77269213)
\curveto(655.62643979,64.41002546)(655.60510646,63.96202546)(655.56243979,63.42869213)
\curveto(655.54110646,62.89535879)(655.50910646,62.35135879)(655.46643979,61.79669213)
\curveto(655.44510646,61.24202546)(655.41310646,60.74069213)(655.37043979,60.29269213)
\curveto(655.32777313,59.86602546)(655.29577313,59.57802546)(655.27443979,59.42869213)
\lineto(663.33843979,71.68469213)
\lineto(669.06643979,71.68469213)
\lineto(669.06643979,54.21269213)
\lineto(664.45843979,54.21269213)
\lineto(664.45843979,61.18869213)
\curveto(664.45843979,61.74335879)(664.47977313,62.37269213)(664.52243979,63.07669213)
\curveto(664.56510646,63.78069213)(664.60777313,64.43135879)(664.65043979,65.02869213)
\curveto(664.71443979,65.64735879)(664.75710646,66.11669213)(664.77843979,66.43669213)
\lineto(656.74643979,54.21269213)
\lineto(651.01843979,54.21269213)
\lineto(651.01843979,71.68469213)
\closepath
\moveto(667.75443979,79.20469213)
\curveto(667.64777313,78.09535879)(667.33843979,77.11402546)(666.82643979,76.26069213)
\curveto(666.31443979,75.42869213)(665.51443979,74.77802546)(664.42643979,74.30869213)
\curveto(663.33843979,73.83935879)(661.87710646,73.60469213)(660.04243979,73.60469213)
\curveto(658.16510646,73.60469213)(656.69310646,73.82869213)(655.62643979,74.27669213)
\curveto(654.58110646,74.72469213)(653.83443979,75.36469213)(653.38643979,76.19669213)
\curveto(652.93843979,77.05002546)(652.67177313,78.05269213)(652.58643979,79.20469213)
\lineto(656.84243979,79.20469213)
\curveto(656.92777313,78.03135879)(657.21577313,77.25269213)(657.70643979,76.86869213)
\curveto(658.21843979,76.48469213)(659.02910646,76.29269213)(660.13843979,76.29269213)
\curveto(661.05577313,76.29269213)(661.80243979,76.49535879)(662.37843979,76.90069213)
\curveto(662.97577313,77.32735879)(663.32777313,78.09535879)(663.43443979,79.20469213)
\closepath
}
}
{
\newrgbcolor{curcolor}{0 0 0}
\pscustom[linestyle=none,fillstyle=solid,fillcolor=curcolor]
{
\newpath
\moveto(173.77868838,174.84189288)
\curveto(171.92268838,174.84189288)(170.50402172,174.14855954)(169.52268838,172.76189288)
\curveto(168.54135505,171.37522621)(168.05068838,169.47655954)(168.05068838,167.06589288)
\curveto(168.05068838,164.63389288)(168.49868838,162.74589288)(169.39468838,161.40189288)
\curveto(170.31202172,160.07922621)(171.77335505,159.41789288)(173.77868838,159.41789288)
\curveto(174.69602172,159.41789288)(175.62402172,159.52455954)(176.56268838,159.73789288)
\curveto(177.50135505,159.95122621)(178.51468838,160.24989288)(179.60268838,160.63389288)
\lineto(179.60268838,156.56989288)
\curveto(178.60002172,156.16455954)(177.60802172,155.86589288)(176.62668838,155.67389288)
\curveto(175.64535505,155.48189288)(174.54668838,155.38589288)(173.33068838,155.38589288)
\curveto(170.96268838,155.38589288)(169.02135505,155.86589288)(167.50668838,156.82589288)
\curveto(165.99202172,157.80722621)(164.87202172,159.17255954)(164.14668838,160.92189288)
\curveto(163.42135505,162.69255954)(163.05868838,164.75122621)(163.05868838,167.09789288)
\curveto(163.05868838,169.40189288)(163.47468838,171.43922621)(164.30668838,173.20989288)
\curveto(165.13868838,174.98055954)(166.34402172,176.36722621)(167.92268838,177.36989288)
\curveto(169.52268838,178.37255954)(171.47468838,178.87389288)(173.77868838,178.87389288)
\curveto(174.90935505,178.87389288)(176.04002172,178.72455954)(177.17068838,178.42589288)
\curveto(178.32268838,178.14855954)(179.42135505,177.76455954)(180.46668838,177.27389288)
\lineto(178.89868838,173.33789288)
\curveto(178.04535505,173.74322621)(177.18135505,174.09522621)(176.30668838,174.39389288)
\curveto(175.45335505,174.69255954)(174.61068838,174.84189288)(173.77868838,174.84189288)
\closepath
}
}
{
\newrgbcolor{curcolor}{0 0 0}
\pscustom[linestyle=none,fillstyle=solid,fillcolor=curcolor]
{
\newpath
\moveto(199.95466055,173.17789288)
\lineto(199.95466055,155.70589288)
\lineto(195.18666055,155.70589288)
\lineto(195.18666055,169.59389288)
\lineto(188.85066055,169.59389288)
\lineto(188.85066055,155.70589288)
\lineto(184.08266055,155.70589288)
\lineto(184.08266055,173.17789288)
\closepath
}
}
{
\newrgbcolor{curcolor}{0 0 0}
\pscustom[linestyle=none,fillstyle=solid,fillcolor=curcolor]
{
\newpath
\moveto(209.55467422,173.17789288)
\lineto(209.55467422,166.26589288)
\curveto(209.55467422,165.90322621)(209.53334089,165.45522621)(209.49067422,164.92189288)
\curveto(209.46934089,164.38855954)(209.43734089,163.84455954)(209.39467422,163.28989288)
\curveto(209.37334089,162.73522621)(209.34134089,162.23389288)(209.29867422,161.78589288)
\curveto(209.25600756,161.35922621)(209.22400756,161.07122621)(209.20267422,160.92189288)
\lineto(217.26667422,173.17789288)
\lineto(222.99467422,173.17789288)
\lineto(222.99467422,155.70589288)
\lineto(218.38667422,155.70589288)
\lineto(218.38667422,162.68189288)
\curveto(218.38667422,163.23655954)(218.40800756,163.86589288)(218.45067422,164.56989288)
\curveto(218.49334089,165.27389288)(218.53600756,165.92455954)(218.57867422,166.52189288)
\curveto(218.64267422,167.14055954)(218.68534089,167.60989288)(218.70667422,167.92989288)
\lineto(210.67467422,155.70589288)
\lineto(204.94667422,155.70589288)
\lineto(204.94667422,173.17789288)
\closepath
}
}
{
\newrgbcolor{curcolor}{0 0 0}
\pscustom[linestyle=none,fillstyle=solid,fillcolor=curcolor]
{
\newpath
\moveto(235.09065225,155.38589288)
\curveto(232.48798558,155.38589288)(230.47198558,156.10055954)(229.04265225,157.52989288)
\curveto(227.63465225,158.95922621)(226.93065225,161.23122621)(226.93065225,164.34589288)
\curveto(226.93065225,166.47922621)(227.29331892,168.21789288)(228.01865225,169.56189288)
\curveto(228.74398558,170.90589288)(229.74665225,171.89789288)(231.02665225,172.53789288)
\curveto(232.32798558,173.17789288)(233.82131892,173.49789288)(235.50665225,173.49789288)
\curveto(236.70131892,173.49789288)(237.73598558,173.38055954)(238.61065225,173.14589288)
\curveto(239.50665225,172.91122621)(240.28531892,172.63389288)(240.94665225,172.31389288)
\lineto(239.53865225,168.63389288)
\curveto(238.79198558,168.93255954)(238.08798558,169.17789288)(237.42665225,169.36989288)
\curveto(236.78665225,169.56189288)(236.14665225,169.65789288)(235.50665225,169.65789288)
\curveto(233.03198558,169.65789288)(231.79465225,167.89789288)(231.79465225,164.37789288)
\curveto(231.79465225,162.62855954)(232.11465225,161.33789288)(232.75465225,160.50589288)
\curveto(233.41598558,159.67389288)(234.33331892,159.25789288)(235.50665225,159.25789288)
\curveto(236.50931892,159.25789288)(237.39465225,159.38589288)(238.16265225,159.64189288)
\curveto(238.93065225,159.91922621)(239.67731892,160.29255954)(240.40265225,160.76189288)
\lineto(240.40265225,156.69789288)
\curveto(239.67731892,156.22855954)(238.90931892,155.89789288)(238.09865225,155.70589288)
\curveto(237.30931892,155.49255954)(236.30665225,155.38589288)(235.09065225,155.38589288)
\closepath
}
}
{
\newrgbcolor{curcolor}{0 0 0}
\pscustom[linestyle=none,fillstyle=solid,fillcolor=curcolor]
{
\newpath
\moveto(260.3066503,164.47389288)
\curveto(260.3066503,161.57255954)(259.5386503,159.33255954)(258.0026503,157.75389288)
\curveto(256.48798363,156.17522621)(254.4186503,155.38589288)(251.7946503,155.38589288)
\curveto(250.17331696,155.38589288)(248.7226503,155.73789288)(247.4426503,156.44189288)
\curveto(246.18398363,157.14589288)(245.19198363,158.16989288)(244.4666503,159.51389288)
\curveto(243.74131696,160.87922621)(243.3786503,162.53255954)(243.3786503,164.47389288)
\curveto(243.3786503,167.37522621)(244.13598363,169.60455954)(245.6506503,171.16189288)
\curveto(247.16531696,172.71922621)(249.24531696,173.49789288)(251.8906503,173.49789288)
\curveto(253.53331696,173.49789288)(254.98398363,173.14589288)(256.2426503,172.44189288)
\curveto(257.50131696,171.73789288)(258.49331696,170.71389288)(259.2186503,169.36989288)
\curveto(259.94398363,168.02589288)(260.3066503,166.39389288)(260.3066503,164.47389288)
\closepath
\moveto(248.2426503,164.47389288)
\curveto(248.2426503,162.74589288)(248.51998363,161.43389288)(249.0746503,160.53789288)
\curveto(249.6506503,159.66322621)(250.5786503,159.22589288)(251.8586503,159.22589288)
\curveto(253.11731696,159.22589288)(254.02398363,159.66322621)(254.5786503,160.53789288)
\curveto(255.1546503,161.43389288)(255.4426503,162.74589288)(255.4426503,164.47389288)
\curveto(255.4426503,166.20189288)(255.1546503,167.49255954)(254.5786503,168.34589288)
\curveto(254.02398363,169.22055954)(253.1066503,169.65789288)(251.8266503,169.65789288)
\curveto(250.56798363,169.65789288)(249.6506503,169.22055954)(249.0746503,168.34589288)
\curveto(248.51998363,167.49255954)(248.2426503,166.20189288)(248.2426503,164.47389288)
\closepath
}
}
{
\newrgbcolor{curcolor}{0 0 0}
\pscustom[linestyle=none,fillstyle=solid,fillcolor=curcolor]
{
\newpath
\moveto(275.69863369,173.17789288)
\lineto(280.94663369,173.17789288)
\lineto(274.03463369,164.79389288)
\lineto(281.55463369,155.70589288)
\lineto(276.14663369,155.70589288)
\lineto(269.01063369,164.56989288)
\lineto(269.01063369,155.70589288)
\lineto(264.24263369,155.70589288)
\lineto(264.24263369,173.17789288)
\lineto(269.01063369,173.17789288)
\lineto(269.01063369,164.69789288)
\closepath
}
}
{
\newrgbcolor{curcolor}{0 0 0}
\pscustom[linestyle=none,fillstyle=solid,fillcolor=curcolor]
{
\newpath
\moveto(123.37868326,133.17789288)
\lineto(128.59468326,133.17789288)
\lineto(131.89068326,123.35389288)
\curveto(132.06134992,122.86322621)(132.18934992,122.37255954)(132.27468326,121.88189288)
\curveto(132.36001659,121.39122621)(132.42401659,120.86855954)(132.46668326,120.31389288)
\lineto(132.56268326,120.31389288)
\curveto(132.62668326,120.86855954)(132.71201659,121.39122621)(132.81868326,121.88189288)
\curveto(132.92534992,122.37255954)(133.06401659,122.86322621)(133.23468326,123.35389288)
\lineto(136.46668326,133.17789288)
\lineto(141.58668326,133.17789288)
\lineto(134.19468326,113.46589288)
\curveto(133.51201659,111.65255954)(132.54134992,110.29789288)(131.28268326,109.40189288)
\curveto(130.02401659,108.48455954)(128.56268326,108.02589288)(126.89868326,108.02589288)
\curveto(126.34401659,108.02589288)(125.87468326,108.05789288)(125.49068326,108.12189288)
\curveto(125.10668326,108.16455954)(124.76534992,108.21789288)(124.46668326,108.28189288)
\lineto(124.46668326,112.05789288)
\curveto(124.68001659,112.01522621)(124.95734992,111.97255954)(125.29868326,111.92989288)
\curveto(125.64001659,111.88722621)(125.99201659,111.86589288)(126.35468326,111.86589288)
\curveto(127.35734992,111.86589288)(128.14668326,112.17522621)(128.72268326,112.79389288)
\curveto(129.29868326,113.39122621)(129.73601659,114.11655954)(130.03468326,114.96989288)
\lineto(130.32268326,115.83389288)
\closepath
}
}
{
\newrgbcolor{curcolor}{0 0 0}
\pscustom[linestyle=none,fillstyle=solid,fillcolor=curcolor]
{
\newpath
\moveto(148.27469107,133.17789288)
\lineto(148.27469107,126.77789288)
\curveto(148.27469107,125.26322621)(148.97869107,124.50589288)(150.38669107,124.50589288)
\curveto(151.3040244,124.50589288)(152.15735773,124.60189288)(152.94669107,124.79389288)
\curveto(153.7360244,125.00722621)(154.52535773,125.28455954)(155.31469107,125.62589288)
\lineto(155.31469107,133.17789288)
\lineto(160.08269107,133.17789288)
\lineto(160.08269107,115.70589288)
\lineto(155.31469107,115.70589288)
\lineto(155.31469107,122.64989288)
\curveto(154.5680244,122.24455954)(153.71469107,121.87122621)(152.75469107,121.52989288)
\curveto(151.79469107,121.20989288)(150.70669107,121.04989288)(149.49069107,121.04989288)
\curveto(147.67735773,121.04989288)(146.22669107,121.50855954)(145.13869107,122.42589288)
\curveto(144.05069107,123.36455954)(143.50669107,124.78322621)(143.50669107,126.68189288)
\lineto(143.50669107,133.17789288)
\closepath
}
}
{
\newrgbcolor{curcolor}{0 0 0}
\pscustom[linestyle=none,fillstyle=solid,fillcolor=curcolor]
{
\newpath
\moveto(172.24269546,133.52989288)
\curveto(174.58936213,133.52989288)(176.38136213,133.01789288)(177.61869546,131.99389288)
\curveto(178.87736213,130.99122621)(179.50669546,129.44455954)(179.50669546,127.35389288)
\lineto(179.50669546,115.70589288)
\lineto(176.17869546,115.70589288)
\lineto(175.25069546,118.07389288)
\lineto(175.12269546,118.07389288)
\curveto(174.3760288,117.13522621)(173.58669546,116.45255954)(172.75469546,116.02589288)
\curveto(171.92269546,115.59922621)(170.78136213,115.38589288)(169.33069546,115.38589288)
\curveto(167.77336213,115.38589288)(166.48269546,115.83389288)(165.45869546,116.72989288)
\curveto(164.43469546,117.62589288)(163.92269546,119.02322621)(163.92269546,120.92189288)
\curveto(163.92269546,122.77789288)(164.57336213,124.14322621)(165.87469546,125.01789288)
\curveto(167.1760288,125.89255954)(169.1280288,126.38322621)(171.73069546,126.48989288)
\lineto(174.77069546,126.58589288)
\lineto(174.77069546,127.35389288)
\curveto(174.77069546,128.27122621)(174.52536213,128.94322621)(174.03469546,129.36989288)
\curveto(173.56536213,129.79655954)(172.9040288,130.00989288)(172.05069546,130.00989288)
\curveto(171.19736213,130.00989288)(170.36536213,129.88189288)(169.55469546,129.62589288)
\curveto(168.7440288,129.39122621)(167.93336213,129.09255954)(167.12269546,128.72989288)
\lineto(165.55469546,131.96189288)
\curveto(166.4720288,132.43122621)(167.50669546,132.80455954)(168.65869546,133.08189288)
\curveto(169.81069546,133.38055954)(171.00536213,133.52989288)(172.24269546,133.52989288)
\closepath
\moveto(174.77069546,123.80189288)
\lineto(172.91469546,123.73789288)
\curveto(171.37869546,123.69522621)(170.3120288,123.41789288)(169.71469546,122.90589288)
\curveto(169.11736213,122.39389288)(168.81869546,121.72189288)(168.81869546,120.88989288)
\curveto(168.81869546,120.16455954)(169.0320288,119.64189288)(169.45869546,119.32189288)
\curveto(169.88536213,119.02322621)(170.4400288,118.87389288)(171.12269546,118.87389288)
\curveto(172.14669546,118.87389288)(173.01069546,119.17255954)(173.71469546,119.76989288)
\curveto(174.41869546,120.38855954)(174.77069546,121.25255954)(174.77069546,122.36189288)
\closepath
}
}
{
\newrgbcolor{curcolor}{0 0 0}
\pscustom[linestyle=none,fillstyle=solid,fillcolor=curcolor]
{
\newpath
\moveto(191.50669839,115.38589288)
\curveto(188.90403173,115.38589288)(186.88803173,116.10055954)(185.45869839,117.52989288)
\curveto(184.05069839,118.95922621)(183.34669839,121.23122621)(183.34669839,124.34589288)
\curveto(183.34669839,126.47922621)(183.70936506,128.21789288)(184.43469839,129.56189288)
\curveto(185.16003173,130.90589288)(186.16269839,131.89789288)(187.44269839,132.53789288)
\curveto(188.74403173,133.17789288)(190.23736506,133.49789288)(191.92269839,133.49789288)
\curveto(193.11736506,133.49789288)(194.15203173,133.38055954)(195.02669839,133.14589288)
\curveto(195.92269839,132.91122621)(196.70136506,132.63389288)(197.36269839,132.31389288)
\lineto(195.95469839,128.63389288)
\curveto(195.20803173,128.93255954)(194.50403173,129.17789288)(193.84269839,129.36989288)
\curveto(193.20269839,129.56189288)(192.56269839,129.65789288)(191.92269839,129.65789288)
\curveto(189.44803173,129.65789288)(188.21069839,127.89789288)(188.21069839,124.37789288)
\curveto(188.21069839,122.62855954)(188.53069839,121.33789288)(189.17069839,120.50589288)
\curveto(189.83203173,119.67389288)(190.74936506,119.25789288)(191.92269839,119.25789288)
\curveto(192.92536506,119.25789288)(193.81069839,119.38589288)(194.57869839,119.64189288)
\curveto(195.34669839,119.91922621)(196.09336506,120.29255954)(196.81869839,120.76189288)
\lineto(196.81869839,116.69789288)
\curveto(196.09336506,116.22855954)(195.32536506,115.89789288)(194.51469839,115.70589288)
\curveto(193.72536506,115.49255954)(192.72269839,115.38589288)(191.50669839,115.38589288)
\closepath
}
}
{
\newrgbcolor{curcolor}{0 0 0}
\pscustom[linestyle=none,fillstyle=solid,fillcolor=curcolor]
{
\newpath
\moveto(215.31469644,129.59389288)
\lineto(209.58669644,129.59389288)
\lineto(209.58669644,115.70589288)
\lineto(204.81869644,115.70589288)
\lineto(204.81869644,129.59389288)
\lineto(199.09069644,129.59389288)
\lineto(199.09069644,133.17789288)
\lineto(215.31469644,133.17789288)
\closepath
}
}
{
\newrgbcolor{curcolor}{0 0 0}
\pscustom[linestyle=none,fillstyle=solid,fillcolor=curcolor]
{
\newpath
\moveto(223.31468033,133.17789288)
\lineto(223.31468033,126.45789288)
\lineto(229.97068033,126.45789288)
\lineto(229.97068033,133.17789288)
\lineto(234.73868033,133.17789288)
\lineto(234.73868033,115.70589288)
\lineto(229.97068033,115.70589288)
\lineto(229.97068033,122.90589288)
\lineto(223.31468033,122.90589288)
\lineto(223.31468033,115.70589288)
\lineto(218.54668033,115.70589288)
\lineto(218.54668033,133.17789288)
\closepath
}
}
{
\newrgbcolor{curcolor}{0 0 0}
\pscustom[linestyle=none,fillstyle=solid,fillcolor=curcolor]
{
\newpath
\moveto(244.33870132,133.17789288)
\lineto(244.33870132,126.26589288)
\curveto(244.33870132,125.90322621)(244.31736799,125.45522621)(244.27470132,124.92189288)
\curveto(244.25336799,124.38855954)(244.22136799,123.84455954)(244.17870132,123.28989288)
\curveto(244.15736799,122.73522621)(244.12536799,122.23389288)(244.08270132,121.78589288)
\curveto(244.04003465,121.35922621)(244.00803465,121.07122621)(243.98670132,120.92189288)
\lineto(252.05070132,133.17789288)
\lineto(257.77870132,133.17789288)
\lineto(257.77870132,115.70589288)
\lineto(253.17070132,115.70589288)
\lineto(253.17070132,122.68189288)
\curveto(253.17070132,123.23655954)(253.19203465,123.86589288)(253.23470132,124.56989288)
\curveto(253.27736799,125.27389288)(253.32003465,125.92455954)(253.36270132,126.52189288)
\curveto(253.42670132,127.14055954)(253.46936799,127.60989288)(253.49070132,127.92989288)
\lineto(245.45870132,115.70589288)
\lineto(239.73070132,115.70589288)
\lineto(239.73070132,133.17789288)
\closepath
}
}
{
\newrgbcolor{curcolor}{0 0 0}
\pscustom[linestyle=none,fillstyle=solid,fillcolor=curcolor]
{
\newpath
\moveto(274.22667935,133.17789288)
\lineto(279.47467935,133.17789288)
\lineto(272.56267935,124.79389288)
\lineto(280.08267935,115.70589288)
\lineto(274.67467935,115.70589288)
\lineto(267.53867935,124.56989288)
\lineto(267.53867935,115.70589288)
\lineto(262.77067935,115.70589288)
\lineto(262.77067935,133.17789288)
\lineto(267.53867935,133.17789288)
\lineto(267.53867935,124.69789288)
\closepath
}
}
{
\newrgbcolor{curcolor}{0 0 0}
\pscustom[linestyle=none,fillstyle=solid,fillcolor=curcolor]
{
\newpath
\moveto(297.8106481,124.47389288)
\curveto(297.8106481,121.57255954)(297.0426481,119.33255954)(295.5066481,117.75389288)
\curveto(293.99198143,116.17522621)(291.9226481,115.38589288)(289.2986481,115.38589288)
\curveto(287.67731477,115.38589288)(286.2266481,115.73789288)(284.9466481,116.44189288)
\curveto(283.68798143,117.14589288)(282.69598143,118.16989288)(281.9706481,119.51389288)
\curveto(281.24531477,120.87922621)(280.8826481,122.53255954)(280.8826481,124.47389288)
\curveto(280.8826481,127.37522621)(281.63998143,129.60455954)(283.1546481,131.16189288)
\curveto(284.66931477,132.71922621)(286.74931477,133.49789288)(289.3946481,133.49789288)
\curveto(291.03731477,133.49789288)(292.48798143,133.14589288)(293.7466481,132.44189288)
\curveto(295.00531477,131.73789288)(295.99731477,130.71389288)(296.7226481,129.36989288)
\curveto(297.44798143,128.02589288)(297.8106481,126.39389288)(297.8106481,124.47389288)
\closepath
\moveto(285.7466481,124.47389288)
\curveto(285.7466481,122.74589288)(286.02398143,121.43389288)(286.5786481,120.53789288)
\curveto(287.1546481,119.66322621)(288.0826481,119.22589288)(289.3626481,119.22589288)
\curveto(290.62131477,119.22589288)(291.52798143,119.66322621)(292.0826481,120.53789288)
\curveto(292.6586481,121.43389288)(292.9466481,122.74589288)(292.9466481,124.47389288)
\curveto(292.9466481,126.20189288)(292.6586481,127.49255954)(292.0826481,128.34589288)
\curveto(291.52798143,129.22055954)(290.6106481,129.65789288)(289.3306481,129.65789288)
\curveto(288.07198143,129.65789288)(287.1546481,129.22055954)(286.5786481,128.34589288)
\curveto(286.02398143,127.49255954)(285.7466481,126.20189288)(285.7466481,124.47389288)
\closepath
}
}
{
\newrgbcolor{curcolor}{0 0 0}
\pscustom[linestyle=none,fillstyle=solid,fillcolor=curcolor]
{
\newpath
\moveto(317.2346315,128.60189288)
\curveto(317.2346315,127.66322621)(316.93596483,126.86322621)(316.3386315,126.20189288)
\curveto(315.7626315,125.54055954)(314.8986315,125.11389288)(313.7466315,124.92189288)
\lineto(313.7466315,124.79389288)
\curveto(314.9626315,124.64455954)(315.93329816,124.21789288)(316.6586315,123.51389288)
\curveto(317.40529816,122.83122621)(317.7786315,121.96722621)(317.7786315,120.92189288)
\curveto(317.7786315,119.91922621)(317.51196483,119.02322621)(316.9786315,118.23389288)
\curveto(316.4666315,117.44455954)(315.64529816,116.82589288)(314.5146315,116.37789288)
\curveto(313.38396483,115.92989288)(311.90129816,115.70589288)(310.0666315,115.70589288)
\lineto(301.7466315,115.70589288)
\lineto(301.7466315,133.17789288)
\lineto(310.0666315,133.17789288)
\curveto(311.43196483,133.17789288)(312.64796483,133.02855954)(313.7146315,132.72989288)
\curveto(314.8026315,132.45255954)(315.65596483,131.97255954)(316.2746315,131.28989288)
\curveto(316.9146315,130.62855954)(317.2346315,129.73255954)(317.2346315,128.60189288)
\closepath
\moveto(312.4026315,128.21789288)
\curveto(312.4026315,129.28455954)(311.55996483,129.81789288)(309.8746315,129.81789288)
\lineto(306.5146315,129.81789288)
\lineto(306.5146315,126.36189288)
\lineto(309.3306315,126.36189288)
\curveto(310.33329816,126.36189288)(311.0906315,126.50055954)(311.6026315,126.77789288)
\curveto(312.13596483,127.07655954)(312.4026315,127.55655954)(312.4026315,128.21789288)
\closepath
\moveto(312.8506315,121.17789288)
\curveto(312.8506315,121.86055954)(312.57329816,122.35122621)(312.0186315,122.64989288)
\curveto(311.48529816,122.96989288)(310.69596483,123.12989288)(309.6506315,123.12989288)
\lineto(306.5146315,123.12989288)
\lineto(306.5146315,119.00189288)
\lineto(309.7466315,119.00189288)
\curveto(310.6426315,119.00189288)(311.3786315,119.16189288)(311.9546315,119.48189288)
\curveto(312.55196483,119.82322621)(312.8506315,120.38855954)(312.8506315,121.17789288)
\closepath
}
}
{
\newrgbcolor{curcolor}{0 0 0}
\pscustom[linestyle=none,fillstyle=solid,fillcolor=curcolor]
{
\newpath
\moveto(159.63466885,93.17789288)
\lineto(164.88266885,93.17789288)
\lineto(157.97066885,84.79389288)
\lineto(165.49066885,75.70589288)
\lineto(160.08266885,75.70589288)
\lineto(152.94666885,84.56989288)
\lineto(152.94666885,75.70589288)
\lineto(148.17866885,75.70589288)
\lineto(148.17866885,93.17789288)
\lineto(152.94666885,93.17789288)
\lineto(152.94666885,84.69789288)
\closepath
}
}
{
\newrgbcolor{curcolor}{0 0 0}
\pscustom[linestyle=none,fillstyle=solid,fillcolor=curcolor]
{
\newpath
\moveto(183.2186376,84.47389288)
\curveto(183.2186376,81.57255954)(182.4506376,79.33255954)(180.9146376,77.75389288)
\curveto(179.39997093,76.17522621)(177.3306376,75.38589288)(174.7066376,75.38589288)
\curveto(173.08530427,75.38589288)(171.6346376,75.73789288)(170.3546376,76.44189288)
\curveto(169.09597093,77.14589288)(168.10397093,78.16989288)(167.3786376,79.51389288)
\curveto(166.65330427,80.87922621)(166.2906376,82.53255954)(166.2906376,84.47389288)
\curveto(166.2906376,87.37522621)(167.04797093,89.60455954)(168.5626376,91.16189288)
\curveto(170.07730427,92.71922621)(172.15730427,93.49789288)(174.8026376,93.49789288)
\curveto(176.44530427,93.49789288)(177.89597093,93.14589288)(179.1546376,92.44189288)
\curveto(180.41330427,91.73789288)(181.40530427,90.71389288)(182.1306376,89.36989288)
\curveto(182.85597093,88.02589288)(183.2186376,86.39389288)(183.2186376,84.47389288)
\closepath
\moveto(171.1546376,84.47389288)
\curveto(171.1546376,82.74589288)(171.43197093,81.43389288)(171.9866376,80.53789288)
\curveto(172.5626376,79.66322621)(173.4906376,79.22589288)(174.7706376,79.22589288)
\curveto(176.02930427,79.22589288)(176.93597093,79.66322621)(177.4906376,80.53789288)
\curveto(178.0666376,81.43389288)(178.3546376,82.74589288)(178.3546376,84.47389288)
\curveto(178.3546376,86.20189288)(178.0666376,87.49255954)(177.4906376,88.34589288)
\curveto(176.93597093,89.22055954)(176.0186376,89.65789288)(174.7386376,89.65789288)
\curveto(173.47997093,89.65789288)(172.5626376,89.22055954)(171.9866376,88.34589288)
\curveto(171.43197093,87.49255954)(171.1546376,86.20189288)(171.1546376,84.47389288)
\closepath
}
}
{
\newrgbcolor{curcolor}{0 0 0}
\pscustom[linestyle=none,fillstyle=solid,fillcolor=curcolor]
{
\newpath
\moveto(209.170621,93.17789288)
\lineto(209.170621,75.70589288)
\lineto(204.722621,75.70589288)
\lineto(204.722621,84.28189288)
\curveto(204.722621,85.13522621)(204.73328767,85.96722621)(204.754621,86.77789288)
\curveto(204.79728767,87.58855954)(204.850621,88.33522621)(204.914621,89.01789288)
\lineto(204.818621,89.01789288)
\lineto(199.986621,75.70589288)
\lineto(196.402621,75.70589288)
\lineto(191.506621,89.04989288)
\lineto(191.378621,89.04989288)
\curveto(191.46395433,88.34589288)(191.51728767,87.58855954)(191.538621,86.77789288)
\curveto(191.58128767,85.98855954)(191.602621,85.11389288)(191.602621,84.15389288)
\lineto(191.602621,75.70589288)
\lineto(187.154621,75.70589288)
\lineto(187.154621,93.17789288)
\lineto(193.906621,93.17789288)
\lineto(198.258621,81.33789288)
\lineto(202.674621,93.17789288)
\closepath
}
}
{
\newrgbcolor{curcolor}{0 0 0}
\pscustom[linestyle=none,fillstyle=solid,fillcolor=curcolor]
{
\newpath
\moveto(218.9306166,93.17789288)
\lineto(218.9306166,86.45789288)
\lineto(225.5866166,86.45789288)
\lineto(225.5866166,93.17789288)
\lineto(230.3546166,93.17789288)
\lineto(230.3546166,75.70589288)
\lineto(225.5866166,75.70589288)
\lineto(225.5866166,82.90589288)
\lineto(218.9306166,82.90589288)
\lineto(218.9306166,75.70589288)
\lineto(214.1626166,75.70589288)
\lineto(214.1626166,93.17789288)
\closepath
}
}
{
\newrgbcolor{curcolor}{0 0 0}
\pscustom[linestyle=none,fillstyle=solid,fillcolor=curcolor]
{
\newpath
\moveto(242.5146376,93.52989288)
\curveto(244.86130427,93.52989288)(246.65330427,93.01789288)(247.8906376,91.99389288)
\curveto(249.14930427,90.99122621)(249.7786376,89.44455954)(249.7786376,87.35389288)
\lineto(249.7786376,75.70589288)
\lineto(246.4506376,75.70589288)
\lineto(245.5226376,78.07389288)
\lineto(245.3946376,78.07389288)
\curveto(244.64797093,77.13522621)(243.8586376,76.45255954)(243.0266376,76.02589288)
\curveto(242.1946376,75.59922621)(241.05330427,75.38589288)(239.6026376,75.38589288)
\curveto(238.04530427,75.38589288)(236.7546376,75.83389288)(235.7306376,76.72989288)
\curveto(234.7066376,77.62589288)(234.1946376,79.02322621)(234.1946376,80.92189288)
\curveto(234.1946376,82.77789288)(234.84530427,84.14322621)(236.1466376,85.01789288)
\curveto(237.44797093,85.89255954)(239.39997093,86.38322621)(242.0026376,86.48989288)
\lineto(245.0426376,86.58589288)
\lineto(245.0426376,87.35389288)
\curveto(245.0426376,88.27122621)(244.79730427,88.94322621)(244.3066376,89.36989288)
\curveto(243.83730427,89.79655954)(243.17597093,90.00989288)(242.3226376,90.00989288)
\curveto(241.46930427,90.00989288)(240.63730427,89.88189288)(239.8266376,89.62589288)
\curveto(239.01597093,89.39122621)(238.20530427,89.09255954)(237.3946376,88.72989288)
\lineto(235.8266376,91.96189288)
\curveto(236.74397093,92.43122621)(237.7786376,92.80455954)(238.9306376,93.08189288)
\curveto(240.0826376,93.38055954)(241.27730427,93.52989288)(242.5146376,93.52989288)
\closepath
\moveto(245.0426376,83.80189288)
\lineto(243.1866376,83.73789288)
\curveto(241.6506376,83.69522621)(240.58397093,83.41789288)(239.9866376,82.90589288)
\curveto(239.38930427,82.39389288)(239.0906376,81.72189288)(239.0906376,80.88989288)
\curveto(239.0906376,80.16455954)(239.30397093,79.64189288)(239.7306376,79.32189288)
\curveto(240.15730427,79.02322621)(240.71197093,78.87389288)(241.3946376,78.87389288)
\curveto(242.4186376,78.87389288)(243.2826376,79.17255954)(243.9866376,79.76989288)
\curveto(244.6906376,80.38855954)(245.0426376,81.25255954)(245.0426376,82.36189288)
\closepath
}
}
{
\newrgbcolor{curcolor}{0 0 0}
\pscustom[linestyle=none,fillstyle=solid,fillcolor=curcolor]
{
\newpath
\moveto(269.13864053,89.59389288)
\lineto(263.41064053,89.59389288)
\lineto(263.41064053,75.70589288)
\lineto(258.64264053,75.70589288)
\lineto(258.64264053,89.59389288)
\lineto(252.91464053,89.59389288)
\lineto(252.91464053,93.17789288)
\lineto(269.13864053,93.17789288)
\closepath
}
}
{
\newrgbcolor{curcolor}{0 0 0}
\pscustom[linestyle=none,fillstyle=solid,fillcolor=curcolor]
{
\newpath
\moveto(272.37062442,75.70589288)
\lineto(272.37062442,93.17789288)
\lineto(277.13862442,93.17789288)
\lineto(277.13862442,86.42589288)
\lineto(279.44262442,86.42589288)
\curveto(282.10929108,86.42589288)(284.08262442,85.99922621)(285.36262442,85.14589288)
\curveto(286.64262442,84.29255954)(287.28262442,83.00189288)(287.28262442,81.27389288)
\curveto(287.28262442,79.56722621)(286.68529108,78.21255954)(285.49062442,77.20989288)
\curveto(284.29595775,76.20722621)(282.33329108,75.70589288)(279.60262442,75.70589288)
\closepath
\moveto(289.81062442,75.70589288)
\lineto(289.81062442,93.17789288)
\lineto(294.57862442,93.17789288)
\lineto(294.57862442,75.70589288)
\closepath
\moveto(277.13862442,79.00189288)
\lineto(279.34662442,79.00189288)
\curveto(280.28529108,79.00189288)(281.04262442,79.16189288)(281.61862442,79.48189288)
\curveto(282.21595775,79.82322621)(282.51462442,80.39922621)(282.51462442,81.20989288)
\curveto(282.51462442,82.48989288)(281.43729108,83.12989288)(279.28262442,83.12989288)
\lineto(277.13862442,83.12989288)
\closepath
}
}
{
\newrgbcolor{curcolor}{0 0 0}
\pscustom[linestyle=none,fillstyle=solid,fillcolor=curcolor]
{
\newpath
\moveto(105.16816812,417.92023)
\lineto(105.16816812,440.76823)
\lineto(123.24816812,440.76823)
\lineto(123.24816812,417.92023)
\lineto(118.41616812,417.92023)
\lineto(118.41616812,436.73623)
\lineto(110.00016812,436.73623)
\lineto(110.00016812,417.92023)
\closepath
}
}
{
\newrgbcolor{curcolor}{0 0 0}
\pscustom[linestyle=none,fillstyle=solid,fillcolor=curcolor]
{
\newpath
\moveto(135.79219497,435.74423)
\curveto(138.13886164,435.74423)(139.93086164,435.23223)(141.16819497,434.20823)
\curveto(142.42686164,433.20556333)(143.05619497,431.65889667)(143.05619497,429.56823)
\lineto(143.05619497,417.92023)
\lineto(139.72819497,417.92023)
\lineto(138.80019497,420.28823)
\lineto(138.67219497,420.28823)
\curveto(137.92552831,419.34956333)(137.13619497,418.66689667)(136.30419497,418.24023)
\curveto(135.47219497,417.81356333)(134.33086164,417.60023)(132.88019497,417.60023)
\curveto(131.32286164,417.60023)(130.03219497,418.04823)(129.00819497,418.94423)
\curveto(127.98419497,419.84023)(127.47219497,421.23756333)(127.47219497,423.13623)
\curveto(127.47219497,424.99223)(128.12286164,426.35756333)(129.42419497,427.23223)
\curveto(130.72552831,428.10689667)(132.67752831,428.59756333)(135.28019497,428.70423)
\lineto(138.32019497,428.80023)
\lineto(138.32019497,429.56823)
\curveto(138.32019497,430.48556333)(138.07486164,431.15756333)(137.58419497,431.58423)
\curveto(137.11486164,432.01089667)(136.45352831,432.22423)(135.60019497,432.22423)
\curveto(134.74686164,432.22423)(133.91486164,432.09623)(133.10419497,431.84023)
\curveto(132.29352831,431.60556333)(131.48286164,431.30689667)(130.67219497,430.94423)
\lineto(129.10419497,434.17623)
\curveto(130.02152831,434.64556333)(131.05619497,435.01889667)(132.20819497,435.29623)
\curveto(133.36019497,435.59489667)(134.55486164,435.74423)(135.79219497,435.74423)
\closepath
\moveto(138.32019497,426.01623)
\lineto(136.46419497,425.95223)
\curveto(134.92819497,425.90956333)(133.86152831,425.63223)(133.26419497,425.12023)
\curveto(132.66686164,424.60823)(132.36819497,423.93623)(132.36819497,423.10423)
\curveto(132.36819497,422.37889667)(132.58152831,421.85623)(133.00819497,421.53623)
\curveto(133.43486164,421.23756333)(133.98952831,421.08823)(134.67219497,421.08823)
\curveto(135.69619497,421.08823)(136.56019497,421.38689667)(137.26419497,421.98423)
\curveto(137.96819497,422.60289667)(138.32019497,423.46689667)(138.32019497,424.57623)
\closepath
}
}
{
\newrgbcolor{curcolor}{0 0 0}
\pscustom[linestyle=none,fillstyle=solid,fillcolor=curcolor]
{
\newpath
\moveto(152.7201979,435.39223)
\lineto(152.7201979,428.67223)
\lineto(159.3761979,428.67223)
\lineto(159.3761979,435.39223)
\lineto(164.1441979,435.39223)
\lineto(164.1441979,417.92023)
\lineto(159.3761979,417.92023)
\lineto(159.3761979,425.12023)
\lineto(152.7201979,425.12023)
\lineto(152.7201979,417.92023)
\lineto(147.9521979,417.92023)
\lineto(147.9521979,435.39223)
\closepath
}
}
{
\newrgbcolor{curcolor}{0 0 0}
\pscustom[linestyle=none,fillstyle=solid,fillcolor=curcolor]
{
\newpath
\moveto(176.3362189,435.71223)
\curveto(178.74688557,435.71223)(180.6562189,435.01889667)(182.0642189,433.63223)
\curveto(183.4722189,432.26689667)(184.1762189,430.31489667)(184.1762189,427.77623)
\lineto(184.1762189,425.47223)
\lineto(172.9122189,425.47223)
\curveto(172.95488557,424.12823)(173.34955223,423.07223)(174.0962189,422.30423)
\curveto(174.8642189,421.53623)(175.9202189,421.15223)(177.2642189,421.15223)
\curveto(178.37355223,421.15223)(179.38688557,421.25889667)(180.3042189,421.47223)
\curveto(181.24288557,421.70689667)(182.20288557,422.05889667)(183.1842189,422.52823)
\lineto(183.1842189,418.84823)
\curveto(182.30955223,418.42156333)(181.40288557,418.11223)(180.4642189,417.92023)
\curveto(179.52555223,417.70689667)(178.3842189,417.60023)(177.0402189,417.60023)
\curveto(175.29088557,417.60023)(173.7442189,417.92023)(172.4002189,418.56023)
\curveto(171.0562189,419.22156333)(170.0002189,420.20289667)(169.2322189,421.50423)
\curveto(168.4642189,422.82689667)(168.0802189,424.50156333)(168.0802189,426.52823)
\curveto(168.0802189,428.55489667)(168.42155223,430.25089667)(169.1042189,431.61623)
\curveto(169.8082189,432.98156333)(170.77888557,434.00556333)(172.0162189,434.68823)
\curveto(173.25355223,435.37089667)(174.69355223,435.71223)(176.3362189,435.71223)
\closepath
\moveto(176.3682189,432.32023)
\curveto(175.42955223,432.32023)(174.66155223,432.02156333)(174.0642189,431.42423)
\curveto(173.46688557,430.82689667)(173.11488557,429.89889667)(173.0082189,428.64023)
\lineto(179.6962189,428.64023)
\curveto(179.67488557,429.68556333)(179.38688557,430.56023)(178.8322189,431.26423)
\curveto(178.29888557,431.96823)(177.47755223,432.32023)(176.3682189,432.32023)
\closepath
}
}
{
\newrgbcolor{curcolor}{0 0 0}
\pscustom[linestyle=none,fillstyle=solid,fillcolor=curcolor]
{
\newpath
\moveto(203.6962062,417.92023)
\lineto(198.9282062,417.92023)
\lineto(198.9282062,431.80823)
\lineto(194.5442062,431.80823)
\curveto(194.26687287,428.39489667)(193.89353954,425.64289667)(193.4242062,423.55223)
\curveto(192.9762062,421.48289667)(192.3362062,419.96823)(191.5042062,419.00823)
\curveto(190.69353954,418.06956333)(189.6162062,417.60023)(188.2722062,417.60023)
\curveto(187.16287287,417.60023)(186.2562062,417.77089667)(185.5522062,418.11223)
\lineto(185.5522062,421.92023)
\curveto(186.04287287,421.70689667)(186.55487287,421.60023)(187.0882062,421.60023)
\curveto(187.4722062,421.60023)(187.8242062,421.79223)(188.1442062,422.17623)
\curveto(188.4642062,422.56023)(188.76287287,423.25356333)(189.0402062,424.25623)
\curveto(189.33887287,425.25889667)(189.60553954,426.65623)(189.8402062,428.44823)
\curveto(190.07487287,430.26156333)(190.2882062,432.57623)(190.4802062,435.39223)
\lineto(203.6962062,435.39223)
\closepath
}
}
{
\newrgbcolor{curcolor}{0 0 0}
\pscustom[linestyle=none,fillstyle=solid,fillcolor=curcolor]
{
\newpath
\moveto(213.45622085,428.64023)
\lineto(216.81622085,428.64023)
\curveto(219.50422085,428.64023)(221.48822085,428.21356333)(222.76822085,427.36023)
\curveto(224.06955419,426.50689667)(224.72022085,425.21623)(224.72022085,423.48823)
\curveto(224.72022085,421.78156333)(224.12288752,420.42689667)(222.92822085,419.42423)
\curveto(221.73355419,418.42156333)(219.76022085,417.92023)(217.00822085,417.92023)
\lineto(208.68822085,417.92023)
\lineto(208.68822085,435.39223)
\lineto(213.45622085,435.39223)
\closepath
\moveto(219.95222085,423.42423)
\curveto(219.95222085,424.70423)(218.87488752,425.34423)(216.72022085,425.34423)
\lineto(213.45622085,425.34423)
\lineto(213.45622085,421.21623)
\lineto(216.78422085,421.21623)
\curveto(217.70155419,421.21623)(218.45888752,421.37623)(219.05622085,421.69623)
\curveto(219.65355419,422.03756333)(219.95222085,422.61356333)(219.95222085,423.42423)
\closepath
}
}
{
\newrgbcolor{curcolor}{0 0 0}
\pscustom[linestyle=none,fillstyle=solid,fillcolor=curcolor]
{
\newpath
\moveto(63.29619424,395.39223)
\lineto(68.51219424,395.39223)
\lineto(71.80819424,385.56823)
\curveto(71.97886091,385.07756333)(72.10686091,384.58689667)(72.19219424,384.09623)
\curveto(72.27752757,383.60556333)(72.34152757,383.08289667)(72.38419424,382.52823)
\lineto(72.48019424,382.52823)
\curveto(72.54419424,383.08289667)(72.62952757,383.60556333)(72.73619424,384.09623)
\curveto(72.84286091,384.58689667)(72.98152757,385.07756333)(73.15219424,385.56823)
\lineto(76.38419424,395.39223)
\lineto(81.50419424,395.39223)
\lineto(74.11219424,375.68023)
\curveto(73.42952757,373.86689667)(72.45886091,372.51223)(71.20019424,371.61623)
\curveto(69.94152757,370.69889667)(68.48019424,370.24023)(66.81619424,370.24023)
\curveto(66.26152757,370.24023)(65.79219424,370.27223)(65.40819424,370.33623)
\curveto(65.02419424,370.37889667)(64.68286091,370.43223)(64.38419424,370.49623)
\lineto(64.38419424,374.27223)
\curveto(64.59752757,374.22956333)(64.87486091,374.18689667)(65.21619424,374.14423)
\curveto(65.55752757,374.10156333)(65.90952757,374.08023)(66.27219424,374.08023)
\curveto(67.27486091,374.08023)(68.06419424,374.38956333)(68.64019424,375.00823)
\curveto(69.21619424,375.60556333)(69.65352757,376.33089667)(69.95219424,377.18423)
\lineto(70.24019424,378.04823)
\closepath
}
}
{
\newrgbcolor{curcolor}{0 0 0}
\pscustom[linestyle=none,fillstyle=solid,fillcolor=curcolor]
{
\newpath
\moveto(99.87220205,395.39223)
\lineto(99.87220205,377.92023)
\lineto(95.10420205,377.92023)
\lineto(95.10420205,391.80823)
\lineto(88.76820205,391.80823)
\lineto(88.76820205,377.92023)
\lineto(84.00020205,377.92023)
\lineto(84.00020205,395.39223)
\closepath
}
}
{
\newrgbcolor{curcolor}{0 0 0}
\pscustom[linestyle=none,fillstyle=solid,fillcolor=curcolor]
{
\newpath
\moveto(114.59221573,395.71223)
\curveto(116.55488239,395.71223)(118.14421573,394.94423)(119.36021573,393.40823)
\curveto(120.57621573,391.89356333)(121.18421573,389.65356333)(121.18421573,386.68823)
\curveto(121.18421573,383.70156333)(120.55488239,381.44023)(119.29621573,379.90423)
\curveto(118.03754906,378.36823)(116.42688239,377.60023)(114.46421573,377.60023)
\curveto(113.20554906,377.60023)(112.20288239,377.82423)(111.45621573,378.27223)
\curveto(110.70954906,378.74156333)(110.10154906,379.26423)(109.63221573,379.84023)
\lineto(109.37621573,379.84023)
\curveto(109.54688239,378.94423)(109.63221573,378.09089667)(109.63221573,377.28023)
\lineto(109.63221573,370.24023)
\lineto(104.86421573,370.24023)
\lineto(104.86421573,395.39223)
\lineto(108.73621573,395.39223)
\lineto(109.40821573,393.12023)
\lineto(109.63221573,393.12023)
\curveto(110.10154906,393.82423)(110.73088239,394.43223)(111.52021573,394.94423)
\curveto(112.30954906,395.45623)(113.33354906,395.71223)(114.59221573,395.71223)
\closepath
\moveto(113.05621573,391.90423)
\curveto(111.81888239,391.90423)(110.94421573,391.50956333)(110.43221573,390.72023)
\curveto(109.92021573,389.95223)(109.65354906,388.78956333)(109.63221573,387.23223)
\lineto(109.63221573,386.72023)
\curveto(109.63221573,385.03489667)(109.87754906,383.73356333)(110.36821573,382.81623)
\curveto(110.88021573,381.92023)(111.79754906,381.47223)(113.12021573,381.47223)
\curveto(114.20821573,381.47223)(115.00821573,381.92023)(115.52021573,382.81623)
\curveto(116.05354906,383.73356333)(116.32021573,385.04556333)(116.32021573,386.75223)
\curveto(116.32021573,390.18689667)(115.23221573,391.90423)(113.05621573,391.90423)
\closepath
}
}
{
\newrgbcolor{curcolor}{0 0 0}
\pscustom[linestyle=none,fillstyle=solid,fillcolor=curcolor]
{
\newpath
\moveto(132.28819717,395.74423)
\curveto(134.63486384,395.74423)(136.42686384,395.23223)(137.66419717,394.20823)
\curveto(138.92286384,393.20556333)(139.55219717,391.65889667)(139.55219717,389.56823)
\lineto(139.55219717,377.92023)
\lineto(136.22419717,377.92023)
\lineto(135.29619717,380.28823)
\lineto(135.16819717,380.28823)
\curveto(134.4215305,379.34956333)(133.63219717,378.66689667)(132.80019717,378.24023)
\curveto(131.96819717,377.81356333)(130.82686384,377.60023)(129.37619717,377.60023)
\curveto(127.81886384,377.60023)(126.52819717,378.04823)(125.50419717,378.94423)
\curveto(124.48019717,379.84023)(123.96819717,381.23756333)(123.96819717,383.13623)
\curveto(123.96819717,384.99223)(124.61886384,386.35756333)(125.92019717,387.23223)
\curveto(127.2215305,388.10689667)(129.1735305,388.59756333)(131.77619717,388.70423)
\lineto(134.81619717,388.80023)
\lineto(134.81619717,389.56823)
\curveto(134.81619717,390.48556333)(134.57086384,391.15756333)(134.08019717,391.58423)
\curveto(133.61086384,392.01089667)(132.9495305,392.22423)(132.09619717,392.22423)
\curveto(131.24286384,392.22423)(130.41086384,392.09623)(129.60019717,391.84023)
\curveto(128.7895305,391.60556333)(127.97886384,391.30689667)(127.16819717,390.94423)
\lineto(125.60019717,394.17623)
\curveto(126.5175305,394.64556333)(127.55219717,395.01889667)(128.70419717,395.29623)
\curveto(129.85619717,395.59489667)(131.05086384,395.74423)(132.28819717,395.74423)
\closepath
\moveto(134.81619717,386.01623)
\lineto(132.96019717,385.95223)
\curveto(131.42419717,385.90956333)(130.3575305,385.63223)(129.76019717,385.12023)
\curveto(129.16286384,384.60823)(128.86419717,383.93623)(128.86419717,383.10423)
\curveto(128.86419717,382.37889667)(129.0775305,381.85623)(129.50419717,381.53623)
\curveto(129.93086384,381.23756333)(130.4855305,381.08823)(131.16819717,381.08823)
\curveto(132.19219717,381.08823)(133.05619717,381.38689667)(133.76019717,381.98423)
\curveto(134.46419717,382.60289667)(134.81619717,383.46689667)(134.81619717,384.57623)
\closepath
}
}
{
\newrgbcolor{curcolor}{0 0 0}
\pscustom[linestyle=none,fillstyle=solid,fillcolor=curcolor]
{
\newpath
\moveto(159.9362001,390.81623)
\curveto(159.9362001,389.87756333)(159.63753343,389.07756333)(159.0402001,388.41623)
\curveto(158.4642001,387.75489667)(157.6002001,387.32823)(156.4482001,387.13623)
\lineto(156.4482001,387.00823)
\curveto(157.6642001,386.85889667)(158.63486677,386.43223)(159.3602001,385.72823)
\curveto(160.10686677,385.04556333)(160.4802001,384.18156333)(160.4802001,383.13623)
\curveto(160.4802001,382.13356333)(160.21353343,381.23756333)(159.6802001,380.44823)
\curveto(159.1682001,379.65889667)(158.34686677,379.04023)(157.2162001,378.59223)
\curveto(156.08553343,378.14423)(154.60286677,377.92023)(152.7682001,377.92023)
\lineto(144.4482001,377.92023)
\lineto(144.4482001,395.39223)
\lineto(152.7682001,395.39223)
\curveto(154.13353343,395.39223)(155.34953343,395.24289667)(156.4162001,394.94423)
\curveto(157.5042001,394.66689667)(158.35753343,394.18689667)(158.9762001,393.50423)
\curveto(159.6162001,392.84289667)(159.9362001,391.94689667)(159.9362001,390.81623)
\closepath
\moveto(155.1042001,390.43223)
\curveto(155.1042001,391.49889667)(154.26153343,392.03223)(152.5762001,392.03223)
\lineto(149.2162001,392.03223)
\lineto(149.2162001,388.57623)
\lineto(152.0322001,388.57623)
\curveto(153.03486677,388.57623)(153.7922001,388.71489667)(154.3042001,388.99223)
\curveto(154.83753343,389.29089667)(155.1042001,389.77089667)(155.1042001,390.43223)
\closepath
\moveto(155.5522001,383.39223)
\curveto(155.5522001,384.07489667)(155.27486677,384.56556333)(154.7202001,384.86423)
\curveto(154.18686677,385.18423)(153.39753343,385.34423)(152.3522001,385.34423)
\lineto(149.2162001,385.34423)
\lineto(149.2162001,381.21623)
\lineto(152.4482001,381.21623)
\curveto(153.3442001,381.21623)(154.0802001,381.37623)(154.6562001,381.69623)
\curveto(155.25353343,382.03756333)(155.5522001,382.60289667)(155.5522001,383.39223)
\closepath
}
}
{
\newrgbcolor{curcolor}{0 0 0}
\pscustom[linestyle=none,fillstyle=solid,fillcolor=curcolor]
{
\newpath
\moveto(180.22419082,377.92023)
\lineto(175.45619082,377.92023)
\lineto(175.45619082,391.80823)
\lineto(171.07219082,391.80823)
\curveto(170.79485749,388.39489667)(170.42152416,385.64289667)(169.95219082,383.55223)
\curveto(169.50419082,381.48289667)(168.86419082,379.96823)(168.03219082,379.00823)
\curveto(167.22152416,378.06956333)(166.14419082,377.60023)(164.80019082,377.60023)
\curveto(163.69085749,377.60023)(162.78419082,377.77089667)(162.08019082,378.11223)
\lineto(162.08019082,381.92023)
\curveto(162.57085749,381.70689667)(163.08285749,381.60023)(163.61619082,381.60023)
\curveto(164.00019082,381.60023)(164.35219082,381.79223)(164.67219082,382.17623)
\curveto(164.99219082,382.56023)(165.29085749,383.25356333)(165.56819082,384.25623)
\curveto(165.86685749,385.25889667)(166.13352416,386.65623)(166.36819082,388.44823)
\curveto(166.60285749,390.26156333)(166.81619082,392.57623)(167.00819082,395.39223)
\lineto(180.22419082,395.39223)
\closepath
}
}
{
\newrgbcolor{curcolor}{0 0 0}
\pscustom[linestyle=none,fillstyle=solid,fillcolor=curcolor]
{
\newpath
\moveto(192.41620547,395.71223)
\curveto(194.82687214,395.71223)(196.73620547,395.01889667)(198.14420547,393.63223)
\curveto(199.55220547,392.26689667)(200.25620547,390.31489667)(200.25620547,387.77623)
\lineto(200.25620547,385.47223)
\lineto(188.99220547,385.47223)
\curveto(189.03487214,384.12823)(189.42953881,383.07223)(190.17620547,382.30423)
\curveto(190.94420547,381.53623)(192.00020547,381.15223)(193.34420547,381.15223)
\curveto(194.45353881,381.15223)(195.46687214,381.25889667)(196.38420547,381.47223)
\curveto(197.32287214,381.70689667)(198.28287214,382.05889667)(199.26420547,382.52823)
\lineto(199.26420547,378.84823)
\curveto(198.38953881,378.42156333)(197.48287214,378.11223)(196.54420547,377.92023)
\curveto(195.60553881,377.70689667)(194.46420547,377.60023)(193.12020547,377.60023)
\curveto(191.37087214,377.60023)(189.82420547,377.92023)(188.48020547,378.56023)
\curveto(187.13620547,379.22156333)(186.08020547,380.20289667)(185.31220547,381.50423)
\curveto(184.54420547,382.82689667)(184.16020547,384.50156333)(184.16020547,386.52823)
\curveto(184.16020547,388.55489667)(184.50153881,390.25089667)(185.18420547,391.61623)
\curveto(185.88820547,392.98156333)(186.85887214,394.00556333)(188.09620547,394.68823)
\curveto(189.33353881,395.37089667)(190.77353881,395.71223)(192.41620547,395.71223)
\closepath
\moveto(192.44820547,392.32023)
\curveto(191.50953881,392.32023)(190.74153881,392.02156333)(190.14420547,391.42423)
\curveto(189.54687214,390.82689667)(189.19487214,389.89889667)(189.08820547,388.64023)
\lineto(195.77620547,388.64023)
\curveto(195.75487214,389.68556333)(195.46687214,390.56023)(194.91220547,391.26423)
\curveto(194.37887214,391.96823)(193.55753881,392.32023)(192.44820547,392.32023)
\closepath
}
}
{
\newrgbcolor{curcolor}{0 0 0}
\pscustom[linestyle=none,fillstyle=solid,fillcolor=curcolor]
{
\newpath
\moveto(208.89619278,395.39223)
\lineto(208.89619278,388.67223)
\lineto(215.55219278,388.67223)
\lineto(215.55219278,395.39223)
\lineto(220.32019278,395.39223)
\lineto(220.32019278,377.92023)
\lineto(215.55219278,377.92023)
\lineto(215.55219278,385.12023)
\lineto(208.89619278,385.12023)
\lineto(208.89619278,377.92023)
\lineto(204.12819278,377.92023)
\lineto(204.12819278,395.39223)
\closepath
}
}
{
\newrgbcolor{curcolor}{0 0 0}
\pscustom[linestyle=none,fillstyle=solid,fillcolor=curcolor]
{
\newpath
\moveto(229.92021377,395.39223)
\lineto(229.92021377,388.48023)
\curveto(229.92021377,388.11756333)(229.89888044,387.66956333)(229.85621377,387.13623)
\curveto(229.83488044,386.60289667)(229.80288044,386.05889667)(229.76021377,385.50423)
\curveto(229.73888044,384.94956333)(229.70688044,384.44823)(229.66421377,384.00023)
\curveto(229.62154711,383.57356333)(229.58954711,383.28556333)(229.56821377,383.13623)
\lineto(237.63221377,395.39223)
\lineto(243.36021377,395.39223)
\lineto(243.36021377,377.92023)
\lineto(238.75221377,377.92023)
\lineto(238.75221377,384.89623)
\curveto(238.75221377,385.45089667)(238.77354711,386.08023)(238.81621377,386.78423)
\curveto(238.85888044,387.48823)(238.90154711,388.13889667)(238.94421377,388.73623)
\curveto(239.00821377,389.35489667)(239.05088044,389.82423)(239.07221377,390.14423)
\lineto(231.04021377,377.92023)
\lineto(225.31221377,377.92023)
\lineto(225.31221377,395.39223)
\closepath
}
}
{
\newrgbcolor{curcolor}{0 0 0}
\pscustom[linestyle=none,fillstyle=solid,fillcolor=curcolor]
{
\newpath
\moveto(251.0081918,377.92023)
\lineto(245.8561918,377.92023)
\lineto(250.5601918,384.83223)
\curveto(249.6641918,385.19489667)(248.8641918,385.78156333)(248.1601918,386.59223)
\curveto(247.47752513,387.42423)(247.1361918,388.55489667)(247.1361918,389.98423)
\curveto(247.1361918,391.73356333)(247.79752513,393.06689667)(249.1201918,393.98423)
\curveto(250.44285847,394.92289667)(252.13885847,395.39223)(254.2081918,395.39223)
\lineto(262.3361918,395.39223)
\lineto(262.3361918,377.92023)
\lineto(257.5681918,377.92023)
\lineto(257.5681918,384.41623)
\lineto(254.9441918,384.41623)
\closepath
\moveto(251.8081918,389.95223)
\curveto(251.8081918,389.22689667)(252.0961918,388.65089667)(252.6721918,388.22423)
\curveto(253.2481918,387.81889667)(253.99485847,387.61623)(254.9121918,387.61623)
\lineto(257.5681918,387.61623)
\lineto(257.5681918,392.03223)
\lineto(254.3041918,392.03223)
\curveto(253.45085847,392.03223)(252.82152513,391.81889667)(252.4161918,391.39223)
\curveto(252.01085847,390.98689667)(251.8081918,390.50689667)(251.8081918,389.95223)
\closepath
}
}
{
\newrgbcolor{curcolor}{0 0 0}
\pscustom[linestyle=none,fillstyle=solid,fillcolor=curcolor]
{
\newpath
\moveto(107.45621695,355.39223)
\lineto(107.45621695,337.92023)
\lineto(102.68821695,337.92023)
\lineto(102.68821695,351.80823)
\lineto(96.35221695,351.80823)
\lineto(96.35221695,337.92023)
\lineto(91.58421695,337.92023)
\lineto(91.58421695,355.39223)
\closepath
}
}
{
\newrgbcolor{curcolor}{0 0 0}
\pscustom[linestyle=none,fillstyle=solid,fillcolor=curcolor]
{
\newpath
\moveto(122.17623062,355.71223)
\curveto(124.13889729,355.71223)(125.72823062,354.94423)(126.94423062,353.40823)
\curveto(128.16023062,351.89356333)(128.76823062,349.65356333)(128.76823062,346.68823)
\curveto(128.76823062,343.70156333)(128.13889729,341.44023)(126.88023062,339.90423)
\curveto(125.62156395,338.36823)(124.01089729,337.60023)(122.04823062,337.60023)
\curveto(120.78956395,337.60023)(119.78689729,337.82423)(119.04023062,338.27223)
\curveto(118.29356395,338.74156333)(117.68556395,339.26423)(117.21623062,339.84023)
\lineto(116.96023062,339.84023)
\curveto(117.13089729,338.94423)(117.21623062,338.09089667)(117.21623062,337.28023)
\lineto(117.21623062,330.24023)
\lineto(112.44823062,330.24023)
\lineto(112.44823062,355.39223)
\lineto(116.32023062,355.39223)
\lineto(116.99223062,353.12023)
\lineto(117.21623062,353.12023)
\curveto(117.68556395,353.82423)(118.31489729,354.43223)(119.10423062,354.94423)
\curveto(119.89356395,355.45623)(120.91756395,355.71223)(122.17623062,355.71223)
\closepath
\moveto(120.64023062,351.90423)
\curveto(119.40289729,351.90423)(118.52823062,351.50956333)(118.01623062,350.72023)
\curveto(117.50423062,349.95223)(117.23756395,348.78956333)(117.21623062,347.23223)
\lineto(117.21623062,346.72023)
\curveto(117.21623062,345.03489667)(117.46156395,343.73356333)(117.95223062,342.81623)
\curveto(118.46423062,341.92023)(119.38156395,341.47223)(120.70423062,341.47223)
\curveto(121.79223062,341.47223)(122.59223062,341.92023)(123.10423062,342.81623)
\curveto(123.63756395,343.73356333)(123.90423062,345.04556333)(123.90423062,346.75223)
\curveto(123.90423062,350.18689667)(122.81623062,351.90423)(120.64023062,351.90423)
\closepath
}
}
{
\newrgbcolor{curcolor}{0 0 0}
\pscustom[linestyle=none,fillstyle=solid,fillcolor=curcolor]
{
\newpath
\moveto(139.87221206,355.74423)
\curveto(142.21887873,355.74423)(144.01087873,355.23223)(145.24821206,354.20823)
\curveto(146.50687873,353.20556333)(147.13621206,351.65889667)(147.13621206,349.56823)
\lineto(147.13621206,337.92023)
\lineto(143.80821206,337.92023)
\lineto(142.88021206,340.28823)
\lineto(142.75221206,340.28823)
\curveto(142.0055454,339.34956333)(141.21621206,338.66689667)(140.38421206,338.24023)
\curveto(139.55221206,337.81356333)(138.41087873,337.60023)(136.96021206,337.60023)
\curveto(135.40287873,337.60023)(134.11221206,338.04823)(133.08821206,338.94423)
\curveto(132.06421206,339.84023)(131.55221206,341.23756333)(131.55221206,343.13623)
\curveto(131.55221206,344.99223)(132.20287873,346.35756333)(133.50421206,347.23223)
\curveto(134.8055454,348.10689667)(136.7575454,348.59756333)(139.36021206,348.70423)
\lineto(142.40021206,348.80023)
\lineto(142.40021206,349.56823)
\curveto(142.40021206,350.48556333)(142.15487873,351.15756333)(141.66421206,351.58423)
\curveto(141.19487873,352.01089667)(140.5335454,352.22423)(139.68021206,352.22423)
\curveto(138.82687873,352.22423)(137.99487873,352.09623)(137.18421206,351.84023)
\curveto(136.3735454,351.60556333)(135.56287873,351.30689667)(134.75221206,350.94423)
\lineto(133.18421206,354.17623)
\curveto(134.1015454,354.64556333)(135.13621206,355.01889667)(136.28821206,355.29623)
\curveto(137.44021206,355.59489667)(138.63487873,355.74423)(139.87221206,355.74423)
\closepath
\moveto(142.40021206,346.01623)
\lineto(140.54421206,345.95223)
\curveto(139.00821206,345.90956333)(137.9415454,345.63223)(137.34421206,345.12023)
\curveto(136.74687873,344.60823)(136.44821206,343.93623)(136.44821206,343.10423)
\curveto(136.44821206,342.37889667)(136.6615454,341.85623)(137.08821206,341.53623)
\curveto(137.51487873,341.23756333)(138.0695454,341.08823)(138.75221206,341.08823)
\curveto(139.77621206,341.08823)(140.64021206,341.38689667)(141.34421206,341.98423)
\curveto(142.04821206,342.60289667)(142.40021206,343.46689667)(142.40021206,344.57623)
\closepath
}
}
{
\newrgbcolor{curcolor}{0 0 0}
\pscustom[linestyle=none,fillstyle=solid,fillcolor=curcolor]
{
\newpath
\moveto(167.52021499,350.81623)
\curveto(167.52021499,349.87756333)(167.22154833,349.07756333)(166.62421499,348.41623)
\curveto(166.04821499,347.75489667)(165.18421499,347.32823)(164.03221499,347.13623)
\lineto(164.03221499,347.00823)
\curveto(165.24821499,346.85889667)(166.21888166,346.43223)(166.94421499,345.72823)
\curveto(167.69088166,345.04556333)(168.06421499,344.18156333)(168.06421499,343.13623)
\curveto(168.06421499,342.13356333)(167.79754833,341.23756333)(167.26421499,340.44823)
\curveto(166.75221499,339.65889667)(165.93088166,339.04023)(164.80021499,338.59223)
\curveto(163.66954833,338.14423)(162.18688166,337.92023)(160.35221499,337.92023)
\lineto(152.03221499,337.92023)
\lineto(152.03221499,355.39223)
\lineto(160.35221499,355.39223)
\curveto(161.71754833,355.39223)(162.93354833,355.24289667)(164.00021499,354.94423)
\curveto(165.08821499,354.66689667)(165.94154833,354.18689667)(166.56021499,353.50423)
\curveto(167.20021499,352.84289667)(167.52021499,351.94689667)(167.52021499,350.81623)
\closepath
\moveto(162.68821499,350.43223)
\curveto(162.68821499,351.49889667)(161.84554833,352.03223)(160.16021499,352.03223)
\lineto(156.80021499,352.03223)
\lineto(156.80021499,348.57623)
\lineto(159.61621499,348.57623)
\curveto(160.61888166,348.57623)(161.37621499,348.71489667)(161.88821499,348.99223)
\curveto(162.42154833,349.29089667)(162.68821499,349.77089667)(162.68821499,350.43223)
\closepath
\moveto(163.13621499,343.39223)
\curveto(163.13621499,344.07489667)(162.85888166,344.56556333)(162.30421499,344.86423)
\curveto(161.77088166,345.18423)(160.98154833,345.34423)(159.93621499,345.34423)
\lineto(156.80021499,345.34423)
\lineto(156.80021499,341.21623)
\lineto(160.03221499,341.21623)
\curveto(160.92821499,341.21623)(161.66421499,341.37623)(162.24021499,341.69623)
\curveto(162.83754833,342.03756333)(163.13621499,342.60289667)(163.13621499,343.39223)
\closepath
}
}
{
\newrgbcolor{curcolor}{0 0 0}
\pscustom[linestyle=none,fillstyle=solid,fillcolor=curcolor]
{
\newpath
\moveto(179.32820572,355.74423)
\curveto(181.67487238,355.74423)(183.46687238,355.23223)(184.70420572,354.20823)
\curveto(185.96287238,353.20556333)(186.59220572,351.65889667)(186.59220572,349.56823)
\lineto(186.59220572,337.92023)
\lineto(183.26420572,337.92023)
\lineto(182.33620572,340.28823)
\lineto(182.20820572,340.28823)
\curveto(181.46153905,339.34956333)(180.67220572,338.66689667)(179.84020572,338.24023)
\curveto(179.00820572,337.81356333)(177.86687238,337.60023)(176.41620572,337.60023)
\curveto(174.85887238,337.60023)(173.56820572,338.04823)(172.54420572,338.94423)
\curveto(171.52020572,339.84023)(171.00820572,341.23756333)(171.00820572,343.13623)
\curveto(171.00820572,344.99223)(171.65887238,346.35756333)(172.96020572,347.23223)
\curveto(174.26153905,348.10689667)(176.21353905,348.59756333)(178.81620572,348.70423)
\lineto(181.85620572,348.80023)
\lineto(181.85620572,349.56823)
\curveto(181.85620572,350.48556333)(181.61087238,351.15756333)(181.12020572,351.58423)
\curveto(180.65087238,352.01089667)(179.98953905,352.22423)(179.13620572,352.22423)
\curveto(178.28287238,352.22423)(177.45087238,352.09623)(176.64020572,351.84023)
\curveto(175.82953905,351.60556333)(175.01887238,351.30689667)(174.20820572,350.94423)
\lineto(172.64020572,354.17623)
\curveto(173.55753905,354.64556333)(174.59220572,355.01889667)(175.74420572,355.29623)
\curveto(176.89620572,355.59489667)(178.09087238,355.74423)(179.32820572,355.74423)
\closepath
\moveto(181.85620572,346.01623)
\lineto(180.00020572,345.95223)
\curveto(178.46420572,345.90956333)(177.39753905,345.63223)(176.80020572,345.12023)
\curveto(176.20287238,344.60823)(175.90420572,343.93623)(175.90420572,343.10423)
\curveto(175.90420572,342.37889667)(176.11753905,341.85623)(176.54420572,341.53623)
\curveto(176.97087238,341.23756333)(177.52553905,341.08823)(178.20820572,341.08823)
\curveto(179.23220572,341.08823)(180.09620572,341.38689667)(180.80020572,341.98423)
\curveto(181.50420572,342.60289667)(181.85620572,343.46689667)(181.85620572,344.57623)
\closepath
}
}
{
\newrgbcolor{curcolor}{0 0 0}
\pscustom[linestyle=none,fillstyle=solid,fillcolor=curcolor]
{
\newpath
\moveto(213.50420865,355.39223)
\lineto(213.50420865,337.92023)
\lineto(209.05620865,337.92023)
\lineto(209.05620865,346.49623)
\curveto(209.05620865,347.34956333)(209.06687531,348.18156333)(209.08820865,348.99223)
\curveto(209.13087531,349.80289667)(209.18420865,350.54956333)(209.24820865,351.23223)
\lineto(209.15220865,351.23223)
\lineto(204.32020865,337.92023)
\lineto(200.73620865,337.92023)
\lineto(195.84020865,351.26423)
\lineto(195.71220865,351.26423)
\curveto(195.79754198,350.56023)(195.85087531,349.80289667)(195.87220865,348.99223)
\curveto(195.91487531,348.20289667)(195.93620865,347.32823)(195.93620865,346.36823)
\lineto(195.93620865,337.92023)
\lineto(191.48820865,337.92023)
\lineto(191.48820865,355.39223)
\lineto(198.24020865,355.39223)
\lineto(202.59220865,343.55223)
\lineto(207.00820865,355.39223)
\closepath
}
}
{
\newrgbcolor{curcolor}{0 0 0}
\pscustom[linestyle=none,fillstyle=solid,fillcolor=curcolor]
{
\newpath
\moveto(223.10420425,355.39223)
\lineto(223.10420425,348.48023)
\curveto(223.10420425,348.11756333)(223.08287092,347.66956333)(223.04020425,347.13623)
\curveto(223.01887092,346.60289667)(222.98687092,346.05889667)(222.94420425,345.50423)
\curveto(222.92287092,344.94956333)(222.89087092,344.44823)(222.84820425,344.00023)
\curveto(222.80553758,343.57356333)(222.77353758,343.28556333)(222.75220425,343.13623)
\lineto(230.81620425,355.39223)
\lineto(236.54420425,355.39223)
\lineto(236.54420425,337.92023)
\lineto(231.93620425,337.92023)
\lineto(231.93620425,344.89623)
\curveto(231.93620425,345.45089667)(231.95753758,346.08023)(232.00020425,346.78423)
\curveto(232.04287092,347.48823)(232.08553758,348.13889667)(232.12820425,348.73623)
\curveto(232.19220425,349.35489667)(232.23487092,349.82423)(232.25620425,350.14423)
\lineto(224.22420425,337.92023)
\lineto(218.49620425,337.92023)
\lineto(218.49620425,355.39223)
\closepath
}
}
{
\newrgbcolor{curcolor}{0 0 0}
\pscustom[linestyle=none,fillstyle=solid,fillcolor=curcolor]
{
\newpath
\moveto(472.98351323,362.038863)
\lineto(467.47951323,362.038863)
\lineto(459.19151323,373.622863)
\lineto(459.19151323,362.038863)
\lineto(454.35951323,362.038863)
\lineto(454.35951323,384.886863)
\lineto(459.19151323,384.886863)
\lineto(459.19151323,373.814863)
\lineto(467.38351323,384.886863)
\lineto(472.53551323,384.886863)
\lineto(464.21551323,373.910863)
\closepath
}
}
{
\newrgbcolor{curcolor}{0 0 0}
\pscustom[linestyle=none,fillstyle=solid,fillcolor=curcolor]
{
\newpath
\moveto(491.12754155,362.038863)
\lineto(486.35954155,362.038863)
\lineto(486.35954155,375.926863)
\lineto(481.97554155,375.926863)
\curveto(481.69820822,372.51352967)(481.32487489,369.76152967)(480.85554155,367.670863)
\curveto(480.40754155,365.60152967)(479.76754155,364.086863)(478.93554155,363.126863)
\curveto(478.12487489,362.18819633)(477.04754155,361.718863)(475.70354155,361.718863)
\curveto(474.59420822,361.718863)(473.68754155,361.88952967)(472.98354155,362.230863)
\lineto(472.98354155,366.038863)
\curveto(473.47420822,365.82552967)(473.98620822,365.718863)(474.51954155,365.718863)
\curveto(474.90354155,365.718863)(475.25554155,365.910863)(475.57554155,366.294863)
\curveto(475.89554155,366.678863)(476.19420822,367.37219633)(476.47154155,368.374863)
\curveto(476.77020822,369.37752967)(477.03687489,370.774863)(477.27154155,372.566863)
\curveto(477.50620822,374.38019633)(477.71954155,376.694863)(477.91154155,379.510863)
\lineto(491.12754155,379.510863)
\closepath
}
}
{
\newrgbcolor{curcolor}{0 0 0}
\pscustom[linestyle=none,fillstyle=solid,fillcolor=curcolor]
{
\newpath
\moveto(500.7275562,379.510863)
\lineto(500.7275562,372.598863)
\curveto(500.7275562,372.23619633)(500.70622287,371.78819633)(500.6635562,371.254863)
\curveto(500.64222287,370.72152967)(500.61022287,370.17752967)(500.5675562,369.622863)
\curveto(500.54622287,369.06819633)(500.51422287,368.566863)(500.4715562,368.118863)
\curveto(500.42888954,367.69219633)(500.39688954,367.40419633)(500.3755562,367.254863)
\lineto(508.4395562,379.510863)
\lineto(514.1675562,379.510863)
\lineto(514.1675562,362.038863)
\lineto(509.5595562,362.038863)
\lineto(509.5595562,369.014863)
\curveto(509.5595562,369.56952967)(509.58088954,370.198863)(509.6235562,370.902863)
\curveto(509.66622287,371.606863)(509.70888954,372.25752967)(509.7515562,372.854863)
\curveto(509.8155562,373.47352967)(509.85822287,373.942863)(509.8795562,374.262863)
\lineto(501.8475562,362.038863)
\lineto(496.1195562,362.038863)
\lineto(496.1195562,379.510863)
\closepath
}
}
{
\newrgbcolor{curcolor}{0 0 0}
\pscustom[linestyle=none,fillstyle=solid,fillcolor=curcolor]
{
\newpath
\moveto(526.35953423,379.830863)
\curveto(528.7702009,379.830863)(530.67953423,379.13752967)(532.08753423,377.750863)
\curveto(533.49553423,376.38552967)(534.19953423,374.43352967)(534.19953423,371.894863)
\lineto(534.19953423,369.590863)
\lineto(522.93553423,369.590863)
\curveto(522.9782009,368.246863)(523.37286756,367.190863)(524.11953423,366.422863)
\curveto(524.88753423,365.654863)(525.94353423,365.270863)(527.28753423,365.270863)
\curveto(528.39686756,365.270863)(529.4102009,365.37752967)(530.32753423,365.590863)
\curveto(531.2662009,365.82552967)(532.2262009,366.17752967)(533.20753423,366.646863)
\lineto(533.20753423,362.966863)
\curveto(532.33286756,362.54019633)(531.4262009,362.230863)(530.48753423,362.038863)
\curveto(529.54886756,361.82552967)(528.40753423,361.718863)(527.06353423,361.718863)
\curveto(525.3142009,361.718863)(523.76753423,362.038863)(522.42353423,362.678863)
\curveto(521.07953423,363.34019633)(520.02353423,364.32152967)(519.25553423,365.622863)
\curveto(518.48753423,366.94552967)(518.10353423,368.62019633)(518.10353423,370.646863)
\curveto(518.10353423,372.67352967)(518.44486756,374.36952967)(519.12753423,375.734863)
\curveto(519.83153423,377.10019633)(520.8022009,378.12419633)(522.03953423,378.806863)
\curveto(523.27686756,379.48952967)(524.71686756,379.830863)(526.35953423,379.830863)
\closepath
\moveto(526.39153423,376.438863)
\curveto(525.45286756,376.438863)(524.68486756,376.14019633)(524.08753423,375.542863)
\curveto(523.4902009,374.94552967)(523.1382009,374.01752967)(523.03153423,372.758863)
\lineto(529.71953423,372.758863)
\curveto(529.6982009,373.80419633)(529.4102009,374.678863)(528.85553423,375.382863)
\curveto(528.3222009,376.086863)(527.50086756,376.438863)(526.39153423,376.438863)
\closepath
}
}
{
\newrgbcolor{curcolor}{0 0 0}
\pscustom[linestyle=none,fillstyle=solid,fillcolor=curcolor]
{
\newpath
\moveto(542.83952153,379.510863)
\lineto(542.83952153,372.790863)
\lineto(549.49552153,372.790863)
\lineto(549.49552153,379.510863)
\lineto(554.26352153,379.510863)
\lineto(554.26352153,362.038863)
\lineto(549.49552153,362.038863)
\lineto(549.49552153,369.238863)
\lineto(542.83952153,369.238863)
\lineto(542.83952153,362.038863)
\lineto(538.07152153,362.038863)
\lineto(538.07152153,379.510863)
\closepath
}
}
{
\newrgbcolor{curcolor}{0 0 0}
\pscustom[linestyle=none,fillstyle=solid,fillcolor=curcolor]
{
\newpath
\moveto(573.71954253,375.926863)
\lineto(567.99154253,375.926863)
\lineto(567.99154253,362.038863)
\lineto(563.22354253,362.038863)
\lineto(563.22354253,375.926863)
\lineto(557.49554253,375.926863)
\lineto(557.49554253,379.510863)
\lineto(573.71954253,379.510863)
\closepath
}
}
\end{pspicture}
}
		\caption{Общая схема клиентской части}
		\label{g5_ink1}
	\end{center}
\end{figure}

Для удобства было решено разделить страницу на три части (Рис.~\ref{g5_img1}). В первой части (левой)
располагается основная информация комнаты, панель управления правами и список участников.
Вторая часть (центральная) содержит апплет \emph{GeoGebra}.
В третьей части (правой) был размещен чат. При этом, для экономии места,
левую и правую части можно свернуть. Когда они свернуты пользователь
может отследить активность чата и, если используется комната с
ограниченными правами, запрос на предоставление прав. Отслеживание
выполняется путем мигания боковых вертикальных кнопок (с помощью которых
и сворачиваются элементы).

\begin{figure}[h]
	\begin{center}
		\includegraphics[width=\linewidth]{g5/img1.png}
		\caption{Веб-интерфейс}
		\label{g5_img1}
	\end{center}
\end{figure}

В комнате с ограниченными правами, если пользователь запрашивает доступ
с помощью специальной кнопки, то приходит уведомление владельцу и в
списке участников комнаты, напротив имени пользователя, запросившего
права, появляются две кнопки $+$ и $-$, нажимая первую, владелец передает
право редактировать доску, соответственно, нажимая вторую, он отказывает
в доступе.

Нeмaлoвaжной функцией является трансляция курсора мыши пользователя,
который в данный момент имеет доступ для редактирования доски. Эта
функция включается соответствующей кнопкой в левой панели.

Еще одной функцией является трансляция координат. Функция позволяет
полностью синхронизировать перемещение координат апплета \emph{GeoGebra},
включается она так же соответствующей кнопкой в левой панели.
Эта функция не была включена по умолчанию, т.е. активируется
только кнопкой, т.к. в некоторых случаях она мешает просмотру.


Все элементы интерфейса были написаны с использованием
шаблонизатора \emph{handlebars}. Он обеспечивает более гибкую
компоновку элементов веб-страниц.

Стоит отметить важность набора формул в чате, поэтому, используя библиотеку
\emph{KaTeX}, была добавлена поддержка синтаксиса \emph{LaTeX}.
Т.е. каждое сообщение, которое приходит пользователю проверяется
библиотекой на наличие специальных символов, в которых помещается
выражение.

Также для эффективного использования экранного места был
добавлен полноэкранный режим с помощью стандартных средств
языка \emph{JavaScript}.

Кроме основной страницы комнаты есть и вспомогательные. Они
обеспечивают простую и удобную систему входа в нее: окно приветствия,
окно создания комнаты, окно выбора типа прав и окно входа в комнату.

Вход в определенную комнату осуществляется с помощью сгенерированной
пары ключей (Рис.~\ref{g5_img2}): ключ комнаты (номер комнаты) и ключ владельца,
последний нужен для работы
ограниченной системы прав. Эти ключи передаются на сервер, и он уже
отправляет пользователя в искомую комнату.
Данные ключи можно скопировать, нажав на соответствующую кнопку
при создании комнаты или, если комната уже создана, скопировать
только ключ комнаты через кнопку в левой панели.

\begin{figure}[h]
	\begin{center}
		\includegraphics[width=\linewidth]{g5/img2.png}
		\caption{Интерфейс подключения к комнате}
		\label{g5_img2}
	\end{center}
\end{figure}

	
	\hfill \break

	\section{\centering Механизма непрерывного обмена данными между клиентом и сервером: сокеты}
	
	Помимо основной маршрутизации запрос/ответ, необходимо организовать схему
непрерывного обмена данными, в частности синхронизации апплетов всех пользователей.
Для этой работы используются веб-сокеты.

\emph{WebSocket} — протокол связи поверх TCP-соединения, предназначенный для обмена сообщениями
между браузером и веб-сервером, используя постоянное соединение.

\noindent
В настоящее время в W3C (Консорциум Всемирной паутины) осуществляется стандартизация API Web Sockets.
Черновой вариант стандарта этого протокола утверждён IETF.

\noindent
\emph{WebSocket} разработан для воплощения в веб-браузерах и веб-серверах, но он может быть
использован для любого клиентского или серверного приложения.
Протокол \emph{WebSocket} — это независимый протокол, основанный на протоколе TCP.
Он делает возможным более тесное взаимодействие между браузером и веб-сайтом,
способствуя распространению интерактивного содержимого и созданию приложений реального времени.

Для работы с веб-сокетами была использована библиотека \emph{socket.io},
специально созданная для платформы \emph{Node.js}.

В проекте веб-сокеты необходимы для взаимодействия веб-интерфейса с
сервером (Рис.~\ref{g6_ink1}), помимо основной маршрутизации фреймворка \emph{Express}.

\begin{figure}[h]
	\begin{center}
		\scalebox{0.6}{%LaTeX with PSTricks extensions
%%Creator: Inkscape 1.2 (dc2aedaf03, 2022-05-15)
%%Please note this file requires PSTricks extensions
\psset{xunit=.5pt,yunit=.5pt,runit=.5pt}
\begin{pspicture}(1024,768)
{
\newrgbcolor{curcolor}{0.50196081 0.50196081 0.50196081}
\pscustom[linestyle=none,fillstyle=solid,fillcolor=curcolor]
{
\newpath
\moveto(742.67016602,374.88122559)
\lineto(988.87597656,374.88122559)
\lineto(988.87597656,232.02107239)
\lineto(742.67016602,232.02107239)
\closepath
}
}
{
\newrgbcolor{curcolor}{0.50196081 0.50196081 0.50196081}
\pscustom[linestyle=none,fillstyle=solid,fillcolor=curcolor]
{
\newpath
\moveto(592.21105957,635.77835083)
\lineto(838.41687012,635.77835083)
\lineto(838.41687012,492.91819763)
\lineto(592.21105957,492.91819763)
\closepath
}
}
{
\newrgbcolor{curcolor}{0.50196081 0.50196081 0.50196081}
\pscustom[linestyle=none,fillstyle=solid,fillcolor=curcolor]
{
\newpath
\moveto(408.82321167,277.10818481)
\lineto(655.02902222,277.10818481)
\lineto(655.02902222,134.24803162)
\lineto(408.82321167,134.24803162)
\closepath
}
}
{
\newrgbcolor{curcolor}{0.50196081 0.50196081 0.50196081}
\pscustom[linestyle=none,fillstyle=solid,fillcolor=curcolor]
{
\newpath
\moveto(164.64379883,635.77836609)
\lineto(410.84960938,635.77836609)
\lineto(410.84960938,492.91821289)
\lineto(164.64379883,492.91821289)
\closepath
}
}
{
\newrgbcolor{curcolor}{0.50196081 0.50196081 0.50196081}
\pscustom[linestyle=none,fillstyle=solid,fillcolor=curcolor]
{
\newpath
\moveto(48.12664413,277.10818481)
\lineto(294.33245468,277.10818481)
\lineto(294.33245468,134.24803162)
\lineto(48.12664413,134.24803162)
\closepath
}
}
{
\newrgbcolor{curcolor}{0.50196081 0.50196081 0.50196081}
\pscustom[linestyle=none,fillstyle=solid,fillcolor=curcolor]
{
\newpath
\moveto(256.70527257,543.55182169)
\lineto(266.24469283,540.13782162)
\lineto(158.70369581,239.64609834)
\lineto(149.16427556,243.06009841)
\closepath
}
}
{
\newrgbcolor{curcolor}{0.50196081 0.50196081 0.50196081}
\pscustom[linestyle=none,fillstyle=solid,fillcolor=curcolor]
{
\newpath
\moveto(316.78465992,510.45248412)
\lineto(325.30889862,515.92909712)
\lineto(497.82219956,247.41559131)
\lineto(489.29796086,241.93897831)
\closepath
}
}
{
\newrgbcolor{curcolor}{0.50196081 0.50196081 0.50196081}
\pscustom[linestyle=none,fillstyle=solid,fillcolor=curcolor]
{
\newpath
\moveto(698.15065346,511.80182843)
\lineto(707.00958234,506.88498854)
\lineto(552.12913365,227.82874258)
\lineto(543.27020477,232.74558246)
\closepath
}
}
{
\newrgbcolor{curcolor}{0.50196081 0.50196081 0.50196081}
\pscustom[linestyle=none,fillstyle=solid,fillcolor=curcolor]
{
\newpath
\moveto(705.04066337,551.46836852)
\lineto(713.16396437,557.5237724)
\lineto(903.90917705,301.6398033)
\lineto(895.78587605,295.58439943)
\closepath
}
}
{
\newrgbcolor{curcolor}{0 0 0}
\pscustom[linestyle=none,fillstyle=solid,fillcolor=curcolor]
{
\newpath
\moveto(238.73911644,593.88350288)
\curveto(236.88311644,593.88350288)(235.46444978,593.19016954)(234.48311644,591.80350288)
\curveto(233.50178311,590.41683621)(233.01111644,588.51816954)(233.01111644,586.10750288)
\curveto(233.01111644,583.67550288)(233.45911644,581.78750288)(234.35511644,580.44350288)
\curveto(235.27244978,579.12083621)(236.73378311,578.45950288)(238.73911644,578.45950288)
\curveto(239.65644978,578.45950288)(240.58444978,578.56616954)(241.52311644,578.77950288)
\curveto(242.46178311,578.99283621)(243.47511644,579.29150288)(244.56311644,579.67550288)
\lineto(244.56311644,575.61150288)
\curveto(243.56044978,575.20616954)(242.56844978,574.90750288)(241.58711644,574.71550288)
\curveto(240.60578311,574.52350288)(239.50711644,574.42750288)(238.29111644,574.42750288)
\curveto(235.92311644,574.42750288)(233.98178311,574.90750288)(232.46711644,575.86750288)
\curveto(230.95244978,576.84883621)(229.83244978,578.21416954)(229.10711644,579.96350288)
\curveto(228.38178311,581.73416954)(228.01911644,583.79283621)(228.01911644,586.13950288)
\curveto(228.01911644,588.44350288)(228.43511644,590.48083621)(229.26711644,592.25150288)
\curveto(230.09911644,594.02216954)(231.30444978,595.40883621)(232.88311644,596.41150288)
\curveto(234.48311644,597.41416954)(236.43511644,597.91550288)(238.73911644,597.91550288)
\curveto(239.86978311,597.91550288)(241.00044978,597.76616954)(242.13111644,597.46750288)
\curveto(243.28311644,597.19016954)(244.38178311,596.80616954)(245.42711644,596.31550288)
\lineto(243.85911644,592.37950288)
\curveto(243.00578311,592.78483621)(242.14178311,593.13683621)(241.26711644,593.43550288)
\curveto(240.41378311,593.73416954)(239.57111644,593.88350288)(238.73911644,593.88350288)
\closepath
}
}
{
\newrgbcolor{curcolor}{0 0 0}
\pscustom[linestyle=none,fillstyle=solid,fillcolor=curcolor]
{
\newpath
\moveto(264.91508861,583.51550288)
\curveto(264.91508861,580.61416954)(264.14708861,578.37416954)(262.61108861,576.79550288)
\curveto(261.09642194,575.21683621)(259.02708861,574.42750288)(256.40308861,574.42750288)
\curveto(254.78175528,574.42750288)(253.33108861,574.77950288)(252.05108861,575.48350288)
\curveto(250.79242194,576.18750288)(249.80042194,577.21150288)(249.07508861,578.55550288)
\curveto(248.34975528,579.92083621)(247.98708861,581.57416954)(247.98708861,583.51550288)
\curveto(247.98708861,586.41683621)(248.74442194,588.64616954)(250.25908861,590.20350288)
\curveto(251.77375528,591.76083621)(253.85375528,592.53950288)(256.49908861,592.53950288)
\curveto(258.14175528,592.53950288)(259.59242194,592.18750288)(260.85108861,591.48350288)
\curveto(262.10975528,590.77950288)(263.10175528,589.75550288)(263.82708861,588.41150288)
\curveto(264.55242194,587.06750288)(264.91508861,585.43550288)(264.91508861,583.51550288)
\closepath
\moveto(252.85108861,583.51550288)
\curveto(252.85108861,581.78750288)(253.12842194,580.47550288)(253.68308861,579.57950288)
\curveto(254.25908861,578.70483621)(255.18708861,578.26750288)(256.46708861,578.26750288)
\curveto(257.72575528,578.26750288)(258.63242194,578.70483621)(259.18708861,579.57950288)
\curveto(259.76308861,580.47550288)(260.05108861,581.78750288)(260.05108861,583.51550288)
\curveto(260.05108861,585.24350288)(259.76308861,586.53416954)(259.18708861,587.38750288)
\curveto(258.63242194,588.26216954)(257.71508861,588.69950288)(256.43508861,588.69950288)
\curveto(255.17642194,588.69950288)(254.25908861,588.26216954)(253.68308861,587.38750288)
\curveto(253.12842194,586.53416954)(252.85108861,585.24350288)(252.85108861,583.51550288)
\closepath
}
}
{
\newrgbcolor{curcolor}{0 0 0}
\pscustom[linestyle=none,fillstyle=solid,fillcolor=curcolor]
{
\newpath
\moveto(280.30707201,592.21950288)
\lineto(285.55507201,592.21950288)
\lineto(278.64307201,583.83550288)
\lineto(286.16307201,574.74750288)
\lineto(280.75507201,574.74750288)
\lineto(273.61907201,583.61150288)
\lineto(273.61907201,574.74750288)
\lineto(268.85107201,574.74750288)
\lineto(268.85107201,592.21950288)
\lineto(273.61907201,592.21950288)
\lineto(273.61907201,583.73950288)
\closepath
}
}
{
\newrgbcolor{curcolor}{0 0 0}
\pscustom[linestyle=none,fillstyle=solid,fillcolor=curcolor]
{
\newpath
\moveto(295.21904076,592.53950288)
\curveto(297.62970743,592.53950288)(299.53904076,591.84616954)(300.94704076,590.45950288)
\curveto(302.35504076,589.09416954)(303.05904076,587.14216954)(303.05904076,584.60350288)
\lineto(303.05904076,582.29950288)
\lineto(291.79504076,582.29950288)
\curveto(291.83770743,580.95550288)(292.23237409,579.89950288)(292.97904076,579.13150288)
\curveto(293.74704076,578.36350288)(294.80304076,577.97950288)(296.14704076,577.97950288)
\curveto(297.25637409,577.97950288)(298.26970743,578.08616954)(299.18704076,578.29950288)
\curveto(300.12570743,578.53416954)(301.08570743,578.88616954)(302.06704076,579.35550288)
\lineto(302.06704076,575.67550288)
\curveto(301.19237409,575.24883621)(300.28570743,574.93950288)(299.34704076,574.74750288)
\curveto(298.40837409,574.53416954)(297.26704076,574.42750288)(295.92304076,574.42750288)
\curveto(294.17370743,574.42750288)(292.62704076,574.74750288)(291.28304076,575.38750288)
\curveto(289.93904076,576.04883621)(288.88304076,577.03016954)(288.11504076,578.33150288)
\curveto(287.34704076,579.65416954)(286.96304076,581.32883621)(286.96304076,583.35550288)
\curveto(286.96304076,585.38216954)(287.30437409,587.07816954)(287.98704076,588.44350288)
\curveto(288.69104076,589.80883621)(289.66170743,590.83283621)(290.89904076,591.51550288)
\curveto(292.13637409,592.19816954)(293.57637409,592.53950288)(295.21904076,592.53950288)
\closepath
\moveto(295.25104076,589.14750288)
\curveto(294.31237409,589.14750288)(293.54437409,588.84883621)(292.94704076,588.25150288)
\curveto(292.34970743,587.65416954)(291.99770743,586.72616954)(291.89104076,585.46750288)
\lineto(298.57904076,585.46750288)
\curveto(298.55770743,586.51283621)(298.26970743,587.38750288)(297.71504076,588.09150288)
\curveto(297.18170743,588.79550288)(296.36037409,589.14750288)(295.25104076,589.14750288)
\closepath
}
}
{
\newrgbcolor{curcolor}{0 0 0}
\pscustom[linestyle=none,fillstyle=solid,fillcolor=curcolor]
{
\newpath
\moveto(321.39502806,588.63550288)
\lineto(315.66702806,588.63550288)
\lineto(315.66702806,574.74750288)
\lineto(310.89902806,574.74750288)
\lineto(310.89902806,588.63550288)
\lineto(305.17102806,588.63550288)
\lineto(305.17102806,592.21950288)
\lineto(321.39502806,592.21950288)
\closepath
}
}
{
\newrgbcolor{curcolor}{0 0 0}
\pscustom[linestyle=none,fillstyle=solid,fillcolor=curcolor]
{
\newpath
\moveto(324.62701195,574.74750288)
\lineto(324.62701195,592.21950288)
\lineto(329.39501195,592.21950288)
\lineto(329.39501195,585.46750288)
\lineto(331.69901195,585.46750288)
\curveto(334.36567862,585.46750288)(336.33901195,585.04083621)(337.61901195,584.18750288)
\curveto(338.89901195,583.33416954)(339.53901195,582.04350288)(339.53901195,580.31550288)
\curveto(339.53901195,578.60883621)(338.94167862,577.25416954)(337.74701195,576.25150288)
\curveto(336.55234528,575.24883621)(334.58967862,574.74750288)(331.85901195,574.74750288)
\closepath
\moveto(342.06701195,574.74750288)
\lineto(342.06701195,592.21950288)
\lineto(346.83501195,592.21950288)
\lineto(346.83501195,574.74750288)
\closepath
\moveto(329.39501195,578.04350288)
\lineto(331.60301195,578.04350288)
\curveto(332.54167862,578.04350288)(333.29901195,578.20350288)(333.87501195,578.52350288)
\curveto(334.47234528,578.86483621)(334.77101195,579.44083621)(334.77101195,580.25150288)
\curveto(334.77101195,581.53150288)(333.69367862,582.17150288)(331.53901195,582.17150288)
\lineto(329.39501195,582.17150288)
\closepath
}
}
{
\newrgbcolor{curcolor}{0 0 0}
\pscustom[linestyle=none,fillstyle=solid,fillcolor=curcolor]
{
\newpath
\moveto(205.57104613,543.51550288)
\curveto(205.57104613,546.11816954)(205.94437946,548.62483621)(206.69104613,551.03550288)
\curveto(207.45904613,553.46750288)(208.6537128,555.65416954)(210.27504613,557.59550288)
\lineto(214.17904613,557.59550288)
\curveto(212.72837946,555.59016954)(211.61904613,553.38216954)(210.85104613,550.97150288)
\curveto(210.10437946,548.56083621)(209.73104613,546.08616954)(209.73104613,543.54750288)
\curveto(209.73104613,541.07283621)(210.10437946,538.63016954)(210.85104613,536.21950288)
\curveto(211.5977128,533.83016954)(212.69637946,531.65416954)(214.14704613,529.69150288)
\lineto(210.27504613,529.69150288)
\curveto(208.6537128,531.56883621)(207.45904613,533.69150288)(206.69104613,536.05950288)
\curveto(205.94437946,538.44883621)(205.57104613,540.93416954)(205.57104613,543.51550288)
\closepath
}
}
{
\newrgbcolor{curcolor}{0 0 0}
\pscustom[linestyle=none,fillstyle=solid,fillcolor=curcolor]
{
\newpath
\moveto(231.49106859,541.08350288)
\curveto(231.49106859,539.05683621)(230.75506859,537.43550288)(229.28306859,536.21950288)
\curveto(227.83240192,535.02483621)(225.76306859,534.42750288)(223.07506859,534.42750288)
\curveto(220.66440192,534.42750288)(218.50973526,534.88616954)(216.61106859,535.80350288)
\lineto(216.61106859,540.31550288)
\curveto(217.69906859,539.84616954)(218.81906859,539.40883621)(219.97106859,539.00350288)
\curveto(221.14440192,538.61950288)(222.30706859,538.42750288)(223.45906859,538.42750288)
\curveto(224.65373526,538.42750288)(225.49640192,538.65150288)(225.98706859,539.09950288)
\curveto(226.49906859,539.56883621)(226.75506859,540.15550288)(226.75506859,540.85950288)
\curveto(226.75506859,541.43550288)(226.55240192,541.92616954)(226.14706859,542.33150288)
\curveto(225.76306859,542.73683621)(225.24040192,543.11016954)(224.57906859,543.45150288)
\curveto(223.91773526,543.81416954)(223.16040192,544.19816954)(222.30706859,544.60350288)
\curveto(221.77373526,544.85950288)(221.19773526,545.15816954)(220.57906859,545.49950288)
\curveto(219.96040192,545.86216954)(219.36306859,546.29950288)(218.78706859,546.81150288)
\curveto(218.23240192,547.34483621)(217.77373526,547.98483621)(217.41106859,548.73150288)
\curveto(217.04840192,549.47816954)(216.86706859,550.37416954)(216.86706859,551.41950288)
\curveto(216.86706859,553.46750288)(217.56040192,555.05683621)(218.94706859,556.18750288)
\curveto(220.33373526,557.33950288)(222.22173526,557.91550288)(224.61106859,557.91550288)
\curveto(225.80573526,557.91550288)(226.93640192,557.77683621)(228.00306859,557.49950288)
\curveto(229.06973526,557.22216954)(230.20040192,556.82750288)(231.39506859,556.31550288)
\lineto(229.82706859,552.53950288)
\curveto(228.78173526,552.96616954)(227.84306859,553.29683621)(227.01106859,553.53150288)
\curveto(226.17906859,553.76616954)(225.32573526,553.88350288)(224.45106859,553.88350288)
\curveto(223.53373526,553.88350288)(222.82973526,553.67016954)(222.33906859,553.24350288)
\curveto(221.84840192,552.81683621)(221.60306859,552.26216954)(221.60306859,551.57950288)
\curveto(221.60306859,550.76883621)(221.96573526,550.12883621)(222.69106859,549.65950288)
\curveto(223.41640192,549.19016954)(224.49373526,548.61416954)(225.92306859,547.93150288)
\curveto(227.09640192,547.37683621)(228.08840192,546.80083621)(228.89906859,546.20350288)
\curveto(229.73106859,545.60616954)(230.37106859,544.90216954)(230.81906859,544.09150288)
\curveto(231.26706859,543.28083621)(231.49106859,542.27816954)(231.49106859,541.08350288)
\closepath
}
}
{
\newrgbcolor{curcolor}{0 0 0}
\pscustom[linestyle=none,fillstyle=solid,fillcolor=curcolor]
{
\newpath
\moveto(251.13908763,543.51550288)
\curveto(251.13908763,540.61416954)(250.37108763,538.37416954)(248.83508763,536.79550288)
\curveto(247.32042097,535.21683621)(245.25108763,534.42750288)(242.62708763,534.42750288)
\curveto(241.0057543,534.42750288)(239.55508763,534.77950288)(238.27508763,535.48350288)
\curveto(237.01642097,536.18750288)(236.02442097,537.21150288)(235.29908763,538.55550288)
\curveto(234.5737543,539.92083621)(234.21108763,541.57416954)(234.21108763,543.51550288)
\curveto(234.21108763,546.41683621)(234.96842097,548.64616954)(236.48308763,550.20350288)
\curveto(237.9977543,551.76083621)(240.0777543,552.53950288)(242.72308763,552.53950288)
\curveto(244.3657543,552.53950288)(245.81642097,552.18750288)(247.07508763,551.48350288)
\curveto(248.3337543,550.77950288)(249.3257543,549.75550288)(250.05108763,548.41150288)
\curveto(250.77642097,547.06750288)(251.13908763,545.43550288)(251.13908763,543.51550288)
\closepath
\moveto(239.07508763,543.51550288)
\curveto(239.07508763,541.78750288)(239.35242097,540.47550288)(239.90708763,539.57950288)
\curveto(240.48308763,538.70483621)(241.41108763,538.26750288)(242.69108763,538.26750288)
\curveto(243.9497543,538.26750288)(244.85642097,538.70483621)(245.41108763,539.57950288)
\curveto(245.98708763,540.47550288)(246.27508763,541.78750288)(246.27508763,543.51550288)
\curveto(246.27508763,545.24350288)(245.98708763,546.53416954)(245.41108763,547.38750288)
\curveto(244.85642097,548.26216954)(243.93908763,548.69950288)(242.65908763,548.69950288)
\curveto(241.40042097,548.69950288)(240.48308763,548.26216954)(239.90708763,547.38750288)
\curveto(239.35242097,546.53416954)(239.07508763,545.24350288)(239.07508763,543.51550288)
\closepath
}
}
{
\newrgbcolor{curcolor}{0 0 0}
\pscustom[linestyle=none,fillstyle=solid,fillcolor=curcolor]
{
\newpath
\moveto(262.17907103,534.42750288)
\curveto(259.57640437,534.42750288)(257.56040437,535.14216954)(256.13107103,536.57150288)
\curveto(254.72307103,538.00083621)(254.01907103,540.27283621)(254.01907103,543.38750288)
\curveto(254.01907103,545.52083621)(254.3817377,547.25950288)(255.10707103,548.60350288)
\curveto(255.83240437,549.94750288)(256.83507103,550.93950288)(258.11507103,551.57950288)
\curveto(259.41640437,552.21950288)(260.9097377,552.53950288)(262.59507103,552.53950288)
\curveto(263.7897377,552.53950288)(264.82440437,552.42216954)(265.69907103,552.18750288)
\curveto(266.59507103,551.95283621)(267.3737377,551.67550288)(268.03507103,551.35550288)
\lineto(266.62707103,547.67550288)
\curveto(265.88040437,547.97416954)(265.17640437,548.21950288)(264.51507103,548.41150288)
\curveto(263.87507103,548.60350288)(263.23507103,548.69950288)(262.59507103,548.69950288)
\curveto(260.12040437,548.69950288)(258.88307103,546.93950288)(258.88307103,543.41950288)
\curveto(258.88307103,541.67016954)(259.20307103,540.37950288)(259.84307103,539.54750288)
\curveto(260.50440437,538.71550288)(261.4217377,538.29950288)(262.59507103,538.29950288)
\curveto(263.5977377,538.29950288)(264.48307103,538.42750288)(265.25107103,538.68350288)
\curveto(266.01907103,538.96083621)(266.7657377,539.33416954)(267.49107103,539.80350288)
\lineto(267.49107103,535.73950288)
\curveto(266.7657377,535.27016954)(265.9977377,534.93950288)(265.18707103,534.74750288)
\curveto(264.3977377,534.53416954)(263.39507103,534.42750288)(262.17907103,534.42750288)
\closepath
}
}
{
\newrgbcolor{curcolor}{0 0 0}
\pscustom[linestyle=none,fillstyle=solid,fillcolor=curcolor]
{
\newpath
\moveto(276.29106908,559.06750288)
\lineto(276.29106908,548.18750288)
\curveto(276.29106908,547.52616954)(276.25906908,546.86483621)(276.19506908,546.20350288)
\curveto(276.15240241,545.56350288)(276.09906908,544.91283621)(276.03506908,544.25150288)
\lineto(276.09906908,544.25150288)
\curveto(276.41906908,544.69950288)(276.74973575,545.14750288)(277.09106908,545.59550288)
\curveto(277.43240241,546.04350288)(277.79506908,546.48083621)(278.17906908,546.90750288)
\lineto(283.07506908,552.21950288)
\lineto(288.45106908,552.21950288)
\lineto(281.50706908,544.63550288)
\lineto(288.86706908,534.74750288)
\lineto(283.36306908,534.74750288)
\lineto(278.33906908,541.81950288)
\lineto(276.29106908,540.18750288)
\lineto(276.29106908,534.74750288)
\lineto(271.52306908,534.74750288)
\lineto(271.52306908,559.06750288)
\closepath
}
}
{
\newrgbcolor{curcolor}{0 0 0}
\pscustom[linestyle=none,fillstyle=solid,fillcolor=curcolor]
{
\newpath
\moveto(298.56309594,552.53950288)
\curveto(300.9737626,552.53950288)(302.88309594,551.84616954)(304.29109594,550.45950288)
\curveto(305.69909594,549.09416954)(306.40309594,547.14216954)(306.40309594,544.60350288)
\lineto(306.40309594,542.29950288)
\lineto(295.13909594,542.29950288)
\curveto(295.1817626,540.95550288)(295.57642927,539.89950288)(296.32309594,539.13150288)
\curveto(297.09109594,538.36350288)(298.14709594,537.97950288)(299.49109594,537.97950288)
\curveto(300.60042927,537.97950288)(301.6137626,538.08616954)(302.53109594,538.29950288)
\curveto(303.4697626,538.53416954)(304.4297626,538.88616954)(305.41109594,539.35550288)
\lineto(305.41109594,535.67550288)
\curveto(304.53642927,535.24883621)(303.6297626,534.93950288)(302.69109594,534.74750288)
\curveto(301.75242927,534.53416954)(300.61109594,534.42750288)(299.26709594,534.42750288)
\curveto(297.5177626,534.42750288)(295.97109594,534.74750288)(294.62709594,535.38750288)
\curveto(293.28309594,536.04883621)(292.22709594,537.03016954)(291.45909594,538.33150288)
\curveto(290.69109594,539.65416954)(290.30709594,541.32883621)(290.30709594,543.35550288)
\curveto(290.30709594,545.38216954)(290.64842927,547.07816954)(291.33109594,548.44350288)
\curveto(292.03509594,549.80883621)(293.0057626,550.83283621)(294.24309594,551.51550288)
\curveto(295.48042927,552.19816954)(296.92042927,552.53950288)(298.56309594,552.53950288)
\closepath
\moveto(298.59509594,549.14750288)
\curveto(297.65642927,549.14750288)(296.88842927,548.84883621)(296.29109594,548.25150288)
\curveto(295.6937626,547.65416954)(295.3417626,546.72616954)(295.23509594,545.46750288)
\lineto(301.92309594,545.46750288)
\curveto(301.9017626,546.51283621)(301.6137626,547.38750288)(301.05909594,548.09150288)
\curveto(300.5257626,548.79550288)(299.70442927,549.14750288)(298.59509594,549.14750288)
\closepath
}
}
{
\newrgbcolor{curcolor}{0 0 0}
\pscustom[linestyle=none,fillstyle=solid,fillcolor=curcolor]
{
\newpath
\moveto(317.63508324,538.23550288)
\curveto(318.16841657,538.23550288)(318.68041657,538.27816954)(319.17108324,538.36350288)
\curveto(319.66174991,538.47016954)(320.15241657,538.60883621)(320.64308324,538.77950288)
\lineto(320.64308324,535.22750288)
\curveto(320.13108324,534.99283621)(319.49108324,534.80083621)(318.72308324,534.65150288)
\curveto(317.97641657,534.50216954)(317.15508324,534.42750288)(316.25908324,534.42750288)
\curveto(315.21374991,534.42750288)(314.27508324,534.59816954)(313.44308324,534.93950288)
\curveto(312.63241657,535.28083621)(311.98174991,535.86750288)(311.49108324,536.69950288)
\curveto(311.02174991,537.53150288)(310.78708324,538.70483621)(310.78708324,540.21950288)
\lineto(310.78708324,548.63550288)
\lineto(308.51508324,548.63550288)
\lineto(308.51508324,550.65150288)
\lineto(311.13908324,552.25150288)
\lineto(312.51508324,555.93150288)
\lineto(315.55508324,555.93150288)
\lineto(315.55508324,552.21950288)
\lineto(320.45108324,552.21950288)
\lineto(320.45108324,548.63550288)
\lineto(315.55508324,548.63550288)
\lineto(315.55508324,540.21950288)
\curveto(315.55508324,539.55816954)(315.74708324,539.05683621)(316.13108324,538.71550288)
\curveto(316.51508324,538.39550288)(317.01641657,538.23550288)(317.63508324,538.23550288)
\closepath
}
}
{
\newrgbcolor{curcolor}{0 0 0}
\pscustom[linestyle=none,fillstyle=solid,fillcolor=curcolor]
{
\newpath
\moveto(323.49108373,536.98750288)
\curveto(323.49108373,537.96883621)(323.75775039,538.65150288)(324.29108373,539.03550288)
\curveto(324.82441706,539.44083621)(325.47508373,539.64350288)(326.24308373,539.64350288)
\curveto(326.98975039,539.64350288)(327.62975039,539.44083621)(328.16308373,539.03550288)
\curveto(328.69641706,538.65150288)(328.96308373,537.96883621)(328.96308373,536.98750288)
\curveto(328.96308373,536.04883621)(328.69641706,535.36616954)(328.16308373,534.93950288)
\curveto(327.62975039,534.53416954)(326.98975039,534.33150288)(326.24308373,534.33150288)
\curveto(325.47508373,534.33150288)(324.82441706,534.53416954)(324.29108373,534.93950288)
\curveto(323.75775039,535.36616954)(323.49108373,536.04883621)(323.49108373,536.98750288)
\closepath
}
}
{
\newrgbcolor{curcolor}{0 0 0}
\pscustom[linestyle=none,fillstyle=solid,fillcolor=curcolor]
{
\newpath
\moveto(335.68307885,559.06750288)
\curveto(336.38707885,559.06750288)(336.99507885,558.89683621)(337.50707885,558.55550288)
\curveto(338.01907885,558.23550288)(338.27507885,557.62750288)(338.27507885,556.73150288)
\curveto(338.27507885,555.85683621)(338.01907885,555.24883621)(337.50707885,554.90750288)
\curveto(336.99507885,554.56616954)(336.38707885,554.39550288)(335.68307885,554.39550288)
\curveto(334.95774551,554.39550288)(334.33907885,554.56616954)(333.82707885,554.90750288)
\curveto(333.33641218,555.24883621)(333.09107885,555.85683621)(333.09107885,556.73150288)
\curveto(333.09107885,557.62750288)(333.33641218,558.23550288)(333.82707885,558.55550288)
\curveto(334.33907885,558.89683621)(334.95774551,559.06750288)(335.68307885,559.06750288)
\closepath
\moveto(338.05107885,552.21950288)
\lineto(338.05107885,534.74750288)
\lineto(333.28307885,534.74750288)
\lineto(333.28307885,552.21950288)
\closepath
}
}
{
\newrgbcolor{curcolor}{0 0 0}
\pscustom[linestyle=none,fillstyle=solid,fillcolor=curcolor]
{
\newpath
\moveto(358.91508861,543.51550288)
\curveto(358.91508861,540.61416954)(358.14708861,538.37416954)(356.61108861,536.79550288)
\curveto(355.09642194,535.21683621)(353.02708861,534.42750288)(350.40308861,534.42750288)
\curveto(348.78175528,534.42750288)(347.33108861,534.77950288)(346.05108861,535.48350288)
\curveto(344.79242194,536.18750288)(343.80042194,537.21150288)(343.07508861,538.55550288)
\curveto(342.34975528,539.92083621)(341.98708861,541.57416954)(341.98708861,543.51550288)
\curveto(341.98708861,546.41683621)(342.74442194,548.64616954)(344.25908861,550.20350288)
\curveto(345.77375528,551.76083621)(347.85375528,552.53950288)(350.49908861,552.53950288)
\curveto(352.14175528,552.53950288)(353.59242194,552.18750288)(354.85108861,551.48350288)
\curveto(356.10975528,550.77950288)(357.10175528,549.75550288)(357.82708861,548.41150288)
\curveto(358.55242194,547.06750288)(358.91508861,545.43550288)(358.91508861,543.51550288)
\closepath
\moveto(346.85108861,543.51550288)
\curveto(346.85108861,541.78750288)(347.12842194,540.47550288)(347.68308861,539.57950288)
\curveto(348.25908861,538.70483621)(349.18708861,538.26750288)(350.46708861,538.26750288)
\curveto(351.72575528,538.26750288)(352.63242194,538.70483621)(353.18708861,539.57950288)
\curveto(353.76308861,540.47550288)(354.05108861,541.78750288)(354.05108861,543.51550288)
\curveto(354.05108861,545.24350288)(353.76308861,546.53416954)(353.18708861,547.38750288)
\curveto(352.63242194,548.26216954)(351.71508861,548.69950288)(350.43508861,548.69950288)
\curveto(349.17642194,548.69950288)(348.25908861,548.26216954)(347.68308861,547.38750288)
\curveto(347.12842194,546.53416954)(346.85108861,545.24350288)(346.85108861,543.51550288)
\closepath
}
}
{
\newrgbcolor{curcolor}{0 0 0}
\pscustom[linestyle=none,fillstyle=solid,fillcolor=curcolor]
{
\newpath
\moveto(369.92307201,543.51550288)
\curveto(369.92307201,540.93416954)(369.53907201,538.44883621)(368.77107201,536.05950288)
\curveto(368.02440534,533.69150288)(366.84040534,531.56883621)(365.21907201,529.69150288)
\lineto(361.34707201,529.69150288)
\curveto(362.77640534,531.65416954)(363.86440534,533.83016954)(364.61107201,536.21950288)
\curveto(365.37907201,538.63016954)(365.76307201,541.07283621)(365.76307201,543.54750288)
\curveto(365.76307201,546.08616954)(365.37907201,548.56083621)(364.61107201,550.97150288)
\curveto(363.86440534,553.38216954)(362.76573868,555.59016954)(361.31507201,557.59550288)
\lineto(365.21907201,557.59550288)
\curveto(366.84040534,555.65416954)(368.02440534,553.46750288)(368.77107201,551.03550288)
\curveto(369.53907201,548.62483621)(369.92307201,546.11816954)(369.92307201,543.51550288)
\closepath
}
}
{
\newrgbcolor{curcolor}{0 0 0}
\pscustom[linestyle=none,fillstyle=solid,fillcolor=curcolor]
{
\newpath
\moveto(129.2191004,216.07562688)
\lineto(123.7151004,216.07562688)
\lineto(115.4271004,227.65962688)
\lineto(115.4271004,216.07562688)
\lineto(110.5951004,216.07562688)
\lineto(110.5951004,238.92362688)
\lineto(115.4271004,238.92362688)
\lineto(115.4271004,227.85162688)
\lineto(123.6191004,238.92362688)
\lineto(128.7711004,238.92362688)
\lineto(120.4511004,227.94762688)
\closepath
}
}
{
\newrgbcolor{curcolor}{0 0 0}
\pscustom[linestyle=none,fillstyle=solid,fillcolor=curcolor]
{
\newpath
\moveto(147.36312872,216.07562688)
\lineto(142.59512872,216.07562688)
\lineto(142.59512872,229.96362688)
\lineto(138.21112872,229.96362688)
\curveto(137.93379538,226.55029354)(137.56046205,223.79829354)(137.09112872,221.70762688)
\curveto(136.64312872,219.63829354)(136.00312872,218.12362688)(135.17112872,217.16362688)
\curveto(134.36046205,216.22496021)(133.28312872,215.75562688)(131.93912872,215.75562688)
\curveto(130.82979538,215.75562688)(129.92312872,215.92629354)(129.21912872,216.26762688)
\lineto(129.21912872,220.07562688)
\curveto(129.70979538,219.86229354)(130.22179538,219.75562688)(130.75512872,219.75562688)
\curveto(131.13912872,219.75562688)(131.49112872,219.94762688)(131.81112872,220.33162688)
\curveto(132.13112872,220.71562688)(132.42979538,221.40896021)(132.70712872,222.41162688)
\curveto(133.00579538,223.41429354)(133.27246205,224.81162688)(133.50712872,226.60362688)
\curveto(133.74179538,228.41696021)(133.95512872,230.73162688)(134.14712872,233.54762688)
\lineto(147.36312872,233.54762688)
\closepath
}
}
{
\newrgbcolor{curcolor}{0 0 0}
\pscustom[linestyle=none,fillstyle=solid,fillcolor=curcolor]
{
\newpath
\moveto(156.96314336,233.54762688)
\lineto(156.96314336,226.63562688)
\curveto(156.96314336,226.27296021)(156.94181003,225.82496021)(156.89914336,225.29162688)
\curveto(156.87781003,224.75829354)(156.84581003,224.21429354)(156.80314336,223.65962688)
\curveto(156.78181003,223.10496021)(156.74981003,222.60362688)(156.70714336,222.15562688)
\curveto(156.6644767,221.72896021)(156.6324767,221.44096021)(156.61114336,221.29162688)
\lineto(164.67514336,233.54762688)
\lineto(170.40314336,233.54762688)
\lineto(170.40314336,216.07562688)
\lineto(165.79514336,216.07562688)
\lineto(165.79514336,223.05162688)
\curveto(165.79514336,223.60629354)(165.8164767,224.23562688)(165.85914336,224.93962688)
\curveto(165.90181003,225.64362688)(165.9444767,226.29429354)(165.98714336,226.89162688)
\curveto(166.05114336,227.51029354)(166.09381003,227.97962688)(166.11514336,228.29962688)
\lineto(158.08314336,216.07562688)
\lineto(152.35514336,216.07562688)
\lineto(152.35514336,233.54762688)
\closepath
}
}
{
\newrgbcolor{curcolor}{0 0 0}
\pscustom[linestyle=none,fillstyle=solid,fillcolor=curcolor]
{
\newpath
\moveto(182.59512139,233.86762688)
\curveto(185.00578806,233.86762688)(186.91512139,233.17429354)(188.32312139,231.78762688)
\curveto(189.73112139,230.42229354)(190.43512139,228.47029354)(190.43512139,225.93162688)
\lineto(190.43512139,223.62762688)
\lineto(179.17112139,223.62762688)
\curveto(179.21378806,222.28362688)(179.60845473,221.22762688)(180.35512139,220.45962688)
\curveto(181.12312139,219.69162688)(182.17912139,219.30762688)(183.52312139,219.30762688)
\curveto(184.63245473,219.30762688)(185.64578806,219.41429354)(186.56312139,219.62762688)
\curveto(187.50178806,219.86229354)(188.46178806,220.21429354)(189.44312139,220.68362688)
\lineto(189.44312139,217.00362688)
\curveto(188.56845473,216.57696021)(187.66178806,216.26762688)(186.72312139,216.07562688)
\curveto(185.78445473,215.86229354)(184.64312139,215.75562688)(183.29912139,215.75562688)
\curveto(181.54978806,215.75562688)(180.00312139,216.07562688)(178.65912139,216.71562688)
\curveto(177.31512139,217.37696021)(176.25912139,218.35829354)(175.49112139,219.65962688)
\curveto(174.72312139,220.98229354)(174.33912139,222.65696021)(174.33912139,224.68362688)
\curveto(174.33912139,226.71029354)(174.68045473,228.40629354)(175.36312139,229.77162688)
\curveto(176.06712139,231.13696021)(177.03778806,232.16096021)(178.27512139,232.84362688)
\curveto(179.51245473,233.52629354)(180.95245473,233.86762688)(182.59512139,233.86762688)
\closepath
\moveto(182.62712139,230.47562688)
\curveto(181.68845473,230.47562688)(180.92045473,230.17696021)(180.32312139,229.57962688)
\curveto(179.72578806,228.98229354)(179.37378806,228.05429354)(179.26712139,226.79562688)
\lineto(185.95512139,226.79562688)
\curveto(185.93378806,227.84096021)(185.64578806,228.71562688)(185.09112139,229.41962688)
\curveto(184.55778806,230.12362688)(183.73645473,230.47562688)(182.62712139,230.47562688)
\closepath
}
}
{
\newrgbcolor{curcolor}{0 0 0}
\pscustom[linestyle=none,fillstyle=solid,fillcolor=curcolor]
{
\newpath
\moveto(199.0751087,233.54762688)
\lineto(199.0751087,226.82762688)
\lineto(205.7311087,226.82762688)
\lineto(205.7311087,233.54762688)
\lineto(210.4991087,233.54762688)
\lineto(210.4991087,216.07562688)
\lineto(205.7311087,216.07562688)
\lineto(205.7311087,223.27562688)
\lineto(199.0751087,223.27562688)
\lineto(199.0751087,216.07562688)
\lineto(194.3071087,216.07562688)
\lineto(194.3071087,233.54762688)
\closepath
}
}
{
\newrgbcolor{curcolor}{0 0 0}
\pscustom[linestyle=none,fillstyle=solid,fillcolor=curcolor]
{
\newpath
\moveto(229.95512969,229.96362688)
\lineto(224.22712969,229.96362688)
\lineto(224.22712969,216.07562688)
\lineto(219.45912969,216.07562688)
\lineto(219.45912969,229.96362688)
\lineto(213.73112969,229.96362688)
\lineto(213.73112969,233.54762688)
\lineto(229.95512969,233.54762688)
\closepath
}
}
{
\newrgbcolor{curcolor}{0 0 0}
\pscustom[linestyle=none,fillstyle=solid,fillcolor=curcolor]
{
\newpath
\moveto(92.06709136,184.84362688)
\curveto(92.06709136,187.44629354)(92.4404247,189.95296021)(93.18709136,192.36362688)
\curveto(93.95509136,194.79562688)(95.14975803,196.98229354)(96.77109136,198.92362688)
\lineto(100.67509136,198.92362688)
\curveto(99.2244247,196.91829354)(98.11509136,194.71029354)(97.34709136,192.29962688)
\curveto(96.6004247,189.88896021)(96.22709136,187.41429354)(96.22709136,184.87562688)
\curveto(96.22709136,182.40096021)(96.6004247,179.95829354)(97.34709136,177.54762688)
\curveto(98.09375803,175.15829354)(99.1924247,172.98229354)(100.64309136,171.01962688)
\lineto(96.77109136,171.01962688)
\curveto(95.14975803,172.89696021)(93.95509136,175.01962688)(93.18709136,177.38762688)
\curveto(92.4404247,179.77696021)(92.06709136,182.26229354)(92.06709136,184.84362688)
\closepath
}
}
{
\newrgbcolor{curcolor}{0 0 0}
\pscustom[linestyle=none,fillstyle=solid,fillcolor=curcolor]
{
\newpath
\moveto(104.51511382,176.07562688)
\lineto(104.51511382,198.92362688)
\lineto(118.94711382,198.92362688)
\lineto(118.94711382,194.92362688)
\lineto(109.34711382,194.92362688)
\lineto(109.34711382,190.15562688)
\lineto(111.26711382,190.15562688)
\curveto(113.42178049,190.15562688)(115.18178049,189.85696021)(116.54711382,189.25962688)
\curveto(117.93378049,188.66229354)(118.95778049,187.84096021)(119.61911382,186.79562688)
\curveto(120.28044716,185.75029354)(120.61111382,184.55562688)(120.61111382,183.21162688)
\curveto(120.61111382,180.95029354)(119.85378049,179.19029354)(118.33911382,177.93162688)
\curveto(116.84578049,176.69429354)(114.45644716,176.07562688)(111.17111382,176.07562688)
\closepath
\moveto(109.34711382,180.04362688)
\lineto(110.97911382,180.04362688)
\curveto(112.45111382,180.04362688)(113.60311382,180.27829354)(114.43511382,180.74762688)
\curveto(115.28844716,181.21696021)(115.71511382,182.03829354)(115.71511382,183.21162688)
\curveto(115.71511382,184.42762688)(115.25644716,185.22762688)(114.33911382,185.61162688)
\curveto(113.42178049,185.99562688)(112.17378049,186.18762688)(110.59511382,186.18762688)
\lineto(109.34711382,186.18762688)
\closepath
}
}
{
\newrgbcolor{curcolor}{0 0 0}
\pscustom[linestyle=none,fillstyle=solid,fillcolor=curcolor]
{
\newpath
\moveto(134.27512945,193.86762688)
\curveto(136.23779612,193.86762688)(137.82712945,193.09962688)(139.04312945,191.56362688)
\curveto(140.25912945,190.04896021)(140.86712945,187.80896021)(140.86712945,184.84362688)
\curveto(140.86712945,181.85696021)(140.23779612,179.59562688)(138.97912945,178.05962688)
\curveto(137.72046278,176.52362688)(136.10979612,175.75562688)(134.14712945,175.75562688)
\curveto(132.88846278,175.75562688)(131.88579612,175.97962688)(131.13912945,176.42762688)
\curveto(130.39246278,176.89696021)(129.78446278,177.41962688)(129.31512945,177.99562688)
\lineto(129.05912945,177.99562688)
\curveto(129.22979612,177.09962688)(129.31512945,176.24629354)(129.31512945,175.43562688)
\lineto(129.31512945,168.39562688)
\lineto(124.54712945,168.39562688)
\lineto(124.54712945,193.54762688)
\lineto(128.41912945,193.54762688)
\lineto(129.09112945,191.27562688)
\lineto(129.31512945,191.27562688)
\curveto(129.78446278,191.97962688)(130.41379612,192.58762688)(131.20312945,193.09962688)
\curveto(131.99246278,193.61162688)(133.01646278,193.86762688)(134.27512945,193.86762688)
\closepath
\moveto(132.73912945,190.05962688)
\curveto(131.50179612,190.05962688)(130.62712945,189.66496021)(130.11512945,188.87562688)
\curveto(129.60312945,188.10762688)(129.33646278,186.94496021)(129.31512945,185.38762688)
\lineto(129.31512945,184.87562688)
\curveto(129.31512945,183.19029354)(129.56046278,181.88896021)(130.05112945,180.97162688)
\curveto(130.56312945,180.07562688)(131.48046278,179.62762688)(132.80312945,179.62762688)
\curveto(133.89112945,179.62762688)(134.69112945,180.07562688)(135.20312945,180.97162688)
\curveto(135.73646278,181.88896021)(136.00312945,183.20096021)(136.00312945,184.90762688)
\curveto(136.00312945,188.34229354)(134.91512945,190.05962688)(132.73912945,190.05962688)
\closepath
}
}
{
\newrgbcolor{curcolor}{0 0 0}
\pscustom[linestyle=none,fillstyle=solid,fillcolor=curcolor]
{
\newpath
\moveto(151.97111089,193.89962688)
\curveto(154.31777756,193.89962688)(156.10977756,193.38762688)(157.34711089,192.36362688)
\curveto(158.60577756,191.36096021)(159.23511089,189.81429354)(159.23511089,187.72362688)
\lineto(159.23511089,176.07562688)
\lineto(155.90711089,176.07562688)
\lineto(154.97911089,178.44362688)
\lineto(154.85111089,178.44362688)
\curveto(154.10444423,177.50496021)(153.31511089,176.82229354)(152.48311089,176.39562688)
\curveto(151.65111089,175.96896021)(150.50977756,175.75562688)(149.05911089,175.75562688)
\curveto(147.50177756,175.75562688)(146.21111089,176.20362688)(145.18711089,177.09962688)
\curveto(144.16311089,177.99562688)(143.65111089,179.39296021)(143.65111089,181.29162688)
\curveto(143.65111089,183.14762688)(144.30177756,184.51296021)(145.60311089,185.38762688)
\curveto(146.90444423,186.26229354)(148.85644423,186.75296021)(151.45911089,186.85962688)
\lineto(154.49911089,186.95562688)
\lineto(154.49911089,187.72362688)
\curveto(154.49911089,188.64096021)(154.25377756,189.31296021)(153.76311089,189.73962688)
\curveto(153.29377756,190.16629354)(152.63244423,190.37962688)(151.77911089,190.37962688)
\curveto(150.92577756,190.37962688)(150.09377756,190.25162688)(149.28311089,189.99562688)
\curveto(148.47244423,189.76096021)(147.66177756,189.46229354)(146.85111089,189.09962688)
\lineto(145.28311089,192.33162688)
\curveto(146.20044423,192.80096021)(147.23511089,193.17429354)(148.38711089,193.45162688)
\curveto(149.53911089,193.75029354)(150.73377756,193.89962688)(151.97111089,193.89962688)
\closepath
\moveto(154.49911089,184.17162688)
\lineto(152.64311089,184.10762688)
\curveto(151.10711089,184.06496021)(150.04044423,183.78762688)(149.44311089,183.27562688)
\curveto(148.84577756,182.76362688)(148.54711089,182.09162688)(148.54711089,181.25962688)
\curveto(148.54711089,180.53429354)(148.76044423,180.01162688)(149.18711089,179.69162688)
\curveto(149.61377756,179.39296021)(150.16844423,179.24362688)(150.85111089,179.24362688)
\curveto(151.87511089,179.24362688)(152.73911089,179.54229354)(153.44311089,180.13962688)
\curveto(154.14711089,180.75829354)(154.49911089,181.62229354)(154.49911089,182.73162688)
\closepath
}
}
{
\newrgbcolor{curcolor}{0 0 0}
\pscustom[linestyle=none,fillstyle=solid,fillcolor=curcolor]
{
\newpath
\moveto(161.63511382,193.54762688)
\lineto(166.85111382,193.54762688)
\lineto(170.14711382,183.72362688)
\curveto(170.31778049,183.23296021)(170.44578049,182.74229354)(170.53111382,182.25162688)
\curveto(170.61644716,181.76096021)(170.68044716,181.23829354)(170.72311382,180.68362688)
\lineto(170.81911382,180.68362688)
\curveto(170.88311382,181.23829354)(170.96844716,181.76096021)(171.07511382,182.25162688)
\curveto(171.18178049,182.74229354)(171.32044716,183.23296021)(171.49111382,183.72362688)
\lineto(174.72311382,193.54762688)
\lineto(179.84311382,193.54762688)
\lineto(172.45111382,173.83562688)
\curveto(171.76844716,172.02229354)(170.79778049,170.66762688)(169.53911382,169.77162688)
\curveto(168.28044716,168.85429354)(166.81911382,168.39562688)(165.15511382,168.39562688)
\curveto(164.60044716,168.39562688)(164.13111382,168.42762688)(163.74711382,168.49162688)
\curveto(163.36311382,168.53429354)(163.02178049,168.58762688)(162.72311382,168.65162688)
\lineto(162.72311382,172.42762688)
\curveto(162.93644716,172.38496021)(163.21378049,172.34229354)(163.55511382,172.29962688)
\curveto(163.89644716,172.25696021)(164.24844716,172.23562688)(164.61111382,172.23562688)
\curveto(165.61378049,172.23562688)(166.40311382,172.54496021)(166.97911382,173.16362688)
\curveto(167.55511382,173.76096021)(167.99244716,174.48629354)(168.29111382,175.33962688)
\lineto(168.57911382,176.20362688)
\closepath
}
}
{
\newrgbcolor{curcolor}{0 0 0}
\pscustom[linestyle=none,fillstyle=solid,fillcolor=curcolor]
{
\newpath
\moveto(188.70712164,193.86762688)
\curveto(189.9657883,193.86762688)(191.13912164,193.69696021)(192.22712164,193.35562688)
\curveto(193.33645497,193.03562688)(194.2217883,192.53429354)(194.88312164,191.85162688)
\curveto(195.5657883,191.16896021)(195.90712164,190.29429354)(195.90712164,189.22762688)
\curveto(195.90712164,188.18229354)(195.58712164,187.35029354)(194.94712164,186.73162688)
\curveto(194.30712164,186.11296021)(193.46445497,185.66496021)(192.41912164,185.38762688)
\lineto(192.41912164,185.22762688)
\curveto(193.1657883,185.05696021)(193.8377883,184.81162688)(194.43512164,184.49162688)
\curveto(195.03245497,184.19296021)(195.5017883,183.77696021)(195.84312164,183.24362688)
\curveto(196.2057883,182.73162688)(196.38712164,182.03829354)(196.38712164,181.16362688)
\curveto(196.38712164,180.20362688)(196.0777883,179.30762688)(195.45912164,178.47562688)
\curveto(194.8617883,177.66496021)(193.92312164,177.00362688)(192.64312164,176.49162688)
\curveto(191.36312164,176.00096021)(189.7417883,175.75562688)(187.77912164,175.75562688)
\curveto(184.8777883,175.75562688)(182.6377883,176.11829354)(181.05912164,176.84362688)
\lineto(181.05912164,180.77962688)
\curveto(181.78445497,180.43829354)(182.6697883,180.12896021)(183.71512164,179.85162688)
\curveto(184.7817883,179.57429354)(185.91245497,179.43562688)(187.10712164,179.43562688)
\curveto(188.40845497,179.43562688)(189.50712164,179.58496021)(190.40312164,179.88362688)
\curveto(191.29912164,180.18229354)(191.74712164,180.70496021)(191.74712164,181.45162688)
\curveto(191.74712164,182.83829354)(190.0937883,183.53162688)(186.78712164,183.53162688)
\lineto(184.93112164,183.53162688)
\lineto(184.93112164,186.82762688)
\lineto(186.69112164,186.82762688)
\curveto(188.2697883,186.82762688)(189.4857883,186.95562688)(190.33912164,187.21162688)
\curveto(191.2137883,187.46762688)(191.65112164,187.94762688)(191.65112164,188.65162688)
\curveto(191.65112164,189.20629354)(191.3737883,189.62229354)(190.81912164,189.89962688)
\curveto(190.26445497,190.19829354)(189.3577883,190.34762688)(188.09912164,190.34762688)
\curveto(187.26712164,190.34762688)(186.3497883,190.25162688)(185.34712164,190.05962688)
\curveto(184.3657883,189.86762688)(183.44845497,189.59029354)(182.59512164,189.22762688)
\lineto(181.18712164,192.55562688)
\curveto(182.1897883,192.93962688)(183.2777883,193.24896021)(184.45112164,193.48362688)
\curveto(185.6457883,193.73962688)(187.06445497,193.86762688)(188.70712164,193.86762688)
\closepath
}
}
{
\newrgbcolor{curcolor}{0 0 0}
\pscustom[linestyle=none,fillstyle=solid,fillcolor=curcolor]
{
\newpath
\moveto(207.2991314,193.86762688)
\curveto(209.70979807,193.86762688)(211.6191314,193.17429354)(213.0271314,191.78762688)
\curveto(214.4351314,190.42229354)(215.1391314,188.47029354)(215.1391314,185.93162688)
\lineto(215.1391314,183.62762688)
\lineto(203.8751314,183.62762688)
\curveto(203.91779807,182.28362688)(204.31246473,181.22762688)(205.0591314,180.45962688)
\curveto(205.8271314,179.69162688)(206.8831314,179.30762688)(208.2271314,179.30762688)
\curveto(209.33646473,179.30762688)(210.34979807,179.41429354)(211.2671314,179.62762688)
\curveto(212.20579807,179.86229354)(213.16579807,180.21429354)(214.1471314,180.68362688)
\lineto(214.1471314,177.00362688)
\curveto(213.27246473,176.57696021)(212.36579807,176.26762688)(211.4271314,176.07562688)
\curveto(210.48846473,175.86229354)(209.3471314,175.75562688)(208.0031314,175.75562688)
\curveto(206.25379807,175.75562688)(204.7071314,176.07562688)(203.3631314,176.71562688)
\curveto(202.0191314,177.37696021)(200.9631314,178.35829354)(200.1951314,179.65962688)
\curveto(199.4271314,180.98229354)(199.0431314,182.65696021)(199.0431314,184.68362688)
\curveto(199.0431314,186.71029354)(199.38446473,188.40629354)(200.0671314,189.77162688)
\curveto(200.7711314,191.13696021)(201.74179807,192.16096021)(202.9791314,192.84362688)
\curveto(204.21646473,193.52629354)(205.65646473,193.86762688)(207.2991314,193.86762688)
\closepath
\moveto(207.3311314,190.47562688)
\curveto(206.39246473,190.47562688)(205.62446473,190.17696021)(205.0271314,189.57962688)
\curveto(204.42979807,188.98229354)(204.07779807,188.05429354)(203.9711314,186.79562688)
\lineto(210.6591314,186.79562688)
\curveto(210.63779807,187.84096021)(210.34979807,188.71562688)(209.7951314,189.41962688)
\curveto(209.26179807,190.12362688)(208.44046473,190.47562688)(207.3311314,190.47562688)
\closepath
}
}
{
\newrgbcolor{curcolor}{0 0 0}
\pscustom[linestyle=none,fillstyle=solid,fillcolor=curcolor]
{
\newpath
\moveto(228.73911871,193.86762688)
\curveto(230.70178537,193.86762688)(232.29111871,193.09962688)(233.50711871,191.56362688)
\curveto(234.72311871,190.04896021)(235.33111871,187.80896021)(235.33111871,184.84362688)
\curveto(235.33111871,181.85696021)(234.70178537,179.59562688)(233.44311871,178.05962688)
\curveto(232.18445204,176.52362688)(230.57378537,175.75562688)(228.61111871,175.75562688)
\curveto(227.35245204,175.75562688)(226.34978537,175.97962688)(225.60311871,176.42762688)
\curveto(224.85645204,176.89696021)(224.24845204,177.41962688)(223.77911871,177.99562688)
\lineto(223.52311871,177.99562688)
\curveto(223.69378537,177.09962688)(223.77911871,176.24629354)(223.77911871,175.43562688)
\lineto(223.77911871,168.39562688)
\lineto(219.01111871,168.39562688)
\lineto(219.01111871,193.54762688)
\lineto(222.88311871,193.54762688)
\lineto(223.55511871,191.27562688)
\lineto(223.77911871,191.27562688)
\curveto(224.24845204,191.97962688)(224.87778537,192.58762688)(225.66711871,193.09962688)
\curveto(226.45645204,193.61162688)(227.48045204,193.86762688)(228.73911871,193.86762688)
\closepath
\moveto(227.20311871,190.05962688)
\curveto(225.96578537,190.05962688)(225.09111871,189.66496021)(224.57911871,188.87562688)
\curveto(224.06711871,188.10762688)(223.80045204,186.94496021)(223.77911871,185.38762688)
\lineto(223.77911871,184.87562688)
\curveto(223.77911871,183.19029354)(224.02445204,181.88896021)(224.51511871,180.97162688)
\curveto(225.02711871,180.07562688)(225.94445204,179.62762688)(227.26711871,179.62762688)
\curveto(228.35511871,179.62762688)(229.15511871,180.07562688)(229.66711871,180.97162688)
\curveto(230.20045204,181.88896021)(230.46711871,183.20096021)(230.46711871,184.90762688)
\curveto(230.46711871,188.34229354)(229.37911871,190.05962688)(227.20311871,190.05962688)
\closepath
}
}
{
\newrgbcolor{curcolor}{0 0 0}
\pscustom[linestyle=none,fillstyle=solid,fillcolor=curcolor]
{
\newpath
\moveto(246.33910015,184.84362688)
\curveto(246.33910015,182.26229354)(245.95510015,179.77696021)(245.18710015,177.38762688)
\curveto(244.44043348,175.01962688)(243.25643348,172.89696021)(241.63510015,171.01962688)
\lineto(237.76310015,171.01962688)
\curveto(239.19243348,172.98229354)(240.28043348,175.15829354)(241.02710015,177.54762688)
\curveto(241.79510015,179.95829354)(242.17910015,182.40096021)(242.17910015,184.87562688)
\curveto(242.17910015,187.41429354)(241.79510015,189.88896021)(241.02710015,192.29962688)
\curveto(240.28043348,194.71029354)(239.18176682,196.91829354)(237.73110015,198.92362688)
\lineto(241.63510015,198.92362688)
\curveto(243.25643348,196.98229354)(244.44043348,194.79562688)(245.18710015,192.36362688)
\curveto(245.95510015,189.95296021)(246.33910015,187.44629354)(246.33910015,184.84362688)
\closepath
}
}
{
\newrgbcolor{curcolor}{0 0 0}
\pscustom[linestyle=none,fillstyle=solid,fillcolor=curcolor]
{
\newpath
\moveto(483.55828179,233.18623725)
\curveto(481.70228179,233.18623725)(480.28361513,232.49290392)(479.30228179,231.10623725)
\curveto(478.32094846,229.71957058)(477.83028179,227.82090392)(477.83028179,225.41023725)
\curveto(477.83028179,222.97823725)(478.27828179,221.09023725)(479.17428179,219.74623725)
\curveto(480.09161513,218.42357058)(481.55294846,217.76223725)(483.55828179,217.76223725)
\curveto(484.47561513,217.76223725)(485.40361513,217.86890392)(486.34228179,218.08223725)
\curveto(487.28094846,218.29557058)(488.29428179,218.59423725)(489.38228179,218.97823725)
\lineto(489.38228179,214.91423725)
\curveto(488.37961513,214.50890392)(487.38761513,214.21023725)(486.40628179,214.01823725)
\curveto(485.42494846,213.82623725)(484.32628179,213.73023725)(483.11028179,213.73023725)
\curveto(480.74228179,213.73023725)(478.80094846,214.21023725)(477.28628179,215.17023725)
\curveto(475.77161513,216.15157058)(474.65161513,217.51690392)(473.92628179,219.26623725)
\curveto(473.20094846,221.03690392)(472.83828179,223.09557058)(472.83828179,225.44223725)
\curveto(472.83828179,227.74623725)(473.25428179,229.78357058)(474.08628179,231.55423725)
\curveto(474.91828179,233.32490392)(476.12361513,234.71157058)(477.70228179,235.71423725)
\curveto(479.30228179,236.71690392)(481.25428179,237.21823725)(483.55828179,237.21823725)
\curveto(484.68894846,237.21823725)(485.81961513,237.06890392)(486.95028179,236.77023725)
\curveto(488.10228179,236.49290392)(489.20094846,236.10890392)(490.24628179,235.61823725)
\lineto(488.67828179,231.68223725)
\curveto(487.82494846,232.08757058)(486.96094846,232.43957058)(486.08628179,232.73823725)
\curveto(485.23294846,233.03690392)(484.39028179,233.18623725)(483.55828179,233.18623725)
\closepath
}
}
{
\newrgbcolor{curcolor}{0 0 0}
\pscustom[linestyle=none,fillstyle=solid,fillcolor=curcolor]
{
\newpath
\moveto(501.06225396,231.84223725)
\curveto(503.47292063,231.84223725)(505.38225396,231.14890392)(506.79025396,229.76223725)
\curveto(508.19825396,228.39690392)(508.90225396,226.44490392)(508.90225396,223.90623725)
\lineto(508.90225396,221.60223725)
\lineto(497.63825396,221.60223725)
\curveto(497.68092063,220.25823725)(498.07558729,219.20223725)(498.82225396,218.43423725)
\curveto(499.59025396,217.66623725)(500.64625396,217.28223725)(501.99025396,217.28223725)
\curveto(503.09958729,217.28223725)(504.11292063,217.38890392)(505.03025396,217.60223725)
\curveto(505.96892063,217.83690392)(506.92892063,218.18890392)(507.91025396,218.65823725)
\lineto(507.91025396,214.97823725)
\curveto(507.03558729,214.55157058)(506.12892063,214.24223725)(505.19025396,214.05023725)
\curveto(504.25158729,213.83690392)(503.11025396,213.73023725)(501.76625396,213.73023725)
\curveto(500.01692063,213.73023725)(498.47025396,214.05023725)(497.12625396,214.69023725)
\curveto(495.78225396,215.35157058)(494.72625396,216.33290392)(493.95825396,217.63423725)
\curveto(493.19025396,218.95690392)(492.80625396,220.63157058)(492.80625396,222.65823725)
\curveto(492.80625396,224.68490392)(493.14758729,226.38090392)(493.83025396,227.74623725)
\curveto(494.53425396,229.11157058)(495.50492063,230.13557058)(496.74225396,230.81823725)
\curveto(497.97958729,231.50090392)(499.41958729,231.84223725)(501.06225396,231.84223725)
\closepath
\moveto(501.09425396,228.45023725)
\curveto(500.15558729,228.45023725)(499.38758729,228.15157058)(498.79025396,227.55423725)
\curveto(498.19292063,226.95690392)(497.84092063,226.02890392)(497.73425396,224.77023725)
\lineto(504.42225396,224.77023725)
\curveto(504.40092063,225.81557058)(504.11292063,226.69023725)(503.55825396,227.39423725)
\curveto(503.02492063,228.09823725)(502.20358729,228.45023725)(501.09425396,228.45023725)
\closepath
}
}
{
\newrgbcolor{curcolor}{0 0 0}
\pscustom[linestyle=none,fillstyle=solid,fillcolor=curcolor]
{
\newpath
\moveto(522.50224126,231.84223725)
\curveto(524.46490793,231.84223725)(526.05424126,231.07423725)(527.27024126,229.53823725)
\curveto(528.48624126,228.02357058)(529.09424126,225.78357058)(529.09424126,222.81823725)
\curveto(529.09424126,219.83157058)(528.46490793,217.57023725)(527.20624126,216.03423725)
\curveto(525.9475746,214.49823725)(524.33690793,213.73023725)(522.37424126,213.73023725)
\curveto(521.1155746,213.73023725)(520.11290793,213.95423725)(519.36624126,214.40223725)
\curveto(518.6195746,214.87157058)(518.0115746,215.39423725)(517.54224126,215.97023725)
\lineto(517.28624126,215.97023725)
\curveto(517.45690793,215.07423725)(517.54224126,214.22090392)(517.54224126,213.41023725)
\lineto(517.54224126,206.37023725)
\lineto(512.77424126,206.37023725)
\lineto(512.77424126,231.52223725)
\lineto(516.64624126,231.52223725)
\lineto(517.31824126,229.25023725)
\lineto(517.54224126,229.25023725)
\curveto(518.0115746,229.95423725)(518.64090793,230.56223725)(519.43024126,231.07423725)
\curveto(520.2195746,231.58623725)(521.2435746,231.84223725)(522.50224126,231.84223725)
\closepath
\moveto(520.96624126,228.03423725)
\curveto(519.72890793,228.03423725)(518.85424126,227.63957058)(518.34224126,226.85023725)
\curveto(517.83024126,226.08223725)(517.5635746,224.91957058)(517.54224126,223.36223725)
\lineto(517.54224126,222.85023725)
\curveto(517.54224126,221.16490392)(517.7875746,219.86357058)(518.27824126,218.94623725)
\curveto(518.79024126,218.05023725)(519.7075746,217.60223725)(521.03024126,217.60223725)
\curveto(522.11824126,217.60223725)(522.91824126,218.05023725)(523.43024126,218.94623725)
\curveto(523.9635746,219.86357058)(524.23024126,221.17557058)(524.23024126,222.88223725)
\curveto(524.23024126,226.31690392)(523.14224126,228.03423725)(520.96624126,228.03423725)
\closepath
}
}
{
\newrgbcolor{curcolor}{0 0 0}
\pscustom[linestyle=none,fillstyle=solid,fillcolor=curcolor]
{
\newpath
\moveto(548.51822271,226.94623725)
\curveto(548.51822271,226.00757058)(548.21955604,225.20757058)(547.62222271,224.54623725)
\curveto(547.04622271,223.88490392)(546.18222271,223.45823725)(545.03022271,223.26623725)
\lineto(545.03022271,223.13823725)
\curveto(546.24622271,222.98890392)(547.21688938,222.56223725)(547.94222271,221.85823725)
\curveto(548.68888938,221.17557058)(549.06222271,220.31157058)(549.06222271,219.26623725)
\curveto(549.06222271,218.26357058)(548.79555604,217.36757058)(548.26222271,216.57823725)
\curveto(547.75022271,215.78890392)(546.92888938,215.17023725)(545.79822271,214.72223725)
\curveto(544.66755604,214.27423725)(543.18488938,214.05023725)(541.35022271,214.05023725)
\lineto(533.03022271,214.05023725)
\lineto(533.03022271,231.52223725)
\lineto(541.35022271,231.52223725)
\curveto(542.71555604,231.52223725)(543.93155604,231.37290392)(544.99822271,231.07423725)
\curveto(546.08622271,230.79690392)(546.93955604,230.31690392)(547.55822271,229.63423725)
\curveto(548.19822271,228.97290392)(548.51822271,228.07690392)(548.51822271,226.94623725)
\closepath
\moveto(543.68622271,226.56223725)
\curveto(543.68622271,227.62890392)(542.84355604,228.16223725)(541.15822271,228.16223725)
\lineto(537.79822271,228.16223725)
\lineto(537.79822271,224.70623725)
\lineto(540.61422271,224.70623725)
\curveto(541.61688938,224.70623725)(542.37422271,224.84490392)(542.88622271,225.12223725)
\curveto(543.41955604,225.42090392)(543.68622271,225.90090392)(543.68622271,226.56223725)
\closepath
\moveto(544.13422271,219.52223725)
\curveto(544.13422271,220.20490392)(543.85688938,220.69557058)(543.30222271,220.99423725)
\curveto(542.76888938,221.31423725)(541.97955604,221.47423725)(540.93422271,221.47423725)
\lineto(537.79822271,221.47423725)
\lineto(537.79822271,217.34623725)
\lineto(541.03022271,217.34623725)
\curveto(541.92622271,217.34623725)(542.66222271,217.50623725)(543.23822271,217.82623725)
\curveto(543.83555604,218.16757058)(544.13422271,218.73290392)(544.13422271,219.52223725)
\closepath
}
}
{
\newrgbcolor{curcolor}{0 0 0}
\pscustom[linestyle=none,fillstyle=solid,fillcolor=curcolor]
{
\newpath
\moveto(560.35821343,231.84223725)
\curveto(562.7688801,231.84223725)(564.67821343,231.14890392)(566.08621343,229.76223725)
\curveto(567.49421343,228.39690392)(568.19821343,226.44490392)(568.19821343,223.90623725)
\lineto(568.19821343,221.60223725)
\lineto(556.93421343,221.60223725)
\curveto(556.9768801,220.25823725)(557.37154677,219.20223725)(558.11821343,218.43423725)
\curveto(558.88621343,217.66623725)(559.94221343,217.28223725)(561.28621343,217.28223725)
\curveto(562.39554677,217.28223725)(563.4088801,217.38890392)(564.32621343,217.60223725)
\curveto(565.2648801,217.83690392)(566.2248801,218.18890392)(567.20621343,218.65823725)
\lineto(567.20621343,214.97823725)
\curveto(566.33154677,214.55157058)(565.4248801,214.24223725)(564.48621343,214.05023725)
\curveto(563.54754677,213.83690392)(562.40621343,213.73023725)(561.06221343,213.73023725)
\curveto(559.3128801,213.73023725)(557.76621343,214.05023725)(556.42221343,214.69023725)
\curveto(555.07821343,215.35157058)(554.02221343,216.33290392)(553.25421343,217.63423725)
\curveto(552.48621343,218.95690392)(552.10221343,220.63157058)(552.10221343,222.65823725)
\curveto(552.10221343,224.68490392)(552.44354677,226.38090392)(553.12621343,227.74623725)
\curveto(553.83021343,229.11157058)(554.8008801,230.13557058)(556.03821343,230.81823725)
\curveto(557.27554677,231.50090392)(558.71554677,231.84223725)(560.35821343,231.84223725)
\closepath
\moveto(560.39021343,228.45023725)
\curveto(559.45154677,228.45023725)(558.68354677,228.15157058)(558.08621343,227.55423725)
\curveto(557.4888801,226.95690392)(557.1368801,226.02890392)(557.03021343,224.77023725)
\lineto(563.71821343,224.77023725)
\curveto(563.6968801,225.81557058)(563.4088801,226.69023725)(562.85421343,227.39423725)
\curveto(562.3208801,228.09823725)(561.49954677,228.45023725)(560.39021343,228.45023725)
\closepath
}
}
{
\newrgbcolor{curcolor}{0 0 0}
\pscustom[linestyle=none,fillstyle=solid,fillcolor=curcolor]
{
\newpath
\moveto(581.79820074,231.84223725)
\curveto(583.7608674,231.84223725)(585.35020074,231.07423725)(586.56620074,229.53823725)
\curveto(587.78220074,228.02357058)(588.39020074,225.78357058)(588.39020074,222.81823725)
\curveto(588.39020074,219.83157058)(587.7608674,217.57023725)(586.50220074,216.03423725)
\curveto(585.24353407,214.49823725)(583.6328674,213.73023725)(581.67020074,213.73023725)
\curveto(580.41153407,213.73023725)(579.4088674,213.95423725)(578.66220074,214.40223725)
\curveto(577.91553407,214.87157058)(577.30753407,215.39423725)(576.83820074,215.97023725)
\lineto(576.58220074,215.97023725)
\curveto(576.7528674,215.07423725)(576.83820074,214.22090392)(576.83820074,213.41023725)
\lineto(576.83820074,206.37023725)
\lineto(572.07020074,206.37023725)
\lineto(572.07020074,231.52223725)
\lineto(575.94220074,231.52223725)
\lineto(576.61420074,229.25023725)
\lineto(576.83820074,229.25023725)
\curveto(577.30753407,229.95423725)(577.9368674,230.56223725)(578.72620074,231.07423725)
\curveto(579.51553407,231.58623725)(580.53953407,231.84223725)(581.79820074,231.84223725)
\closepath
\moveto(580.26220074,228.03423725)
\curveto(579.0248674,228.03423725)(578.15020074,227.63957058)(577.63820074,226.85023725)
\curveto(577.12620074,226.08223725)(576.85953407,224.91957058)(576.83820074,223.36223725)
\lineto(576.83820074,222.85023725)
\curveto(576.83820074,221.16490392)(577.08353407,219.86357058)(577.57420074,218.94623725)
\curveto(578.08620074,218.05023725)(579.00353407,217.60223725)(580.32620074,217.60223725)
\curveto(581.41420074,217.60223725)(582.21420074,218.05023725)(582.72620074,218.94623725)
\curveto(583.25953407,219.86357058)(583.52620074,221.17557058)(583.52620074,222.88223725)
\curveto(583.52620074,226.31690392)(582.43820074,228.03423725)(580.26220074,228.03423725)
\closepath
}
}
{
\newrgbcolor{curcolor}{0 0 0}
\pscustom[linestyle=none,fillstyle=solid,fillcolor=curcolor]
{
\newpath
\moveto(460.95023956,182.81823725)
\curveto(460.95023956,185.42090392)(461.32357289,187.92757058)(462.07023956,190.33823725)
\curveto(462.83823956,192.77023725)(464.03290622,194.95690392)(465.65423956,196.89823725)
\lineto(469.55823956,196.89823725)
\curveto(468.10757289,194.89290392)(466.99823956,192.68490392)(466.23023956,190.27423725)
\curveto(465.48357289,187.86357058)(465.11023956,185.38890392)(465.11023956,182.85023725)
\curveto(465.11023956,180.37557058)(465.48357289,177.93290392)(466.23023956,175.52223725)
\curveto(466.97690622,173.13290392)(468.07557289,170.95690392)(469.52623956,168.99423725)
\lineto(465.65423956,168.99423725)
\curveto(464.03290622,170.87157058)(462.83823956,172.99423725)(462.07023956,175.36223725)
\curveto(461.32357289,177.75157058)(460.95023956,180.23690392)(460.95023956,182.81823725)
\closepath
}
}
{
\newrgbcolor{curcolor}{0 0 0}
\pscustom[linestyle=none,fillstyle=solid,fillcolor=curcolor]
{
\newpath
\moveto(493.65426202,174.05023725)
\lineto(487.51026202,174.05023725)
\lineto(477.55826202,191.33023725)
\lineto(477.43026202,191.33023725)
\lineto(477.55826202,188.06623725)
\curveto(477.62226202,186.97823725)(477.67559535,185.89023725)(477.71826202,184.80223725)
\lineto(477.71826202,174.05023725)
\lineto(473.39826202,174.05023725)
\lineto(473.39826202,196.89823725)
\lineto(479.51026202,196.89823725)
\lineto(489.43026202,179.77823725)
\lineto(489.52626202,179.77823725)
\lineto(489.39826202,182.91423725)
\curveto(489.35559535,183.95957058)(489.32359535,185.01557058)(489.30226202,186.08223725)
\lineto(489.30226202,196.89823725)
\lineto(493.65426202,196.89823725)
\closepath
}
}
{
\newrgbcolor{curcolor}{0 0 0}
\pscustom[linestyle=none,fillstyle=solid,fillcolor=curcolor]
{
\newpath
\moveto(514.90225323,182.81823725)
\curveto(514.90225323,179.91690392)(514.13425323,177.67690392)(512.59825323,176.09823725)
\curveto(511.08358656,174.51957058)(509.01425323,173.73023725)(506.39025323,173.73023725)
\curveto(504.76891989,173.73023725)(503.31825323,174.08223725)(502.03825323,174.78623725)
\curveto(500.77958656,175.49023725)(499.78758656,176.51423725)(499.06225323,177.85823725)
\curveto(498.33691989,179.22357058)(497.97425323,180.87690392)(497.97425323,182.81823725)
\curveto(497.97425323,185.71957058)(498.73158656,187.94890392)(500.24625323,189.50623725)
\curveto(501.76091989,191.06357058)(503.84091989,191.84223725)(506.48625323,191.84223725)
\curveto(508.12891989,191.84223725)(509.57958656,191.49023725)(510.83825323,190.78623725)
\curveto(512.09691989,190.08223725)(513.08891989,189.05823725)(513.81425323,187.71423725)
\curveto(514.53958656,186.37023725)(514.90225323,184.73823725)(514.90225323,182.81823725)
\closepath
\moveto(502.83825323,182.81823725)
\curveto(502.83825323,181.09023725)(503.11558656,179.77823725)(503.67025323,178.88223725)
\curveto(504.24625323,178.00757058)(505.17425323,177.57023725)(506.45425323,177.57023725)
\curveto(507.71291989,177.57023725)(508.61958656,178.00757058)(509.17425323,178.88223725)
\curveto(509.75025323,179.77823725)(510.03825323,181.09023725)(510.03825323,182.81823725)
\curveto(510.03825323,184.54623725)(509.75025323,185.83690392)(509.17425323,186.69023725)
\curveto(508.61958656,187.56490392)(507.70225323,188.00223725)(506.42225323,188.00223725)
\curveto(505.16358656,188.00223725)(504.24625323,187.56490392)(503.67025323,186.69023725)
\curveto(503.11558656,185.83690392)(502.83825323,184.54623725)(502.83825323,182.81823725)
\closepath
}
}
{
\newrgbcolor{curcolor}{0 0 0}
\pscustom[linestyle=none,fillstyle=solid,fillcolor=curcolor]
{
\newpath
\moveto(524.37423663,173.73023725)
\curveto(522.43290329,173.73023725)(520.84356996,174.48757058)(519.60623663,176.00223725)
\curveto(518.39023663,177.53823725)(517.78223663,179.78890392)(517.78223663,182.75423725)
\curveto(517.78223663,185.74090392)(518.40090329,188.00223725)(519.63823663,189.53823725)
\curveto(520.87556996,191.07423725)(522.49690329,191.84223725)(524.50223663,191.84223725)
\curveto(525.76090329,191.84223725)(526.79556996,191.59690392)(527.60623663,191.10623725)
\curveto(528.41690329,190.61557058)(529.05690329,190.00757058)(529.52623663,189.28223725)
\lineto(529.68623663,189.28223725)
\curveto(529.62223663,189.62357058)(529.54756996,190.11423725)(529.46223663,190.75423725)
\curveto(529.37690329,191.41557058)(529.33423663,192.08757058)(529.33423663,192.77023725)
\lineto(529.33423663,198.37023725)
\lineto(534.10223663,198.37023725)
\lineto(534.10223663,174.05023725)
\lineto(530.45423663,174.05023725)
\lineto(529.52623663,176.32223725)
\lineto(529.33423663,176.32223725)
\curveto(528.86490329,175.59690392)(528.23556996,174.97823725)(527.44623663,174.46623725)
\curveto(526.65690329,173.97557058)(525.63290329,173.73023725)(524.37423663,173.73023725)
\closepath
\moveto(526.03823663,177.53823725)
\curveto(527.33956996,177.53823725)(528.25690329,177.92223725)(528.79023663,178.69023725)
\curveto(529.32356996,179.47957058)(529.61156996,180.65290392)(529.65423663,182.21023725)
\lineto(529.65423663,182.72223725)
\curveto(529.65423663,184.40757058)(529.38756996,185.69823725)(528.85423663,186.59423725)
\curveto(528.34223663,187.51157058)(527.38223663,187.97023725)(525.97423663,187.97023725)
\curveto(524.92890329,187.97023725)(524.10756996,187.51157058)(523.51023663,186.59423725)
\curveto(522.91290329,185.69823725)(522.61423663,184.39690392)(522.61423663,182.69023725)
\curveto(522.61423663,180.98357058)(522.91290329,179.69290392)(523.51023663,178.81823725)
\curveto(524.10756996,177.96490392)(524.95023663,177.53823725)(526.03823663,177.53823725)
\closepath
}
}
{
\newrgbcolor{curcolor}{0 0 0}
\pscustom[linestyle=none,fillstyle=solid,fillcolor=curcolor]
{
\newpath
\moveto(546.29421807,191.84223725)
\curveto(548.70488474,191.84223725)(550.61421807,191.14890392)(552.02221807,189.76223725)
\curveto(553.43021807,188.39690392)(554.13421807,186.44490392)(554.13421807,183.90623725)
\lineto(554.13421807,181.60223725)
\lineto(542.87021807,181.60223725)
\curveto(542.91288474,180.25823725)(543.3075514,179.20223725)(544.05421807,178.43423725)
\curveto(544.82221807,177.66623725)(545.87821807,177.28223725)(547.22221807,177.28223725)
\curveto(548.3315514,177.28223725)(549.34488474,177.38890392)(550.26221807,177.60223725)
\curveto(551.20088474,177.83690392)(552.16088474,178.18890392)(553.14221807,178.65823725)
\lineto(553.14221807,174.97823725)
\curveto(552.2675514,174.55157058)(551.36088474,174.24223725)(550.42221807,174.05023725)
\curveto(549.4835514,173.83690392)(548.34221807,173.73023725)(546.99821807,173.73023725)
\curveto(545.24888474,173.73023725)(543.70221807,174.05023725)(542.35821807,174.69023725)
\curveto(541.01421807,175.35157058)(539.95821807,176.33290392)(539.19021807,177.63423725)
\curveto(538.42221807,178.95690392)(538.03821807,180.63157058)(538.03821807,182.65823725)
\curveto(538.03821807,184.68490392)(538.3795514,186.38090392)(539.06221807,187.74623725)
\curveto(539.76621807,189.11157058)(540.73688474,190.13557058)(541.97421807,190.81823725)
\curveto(543.2115514,191.50090392)(544.6515514,191.84223725)(546.29421807,191.84223725)
\closepath
\moveto(546.32621807,188.45023725)
\curveto(545.3875514,188.45023725)(544.6195514,188.15157058)(544.02221807,187.55423725)
\curveto(543.42488474,186.95690392)(543.07288474,186.02890392)(542.96621807,184.77023725)
\lineto(549.65421807,184.77023725)
\curveto(549.63288474,185.81557058)(549.34488474,186.69023725)(548.79021807,187.39423725)
\curveto(548.25688474,188.09823725)(547.4355514,188.45023725)(546.32621807,188.45023725)
\closepath
}
}
{
\newrgbcolor{curcolor}{0 0 0}
\pscustom[linestyle=none,fillstyle=solid,fillcolor=curcolor]
{
\newpath
\moveto(557.33420538,176.29023725)
\curveto(557.33420538,177.27157058)(557.60087204,177.95423725)(558.13420538,178.33823725)
\curveto(558.66753871,178.74357058)(559.31820538,178.94623725)(560.08620538,178.94623725)
\curveto(560.83287204,178.94623725)(561.47287204,178.74357058)(562.00620538,178.33823725)
\curveto(562.53953871,177.95423725)(562.80620538,177.27157058)(562.80620538,176.29023725)
\curveto(562.80620538,175.35157058)(562.53953871,174.66890392)(562.00620538,174.24223725)
\curveto(561.47287204,173.83690392)(560.83287204,173.63423725)(560.08620538,173.63423725)
\curveto(559.31820538,173.63423725)(558.66753871,173.83690392)(558.13420538,174.24223725)
\curveto(557.60087204,174.66890392)(557.33420538,175.35157058)(557.33420538,176.29023725)
\closepath
}
}
{
\newrgbcolor{curcolor}{0 0 0}
\pscustom[linestyle=none,fillstyle=solid,fillcolor=curcolor]
{
\newpath
\moveto(566.93420049,196.03423725)
\curveto(566.93420049,196.93023725)(567.17953383,197.53823725)(567.67020049,197.85823725)
\curveto(568.18220049,198.19957058)(568.80086716,198.37023725)(569.52620049,198.37023725)
\curveto(570.23020049,198.37023725)(570.83820049,198.19957058)(571.35020049,197.85823725)
\curveto(571.86220049,197.53823725)(572.11820049,196.93023725)(572.11820049,196.03423725)
\curveto(572.11820049,195.15957058)(571.86220049,194.55157058)(571.35020049,194.21023725)
\curveto(570.83820049,193.86890392)(570.23020049,193.69823725)(569.52620049,193.69823725)
\curveto(568.80086716,193.69823725)(568.18220049,193.86890392)(567.67020049,194.21023725)
\curveto(567.17953383,194.55157058)(566.93420049,195.15957058)(566.93420049,196.03423725)
\closepath
\moveto(565.71820049,166.37023725)
\curveto(565.16353383,166.37023725)(564.59820049,166.41290392)(564.02220049,166.49823725)
\curveto(563.44620049,166.56223725)(562.96620049,166.64757058)(562.58220049,166.75423725)
\lineto(562.58220049,170.49823725)
\curveto(562.96620049,170.39157058)(563.32886716,170.31690392)(563.67020049,170.27423725)
\curveto(564.01153383,170.23157058)(564.39553383,170.21023725)(564.82220049,170.21023725)
\curveto(565.46220049,170.21023725)(566.00620049,170.39157058)(566.45420049,170.75423725)
\curveto(566.90220049,171.11690392)(567.12620049,171.82090392)(567.12620049,172.86623725)
\lineto(567.12620049,191.52223725)
\lineto(571.89420049,191.52223725)
\lineto(571.89420049,172.16223725)
\curveto(571.89420049,171.09557058)(571.69153383,170.12490392)(571.28620049,169.25023725)
\curveto(570.88086716,168.37557058)(570.21953383,167.68223725)(569.30220049,167.17023725)
\curveto(568.40620049,166.63690392)(567.21153383,166.37023725)(565.71820049,166.37023725)
\closepath
}
}
{
\newrgbcolor{curcolor}{0 0 0}
\pscustom[linestyle=none,fillstyle=solid,fillcolor=curcolor]
{
\newpath
\moveto(589.07821026,179.23423725)
\curveto(589.07821026,177.46357058)(588.44887693,176.09823725)(587.19021026,175.13823725)
\curveto(585.95287693,174.19957058)(584.09687693,173.73023725)(581.62221026,173.73023725)
\curveto(580.40621026,173.73023725)(579.36087693,173.81557058)(578.48621026,173.98623725)
\curveto(577.61154359,174.13557058)(576.73687693,174.39157058)(575.86221026,174.75423725)
\lineto(575.86221026,178.69023725)
\curveto(576.80087693,178.26357058)(577.81421026,177.91157058)(578.90221026,177.63423725)
\curveto(579.99021026,177.35690392)(580.95021026,177.21823725)(581.78221026,177.21823725)
\curveto(582.69954359,177.21823725)(583.36087693,177.35690392)(583.76621026,177.63423725)
\curveto(584.17154359,177.91157058)(584.37421026,178.27423725)(584.37421026,178.72223725)
\curveto(584.37421026,179.02090392)(584.28887693,179.28757058)(584.11821026,179.52223725)
\curveto(583.96887693,179.75690392)(583.62754359,180.02357058)(583.09421026,180.32223725)
\curveto(582.56087693,180.62090392)(581.72887693,181.00490392)(580.59821026,181.47423725)
\curveto(579.48887693,181.94357058)(578.58221026,182.40223725)(577.87821026,182.85023725)
\curveto(577.19554359,183.31957058)(576.68354359,183.87423725)(576.34221026,184.51423725)
\curveto(576.00087693,185.17557058)(575.83021026,185.99690392)(575.83021026,186.97823725)
\curveto(575.83021026,188.59957058)(576.45954359,189.81557058)(577.71821026,190.62623725)
\curveto(578.97687693,191.43690392)(580.65154359,191.84223725)(582.74221026,191.84223725)
\curveto(583.83021026,191.84223725)(584.86487693,191.73557058)(585.84621026,191.52223725)
\curveto(586.82754359,191.30890392)(587.84087693,190.95690392)(588.88621026,190.46623725)
\lineto(587.44621026,187.04223725)
\curveto(586.59287693,187.40490392)(585.78221026,187.70357058)(585.01421026,187.93823725)
\curveto(584.24621026,188.19423725)(583.46754359,188.32223725)(582.67821026,188.32223725)
\curveto(581.27021026,188.32223725)(580.56621026,187.93823725)(580.56621026,187.17023725)
\curveto(580.56621026,186.89290392)(580.65154359,186.63690392)(580.82221026,186.40223725)
\curveto(581.01421026,186.18890392)(581.36621026,185.95423725)(581.87821026,185.69823725)
\curveto(582.41154359,185.44223725)(583.19021026,185.10090392)(584.21421026,184.67423725)
\curveto(585.21687693,184.26890392)(586.08087693,183.84223725)(586.80621026,183.39423725)
\curveto(587.53154359,182.96757058)(588.08621026,182.42357058)(588.47021026,181.76223725)
\curveto(588.87554359,181.10090392)(589.07821026,180.25823725)(589.07821026,179.23423725)
\closepath
}
}
{
\newrgbcolor{curcolor}{0 0 0}
\pscustom[linestyle=none,fillstyle=solid,fillcolor=curcolor]
{
\newpath
\moveto(599.86220196,182.81823725)
\curveto(599.86220196,180.23690392)(599.47820196,177.75157058)(598.71020196,175.36223725)
\curveto(597.96353529,172.99423725)(596.77953529,170.87157058)(595.15820196,168.99423725)
\lineto(591.28620196,168.99423725)
\curveto(592.71553529,170.95690392)(593.80353529,173.13290392)(594.55020196,175.52223725)
\curveto(595.31820196,177.93290392)(595.70220196,180.37557058)(595.70220196,182.85023725)
\curveto(595.70220196,185.38890392)(595.31820196,187.86357058)(594.55020196,190.27423725)
\curveto(593.80353529,192.68490392)(592.70486862,194.89290392)(591.25420196,196.89823725)
\lineto(595.15820196,196.89823725)
\curveto(596.77953529,194.95690392)(597.96353529,192.77023725)(598.71020196,190.33823725)
\curveto(599.47820196,187.92757058)(599.86220196,185.42090392)(599.86220196,182.81823725)
\closepath
}
}
{
\newrgbcolor{curcolor}{0 0 0}
\pscustom[linestyle=none,fillstyle=solid,fillcolor=curcolor]
{
\newpath
\moveto(611.80517239,574.747023)
\lineto(611.80517239,597.595023)
\lineto(626.23717239,597.595023)
\lineto(626.23717239,593.595023)
\lineto(616.63717239,593.595023)
\lineto(616.63717239,588.827023)
\lineto(618.55717239,588.827023)
\curveto(620.71183906,588.827023)(622.47183906,588.52835633)(623.83717239,587.931023)
\curveto(625.22383906,587.33368967)(626.24783906,586.51235633)(626.90917239,585.467023)
\curveto(627.57050572,584.42168967)(627.90117239,583.227023)(627.90117239,581.883023)
\curveto(627.90117239,579.62168967)(627.14383906,577.86168967)(625.62917239,576.603023)
\curveto(624.13583906,575.36568967)(621.74650572,574.747023)(618.46117239,574.747023)
\closepath
\moveto(616.63717239,578.715023)
\lineto(618.26917239,578.715023)
\curveto(619.74117239,578.715023)(620.89317239,578.94968967)(621.72517239,579.419023)
\curveto(622.57850572,579.88835633)(623.00517239,580.70968967)(623.00517239,581.883023)
\curveto(623.00517239,583.099023)(622.54650572,583.899023)(621.62917239,584.283023)
\curveto(620.71183906,584.667023)(619.46383906,584.859023)(617.88517239,584.859023)
\lineto(616.63717239,584.859023)
\closepath
}
}
{
\newrgbcolor{curcolor}{0 0 0}
\pscustom[linestyle=none,fillstyle=solid,fillcolor=curcolor]
{
\newpath
\moveto(639.00518801,592.571023)
\curveto(641.35185468,592.571023)(643.14385468,592.059023)(644.38118801,591.035023)
\curveto(645.63985468,590.03235633)(646.26918801,588.48568967)(646.26918801,586.395023)
\lineto(646.26918801,574.747023)
\lineto(642.94118801,574.747023)
\lineto(642.01318801,577.115023)
\lineto(641.88518801,577.115023)
\curveto(641.13852135,576.17635633)(640.34918801,575.49368967)(639.51718801,575.067023)
\curveto(638.68518801,574.64035633)(637.54385468,574.427023)(636.09318801,574.427023)
\curveto(634.53585468,574.427023)(633.24518801,574.875023)(632.22118801,575.771023)
\curveto(631.19718801,576.667023)(630.68518801,578.06435633)(630.68518801,579.963023)
\curveto(630.68518801,581.819023)(631.33585468,583.18435633)(632.63718801,584.059023)
\curveto(633.93852135,584.93368967)(635.89052135,585.42435633)(638.49318801,585.531023)
\lineto(641.53318801,585.627023)
\lineto(641.53318801,586.395023)
\curveto(641.53318801,587.31235633)(641.28785468,587.98435633)(640.79718801,588.411023)
\curveto(640.32785468,588.83768967)(639.66652135,589.051023)(638.81318801,589.051023)
\curveto(637.95985468,589.051023)(637.12785468,588.923023)(636.31718801,588.667023)
\curveto(635.50652135,588.43235633)(634.69585468,588.13368967)(633.88518801,587.771023)
\lineto(632.31718801,591.003023)
\curveto(633.23452135,591.47235633)(634.26918801,591.84568967)(635.42118801,592.123023)
\curveto(636.57318801,592.42168967)(637.76785468,592.571023)(639.00518801,592.571023)
\closepath
\moveto(641.53318801,582.843023)
\lineto(639.67718801,582.779023)
\curveto(638.14118801,582.73635633)(637.07452135,582.459023)(636.47718801,581.947023)
\curveto(635.87985468,581.435023)(635.58118801,580.763023)(635.58118801,579.931023)
\curveto(635.58118801,579.20568967)(635.79452135,578.683023)(636.22118801,578.363023)
\curveto(636.64785468,578.06435633)(637.20252135,577.915023)(637.88518801,577.915023)
\curveto(638.90918801,577.915023)(639.77318801,578.21368967)(640.47718801,578.811023)
\curveto(641.18118801,579.42968967)(641.53318801,580.29368967)(641.53318801,581.403023)
\closepath
}
}
{
\newrgbcolor{curcolor}{0 0 0}
\pscustom[linestyle=none,fillstyle=solid,fillcolor=curcolor]
{
\newpath
\moveto(657.53319094,592.539023)
\curveto(658.79185761,592.539023)(659.96519094,592.36835633)(661.05319094,592.027023)
\curveto(662.16252428,591.707023)(663.04785761,591.20568967)(663.70919094,590.523023)
\curveto(664.39185761,589.84035633)(664.73319094,588.96568967)(664.73319094,587.899023)
\curveto(664.73319094,586.85368967)(664.41319094,586.02168967)(663.77319094,585.403023)
\curveto(663.13319094,584.78435633)(662.29052428,584.33635633)(661.24519094,584.059023)
\lineto(661.24519094,583.899023)
\curveto(661.99185761,583.72835633)(662.66385761,583.483023)(663.26119094,583.163023)
\curveto(663.85852428,582.86435633)(664.32785761,582.44835633)(664.66919094,581.915023)
\curveto(665.03185761,581.403023)(665.21319094,580.70968967)(665.21319094,579.835023)
\curveto(665.21319094,578.875023)(664.90385761,577.979023)(664.28519094,577.147023)
\curveto(663.68785761,576.33635633)(662.74919094,575.675023)(661.46919094,575.163023)
\curveto(660.18919094,574.67235633)(658.56785761,574.427023)(656.60519094,574.427023)
\curveto(653.70385761,574.427023)(651.46385761,574.78968967)(649.88519094,575.515023)
\lineto(649.88519094,579.451023)
\curveto(650.61052428,579.10968967)(651.49585761,578.80035633)(652.54119094,578.523023)
\curveto(653.60785761,578.24568967)(654.73852428,578.107023)(655.93319094,578.107023)
\curveto(657.23452428,578.107023)(658.33319094,578.25635633)(659.22919094,578.555023)
\curveto(660.12519094,578.85368967)(660.57319094,579.37635633)(660.57319094,580.123023)
\curveto(660.57319094,581.50968967)(658.91985761,582.203023)(655.61319094,582.203023)
\lineto(653.75719094,582.203023)
\lineto(653.75719094,585.499023)
\lineto(655.51719094,585.499023)
\curveto(657.09585761,585.499023)(658.31185761,585.627023)(659.16519094,585.883023)
\curveto(660.03985761,586.139023)(660.47719094,586.619023)(660.47719094,587.323023)
\curveto(660.47719094,587.87768967)(660.19985761,588.29368967)(659.64519094,588.571023)
\curveto(659.09052428,588.86968967)(658.18385761,589.019023)(656.92519094,589.019023)
\curveto(656.09319094,589.019023)(655.17585761,588.923023)(654.17319094,588.731023)
\curveto(653.19185761,588.539023)(652.27452428,588.26168967)(651.42119094,587.899023)
\lineto(650.01319094,591.227023)
\curveto(651.01585761,591.611023)(652.10385761,591.92035633)(653.27719094,592.155023)
\curveto(654.47185761,592.411023)(655.89052428,592.539023)(657.53319094,592.539023)
\closepath
}
}
{
\newrgbcolor{curcolor}{0 0 0}
\pscustom[linestyle=none,fillstyle=solid,fillcolor=curcolor]
{
\newpath
\moveto(676.09320071,592.571023)
\curveto(678.43986738,592.571023)(680.23186738,592.059023)(681.46920071,591.035023)
\curveto(682.72786738,590.03235633)(683.35720071,588.48568967)(683.35720071,586.395023)
\lineto(683.35720071,574.747023)
\lineto(680.02920071,574.747023)
\lineto(679.10120071,577.115023)
\lineto(678.97320071,577.115023)
\curveto(678.22653404,576.17635633)(677.43720071,575.49368967)(676.60520071,575.067023)
\curveto(675.77320071,574.64035633)(674.63186738,574.427023)(673.18120071,574.427023)
\curveto(671.62386738,574.427023)(670.33320071,574.875023)(669.30920071,575.771023)
\curveto(668.28520071,576.667023)(667.77320071,578.06435633)(667.77320071,579.963023)
\curveto(667.77320071,581.819023)(668.42386738,583.18435633)(669.72520071,584.059023)
\curveto(671.02653404,584.93368967)(672.97853404,585.42435633)(675.58120071,585.531023)
\lineto(678.62120071,585.627023)
\lineto(678.62120071,586.395023)
\curveto(678.62120071,587.31235633)(678.37586738,587.98435633)(677.88520071,588.411023)
\curveto(677.41586738,588.83768967)(676.75453404,589.051023)(675.90120071,589.051023)
\curveto(675.04786738,589.051023)(674.21586738,588.923023)(673.40520071,588.667023)
\curveto(672.59453404,588.43235633)(671.78386738,588.13368967)(670.97320071,587.771023)
\lineto(669.40520071,591.003023)
\curveto(670.32253404,591.47235633)(671.35720071,591.84568967)(672.50920071,592.123023)
\curveto(673.66120071,592.42168967)(674.85586738,592.571023)(676.09320071,592.571023)
\closepath
\moveto(678.62120071,582.843023)
\lineto(676.76520071,582.779023)
\curveto(675.22920071,582.73635633)(674.16253404,582.459023)(673.56520071,581.947023)
\curveto(672.96786738,581.435023)(672.66920071,580.763023)(672.66920071,579.931023)
\curveto(672.66920071,579.20568967)(672.88253404,578.683023)(673.30920071,578.363023)
\curveto(673.73586738,578.06435633)(674.29053404,577.915023)(674.97320071,577.915023)
\curveto(675.99720071,577.915023)(676.86120071,578.21368967)(677.56520071,578.811023)
\curveto(678.26920071,579.42968967)(678.62120071,580.29368967)(678.62120071,581.403023)
\closepath
}
}
{
\newrgbcolor{curcolor}{0 0 0}
\pscustom[linestyle=none,fillstyle=solid,fillcolor=curcolor]
{
\newpath
\moveto(712.28521096,592.219023)
\lineto(712.28521096,578.235023)
\lineto(714.84521096,578.235023)
\lineto(714.84521096,568.475023)
\lineto(710.55721096,568.475023)
\lineto(710.55721096,574.747023)
\lineto(698.81321096,574.747023)
\lineto(698.81321096,568.475023)
\lineto(694.52521096,568.475023)
\lineto(694.52521096,578.235023)
\lineto(695.99721096,578.235023)
\curveto(696.76521096,579.40835633)(697.41587763,580.74168967)(697.94921096,582.235023)
\curveto(698.4825443,583.74968967)(698.90921096,585.34968967)(699.22921096,587.035023)
\curveto(699.54921096,588.74168967)(699.78387763,590.46968967)(699.93321096,592.219023)
\closepath
\moveto(707.51721096,588.635023)
\lineto(703.93321096,588.635023)
\curveto(703.67721096,586.69368967)(703.32521096,584.84835633)(702.87721096,583.099023)
\curveto(702.42921096,581.371023)(701.79987763,579.74968967)(700.98921096,578.235023)
\lineto(707.51721096,578.235023)
\closepath
}
}
{
\newrgbcolor{curcolor}{0 0 0}
\pscustom[linestyle=none,fillstyle=solid,fillcolor=curcolor]
{
\newpath
\moveto(724.9891968,592.571023)
\curveto(727.33586347,592.571023)(729.12786347,592.059023)(730.3651968,591.035023)
\curveto(731.62386347,590.03235633)(732.2531968,588.48568967)(732.2531968,586.395023)
\lineto(732.2531968,574.747023)
\lineto(728.9251968,574.747023)
\lineto(727.9971968,577.115023)
\lineto(727.8691968,577.115023)
\curveto(727.12253014,576.17635633)(726.3331968,575.49368967)(725.5011968,575.067023)
\curveto(724.6691968,574.64035633)(723.52786347,574.427023)(722.0771968,574.427023)
\curveto(720.51986347,574.427023)(719.2291968,574.875023)(718.2051968,575.771023)
\curveto(717.1811968,576.667023)(716.6691968,578.06435633)(716.6691968,579.963023)
\curveto(716.6691968,581.819023)(717.31986347,583.18435633)(718.6211968,584.059023)
\curveto(719.92253014,584.93368967)(721.87453014,585.42435633)(724.4771968,585.531023)
\lineto(727.5171968,585.627023)
\lineto(727.5171968,586.395023)
\curveto(727.5171968,587.31235633)(727.27186347,587.98435633)(726.7811968,588.411023)
\curveto(726.31186347,588.83768967)(725.65053014,589.051023)(724.7971968,589.051023)
\curveto(723.94386347,589.051023)(723.11186347,588.923023)(722.3011968,588.667023)
\curveto(721.49053014,588.43235633)(720.67986347,588.13368967)(719.8691968,587.771023)
\lineto(718.3011968,591.003023)
\curveto(719.21853014,591.47235633)(720.2531968,591.84568967)(721.4051968,592.123023)
\curveto(722.5571968,592.42168967)(723.75186347,592.571023)(724.9891968,592.571023)
\closepath
\moveto(727.5171968,582.843023)
\lineto(725.6611968,582.779023)
\curveto(724.1251968,582.73635633)(723.05853014,582.459023)(722.4611968,581.947023)
\curveto(721.86386347,581.435023)(721.5651968,580.763023)(721.5651968,579.931023)
\curveto(721.5651968,579.20568967)(721.77853014,578.683023)(722.2051968,578.363023)
\curveto(722.63186347,578.06435633)(723.18653014,577.915023)(723.8691968,577.915023)
\curveto(724.8931968,577.915023)(725.7571968,578.21368967)(726.4611968,578.811023)
\curveto(727.1651968,579.42968967)(727.5171968,580.29368967)(727.5171968,581.403023)
\closepath
}
}
{
\newrgbcolor{curcolor}{0 0 0}
\pscustom[linestyle=none,fillstyle=solid,fillcolor=curcolor]
{
\newpath
\moveto(741.91719973,592.219023)
\lineto(741.91719973,585.499023)
\lineto(748.57319973,585.499023)
\lineto(748.57319973,592.219023)
\lineto(753.34119973,592.219023)
\lineto(753.34119973,574.747023)
\lineto(748.57319973,574.747023)
\lineto(748.57319973,581.947023)
\lineto(741.91719973,581.947023)
\lineto(741.91719973,574.747023)
\lineto(737.14919973,574.747023)
\lineto(737.14919973,592.219023)
\closepath
}
}
{
\newrgbcolor{curcolor}{0 0 0}
\pscustom[linestyle=none,fillstyle=solid,fillcolor=curcolor]
{
\newpath
\moveto(763.10122073,592.219023)
\lineto(763.10122073,585.499023)
\lineto(769.75722073,585.499023)
\lineto(769.75722073,592.219023)
\lineto(774.52522073,592.219023)
\lineto(774.52522073,574.747023)
\lineto(769.75722073,574.747023)
\lineto(769.75722073,581.947023)
\lineto(763.10122073,581.947023)
\lineto(763.10122073,574.747023)
\lineto(758.33322073,574.747023)
\lineto(758.33322073,592.219023)
\closepath
}
}
{
\newrgbcolor{curcolor}{0 0 0}
\pscustom[linestyle=none,fillstyle=solid,fillcolor=curcolor]
{
\newpath
\moveto(779.51724172,574.747023)
\lineto(779.51724172,592.219023)
\lineto(784.28524172,592.219023)
\lineto(784.28524172,585.467023)
\lineto(786.58924172,585.467023)
\curveto(789.25590839,585.467023)(791.22924172,585.04035633)(792.50924172,584.187023)
\curveto(793.78924172,583.33368967)(794.42924172,582.043023)(794.42924172,580.315023)
\curveto(794.42924172,578.60835633)(793.83190839,577.25368967)(792.63724172,576.251023)
\curveto(791.44257506,575.24835633)(789.47990839,574.747023)(786.74924172,574.747023)
\closepath
\moveto(796.95724172,574.747023)
\lineto(796.95724172,592.219023)
\lineto(801.72524172,592.219023)
\lineto(801.72524172,574.747023)
\closepath
\moveto(784.28524172,578.043023)
\lineto(786.49324172,578.043023)
\curveto(787.43190839,578.043023)(788.18924172,578.203023)(788.76524172,578.523023)
\curveto(789.36257506,578.86435633)(789.66124172,579.44035633)(789.66124172,580.251023)
\curveto(789.66124172,581.531023)(788.58390839,582.171023)(786.42924172,582.171023)
\lineto(784.28524172,582.171023)
\closepath
}
}
{
\newrgbcolor{curcolor}{0 0 0}
\pscustom[linestyle=none,fillstyle=solid,fillcolor=curcolor]
{
\newpath
\moveto(810.30125393,583.675023)
\lineto(804.66925393,592.219023)
\lineto(810.07725393,592.219023)
\lineto(813.46925393,586.651023)
\lineto(816.89325393,592.219023)
\lineto(822.30125393,592.219023)
\lineto(816.60525393,583.675023)
\lineto(822.55725393,574.747023)
\lineto(817.14925393,574.747023)
\lineto(813.46925393,580.731023)
\lineto(809.78925393,574.747023)
\lineto(804.38125393,574.747023)
\closepath
}
}
{
\newrgbcolor{curcolor}{0 0 0}
\pscustom[linestyle=none,fillstyle=solid,fillcolor=curcolor]
{
\newpath
\moveto(672.70117459,543.515023)
\curveto(672.70117459,546.11768967)(673.07450792,548.62435633)(673.82117459,551.035023)
\curveto(674.58917459,553.467023)(675.78384125,555.65368967)(677.40517459,557.595023)
\lineto(681.30917459,557.595023)
\curveto(679.85850792,555.58968967)(678.74917459,553.38168967)(677.98117459,550.971023)
\curveto(677.23450792,548.56035633)(676.86117459,546.08568967)(676.86117459,543.547023)
\curveto(676.86117459,541.07235633)(677.23450792,538.62968967)(677.98117459,536.219023)
\curveto(678.72784125,533.82968967)(679.82650792,531.65368967)(681.27717459,529.691023)
\lineto(677.40517459,529.691023)
\curveto(675.78384125,531.56835633)(674.58917459,533.691023)(673.82117459,536.059023)
\curveto(673.07450792,538.44835633)(672.70117459,540.93368967)(672.70117459,543.515023)
\closepath
}
}
{
\newrgbcolor{curcolor}{0 0 0}
\pscustom[linestyle=none,fillstyle=solid,fillcolor=curcolor]
{
\newpath
\moveto(705.88519705,546.203023)
\curveto(705.88519705,543.835023)(705.49053038,541.76568967)(704.70119705,539.995023)
\curveto(703.93319705,538.24568967)(702.73853038,536.88035633)(701.11719705,535.899023)
\curveto(699.51719705,534.91768967)(697.47986371,534.427023)(695.00519705,534.427023)
\curveto(692.53053038,534.427023)(690.48253038,534.91768967)(688.86119705,535.899023)
\curveto(687.26119705,536.88035633)(686.06653038,538.25635633)(685.27719705,540.027023)
\curveto(684.50919705,541.79768967)(684.12519705,543.867023)(684.12519705,546.235023)
\curveto(684.12519705,548.603023)(684.50919705,550.66168967)(685.27719705,552.411023)
\curveto(686.06653038,554.16035633)(687.26119705,555.515023)(688.86119705,556.475023)
\curveto(690.48253038,557.45635633)(692.54119705,557.947023)(695.03719705,557.947023)
\curveto(697.51186371,557.947023)(699.54919705,557.45635633)(701.14919705,556.475023)
\curveto(702.74919705,555.515023)(703.93319705,554.14968967)(704.70119705,552.379023)
\curveto(705.49053038,550.62968967)(705.88519705,548.571023)(705.88519705,546.203023)
\closepath
\moveto(689.21319705,546.203023)
\curveto(689.21319705,543.81368967)(689.67186371,541.92568967)(690.58919705,540.539023)
\curveto(691.50653038,539.17368967)(692.97853038,538.491023)(695.00519705,538.491023)
\curveto(697.07453038,538.491023)(698.55719705,539.17368967)(699.45319705,540.539023)
\curveto(700.34919705,541.92568967)(700.79719705,543.81368967)(700.79719705,546.203023)
\curveto(700.79719705,548.59235633)(700.34919705,550.46968967)(699.45319705,551.835023)
\curveto(698.55719705,553.22168967)(697.08519705,553.915023)(695.03719705,553.915023)
\curveto(692.98919705,553.915023)(691.50653038,553.22168967)(690.58919705,551.835023)
\curveto(689.67186371,550.46968967)(689.21319705,548.59235633)(689.21319705,546.203023)
\closepath
}
}
{
\newrgbcolor{curcolor}{0 0 0}
\pscustom[linestyle=none,fillstyle=solid,fillcolor=curcolor]
{
\newpath
\moveto(726.07717459,552.251023)
\curveto(726.07717459,550.715023)(725.53317459,549.47768967)(724.44517459,548.539023)
\curveto(723.35717459,547.60035633)(721.97050792,547.01368967)(720.28517459,546.779023)
\lineto(720.28517459,546.683023)
\curveto(722.37584125,546.46968967)(723.97584125,545.883023)(725.08517459,544.923023)
\curveto(726.21584125,543.98435633)(726.78117459,542.75768967)(726.78117459,541.243023)
\curveto(726.78117459,539.23768967)(725.95984125,537.595023)(724.31717459,536.315023)
\curveto(722.69584125,535.05635633)(720.30650792,534.427023)(717.14917459,534.427023)
\curveto(715.42117459,534.427023)(713.88517459,534.53368967)(712.54117459,534.747023)
\curveto(711.21850792,534.96035633)(710.06650792,535.26968967)(709.08517459,535.675023)
\lineto(709.08517459,539.739023)
\curveto(709.74650792,539.419023)(710.49317459,539.14168967)(711.32517459,538.907023)
\curveto(712.15717459,538.69368967)(712.99984125,538.523023)(713.85317459,538.395023)
\curveto(714.70650792,538.28835633)(715.49584125,538.235023)(716.22117459,538.235023)
\curveto(718.24784125,538.235023)(719.70917459,538.523023)(720.60517459,539.099023)
\curveto(721.52250792,539.69635633)(721.98117459,540.52835633)(721.98117459,541.595023)
\curveto(721.98117459,542.683023)(721.31984125,543.47235633)(719.99717459,543.963023)
\curveto(718.67450792,544.475023)(716.89317459,544.731023)(714.65317459,544.731023)
\lineto(712.50917459,544.731023)
\lineto(712.50917459,548.507023)
\lineto(714.42917459,548.507023)
\curveto(716.24250792,548.507023)(717.65050792,548.62435633)(718.65317459,548.859023)
\curveto(719.65584125,549.09368967)(720.35984125,549.435023)(720.76517459,549.883023)
\curveto(721.17050792,550.331023)(721.37317459,550.86435633)(721.37317459,551.483023)
\curveto(721.37317459,552.27235633)(721.02117459,552.891023)(720.31717459,553.339023)
\curveto(719.61317459,553.80835633)(718.57850792,554.043023)(717.21317459,554.043023)
\curveto(716.01850792,554.043023)(714.90917459,553.87235633)(713.88517459,553.531023)
\curveto(712.86117459,553.18968967)(711.90117459,552.74168967)(711.00517459,552.187023)
\lineto(708.89317459,555.419023)
\curveto(710.06650792,556.187023)(711.36784125,556.795023)(712.79717459,557.243023)
\curveto(714.22650792,557.691023)(715.92250792,557.915023)(717.88517459,557.915023)
\curveto(720.46650792,557.915023)(722.47184125,557.38168967)(723.90117459,556.315023)
\curveto(725.35184125,555.24835633)(726.07717459,553.89368967)(726.07717459,552.251023)
\closepath
}
}
{
\newrgbcolor{curcolor}{0 0 0}
\pscustom[linestyle=none,fillstyle=solid,fillcolor=curcolor]
{
\newpath
\moveto(749.37319754,557.595023)
\lineto(742.30119754,541.435023)
\curveto(741.66119754,539.963023)(740.97853087,538.70435633)(740.25319754,537.659023)
\curveto(739.5278642,536.61368967)(738.61053087,535.81368967)(737.50119754,535.259023)
\curveto(736.41319754,534.70435633)(734.97319754,534.427023)(733.18119754,534.427023)
\curveto(732.62653087,534.427023)(732.01853087,534.46968967)(731.35719754,534.555023)
\curveto(730.6958642,534.64035633)(730.0878642,534.75768967)(729.53319754,534.907023)
\lineto(729.53319754,539.067023)
\curveto(730.04519754,538.85368967)(730.5998642,538.70435633)(731.19719754,538.619023)
\curveto(731.8158642,538.53368967)(732.40253087,538.491023)(732.95719754,538.491023)
\curveto(734.0238642,538.491023)(734.7918642,538.747023)(735.26119754,539.259023)
\curveto(735.73053087,539.79235633)(736.1038642,540.43235633)(736.38119754,541.179023)
\lineto(728.47719754,557.595023)
\lineto(733.59719754,557.595023)
\lineto(737.85319754,547.707023)
\curveto(738.00253087,547.387023)(738.20519754,546.92835633)(738.46119754,546.331023)
\curveto(738.71719754,545.755023)(738.90919754,545.26435633)(739.03719754,544.859023)
\lineto(739.19719754,544.859023)
\curveto(739.32519754,545.243023)(739.50653087,545.74435633)(739.74119754,546.363023)
\curveto(739.99719754,546.98168967)(740.22119754,547.52568967)(740.41319754,547.995023)
\lineto(744.38119754,557.595023)
\closepath
}
}
{
\newrgbcolor{curcolor}{0 0 0}
\pscustom[linestyle=none,fillstyle=solid,fillcolor=curcolor]
{
\newpath
\moveto(758.94120095,543.515023)
\curveto(758.94120095,540.93368967)(758.55720095,538.44835633)(757.78920095,536.059023)
\curveto(757.04253429,533.691023)(755.85853429,531.56835633)(754.23720095,529.691023)
\lineto(750.36520095,529.691023)
\curveto(751.79453429,531.65368967)(752.88253429,533.82968967)(753.62920095,536.219023)
\curveto(754.39720095,538.62968967)(754.78120095,541.07235633)(754.78120095,543.547023)
\curveto(754.78120095,546.08568967)(754.39720095,548.56035633)(753.62920095,550.971023)
\curveto(752.88253429,553.38168967)(751.78386762,555.58968967)(750.33320095,557.595023)
\lineto(754.23720095,557.595023)
\curveto(755.85853429,555.65368967)(757.04253429,553.467023)(757.78920095,551.035023)
\curveto(758.55720095,548.62435633)(758.94120095,546.11768967)(758.94120095,543.515023)
\closepath
}
}
{
\newrgbcolor{curcolor}{0 0 0}
\pscustom[linestyle=none,fillstyle=solid,fillcolor=curcolor]
{
\newpath
\moveto(764.29063203,314.3569135)
\lineto(764.29063203,337.2049135)
\lineto(778.72263203,337.2049135)
\lineto(778.72263203,333.2049135)
\lineto(769.12263203,333.2049135)
\lineto(769.12263203,328.4369135)
\lineto(771.04263203,328.4369135)
\curveto(773.19729869,328.4369135)(774.95729869,328.13824683)(776.32263203,327.5409135)
\curveto(777.70929869,326.94358017)(778.73329869,326.12224683)(779.39463203,325.0769135)
\curveto(780.05596536,324.03158017)(780.38663203,322.8369135)(780.38663203,321.4929135)
\curveto(780.38663203,319.23158017)(779.62929869,317.47158017)(778.11463203,316.2129135)
\curveto(776.62129869,314.97558017)(774.23196536,314.3569135)(770.94663203,314.3569135)
\closepath
\moveto(769.12263203,318.3249135)
\lineto(770.75463203,318.3249135)
\curveto(772.22663203,318.3249135)(773.37863203,318.55958017)(774.21063203,319.0289135)
\curveto(775.06396536,319.49824683)(775.49063203,320.31958017)(775.49063203,321.4929135)
\curveto(775.49063203,322.7089135)(775.03196536,323.5089135)(774.11463203,323.8929135)
\curveto(773.19729869,324.2769135)(771.94929869,324.4689135)(770.37063203,324.4689135)
\lineto(769.12263203,324.4689135)
\closepath
}
}
{
\newrgbcolor{curcolor}{0 0 0}
\pscustom[linestyle=none,fillstyle=solid,fillcolor=curcolor]
{
\newpath
\moveto(791.49064765,332.1809135)
\curveto(793.83731432,332.1809135)(795.62931432,331.6689135)(796.86664765,330.6449135)
\curveto(798.12531432,329.64224683)(798.75464765,328.09558017)(798.75464765,326.0049135)
\lineto(798.75464765,314.3569135)
\lineto(795.42664765,314.3569135)
\lineto(794.49864765,316.7249135)
\lineto(794.37064765,316.7249135)
\curveto(793.62398098,315.78624683)(792.83464765,315.10358017)(792.00264765,314.6769135)
\curveto(791.17064765,314.25024683)(790.02931432,314.0369135)(788.57864765,314.0369135)
\curveto(787.02131432,314.0369135)(785.73064765,314.4849135)(784.70664765,315.3809135)
\curveto(783.68264765,316.2769135)(783.17064765,317.67424683)(783.17064765,319.5729135)
\curveto(783.17064765,321.4289135)(783.82131432,322.79424683)(785.12264765,323.6689135)
\curveto(786.42398098,324.54358017)(788.37598098,325.03424683)(790.97864765,325.1409135)
\lineto(794.01864765,325.2369135)
\lineto(794.01864765,326.0049135)
\curveto(794.01864765,326.92224683)(793.77331432,327.59424683)(793.28264765,328.0209135)
\curveto(792.81331432,328.44758017)(792.15198098,328.6609135)(791.29864765,328.6609135)
\curveto(790.44531432,328.6609135)(789.61331432,328.5329135)(788.80264765,328.2769135)
\curveto(787.99198098,328.04224683)(787.18131432,327.74358017)(786.37064765,327.3809135)
\lineto(784.80264765,330.6129135)
\curveto(785.71998098,331.08224683)(786.75464765,331.45558017)(787.90664765,331.7329135)
\curveto(789.05864765,332.03158017)(790.25331432,332.1809135)(791.49064765,332.1809135)
\closepath
\moveto(794.01864765,322.4529135)
\lineto(792.16264765,322.3889135)
\curveto(790.62664765,322.34624683)(789.55998098,322.0689135)(788.96264765,321.5569135)
\curveto(788.36531432,321.0449135)(788.06664765,320.3729135)(788.06664765,319.5409135)
\curveto(788.06664765,318.81558017)(788.27998098,318.2929135)(788.70664765,317.9729135)
\curveto(789.13331432,317.67424683)(789.68798098,317.5249135)(790.37064765,317.5249135)
\curveto(791.39464765,317.5249135)(792.25864765,317.82358017)(792.96264765,318.4209135)
\curveto(793.66664765,319.03958017)(794.01864765,319.90358017)(794.01864765,321.0129135)
\closepath
}
}
{
\newrgbcolor{curcolor}{0 0 0}
\pscustom[linestyle=none,fillstyle=solid,fillcolor=curcolor]
{
\newpath
\moveto(810.01865058,332.1489135)
\curveto(811.27731725,332.1489135)(812.45065058,331.97824683)(813.53865058,331.6369135)
\curveto(814.64798391,331.3169135)(815.53331725,330.81558017)(816.19465058,330.1329135)
\curveto(816.87731725,329.45024683)(817.21865058,328.57558017)(817.21865058,327.5089135)
\curveto(817.21865058,326.46358017)(816.89865058,325.63158017)(816.25865058,325.0129135)
\curveto(815.61865058,324.39424683)(814.77598391,323.94624683)(813.73065058,323.6689135)
\lineto(813.73065058,323.5089135)
\curveto(814.47731725,323.33824683)(815.14931725,323.0929135)(815.74665058,322.7729135)
\curveto(816.34398391,322.47424683)(816.81331725,322.05824683)(817.15465058,321.5249135)
\curveto(817.51731725,321.0129135)(817.69865058,320.31958017)(817.69865058,319.4449135)
\curveto(817.69865058,318.4849135)(817.38931725,317.5889135)(816.77065058,316.7569135)
\curveto(816.17331725,315.94624683)(815.23465058,315.2849135)(813.95465058,314.7729135)
\curveto(812.67465058,314.28224683)(811.05331725,314.0369135)(809.09065058,314.0369135)
\curveto(806.18931725,314.0369135)(803.94931725,314.39958017)(802.37065058,315.1249135)
\lineto(802.37065058,319.0609135)
\curveto(803.09598391,318.71958017)(803.98131725,318.41024683)(805.02665058,318.1329135)
\curveto(806.09331725,317.85558017)(807.22398391,317.7169135)(808.41865058,317.7169135)
\curveto(809.71998391,317.7169135)(810.81865058,317.86624683)(811.71465058,318.1649135)
\curveto(812.61065058,318.46358017)(813.05865058,318.98624683)(813.05865058,319.7329135)
\curveto(813.05865058,321.11958017)(811.40531725,321.8129135)(808.09865058,321.8129135)
\lineto(806.24265058,321.8129135)
\lineto(806.24265058,325.1089135)
\lineto(808.00265058,325.1089135)
\curveto(809.58131725,325.1089135)(810.79731725,325.2369135)(811.65065058,325.4929135)
\curveto(812.52531725,325.7489135)(812.96265058,326.2289135)(812.96265058,326.9329135)
\curveto(812.96265058,327.48758017)(812.68531725,327.90358017)(812.13065058,328.1809135)
\curveto(811.57598391,328.47958017)(810.66931725,328.6289135)(809.41065058,328.6289135)
\curveto(808.57865058,328.6289135)(807.66131725,328.5329135)(806.65865058,328.3409135)
\curveto(805.67731725,328.1489135)(804.75998391,327.87158017)(803.90665058,327.5089135)
\lineto(802.49865058,330.8369135)
\curveto(803.50131725,331.2209135)(804.58931725,331.53024683)(805.76265058,331.7649135)
\curveto(806.95731725,332.0209135)(808.37598391,332.1489135)(810.01865058,332.1489135)
\closepath
}
}
{
\newrgbcolor{curcolor}{0 0 0}
\pscustom[linestyle=none,fillstyle=solid,fillcolor=curcolor]
{
\newpath
\moveto(828.57866035,332.1809135)
\curveto(830.92532701,332.1809135)(832.71732701,331.6689135)(833.95466035,330.6449135)
\curveto(835.21332701,329.64224683)(835.84266035,328.09558017)(835.84266035,326.0049135)
\lineto(835.84266035,314.3569135)
\lineto(832.51466035,314.3569135)
\lineto(831.58666035,316.7249135)
\lineto(831.45866035,316.7249135)
\curveto(830.71199368,315.78624683)(829.92266035,315.10358017)(829.09066035,314.6769135)
\curveto(828.25866035,314.25024683)(827.11732701,314.0369135)(825.66666035,314.0369135)
\curveto(824.10932701,314.0369135)(822.81866035,314.4849135)(821.79466035,315.3809135)
\curveto(820.77066035,316.2769135)(820.25866035,317.67424683)(820.25866035,319.5729135)
\curveto(820.25866035,321.4289135)(820.90932701,322.79424683)(822.21066035,323.6689135)
\curveto(823.51199368,324.54358017)(825.46399368,325.03424683)(828.06666035,325.1409135)
\lineto(831.10666035,325.2369135)
\lineto(831.10666035,326.0049135)
\curveto(831.10666035,326.92224683)(830.86132701,327.59424683)(830.37066035,328.0209135)
\curveto(829.90132701,328.44758017)(829.23999368,328.6609135)(828.38666035,328.6609135)
\curveto(827.53332701,328.6609135)(826.70132701,328.5329135)(825.89066035,328.2769135)
\curveto(825.07999368,328.04224683)(824.26932701,327.74358017)(823.45866035,327.3809135)
\lineto(821.89066035,330.6129135)
\curveto(822.80799368,331.08224683)(823.84266035,331.45558017)(824.99466035,331.7329135)
\curveto(826.14666035,332.03158017)(827.34132701,332.1809135)(828.57866035,332.1809135)
\closepath
\moveto(831.10666035,322.4529135)
\lineto(829.25066035,322.3889135)
\curveto(827.71466035,322.34624683)(826.64799368,322.0689135)(826.05066035,321.5569135)
\curveto(825.45332701,321.0449135)(825.15466035,320.3729135)(825.15466035,319.5409135)
\curveto(825.15466035,318.81558017)(825.36799368,318.2929135)(825.79466035,317.9729135)
\curveto(826.22132701,317.67424683)(826.77599368,317.5249135)(827.45866035,317.5249135)
\curveto(828.48266035,317.5249135)(829.34666035,317.82358017)(830.05066035,318.4209135)
\curveto(830.75466035,319.03958017)(831.10666035,319.90358017)(831.10666035,321.0129135)
\closepath
}
}
{
\newrgbcolor{curcolor}{0 0 0}
\pscustom[linestyle=none,fillstyle=solid,fillcolor=curcolor]
{
\newpath
\moveto(864.7706706,331.8289135)
\lineto(864.7706706,317.8449135)
\lineto(867.3306706,317.8449135)
\lineto(867.3306706,308.0849135)
\lineto(863.0426706,308.0849135)
\lineto(863.0426706,314.3569135)
\lineto(851.2986706,314.3569135)
\lineto(851.2986706,308.0849135)
\lineto(847.0106706,308.0849135)
\lineto(847.0106706,317.8449135)
\lineto(848.4826706,317.8449135)
\curveto(849.2506706,319.01824683)(849.90133727,320.35158017)(850.4346706,321.8449135)
\curveto(850.96800393,323.35958017)(851.3946706,324.95958017)(851.7146706,326.6449135)
\curveto(852.0346706,328.35158017)(852.26933727,330.07958017)(852.4186706,331.8289135)
\closepath
\moveto(860.0026706,328.2449135)
\lineto(856.4186706,328.2449135)
\curveto(856.1626706,326.30358017)(855.8106706,324.45824683)(855.3626706,322.7089135)
\curveto(854.9146706,320.9809135)(854.28533727,319.35958017)(853.4746706,317.8449135)
\lineto(860.0026706,317.8449135)
\closepath
}
}
{
\newrgbcolor{curcolor}{0 0 0}
\pscustom[linestyle=none,fillstyle=solid,fillcolor=curcolor]
{
\newpath
\moveto(877.47465644,332.1809135)
\curveto(879.82132311,332.1809135)(881.61332311,331.6689135)(882.85065644,330.6449135)
\curveto(884.10932311,329.64224683)(884.73865644,328.09558017)(884.73865644,326.0049135)
\lineto(884.73865644,314.3569135)
\lineto(881.41065644,314.3569135)
\lineto(880.48265644,316.7249135)
\lineto(880.35465644,316.7249135)
\curveto(879.60798977,315.78624683)(878.81865644,315.10358017)(877.98665644,314.6769135)
\curveto(877.15465644,314.25024683)(876.01332311,314.0369135)(874.56265644,314.0369135)
\curveto(873.00532311,314.0369135)(871.71465644,314.4849135)(870.69065644,315.3809135)
\curveto(869.66665644,316.2769135)(869.15465644,317.67424683)(869.15465644,319.5729135)
\curveto(869.15465644,321.4289135)(869.80532311,322.79424683)(871.10665644,323.6689135)
\curveto(872.40798977,324.54358017)(874.35998977,325.03424683)(876.96265644,325.1409135)
\lineto(880.00265644,325.2369135)
\lineto(880.00265644,326.0049135)
\curveto(880.00265644,326.92224683)(879.75732311,327.59424683)(879.26665644,328.0209135)
\curveto(878.79732311,328.44758017)(878.13598977,328.6609135)(877.28265644,328.6609135)
\curveto(876.42932311,328.6609135)(875.59732311,328.5329135)(874.78665644,328.2769135)
\curveto(873.97598977,328.04224683)(873.16532311,327.74358017)(872.35465644,327.3809135)
\lineto(870.78665644,330.6129135)
\curveto(871.70398977,331.08224683)(872.73865644,331.45558017)(873.89065644,331.7329135)
\curveto(875.04265644,332.03158017)(876.23732311,332.1809135)(877.47465644,332.1809135)
\closepath
\moveto(880.00265644,322.4529135)
\lineto(878.14665644,322.3889135)
\curveto(876.61065644,322.34624683)(875.54398977,322.0689135)(874.94665644,321.5569135)
\curveto(874.34932311,321.0449135)(874.05065644,320.3729135)(874.05065644,319.5409135)
\curveto(874.05065644,318.81558017)(874.26398977,318.2929135)(874.69065644,317.9729135)
\curveto(875.11732311,317.67424683)(875.67198977,317.5249135)(876.35465644,317.5249135)
\curveto(877.37865644,317.5249135)(878.24265644,317.82358017)(878.94665644,318.4209135)
\curveto(879.65065644,319.03958017)(880.00265644,319.90358017)(880.00265644,321.0129135)
\closepath
}
}
{
\newrgbcolor{curcolor}{0 0 0}
\pscustom[linestyle=none,fillstyle=solid,fillcolor=curcolor]
{
\newpath
\moveto(894.40265937,331.8289135)
\lineto(894.40265937,325.1089135)
\lineto(901.05865937,325.1089135)
\lineto(901.05865937,331.8289135)
\lineto(905.82665937,331.8289135)
\lineto(905.82665937,314.3569135)
\lineto(901.05865937,314.3569135)
\lineto(901.05865937,321.5569135)
\lineto(894.40265937,321.5569135)
\lineto(894.40265937,314.3569135)
\lineto(889.63465937,314.3569135)
\lineto(889.63465937,331.8289135)
\closepath
}
}
{
\newrgbcolor{curcolor}{0 0 0}
\pscustom[linestyle=none,fillstyle=solid,fillcolor=curcolor]
{
\newpath
\moveto(915.58668037,331.8289135)
\lineto(915.58668037,325.1089135)
\lineto(922.24268037,325.1089135)
\lineto(922.24268037,331.8289135)
\lineto(927.01068037,331.8289135)
\lineto(927.01068037,314.3569135)
\lineto(922.24268037,314.3569135)
\lineto(922.24268037,321.5569135)
\lineto(915.58668037,321.5569135)
\lineto(915.58668037,314.3569135)
\lineto(910.81868037,314.3569135)
\lineto(910.81868037,331.8289135)
\closepath
}
}
{
\newrgbcolor{curcolor}{0 0 0}
\pscustom[linestyle=none,fillstyle=solid,fillcolor=curcolor]
{
\newpath
\moveto(932.00270136,314.3569135)
\lineto(932.00270136,331.8289135)
\lineto(936.77070136,331.8289135)
\lineto(936.77070136,325.0769135)
\lineto(939.07470136,325.0769135)
\curveto(941.74136803,325.0769135)(943.71470136,324.65024683)(944.99470136,323.7969135)
\curveto(946.27470136,322.94358017)(946.91470136,321.6529135)(946.91470136,319.9249135)
\curveto(946.91470136,318.21824683)(946.31736803,316.86358017)(945.12270136,315.8609135)
\curveto(943.9280347,314.85824683)(941.96536803,314.3569135)(939.23470136,314.3569135)
\closepath
\moveto(949.44270136,314.3569135)
\lineto(949.44270136,331.8289135)
\lineto(954.21070136,331.8289135)
\lineto(954.21070136,314.3569135)
\closepath
\moveto(936.77070136,317.6529135)
\lineto(938.97870136,317.6529135)
\curveto(939.91736803,317.6529135)(940.67470136,317.8129135)(941.25070136,318.1329135)
\curveto(941.8480347,318.47424683)(942.14670136,319.05024683)(942.14670136,319.8609135)
\curveto(942.14670136,321.1409135)(941.06936803,321.7809135)(938.91470136,321.7809135)
\lineto(936.77070136,321.7809135)
\closepath
}
}
{
\newrgbcolor{curcolor}{0 0 0}
\pscustom[linestyle=none,fillstyle=solid,fillcolor=curcolor]
{
\newpath
\moveto(962.78671357,323.2849135)
\lineto(957.15471357,331.8289135)
\lineto(962.56271357,331.8289135)
\lineto(965.95471357,326.2609135)
\lineto(969.37871357,331.8289135)
\lineto(974.78671357,331.8289135)
\lineto(969.09071357,323.2849135)
\lineto(975.04271357,314.3569135)
\lineto(969.63471357,314.3569135)
\lineto(965.95471357,320.3409135)
\lineto(962.27471357,314.3569135)
\lineto(956.86671357,314.3569135)
\closepath
}
}
{
\newrgbcolor{curcolor}{0 0 0}
\pscustom[linestyle=none,fillstyle=solid,fillcolor=curcolor]
{
\newpath
\moveto(757.10667817,283.1249135)
\curveto(757.10667817,285.72758017)(757.4800115,288.23424683)(758.22667817,290.6449135)
\curveto(758.99467817,293.0769135)(760.18934484,295.26358017)(761.81067817,297.2049135)
\lineto(765.71467817,297.2049135)
\curveto(764.2640115,295.19958017)(763.15467817,292.99158017)(762.38667817,290.5809135)
\curveto(761.6400115,288.17024683)(761.26667817,285.69558017)(761.26667817,283.1569135)
\curveto(761.26667817,280.68224683)(761.6400115,278.23958017)(762.38667817,275.8289135)
\curveto(763.13334484,273.43958017)(764.2320115,271.26358017)(765.68267817,269.3009135)
\lineto(761.81067817,269.3009135)
\curveto(760.18934484,271.17824683)(758.99467817,273.3009135)(758.22667817,275.6689135)
\curveto(757.4800115,278.05824683)(757.10667817,280.54358017)(757.10667817,283.1249135)
\closepath
}
}
{
\newrgbcolor{curcolor}{0 0 0}
\pscustom[linestyle=none,fillstyle=solid,fillcolor=curcolor]
{
\newpath
\moveto(788.27470063,274.3569135)
\lineto(783.44270063,274.3569135)
\lineto(783.44270063,284.2129135)
\lineto(774.38670063,284.2129135)
\lineto(774.38670063,274.3569135)
\lineto(769.55470063,274.3569135)
\lineto(769.55470063,297.2049135)
\lineto(774.38670063,297.2049135)
\lineto(774.38670063,288.2449135)
\lineto(783.44270063,288.2449135)
\lineto(783.44270063,297.2049135)
\lineto(788.27470063,297.2049135)
\closepath
}
}
{
\newrgbcolor{curcolor}{0 0 0}
\pscustom[linestyle=none,fillstyle=solid,fillcolor=curcolor]
{
\newpath
\moveto(800.8186811,292.1809135)
\curveto(803.16534777,292.1809135)(804.95734777,291.6689135)(806.1946811,290.6449135)
\curveto(807.45334777,289.64224683)(808.0826811,288.09558017)(808.0826811,286.0049135)
\lineto(808.0826811,274.3569135)
\lineto(804.7546811,274.3569135)
\lineto(803.8266811,276.7249135)
\lineto(803.6986811,276.7249135)
\curveto(802.95201443,275.78624683)(802.1626811,275.10358017)(801.3306811,274.6769135)
\curveto(800.4986811,274.25024683)(799.35734777,274.0369135)(797.9066811,274.0369135)
\curveto(796.34934777,274.0369135)(795.0586811,274.4849135)(794.0346811,275.3809135)
\curveto(793.0106811,276.2769135)(792.4986811,277.67424683)(792.4986811,279.5729135)
\curveto(792.4986811,281.4289135)(793.14934777,282.79424683)(794.4506811,283.6689135)
\curveto(795.75201443,284.54358017)(797.70401443,285.03424683)(800.3066811,285.1409135)
\lineto(803.3466811,285.2369135)
\lineto(803.3466811,286.0049135)
\curveto(803.3466811,286.92224683)(803.10134777,287.59424683)(802.6106811,288.0209135)
\curveto(802.14134777,288.44758017)(801.48001443,288.6609135)(800.6266811,288.6609135)
\curveto(799.77334777,288.6609135)(798.94134777,288.5329135)(798.1306811,288.2769135)
\curveto(797.32001443,288.04224683)(796.50934777,287.74358017)(795.6986811,287.3809135)
\lineto(794.1306811,290.6129135)
\curveto(795.04801443,291.08224683)(796.0826811,291.45558017)(797.2346811,291.7329135)
\curveto(798.3866811,292.03158017)(799.58134777,292.1809135)(800.8186811,292.1809135)
\closepath
\moveto(803.3466811,282.4529135)
\lineto(801.4906811,282.3889135)
\curveto(799.9546811,282.34624683)(798.88801443,282.0689135)(798.2906811,281.5569135)
\curveto(797.69334777,281.0449135)(797.3946811,280.3729135)(797.3946811,279.5409135)
\curveto(797.3946811,278.81558017)(797.60801443,278.2929135)(798.0346811,277.9729135)
\curveto(798.46134777,277.67424683)(799.01601443,277.5249135)(799.6986811,277.5249135)
\curveto(800.7226811,277.5249135)(801.5866811,277.82358017)(802.2906811,278.4209135)
\curveto(802.9946811,279.03958017)(803.3466811,279.90358017)(803.3466811,281.0129135)
\closepath
}
}
{
\newrgbcolor{curcolor}{0 0 0}
\pscustom[linestyle=none,fillstyle=solid,fillcolor=curcolor]
{
\newpath
\moveto(824.43468403,291.8289135)
\lineto(829.68268403,291.8289135)
\lineto(822.77068403,283.4449135)
\lineto(830.29068403,274.3569135)
\lineto(824.88268403,274.3569135)
\lineto(817.74668403,283.2209135)
\lineto(817.74668403,274.3569135)
\lineto(812.97868403,274.3569135)
\lineto(812.97868403,291.8289135)
\lineto(817.74668403,291.8289135)
\lineto(817.74668403,283.3489135)
\closepath
}
}
{
\newrgbcolor{curcolor}{0 0 0}
\pscustom[linestyle=none,fillstyle=solid,fillcolor=curcolor]
{
\newpath
\moveto(848.01865278,283.1249135)
\curveto(848.01865278,280.22358017)(847.25065278,277.98358017)(845.71465278,276.4049135)
\curveto(844.19998611,274.82624683)(842.13065278,274.0369135)(839.50665278,274.0369135)
\curveto(837.88531944,274.0369135)(836.43465278,274.3889135)(835.15465278,275.0929135)
\curveto(833.89598611,275.7969135)(832.90398611,276.8209135)(832.17865278,278.1649135)
\curveto(831.45331944,279.53024683)(831.09065278,281.18358017)(831.09065278,283.1249135)
\curveto(831.09065278,286.02624683)(831.84798611,288.25558017)(833.36265278,289.8129135)
\curveto(834.87731944,291.37024683)(836.95731944,292.1489135)(839.60265278,292.1489135)
\curveto(841.24531944,292.1489135)(842.69598611,291.7969135)(843.95465278,291.0929135)
\curveto(845.21331944,290.3889135)(846.20531944,289.3649135)(846.93065278,288.0209135)
\curveto(847.65598611,286.6769135)(848.01865278,285.0449135)(848.01865278,283.1249135)
\closepath
\moveto(835.95465278,283.1249135)
\curveto(835.95465278,281.3969135)(836.23198611,280.0849135)(836.78665278,279.1889135)
\curveto(837.36265278,278.31424683)(838.29065278,277.8769135)(839.57065278,277.8769135)
\curveto(840.82931944,277.8769135)(841.73598611,278.31424683)(842.29065278,279.1889135)
\curveto(842.86665278,280.0849135)(843.15465278,281.3969135)(843.15465278,283.1249135)
\curveto(843.15465278,284.8529135)(842.86665278,286.14358017)(842.29065278,286.9969135)
\curveto(841.73598611,287.87158017)(840.81865278,288.3089135)(839.53865278,288.3089135)
\curveto(838.27998611,288.3089135)(837.36265278,287.87158017)(836.78665278,286.9969135)
\curveto(836.23198611,286.14358017)(835.95465278,284.8529135)(835.95465278,283.1249135)
\closepath
}
}
{
\newrgbcolor{curcolor}{0 0 0}
\pscustom[linestyle=none,fillstyle=solid,fillcolor=curcolor]
{
\newpath
\moveto(867.82663618,291.8289135)
\lineto(867.82663618,274.3569135)
\lineto(863.05863618,274.3569135)
\lineto(863.05863618,288.2449135)
\lineto(856.72263618,288.2449135)
\lineto(856.72263618,274.3569135)
\lineto(851.95463618,274.3569135)
\lineto(851.95463618,291.8289135)
\closepath
}
}
{
\newrgbcolor{curcolor}{0 0 0}
\pscustom[linestyle=none,fillstyle=solid,fillcolor=curcolor]
{
\newpath
\moveto(877.42664985,291.8289135)
\lineto(877.42664985,284.9169135)
\curveto(877.42664985,284.55424683)(877.40531652,284.10624683)(877.36264985,283.5729135)
\curveto(877.34131652,283.03958017)(877.30931652,282.49558017)(877.26664985,281.9409135)
\curveto(877.24531652,281.38624683)(877.21331652,280.8849135)(877.17064985,280.4369135)
\curveto(877.12798318,280.01024683)(877.09598318,279.72224683)(877.07464985,279.5729135)
\lineto(885.13864985,291.8289135)
\lineto(890.86664985,291.8289135)
\lineto(890.86664985,274.3569135)
\lineto(886.25864985,274.3569135)
\lineto(886.25864985,281.3329135)
\curveto(886.25864985,281.88758017)(886.27998318,282.5169135)(886.32264985,283.2209135)
\curveto(886.36531652,283.9249135)(886.40798318,284.57558017)(886.45064985,285.1729135)
\curveto(886.51464985,285.79158017)(886.55731652,286.2609135)(886.57864985,286.5809135)
\lineto(878.54664985,274.3569135)
\lineto(872.81864985,274.3569135)
\lineto(872.81864985,291.8289135)
\closepath
}
}
{
\newrgbcolor{curcolor}{0 0 0}
\pscustom[linestyle=none,fillstyle=solid,fillcolor=curcolor]
{
\newpath
\moveto(910.32262788,288.2449135)
\lineto(904.59462788,288.2449135)
\lineto(904.59462788,274.3569135)
\lineto(899.82662788,274.3569135)
\lineto(899.82662788,288.2449135)
\lineto(894.09862788,288.2449135)
\lineto(894.09862788,291.8289135)
\lineto(910.32262788,291.8289135)
\closepath
}
}
{
\newrgbcolor{curcolor}{0 0 0}
\pscustom[linestyle=none,fillstyle=solid,fillcolor=curcolor]
{
\newpath
\moveto(920.43460444,292.1489135)
\curveto(922.8452711,292.1489135)(924.75460444,291.45558017)(926.16260444,290.0689135)
\curveto(927.57060444,288.70358017)(928.27460444,286.75158017)(928.27460444,284.2129135)
\lineto(928.27460444,281.9089135)
\lineto(917.01060444,281.9089135)
\curveto(917.0532711,280.5649135)(917.44793777,279.5089135)(918.19460444,278.7409135)
\curveto(918.96260444,277.9729135)(920.01860444,277.5889135)(921.36260444,277.5889135)
\curveto(922.47193777,277.5889135)(923.4852711,277.69558017)(924.40260444,277.9089135)
\curveto(925.3412711,278.14358017)(926.3012711,278.49558017)(927.28260444,278.9649135)
\lineto(927.28260444,275.2849135)
\curveto(926.40793777,274.85824683)(925.5012711,274.5489135)(924.56260444,274.3569135)
\curveto(923.62393777,274.14358017)(922.48260444,274.0369135)(921.13860444,274.0369135)
\curveto(919.3892711,274.0369135)(917.84260444,274.3569135)(916.49860444,274.9969135)
\curveto(915.15460444,275.65824683)(914.09860444,276.63958017)(913.33060444,277.9409135)
\curveto(912.56260444,279.26358017)(912.17860444,280.93824683)(912.17860444,282.9649135)
\curveto(912.17860444,284.99158017)(912.51993777,286.68758017)(913.20260444,288.0529135)
\curveto(913.90660444,289.41824683)(914.8772711,290.44224683)(916.11460444,291.1249135)
\curveto(917.35193777,291.80758017)(918.79193777,292.1489135)(920.43460444,292.1489135)
\closepath
\moveto(920.46660444,288.7569135)
\curveto(919.52793777,288.7569135)(918.75993777,288.45824683)(918.16260444,287.8609135)
\curveto(917.5652711,287.26358017)(917.2132711,286.33558017)(917.10660444,285.0769135)
\lineto(923.79460444,285.0769135)
\curveto(923.7732711,286.12224683)(923.4852711,286.9969135)(922.93060444,287.7009135)
\curveto(922.3972711,288.4049135)(921.57593777,288.7569135)(920.46660444,288.7569135)
\closepath
}
}
{
\newrgbcolor{curcolor}{0 0 0}
\pscustom[linestyle=none,fillstyle=solid,fillcolor=curcolor]
{
\newpath
\moveto(947.79459174,274.3569135)
\lineto(943.02659174,274.3569135)
\lineto(943.02659174,288.2449135)
\lineto(938.64259174,288.2449135)
\curveto(938.36525841,284.83158017)(937.99192508,282.07958017)(937.52259174,279.9889135)
\curveto(937.07459174,277.91958017)(936.43459174,276.4049135)(935.60259174,275.4449135)
\curveto(934.79192508,274.50624683)(933.71459174,274.0369135)(932.37059174,274.0369135)
\curveto(931.26125841,274.0369135)(930.35459174,274.20758017)(929.65059174,274.5489135)
\lineto(929.65059174,278.3569135)
\curveto(930.14125841,278.14358017)(930.65325841,278.0369135)(931.18659174,278.0369135)
\curveto(931.57059174,278.0369135)(931.92259174,278.2289135)(932.24259174,278.6129135)
\curveto(932.56259174,278.9969135)(932.86125841,279.69024683)(933.13859174,280.6929135)
\curveto(933.43725841,281.69558017)(933.70392508,283.0929135)(933.93859174,284.8849135)
\curveto(934.17325841,286.69824683)(934.38659174,289.0129135)(934.57859174,291.8289135)
\lineto(947.79459174,291.8289135)
\closepath
}
}
{
\newrgbcolor{curcolor}{0 0 0}
\pscustom[linestyle=none,fillstyle=solid,fillcolor=curcolor]
{
\newpath
\moveto(957.55460639,285.0769135)
\lineto(960.91460639,285.0769135)
\curveto(963.60260639,285.0769135)(965.58660639,284.65024683)(966.86660639,283.7969135)
\curveto(968.16793972,282.94358017)(968.81860639,281.6529135)(968.81860639,279.9249135)
\curveto(968.81860639,278.21824683)(968.22127306,276.86358017)(967.02660639,275.8609135)
\curveto(965.83193972,274.85824683)(963.85860639,274.3569135)(961.10660639,274.3569135)
\lineto(952.78660639,274.3569135)
\lineto(952.78660639,291.8289135)
\lineto(957.55460639,291.8289135)
\closepath
\moveto(964.05060639,279.8609135)
\curveto(964.05060639,281.1409135)(962.97327306,281.7809135)(960.81860639,281.7809135)
\lineto(957.55460639,281.7809135)
\lineto(957.55460639,277.6529135)
\lineto(960.88260639,277.6529135)
\curveto(961.79993972,277.6529135)(962.55727306,277.8129135)(963.15460639,278.1329135)
\curveto(963.75193972,278.47424683)(964.05060639,279.05024683)(964.05060639,279.8609135)
\closepath
}
}
{
\newrgbcolor{curcolor}{0 0 0}
\pscustom[linestyle=none,fillstyle=solid,fillcolor=curcolor]
{
\newpath
\moveto(979.50661665,283.1249135)
\curveto(979.50661665,280.54358017)(979.12261665,278.05824683)(978.35461665,275.6689135)
\curveto(977.60794998,273.3009135)(976.42394998,271.17824683)(974.80261665,269.3009135)
\lineto(970.93061665,269.3009135)
\curveto(972.35994998,271.26358017)(973.44794998,273.43958017)(974.19461665,275.8289135)
\curveto(974.96261665,278.23958017)(975.34661665,280.68224683)(975.34661665,283.1569135)
\curveto(975.34661665,285.69558017)(974.96261665,288.17024683)(974.19461665,290.5809135)
\curveto(973.44794998,292.99158017)(972.34928331,295.19958017)(970.89861665,297.2049135)
\lineto(974.80261665,297.2049135)
\curveto(976.42394998,295.26358017)(977.60794998,293.0769135)(978.35461665,290.6449135)
\curveto(979.12261665,288.23424683)(979.50661665,285.72758017)(979.50661665,283.1249135)
\closepath
}
}
\end{pspicture}
}
		\caption{Схема клиент-серверного приложения}
		\label{g6_ink1}
	\end{center}
\end{figure}

Первоочередная задача — создание инструкций передачи данных
апплета \emph{GeoGebra} как на сервер, так и из него. В базовом API апплета
предусмотрено несколько вариантов сохранения/загрузки данных,
в том числе с помощью \emph{json}-файлов и \emph{XML}-файлов.
Опытным путем был выбран вариант работы с \emph{XML}-файлами
как основной и с \emph{json}-файлами как вспомогательный. Соответственно была написана
функция передачи/приема этих файлов с помощью сокетов [Исходный код \ref{sc12_s_skt_xml}] [Исходный код \ref{sc32_k_js_xml}].

Рассмотрим работу инструкций по этапам на примере:

1. Пользователь, владеющий правами на редактирование доски,
добавляет точку в координатную плоскость апплета.

2. В API апплета есть обработчики событий, которые реагируют
на изменение состояния апплета, в нашем случае была добавлена точка,
и вызывают функцию сохранения/передачи.

3. Функция сохранения/передачи копирует \emph{XML}-код
апплета и, если есть возможность, редактирует его.
После этого через сокет отправляет этот код на сервер вместе с
номером комнаты.

4. Сервер принимает данные из сокета, т.е. \emph{XML}-код и номер
комнаты. Далее по номеру комнаты ищет всех ее участников и отправляет
им через сокет полученный \emph{XML}-код.

5. Участники через сокет получают \emph{XML}-код от сервера.
Вызывается функция загрузки \emph{XML}-кода в апплет.

Причем перед отправкой любых данных через веб-сокеты, сервер проверяет
id пользователя, его принадлежность к комнате, и наличие у него прав
владельца комнаты (для комнаты с ограниченными правами). Тем самым
отсеивая любую возможность взаимодействия участников одной комнаты с другой.

Таким образом происходит синхронизация \emph{XML}-кода апплета
пользователя с правами на редактирование доски с другими участниками
комнаты.

Стоит отметить, что был проведен анализ, какие элементы в
\emph{XML}-коде постоянны, а какие изменяются. Исходя
из этого, для уменьшения нагрузки на сеть, \emph{XML}-код
перед отправкой обрезается и в нем остаются только изменяющиеся
элементы.

Координаты плоскости апплета так же входят в \emph{XML}-код,
соответственно функция трансляции координат отключается путем той самой обрезки
части \emph{XML}-кода, если данная функция включена, то эта часть с 
координатами остается и передается вместе с остальным \emph{XML}-кодом.

Следующая задача - инструкции для передачи команд управления правами
и работы чата [Исходный код \ref{sc31_k_js_permission}]. Это достаточно тривиальная задача, т.к. в ней требуется
передавать через веб-сокеты малый объем информации.

Инструкции для передачи команд управления правами работают просто.
Рассмотрим пример. Пользователь запрашивает права у владельца, нажимая
соответствующую кнопку. Обработчик клика этой кнопки вызывает
функцию, которая через сокет передает номер комнаты, специальный
id участника и команду для запроса прав. Сервер принимает данные
сокета, по номеру комнаты ищет владельца и отправляет через сокет
id участника, запросившего права. Владелец получает уведомление
и принимает решение: передать права или нет. Это решение (true или false)
отправляется на сервер через сокет вместе с номером комнаты и id участника.
Сервер рассматривает решение владельца, и в соответствии с ним
либо отдает права участнику, либо нет, отправляя ему соответствующее
уведомление через сокет. В свободной системе прав все происходит
намного проще, но общая схема остается такой же.

В комнате с ограниченной системой прав, если участник отказывается
от права редактирования доски (повторно нажимая кнопку, которой он запрашивал
права), то права на редактирование немедленно возвращаются владельцу
комнаты.

В комнате со свободной системой прав, если участник отказывается от прав
на редактирование доски, то права никому не передаются, пока их не
запросит другой пользователь.

Инструкции работы чата оказались самой простой задачей [Исходный код \ref{sc34_k_js_messages}], т.к. в документации
\emph{socket.io} много информации об этом.
Снова рассмотрим пример. В случае чата нет никакой разницы,
какая система прав используется. Участник пишет сообщение, нажимает
кнопку, на которой размещен обработчик клика. Обработчик вызывает
функцию, которая отправляет через сокет сообщение пользователя, 
номер комнаты и имя пользователя на сервер. Сервер принимает все данные,
по номеру комнаты ищет всех участников и через сокет отправляет
это сообщение, соединенное с именем пользователя, всем участникам комнаты.
Участники принимают это сообщение, оно оборачивается в элемент списка
и помещается в контейнер чата.

Аналогично чату работает и голосовой канал [Исходный код \ref{sc37_k_js_voice}],
различие заключается только в том, что передача осуществляется чанками и непрерывно.

Так же, через систему сокетов, связана и трансляция курсора мыши, только передаются уже
координаты. Однако, в чистом виде, как они есть, передавать координаты
нельзя, т.к. в этом случае существует привязка к разрешению экрана пользователя.
Поэтому используются методы преобразования координат. Когда другой пользователь получает такие данные,
то происходит обратное преобразование, но с учетом его разрешения.

Помимо этого, для оптимизации, каждый сокет был связан с таймером,
который ограничивает количество тиков передачи данных, тем
самым уменьшая нагрузку как на сеть, так и на ПК пользователя.

	
	\hfill \break

	\section{\centering База данных для хранения результатов работы с комнатами}
	
	Немаловажной задачей стала возможность записи результатов работы с комнатами.
Это связано с тем, чтобы после выхода всех участников из комнаты была
возможность вернуться в эту комнату и не потерять всю историю действий.
Было рассмотрено несколько библиотечных баз данных, взаимодействующих
с платформой \emph{Node.js}, но все они оказались слишком тяжелыми
для такой простой задачи. Пример: БД \emph{MySQL} требует отдельный
сервер для работы и большое количество оперативной памяти, что
является недопустимым для проекта, т.к. он должен
работать на маломощном сервере. Поэтому решено было написать собственную
базу данных.

Структура базы данных представляла систему, состоящую из двух блоков.
Первый блок [Исходный код \ref{sc4_s_database}] должен был выполнять функцию хранения данных на локальном
диске с помощью \emph{json}-файлов. Второй блок [Исходный код \ref{sc3_s_dbRAM}] хранил данные в
оперативной памяти. Взаимодействуют они следующим образом: пока
в комнате есть участники — данные комнаты хранятся в оперативной
памяти для быстрого доступа к ним. Как только все участники покидают
комнату — данные записываются на локальный диск в \emph{json}-файл и
освобождаются из оперативной памяти. Соответственно, если в пустую комнату
заходит хотя бы один участник — данные этой комнаты считываются с
\emph{json}-файла и помещаются в оперативную память для последующего использования.

	
	\hfill \break

	\section{\centering Тестирование и отладка приложения}
	
	На момент написания дипломной работы произведена отладка и тестирование проекта
«\emph{Geometry Room}». Были выявлены следующие проблемы в порядке важности, требующие решения:
\begin{enumerate}
    \item Мерцание апплета «\emph{GeoGebra}» при его обновлении,
    что происходит довольно часто (при любых масштабных изменениях данных апплета).
    Данная проблема крайне заметна и мешает долгому использованию приложения.
    Проявляется она по причине того, что API апплетов не предназначено для постоянного
    редактирования данных и требует постоянного обновления окна апплета.

    Решение данной проблемы найдено, но еще не реализовано. Оно заключается в
    создании механизма двойной буферизации. Создаются два апплета: один
    отображается у пользователя, другой принимает данные и подготавливается к отображению.
    Далее они меняются местами. Смена апплетов будет осуществляться средствами
    \emph{JavaScript} и \emph{CSS}, в таком варианте мерцание
    на специально подготовленном для тестирования апплете не проявляется.

    \item Потеря некоторых функций, таких как рисование пером на апплете и вставка изображений.
    Это связано с использованием XML-кода, который, в рамках данного случая, не поддерживает эти
    операции.
    
    Решение проблемы заключается в большем использовании возможностей JSON-файлов при обмене данных между
    апплетами. Это требует многих оптимизаций, так как файлы данного формата имеют большой вес.

    \item Масштабируемость веб-интерфейса работает не на всех разрешениях и не на всех форматах экрана,
    а только на тех, которые были доступны для продолжительного тестирования: разрешение не ниже FHD c 
    форматом экрана 16:9.

    Решение данной проблемы в изучении и тестировании средств \emph{CSS} для адаптивной верстки.

    \item Проблема в работе голосового канала: замечены случаи потери пакетов данных.

    Решение на данный момент пока не найдено.
\end{enumerate}


	\newpage
	\section{\centering Дерево проекта}
	
	\begin{figure}[H]
	\begin{center}
		\scalebox{1}{%LaTeX with PSTricks extensions
%%Creator: Inkscape 1.2 (dc2aedaf03, 2022-05-15)
%%Please note this file requires PSTricks extensions
\psset{xunit=.5pt,yunit=.5pt,runit=.5pt}
\begin{pspicture}(566.92913386,1228.34645669)
{
\newrgbcolor{curcolor}{0.50196081 0.50196081 0.50196081}
\pscustom[linestyle=none,fillstyle=solid,fillcolor=curcolor]
{
\newpath
\moveto(16.29855659,1214.12744207)
\lineto(26.43048324,1214.12744207)
\lineto(26.43048324,20.0305317)
\lineto(16.29855659,20.0305317)
\closepath
}
}
{
\newrgbcolor{curcolor}{0.50196081 0.50196081 0.50196081}
\pscustom[linestyle=none,fillstyle=solid,fillcolor=curcolor]
{
\newpath
\moveto(59.92917279,1188.8609374)
\lineto(70.06109944,1188.8609374)
\lineto(70.06109944,20.83928386)
\lineto(59.92917279,20.83928386)
\closepath
}
}
{
\newrgbcolor{curcolor}{0.50196081 0.50196081 0.50196081}
\pscustom[linestyle=none,fillstyle=solid,fillcolor=curcolor]
{
\newpath
\moveto(104.09951037,1158.7358706)
\lineto(114.23143702,1158.7358706)
\lineto(114.23143702,837.65183817)
\lineto(104.09951037,837.65183817)
\closepath
}
}
{
\newrgbcolor{curcolor}{0.50196081 0.50196081 0.50196081}
\pscustom[linestyle=none,fillstyle=solid,fillcolor=curcolor]
{
\newpath
\moveto(150.03414124,1126.89172459)
\lineto(160.16606789,1126.89172459)
\lineto(160.16606789,952.94801727)
\lineto(150.03414124,952.94801727)
\closepath
}
}
{
\newrgbcolor{curcolor}{0.50196081 0.50196081 0.50196081}
\pscustom[linestyle=none,fillstyle=solid,fillcolor=curcolor]
{
\newpath
\moveto(30.48324705,1203.99553975)
\lineto(69.99775684,1203.99553975)
\lineto(69.99775684,1193.8636131)
\lineto(30.48324705,1193.8636131)
\closepath
}
}
{
\newrgbcolor{curcolor}{0.50196081 0.50196081 0.50196081}
\pscustom[linestyle=none,fillstyle=solid,fillcolor=curcolor]
{
\newpath
\moveto(74.7647311,1173.35030383)
\lineto(114.27924089,1173.35030383)
\lineto(114.27924089,1163.21837718)
\lineto(74.7647311,1163.21837718)
\closepath
}
}
{
\newrgbcolor{curcolor}{0.50196081 0.50196081 0.50196081}
\pscustom[linestyle=none,fillstyle=solid,fillcolor=curcolor]
{
\newpath
\moveto(104.86836591,803.5533034)
\lineto(115.00029256,803.5533034)
\lineto(115.00029256,258.55466828)
\lineto(104.86836591,258.55466828)
\closepath
}
}
{
\newrgbcolor{curcolor}{0.50196081 0.50196081 0.50196081}
\pscustom[linestyle=none,fillstyle=solid,fillcolor=curcolor]
{
\newpath
\moveto(75.53358664,818.16774383)
\lineto(115.04809642,818.16774383)
\lineto(115.04809642,808.03581718)
\lineto(75.53358664,808.03581718)
\closepath
}
}
{
\newrgbcolor{curcolor}{0.50196081 0.50196081 0.50196081}
\pscustom[linestyle=none,fillstyle=solid,fillcolor=curcolor]
{
\newpath
\moveto(120.03755104,1141.68418103)
\lineto(159.55206083,1141.68418103)
\lineto(159.55206083,1131.55225438)
\lineto(120.03755104,1131.55225438)
\closepath
}
}
{
\newrgbcolor{curcolor}{0.50196081 0.50196081 0.50196081}
\pscustom[linestyle=none,fillstyle=solid,fillcolor=curcolor]
{
\newpath
\moveto(119.9256909,932.87977324)
\lineto(159.44020069,932.87977324)
\lineto(159.44020069,922.74784659)
\lineto(119.9256909,922.74784659)
\closepath
}
}
{
\newrgbcolor{curcolor}{0.50196081 0.50196081 0.50196081}
\pscustom[linestyle=none,fillstyle=solid,fillcolor=curcolor]
{
\newpath
\moveto(166.11224317,1112.46635137)
\lineto(205.62675296,1112.46635137)
\lineto(205.62675296,1102.33442472)
\lineto(166.11224317,1102.33442472)
\closepath
}
}
{
\newrgbcolor{curcolor}{0.50196081 0.50196081 0.50196081}
\pscustom[linestyle=none,fillstyle=solid,fillcolor=curcolor]
{
\newpath
\moveto(166.11224317,1091.14932083)
\lineto(205.62675296,1091.14932083)
\lineto(205.62675296,1081.01739418)
\lineto(166.11224317,1081.01739418)
\closepath
}
}
{
\newrgbcolor{curcolor}{0.50196081 0.50196081 0.50196081}
\pscustom[linestyle=none,fillstyle=solid,fillcolor=curcolor]
{
\newpath
\moveto(166.11224317,1069.8322975)
\lineto(205.62675296,1069.8322975)
\lineto(205.62675296,1059.70037085)
\lineto(166.11224317,1059.70037085)
\closepath
}
}
{
\newrgbcolor{curcolor}{0.50196081 0.50196081 0.50196081}
\pscustom[linestyle=none,fillstyle=solid,fillcolor=curcolor]
{
\newpath
\moveto(166.11224317,1048.51527417)
\lineto(205.62675296,1048.51527417)
\lineto(205.62675296,1038.38334752)
\lineto(166.11224317,1038.38334752)
\closepath
}
}
{
\newrgbcolor{curcolor}{0.50196081 0.50196081 0.50196081}
\pscustom[linestyle=none,fillstyle=solid,fillcolor=curcolor]
{
\newpath
\moveto(119.9256909,911.64550781)
\lineto(159.44020069,911.64550781)
\lineto(159.44020069,901.51358116)
\lineto(119.9256909,901.51358116)
\closepath
}
}
{
\newrgbcolor{curcolor}{0.50196081 0.50196081 0.50196081}
\pscustom[linestyle=none,fillstyle=solid,fillcolor=curcolor]
{
\newpath
\moveto(119.9256909,890.41121355)
\lineto(159.44020069,890.41121355)
\lineto(159.44020069,880.2792869)
\lineto(119.9256909,880.2792869)
\closepath
}
}
{
\newrgbcolor{curcolor}{0.50196081 0.50196081 0.50196081}
\pscustom[linestyle=none,fillstyle=solid,fillcolor=curcolor]
{
\newpath
\moveto(119.9256909,869.17694813)
\lineto(159.44020069,869.17694813)
\lineto(159.44020069,859.04502148)
\lineto(119.9256909,859.04502148)
\closepath
}
}
{
\newrgbcolor{curcolor}{0.50196081 0.50196081 0.50196081}
\pscustom[linestyle=none,fillstyle=solid,fillcolor=curcolor]
{
\newpath
\moveto(119.9256909,847.9426827)
\lineto(159.44020069,847.9426827)
\lineto(159.44020069,837.81075605)
\lineto(119.9256909,837.81075605)
\closepath
}
}
{
\newrgbcolor{curcolor}{0.50196081 0.50196081 0.50196081}
\pscustom[linestyle=none,fillstyle=solid,fillcolor=curcolor]
{
\newpath
\moveto(166.11224317,1027.19825084)
\lineto(205.62675296,1027.19825084)
\lineto(205.62675296,1017.06632419)
\lineto(166.11224317,1017.06632419)
\closepath
}
}
{
\newrgbcolor{curcolor}{0.50196081 0.50196081 0.50196081}
\pscustom[linestyle=none,fillstyle=solid,fillcolor=curcolor]
{
\newpath
\moveto(166.11224317,1005.88121309)
\lineto(205.62675296,1005.88121309)
\lineto(205.62675296,995.74928644)
\lineto(166.11224317,995.74928644)
\closepath
}
}
{
\newrgbcolor{curcolor}{0.50196081 0.50196081 0.50196081}
\pscustom[linestyle=none,fillstyle=solid,fillcolor=curcolor]
{
\newpath
\moveto(166.11224317,984.56418976)
\lineto(205.62675296,984.56418976)
\lineto(205.62675296,974.43226311)
\lineto(166.11224317,974.43226311)
\closepath
}
}
{
\newrgbcolor{curcolor}{0.50196081 0.50196081 0.50196081}
\pscustom[linestyle=none,fillstyle=solid,fillcolor=curcolor]
{
\newpath
\moveto(166.11224317,963.24716643)
\lineto(205.62675296,963.24716643)
\lineto(205.62675296,953.11523978)
\lineto(166.11224317,953.11523978)
\closepath
}
}
{
\newrgbcolor{curcolor}{0 0 0}
\pscustom[linestyle=none,fillstyle=solid,fillcolor=curcolor]
{
\newpath
\moveto(80.4951291,1204.48248341)
\curveto(81.75202024,1204.48248341)(82.73513312,1203.98470474)(83.44446772,1202.9891474)
\lineto(83.51913452,1202.9891474)
\lineto(83.74313492,1204.29581641)
\lineto(86.09513914,1204.29581641)
\lineto(86.09513914,1194.08513144)
\curveto(86.09513914,1192.62912883)(85.66580504,1191.52157129)(84.80713683,1190.76245882)
\curveto(83.94846862,1190.00334635)(82.67913302,1189.62379011)(80.99913,1189.62379011)
\curveto(80.27735093,1189.62379011)(79.60534973,1189.66734574)(78.98312639,1189.75445701)
\curveto(78.37334752,1189.84156828)(77.77601312,1189.99712411)(77.19112318,1190.22112451)
\lineto(77.19112318,1192.44246183)
\curveto(78.43556985,1191.91979423)(79.76090556,1191.65846042)(81.16713031,1191.65846042)
\curveto(82.59824398,1191.65846042)(83.31380082,1192.43001736)(83.31380082,1193.97313124)
\lineto(83.31380082,1194.17846494)
\curveto(83.31380082,1194.37757641)(83.32002305,1194.58913234)(83.33246752,1194.81313275)
\curveto(83.34491199,1195.04957761)(83.36357869,1195.25491131)(83.38846762,1195.42913385)
\lineto(83.31380082,1195.42913385)
\curveto(82.96535575,1194.89402178)(82.54846612,1194.50824331)(82.06313191,1194.27179844)
\curveto(81.57779771,1194.03535357)(81.03024117,1193.91713114)(80.4204623,1193.91713114)
\curveto(79.21334903,1193.91713114)(78.26756955,1194.37757641)(77.58312388,1195.29846695)
\curveto(76.91112268,1196.23180195)(76.57512207,1197.5260265)(76.57512207,1199.18114057)
\curveto(76.57512207,1200.84869912)(76.92356714,1202.1491459)(77.62045728,1203.0824809)
\curveto(78.31734742,1204.01581591)(79.27557136,1204.48248341)(80.4951291,1204.48248341)
\closepath
\moveto(81.37246401,1202.2238127)
\curveto(80.065795,1202.2238127)(79.41246049,1201.19714419)(79.41246049,1199.14380717)
\curveto(79.41246049,1197.11535909)(80.07823947,1196.10113505)(81.40979741,1196.10113505)
\curveto(82.11913201,1196.10113505)(82.64179962,1196.30024652)(82.97780022,1196.69846946)
\curveto(83.32624529,1197.10913686)(83.50046782,1197.81847147)(83.50046782,1198.82647327)
\lineto(83.50046782,1199.16247387)
\curveto(83.50046782,1200.25758695)(83.33246752,1201.04158835)(82.99646692,1201.51447809)
\curveto(82.66046632,1201.98736783)(82.11913201,1202.2238127)(81.37246401,1202.2238127)
\closepath
}
}
{
\newrgbcolor{curcolor}{0 0 0}
\pscustom[linestyle=none,fillstyle=solid,fillcolor=curcolor]
{
\newpath
\moveto(93.20714103,1204.48248341)
\curveto(94.61336578,1204.48248341)(95.72714555,1204.07803824)(96.54848036,1203.2691479)
\curveto(97.36981516,1202.47270203)(97.78048257,1201.33403332)(97.78048257,1199.85314178)
\lineto(97.78048257,1198.50913937)
\lineto(91.20980412,1198.50913937)
\curveto(91.23469305,1197.72513797)(91.46491569,1197.10913686)(91.90047202,1196.66113606)
\curveto(92.34847283,1196.21313525)(92.96447393,1195.98913485)(93.74847534,1195.98913485)
\curveto(94.39558761,1195.98913485)(94.98669978,1196.05135719)(95.52181185,1196.17580185)
\curveto(96.06936839,1196.31269099)(96.62936939,1196.51802469)(97.20181486,1196.79180296)
\lineto(97.20181486,1194.64513244)
\curveto(96.69159172,1194.39624311)(96.16270189,1194.21579834)(95.61514535,1194.10379814)
\curveto(95.06758881,1193.97935347)(94.40180984,1193.91713114)(93.61780844,1193.91713114)
\curveto(92.59736216,1193.91713114)(91.69513832,1194.10379814)(90.91113692,1194.47713214)
\curveto(90.12713551,1194.86291061)(89.51113441,1195.43535608)(89.06313361,1196.19446855)
\curveto(88.6151328,1196.96602549)(88.3911324,1197.94291613)(88.3911324,1199.12514047)
\curveto(88.3911324,1200.30736482)(88.59024387,1201.29669992)(88.9884668,1202.09314579)
\curveto(89.39913421,1202.88959167)(89.96535744,1203.48692607)(90.68713652,1203.88514901)
\curveto(91.40891559,1204.28337194)(92.24891709,1204.48248341)(93.20714103,1204.48248341)
\closepath
\moveto(93.22580773,1202.5038132)
\curveto(92.6782512,1202.5038132)(92.23025039,1202.32959066)(91.88180532,1201.98114559)
\curveto(91.53336026,1201.63270053)(91.32802655,1201.09136622)(91.26580422,1200.35714268)
\lineto(95.16714455,1200.35714268)
\curveto(95.15470008,1200.96692155)(94.98669978,1201.47714469)(94.66314364,1201.88781209)
\curveto(94.35203197,1202.2984795)(93.87292,1202.5038132)(93.22580773,1202.5038132)
\closepath
}
}
{
\newrgbcolor{curcolor}{0 0 0}
\pscustom[linestyle=none,fillstyle=solid,fillcolor=curcolor]
{
\newpath
\moveto(109.29782862,1199.21847398)
\curveto(109.29782862,1197.5260265)(108.84982781,1196.21935749)(107.95382621,1195.29846695)
\curveto(107.07026907,1194.37757641)(105.86315579,1193.91713114)(104.33248638,1193.91713114)
\curveto(103.38670691,1193.91713114)(102.54048317,1194.12246484)(101.79381516,1194.53313224)
\curveto(101.05959163,1194.94379965)(100.48092392,1195.54113405)(100.05781205,1196.32513546)
\curveto(99.63470018,1197.12158133)(99.42314425,1198.0860275)(99.42314425,1199.21847398)
\curveto(99.42314425,1200.91092145)(99.86492282,1202.21136823)(100.74847996,1203.1198143)
\curveto(101.6320371,1204.02826037)(102.8453726,1204.48248341)(104.38848648,1204.48248341)
\curveto(105.34671042,1204.48248341)(106.19293416,1204.27714971)(106.9271577,1203.86648231)
\curveto(107.66138124,1203.4558149)(108.24004894,1202.8584805)(108.66316081,1202.07447909)
\curveto(109.08627268,1201.29047769)(109.29782862,1200.33847598)(109.29782862,1199.21847398)
\closepath
\moveto(102.26048267,1199.21847398)
\curveto(102.26048267,1198.21047217)(102.42226073,1197.44513746)(102.74581687,1196.92246986)
\curveto(103.08181747,1196.41224672)(103.62315178,1196.15713515)(104.36981978,1196.15713515)
\curveto(105.10404332,1196.15713515)(105.63293316,1196.41224672)(105.95648929,1196.92246986)
\curveto(106.29248989,1197.44513746)(106.4604902,1198.21047217)(106.4604902,1199.21847398)
\curveto(106.4604902,1200.22647578)(106.29248989,1200.97936602)(105.95648929,1201.47714469)
\curveto(105.63293316,1201.98736783)(105.09782109,1202.2424794)(104.35115308,1202.2424794)
\curveto(103.61692954,1202.2424794)(103.08181747,1201.98736783)(102.74581687,1201.47714469)
\curveto(102.42226073,1200.97936602)(102.26048267,1200.22647578)(102.26048267,1199.21847398)
\closepath
}
}
{
\newrgbcolor{curcolor}{0 0 0}
\pscustom[linestyle=none,fillstyle=solid,fillcolor=curcolor]
{
\newpath
\moveto(123.55917986,1204.48248341)
\curveto(124.71651527,1204.48248341)(125.58762794,1204.18381621)(126.17251788,1203.58648181)
\curveto(126.76985228,1203.00159187)(127.06851948,1202.05581239)(127.06851948,1200.74914339)
\lineto(127.06851948,1194.10379814)
\lineto(124.28718116,1194.10379814)
\lineto(124.28718116,1200.05847548)
\curveto(124.28718116,1201.52692256)(123.77695803,1202.2611461)(122.75651175,1202.2611461)
\curveto(122.02228821,1202.2611461)(121.49962061,1201.99981229)(121.18850894,1201.47714469)
\curveto(120.87739727,1200.95447709)(120.72184144,1200.20158685)(120.72184144,1199.21847398)
\lineto(120.72184144,1194.10379814)
\lineto(117.94050312,1194.10379814)
\lineto(117.94050312,1200.05847548)
\curveto(117.94050312,1201.52692256)(117.43027998,1202.2611461)(116.40983371,1202.2611461)
\curveto(115.63827677,1202.2611461)(115.1031647,1201.96870113)(114.8044975,1201.38381119)
\curveto(114.51827476,1200.81136572)(114.3751634,1199.98380868)(114.3751634,1198.90114007)
\lineto(114.3751634,1194.10379814)
\lineto(111.59382508,1194.10379814)
\lineto(111.59382508,1204.29581641)
\lineto(113.72182889,1204.29581641)
\lineto(114.09516289,1202.9891474)
\lineto(114.24449649,1202.9891474)
\curveto(114.55560816,1203.511815)(114.97872003,1203.89137124)(115.5138321,1204.12781611)
\curveto(116.06138864,1204.36426098)(116.62761188,1204.48248341)(117.21250182,1204.48248341)
\curveto(117.95916982,1204.48248341)(118.58761539,1204.35803874)(119.09783853,1204.10914941)
\curveto(119.62050613,1203.87270454)(120.0249513,1203.49937054)(120.31117404,1202.9891474)
\lineto(120.55384114,1202.9891474)
\curveto(120.86495281,1203.511815)(121.29428691,1203.89137124)(121.84184345,1204.12781611)
\curveto(122.40184445,1204.36426098)(122.97428992,1204.48248341)(123.55917986,1204.48248341)
\closepath
}
}
{
\newrgbcolor{curcolor}{0 0 0}
\pscustom[linestyle=none,fillstyle=solid,fillcolor=curcolor]
{
\newpath
\moveto(134.12453611,1204.48248341)
\curveto(135.53076085,1204.48248341)(136.64454063,1204.07803824)(137.46587543,1203.2691479)
\curveto(138.28721024,1202.47270203)(138.69787764,1201.33403332)(138.69787764,1199.85314178)
\lineto(138.69787764,1198.50913937)
\lineto(132.1271992,1198.50913937)
\curveto(132.15208813,1197.72513797)(132.38231076,1197.10913686)(132.8178671,1196.66113606)
\curveto(133.2658679,1196.21313525)(133.88186901,1195.98913485)(134.66587041,1195.98913485)
\curveto(135.31298268,1195.98913485)(135.90409486,1196.05135719)(136.43920693,1196.17580185)
\curveto(136.98676346,1196.31269099)(137.54676447,1196.51802469)(138.11920994,1196.79180296)
\lineto(138.11920994,1194.64513244)
\curveto(137.6089868,1194.39624311)(137.08009696,1194.21579834)(136.53254043,1194.10379814)
\curveto(135.98498389,1193.97935347)(135.31920492,1193.91713114)(134.53520351,1193.91713114)
\curveto(133.51475724,1193.91713114)(132.6125334,1194.10379814)(131.82853199,1194.47713214)
\curveto(131.04453059,1194.86291061)(130.42852948,1195.43535608)(129.98052868,1196.19446855)
\curveto(129.53252788,1196.96602549)(129.30852748,1197.94291613)(129.30852748,1199.12514047)
\curveto(129.30852748,1200.30736482)(129.50763895,1201.29669992)(129.90586188,1202.09314579)
\curveto(130.31652928,1202.88959167)(130.88275252,1203.48692607)(131.60453159,1203.88514901)
\curveto(132.32631066,1204.28337194)(133.16631217,1204.48248341)(134.12453611,1204.48248341)
\closepath
\moveto(134.14320281,1202.5038132)
\curveto(133.59564627,1202.5038132)(133.14764547,1202.32959066)(132.7992004,1201.98114559)
\curveto(132.45075533,1201.63270053)(132.24542163,1201.09136622)(132.1831993,1200.35714268)
\lineto(136.08453962,1200.35714268)
\curveto(136.07209516,1200.96692155)(135.90409486,1201.47714469)(135.58053872,1201.88781209)
\curveto(135.26942705,1202.2984795)(134.79031508,1202.5038132)(134.14320281,1202.5038132)
\closepath
}
}
{
\newrgbcolor{curcolor}{0 0 0}
\pscustom[linestyle=none,fillstyle=solid,fillcolor=curcolor]
{
\newpath
\moveto(145.24988241,1196.13846845)
\curveto(145.56099408,1196.13846845)(145.85966128,1196.16335739)(146.14588402,1196.21313525)
\curveto(146.43210675,1196.27535759)(146.71832949,1196.35624662)(147.00455222,1196.45580236)
\lineto(147.00455222,1194.38379864)
\curveto(146.70588502,1194.24690951)(146.33255102,1194.13490931)(145.88455022,1194.04779804)
\curveto(145.44899388,1193.96068677)(144.96988191,1193.91713114)(144.44721431,1193.91713114)
\curveto(143.83743543,1193.91713114)(143.2898789,1194.01668687)(142.80454469,1194.21579834)
\curveto(142.33165496,1194.41490981)(141.95209872,1194.75713264)(141.66587599,1195.24246685)
\curveto(141.39209772,1195.72780105)(141.25520858,1196.41224672)(141.25520858,1197.29580386)
\lineto(141.25520858,1202.205146)
\lineto(139.92987287,1202.205146)
\lineto(139.92987287,1203.3811481)
\lineto(141.46054229,1204.31448311)
\lineto(142.26321039,1206.46115362)
\lineto(144.0365469,1206.46115362)
\lineto(144.0365469,1204.29581641)
\lineto(146.89255202,1204.29581641)
\lineto(146.89255202,1202.205146)
\lineto(144.0365469,1202.205146)
\lineto(144.0365469,1197.29580386)
\curveto(144.0365469,1196.91002539)(144.1485471,1196.61758042)(144.37254751,1196.41846896)
\curveto(144.59654791,1196.23180195)(144.88899288,1196.13846845)(145.24988241,1196.13846845)
\closepath
}
}
{
\newrgbcolor{curcolor}{0 0 0}
\pscustom[linestyle=none,fillstyle=solid,fillcolor=curcolor]
{
\newpath
\moveto(154.7512337,1204.48248341)
\curveto(154.88812283,1204.48248341)(155.0499009,1204.47626118)(155.2365679,1204.46381671)
\curveto(155.4232349,1204.45137224)(155.5725685,1204.43270554)(155.6845687,1204.40781661)
\lineto(155.479235,1201.79447859)
\curveto(155.37967927,1201.81936753)(155.24901237,1201.83803423)(155.0872343,1201.85047869)
\curveto(154.92545623,1201.87536763)(154.78234486,1201.88781209)(154.65790019,1201.88781209)
\curveto(154.18501046,1201.88781209)(153.73078742,1201.80070083)(153.29523109,1201.62647829)
\curveto(152.85967475,1201.46470022)(152.50500745,1201.19714419)(152.23122918,1200.82381019)
\curveto(151.96989538,1200.45047618)(151.83922848,1199.94025305)(151.83922848,1199.29314078)
\lineto(151.83922848,1194.10379814)
\lineto(149.05789016,1194.10379814)
\lineto(149.05789016,1204.29581641)
\lineto(151.16722727,1204.29581641)
\lineto(151.57789467,1202.57848)
\lineto(151.70856157,1202.57848)
\curveto(152.00722878,1203.1011476)(152.41789618,1203.5491484)(152.94056378,1203.92248241)
\curveto(153.46323139,1204.29581641)(154.06678802,1204.48248341)(154.7512337,1204.48248341)
\closepath
}
}
{
\newrgbcolor{curcolor}{0 0 0}
\pscustom[linestyle=none,fillstyle=solid,fillcolor=curcolor]
{
\newpath
\moveto(156.07658074,1204.29581641)
\lineto(159.11925286,1204.29581641)
\lineto(161.04192297,1198.56513947)
\curveto(161.14147871,1198.27891674)(161.21614551,1197.992694)(161.26592337,1197.70647126)
\curveto(161.31570124,1197.42024853)(161.35303464,1197.11535909)(161.37792357,1196.79180296)
\lineto(161.43392367,1196.79180296)
\curveto(161.47125708,1197.11535909)(161.52103494,1197.42024853)(161.58325728,1197.70647126)
\curveto(161.64547961,1197.992694)(161.72636864,1198.27891674)(161.82592438,1198.56513947)
\lineto(163.71126109,1204.29581641)
\lineto(166.69793311,1204.29581641)
\lineto(162.38592538,1192.79712913)
\curveto(161.98770245,1191.73934946)(161.42147921,1190.94912582)(160.68725567,1190.42645821)
\curveto(159.95303213,1189.89134614)(159.10058616,1189.62379011)(158.12991775,1189.62379011)
\curveto(157.80636162,1189.62379011)(157.53258335,1189.64245681)(157.30858295,1189.67979021)
\curveto(157.08458255,1189.70467914)(156.88547108,1189.73579031)(156.71124854,1189.77312371)
\lineto(156.71124854,1191.97579433)
\curveto(156.83569321,1191.95090539)(156.99747128,1191.92601646)(157.19658275,1191.90112753)
\curveto(157.39569421,1191.87623859)(157.60102792,1191.86379412)(157.81258385,1191.86379412)
\curveto(158.39747379,1191.86379412)(158.85791906,1192.04423889)(159.19391966,1192.40512843)
\curveto(159.52992026,1192.7535735)(159.78503183,1193.17668537)(159.95925436,1193.67446404)
\lineto(160.12725467,1194.17846494)
\closepath
}
}
{
\newrgbcolor{curcolor}{0 0 0}
\pscustom[linestyle=none,fillstyle=solid,fillcolor=curcolor]
{
\newpath
\moveto(173.8472799,1204.48248341)
\curveto(173.98416904,1204.48248341)(174.1459471,1204.47626118)(174.33261411,1204.46381671)
\curveto(174.51928111,1204.45137224)(174.66861471,1204.43270554)(174.78061491,1204.40781661)
\lineto(174.57528121,1201.79447859)
\curveto(174.47572547,1201.81936753)(174.34505857,1201.83803423)(174.18328051,1201.85047869)
\curveto(174.02150244,1201.87536763)(173.87839107,1201.88781209)(173.7539464,1201.88781209)
\curveto(173.28105667,1201.88781209)(172.82683363,1201.80070083)(172.39127729,1201.62647829)
\curveto(171.95572096,1201.46470022)(171.60105365,1201.19714419)(171.32727539,1200.82381019)
\curveto(171.06594158,1200.45047618)(170.93527468,1199.94025305)(170.93527468,1199.29314078)
\lineto(170.93527468,1194.10379814)
\lineto(168.15393636,1194.10379814)
\lineto(168.15393636,1204.29581641)
\lineto(170.26327348,1204.29581641)
\lineto(170.67394088,1202.57848)
\lineto(170.80460778,1202.57848)
\curveto(171.10327498,1203.1011476)(171.51394239,1203.5491484)(172.03660999,1203.92248241)
\curveto(172.55927759,1204.29581641)(173.16283423,1204.48248341)(173.8472799,1204.48248341)
\closepath
}
}
{
\newrgbcolor{curcolor}{0 0 0}
\pscustom[linestyle=none,fillstyle=solid,fillcolor=curcolor]
{
\newpath
\moveto(185.5139684,1199.21847398)
\curveto(185.5139684,1197.5260265)(185.0659676,1196.21935749)(184.16996599,1195.29846695)
\curveto(183.28640885,1194.37757641)(182.07929558,1193.91713114)(180.54862617,1193.91713114)
\curveto(179.60284669,1193.91713114)(178.75662296,1194.12246484)(178.00995495,1194.53313224)
\curveto(177.27573141,1194.94379965)(176.69706371,1195.54113405)(176.27395184,1196.32513546)
\curveto(175.85083997,1197.12158133)(175.63928403,1198.0860275)(175.63928403,1199.21847398)
\curveto(175.63928403,1200.91092145)(176.0810626,1202.21136823)(176.96461974,1203.1198143)
\curveto(177.84817688,1204.02826037)(179.06151239,1204.48248341)(180.60462627,1204.48248341)
\curveto(181.56285021,1204.48248341)(182.40907395,1204.27714971)(183.14329749,1203.86648231)
\curveto(183.87752102,1203.4558149)(184.45618873,1202.8584805)(184.8793006,1202.07447909)
\curveto(185.30241247,1201.29047769)(185.5139684,1200.33847598)(185.5139684,1199.21847398)
\closepath
\moveto(178.47662245,1199.21847398)
\curveto(178.47662245,1198.21047217)(178.63840052,1197.44513746)(178.96195666,1196.92246986)
\curveto(179.29795726,1196.41224672)(179.83929156,1196.15713515)(180.58595957,1196.15713515)
\curveto(181.32018311,1196.15713515)(181.84907294,1196.41224672)(182.17262908,1196.92246986)
\curveto(182.50862968,1197.44513746)(182.67662998,1198.21047217)(182.67662998,1199.21847398)
\curveto(182.67662998,1200.22647578)(182.50862968,1200.97936602)(182.17262908,1201.47714469)
\curveto(181.84907294,1201.98736783)(181.31396087,1202.2424794)(180.56729287,1202.2424794)
\curveto(179.83306933,1202.2424794)(179.29795726,1201.98736783)(178.96195666,1201.47714469)
\curveto(178.63840052,1200.97936602)(178.47662245,1200.22647578)(178.47662245,1199.21847398)
\closepath
}
}
{
\newrgbcolor{curcolor}{0 0 0}
\pscustom[linestyle=none,fillstyle=solid,fillcolor=curcolor]
{
\newpath
\moveto(197.06864813,1199.21847398)
\curveto(197.06864813,1197.5260265)(196.62064732,1196.21935749)(195.72464572,1195.29846695)
\curveto(194.84108858,1194.37757641)(193.6339753,1193.91713114)(192.10330589,1193.91713114)
\curveto(191.15752642,1193.91713114)(190.31130268,1194.12246484)(189.56463468,1194.53313224)
\curveto(188.83041114,1194.94379965)(188.25174343,1195.54113405)(187.82863156,1196.32513546)
\curveto(187.40551969,1197.12158133)(187.19396376,1198.0860275)(187.19396376,1199.21847398)
\curveto(187.19396376,1200.91092145)(187.63574233,1202.21136823)(188.51929947,1203.1198143)
\curveto(189.40285661,1204.02826037)(190.61619212,1204.48248341)(192.15930599,1204.48248341)
\curveto(193.11752993,1204.48248341)(193.96375367,1204.27714971)(194.69797721,1203.86648231)
\curveto(195.43220075,1203.4558149)(196.01086845,1202.8584805)(196.43398032,1202.07447909)
\curveto(196.85709219,1201.29047769)(197.06864813,1200.33847598)(197.06864813,1199.21847398)
\closepath
\moveto(190.03130218,1199.21847398)
\curveto(190.03130218,1198.21047217)(190.19308025,1197.44513746)(190.51663638,1196.92246986)
\curveto(190.85263698,1196.41224672)(191.39397129,1196.15713515)(192.14063929,1196.15713515)
\curveto(192.87486283,1196.15713515)(193.40375267,1196.41224672)(193.7273088,1196.92246986)
\curveto(194.06330941,1197.44513746)(194.23130971,1198.21047217)(194.23130971,1199.21847398)
\curveto(194.23130971,1200.22647578)(194.06330941,1200.97936602)(193.7273088,1201.47714469)
\curveto(193.40375267,1201.98736783)(192.8686406,1202.2424794)(192.12197259,1202.2424794)
\curveto(191.38774905,1202.2424794)(190.85263698,1201.98736783)(190.51663638,1201.47714469)
\curveto(190.19308025,1200.97936602)(190.03130218,1200.22647578)(190.03130218,1199.21847398)
\closepath
}
}
{
\newrgbcolor{curcolor}{0 0 0}
\pscustom[linestyle=none,fillstyle=solid,fillcolor=curcolor]
{
\newpath
\moveto(211.32999937,1204.48248341)
\curveto(212.48733478,1204.48248341)(213.35844745,1204.18381621)(213.94333739,1203.58648181)
\curveto(214.54067179,1203.00159187)(214.83933899,1202.05581239)(214.83933899,1200.74914339)
\lineto(214.83933899,1194.10379814)
\lineto(212.05800068,1194.10379814)
\lineto(212.05800068,1200.05847548)
\curveto(212.05800068,1201.52692256)(211.54777754,1202.2611461)(210.52733127,1202.2611461)
\curveto(209.79310773,1202.2611461)(209.27044012,1201.99981229)(208.95932845,1201.47714469)
\curveto(208.64821679,1200.95447709)(208.49266095,1200.20158685)(208.49266095,1199.21847398)
\lineto(208.49266095,1194.10379814)
\lineto(205.71132263,1194.10379814)
\lineto(205.71132263,1200.05847548)
\curveto(205.71132263,1201.52692256)(205.2010995,1202.2611461)(204.18065322,1202.2611461)
\curveto(203.40909628,1202.2611461)(202.87398421,1201.96870113)(202.57531701,1201.38381119)
\curveto(202.28909428,1200.81136572)(202.14598291,1199.98380868)(202.14598291,1198.90114007)
\lineto(202.14598291,1194.10379814)
\lineto(199.36464459,1194.10379814)
\lineto(199.36464459,1204.29581641)
\lineto(201.4926484,1204.29581641)
\lineto(201.86598241,1202.9891474)
\lineto(202.01531601,1202.9891474)
\curveto(202.32642768,1203.511815)(202.74953955,1203.89137124)(203.28465162,1204.12781611)
\curveto(203.83220815,1204.36426098)(204.39843139,1204.48248341)(204.98332133,1204.48248341)
\curveto(205.72998933,1204.48248341)(206.3584349,1204.35803874)(206.86865804,1204.10914941)
\curveto(207.39132564,1203.87270454)(207.79577081,1203.49937054)(208.08199355,1202.9891474)
\lineto(208.32466065,1202.9891474)
\curveto(208.63577232,1203.511815)(209.06510642,1203.89137124)(209.61266296,1204.12781611)
\curveto(210.17266396,1204.36426098)(210.74510943,1204.48248341)(211.32999937,1204.48248341)
\closepath
}
}
{
\newrgbcolor{curcolor}{0 0 0}
\pscustom[linestyle=none,fillstyle=solid,fillcolor=curcolor]
{
\newpath
\moveto(130.4179677,1166.22520182)
\curveto(130.4179677,1165.19231108)(130.05085593,1164.39586521)(129.31663239,1163.83586421)
\curveto(128.59485332,1163.28830767)(127.51218471,1163.0145294)(126.06862657,1163.0145294)
\curveto(125.35929196,1163.0145294)(124.74951309,1163.06430727)(124.23928996,1163.163863)
\curveto(123.72906682,1163.25097427)(123.21884368,1163.40030787)(122.70862054,1163.61186381)
\lineto(122.70862054,1165.90786792)
\curveto(123.25617708,1165.65897859)(123.84728925,1165.45364489)(124.48195706,1165.29186682)
\curveto(125.11662486,1165.13008875)(125.67662587,1165.04919972)(126.16196007,1165.04919972)
\curveto(126.69707214,1165.04919972)(127.08285061,1165.13008875)(127.31929548,1165.29186682)
\curveto(127.55574034,1165.45364489)(127.67396278,1165.66520082)(127.67396278,1165.92653462)
\curveto(127.67396278,1166.10075716)(127.62418491,1166.25631299)(127.52462918,1166.39320213)
\curveto(127.43751791,1166.53009126)(127.23840644,1166.68564709)(126.92729477,1166.85986963)
\curveto(126.61618311,1167.03409216)(126.1308489,1167.25809257)(125.47129216,1167.53187083)
\curveto(124.82417989,1167.8056491)(124.29529006,1168.07320514)(123.88462265,1168.33453894)
\curveto(123.48639972,1168.60831721)(123.18773251,1168.93187334)(122.98862105,1169.30520735)
\curveto(122.78950958,1169.69098582)(122.68995384,1170.17009779)(122.68995384,1170.74254326)
\curveto(122.68995384,1171.68832273)(123.05706561,1172.39765733)(123.79128915,1172.87054707)
\curveto(124.52551269,1173.34343681)(125.50240333,1173.57988168)(126.72196107,1173.57988168)
\curveto(127.35662888,1173.57988168)(127.96018551,1173.51765934)(128.53263098,1173.39321467)
\curveto(129.10507646,1173.26877001)(129.69618863,1173.06343631)(130.3059675,1172.77721357)
\lineto(129.46596599,1170.77987666)
\curveto(128.96818732,1170.99143259)(128.49529758,1171.16565513)(128.04729678,1171.30254426)
\curveto(127.59929598,1171.45187786)(127.14507294,1171.52654466)(126.68462767,1171.52654466)
\curveto(125.86329287,1171.52654466)(125.45262546,1171.30254426)(125.45262546,1170.85454346)
\curveto(125.45262546,1170.69276539)(125.50240333,1170.54343179)(125.60195906,1170.40654265)
\curveto(125.71395927,1170.28209799)(125.91929297,1170.14520885)(126.21796017,1169.99587525)
\curveto(126.52907184,1169.84654165)(126.98329487,1169.64743018)(127.58062928,1169.39854085)
\curveto(128.16551922,1169.16209598)(128.66952012,1168.91320664)(129.09263199,1168.65187284)
\curveto(129.51574386,1168.40298351)(129.83929999,1168.0856496)(130.0633004,1167.69987113)
\curveto(130.29974526,1167.31409267)(130.4179677,1166.82253623)(130.4179677,1166.22520182)
\closepath
}
}
{
\newrgbcolor{curcolor}{0 0 0}
\pscustom[linestyle=none,fillstyle=solid,fillcolor=curcolor]
{
\newpath
\moveto(136.78330802,1173.57988168)
\curveto(138.18953276,1173.57988168)(139.30331254,1173.17543651)(140.12464734,1172.36654617)
\curveto(140.94598215,1171.57010029)(141.35664955,1170.43143159)(141.35664955,1168.95054004)
\lineto(141.35664955,1167.60653763)
\lineto(134.7859711,1167.60653763)
\curveto(134.81086004,1166.82253623)(135.04108267,1166.20653512)(135.47663901,1165.75853432)
\curveto(135.92463981,1165.31053352)(136.54064092,1165.08653312)(137.32464232,1165.08653312)
\curveto(137.97175459,1165.08653312)(138.56286676,1165.14875545)(139.09797883,1165.27320012)
\curveto(139.64553537,1165.41008925)(140.20553638,1165.61542295)(140.77798185,1165.88920122)
\lineto(140.77798185,1163.74253071)
\curveto(140.26775871,1163.49364137)(139.73886887,1163.3131966)(139.19131233,1163.2011964)
\curveto(138.6437558,1163.07675174)(137.97797683,1163.0145294)(137.19397542,1163.0145294)
\curveto(136.17352915,1163.0145294)(135.27130531,1163.2011964)(134.4873039,1163.57453041)
\curveto(133.7033025,1163.96030888)(133.08730139,1164.53275435)(132.63930059,1165.29186682)
\curveto(132.19129979,1166.06342376)(131.96729939,1167.0403144)(131.96729939,1168.22253874)
\curveto(131.96729939,1169.40476308)(132.16641085,1170.39409819)(132.56463379,1171.19054406)
\curveto(132.97530119,1171.98698993)(133.54152443,1172.58432434)(134.2633035,1172.98254727)
\curveto(134.98508257,1173.38077021)(135.82508408,1173.57988168)(136.78330802,1173.57988168)
\closepath
\moveto(136.80197472,1171.60121146)
\curveto(136.25441818,1171.60121146)(135.80641738,1171.42698893)(135.45797231,1171.07854386)
\curveto(135.10952724,1170.73009879)(134.90419354,1170.18876449)(134.84197121,1169.45454095)
\lineto(138.74331153,1169.45454095)
\curveto(138.73086707,1170.06431982)(138.56286676,1170.57454295)(138.23931063,1170.98521036)
\curveto(137.92819896,1171.39587776)(137.44908699,1171.60121146)(136.80197472,1171.60121146)
\closepath
}
}
{
\newrgbcolor{curcolor}{0 0 0}
\pscustom[linestyle=none,fillstyle=solid,fillcolor=curcolor]
{
\newpath
\moveto(149.30865683,1173.57988168)
\curveto(149.44554596,1173.57988168)(149.60732403,1173.57365944)(149.79399103,1173.56121498)
\curveto(149.98065803,1173.54877051)(150.12999163,1173.53010381)(150.24199184,1173.50521487)
\lineto(150.03665813,1170.89187686)
\curveto(149.9371024,1170.91676579)(149.8064355,1170.93543249)(149.64465743,1170.94787696)
\curveto(149.48287936,1170.97276589)(149.339768,1170.98521036)(149.21532333,1170.98521036)
\curveto(148.74243359,1170.98521036)(148.28821056,1170.89809909)(147.85265422,1170.72387656)
\curveto(147.41709788,1170.56209849)(147.06243058,1170.29454245)(146.78865231,1169.92120845)
\curveto(146.52731851,1169.54787445)(146.39665161,1169.03765131)(146.39665161,1168.39053904)
\lineto(146.39665161,1163.2011964)
\lineto(143.61531329,1163.2011964)
\lineto(143.61531329,1173.39321467)
\lineto(145.7246504,1173.39321467)
\lineto(146.13531781,1171.67587826)
\lineto(146.26598471,1171.67587826)
\curveto(146.56465191,1172.19854587)(146.97531931,1172.64654667)(147.49798692,1173.01988067)
\curveto(148.02065452,1173.39321467)(148.62421116,1173.57988168)(149.30865683,1173.57988168)
\closepath
}
}
{
\newrgbcolor{curcolor}{0 0 0}
\pscustom[linestyle=none,fillstyle=solid,fillcolor=curcolor]
{
\newpath
\moveto(154.51667559,1163.2011964)
\lineto(150.63400196,1173.39321467)
\lineto(153.54600718,1173.39321467)
\lineto(155.5060107,1167.58787093)
\curveto(155.6180109,1167.23942587)(155.70512217,1166.87853633)(155.7673445,1166.50520233)
\curveto(155.82956683,1166.13186832)(155.87312247,1165.79586772)(155.8980114,1165.49720052)
\lineto(155.9726782,1165.49720052)
\curveto(156.0100116,1166.16920172)(156.14690074,1166.86609186)(156.3833456,1167.58787093)
\lineto(158.34334912,1173.39321467)
\lineto(161.25535434,1173.39321467)
\lineto(157.37268071,1163.2011964)
\closepath
}
}
{
\newrgbcolor{curcolor}{0 0 0}
\pscustom[linestyle=none,fillstyle=solid,fillcolor=curcolor]
{
\newpath
\moveto(166.91136893,1173.57988168)
\curveto(168.31759368,1173.57988168)(169.43137345,1173.17543651)(170.25270826,1172.36654617)
\curveto(171.07404306,1171.57010029)(171.48471046,1170.43143159)(171.48471046,1168.95054004)
\lineto(171.48471046,1167.60653763)
\lineto(164.91403202,1167.60653763)
\curveto(164.93892095,1166.82253623)(165.16914359,1166.20653512)(165.60469992,1165.75853432)
\curveto(166.05270073,1165.31053352)(166.66870183,1165.08653312)(167.45270324,1165.08653312)
\curveto(168.09981551,1165.08653312)(168.69092768,1165.14875545)(169.22603975,1165.27320012)
\curveto(169.77359629,1165.41008925)(170.33359729,1165.61542295)(170.90604276,1165.88920122)
\lineto(170.90604276,1163.74253071)
\curveto(170.39581962,1163.49364137)(169.86692979,1163.3131966)(169.31937325,1163.2011964)
\curveto(168.77181671,1163.07675174)(168.10603774,1163.0145294)(167.32203634,1163.0145294)
\curveto(166.30159006,1163.0145294)(165.39936622,1163.2011964)(164.61536482,1163.57453041)
\curveto(163.83136341,1163.96030888)(163.21536231,1164.53275435)(162.7673615,1165.29186682)
\curveto(162.3193607,1166.06342376)(162.0953603,1167.0403144)(162.0953603,1168.22253874)
\curveto(162.0953603,1169.40476308)(162.29447177,1170.39409819)(162.6926947,1171.19054406)
\curveto(163.10336211,1171.98698993)(163.66958534,1172.58432434)(164.39136442,1172.98254727)
\curveto(165.11314349,1173.38077021)(165.95314499,1173.57988168)(166.91136893,1173.57988168)
\closepath
\moveto(166.93003563,1171.60121146)
\curveto(166.3824791,1171.60121146)(165.93447829,1171.42698893)(165.58603322,1171.07854386)
\curveto(165.23758816,1170.73009879)(165.03225445,1170.18876449)(164.97003212,1169.45454095)
\lineto(168.87137245,1169.45454095)
\curveto(168.85892798,1170.06431982)(168.69092768,1170.57454295)(168.36737154,1170.98521036)
\curveto(168.05625987,1171.39587776)(167.5771479,1171.60121146)(166.93003563,1171.60121146)
\closepath
}
}
{
\newrgbcolor{curcolor}{0 0 0}
\pscustom[linestyle=none,fillstyle=solid,fillcolor=curcolor]
{
\newpath
\moveto(179.43671774,1173.57988168)
\curveto(179.57360688,1173.57988168)(179.73538495,1173.57365944)(179.92205195,1173.56121498)
\curveto(180.10871895,1173.54877051)(180.25805255,1173.53010381)(180.37005275,1173.50521487)
\lineto(180.16471905,1170.89187686)
\curveto(180.06516331,1170.91676579)(179.93449641,1170.93543249)(179.77271835,1170.94787696)
\curveto(179.61094028,1170.97276589)(179.46782891,1170.98521036)(179.34338424,1170.98521036)
\curveto(178.87049451,1170.98521036)(178.41627147,1170.89809909)(177.98071513,1170.72387656)
\curveto(177.5451588,1170.56209849)(177.1904915,1170.29454245)(176.91671323,1169.92120845)
\curveto(176.65537942,1169.54787445)(176.52471252,1169.03765131)(176.52471252,1168.39053904)
\lineto(176.52471252,1163.2011964)
\lineto(173.7433742,1163.2011964)
\lineto(173.7433742,1173.39321467)
\lineto(175.85271132,1173.39321467)
\lineto(176.26337872,1171.67587826)
\lineto(176.39404562,1171.67587826)
\curveto(176.69271283,1172.19854587)(177.10338023,1172.64654667)(177.62604783,1173.01988067)
\curveto(178.14871544,1173.39321467)(178.75227207,1173.57988168)(179.43671774,1173.57988168)
\closepath
}
}
{
\newrgbcolor{curcolor}{0 0 0}
\pscustom[linestyle=none,fillstyle=solid,fillcolor=curcolor]
{
\newpath
\moveto(176.24773237,1134.27993443)
\curveto(176.24773237,1133.24704369)(175.8806206,1132.45059782)(175.14639707,1131.89059682)
\curveto(174.42461799,1131.34304028)(173.34194939,1131.06926201)(171.89839124,1131.06926201)
\curveto(171.18905664,1131.06926201)(170.57927777,1131.11903988)(170.06905463,1131.21859561)
\curveto(169.55883149,1131.30570688)(169.04860836,1131.45504048)(168.53838522,1131.66659642)
\lineto(168.53838522,1133.96260053)
\curveto(169.08594176,1133.7137112)(169.67705393,1133.50837749)(170.31172173,1133.34659943)
\curveto(170.94638954,1133.18482136)(171.50639054,1133.10393232)(171.99172474,1133.10393232)
\curveto(172.52683681,1133.10393232)(172.91261528,1133.18482136)(173.14906015,1133.34659943)
\curveto(173.38550502,1133.50837749)(173.50372745,1133.71993343)(173.50372745,1133.98126723)
\curveto(173.50372745,1134.15548977)(173.45394959,1134.3110456)(173.35439385,1134.44793473)
\curveto(173.26728259,1134.58482387)(173.06817112,1134.7403797)(172.75705945,1134.91460224)
\curveto(172.44594778,1135.08882477)(171.96061358,1135.31282517)(171.30105684,1135.58660344)
\curveto(170.65394457,1135.86038171)(170.12505473,1136.12793775)(169.71438733,1136.38927155)
\curveto(169.31616439,1136.66304982)(169.01749719,1136.98660595)(168.81838572,1137.35993995)
\curveto(168.61927425,1137.74571842)(168.51971852,1138.22483039)(168.51971852,1138.79727586)
\curveto(168.51971852,1139.74305534)(168.88683029,1140.45238994)(169.62105383,1140.92527968)
\curveto(170.35527737,1141.39816942)(171.33216801,1141.63461428)(172.55172575,1141.63461428)
\curveto(173.18639355,1141.63461428)(173.78995019,1141.57239195)(174.36239566,1141.44794728)
\curveto(174.93484113,1141.32350261)(175.5259533,1141.11816891)(176.13573217,1140.83194618)
\lineto(175.29573067,1138.83460926)
\curveto(174.797952,1139.0461652)(174.32506226,1139.22038773)(173.87706146,1139.35727687)
\curveto(173.42906065,1139.50661047)(172.97483762,1139.58127727)(172.51439235,1139.58127727)
\curveto(171.69305754,1139.58127727)(171.28239014,1139.35727687)(171.28239014,1138.90927606)
\curveto(171.28239014,1138.747498)(171.33216801,1138.5981644)(171.43172374,1138.46127526)
\curveto(171.54372394,1138.33683059)(171.74905764,1138.19994146)(172.04772484,1138.05060786)
\curveto(172.35883651,1137.90127426)(172.81305955,1137.70216279)(173.41039395,1137.45327345)
\curveto(173.99528389,1137.21682859)(174.49928479,1136.96793925)(174.92239666,1136.70660545)
\curveto(175.34550853,1136.45771611)(175.66906467,1136.14038221)(175.89306507,1135.75460374)
\curveto(176.12950994,1135.36882527)(176.24773237,1134.87726884)(176.24773237,1134.27993443)
\closepath
}
}
{
\newrgbcolor{curcolor}{0 0 0}
\pscustom[linestyle=none,fillstyle=solid,fillcolor=curcolor]
{
\newpath
\moveto(187.67174843,1136.37060485)
\curveto(187.67174843,1134.67815737)(187.22374762,1133.37148836)(186.32774602,1132.45059782)
\curveto(185.44418888,1131.52970728)(184.2370756,1131.06926201)(182.70640619,1131.06926201)
\curveto(181.76062672,1131.06926201)(180.91440298,1131.27459571)(180.16773498,1131.68526312)
\curveto(179.43351144,1132.09593052)(178.85484373,1132.69326492)(178.43173186,1133.47726633)
\curveto(178.00861999,1134.2737122)(177.79706406,1135.23815837)(177.79706406,1136.37060485)
\curveto(177.79706406,1138.06305233)(178.23884263,1139.3634991)(179.12239977,1140.27194517)
\curveto(180.00595691,1141.18039125)(181.21929242,1141.63461428)(182.76240629,1141.63461428)
\curveto(183.72063023,1141.63461428)(184.56685397,1141.42928058)(185.30107751,1141.01861318)
\curveto(186.03530105,1140.60794578)(186.61396875,1140.01061137)(187.03708062,1139.22660997)
\curveto(187.46019249,1138.44260856)(187.67174843,1137.49060686)(187.67174843,1136.37060485)
\closepath
\moveto(180.63440248,1136.37060485)
\curveto(180.63440248,1135.36260304)(180.79618055,1134.59726834)(181.11973668,1134.07460073)
\curveto(181.45573729,1133.56437759)(181.99707159,1133.30926603)(182.74373959,1133.30926603)
\curveto(183.47796313,1133.30926603)(184.00685297,1133.56437759)(184.3304091,1134.07460073)
\curveto(184.66640971,1134.59726834)(184.83441001,1135.36260304)(184.83441001,1136.37060485)
\curveto(184.83441001,1137.37860665)(184.66640971,1138.13149689)(184.3304091,1138.62927556)
\curveto(184.00685297,1139.1394987)(183.4717409,1139.39461027)(182.72507289,1139.39461027)
\curveto(181.99084936,1139.39461027)(181.45573729,1139.1394987)(181.11973668,1138.62927556)
\curveto(180.79618055,1138.13149689)(180.63440248,1137.37860665)(180.63440248,1136.37060485)
\closepath
}
}
{
\newrgbcolor{curcolor}{0 0 0}
\pscustom[linestyle=none,fillstyle=solid,fillcolor=curcolor]
{
\newpath
\moveto(194.11175041,1131.06926201)
\curveto(192.59352547,1131.06926201)(191.41752336,1131.48615165)(190.58374409,1132.31993092)
\curveto(189.76240928,1133.15371019)(189.35174188,1134.4790459)(189.35174188,1136.29593805)
\curveto(189.35174188,1137.54038472)(189.56329781,1138.55460876)(189.98640968,1139.33861017)
\curveto(190.40952155,1140.12261157)(190.99441149,1140.70127928)(191.74107949,1141.07461328)
\curveto(192.50019197,1141.44794728)(193.37130464,1141.63461428)(194.35441751,1141.63461428)
\curveto(195.05130765,1141.63461428)(195.65486429,1141.56616972)(196.16508742,1141.42928058)
\curveto(196.68775503,1141.29239145)(197.14197806,1141.13061338)(197.52775653,1140.94394638)
\lineto(196.70642173,1138.79727586)
\curveto(196.27086539,1138.9714984)(195.86019799,1139.11460977)(195.47441952,1139.22660997)
\curveto(195.10108552,1139.33861017)(194.72775151,1139.39461027)(194.35441751,1139.39461027)
\curveto(192.91085937,1139.39461027)(192.1890803,1138.36794176)(192.1890803,1136.31460475)
\curveto(192.1890803,1135.29415847)(192.3757473,1134.54126823)(192.7490813,1134.05593403)
\curveto(193.13485977,1133.57059983)(193.66997184,1133.32793273)(194.35441751,1133.32793273)
\curveto(194.93930745,1133.32793273)(195.45575282,1133.40259953)(195.90375362,1133.55193313)
\curveto(196.35175443,1133.7137112)(196.78731076,1133.93148936)(197.21042263,1134.20526763)
\lineto(197.21042263,1131.83459672)
\curveto(196.78731076,1131.56081845)(196.33930996,1131.36792921)(195.86642022,1131.25592901)
\curveto(195.40597495,1131.13148434)(194.82108502,1131.06926201)(194.11175041,1131.06926201)
\closepath
}
}
{
\newrgbcolor{curcolor}{0 0 0}
\pscustom[linestyle=none,fillstyle=solid,fillcolor=curcolor]
{
\newpath
\moveto(202.34376387,1145.44262111)
\lineto(202.34376387,1139.09594307)
\curveto(202.34376387,1138.7101646)(202.32509717,1138.32438613)(202.28776377,1137.93860766)
\curveto(202.26287484,1137.56527366)(202.23176367,1137.18571742)(202.19443027,1136.79993895)
\lineto(202.23176367,1136.79993895)
\curveto(202.41843067,1137.06127275)(202.61131991,1137.32260655)(202.81043138,1137.58394036)
\curveto(203.00954284,1137.84527416)(203.22109878,1138.10038573)(203.44509918,1138.34927506)
\lineto(206.3011043,1141.44794728)
\lineto(209.43710992,1141.44794728)
\lineto(205.38643599,1137.02393935)
\lineto(209.67977702,1131.25592901)
\lineto(206.4691046,1131.25592901)
\lineto(203.53843268,1135.38126974)
\lineto(202.34376387,1134.42926803)
\lineto(202.34376387,1131.25592901)
\lineto(199.56242555,1131.25592901)
\lineto(199.56242555,1145.44262111)
\closepath
}
}
{
\newrgbcolor{curcolor}{0 0 0}
\pscustom[linestyle=none,fillstyle=solid,fillcolor=curcolor]
{
\newpath
\moveto(215.33580292,1141.63461428)
\curveto(216.74202766,1141.63461428)(217.85580743,1141.23016911)(218.67714224,1140.42127878)
\curveto(219.49847705,1139.6248329)(219.90914445,1138.4861642)(219.90914445,1137.00527265)
\lineto(219.90914445,1135.66127024)
\lineto(213.338466,1135.66127024)
\curveto(213.36335494,1134.87726884)(213.59357757,1134.26126773)(214.02913391,1133.81326693)
\curveto(214.47713471,1133.36526613)(215.09313582,1133.14126573)(215.87713722,1133.14126573)
\curveto(216.52424949,1133.14126573)(217.11536166,1133.20348806)(217.65047373,1133.32793273)
\curveto(218.19803027,1133.46482186)(218.75803127,1133.67015556)(219.33047674,1133.94393383)
\lineto(219.33047674,1131.79726332)
\curveto(218.82025361,1131.54837398)(218.29136377,1131.36792921)(217.74380723,1131.25592901)
\curveto(217.1962507,1131.13148434)(216.53047173,1131.06926201)(215.74647032,1131.06926201)
\curveto(214.72602405,1131.06926201)(213.82380021,1131.25592901)(213.0397988,1131.62926301)
\curveto(212.2557974,1132.01504148)(211.63979629,1132.58748695)(211.19179549,1133.34659943)
\curveto(210.74379469,1134.11815637)(210.51979428,1135.09504701)(210.51979428,1136.27727135)
\curveto(210.51979428,1137.45949569)(210.71890575,1138.44883079)(211.11712869,1139.24527667)
\curveto(211.52779609,1140.04172254)(212.09401933,1140.63905694)(212.8157984,1141.03727988)
\curveto(213.53757747,1141.43550282)(214.37757898,1141.63461428)(215.33580292,1141.63461428)
\closepath
\moveto(215.35446962,1139.65594407)
\curveto(214.80691308,1139.65594407)(214.35891228,1139.48172154)(214.01046721,1139.13327647)
\curveto(213.66202214,1138.7848314)(213.45668844,1138.24349709)(213.3944661,1137.50927356)
\lineto(217.29580643,1137.50927356)
\curveto(217.28336196,1138.11905243)(217.11536166,1138.62927556)(216.79180553,1139.03994297)
\curveto(216.48069386,1139.45061037)(216.00158189,1139.65594407)(215.35446962,1139.65594407)
\closepath
}
}
{
\newrgbcolor{curcolor}{0 0 0}
\pscustom[linestyle=none,fillstyle=solid,fillcolor=curcolor]
{
\newpath
\moveto(226.46114922,1133.29059933)
\curveto(226.77226089,1133.29059933)(227.07092809,1133.31548826)(227.35715082,1133.36526613)
\curveto(227.64337356,1133.42748846)(227.9295963,1133.50837749)(228.21581903,1133.60793323)
\lineto(228.21581903,1131.53592951)
\curveto(227.91715183,1131.39904038)(227.54381783,1131.28704018)(227.09581702,1131.19992891)
\curveto(226.66026069,1131.11281764)(226.18114872,1131.06926201)(225.65848111,1131.06926201)
\curveto(225.04870224,1131.06926201)(224.50114571,1131.16881775)(224.0158115,1131.36792921)
\curveto(223.54292177,1131.56704068)(223.16336553,1131.90926352)(222.87714279,1132.39459772)
\curveto(222.60336453,1132.87993192)(222.46647539,1133.56437759)(222.46647539,1134.44793473)
\lineto(222.46647539,1139.35727687)
\lineto(221.14113968,1139.35727687)
\lineto(221.14113968,1140.53327898)
\lineto(222.67180909,1141.46661398)
\lineto(223.4744772,1143.6132845)
\lineto(225.24781371,1143.6132845)
\lineto(225.24781371,1141.44794728)
\lineto(228.10381883,1141.44794728)
\lineto(228.10381883,1139.35727687)
\lineto(225.24781371,1139.35727687)
\lineto(225.24781371,1134.44793473)
\curveto(225.24781371,1134.06215626)(225.35981391,1133.7697113)(225.58381431,1133.57059983)
\curveto(225.80781471,1133.38393283)(226.10025968,1133.29059933)(226.46114922,1133.29059933)
\closepath
}
}
{
\newrgbcolor{curcolor}{0 0 0}
\pscustom[linestyle=none,fillstyle=solid,fillcolor=curcolor]
{
\newpath
\moveto(237.38116971,1134.27993443)
\curveto(237.38116971,1133.24704369)(237.01405794,1132.45059782)(236.27983441,1131.89059682)
\curveto(235.55805533,1131.34304028)(234.47538673,1131.06926201)(233.03182858,1131.06926201)
\curveto(232.32249398,1131.06926201)(231.71271511,1131.11903988)(231.20249197,1131.21859561)
\curveto(230.69226883,1131.30570688)(230.1820457,1131.45504048)(229.67182256,1131.66659642)
\lineto(229.67182256,1133.96260053)
\curveto(230.2193791,1133.7137112)(230.81049127,1133.50837749)(231.44515907,1133.34659943)
\curveto(232.07982688,1133.18482136)(232.63982788,1133.10393232)(233.12516208,1133.10393232)
\curveto(233.66027415,1133.10393232)(234.04605262,1133.18482136)(234.28249749,1133.34659943)
\curveto(234.51894236,1133.50837749)(234.63716479,1133.71993343)(234.63716479,1133.98126723)
\curveto(234.63716479,1134.15548977)(234.58738693,1134.3110456)(234.48783119,1134.44793473)
\curveto(234.40071993,1134.58482387)(234.20160846,1134.7403797)(233.89049679,1134.91460224)
\curveto(233.57938512,1135.08882477)(233.09405092,1135.31282517)(232.43449418,1135.58660344)
\curveto(231.78738191,1135.86038171)(231.25849207,1136.12793775)(230.84782467,1136.38927155)
\curveto(230.44960173,1136.66304982)(230.15093453,1136.98660595)(229.95182306,1137.35993995)
\curveto(229.75271159,1137.74571842)(229.65315586,1138.22483039)(229.65315586,1138.79727586)
\curveto(229.65315586,1139.74305534)(230.02026763,1140.45238994)(230.75449117,1140.92527968)
\curveto(231.48871471,1141.39816942)(232.46560535,1141.63461428)(233.68516309,1141.63461428)
\curveto(234.31983089,1141.63461428)(234.92338753,1141.57239195)(235.495833,1141.44794728)
\curveto(236.06827847,1141.32350261)(236.65939064,1141.11816891)(237.26916951,1140.83194618)
\lineto(236.42916801,1138.83460926)
\curveto(235.93138934,1139.0461652)(235.4584996,1139.22038773)(235.0104988,1139.35727687)
\curveto(234.56249799,1139.50661047)(234.10827496,1139.58127727)(233.64782969,1139.58127727)
\curveto(232.82649488,1139.58127727)(232.41582748,1139.35727687)(232.41582748,1138.90927606)
\curveto(232.41582748,1138.747498)(232.46560535,1138.5981644)(232.56516108,1138.46127526)
\curveto(232.67716128,1138.33683059)(232.88249498,1138.19994146)(233.18116218,1138.05060786)
\curveto(233.49227385,1137.90127426)(233.94649689,1137.70216279)(234.54383129,1137.45327345)
\curveto(235.12872123,1137.21682859)(235.63272213,1136.96793925)(236.055834,1136.70660545)
\curveto(236.47894587,1136.45771611)(236.80250201,1136.14038221)(237.02650241,1135.75460374)
\curveto(237.26294728,1135.36882527)(237.38116971,1134.87726884)(237.38116971,1134.27993443)
\closepath
}
}
{
\newrgbcolor{curcolor}{0.50196081 0.50196081 0.50196081}
\pscustom[linestyle=none,fillstyle=solid,fillcolor=curcolor]
{
\newpath
\moveto(150.51327202,769.21782726)
\lineto(160.64519867,769.21782726)
\lineto(160.64519867,616.46945743)
\lineto(150.51327202,616.46945743)
\closepath
}
}
{
\newrgbcolor{curcolor}{0.50196081 0.50196081 0.50196081}
\pscustom[linestyle=none,fillstyle=solid,fillcolor=curcolor]
{
\newpath
\moveto(120.51668903,784.01024045)
\lineto(160.03119882,784.01024045)
\lineto(160.03119882,773.8783138)
\lineto(120.51668903,773.8783138)
\closepath
}
}
{
\newrgbcolor{curcolor}{0.50196081 0.50196081 0.50196081}
\pscustom[linestyle=none,fillstyle=solid,fillcolor=curcolor]
{
\newpath
\moveto(166.59137395,754.79243241)
\lineto(206.10588374,754.79243241)
\lineto(206.10588374,744.66050576)
\lineto(166.59137395,744.66050576)
\closepath
}
}
{
\newrgbcolor{curcolor}{0.50196081 0.50196081 0.50196081}
\pscustom[linestyle=none,fillstyle=solid,fillcolor=curcolor]
{
\newpath
\moveto(166.59137395,733.47540908)
\lineto(206.10588374,733.47540908)
\lineto(206.10588374,723.34348243)
\lineto(166.59137395,723.34348243)
\closepath
}
}
{
\newrgbcolor{curcolor}{0.50196081 0.50196081 0.50196081}
\pscustom[linestyle=none,fillstyle=solid,fillcolor=curcolor]
{
\newpath
\moveto(166.59137395,712.15835691)
\lineto(206.10588374,712.15835691)
\lineto(206.10588374,702.02643027)
\lineto(166.59137395,702.02643027)
\closepath
}
}
{
\newrgbcolor{curcolor}{0.50196081 0.50196081 0.50196081}
\pscustom[linestyle=none,fillstyle=solid,fillcolor=curcolor]
{
\newpath
\moveto(166.59137395,690.84130475)
\lineto(206.10588374,690.84130475)
\lineto(206.10588374,680.7093781)
\lineto(166.59137395,680.7093781)
\closepath
}
}
{
\newrgbcolor{curcolor}{0.50196081 0.50196081 0.50196081}
\pscustom[linestyle=none,fillstyle=solid,fillcolor=curcolor]
{
\newpath
\moveto(166.59137395,669.52436793)
\lineto(206.10588374,669.52436793)
\lineto(206.10588374,659.39244128)
\lineto(166.59137395,659.39244128)
\closepath
}
}
{
\newrgbcolor{curcolor}{0.50196081 0.50196081 0.50196081}
\pscustom[linestyle=none,fillstyle=solid,fillcolor=curcolor]
{
\newpath
\moveto(166.59137395,648.20731576)
\lineto(206.10588374,648.20731576)
\lineto(206.10588374,638.07538911)
\lineto(166.59137395,638.07538911)
\closepath
}
}
{
\newrgbcolor{curcolor}{0.50196081 0.50196081 0.50196081}
\pscustom[linestyle=none,fillstyle=solid,fillcolor=curcolor]
{
\newpath
\moveto(166.59137395,626.89026359)
\lineto(206.10588374,626.89026359)
\lineto(206.10588374,616.75833695)
\lineto(166.59137395,616.75833695)
\closepath
}
}
{
\newrgbcolor{curcolor}{0.50196081 0.50196081 0.50196081}
\pscustom[linestyle=none,fillstyle=solid,fillcolor=curcolor]
{
\newpath
\moveto(152.32197919,254.01278759)
\lineto(162.45390584,254.01278759)
\lineto(162.45390584,144.16414756)
\lineto(152.32197919,144.16414756)
\closepath
}
}
{
\newrgbcolor{curcolor}{0.50196081 0.50196081 0.50196081}
\pscustom[linestyle=none,fillstyle=solid,fillcolor=curcolor]
{
\newpath
\moveto(122.3253962,268.80517194)
\lineto(161.83990599,268.80517194)
\lineto(161.83990599,258.67324529)
\lineto(122.3253962,258.67324529)
\closepath
}
}
{
\newrgbcolor{curcolor}{0.50196081 0.50196081 0.50196081}
\pscustom[linestyle=none,fillstyle=solid,fillcolor=curcolor]
{
\newpath
\moveto(168.40008112,239.58742157)
\lineto(207.91459091,239.58742157)
\lineto(207.91459091,229.45549492)
\lineto(168.40008112,229.45549492)
\closepath
}
}
{
\newrgbcolor{curcolor}{0.50196081 0.50196081 0.50196081}
\pscustom[linestyle=none,fillstyle=solid,fillcolor=curcolor]
{
\newpath
\moveto(168.40008112,218.2703694)
\lineto(207.91459091,218.2703694)
\lineto(207.91459091,208.13844275)
\lineto(168.40008112,208.13844275)
\closepath
}
}
{
\newrgbcolor{curcolor}{0.50196081 0.50196081 0.50196081}
\pscustom[linestyle=none,fillstyle=solid,fillcolor=curcolor]
{
\newpath
\moveto(168.40008112,196.95331724)
\lineto(207.91459091,196.95331724)
\lineto(207.91459091,186.82139059)
\lineto(168.40008112,186.82139059)
\closepath
}
}
{
\newrgbcolor{curcolor}{0.50196081 0.50196081 0.50196081}
\pscustom[linestyle=none,fillstyle=solid,fillcolor=curcolor]
{
\newpath
\moveto(168.40008112,175.63626507)
\lineto(207.91459091,175.63626507)
\lineto(207.91459091,165.50433842)
\lineto(168.40008112,165.50433842)
\closepath
}
}
{
\newrgbcolor{curcolor}{0.50196081 0.50196081 0.50196081}
\pscustom[linestyle=none,fillstyle=solid,fillcolor=curcolor]
{
\newpath
\moveto(168.40008112,154.31932825)
\lineto(207.91459091,154.31932825)
\lineto(207.91459091,144.1874016)
\lineto(168.40008112,144.1874016)
\closepath
}
}
{
\newrgbcolor{curcolor}{0.50196081 0.50196081 0.50196081}
\pscustom[linestyle=none,fillstyle=solid,fillcolor=curcolor]
{
\newpath
\moveto(75.79028753,116.62255475)
\lineto(115.30479732,116.62255475)
\lineto(115.30479732,106.4906281)
\lineto(75.79028753,106.4906281)
\closepath
}
}
{
\newrgbcolor{curcolor}{0.50196081 0.50196081 0.50196081}
\pscustom[linestyle=none,fillstyle=solid,fillcolor=curcolor]
{
\newpath
\moveto(75.79028753,95.30550258)
\lineto(115.30479732,95.30550258)
\lineto(115.30479732,85.17357593)
\lineto(75.79028753,85.17357593)
\closepath
}
}
{
\newrgbcolor{curcolor}{0.50196081 0.50196081 0.50196081}
\pscustom[linestyle=none,fillstyle=solid,fillcolor=curcolor]
{
\newpath
\moveto(75.79028753,73.98845042)
\lineto(115.30479732,73.98845042)
\lineto(115.30479732,63.85652377)
\lineto(75.79028753,63.85652377)
\closepath
}
}
{
\newrgbcolor{curcolor}{0.50196081 0.50196081 0.50196081}
\pscustom[linestyle=none,fillstyle=solid,fillcolor=curcolor]
{
\newpath
\moveto(75.79028753,52.67139825)
\lineto(115.30479732,52.67139825)
\lineto(115.30479732,42.5394716)
\lineto(75.79028753,42.5394716)
\closepath
}
}
{
\newrgbcolor{curcolor}{0.50196081 0.50196081 0.50196081}
\pscustom[linestyle=none,fillstyle=solid,fillcolor=curcolor]
{
\newpath
\moveto(75.79028753,31.35446143)
\lineto(115.30479732,31.35446143)
\lineto(115.30479732,21.22253478)
\lineto(75.79028753,21.22253478)
\closepath
}
}
{
\newrgbcolor{curcolor}{0.50196081 0.50196081 0.50196081}
\pscustom[linestyle=none,fillstyle=solid,fillcolor=curcolor]
{
\newpath
\moveto(150.84107406,580.57040477)
\lineto(160.97300071,580.57040477)
\lineto(160.97300071,555.981193)
\lineto(150.84107406,555.981193)
\closepath
}
}
{
\newrgbcolor{curcolor}{0.50196081 0.50196081 0.50196081}
\pscustom[linestyle=none,fillstyle=solid,fillcolor=curcolor]
{
\newpath
\moveto(120.84449828,595.3628468)
\lineto(160.35900807,595.3628468)
\lineto(160.35900807,585.23092015)
\lineto(120.84449828,585.23092015)
\closepath
}
}
{
\newrgbcolor{curcolor}{0.50196081 0.50196081 0.50196081}
\pscustom[linestyle=none,fillstyle=solid,fillcolor=curcolor]
{
\newpath
\moveto(166.919176,566.14503875)
\lineto(206.43368578,566.14503875)
\lineto(206.43368578,556.01311211)
\lineto(166.919176,556.01311211)
\closepath
}
}
{
\newrgbcolor{curcolor}{0.50196081 0.50196081 0.50196081}
\pscustom[linestyle=none,fillstyle=solid,fillcolor=curcolor]
{
\newpath
\moveto(196.22527836,550.78828586)
\lineto(206.35720501,550.78828586)
\lineto(206.35720501,290.5113016)
\lineto(196.22527836,290.5113016)
\closepath
}
}
{
\newrgbcolor{curcolor}{0.50196081 0.50196081 0.50196081}
\pscustom[linestyle=none,fillstyle=solid,fillcolor=curcolor]
{
\newpath
\moveto(212.30338029,536.36286217)
\lineto(251.81789008,536.36286217)
\lineto(251.81789008,526.23093552)
\lineto(212.30338029,526.23093552)
\closepath
}
}
{
\newrgbcolor{curcolor}{0.50196081 0.50196081 0.50196081}
\pscustom[linestyle=none,fillstyle=solid,fillcolor=curcolor]
{
\newpath
\moveto(212.30338029,514.9443667)
\lineto(251.81789008,514.9443667)
\lineto(251.81789008,504.81244005)
\lineto(212.30338029,504.81244005)
\closepath
}
}
{
\newrgbcolor{curcolor}{0.50196081 0.50196081 0.50196081}
\pscustom[linestyle=none,fillstyle=solid,fillcolor=curcolor]
{
\newpath
\moveto(212.30338029,493.52587122)
\lineto(251.81789008,493.52587122)
\lineto(251.81789008,483.39394457)
\lineto(212.30338029,483.39394457)
\closepath
}
}
{
\newrgbcolor{curcolor}{0.50196081 0.50196081 0.50196081}
\pscustom[linestyle=none,fillstyle=solid,fillcolor=curcolor]
{
\newpath
\moveto(212.30338029,472.10737574)
\lineto(251.81789008,472.10737574)
\lineto(251.81789008,461.97544909)
\lineto(212.30338029,461.97544909)
\closepath
}
}
{
\newrgbcolor{curcolor}{0.50196081 0.50196081 0.50196081}
\pscustom[linestyle=none,fillstyle=solid,fillcolor=curcolor]
{
\newpath
\moveto(212.30338029,450.68888026)
\lineto(251.81789008,450.68888026)
\lineto(251.81789008,440.55695361)
\lineto(212.30338029,440.55695361)
\closepath
}
}
{
\newrgbcolor{curcolor}{0.50196081 0.50196081 0.50196081}
\pscustom[linestyle=none,fillstyle=solid,fillcolor=curcolor]
{
\newpath
\moveto(212.30338029,386.43345149)
\lineto(251.81789008,386.43345149)
\lineto(251.81789008,376.30152485)
\lineto(212.30338029,376.30152485)
\closepath
}
}
{
\newrgbcolor{curcolor}{0.50196081 0.50196081 0.50196081}
\pscustom[linestyle=none,fillstyle=solid,fillcolor=curcolor]
{
\newpath
\moveto(212.30338029,365.01495602)
\lineto(251.81789008,365.01495602)
\lineto(251.81789008,354.88302937)
\lineto(212.30338029,354.88302937)
\closepath
}
}
{
\newrgbcolor{curcolor}{0.50196081 0.50196081 0.50196081}
\pscustom[linestyle=none,fillstyle=solid,fillcolor=curcolor]
{
\newpath
\moveto(212.30338029,343.59646054)
\lineto(251.81789008,343.59646054)
\lineto(251.81789008,333.46453389)
\lineto(212.30338029,333.46453389)
\closepath
}
}
{
\newrgbcolor{curcolor}{0.50196081 0.50196081 0.50196081}
\pscustom[linestyle=none,fillstyle=solid,fillcolor=curcolor]
{
\newpath
\moveto(212.30338029,322.17796506)
\lineto(251.81789008,322.17796506)
\lineto(251.81789008,312.04603841)
\lineto(212.30338029,312.04603841)
\closepath
}
}
{
\newrgbcolor{curcolor}{0.50196081 0.50196081 0.50196081}
\pscustom[linestyle=none,fillstyle=solid,fillcolor=curcolor]
{
\newpath
\moveto(212.30338029,300.75946958)
\lineto(251.81789008,300.75946958)
\lineto(251.81789008,290.62754293)
\lineto(212.30338029,290.62754293)
\closepath
}
}
{
\newrgbcolor{curcolor}{0.50196081 0.50196081 0.50196081}
\pscustom[linestyle=none,fillstyle=solid,fillcolor=curcolor]
{
\newpath
\moveto(212.30338029,429.27038478)
\lineto(251.81789008,429.27038478)
\lineto(251.81789008,419.13845813)
\lineto(212.30338029,419.13845813)
\closepath
}
}
{
\newrgbcolor{curcolor}{0.50196081 0.50196081 0.50196081}
\pscustom[linestyle=none,fillstyle=solid,fillcolor=curcolor]
{
\newpath
\moveto(212.30338029,407.85194697)
\lineto(251.81789008,407.85194697)
\lineto(251.81789008,397.72002032)
\lineto(212.30338029,397.72002032)
\closepath
}
}
{
\newrgbcolor{curcolor}{0 0 0}
\pscustom[linestyle=none,fillstyle=solid,fillcolor=curcolor]
{
\newpath
\moveto(128.66624002,810.67213965)
\curveto(128.66624002,809.63924891)(128.29912826,808.84280304)(127.56490472,808.28280203)
\curveto(126.84312565,807.73524549)(125.76045704,807.46146723)(124.31689889,807.46146723)
\curveto(123.60756429,807.46146723)(122.99778542,807.51124509)(122.48756228,807.61080083)
\curveto(121.97733915,807.69791209)(121.46711601,807.8472457)(120.95689287,808.05880163)
\lineto(120.95689287,810.35480575)
\curveto(121.50444941,810.10591641)(122.09556158,809.90058271)(122.73022938,809.73880464)
\curveto(123.36489719,809.57702657)(123.92489819,809.49613754)(124.4102324,809.49613754)
\curveto(124.94534447,809.49613754)(125.33112294,809.57702657)(125.5675678,809.73880464)
\curveto(125.80401267,809.90058271)(125.92223511,810.11213864)(125.92223511,810.37347245)
\curveto(125.92223511,810.54769498)(125.87245724,810.70325082)(125.7729015,810.84013995)
\curveto(125.68579024,810.97702908)(125.48667877,811.13258492)(125.1755671,811.30680745)
\curveto(124.86445543,811.48102999)(124.37912123,811.70503039)(123.71956449,811.97880866)
\curveto(123.07245222,812.25258693)(122.54356238,812.52014296)(122.13289498,812.78147676)
\curveto(121.73467204,813.05525503)(121.43600484,813.37881117)(121.23689337,813.75214517)
\curveto(121.03778191,814.13792364)(120.93822617,814.61703561)(120.93822617,815.18948108)
\curveto(120.93822617,816.13526055)(121.30533794,816.84459516)(122.03956148,817.31748489)
\curveto(122.77378502,817.79037463)(123.75067566,818.0268195)(124.9702334,818.0268195)
\curveto(125.6049012,818.0268195)(126.20845784,817.96459716)(126.78090331,817.8401525)
\curveto(127.35334878,817.71570783)(127.94446095,817.51037413)(128.55423982,817.22415139)
\lineto(127.71423832,815.22681448)
\curveto(127.21645965,815.43837041)(126.74356991,815.61259295)(126.29556911,815.74948208)
\curveto(125.84756831,815.89881568)(125.39334527,815.97348248)(124.9329,815.97348248)
\curveto(124.11156519,815.97348248)(123.70089779,815.74948208)(123.70089779,815.30148128)
\curveto(123.70089779,815.13970321)(123.75067566,814.99036961)(123.85023139,814.85348048)
\curveto(123.96223159,814.72903581)(124.16756529,814.59214668)(124.4662325,814.44281307)
\curveto(124.77734416,814.29347947)(125.2315672,814.09436801)(125.82890161,813.84547867)
\curveto(126.41379154,813.6090338)(126.91779245,813.36014447)(127.34090432,813.09881066)
\curveto(127.76401619,812.84992133)(128.08757232,812.53258743)(128.31157272,812.14680896)
\curveto(128.54801759,811.76103049)(128.66624002,811.26947405)(128.66624002,810.67213965)
\closepath
}
}
{
\newrgbcolor{curcolor}{0 0 0}
\pscustom[linestyle=none,fillstyle=solid,fillcolor=curcolor]
{
\newpath
\moveto(133.61291114,821.83482632)
\lineto(133.61291114,818.9414878)
\curveto(133.61291114,818.43126467)(133.59424444,817.94593046)(133.55691104,817.48548519)
\curveto(133.5320221,817.03748439)(133.50713317,816.72015049)(133.48224423,816.53348349)
\lineto(133.63157784,816.53348349)
\curveto(133.95513397,817.05615109)(134.37202361,817.43570733)(134.88224674,817.6721522)
\curveto(135.39246988,817.90859706)(135.95869312,818.0268195)(136.58091646,818.0268195)
\curveto(137.67602953,818.0268195)(138.55958667,817.7281523)(139.23158787,817.13081789)
\curveto(139.90358908,816.54592796)(140.23958968,815.60014848)(140.23958968,814.29347947)
\lineto(140.23958968,807.64813423)
\lineto(137.45825136,807.64813423)
\lineto(137.45825136,813.60281157)
\curveto(137.45825136,815.07125865)(136.91069482,815.80548218)(135.81558175,815.80548218)
\curveto(134.98180248,815.80548218)(134.40313477,815.51303721)(134.07957864,814.92814728)
\curveto(133.76846697,814.35570181)(133.61291114,813.52814477)(133.61291114,812.44547616)
\lineto(133.61291114,807.64813423)
\lineto(130.83157282,807.64813423)
\lineto(130.83157282,821.83482632)
\closepath
}
}
{
\newrgbcolor{curcolor}{0 0 0}
\pscustom[linestyle=none,fillstyle=solid,fillcolor=curcolor]
{
\newpath
\moveto(147.29559348,818.0268195)
\curveto(148.70181822,818.0268195)(149.815598,817.62237433)(150.6369328,816.81348399)
\curveto(151.45826761,816.01703812)(151.86893501,814.87836941)(151.86893501,813.39747787)
\lineto(151.86893501,812.05347546)
\lineto(145.29825657,812.05347546)
\curveto(145.3231455,811.26947405)(145.55336814,810.65347295)(145.98892447,810.20547215)
\curveto(146.43692528,809.75747134)(147.05292638,809.53347094)(147.83692778,809.53347094)
\curveto(148.48404006,809.53347094)(149.07515223,809.59569327)(149.6102643,809.72013794)
\curveto(150.15782083,809.85702708)(150.71782184,810.06236078)(151.29026731,810.33613905)
\lineto(151.29026731,808.18946853)
\curveto(150.78004417,807.9405792)(150.25115433,807.76013443)(149.7035978,807.64813423)
\curveto(149.15604126,807.52368956)(148.49026229,807.46146723)(147.70626088,807.46146723)
\curveto(146.68581461,807.46146723)(145.78359077,807.64813423)(144.99958937,808.02146823)
\curveto(144.21558796,808.4072467)(143.59958686,808.97969217)(143.15158605,809.73880464)
\curveto(142.70358525,810.51036158)(142.47958485,811.48725222)(142.47958485,812.66947656)
\curveto(142.47958485,813.8517009)(142.67869632,814.84103601)(143.07691925,815.63748188)
\curveto(143.48758665,816.43392775)(144.05380989,817.03126216)(144.77558896,817.42948509)
\curveto(145.49736804,817.82770803)(146.33736954,818.0268195)(147.29559348,818.0268195)
\closepath
\moveto(147.31426018,816.04814929)
\curveto(146.76670364,816.04814929)(146.31870284,815.87392675)(145.97025777,815.52548168)
\curveto(145.6218127,815.17703661)(145.416479,814.63570231)(145.35425667,813.90147877)
\lineto(149.25559699,813.90147877)
\curveto(149.24315253,814.51125764)(149.07515223,815.02148078)(148.75159609,815.43214818)
\curveto(148.44048442,815.84281558)(147.96137245,816.04814929)(147.31426018,816.04814929)
\closepath
}
}
{
\newrgbcolor{curcolor}{0 0 0}
\pscustom[linestyle=none,fillstyle=solid,fillcolor=curcolor]
{
\newpath
\moveto(156.90893707,807.64813423)
\lineto(154.12759875,807.64813423)
\lineto(154.12759875,821.83482632)
\lineto(156.90893707,821.83482632)
\closepath
}
}
{
\newrgbcolor{curcolor}{0 0 0}
\pscustom[linestyle=none,fillstyle=solid,fillcolor=curcolor]
{
\newpath
\moveto(162.60228875,807.64813423)
\lineto(159.82095043,807.64813423)
\lineto(159.82095043,821.83482632)
\lineto(162.60228875,821.83482632)
\closepath
}
}
{
\newrgbcolor{curcolor}{0 0 0}
\pscustom[linestyle=none,fillstyle=solid,fillcolor=curcolor]
{
\newpath
\moveto(172.33811093,784.17206555)
\curveto(173.48300187,784.17206555)(174.41011464,783.72406475)(175.11944925,782.82806314)
\curveto(175.82878385,781.94450601)(176.18345115,780.637837)(176.18345115,778.90805612)
\curveto(176.18345115,777.16583077)(175.81633938,775.8467173)(175.08211584,774.95071569)
\curveto(174.34789231,774.05471408)(173.40833507,773.60671328)(172.26344413,773.60671328)
\curveto(171.52922059,773.60671328)(170.94433065,773.73738018)(170.50877431,773.99871398)
\curveto(170.07321798,774.27249225)(169.71855067,774.57738169)(169.44477241,774.91338229)
\lineto(169.29543881,774.91338229)
\curveto(169.39499454,774.39071469)(169.44477241,773.89293602)(169.44477241,773.42004628)
\lineto(169.44477241,769.31337225)
\lineto(166.66343409,769.31337225)
\lineto(166.66343409,783.98539855)
\lineto(168.9221048,783.98539855)
\lineto(169.31410551,782.66006284)
\lineto(169.44477241,782.66006284)
\curveto(169.71855067,783.07073025)(170.08566244,783.42539755)(170.54610771,783.72406475)
\curveto(171.00655298,784.02273195)(171.60388739,784.17206555)(172.33811093,784.17206555)
\closepath
\moveto(171.44210932,781.95072824)
\curveto(170.72033025,781.95072824)(170.21010711,781.7205056)(169.91143991,781.26006033)
\curveto(169.61277271,780.81205953)(169.45721687,780.13383609)(169.44477241,779.22539002)
\lineto(169.44477241,778.92672282)
\curveto(169.44477241,777.94360994)(169.58788377,777.18449747)(169.87410651,776.6493854)
\curveto(170.17277371,776.1267178)(170.70788578,775.865384)(171.47944272,775.865384)
\curveto(172.11411052,775.865384)(172.58077803,776.1267178)(172.87944523,776.6493854)
\curveto(173.1905569,777.18449747)(173.34611273,777.94983218)(173.34611273,778.94538952)
\curveto(173.34611273,780.94894866)(172.71144493,781.95072824)(171.44210932,781.95072824)
\closepath
}
}
{
\newrgbcolor{curcolor}{0 0 0}
\pscustom[linestyle=none,fillstyle=solid,fillcolor=curcolor]
{
\newpath
\moveto(182.66078524,784.19073225)
\curveto(184.02967658,784.19073225)(185.07501179,783.89206505)(185.79679086,783.29473065)
\curveto(186.5310144,782.70984071)(186.89812617,781.80761687)(186.89812617,780.58805913)
\lineto(186.89812617,773.79338028)
\lineto(184.95678936,773.79338028)
\lineto(184.41545505,775.17471609)
\lineto(184.34078825,775.17471609)
\curveto(183.90523192,774.62715955)(183.44478665,774.22893662)(182.95945244,773.98004728)
\curveto(182.47411824,773.73115795)(181.80833927,773.60671328)(180.96211553,773.60671328)
\curveto(180.05366946,773.60671328)(179.30077922,773.86804708)(178.70344482,774.39071469)
\curveto(178.10611041,774.91338229)(177.80744321,775.72849486)(177.80744321,776.8360524)
\curveto(177.80744321,777.91872101)(178.18699944,778.71516688)(178.94611192,779.22539002)
\curveto(179.70522439,779.73561316)(180.8438931,780.02183589)(182.36211804,780.08405823)
\lineto(184.13545455,780.14005833)
\lineto(184.13545455,780.58805913)
\curveto(184.13545455,781.1231712)(183.99234318,781.5151719)(183.70612045,781.76406124)
\curveto(183.43234218,782.01295057)(183.04656371,782.13739524)(182.54878504,782.13739524)
\curveto(182.05100637,782.13739524)(181.56567217,782.06272844)(181.09278243,781.91339484)
\curveto(180.61989269,781.7765057)(180.14700296,781.60228317)(179.67411322,781.39072723)
\lineto(178.75944492,783.27606395)
\curveto(179.29455699,783.54984222)(179.89811362,783.76762038)(180.57011483,783.92939845)
\curveto(181.24211603,784.10362099)(181.93900617,784.19073225)(182.66078524,784.19073225)
\closepath
\moveto(184.13545455,778.51605541)
\lineto(183.05278594,778.47872201)
\curveto(182.15678434,778.45383308)(181.534561,778.29205501)(181.18611593,777.99338781)
\curveto(180.83767086,777.69472061)(180.66344833,777.30271991)(180.66344833,776.8173857)
\curveto(180.66344833,776.39427383)(180.787893,776.0893844)(181.03678233,775.9027174)
\curveto(181.28567167,775.72849486)(181.6092278,775.6413836)(182.00745074,775.6413836)
\curveto(182.60478514,775.6413836)(183.10878605,775.81560613)(183.51945345,776.1640512)
\curveto(183.93012085,776.52494073)(184.13545455,777.02894164)(184.13545455,777.67605391)
\closepath
}
}
{
\newrgbcolor{curcolor}{0 0 0}
\pscustom[linestyle=none,fillstyle=solid,fillcolor=curcolor]
{
\newpath
\moveto(193.05813796,784.17206555)
\curveto(194.3150291,784.17206555)(195.29814198,783.67428688)(196.00747658,782.67872954)
\lineto(196.08214338,782.67872954)
\lineto(196.30614378,783.98539855)
\lineto(198.658148,783.98539855)
\lineto(198.658148,773.77471358)
\curveto(198.658148,772.31871097)(198.2288139,771.21115343)(197.37014569,770.45204096)
\curveto(196.51147748,769.69292849)(195.24214188,769.31337225)(193.56213886,769.31337225)
\curveto(192.84035979,769.31337225)(192.16835859,769.35692789)(191.54613525,769.44403915)
\curveto(190.93635638,769.53115042)(190.33902198,769.68670625)(189.75413204,769.91070666)
\lineto(189.75413204,772.13204397)
\curveto(190.99857871,771.60937637)(192.32391442,771.34804257)(193.73013917,771.34804257)
\curveto(195.16125284,771.34804257)(195.87680968,772.1195995)(195.87680968,773.66271338)
\lineto(195.87680968,773.86804708)
\curveto(195.87680968,774.06715855)(195.88303191,774.27871449)(195.89547638,774.50271489)
\curveto(195.90792085,774.73915976)(195.92658755,774.94449346)(195.95147648,775.11871599)
\lineto(195.87680968,775.11871599)
\curveto(195.52836461,774.58360392)(195.11147498,774.19782545)(194.62614077,773.96138058)
\curveto(194.14080657,773.72493572)(193.59325003,773.60671328)(192.98347116,773.60671328)
\curveto(191.77635789,773.60671328)(190.83057841,774.06715855)(190.14613274,774.98804909)
\curveto(189.47413154,775.9213841)(189.13813093,777.21560864)(189.13813093,778.87072272)
\curveto(189.13813093,780.53828126)(189.486576,781.83872804)(190.18346614,782.77206304)
\curveto(190.88035628,783.70539805)(191.83858022,784.17206555)(193.05813796,784.17206555)
\closepath
\moveto(193.93547287,781.91339484)
\curveto(192.62880386,781.91339484)(191.97546935,780.88672633)(191.97546935,778.83338932)
\curveto(191.97546935,776.80494124)(192.64124832,775.7907172)(193.97280627,775.7907172)
\curveto(194.68214087,775.7907172)(195.20480848,775.98982866)(195.54080908,776.3880516)
\curveto(195.88925415,776.798719)(196.06347668,777.50805361)(196.06347668,778.51605541)
\lineto(196.06347668,778.85205602)
\curveto(196.06347668,779.94716909)(195.89547638,780.7311705)(195.55947578,781.20406023)
\curveto(195.22347518,781.67694997)(194.68214087,781.91339484)(193.93547287,781.91339484)
\closepath
}
}
{
\newrgbcolor{curcolor}{0 0 0}
\pscustom[linestyle=none,fillstyle=solid,fillcolor=curcolor]
{
\newpath
\moveto(205.77014894,784.17206555)
\curveto(207.17637368,784.17206555)(208.29015346,783.76762038)(209.11148826,782.95873005)
\curveto(209.93282307,782.16228417)(210.34349047,781.02361547)(210.34349047,779.54272392)
\lineto(210.34349047,778.19872151)
\lineto(203.77281203,778.19872151)
\curveto(203.79770096,777.41472011)(204.02792359,776.798719)(204.46347993,776.3507182)
\curveto(204.91148073,775.9027174)(205.52748184,775.678717)(206.31148324,775.678717)
\curveto(206.95859551,775.678717)(207.54970769,775.74093933)(208.08481976,775.865384)
\curveto(208.63237629,776.00227313)(209.1923773,776.20760683)(209.76482277,776.4813851)
\lineto(209.76482277,774.33471459)
\curveto(209.25459963,774.08582525)(208.72570979,773.90538048)(208.17815326,773.79338028)
\curveto(207.63059672,773.66893562)(206.96481775,773.60671328)(206.18081634,773.60671328)
\curveto(205.16037007,773.60671328)(204.25814623,773.79338028)(203.47414482,774.16671429)
\curveto(202.69014342,774.55249275)(202.07414231,775.12493823)(201.62614151,775.8840507)
\curveto(201.17814071,776.65560764)(200.95414031,777.63249828)(200.95414031,778.81472262)
\curveto(200.95414031,779.99694696)(201.15325177,780.98628207)(201.55147471,781.78272794)
\curveto(201.96214211,782.57917381)(202.52836535,783.17650821)(203.25014442,783.57473115)
\curveto(203.97192349,783.97295409)(204.811925,784.17206555)(205.77014894,784.17206555)
\closepath
\moveto(205.78881564,782.19339534)
\curveto(205.2412591,782.19339534)(204.7932583,782.01917281)(204.44481323,781.67072774)
\curveto(204.09636816,781.32228267)(203.89103446,780.78094836)(203.82881213,780.04672483)
\lineto(207.73015245,780.04672483)
\curveto(207.71770799,780.6565037)(207.54970769,781.16672683)(207.22615155,781.57739424)
\curveto(206.91503988,781.98806164)(206.43592791,782.19339534)(205.78881564,782.19339534)
\closepath
}
}
{
\newrgbcolor{curcolor}{0 0 0}
\pscustom[linestyle=none,fillstyle=solid,fillcolor=curcolor]
{
\newpath
\moveto(219.71416696,776.8173857)
\curveto(219.71416696,775.78449496)(219.34705519,774.98804909)(218.61283165,774.42804809)
\curveto(217.89105258,773.88049155)(216.80838397,773.60671328)(215.36482583,773.60671328)
\curveto(214.65549123,773.60671328)(214.04571235,773.65649115)(213.53548922,773.75604688)
\curveto(213.02526608,773.84315815)(212.51504294,773.99249175)(212.00481981,774.20404769)
\lineto(212.00481981,776.5000518)
\curveto(212.55237634,776.25116247)(213.14348852,776.04582876)(213.77815632,775.8840507)
\curveto(214.41282412,775.72227263)(214.97282513,775.6413836)(215.45815933,775.6413836)
\curveto(215.9932714,775.6413836)(216.37904987,775.72227263)(216.61549474,775.8840507)
\curveto(216.85193961,776.04582876)(216.97016204,776.2573847)(216.97016204,776.5187185)
\curveto(216.97016204,776.69294104)(216.92038417,776.84849687)(216.82082844,776.985386)
\curveto(216.73371717,777.12227514)(216.53460571,777.27783097)(216.22349404,777.45205351)
\curveto(215.91238237,777.62627604)(215.42704816,777.85027644)(214.76749143,778.12405471)
\curveto(214.12037916,778.39783298)(213.59148932,778.66538902)(213.18082192,778.92672282)
\curveto(212.78259898,779.20050109)(212.48393178,779.52405722)(212.28482031,779.89739122)
\curveto(212.08570884,780.28316969)(211.98615311,780.76228166)(211.98615311,781.33472713)
\curveto(211.98615311,782.28050661)(212.35326488,782.98984121)(213.08748841,783.46273095)
\curveto(213.82171195,783.93562069)(214.79860259,784.17206555)(216.01816033,784.17206555)
\curveto(216.65282814,784.17206555)(217.25638478,784.10984322)(217.82883025,783.98539855)
\curveto(218.40127572,783.86095388)(218.99238789,783.65562018)(219.60216676,783.36939745)
\lineto(218.76216525,781.37206053)
\curveto(218.26438658,781.58361647)(217.79149685,781.757839)(217.34349604,781.89472814)
\curveto(216.89549524,782.04406174)(216.4412722,782.11872854)(215.98082693,782.11872854)
\curveto(215.15949213,782.11872854)(214.74882473,781.89472814)(214.74882473,781.44672734)
\curveto(214.74882473,781.28494927)(214.79860259,781.13561567)(214.89815833,780.99872653)
\curveto(215.01015853,780.87428186)(215.21549223,780.73739273)(215.51415943,780.58805913)
\curveto(215.8252711,780.43872553)(216.27949414,780.23961406)(216.87682854,779.99072473)
\curveto(217.46171848,779.75427986)(217.96571938,779.50539052)(218.38883125,779.24405672)
\curveto(218.81194312,778.99516738)(219.13549926,778.67783348)(219.35949966,778.29205501)
\curveto(219.59594453,777.90627654)(219.71416696,777.41472011)(219.71416696,776.8173857)
\closepath
}
}
{
\newrgbcolor{curcolor}{0 0 0}
\pscustom[linestyle=none,fillstyle=solid,fillcolor=curcolor]
{
\newpath
\moveto(174.71624307,588.31823848)
\curveto(174.71624307,587.28534774)(174.3491313,586.48890187)(173.61490776,585.92890086)
\curveto(172.89312869,585.38134433)(171.81046008,585.10756606)(170.36690194,585.10756606)
\curveto(169.65756734,585.10756606)(169.04778846,585.15734392)(168.53756533,585.25689966)
\curveto(168.02734219,585.34401093)(167.51711905,585.49334453)(167.00689592,585.70490046)
\lineto(167.00689592,588.00090458)
\curveto(167.55445245,587.75201524)(168.14556463,587.54668154)(168.78023243,587.38490347)
\curveto(169.41490023,587.22312541)(169.97490124,587.14223637)(170.46023544,587.14223637)
\curveto(170.99534751,587.14223637)(171.38112598,587.22312541)(171.61757085,587.38490347)
\curveto(171.85401572,587.54668154)(171.97223815,587.75823748)(171.97223815,588.01957128)
\curveto(171.97223815,588.19379381)(171.92246028,588.34934965)(171.82290455,588.48623878)
\curveto(171.73579328,588.62312792)(171.53668182,588.77868375)(171.22557015,588.95290628)
\curveto(170.91445848,589.12712882)(170.42912427,589.35112922)(169.76956754,589.62490749)
\curveto(169.12245527,589.89868576)(168.59356543,590.16624179)(168.18289803,590.42757559)
\curveto(167.78467509,590.70135386)(167.48600789,591.02491)(167.28689642,591.398244)
\curveto(167.08778495,591.78402247)(166.98822922,592.26313444)(166.98822922,592.83557991)
\curveto(166.98822922,593.78135938)(167.35534099,594.49069399)(168.08956452,594.96358373)
\curveto(168.82378806,595.43647346)(169.8006787,595.67291833)(171.02023645,595.67291833)
\curveto(171.65490425,595.67291833)(172.25846089,595.610696)(172.83090636,595.48625133)
\curveto(173.40335183,595.36180666)(173.994464,595.15647296)(174.60424287,594.87025022)
\lineto(173.76424136,592.87291331)
\curveto(173.26646269,593.08446925)(172.79357296,593.25869178)(172.34557215,593.39558091)
\curveto(171.89757135,593.54491452)(171.44334831,593.61958132)(170.98290304,593.61958132)
\curveto(170.16156824,593.61958132)(169.75090084,593.39558091)(169.75090084,592.94758011)
\curveto(169.75090084,592.78580204)(169.8006787,592.63646844)(169.90023444,592.49957931)
\curveto(170.01223464,592.37513464)(170.21756834,592.23824551)(170.51623554,592.08891191)
\curveto(170.82734721,591.9395783)(171.28157025,591.74046684)(171.87890465,591.4915775)
\curveto(172.46379459,591.25513263)(172.96779549,591.0062433)(173.39090736,590.7449095)
\curveto(173.81401923,590.49602016)(174.13757537,590.17868626)(174.36157577,589.79290779)
\curveto(174.59802064,589.40712932)(174.71624307,588.91557288)(174.71624307,588.31823848)
\closepath
}
}
{
\newrgbcolor{curcolor}{0 0 0}
\pscustom[linestyle=none,fillstyle=solid,fillcolor=curcolor]
{
\newpath
\moveto(181.02558329,585.10756606)
\curveto(179.50735835,585.10756606)(178.33135624,585.52445569)(177.49757697,586.35823497)
\curveto(176.67624216,587.19201424)(176.26557476,588.51734995)(176.26557476,590.33424209)
\curveto(176.26557476,591.57868877)(176.47713069,592.59291281)(176.90024256,593.37691421)
\curveto(177.32335443,594.16091562)(177.90824437,594.73958332)(178.65491237,595.11291733)
\curveto(179.41402485,595.48625133)(180.28513752,595.67291833)(181.26825039,595.67291833)
\curveto(181.96514053,595.67291833)(182.56869717,595.60447376)(183.0789203,595.46758463)
\curveto(183.60158791,595.33069549)(184.05581094,595.16891743)(184.44158941,594.98225043)
\lineto(183.62025461,592.83557991)
\curveto(183.18469827,593.00980245)(182.77403087,593.15291381)(182.3882524,593.26491401)
\curveto(182.0149184,593.37691421)(181.64158439,593.43291431)(181.26825039,593.43291431)
\curveto(179.82469225,593.43291431)(179.10291318,592.40624581)(179.10291318,590.35290879)
\curveto(179.10291318,589.33246252)(179.28958018,588.57957228)(179.66291418,588.09423808)
\curveto(180.04869265,587.60890387)(180.58380472,587.36623677)(181.26825039,587.36623677)
\curveto(181.85314033,587.36623677)(182.3695857,587.44090357)(182.8175865,587.59023717)
\curveto(183.26558731,587.75201524)(183.70114364,587.96979341)(184.12425551,588.24357168)
\lineto(184.12425551,585.87290076)
\curveto(183.70114364,585.59912249)(183.25314284,585.40623326)(182.7802531,585.29423306)
\curveto(182.31980783,585.16978839)(181.7349179,585.10756606)(181.02558329,585.10756606)
\closepath
}
}
{
\newrgbcolor{curcolor}{0 0 0}
\pscustom[linestyle=none,fillstyle=solid,fillcolor=curcolor]
{
\newpath
\moveto(192.16960197,595.67291833)
\curveto(192.30649111,595.67291833)(192.46826918,595.6666961)(192.65493618,595.65425163)
\curveto(192.84160318,595.64180716)(192.99093678,595.62314046)(193.10293698,595.59825153)
\lineto(192.89760328,592.98491351)
\curveto(192.79804754,593.00980245)(192.66738064,593.02846915)(192.50560258,593.04091361)
\curveto(192.34382451,593.06580255)(192.20071314,593.07824701)(192.07626847,593.07824701)
\curveto(191.60337874,593.07824701)(191.1491557,592.99113574)(190.71359936,592.81691321)
\curveto(190.27804303,592.65513514)(189.92337573,592.38757911)(189.64959746,592.0142451)
\curveto(189.38826365,591.6409111)(189.25759675,591.13068797)(189.25759675,590.48357569)
\lineto(189.25759675,585.29423306)
\lineto(186.47625843,585.29423306)
\lineto(186.47625843,595.48625133)
\lineto(188.58559555,595.48625133)
\lineto(188.99626295,593.76891492)
\lineto(189.12692985,593.76891492)
\curveto(189.42559705,594.29158252)(189.83626446,594.73958332)(190.35893206,595.11291733)
\curveto(190.88159966,595.48625133)(191.4851563,595.67291833)(192.16960197,595.67291833)
\closepath
}
}
{
\newrgbcolor{curcolor}{0 0 0}
\pscustom[linestyle=none,fillstyle=solid,fillcolor=curcolor]
{
\newpath
\moveto(196.35095032,599.48092516)
\curveto(196.76161772,599.48092516)(197.11628503,599.38136942)(197.41495223,599.18225795)
\curveto(197.71361943,598.99559095)(197.86295303,598.64092365)(197.86295303,598.11825605)
\curveto(197.86295303,597.60803291)(197.71361943,597.25336561)(197.41495223,597.05425414)
\curveto(197.11628503,596.85514267)(196.76161772,596.75558694)(196.35095032,596.75558694)
\curveto(195.92783845,596.75558694)(195.56694892,596.85514267)(195.26828171,597.05425414)
\curveto(194.98205898,597.25336561)(194.83894761,597.60803291)(194.83894761,598.11825605)
\curveto(194.83894761,598.64092365)(194.98205898,598.99559095)(195.26828171,599.18225795)
\curveto(195.56694892,599.38136942)(195.92783845,599.48092516)(196.35095032,599.48092516)
\closepath
\moveto(197.73228613,595.48625133)
\lineto(197.73228613,585.29423306)
\lineto(194.95094781,585.29423306)
\lineto(194.95094781,595.48625133)
\closepath
}
}
{
\newrgbcolor{curcolor}{0 0 0}
\pscustom[linestyle=none,fillstyle=solid,fillcolor=curcolor]
{
\newpath
\moveto(206.31897632,595.67291833)
\curveto(207.46386727,595.67291833)(208.39098004,595.22491753)(209.10031464,594.32891592)
\curveto(209.80964925,593.44535878)(210.16431655,592.13868977)(210.16431655,590.40890889)
\curveto(210.16431655,588.66668355)(209.79720478,587.34757007)(209.06298124,586.45156847)
\curveto(208.3287577,585.55556686)(207.38920047,585.10756606)(206.24430952,585.10756606)
\curveto(205.51008599,585.10756606)(204.92519605,585.23823296)(204.48963971,585.49956676)
\curveto(204.05408338,585.77334503)(203.69941607,586.07823446)(203.4256378,586.41423507)
\lineto(203.2763042,586.41423507)
\curveto(203.37585994,585.89156746)(203.4256378,585.39378879)(203.4256378,584.92089906)
\lineto(203.4256378,580.81422503)
\lineto(200.64429949,580.81422503)
\lineto(200.64429949,595.48625133)
\lineto(202.9029702,595.48625133)
\lineto(203.2949709,594.16091562)
\lineto(203.4256378,594.16091562)
\curveto(203.69941607,594.57158302)(204.06652784,594.92625032)(204.52697311,595.22491753)
\curveto(204.98741838,595.52358473)(205.58475279,595.67291833)(206.31897632,595.67291833)
\closepath
\moveto(205.42297472,593.45158101)
\curveto(204.70119565,593.45158101)(204.19097251,593.22135838)(203.89230531,592.76091311)
\curveto(203.59363811,592.31291231)(203.43808227,591.63468887)(203.4256378,590.7262428)
\lineto(203.4256378,590.42757559)
\curveto(203.4256378,589.44446272)(203.56874917,588.68535025)(203.85497191,588.15023818)
\curveto(204.15363911,587.62757057)(204.68875118,587.36623677)(205.46030812,587.36623677)
\curveto(206.09497592,587.36623677)(206.56164343,587.62757057)(206.86031063,588.15023818)
\curveto(207.1714223,588.68535025)(207.32697813,589.45068495)(207.32697813,590.44624229)
\curveto(207.32697813,592.44980144)(206.69231033,593.45158101)(205.42297472,593.45158101)
\closepath
}
}
{
\newrgbcolor{curcolor}{0 0 0}
\pscustom[linestyle=none,fillstyle=solid,fillcolor=curcolor]
{
\newpath
\moveto(216.75364989,587.32890337)
\curveto(217.06476156,587.32890337)(217.36342876,587.35379231)(217.64965149,587.40357017)
\curveto(217.93587423,587.46579251)(218.22209696,587.54668154)(218.5083197,587.64623727)
\lineto(218.5083197,585.57423356)
\curveto(218.2096525,585.43734443)(217.83631849,585.32534423)(217.38831769,585.23823296)
\curveto(216.95276136,585.15112169)(216.47364939,585.10756606)(215.95098178,585.10756606)
\curveto(215.34120291,585.10756606)(214.79364637,585.20712179)(214.30831217,585.40623326)
\curveto(213.83542243,585.60534473)(213.4558662,585.94756756)(213.16964346,586.43290177)
\curveto(212.89586519,586.91823597)(212.75897606,587.60268164)(212.75897606,588.48623878)
\lineto(212.75897606,593.39558091)
\lineto(211.43364035,593.39558091)
\lineto(211.43364035,594.57158302)
\lineto(212.96430976,595.50491803)
\lineto(213.76697787,597.65158854)
\lineto(215.54031438,597.65158854)
\lineto(215.54031438,595.48625133)
\lineto(218.3963195,595.48625133)
\lineto(218.3963195,593.39558091)
\lineto(215.54031438,593.39558091)
\lineto(215.54031438,588.48623878)
\curveto(215.54031438,588.10046031)(215.65231458,587.80801534)(215.87631498,587.60890387)
\curveto(216.10031538,587.42223687)(216.39276035,587.32890337)(216.75364989,587.32890337)
\closepath
}
}
{
\newrgbcolor{curcolor}{0 0 0}
\pscustom[linestyle=none,fillstyle=solid,fillcolor=curcolor]
{
\newpath
\moveto(227.67367038,588.31823848)
\curveto(227.67367038,587.28534774)(227.30655861,586.48890187)(226.57233507,585.92890086)
\curveto(225.850556,585.38134433)(224.7678874,585.10756606)(223.32432925,585.10756606)
\curveto(222.61499465,585.10756606)(222.00521578,585.15734392)(221.49499264,585.25689966)
\curveto(220.9847695,585.34401093)(220.47454637,585.49334453)(219.96432323,585.70490046)
\lineto(219.96432323,588.00090458)
\curveto(220.51187977,587.75201524)(221.10299194,587.54668154)(221.73765974,587.38490347)
\curveto(222.37232755,587.22312541)(222.93232855,587.14223637)(223.41766275,587.14223637)
\curveto(223.95277482,587.14223637)(224.33855329,587.22312541)(224.57499816,587.38490347)
\curveto(224.81144303,587.54668154)(224.92966546,587.75823748)(224.92966546,588.01957128)
\curveto(224.92966546,588.19379381)(224.8798876,588.34934965)(224.78033186,588.48623878)
\curveto(224.69322059,588.62312792)(224.49410913,588.77868375)(224.18299746,588.95290628)
\curveto(223.87188579,589.12712882)(223.38655159,589.35112922)(222.72699485,589.62490749)
\curveto(222.07988258,589.89868576)(221.55099274,590.16624179)(221.14032534,590.42757559)
\curveto(220.7421024,590.70135386)(220.4434352,591.02491)(220.24432373,591.398244)
\curveto(220.04521226,591.78402247)(219.94565653,592.26313444)(219.94565653,592.83557991)
\curveto(219.94565653,593.78135938)(220.3127683,594.49069399)(221.04699184,594.96358373)
\curveto(221.78121537,595.43647346)(222.75810601,595.67291833)(223.97766376,595.67291833)
\curveto(224.61233156,595.67291833)(225.2158882,595.610696)(225.78833367,595.48625133)
\curveto(226.36077914,595.36180666)(226.95189131,595.15647296)(227.56167018,594.87025022)
\lineto(226.72166868,592.87291331)
\curveto(226.22389001,593.08446925)(225.75100027,593.25869178)(225.30299947,593.39558091)
\curveto(224.85499866,593.54491452)(224.40077563,593.61958132)(223.94033036,593.61958132)
\curveto(223.11899555,593.61958132)(222.70832815,593.39558091)(222.70832815,592.94758011)
\curveto(222.70832815,592.78580204)(222.75810601,592.63646844)(222.85766175,592.49957931)
\curveto(222.96966195,592.37513464)(223.17499565,592.23824551)(223.47366285,592.08891191)
\curveto(223.78477452,591.9395783)(224.23899756,591.74046684)(224.83633196,591.4915775)
\curveto(225.4212219,591.25513263)(225.9252228,591.0062433)(226.34833467,590.7449095)
\curveto(226.77144654,590.49602016)(227.09500268,590.17868626)(227.31900308,589.79290779)
\curveto(227.55544795,589.40712932)(227.67367038,588.91557288)(227.67367038,588.31823848)
\closepath
}
}
{
\newrgbcolor{curcolor}{0 0 0}
\pscustom[linestyle=none,fillstyle=solid,fillcolor=curcolor]
{
\newpath
\moveto(176.60321025,261.39186581)
\curveto(176.60321025,260.35897507)(176.23609848,259.56252919)(175.50187494,259.00252819)
\curveto(174.78009587,258.45497165)(173.69742726,258.18119338)(172.25386912,258.18119338)
\curveto(171.54453452,258.18119338)(170.93475565,258.23097125)(170.42453251,258.33052698)
\curveto(169.91430937,258.41763825)(169.40408623,258.56697185)(168.8938631,258.77852779)
\lineto(168.8938631,261.0745319)
\curveto(169.44141964,260.82564257)(170.03253181,260.62030887)(170.66719961,260.4585308)
\curveto(171.30186741,260.29675273)(171.86186842,260.2158637)(172.34720262,260.2158637)
\curveto(172.88231469,260.2158637)(173.26809316,260.29675273)(173.50453803,260.4585308)
\curveto(173.7409829,260.62030887)(173.85920533,260.8318648)(173.85920533,261.0931986)
\curveto(173.85920533,261.26742114)(173.80942747,261.42297697)(173.70987173,261.55986611)
\curveto(173.62276046,261.69675524)(173.423649,261.85231108)(173.11253733,262.02653361)
\curveto(172.80142566,262.20075614)(172.31609145,262.42475655)(171.65653472,262.69853481)
\curveto(171.00942245,262.97231308)(170.48053261,263.23986912)(170.06986521,263.50120292)
\curveto(169.67164227,263.77498119)(169.37297507,264.09853732)(169.1738636,264.47187133)
\curveto(168.97475213,264.8576498)(168.8751964,265.33676177)(168.8751964,265.90920724)
\curveto(168.8751964,266.85498671)(169.24230817,267.56432131)(169.97653171,268.03721105)
\curveto(170.71075524,268.51010079)(171.68764588,268.74654566)(172.90720363,268.74654566)
\curveto(173.54187143,268.74654566)(174.14542807,268.68432332)(174.71787354,268.55987866)
\curveto(175.29031901,268.43543399)(175.88143118,268.23010029)(176.49121005,267.94387755)
\lineto(175.65120854,265.94654064)
\curveto(175.15342987,266.15809657)(174.68054014,266.33231911)(174.23253933,266.46920824)
\curveto(173.78453853,266.61854184)(173.3303155,266.69320864)(172.86987023,266.69320864)
\curveto(172.04853542,266.69320864)(171.63786802,266.46920824)(171.63786802,266.02120744)
\curveto(171.63786802,265.85942937)(171.68764588,265.71009577)(171.78720162,265.57320663)
\curveto(171.89920182,265.44876197)(172.10453552,265.31187283)(172.40320272,265.16253923)
\curveto(172.71431439,265.01320563)(173.16853743,264.81409416)(173.76587183,264.56520483)
\curveto(174.35076177,264.32875996)(174.85476267,264.07987062)(175.27787454,263.81853682)
\curveto(175.70098641,263.56964749)(176.02454255,263.25231359)(176.24854295,262.86653512)
\curveto(176.48498782,262.48075665)(176.60321025,261.98920021)(176.60321025,261.39186581)
\closepath
}
}
{
\newrgbcolor{curcolor}{0 0 0}
\pscustom[linestyle=none,fillstyle=solid,fillcolor=curcolor]
{
\newpath
\moveto(183.06188407,260.4025307)
\curveto(183.37299574,260.4025307)(183.67166294,260.42741963)(183.95788568,260.4771975)
\curveto(184.24410841,260.53941983)(184.53033115,260.62030887)(184.81655388,260.7198646)
\lineto(184.81655388,258.64786089)
\curveto(184.51788668,258.51097175)(184.14455268,258.39897155)(183.69655188,258.31186028)
\curveto(183.26099554,258.22474902)(182.78188357,258.18119338)(182.25921597,258.18119338)
\curveto(181.6494371,258.18119338)(181.10188056,258.28074912)(180.61654636,258.47986059)
\curveto(180.14365662,258.67897205)(179.76410038,259.02119489)(179.47787765,259.50652909)
\curveto(179.20409938,259.9918633)(179.06721024,260.67630897)(179.06721024,261.55986611)
\lineto(179.06721024,266.46920824)
\lineto(177.74187454,266.46920824)
\lineto(177.74187454,267.64521035)
\lineto(179.27254395,268.57854536)
\lineto(180.07521205,270.72521587)
\lineto(181.84854856,270.72521587)
\lineto(181.84854856,268.55987866)
\lineto(184.70455368,268.55987866)
\lineto(184.70455368,266.46920824)
\lineto(181.84854856,266.46920824)
\lineto(181.84854856,261.55986611)
\curveto(181.84854856,261.17408764)(181.96054876,260.88164267)(182.18454917,260.6825312)
\curveto(182.40854957,260.4958642)(182.70099454,260.4025307)(183.06188407,260.4025307)
\closepath
}
}
{
\newrgbcolor{curcolor}{0 0 0}
\pscustom[linestyle=none,fillstyle=solid,fillcolor=curcolor]
{
\newpath
\moveto(185.4138873,268.55987866)
\lineto(188.45655942,268.55987866)
\lineto(190.37922953,262.82920172)
\curveto(190.47878527,262.54297898)(190.55345207,262.25675625)(190.60322994,261.97053351)
\curveto(190.6530078,261.68431077)(190.6903412,261.37942134)(190.71523014,261.0558652)
\lineto(190.77123024,261.0558652)
\curveto(190.80856364,261.37942134)(190.8583415,261.68431077)(190.92056384,261.97053351)
\curveto(190.98278617,262.25675625)(191.06367521,262.54297898)(191.16323094,262.82920172)
\lineto(193.04856765,268.55987866)
\lineto(196.03523967,268.55987866)
\lineto(191.72323194,257.06119138)
\curveto(191.32500901,256.0034117)(190.75878577,255.21318806)(190.02456223,254.69052046)
\curveto(189.29033869,254.15540839)(188.43789272,253.88785235)(187.46722431,253.88785235)
\curveto(187.14366818,253.88785235)(186.86988991,253.90651905)(186.64588951,253.94385245)
\curveto(186.42188911,253.96874139)(186.22277764,253.99985256)(186.0485551,254.03718596)
\lineto(186.0485551,256.23985657)
\curveto(186.17299977,256.21496764)(186.33477784,256.1900787)(186.53388931,256.16518977)
\curveto(186.73300078,256.14030084)(186.93833448,256.12785637)(187.14989041,256.12785637)
\curveto(187.73478035,256.12785637)(188.19522562,256.30830114)(188.53122622,256.66919067)
\curveto(188.86722682,257.01763574)(189.12233839,257.44074761)(189.29656093,257.93852628)
\lineto(189.46456123,258.44252719)
\closepath
}
}
{
\newrgbcolor{curcolor}{0 0 0}
\pscustom[linestyle=none,fillstyle=solid,fillcolor=curcolor]
{
\newpath
\moveto(200.27258506,258.36786039)
\lineto(197.49124674,258.36786039)
\lineto(197.49124674,272.55455248)
\lineto(200.27258506,272.55455248)
\closepath
}
}
{
\newrgbcolor{curcolor}{0 0 0}
\pscustom[linestyle=none,fillstyle=solid,fillcolor=curcolor]
{
\newpath
\moveto(207.38460594,268.74654566)
\curveto(208.79083069,268.74654566)(209.90461046,268.34210049)(210.72594527,267.53321015)
\curveto(211.54728007,266.73676428)(211.95794747,265.59809557)(211.95794747,264.11720402)
\lineto(211.95794747,262.77320162)
\lineto(205.38726903,262.77320162)
\curveto(205.41215796,261.98920021)(205.6423806,261.37319911)(206.07793693,260.9251983)
\curveto(206.52593774,260.4771975)(207.14193884,260.2531971)(207.92594025,260.2531971)
\curveto(208.57305252,260.2531971)(209.16416469,260.31541943)(209.69927676,260.4398641)
\curveto(210.2468333,260.57675323)(210.8068343,260.78208694)(211.37927977,261.0558652)
\lineto(211.37927977,258.90919469)
\curveto(210.86905663,258.66030535)(210.3401668,258.47986059)(209.79261026,258.36786039)
\curveto(209.24505372,258.24341572)(208.57927475,258.18119338)(207.79527335,258.18119338)
\curveto(206.77482707,258.18119338)(205.87260323,258.36786039)(205.08860183,258.74119439)
\curveto(204.30460042,259.12697286)(203.68859932,259.69941833)(203.24059851,260.4585308)
\curveto(202.79259771,261.23008774)(202.56859731,262.20697838)(202.56859731,263.38920272)
\curveto(202.56859731,264.57142706)(202.76770878,265.56076217)(203.16593171,266.35720804)
\curveto(203.57659912,267.15365391)(204.14282235,267.75098832)(204.86460143,268.14921125)
\curveto(205.5863805,268.54743419)(206.426382,268.74654566)(207.38460594,268.74654566)
\closepath
\moveto(207.40327264,266.76787544)
\curveto(206.85571611,266.76787544)(206.4077153,266.59365291)(206.05927023,266.24520784)
\curveto(205.71082517,265.89676277)(205.50549146,265.35542847)(205.44326913,264.62120493)
\lineto(209.34460946,264.62120493)
\curveto(209.33216499,265.2309838)(209.16416469,265.74120694)(208.84060855,266.15187434)
\curveto(208.52949688,266.56254174)(208.05038491,266.76787544)(207.40327264,266.76787544)
\closepath
}
}
{
\newrgbcolor{curcolor}{0 0 0}
\pscustom[linestyle=none,fillstyle=solid,fillcolor=curcolor]
{
\newpath
\moveto(221.32862396,261.39186581)
\curveto(221.32862396,260.35897507)(220.96151219,259.56252919)(220.22728866,259.00252819)
\curveto(219.50550958,258.45497165)(218.42284098,258.18119338)(216.97928283,258.18119338)
\curveto(216.26994823,258.18119338)(215.66016936,258.23097125)(215.14994622,258.33052698)
\curveto(214.63972308,258.41763825)(214.12949995,258.56697185)(213.61927681,258.77852779)
\lineto(213.61927681,261.0745319)
\curveto(214.16683335,260.82564257)(214.75794552,260.62030887)(215.39261332,260.4585308)
\curveto(216.02728113,260.29675273)(216.58728213,260.2158637)(217.07261633,260.2158637)
\curveto(217.6077284,260.2158637)(217.99350687,260.29675273)(218.22995174,260.4585308)
\curveto(218.46639661,260.62030887)(218.58461904,260.8318648)(218.58461904,261.0931986)
\curveto(218.58461904,261.26742114)(218.53484118,261.42297697)(218.43528544,261.55986611)
\curveto(218.34817418,261.69675524)(218.14906271,261.85231108)(217.83795104,262.02653361)
\curveto(217.52683937,262.20075614)(217.04150517,262.42475655)(216.38194843,262.69853481)
\curveto(215.73483616,262.97231308)(215.20594632,263.23986912)(214.79527892,263.50120292)
\curveto(214.39705598,263.77498119)(214.09838878,264.09853732)(213.89927731,264.47187133)
\curveto(213.70016584,264.8576498)(213.60061011,265.33676177)(213.60061011,265.90920724)
\curveto(213.60061011,266.85498671)(213.96772188,267.56432131)(214.70194542,268.03721105)
\curveto(215.43616896,268.51010079)(216.4130596,268.74654566)(217.63261734,268.74654566)
\curveto(218.26728514,268.74654566)(218.87084178,268.68432332)(219.44328725,268.55987866)
\curveto(220.01573272,268.43543399)(220.60684489,268.23010029)(221.21662376,267.94387755)
\lineto(220.37662226,265.94654064)
\curveto(219.87884359,266.15809657)(219.40595385,266.33231911)(218.95795305,266.46920824)
\curveto(218.50995224,266.61854184)(218.05572921,266.69320864)(217.59528394,266.69320864)
\curveto(216.77394913,266.69320864)(216.36328173,266.46920824)(216.36328173,266.02120744)
\curveto(216.36328173,265.85942937)(216.4130596,265.71009577)(216.51261533,265.57320663)
\curveto(216.62461553,265.44876197)(216.82994923,265.31187283)(217.12861643,265.16253923)
\curveto(217.4397281,265.01320563)(217.89395114,264.81409416)(218.49128554,264.56520483)
\curveto(219.07617548,264.32875996)(219.58017638,264.07987062)(220.00328825,263.81853682)
\curveto(220.42640012,263.56964749)(220.74995626,263.25231359)(220.97395666,262.86653512)
\curveto(221.21040153,262.48075665)(221.32862396,261.98920021)(221.32862396,261.39186581)
\closepath
}
}
{
\newrgbcolor{curcolor}{0 0 0}
\pscustom[linestyle=none,fillstyle=solid,fillcolor=curcolor]
{
\newpath
\moveto(218.63613456,1102.20732944)
\curveto(217.11790961,1102.20732944)(215.9419075,1102.62421908)(215.10812823,1103.45799835)
\curveto(214.28679343,1104.29177762)(213.87612602,1105.61711333)(213.87612602,1107.43400548)
\curveto(213.87612602,1108.67845215)(214.08768196,1109.69267619)(214.51079383,1110.4766776)
\curveto(214.9339057,1111.260679)(215.51879563,1111.83934671)(216.26546364,1112.21268071)
\curveto(217.02457611,1112.58601471)(217.89568878,1112.77268171)(218.87880166,1112.77268171)
\curveto(219.5756918,1112.77268171)(220.17924843,1112.70423715)(220.68947157,1112.56734801)
\curveto(221.21213917,1112.43045888)(221.66636221,1112.26868081)(222.05214068,1112.08201381)
\lineto(221.23080587,1109.93534329)
\curveto(220.79524954,1110.10956583)(220.38458213,1110.2526772)(219.99880367,1110.3646774)
\curveto(219.62546966,1110.4766776)(219.25213566,1110.5326777)(218.87880166,1110.5326777)
\curveto(217.43524351,1110.5326777)(216.71346444,1109.50600919)(216.71346444,1107.45267218)
\curveto(216.71346444,1106.4322259)(216.90013144,1105.67933566)(217.27346545,1105.19400146)
\curveto(217.65924392,1104.70866726)(218.19435599,1104.46600016)(218.87880166,1104.46600016)
\curveto(219.4636916,1104.46600016)(219.98013697,1104.54066696)(220.42813777,1104.69000056)
\curveto(220.87613857,1104.85177863)(221.31169491,1105.06955679)(221.73480678,1105.34333506)
\lineto(221.73480678,1102.97266415)
\curveto(221.31169491,1102.69888588)(220.8636941,1102.50599664)(220.39080437,1102.39399644)
\curveto(219.9303591,1102.26955177)(219.34546916,1102.20732944)(218.63613456,1102.20732944)
\closepath
}
}
{
\newrgbcolor{curcolor}{0 0 0}
\pscustom[linestyle=none,fillstyle=solid,fillcolor=curcolor]
{
\newpath
\moveto(233.34549296,1107.50867228)
\curveto(233.34549296,1105.8162248)(232.89749216,1104.50955579)(232.00149055,1103.58866525)
\curveto(231.11793342,1102.66777471)(229.91082014,1102.20732944)(228.38015073,1102.20732944)
\curveto(227.43437126,1102.20732944)(226.58814752,1102.41266314)(225.84147951,1102.82333055)
\curveto(225.10725597,1103.23399795)(224.52858827,1103.83133235)(224.1054764,1104.61533376)
\curveto(223.68236453,1105.41177963)(223.4708086,1106.3762258)(223.4708086,1107.50867228)
\curveto(223.4708086,1109.20111976)(223.91258717,1110.50156653)(224.79614431,1111.4100126)
\curveto(225.67970144,1112.31845868)(226.89303695,1112.77268171)(228.43615083,1112.77268171)
\curveto(229.39437477,1112.77268171)(230.24059851,1112.56734801)(230.97482205,1112.15668061)
\curveto(231.70904559,1111.74601321)(232.28771329,1111.1486788)(232.71082516,1110.3646774)
\curveto(233.13393703,1109.58067599)(233.34549296,1108.62867429)(233.34549296,1107.50867228)
\closepath
\moveto(226.30814702,1107.50867228)
\curveto(226.30814702,1106.50067047)(226.46992508,1105.73533577)(226.79348122,1105.21266816)
\curveto(227.12948182,1104.70244502)(227.67081613,1104.44733346)(228.41748413,1104.44733346)
\curveto(229.15170767,1104.44733346)(229.68059751,1104.70244502)(230.00415364,1105.21266816)
\curveto(230.34015424,1105.73533577)(230.50815454,1106.50067047)(230.50815454,1107.50867228)
\curveto(230.50815454,1108.51667408)(230.34015424,1109.26956432)(230.00415364,1109.76734299)
\curveto(229.68059751,1110.27756613)(229.14548544,1110.5326777)(228.39881743,1110.5326777)
\curveto(227.66459389,1110.5326777)(227.12948182,1110.27756613)(226.79348122,1109.76734299)
\curveto(226.46992508,1109.26956432)(226.30814702,1108.51667408)(226.30814702,1107.50867228)
\closepath
}
}
{
\newrgbcolor{curcolor}{0 0 0}
\pscustom[linestyle=none,fillstyle=solid,fillcolor=curcolor]
{
\newpath
\moveto(241.42816456,1112.77268171)
\curveto(242.52327763,1112.77268171)(243.40061254,1112.47401451)(244.06016928,1111.87668011)
\curveto(244.71972601,1111.29179017)(245.04950438,1110.3460107)(245.04950438,1109.03934169)
\lineto(245.04950438,1102.39399644)
\lineto(242.26816606,1102.39399644)
\lineto(242.26816606,1108.34867378)
\curveto(242.26816606,1109.08289732)(242.13749916,1109.63045386)(241.87616536,1109.99134339)
\curveto(241.61483156,1110.3646774)(241.19794192,1110.5513444)(240.62549645,1110.5513444)
\curveto(239.77927271,1110.5513444)(239.20060501,1110.25889943)(238.88949334,1109.67400949)
\curveto(238.57838167,1109.10156402)(238.42282584,1108.27400698)(238.42282584,1107.19133838)
\lineto(238.42282584,1102.39399644)
\lineto(235.64148752,1102.39399644)
\lineto(235.64148752,1112.58601471)
\lineto(237.76949133,1112.58601471)
\lineto(238.14282534,1111.2793457)
\lineto(238.29215894,1111.2793457)
\curveto(238.61571507,1111.80201331)(239.05749364,1112.18156954)(239.61749465,1112.41801441)
\curveto(240.18994012,1112.65445928)(240.79349675,1112.77268171)(241.42816456,1112.77268171)
\closepath
}
}
{
\newrgbcolor{curcolor}{0 0 0}
\pscustom[linestyle=none,fillstyle=solid,fillcolor=curcolor]
{
\newpath
\moveto(253.6921796,1112.77268171)
\curveto(254.78729268,1112.77268171)(255.66462758,1112.47401451)(256.32418432,1111.87668011)
\curveto(256.98374106,1111.29179017)(257.31351943,1110.3460107)(257.31351943,1109.03934169)
\lineto(257.31351943,1102.39399644)
\lineto(254.53218111,1102.39399644)
\lineto(254.53218111,1108.34867378)
\curveto(254.53218111,1109.08289732)(254.40151421,1109.63045386)(254.1401804,1109.99134339)
\curveto(253.8788466,1110.3646774)(253.46195697,1110.5513444)(252.8895115,1110.5513444)
\curveto(252.04328776,1110.5513444)(251.46462005,1110.25889943)(251.15350838,1109.67400949)
\curveto(250.84239671,1109.10156402)(250.68684088,1108.27400698)(250.68684088,1107.19133838)
\lineto(250.68684088,1102.39399644)
\lineto(247.90550256,1102.39399644)
\lineto(247.90550256,1112.58601471)
\lineto(250.03350638,1112.58601471)
\lineto(250.40684038,1111.2793457)
\lineto(250.55617398,1111.2793457)
\curveto(250.87973012,1111.80201331)(251.32150868,1112.18156954)(251.88150969,1112.41801441)
\curveto(252.45395516,1112.65445928)(253.0575118,1112.77268171)(253.6921796,1112.77268171)
\closepath
}
}
{
\newrgbcolor{curcolor}{0 0 0}
\pscustom[linestyle=none,fillstyle=solid,fillcolor=curcolor]
{
\newpath
\moveto(264.36952513,1112.77268171)
\curveto(265.77574988,1112.77268171)(266.88952965,1112.36823654)(267.71086446,1111.55934621)
\curveto(268.53219926,1110.76290033)(268.94286666,1109.62423163)(268.94286666,1108.14334008)
\lineto(268.94286666,1106.79933767)
\lineto(262.37218822,1106.79933767)
\curveto(262.39707715,1106.01533627)(262.62729979,1105.39933516)(263.06285612,1104.95133436)
\curveto(263.51085693,1104.50333356)(264.12685803,1104.27933316)(264.91085944,1104.27933316)
\curveto(265.55797171,1104.27933316)(266.14908388,1104.34155549)(266.68419595,1104.46600016)
\curveto(267.23175249,1104.60288929)(267.79175349,1104.80822299)(268.36419896,1105.08200126)
\lineto(268.36419896,1102.93533075)
\curveto(267.85397582,1102.68644141)(267.32508599,1102.50599664)(266.77752945,1102.39399644)
\curveto(266.22997291,1102.26955177)(265.56419394,1102.20732944)(264.78019254,1102.20732944)
\curveto(263.75974626,1102.20732944)(262.85752242,1102.39399644)(262.07352102,1102.76733044)
\curveto(261.28951961,1103.15310891)(260.67351851,1103.72555438)(260.22551771,1104.48466686)
\curveto(259.7775169,1105.2562238)(259.5535165,1106.23311444)(259.5535165,1107.41533878)
\curveto(259.5535165,1108.59756312)(259.75262797,1109.58689822)(260.1508509,1110.3833441)
\curveto(260.56151831,1111.17978997)(261.12774154,1111.77712437)(261.84952062,1112.17534731)
\curveto(262.57129969,1112.57357025)(263.41130119,1112.77268171)(264.36952513,1112.77268171)
\closepath
\moveto(264.38819183,1110.7940115)
\curveto(263.8406353,1110.7940115)(263.39263449,1110.61978897)(263.04418942,1110.2713439)
\curveto(262.69574436,1109.92289883)(262.49041065,1109.38156452)(262.42818832,1108.64734099)
\lineto(266.32952865,1108.64734099)
\curveto(266.31708418,1109.25711986)(266.14908388,1109.76734299)(265.82552774,1110.1780104)
\curveto(265.51441607,1110.5886778)(265.0353041,1110.7940115)(264.38819183,1110.7940115)
\closepath
}
}
{
\newrgbcolor{curcolor}{0 0 0}
\pscustom[linestyle=none,fillstyle=solid,fillcolor=curcolor]
{
\newpath
\moveto(275.34553783,1102.20732944)
\curveto(273.82731289,1102.20732944)(272.65131078,1102.62421908)(271.81753151,1103.45799835)
\curveto(270.9961967,1104.29177762)(270.5855293,1105.61711333)(270.5855293,1107.43400548)
\curveto(270.5855293,1108.67845215)(270.79708524,1109.69267619)(271.22019711,1110.4766776)
\curveto(271.64330897,1111.260679)(272.22819891,1111.83934671)(272.97486692,1112.21268071)
\curveto(273.73397939,1112.58601471)(274.60509206,1112.77268171)(275.58820494,1112.77268171)
\curveto(276.28509507,1112.77268171)(276.88865171,1112.70423715)(277.39887485,1112.56734801)
\curveto(277.92154245,1112.43045888)(278.37576549,1112.26868081)(278.76154396,1112.08201381)
\lineto(277.94020915,1109.93534329)
\curveto(277.50465282,1110.10956583)(277.09398541,1110.2526772)(276.70820694,1110.3646774)
\curveto(276.33487294,1110.4766776)(275.96153894,1110.5326777)(275.58820494,1110.5326777)
\curveto(274.14464679,1110.5326777)(273.42286772,1109.50600919)(273.42286772,1107.45267218)
\curveto(273.42286772,1106.4322259)(273.60953472,1105.67933566)(273.98286872,1105.19400146)
\curveto(274.36864719,1104.70866726)(274.90375926,1104.46600016)(275.58820494,1104.46600016)
\curveto(276.17309487,1104.46600016)(276.68954024,1104.54066696)(277.13754105,1104.69000056)
\curveto(277.58554185,1104.85177863)(278.02109819,1105.06955679)(278.44421005,1105.34333506)
\lineto(278.44421005,1102.97266415)
\curveto(278.02109819,1102.69888588)(277.57309738,1102.50599664)(277.10020765,1102.39399644)
\curveto(276.63976238,1102.26955177)(276.05487244,1102.20732944)(275.34553783,1102.20732944)
\closepath
}
}
{
\newrgbcolor{curcolor}{0 0 0}
\pscustom[linestyle=none,fillstyle=solid,fillcolor=curcolor]
{
\newpath
\moveto(285.0895521,1104.42866676)
\curveto(285.40066377,1104.42866676)(285.69933097,1104.45355569)(285.98555371,1104.50333356)
\curveto(286.27177644,1104.56555589)(286.55799918,1104.64644492)(286.84422191,1104.74600066)
\lineto(286.84422191,1102.67399694)
\curveto(286.54555471,1102.53710781)(286.17222071,1102.42510761)(285.7242199,1102.33799634)
\curveto(285.28866357,1102.25088507)(284.8095516,1102.20732944)(284.28688399,1102.20732944)
\curveto(283.67710512,1102.20732944)(283.12954859,1102.30688518)(282.64421438,1102.50599664)
\curveto(282.17132465,1102.70510811)(281.79176841,1103.04733095)(281.50554568,1103.53266515)
\curveto(281.23176741,1104.01799935)(281.09487827,1104.70244502)(281.09487827,1105.58600216)
\lineto(281.09487827,1110.4953443)
\lineto(279.76954256,1110.4953443)
\lineto(279.76954256,1111.67134641)
\lineto(281.30021197,1112.60468141)
\lineto(282.10288008,1114.75135193)
\lineto(283.87621659,1114.75135193)
\lineto(283.87621659,1112.58601471)
\lineto(286.73222171,1112.58601471)
\lineto(286.73222171,1110.4953443)
\lineto(283.87621659,1110.4953443)
\lineto(283.87621659,1105.58600216)
\curveto(283.87621659,1105.20022369)(283.98821679,1104.90777873)(284.21221719,1104.70866726)
\curveto(284.4362176,1104.52200026)(284.72866256,1104.42866676)(285.0895521,1104.42866676)
\closepath
}
}
{
\newrgbcolor{curcolor}{0 0 0}
\pscustom[linestyle=none,fillstyle=solid,fillcolor=curcolor]
{
\newpath
\moveto(288.50555914,1103.70066545)
\curveto(288.50555914,1104.27311092)(288.66111498,1104.67133386)(288.97222665,1104.89533426)
\curveto(289.28333831,1105.13177913)(289.66289455,1105.25000156)(290.11089535,1105.25000156)
\curveto(290.54645169,1105.25000156)(290.91978569,1105.13177913)(291.23089736,1104.89533426)
\curveto(291.54200903,1104.67133386)(291.69756486,1104.27311092)(291.69756486,1103.70066545)
\curveto(291.69756486,1103.15310891)(291.54200903,1102.75488598)(291.23089736,1102.50599664)
\curveto(290.91978569,1102.26955177)(290.54645169,1102.15132934)(290.11089535,1102.15132934)
\curveto(289.66289455,1102.15132934)(289.28333831,1102.26955177)(288.97222665,1102.50599664)
\curveto(288.66111498,1102.75488598)(288.50555914,1103.15310891)(288.50555914,1103.70066545)
\closepath
}
}
{
\newrgbcolor{curcolor}{0 0 0}
\pscustom[linestyle=none,fillstyle=solid,fillcolor=curcolor]
{
\newpath
\moveto(294.1055669,1115.21801943)
\curveto(294.1055669,1115.74068703)(294.24867827,1116.09535434)(294.534901,1116.28202134)
\curveto(294.83356821,1116.48113281)(295.19445774,1116.58068854)(295.61756961,1116.58068854)
\curveto(296.02823701,1116.58068854)(296.38290432,1116.48113281)(296.68157152,1116.28202134)
\curveto(296.98023872,1116.09535434)(297.12957232,1115.74068703)(297.12957232,1115.21801943)
\curveto(297.12957232,1114.70779629)(296.98023872,1114.35312899)(296.68157152,1114.15401752)
\curveto(296.38290432,1113.95490605)(296.02823701,1113.85535032)(295.61756961,1113.85535032)
\curveto(295.19445774,1113.85535032)(294.83356821,1113.95490605)(294.534901,1114.15401752)
\curveto(294.24867827,1114.35312899)(294.1055669,1114.70779629)(294.1055669,1115.21801943)
\closepath
\moveto(293.3962323,1097.91398841)
\curveto(293.07267616,1097.91398841)(292.74289779,1097.93887735)(292.40689719,1097.98865521)
\curveto(292.07089659,1098.02598861)(291.79089609,1098.07576648)(291.56689568,1098.13798881)
\lineto(291.56689568,1100.32199273)
\curveto(291.79089609,1100.25977039)(292.00245202,1100.21621476)(292.20156349,1100.19132583)
\curveto(292.40067496,1100.16643689)(292.62467536,1100.15399243)(292.87356469,1100.15399243)
\curveto(293.2468987,1100.15399243)(293.5642326,1100.25977039)(293.8255664,1100.47132633)
\curveto(294.0869002,1100.68288226)(294.2175671,1101.09354967)(294.2175671,1101.70332854)
\lineto(294.2175671,1112.58601471)
\lineto(296.99890542,1112.58601471)
\lineto(296.99890542,1101.29266113)
\curveto(296.99890542,1100.6704378)(296.88068299,1100.10421456)(296.64423812,1099.59399142)
\curveto(296.40779325,1099.08376829)(296.02201478,1098.67932312)(295.48690271,1098.38065591)
\curveto(294.96423511,1098.06954425)(294.26734497,1097.91398841)(293.3962323,1097.91398841)
\closepath
}
}
{
\newrgbcolor{curcolor}{0 0 0}
\pscustom[linestyle=none,fillstyle=solid,fillcolor=curcolor]
{
\newpath
\moveto(307.02293153,1105.41800186)
\curveto(307.02293153,1104.38511112)(306.65581976,1103.58866525)(305.92159622,1103.02866425)
\curveto(305.19981715,1102.48110771)(304.11714854,1102.20732944)(302.6735904,1102.20732944)
\curveto(301.96425579,1102.20732944)(301.35447692,1102.25710731)(300.84425378,1102.35666304)
\curveto(300.33403065,1102.44377431)(299.82380751,1102.59310791)(299.31358437,1102.80466385)
\lineto(299.31358437,1105.10066796)
\curveto(299.86114091,1104.85177863)(300.45225308,1104.64644492)(301.08692088,1104.48466686)
\curveto(301.72158869,1104.32288879)(302.28158969,1104.24199975)(302.7669239,1104.24199975)
\curveto(303.30203597,1104.24199975)(303.68781444,1104.32288879)(303.9242593,1104.48466686)
\curveto(304.16070417,1104.64644492)(304.27892661,1104.85800086)(304.27892661,1105.11933466)
\curveto(304.27892661,1105.2935572)(304.22914874,1105.44911303)(304.12959301,1105.58600216)
\curveto(304.04248174,1105.7228913)(303.84337027,1105.87844713)(303.5322586,1106.05266967)
\curveto(303.22114693,1106.2268922)(302.73581273,1106.4508926)(302.07625599,1106.72467087)
\curveto(301.42914372,1106.99844914)(300.90025388,1107.26600518)(300.48958648,1107.52733898)
\curveto(300.09136354,1107.80111725)(299.79269634,1108.12467338)(299.59358487,1108.49800738)
\curveto(299.39447341,1108.88378585)(299.29491767,1109.36289782)(299.29491767,1109.93534329)
\curveto(299.29491767,1110.88112277)(299.66202944,1111.59045737)(300.39625298,1112.06334711)
\curveto(301.13047652,1112.53623685)(302.10736716,1112.77268171)(303.3269249,1112.77268171)
\curveto(303.9615927,1112.77268171)(304.56514934,1112.71045938)(305.13759481,1112.58601471)
\curveto(305.71004028,1112.46157004)(306.30115245,1112.25623634)(306.91093132,1111.97001361)
\lineto(306.07092982,1109.97267669)
\curveto(305.57315115,1110.18423263)(305.10026141,1110.35845516)(304.65226061,1110.4953443)
\curveto(304.20425981,1110.6446779)(303.75003677,1110.7193447)(303.2895915,1110.7193447)
\curveto(302.46825669,1110.7193447)(302.05758929,1110.4953443)(302.05758929,1110.04734349)
\curveto(302.05758929,1109.88556543)(302.10736716,1109.73623183)(302.20692289,1109.59934269)
\curveto(302.31892309,1109.47489802)(302.52425679,1109.33800889)(302.822924,1109.18867529)
\curveto(303.13403567,1109.03934169)(303.5882587,1108.84023022)(304.18559311,1108.59134088)
\curveto(304.77048304,1108.35489602)(305.27448395,1108.10600668)(305.69759582,1107.84467288)
\curveto(306.12070769,1107.59578354)(306.44426382,1107.27844964)(306.66826422,1106.89267117)
\curveto(306.90470909,1106.5068927)(307.02293153,1106.01533627)(307.02293153,1105.41800186)
\closepath
}
}
{
\newrgbcolor{curcolor}{0 0 0}
\pscustom[linestyle=none,fillstyle=solid,fillcolor=curcolor]
{
\newpath
\moveto(218.23186378,1080.56790638)
\curveto(216.71363884,1080.56790638)(215.53763673,1080.98479602)(214.70385746,1081.81857529)
\curveto(213.88252265,1082.65235456)(213.47185525,1083.97769027)(213.47185525,1085.79458242)
\curveto(213.47185525,1087.03902909)(213.68341119,1088.05325313)(214.10652306,1088.83725454)
\curveto(214.52963492,1089.62125595)(215.11452486,1090.19992365)(215.86119287,1090.57325765)
\curveto(216.62030534,1090.94659165)(217.49141801,1091.13325866)(218.47453089,1091.13325866)
\curveto(219.17142102,1091.13325866)(219.77497766,1091.06481409)(220.2852008,1090.92792495)
\curveto(220.8078684,1090.79103582)(221.26209144,1090.62925775)(221.64786991,1090.44259075)
\lineto(220.8265351,1088.29592024)
\curveto(220.39097877,1088.47014277)(219.98031136,1088.61325414)(219.59453289,1088.72525434)
\curveto(219.22119889,1088.83725454)(218.84786489,1088.89325464)(218.47453089,1088.89325464)
\curveto(217.03097274,1088.89325464)(216.30919367,1087.86658613)(216.30919367,1085.81324912)
\curveto(216.30919367,1084.79280285)(216.49586067,1084.03991261)(216.86919467,1083.5545784)
\curveto(217.25497314,1083.0692442)(217.79008521,1082.8265771)(218.47453089,1082.8265771)
\curveto(219.05942082,1082.8265771)(219.57586619,1082.9012439)(220.023867,1083.0505775)
\curveto(220.4718678,1083.21235557)(220.90742414,1083.43013374)(221.330536,1083.703912)
\lineto(221.330536,1081.33324109)
\curveto(220.90742414,1081.05946282)(220.45942333,1080.86657359)(219.9865336,1080.75457338)
\curveto(219.52608833,1080.63012872)(218.94119839,1080.56790638)(218.23186378,1080.56790638)
\closepath
}
}
{
\newrgbcolor{curcolor}{0 0 0}
\pscustom[linestyle=none,fillstyle=solid,fillcolor=curcolor]
{
\newpath
\moveto(233.03455569,1090.94659165)
\lineto(233.03455569,1080.75457338)
\lineto(230.90655188,1080.75457338)
\lineto(230.53321787,1082.06124239)
\lineto(230.38388427,1082.06124239)
\curveto(230.06032814,1081.53857479)(229.61232734,1081.15901855)(229.03988186,1080.92257369)
\curveto(228.47988086,1080.68612882)(227.88254646,1080.56790638)(227.24787865,1080.56790638)
\curveto(226.15276558,1080.56790638)(225.27543067,1080.86035135)(224.61587393,1081.44524129)
\curveto(223.9563172,1082.04257569)(223.62653883,1082.9945774)(223.62653883,1084.30124641)
\lineto(223.62653883,1090.94659165)
\lineto(226.40787715,1090.94659165)
\lineto(226.40787715,1084.99191431)
\curveto(226.40787715,1084.25769078)(226.53854405,1083.703912)(226.79987785,1083.330578)
\curveto(227.06121165,1082.96968847)(227.47810129,1082.7892437)(228.05054676,1082.7892437)
\curveto(228.8967705,1082.7892437)(229.4754382,1083.07546643)(229.78654987,1083.6479119)
\curveto(230.09766154,1084.23280184)(230.25321737,1085.06658111)(230.25321737,1086.14924972)
\lineto(230.25321737,1090.94659165)
\closepath
}
}
{
\newrgbcolor{curcolor}{0 0 0}
\pscustom[linestyle=none,fillstyle=solid,fillcolor=curcolor]
{
\newpath
\moveto(241.6398956,1091.13325866)
\curveto(241.77678474,1091.13325866)(241.9385628,1091.12703642)(242.12522981,1091.11459196)
\curveto(242.31189681,1091.10214749)(242.46123041,1091.08348079)(242.57323061,1091.05859186)
\lineto(242.36789691,1088.44525384)
\curveto(242.26834117,1088.47014277)(242.13767427,1088.48880947)(241.9758962,1088.50125394)
\curveto(241.81411814,1088.52614287)(241.67100677,1088.53858734)(241.5465621,1088.53858734)
\curveto(241.07367237,1088.53858734)(240.61944933,1088.45147607)(240.18389299,1088.27725354)
\curveto(239.74833666,1088.11547547)(239.39366935,1087.84791943)(239.11989109,1087.47458543)
\curveto(238.85855728,1087.10125143)(238.72789038,1086.59102829)(238.72789038,1085.94391602)
\lineto(238.72789038,1080.75457338)
\lineto(235.94655206,1080.75457338)
\lineto(235.94655206,1090.94659165)
\lineto(238.05588918,1090.94659165)
\lineto(238.46655658,1089.22925524)
\lineto(238.59722348,1089.22925524)
\curveto(238.89589068,1089.75192285)(239.30655809,1090.19992365)(239.82922569,1090.57325765)
\curveto(240.35189329,1090.94659165)(240.95544993,1091.13325866)(241.6398956,1091.13325866)
\closepath
}
}
{
\newrgbcolor{curcolor}{0 0 0}
\pscustom[linestyle=none,fillstyle=solid,fillcolor=curcolor]
{
\newpath
\moveto(251.5332561,1083.77857881)
\curveto(251.5332561,1082.74568806)(251.16614433,1081.94924219)(250.43192079,1081.38924119)
\curveto(249.71014172,1080.84168465)(248.62747311,1080.56790638)(247.18391497,1080.56790638)
\curveto(246.47458036,1080.56790638)(245.86480149,1080.61768425)(245.35457835,1080.71723998)
\curveto(244.84435522,1080.80435125)(244.33413208,1080.95368485)(243.82390894,1081.16524079)
\lineto(243.82390894,1083.4612449)
\curveto(244.37146548,1083.21235557)(244.96257765,1083.00702187)(245.59724546,1082.8452438)
\curveto(246.23191326,1082.68346573)(246.79191426,1082.6025767)(247.27724847,1082.6025767)
\curveto(247.81236054,1082.6025767)(248.19813901,1082.68346573)(248.43458388,1082.8452438)
\curveto(248.67102874,1083.00702187)(248.78925118,1083.2185778)(248.78925118,1083.4799116)
\curveto(248.78925118,1083.65413414)(248.73947331,1083.80968997)(248.63991758,1083.94657911)
\curveto(248.55280631,1084.08346824)(248.35369484,1084.23902408)(248.04258317,1084.41324661)
\curveto(247.7314715,1084.58746914)(247.2461373,1084.81146955)(246.58658056,1085.08524781)
\curveto(245.93946829,1085.35902608)(245.41057845,1085.62658212)(244.99991105,1085.88791592)
\curveto(244.60168812,1086.16169419)(244.30302091,1086.48525032)(244.10390945,1086.85858433)
\curveto(243.90479798,1087.2443628)(243.80524224,1087.72347477)(243.80524224,1088.29592024)
\curveto(243.80524224,1089.24169971)(244.17235401,1089.95103431)(244.90657755,1090.42392405)
\curveto(245.64080109,1090.89681379)(246.61769173,1091.13325866)(247.83724947,1091.13325866)
\curveto(248.47191728,1091.13325866)(249.07547391,1091.07103632)(249.64791938,1090.94659165)
\curveto(250.22036485,1090.82214699)(250.81147703,1090.61681329)(251.4212559,1090.33059055)
\lineto(250.58125439,1088.33325364)
\curveto(250.08347572,1088.54480957)(249.61058598,1088.71903211)(249.16258518,1088.85592124)
\curveto(248.71458438,1089.00525484)(248.26036134,1089.07992164)(247.79991607,1089.07992164)
\curveto(246.97858127,1089.07992164)(246.56791386,1088.85592124)(246.56791386,1088.40792044)
\curveto(246.56791386,1088.24614237)(246.61769173,1088.09680877)(246.71724746,1087.95991963)
\curveto(246.82924766,1087.83547497)(247.03458137,1087.69858583)(247.33324857,1087.54925223)
\curveto(247.64436024,1087.39991863)(248.09858327,1087.20080716)(248.69591768,1086.95191783)
\curveto(249.28080761,1086.71547296)(249.78480852,1086.46658362)(250.20792039,1086.20524982)
\curveto(250.63103226,1085.95636049)(250.95458839,1085.63902658)(251.17858879,1085.25324812)
\curveto(251.41503366,1084.86746965)(251.5332561,1084.37591321)(251.5332561,1083.77857881)
\closepath
}
}
{
\newrgbcolor{curcolor}{0 0 0}
\pscustom[linestyle=none,fillstyle=solid,fillcolor=curcolor]
{
\newpath
\moveto(262.95727215,1085.86924922)
\curveto(262.95727215,1084.17680174)(262.50927135,1082.87013273)(261.61326974,1081.94924219)
\curveto(260.7297126,1081.02835165)(259.52259933,1080.56790638)(257.99192992,1080.56790638)
\curveto(257.04615044,1080.56790638)(256.19992671,1080.77324008)(255.4532587,1081.18390749)
\curveto(254.71903516,1081.59457489)(254.14036746,1082.19190929)(253.71725559,1082.9759107)
\curveto(253.29414372,1083.77235657)(253.08258778,1084.73680275)(253.08258778,1085.86924922)
\curveto(253.08258778,1087.5616967)(253.52436635,1088.86214347)(254.40792349,1089.77058955)
\curveto(255.29148063,1090.67903562)(256.50481614,1091.13325866)(258.04793002,1091.13325866)
\curveto(259.00615396,1091.13325866)(259.8523777,1090.92792495)(260.58660124,1090.51725755)
\curveto(261.32082477,1090.10659015)(261.89949248,1089.50925574)(262.32260435,1088.72525434)
\curveto(262.74571622,1087.94125293)(262.95727215,1086.98925123)(262.95727215,1085.86924922)
\closepath
\moveto(255.9199262,1085.86924922)
\curveto(255.9199262,1084.86124741)(256.08170427,1084.09591271)(256.40526041,1083.5732451)
\curveto(256.74126101,1083.06302197)(257.28259531,1082.8079104)(258.02926332,1082.8079104)
\curveto(258.76348686,1082.8079104)(259.29237669,1083.06302197)(259.61593283,1083.5732451)
\curveto(259.95193343,1084.09591271)(260.11993373,1084.86124741)(260.11993373,1085.86924922)
\curveto(260.11993373,1086.87725103)(259.95193343,1087.63014127)(259.61593283,1088.12791994)
\curveto(259.29237669,1088.63814307)(258.75726462,1088.89325464)(258.01059662,1088.89325464)
\curveto(257.27637308,1088.89325464)(256.74126101,1088.63814307)(256.40526041,1088.12791994)
\curveto(256.08170427,1087.63014127)(255.9199262,1086.87725103)(255.9199262,1085.86924922)
\closepath
}
}
{
\newrgbcolor{curcolor}{0 0 0}
\pscustom[linestyle=none,fillstyle=solid,fillcolor=curcolor]
{
\newpath
\moveto(270.94661215,1091.13325866)
\curveto(271.08350129,1091.13325866)(271.24527935,1091.12703642)(271.43194636,1091.11459196)
\curveto(271.61861336,1091.10214749)(271.76794696,1091.08348079)(271.87994716,1091.05859186)
\lineto(271.67461346,1088.44525384)
\curveto(271.57505772,1088.47014277)(271.44439082,1088.48880947)(271.28261275,1088.50125394)
\curveto(271.12083469,1088.52614287)(270.97772332,1088.53858734)(270.85327865,1088.53858734)
\curveto(270.38038892,1088.53858734)(269.92616588,1088.45147607)(269.49060954,1088.27725354)
\curveto(269.05505321,1088.11547547)(268.7003859,1087.84791943)(268.42660764,1087.47458543)
\curveto(268.16527383,1087.10125143)(268.03460693,1086.59102829)(268.03460693,1085.94391602)
\lineto(268.03460693,1080.75457338)
\lineto(265.25326861,1080.75457338)
\lineto(265.25326861,1090.94659165)
\lineto(267.36260573,1090.94659165)
\lineto(267.77327313,1089.22925524)
\lineto(267.90394003,1089.22925524)
\curveto(268.20260723,1089.75192285)(268.61327464,1090.19992365)(269.13594224,1090.57325765)
\curveto(269.65860984,1090.94659165)(270.26216648,1091.13325866)(270.94661215,1091.13325866)
\closepath
}
}
{
\newrgbcolor{curcolor}{0 0 0}
\pscustom[linestyle=none,fillstyle=solid,fillcolor=curcolor]
{
\newpath
\moveto(272.215966,1082.06124239)
\curveto(272.215966,1082.63368786)(272.37152183,1083.0319108)(272.6826335,1083.2559112)
\curveto(272.99374517,1083.49235607)(273.37330141,1083.6105785)(273.82130221,1083.6105785)
\curveto(274.25685855,1083.6105785)(274.63019255,1083.49235607)(274.94130422,1083.2559112)
\curveto(275.25241589,1083.0319108)(275.40797172,1082.63368786)(275.40797172,1082.06124239)
\curveto(275.40797172,1081.51368586)(275.25241589,1081.11546292)(274.94130422,1080.86657359)
\curveto(274.63019255,1080.63012872)(274.25685855,1080.51190628)(273.82130221,1080.51190628)
\curveto(273.37330141,1080.51190628)(272.99374517,1080.63012872)(272.6826335,1080.86657359)
\curveto(272.37152183,1081.11546292)(272.215966,1081.51368586)(272.215966,1082.06124239)
\closepath
}
}
{
\newrgbcolor{curcolor}{0 0 0}
\pscustom[linestyle=none,fillstyle=solid,fillcolor=curcolor]
{
\newpath
\moveto(277.81597376,1093.57859637)
\curveto(277.81597376,1094.10126398)(277.95908513,1094.45593128)(278.24530786,1094.64259828)
\curveto(278.54397506,1094.84170975)(278.9048646,1094.94126548)(279.32797647,1094.94126548)
\curveto(279.73864387,1094.94126548)(280.09331117,1094.84170975)(280.39197838,1094.64259828)
\curveto(280.69064558,1094.45593128)(280.83997918,1094.10126398)(280.83997918,1093.57859637)
\curveto(280.83997918,1093.06837324)(280.69064558,1092.71370593)(280.39197838,1092.51459447)
\curveto(280.09331117,1092.315483)(279.73864387,1092.21592726)(279.32797647,1092.21592726)
\curveto(278.9048646,1092.21592726)(278.54397506,1092.315483)(278.24530786,1092.51459447)
\curveto(277.95908513,1092.71370593)(277.81597376,1093.06837324)(277.81597376,1093.57859637)
\closepath
\moveto(277.10663915,1076.27456535)
\curveto(276.78308302,1076.27456535)(276.45330465,1076.29945429)(276.11730405,1076.34923215)
\curveto(275.78130344,1076.38656555)(275.50130294,1076.43634342)(275.27730254,1076.49856576)
\lineto(275.27730254,1078.68256967)
\curveto(275.50130294,1078.62034734)(275.71285888,1078.5767917)(275.91197035,1078.55190277)
\curveto(276.11108181,1078.52701384)(276.33508221,1078.51456937)(276.58397155,1078.51456937)
\curveto(276.95730555,1078.51456937)(277.27463945,1078.62034734)(277.53597326,1078.83190327)
\curveto(277.79730706,1079.04345921)(277.92797396,1079.45412661)(277.92797396,1080.06390548)
\lineto(277.92797396,1090.94659165)
\lineto(280.70931228,1090.94659165)
\lineto(280.70931228,1079.65323808)
\curveto(280.70931228,1079.03101474)(280.59108984,1078.4647915)(280.35464498,1077.95456837)
\curveto(280.11820011,1077.44434523)(279.73242164,1077.03990006)(279.19730957,1076.74123286)
\curveto(278.67464196,1076.43012119)(277.97775183,1076.27456535)(277.10663915,1076.27456535)
\closepath
}
}
{
\newrgbcolor{curcolor}{0 0 0}
\pscustom[linestyle=none,fillstyle=solid,fillcolor=curcolor]
{
\newpath
\moveto(290.73333075,1083.77857881)
\curveto(290.73333075,1082.74568806)(290.36621898,1081.94924219)(289.63199544,1081.38924119)
\curveto(288.91021637,1080.84168465)(287.82754777,1080.56790638)(286.38398962,1080.56790638)
\curveto(285.67465502,1080.56790638)(285.06487615,1080.61768425)(284.55465301,1080.71723998)
\curveto(284.04442987,1080.80435125)(283.53420674,1080.95368485)(283.0239836,1081.16524079)
\lineto(283.0239836,1083.4612449)
\curveto(283.57154014,1083.21235557)(284.16265231,1083.00702187)(284.79732011,1082.8452438)
\curveto(285.43198792,1082.68346573)(285.99198892,1082.6025767)(286.47732312,1082.6025767)
\curveto(287.01243519,1082.6025767)(287.39821366,1082.68346573)(287.63465853,1082.8452438)
\curveto(287.8711034,1083.00702187)(287.98932583,1083.2185778)(287.98932583,1083.4799116)
\curveto(287.98932583,1083.65413414)(287.93954797,1083.80968997)(287.83999223,1083.94657911)
\curveto(287.75288097,1084.08346824)(287.5537695,1084.23902408)(287.24265783,1084.41324661)
\curveto(286.93154616,1084.58746914)(286.44621196,1084.81146955)(285.78665522,1085.08524781)
\curveto(285.13954295,1085.35902608)(284.61065311,1085.62658212)(284.19998571,1085.88791592)
\curveto(283.80176277,1086.16169419)(283.50309557,1086.48525032)(283.3039841,1086.85858433)
\curveto(283.10487263,1087.2443628)(283.0053169,1087.72347477)(283.0053169,1088.29592024)
\curveto(283.0053169,1089.24169971)(283.37242867,1089.95103431)(284.10665221,1090.42392405)
\curveto(284.84087575,1090.89681379)(285.81776639,1091.13325866)(287.03732413,1091.13325866)
\curveto(287.67199193,1091.13325866)(288.27554857,1091.07103632)(288.84799404,1090.94659165)
\curveto(289.42043951,1090.82214699)(290.01155168,1090.61681329)(290.62133055,1090.33059055)
\lineto(289.78132905,1088.33325364)
\curveto(289.28355038,1088.54480957)(288.81066064,1088.71903211)(288.36265984,1088.85592124)
\curveto(287.91465903,1089.00525484)(287.460436,1089.07992164)(286.99999073,1089.07992164)
\curveto(286.17865592,1089.07992164)(285.76798852,1088.85592124)(285.76798852,1088.40792044)
\curveto(285.76798852,1088.24614237)(285.81776639,1088.09680877)(285.91732212,1087.95991963)
\curveto(286.02932232,1087.83547497)(286.23465602,1087.69858583)(286.53332322,1087.54925223)
\curveto(286.84443489,1087.39991863)(287.29865793,1087.20080716)(287.89599233,1086.95191783)
\curveto(288.48088227,1086.71547296)(288.98488317,1086.46658362)(289.40799504,1086.20524982)
\curveto(289.83110691,1085.95636049)(290.15466305,1085.63902658)(290.37866345,1085.25324812)
\curveto(290.61510832,1084.86746965)(290.73333075,1084.37591321)(290.73333075,1083.77857881)
\closepath
}
}
{
\newrgbcolor{curcolor}{0 0 0}
\pscustom[linestyle=none,fillstyle=solid,fillcolor=curcolor]
{
\newpath
\moveto(215.84046472,1058.88590098)
\curveto(214.70801825,1058.88590098)(213.78090548,1059.32767955)(213.05912641,1060.21123669)
\curveto(212.3497918,1061.1072383)(211.9951245,1062.42012954)(211.9951245,1064.14991042)
\curveto(211.9951245,1065.89213576)(212.35601403,1067.21124924)(213.07779311,1068.10725085)
\curveto(213.79957218,1069.00325245)(214.74535165,1069.45125326)(215.91513153,1069.45125326)
\curveto(216.64935506,1069.45125326)(217.2529117,1069.30814189)(217.72580144,1069.02191915)
\curveto(218.19869117,1068.73569642)(218.57202518,1068.38102911)(218.84580345,1067.95791725)
\lineto(218.93913695,1067.95791725)
\curveto(218.90180355,1068.15702871)(218.85824791,1068.44325145)(218.80847005,1068.81658545)
\curveto(218.75869218,1069.20236392)(218.73380324,1069.59436462)(218.73380324,1069.99258756)
\lineto(218.73380324,1073.25926008)
\lineto(221.51514156,1073.25926008)
\lineto(221.51514156,1059.07256798)
\lineto(219.38713775,1059.07256798)
\lineto(218.84580345,1060.39790369)
\lineto(218.73380324,1060.39790369)
\curveto(218.46002498,1059.97479182)(218.09291321,1059.61390229)(217.63246794,1059.31523509)
\curveto(217.17202267,1059.02901235)(216.57468826,1058.88590098)(215.84046472,1058.88590098)
\closepath
\moveto(216.81113313,1061.1072383)
\curveto(217.5702456,1061.1072383)(218.10535767,1061.3312387)(218.41646934,1061.7792395)
\curveto(218.72758101,1062.23968477)(218.89558131,1062.92413044)(218.92047025,1063.83257652)
\lineto(218.92047025,1064.13124372)
\curveto(218.92047025,1065.11435659)(218.76491441,1065.86724683)(218.45380274,1066.38991443)
\curveto(218.15513554,1066.9250265)(217.59513454,1067.19258254)(216.77379973,1067.19258254)
\curveto(216.16402086,1067.19258254)(215.68490889,1066.9250265)(215.33646382,1066.38991443)
\curveto(214.98801875,1065.86724683)(214.81379622,1065.10813436)(214.81379622,1064.11257702)
\curveto(214.81379622,1063.11701968)(214.98801875,1062.36412944)(215.33646382,1061.8539063)
\curveto(215.68490889,1061.35612763)(216.17646533,1061.1072383)(216.81113313,1061.1072383)
\closepath
}
}
{
\newrgbcolor{curcolor}{0 0 0}
\pscustom[linestyle=none,fillstyle=solid,fillcolor=curcolor]
{
\newpath
\moveto(225.82713844,1073.25926008)
\curveto(226.23780584,1073.25926008)(226.59247314,1073.15970435)(226.89114035,1072.96059288)
\curveto(227.18980755,1072.77392588)(227.33914115,1072.41925858)(227.33914115,1071.89659097)
\curveto(227.33914115,1071.38636784)(227.18980755,1071.03170053)(226.89114035,1070.83258906)
\curveto(226.59247314,1070.6334776)(226.23780584,1070.53392186)(225.82713844,1070.53392186)
\curveto(225.40402657,1070.53392186)(225.04313703,1070.6334776)(224.74446983,1070.83258906)
\curveto(224.4582471,1071.03170053)(224.31513573,1071.38636784)(224.31513573,1071.89659097)
\curveto(224.31513573,1072.41925858)(224.4582471,1072.77392588)(224.74446983,1072.96059288)
\curveto(225.04313703,1073.15970435)(225.40402657,1073.25926008)(225.82713844,1073.25926008)
\closepath
\moveto(227.20847425,1069.26458625)
\lineto(227.20847425,1059.07256798)
\lineto(224.42713593,1059.07256798)
\lineto(224.42713593,1069.26458625)
\closepath
}
}
{
\newrgbcolor{curcolor}{0 0 0}
\pscustom[linestyle=none,fillstyle=solid,fillcolor=curcolor]
{
\newpath
\moveto(237.23249749,1062.0965734)
\curveto(237.23249749,1061.06368266)(236.86538572,1060.26723679)(236.13116218,1059.70723579)
\curveto(235.40938311,1059.15967925)(234.3267145,1058.88590098)(232.88315636,1058.88590098)
\curveto(232.17382176,1058.88590098)(231.56404288,1058.93567885)(231.05381975,1059.03523458)
\curveto(230.54359661,1059.12234585)(230.03337347,1059.27167945)(229.52315034,1059.48323539)
\lineto(229.52315034,1061.7792395)
\curveto(230.07070687,1061.53035017)(230.66181905,1061.32501647)(231.29648685,1061.1632384)
\curveto(231.93115465,1061.00146033)(232.49115566,1060.9205713)(232.97648986,1060.9205713)
\curveto(233.51160193,1060.9205713)(233.8973804,1061.00146033)(234.13382527,1061.1632384)
\curveto(234.37027014,1061.32501647)(234.48849257,1061.5365724)(234.48849257,1061.7979062)
\curveto(234.48849257,1061.97212874)(234.4387147,1062.12768457)(234.33915897,1062.26457371)
\curveto(234.2520477,1062.40146284)(234.05293624,1062.55701867)(233.74182457,1062.73124121)
\curveto(233.4307129,1062.90546374)(232.94537869,1063.12946415)(232.28582196,1063.40324241)
\curveto(231.63870969,1063.67702068)(231.10981985,1063.94457672)(230.69915245,1064.20591052)
\curveto(230.30092951,1064.47968879)(230.00226231,1064.80324492)(229.80315084,1065.17657893)
\curveto(229.60403937,1065.5623574)(229.50448364,1066.04146937)(229.50448364,1066.61391484)
\curveto(229.50448364,1067.55969431)(229.87159541,1068.26902891)(230.60581895,1068.74191865)
\curveto(231.34004248,1069.21480839)(232.31693312,1069.45125326)(233.53649087,1069.45125326)
\curveto(234.17115867,1069.45125326)(234.77471531,1069.38903092)(235.34716078,1069.26458625)
\curveto(235.91960625,1069.14014159)(236.51071842,1068.93480789)(237.12049729,1068.64858515)
\lineto(236.28049578,1066.65124824)
\curveto(235.78271711,1066.86280417)(235.30982738,1067.03702671)(234.86182657,1067.17391584)
\curveto(234.41382577,1067.32324944)(233.95960273,1067.39791624)(233.49915746,1067.39791624)
\curveto(232.67782266,1067.39791624)(232.26715526,1067.17391584)(232.26715526,1066.72591504)
\curveto(232.26715526,1066.56413697)(232.31693312,1066.41480337)(232.41648886,1066.27791423)
\curveto(232.52848906,1066.15346957)(232.73382276,1066.01658043)(233.03248996,1065.86724683)
\curveto(233.34360163,1065.71791323)(233.79782467,1065.51880176)(234.39515907,1065.26991243)
\curveto(234.98004901,1065.03346756)(235.48404991,1064.78457822)(235.90716178,1064.52324442)
\curveto(236.33027365,1064.27435509)(236.65382979,1063.95702118)(236.87783019,1063.57124272)
\curveto(237.11427506,1063.18546425)(237.23249749,1062.69390781)(237.23249749,1062.0965734)
\closepath
}
}
{
\newrgbcolor{curcolor}{0 0 0}
\pscustom[linestyle=none,fillstyle=solid,fillcolor=curcolor]
{
\newpath
\moveto(243.54183771,1058.88590098)
\curveto(242.02361277,1058.88590098)(240.84761066,1059.30279062)(240.01383139,1060.13656989)
\curveto(239.19249658,1060.97034916)(238.78182918,1062.29568487)(238.78182918,1064.11257702)
\curveto(238.78182918,1065.35702369)(238.99338511,1066.37124773)(239.41649698,1067.15524914)
\curveto(239.83960885,1067.93925054)(240.42449879,1068.51791825)(241.17116679,1068.89125225)
\curveto(241.93027927,1069.26458625)(242.80139194,1069.45125326)(243.78450481,1069.45125326)
\curveto(244.48139495,1069.45125326)(245.08495159,1069.38280869)(245.59517472,1069.24591955)
\curveto(246.11784233,1069.10903042)(246.57206536,1068.94725235)(246.95784383,1068.76058535)
\lineto(246.13650903,1066.61391484)
\curveto(245.70095269,1066.78813737)(245.29028529,1066.93124874)(244.90450682,1067.04324894)
\curveto(244.53117282,1067.15524914)(244.15783881,1067.21124924)(243.78450481,1067.21124924)
\curveto(242.34094667,1067.21124924)(241.6191676,1066.18458073)(241.6191676,1064.13124372)
\curveto(241.6191676,1063.11079745)(241.8058346,1062.35790721)(242.1791686,1061.872573)
\curveto(242.56494707,1061.3872388)(243.10005914,1061.1445717)(243.78450481,1061.1445717)
\curveto(244.36939475,1061.1445717)(244.88584012,1061.2192385)(245.33384092,1061.3685721)
\curveto(245.78184173,1061.53035017)(246.21739806,1061.74812834)(246.64050993,1062.0219066)
\lineto(246.64050993,1059.65123569)
\curveto(246.21739806,1059.37745742)(245.76939726,1059.18456818)(245.29650752,1059.07256798)
\curveto(244.83606225,1058.94812332)(244.25117232,1058.88590098)(243.54183771,1058.88590098)
\closepath
}
}
{
\newrgbcolor{curcolor}{0 0 0}
\pscustom[linestyle=none,fillstyle=solid,fillcolor=curcolor]
{
\newpath
\moveto(258.25119803,1064.18724382)
\curveto(258.25119803,1062.49479634)(257.80319722,1061.18812733)(256.90719562,1060.26723679)
\curveto(256.02363848,1059.34634625)(254.8165252,1058.88590098)(253.28585579,1058.88590098)
\curveto(252.34007632,1058.88590098)(251.49385258,1059.09123468)(250.74718457,1059.50190209)
\curveto(250.01296104,1059.91256949)(249.43429333,1060.50990389)(249.01118146,1061.2939053)
\curveto(248.58806959,1062.09035117)(248.37651366,1063.05479734)(248.37651366,1064.18724382)
\curveto(248.37651366,1065.8796913)(248.81829223,1067.18013807)(249.70184937,1068.08858415)
\curveto(250.58540651,1068.99703022)(251.79874201,1069.45125326)(253.34185589,1069.45125326)
\curveto(254.30007983,1069.45125326)(255.14630357,1069.24591955)(255.88052711,1068.83525215)
\curveto(256.61475065,1068.42458475)(257.19341835,1067.82725034)(257.61653022,1067.04324894)
\curveto(258.03964209,1066.25924753)(258.25119803,1065.30724583)(258.25119803,1064.18724382)
\closepath
\moveto(251.21385208,1064.18724382)
\curveto(251.21385208,1063.17924201)(251.37563015,1062.41390731)(251.69918628,1061.8912397)
\curveto(252.03518688,1061.38101657)(252.57652119,1061.125905)(253.32318919,1061.125905)
\curveto(254.05741273,1061.125905)(254.58630257,1061.38101657)(254.9098587,1061.8912397)
\curveto(255.24585931,1062.41390731)(255.41385961,1063.17924201)(255.41385961,1064.18724382)
\curveto(255.41385961,1065.19524563)(255.24585931,1065.94813586)(254.9098587,1066.44591453)
\curveto(254.58630257,1066.95613767)(254.0511905,1067.21124924)(253.30452249,1067.21124924)
\curveto(252.57029895,1067.21124924)(252.03518688,1066.95613767)(251.69918628,1066.44591453)
\curveto(251.37563015,1065.94813586)(251.21385208,1065.19524563)(251.21385208,1064.18724382)
\closepath
}
}
{
\newrgbcolor{curcolor}{0 0 0}
\pscustom[linestyle=none,fillstyle=solid,fillcolor=curcolor]
{
\newpath
\moveto(266.33386771,1069.45125326)
\curveto(267.42898079,1069.45125326)(268.30631569,1069.15258605)(268.96587243,1068.55525165)
\curveto(269.62542917,1067.97036171)(269.95520754,1067.02458224)(269.95520754,1065.71791323)
\lineto(269.95520754,1059.07256798)
\lineto(267.17386922,1059.07256798)
\lineto(267.17386922,1065.02724533)
\curveto(267.17386922,1065.76146886)(267.04320232,1066.3090254)(266.78186852,1066.66991494)
\curveto(266.52053471,1067.04324894)(266.10364508,1067.22991594)(265.53119961,1067.22991594)
\curveto(264.68497587,1067.22991594)(264.10630816,1066.93747097)(263.79519649,1066.35258103)
\curveto(263.48408483,1065.78013556)(263.32852899,1064.95257852)(263.32852899,1063.86990992)
\lineto(263.32852899,1059.07256798)
\lineto(260.54719067,1059.07256798)
\lineto(260.54719067,1069.26458625)
\lineto(262.67519449,1069.26458625)
\lineto(263.04852849,1067.95791725)
\lineto(263.19786209,1067.95791725)
\curveto(263.52141823,1068.48058485)(263.9631968,1068.86014108)(264.5231978,1069.09658595)
\curveto(265.09564327,1069.33303082)(265.69919991,1069.45125326)(266.33386771,1069.45125326)
\closepath
}
}
{
\newrgbcolor{curcolor}{0 0 0}
\pscustom[linestyle=none,fillstyle=solid,fillcolor=curcolor]
{
\newpath
\moveto(278.59788276,1069.45125326)
\curveto(279.69299583,1069.45125326)(280.57033074,1069.15258605)(281.22988747,1068.55525165)
\curveto(281.88944421,1067.97036171)(282.21922258,1067.02458224)(282.21922258,1065.71791323)
\lineto(282.21922258,1059.07256798)
\lineto(279.43788426,1059.07256798)
\lineto(279.43788426,1065.02724533)
\curveto(279.43788426,1065.76146886)(279.30721736,1066.3090254)(279.04588356,1066.66991494)
\curveto(278.78454976,1067.04324894)(278.36766012,1067.22991594)(277.79521465,1067.22991594)
\curveto(276.94899091,1067.22991594)(276.37032321,1066.93747097)(276.05921154,1066.35258103)
\curveto(275.74809987,1065.78013556)(275.59254403,1064.95257852)(275.59254403,1063.86990992)
\lineto(275.59254403,1059.07256798)
\lineto(272.81120572,1059.07256798)
\lineto(272.81120572,1069.26458625)
\lineto(274.93920953,1069.26458625)
\lineto(275.31254353,1067.95791725)
\lineto(275.46187713,1067.95791725)
\curveto(275.78543327,1068.48058485)(276.22721184,1068.86014108)(276.78721284,1069.09658595)
\curveto(277.35965831,1069.33303082)(277.96321495,1069.45125326)(278.59788276,1069.45125326)
\closepath
}
}
{
\newrgbcolor{curcolor}{0 0 0}
\pscustom[linestyle=none,fillstyle=solid,fillcolor=curcolor]
{
\newpath
\moveto(289.27522829,1069.45125326)
\curveto(290.68145303,1069.45125326)(291.79523281,1069.04680809)(292.61656761,1068.23791775)
\curveto(293.43790242,1067.44147187)(293.84856982,1066.30280317)(293.84856982,1064.82191162)
\lineto(293.84856982,1063.47790921)
\lineto(287.27789137,1063.47790921)
\curveto(287.30278031,1062.69390781)(287.53300294,1062.0779067)(287.96855928,1061.6299059)
\curveto(288.41656008,1061.1819051)(289.03256119,1060.9579047)(289.81656259,1060.9579047)
\curveto(290.46367486,1060.9579047)(291.05478703,1061.02012703)(291.5898991,1061.1445717)
\curveto(292.13745564,1061.28146083)(292.69745664,1061.48679453)(293.26990212,1061.7605728)
\lineto(293.26990212,1059.61390229)
\curveto(292.75967898,1059.36501295)(292.23078914,1059.18456818)(291.6832326,1059.07256798)
\curveto(291.13567607,1058.94812332)(290.4698971,1058.88590098)(289.68589569,1058.88590098)
\curveto(288.66544942,1058.88590098)(287.76322558,1059.07256798)(286.97922417,1059.44590199)
\curveto(286.19522277,1059.83168046)(285.57922166,1060.40412593)(285.13122086,1061.1632384)
\curveto(284.68322006,1061.93479534)(284.45921965,1062.91168598)(284.45921965,1064.09391032)
\curveto(284.45921965,1065.27613466)(284.65833112,1066.26546977)(285.05655406,1067.06191564)
\curveto(285.46722146,1067.85836151)(286.0334447,1068.45569592)(286.75522377,1068.85391885)
\curveto(287.47700284,1069.25214179)(288.31700435,1069.45125326)(289.27522829,1069.45125326)
\closepath
\moveto(289.29389499,1067.47258304)
\curveto(288.74633845,1067.47258304)(288.29833765,1067.29836051)(287.94989258,1066.94991544)
\curveto(287.60144751,1066.60147037)(287.39611381,1066.06013607)(287.33389147,1065.32591253)
\lineto(291.2352318,1065.32591253)
\curveto(291.22278733,1065.9356914)(291.05478703,1066.44591453)(290.7312309,1066.85658194)
\curveto(290.42011923,1067.26724934)(289.94100726,1067.47258304)(289.29389499,1067.47258304)
\closepath
}
}
{
\newrgbcolor{curcolor}{0 0 0}
\pscustom[linestyle=none,fillstyle=solid,fillcolor=curcolor]
{
\newpath
\moveto(300.25124099,1058.88590098)
\curveto(298.73301604,1058.88590098)(297.55701394,1059.30279062)(296.72323466,1060.13656989)
\curveto(295.90189986,1060.97034916)(295.49123246,1062.29568487)(295.49123246,1064.11257702)
\curveto(295.49123246,1065.35702369)(295.70278839,1066.37124773)(296.12590026,1067.15524914)
\curveto(296.54901213,1067.93925054)(297.13390207,1068.51791825)(297.88057007,1068.89125225)
\curveto(298.63968254,1069.26458625)(299.51079522,1069.45125326)(300.49390809,1069.45125326)
\curveto(301.19079823,1069.45125326)(301.79435487,1069.38280869)(302.304578,1069.24591955)
\curveto(302.82724561,1069.10903042)(303.28146864,1068.94725235)(303.66724711,1068.76058535)
\lineto(302.84591231,1066.61391484)
\curveto(302.41035597,1066.78813737)(301.99968857,1066.93124874)(301.6139101,1067.04324894)
\curveto(301.24057609,1067.15524914)(300.86724209,1067.21124924)(300.49390809,1067.21124924)
\curveto(299.05034995,1067.21124924)(298.32857087,1066.18458073)(298.32857087,1064.13124372)
\curveto(298.32857087,1063.11079745)(298.51523788,1062.35790721)(298.88857188,1061.872573)
\curveto(299.27435035,1061.3872388)(299.80946242,1061.1445717)(300.49390809,1061.1445717)
\curveto(301.07879803,1061.1445717)(301.5952434,1061.2192385)(302.0432442,1061.3685721)
\curveto(302.491245,1061.53035017)(302.92680134,1061.74812834)(303.34991321,1062.0219066)
\lineto(303.34991321,1059.65123569)
\curveto(302.92680134,1059.37745742)(302.47880054,1059.18456818)(302.0059108,1059.07256798)
\curveto(301.54546553,1058.94812332)(300.96057559,1058.88590098)(300.25124099,1058.88590098)
\closepath
}
}
{
\newrgbcolor{curcolor}{0 0 0}
\pscustom[linestyle=none,fillstyle=solid,fillcolor=curcolor]
{
\newpath
\moveto(309.99525525,1061.1072383)
\curveto(310.30636692,1061.1072383)(310.60503412,1061.13212723)(310.89125686,1061.1819051)
\curveto(311.1774796,1061.24412743)(311.46370233,1061.32501647)(311.74992507,1061.4245722)
\lineto(311.74992507,1059.35256849)
\curveto(311.45125786,1059.21567935)(311.07792386,1059.10367915)(310.62992306,1059.01656788)
\curveto(310.19436672,1058.92945662)(309.71525475,1058.88590098)(309.19258715,1058.88590098)
\curveto(308.58280828,1058.88590098)(308.03525174,1058.98545672)(307.54991754,1059.18456818)
\curveto(307.0770278,1059.38367965)(306.69747156,1059.72590249)(306.41124883,1060.21123669)
\curveto(306.13747056,1060.6965709)(306.00058143,1061.38101657)(306.00058143,1062.26457371)
\lineto(306.00058143,1067.17391584)
\lineto(304.67524572,1067.17391584)
\lineto(304.67524572,1068.34991795)
\lineto(306.20591513,1069.28325295)
\lineto(307.00858323,1071.42992347)
\lineto(308.78191975,1071.42992347)
\lineto(308.78191975,1069.26458625)
\lineto(311.63792487,1069.26458625)
\lineto(311.63792487,1067.17391584)
\lineto(308.78191975,1067.17391584)
\lineto(308.78191975,1062.26457371)
\curveto(308.78191975,1061.87879524)(308.89391995,1061.58635027)(309.11792035,1061.3872388)
\curveto(309.34192075,1061.2005718)(309.63436572,1061.1072383)(309.99525525,1061.1072383)
\closepath
}
}
{
\newrgbcolor{curcolor}{0 0 0}
\pscustom[linestyle=none,fillstyle=solid,fillcolor=curcolor]
{
\newpath
\moveto(313.4112623,1060.37923699)
\curveto(313.4112623,1060.95168246)(313.56681813,1061.3499054)(313.8779298,1061.5739058)
\curveto(314.18904147,1061.81035067)(314.5685977,1061.9285731)(315.01659851,1061.9285731)
\curveto(315.45215484,1061.9285731)(315.82548885,1061.81035067)(316.13660052,1061.5739058)
\curveto(316.44771218,1061.3499054)(316.60326802,1060.95168246)(316.60326802,1060.37923699)
\curveto(316.60326802,1059.83168046)(316.44771218,1059.43345752)(316.13660052,1059.18456818)
\curveto(315.82548885,1058.94812332)(315.45215484,1058.82990088)(315.01659851,1058.82990088)
\curveto(314.5685977,1058.82990088)(314.18904147,1058.94812332)(313.8779298,1059.18456818)
\curveto(313.56681813,1059.43345752)(313.4112623,1059.83168046)(313.4112623,1060.37923699)
\closepath
}
}
{
\newrgbcolor{curcolor}{0 0 0}
\pscustom[linestyle=none,fillstyle=solid,fillcolor=curcolor]
{
\newpath
\moveto(319.01127006,1071.89659097)
\curveto(319.01127006,1072.41925858)(319.15438142,1072.77392588)(319.44060416,1072.96059288)
\curveto(319.73927136,1073.15970435)(320.1001609,1073.25926008)(320.52327277,1073.25926008)
\curveto(320.93394017,1073.25926008)(321.28860747,1073.15970435)(321.58727467,1072.96059288)
\curveto(321.88594188,1072.77392588)(322.03527548,1072.41925858)(322.03527548,1071.89659097)
\curveto(322.03527548,1071.38636784)(321.88594188,1071.03170053)(321.58727467,1070.83258906)
\curveto(321.28860747,1070.6334776)(320.93394017,1070.53392186)(320.52327277,1070.53392186)
\curveto(320.1001609,1070.53392186)(319.73927136,1070.6334776)(319.44060416,1070.83258906)
\curveto(319.15438142,1071.03170053)(319.01127006,1071.38636784)(319.01127006,1071.89659097)
\closepath
\moveto(318.30193545,1054.59255995)
\curveto(317.97837932,1054.59255995)(317.64860095,1054.61744889)(317.31260034,1054.66722675)
\curveto(316.97659974,1054.70456015)(316.69659924,1054.75433802)(316.47259884,1054.81656036)
\lineto(316.47259884,1057.00056427)
\curveto(316.69659924,1056.93834194)(316.90815517,1056.8947863)(317.10726664,1056.86989737)
\curveto(317.30637811,1056.84500844)(317.53037851,1056.83256397)(317.77926785,1056.83256397)
\curveto(318.15260185,1056.83256397)(318.46993575,1056.93834194)(318.73126955,1057.14989787)
\curveto(318.99260336,1057.36145381)(319.12327026,1057.77212121)(319.12327026,1058.38190008)
\lineto(319.12327026,1069.26458625)
\lineto(321.90460858,1069.26458625)
\lineto(321.90460858,1057.97123268)
\curveto(321.90460858,1057.34900934)(321.78638614,1056.7827861)(321.54994127,1056.27256297)
\curveto(321.31349641,1055.76233983)(320.92771794,1055.35789466)(320.39260587,1055.05922746)
\curveto(319.86993826,1054.74811579)(319.17304812,1054.59255995)(318.30193545,1054.59255995)
\closepath
}
}
{
\newrgbcolor{curcolor}{0 0 0}
\pscustom[linestyle=none,fillstyle=solid,fillcolor=curcolor]
{
\newpath
\moveto(331.92863468,1062.0965734)
\curveto(331.92863468,1061.06368266)(331.56152291,1060.26723679)(330.82729937,1059.70723579)
\curveto(330.1055203,1059.15967925)(329.02285169,1058.88590098)(327.57929355,1058.88590098)
\curveto(326.86995894,1058.88590098)(326.26018007,1058.93567885)(325.74995694,1059.03523458)
\curveto(325.2397338,1059.12234585)(324.72951066,1059.27167945)(324.21928753,1059.48323539)
\lineto(324.21928753,1061.7792395)
\curveto(324.76684406,1061.53035017)(325.35795623,1061.32501647)(325.99262404,1061.1632384)
\curveto(326.62729184,1061.00146033)(327.18729285,1060.9205713)(327.67262705,1060.9205713)
\curveto(328.20773912,1060.9205713)(328.59351759,1061.00146033)(328.82996246,1061.1632384)
\curveto(329.06640733,1061.32501647)(329.18462976,1061.5365724)(329.18462976,1061.7979062)
\curveto(329.18462976,1061.97212874)(329.13485189,1062.12768457)(329.03529616,1062.26457371)
\curveto(328.94818489,1062.40146284)(328.74907342,1062.55701867)(328.43796176,1062.73124121)
\curveto(328.12685009,1062.90546374)(327.64151588,1063.12946415)(326.98195915,1063.40324241)
\curveto(326.33484687,1063.67702068)(325.80595704,1063.94457672)(325.39528963,1064.20591052)
\curveto(324.9970667,1064.47968879)(324.6983995,1064.80324492)(324.49928803,1065.17657893)
\curveto(324.30017656,1065.5623574)(324.20062083,1066.04146937)(324.20062083,1066.61391484)
\curveto(324.20062083,1067.55969431)(324.5677326,1068.26902891)(325.30195613,1068.74191865)
\curveto(326.03617967,1069.21480839)(327.01307031,1069.45125326)(328.23262805,1069.45125326)
\curveto(328.86729586,1069.45125326)(329.4708525,1069.38903092)(330.04329797,1069.26458625)
\curveto(330.61574344,1069.14014159)(331.20685561,1068.93480789)(331.81663448,1068.64858515)
\lineto(330.97663297,1066.65124824)
\curveto(330.4788543,1066.86280417)(330.00596457,1067.03702671)(329.55796376,1067.17391584)
\curveto(329.10996296,1067.32324944)(328.65573992,1067.39791624)(328.19529465,1067.39791624)
\curveto(327.37395985,1067.39791624)(326.96329245,1067.17391584)(326.96329245,1066.72591504)
\curveto(326.96329245,1066.56413697)(327.01307031,1066.41480337)(327.11262605,1066.27791423)
\curveto(327.22462625,1066.15346957)(327.42995995,1066.01658043)(327.72862715,1065.86724683)
\curveto(328.03973882,1065.71791323)(328.49396186,1065.51880176)(329.09129626,1065.26991243)
\curveto(329.6761862,1065.03346756)(330.1801871,1064.78457822)(330.60329897,1064.52324442)
\curveto(331.02641084,1064.27435509)(331.34996698,1063.95702118)(331.57396738,1063.57124272)
\curveto(331.81041225,1063.18546425)(331.92863468,1062.69390781)(331.92863468,1062.0965734)
\closepath
}
}
{
\newrgbcolor{curcolor}{0 0 0}
\pscustom[linestyle=none,fillstyle=solid,fillcolor=curcolor]
{
\newpath
\moveto(224.30408314,1048.25646131)
\curveto(225.46141855,1048.25646131)(226.33253122,1047.9577941)(226.91742116,1047.3604597)
\curveto(227.51475556,1046.77556976)(227.81342276,1045.82979029)(227.81342276,1044.52312128)
\lineto(227.81342276,1037.87777603)
\lineto(225.03208444,1037.87777603)
\lineto(225.03208444,1043.83245337)
\curveto(225.03208444,1045.30090045)(224.52186131,1046.03512399)(223.50141503,1046.03512399)
\curveto(222.76719149,1046.03512399)(222.24452389,1045.77379019)(221.93341222,1045.25112258)
\curveto(221.62230055,1044.72845498)(221.46674472,1043.97556474)(221.46674472,1042.99245187)
\lineto(221.46674472,1037.87777603)
\lineto(218.6854064,1037.87777603)
\lineto(218.6854064,1043.83245337)
\curveto(218.6854064,1045.30090045)(218.17518326,1046.03512399)(217.15473699,1046.03512399)
\curveto(216.38318005,1046.03512399)(215.84806798,1045.74267902)(215.54940078,1045.15778908)
\curveto(215.26317804,1044.58534361)(215.12006668,1043.75778657)(215.12006668,1042.67511797)
\lineto(215.12006668,1037.87777603)
\lineto(212.33872836,1037.87777603)
\lineto(212.33872836,1048.0697943)
\lineto(214.46673217,1048.0697943)
\lineto(214.84006617,1046.76312529)
\lineto(214.98939977,1046.76312529)
\curveto(215.30051144,1047.2857929)(215.72362331,1047.66534913)(216.25873538,1047.901794)
\curveto(216.80629192,1048.13823887)(217.37251516,1048.25646131)(217.9574051,1048.25646131)
\curveto(218.7040731,1048.25646131)(219.33251867,1048.13201664)(219.84274181,1047.8831273)
\curveto(220.36540941,1047.64668243)(220.76985458,1047.27334843)(221.05607732,1046.76312529)
\lineto(221.29874442,1046.76312529)
\curveto(221.60985609,1047.2857929)(222.03919019,1047.66534913)(222.58674673,1047.901794)
\curveto(223.14674773,1048.13823887)(223.7191932,1048.25646131)(224.30408314,1048.25646131)
\closepath
}
}
{
\newrgbcolor{curcolor}{0 0 0}
\pscustom[linestyle=none,fillstyle=solid,fillcolor=curcolor]
{
\newpath
\moveto(234.8694413,1048.25646131)
\curveto(236.27566604,1048.25646131)(237.38944581,1047.85201614)(238.21078062,1047.0431258)
\curveto(239.03211543,1046.24667992)(239.44278283,1045.10801122)(239.44278283,1043.62711967)
\lineto(239.44278283,1042.28311726)
\lineto(232.87210438,1042.28311726)
\curveto(232.89699332,1041.49911586)(233.12721595,1040.88311475)(233.56277229,1040.43511395)
\curveto(234.01077309,1039.98711315)(234.6267742,1039.76311275)(235.4107756,1039.76311275)
\curveto(236.05788787,1039.76311275)(236.64900004,1039.82533508)(237.18411211,1039.94977975)
\curveto(237.73166865,1040.08666888)(238.29166965,1040.29200258)(238.86411512,1040.56578085)
\lineto(238.86411512,1038.41911034)
\curveto(238.35389199,1038.170221)(237.82500215,1037.98977623)(237.27744561,1037.87777603)
\curveto(236.72988908,1037.75333137)(236.06411011,1037.69110903)(235.2801087,1037.69110903)
\curveto(234.25966243,1037.69110903)(233.35743859,1037.87777603)(232.57343718,1038.25111004)
\curveto(231.78943578,1038.63688851)(231.17343467,1039.20933398)(230.72543387,1039.96844645)
\curveto(230.27743307,1040.74000339)(230.05343266,1041.71689403)(230.05343266,1042.89911837)
\curveto(230.05343266,1044.08134271)(230.25254413,1045.07067782)(230.65076707,1045.86712369)
\curveto(231.06143447,1046.66356956)(231.62765771,1047.26090396)(232.34943678,1047.6591269)
\curveto(233.07121585,1048.05734984)(233.91121736,1048.25646131)(234.8694413,1048.25646131)
\closepath
\moveto(234.888108,1046.27779109)
\curveto(234.34055146,1046.27779109)(233.89255066,1046.10356856)(233.54410559,1045.75512349)
\curveto(233.19566052,1045.40667842)(232.99032682,1044.86534412)(232.92810448,1044.13112058)
\lineto(236.82944481,1044.13112058)
\curveto(236.81700034,1044.74089945)(236.64900004,1045.25112258)(236.32544391,1045.66178999)
\curveto(236.01433224,1046.07245739)(235.53522027,1046.27779109)(234.888108,1046.27779109)
\closepath
}
}
{
\newrgbcolor{curcolor}{0 0 0}
\pscustom[linestyle=none,fillstyle=solid,fillcolor=curcolor]
{
\newpath
\moveto(248.81345932,1040.90178145)
\curveto(248.81345932,1039.86889071)(248.44634755,1039.07244484)(247.71212401,1038.51244384)
\curveto(246.99034494,1037.9648873)(245.90767633,1037.69110903)(244.46411819,1037.69110903)
\curveto(243.75478358,1037.69110903)(243.14500471,1037.7408869)(242.63478158,1037.84044263)
\curveto(242.12455844,1037.9275539)(241.6143353,1038.0768875)(241.10411216,1038.28844344)
\lineto(241.10411216,1040.58444755)
\curveto(241.6516687,1040.33555822)(242.24278087,1040.13022452)(242.87744868,1039.96844645)
\curveto(243.51211648,1039.80666838)(244.07211749,1039.72577935)(244.55745169,1039.72577935)
\curveto(245.09256376,1039.72577935)(245.47834223,1039.80666838)(245.7147871,1039.96844645)
\curveto(245.95123196,1040.13022452)(246.0694544,1040.34178045)(246.0694544,1040.60311425)
\curveto(246.0694544,1040.77733679)(246.01967653,1040.93289262)(245.9201208,1041.06978176)
\curveto(245.83300953,1041.20667089)(245.63389806,1041.36222672)(245.32278639,1041.53644926)
\curveto(245.01167473,1041.71067179)(244.52634052,1041.9346722)(243.86678378,1042.20845046)
\curveto(243.21967151,1042.48222873)(242.69078168,1042.74978477)(242.28011427,1043.01111857)
\curveto(241.88189134,1043.28489684)(241.58322413,1043.60845297)(241.38411267,1043.98178698)
\curveto(241.1850012,1044.36756545)(241.08544546,1044.84667742)(241.08544546,1045.41912289)
\curveto(241.08544546,1046.36490236)(241.45255723,1047.07423696)(242.18678077,1047.5471267)
\curveto(242.92100431,1048.02001644)(243.89789495,1048.25646131)(245.11745269,1048.25646131)
\curveto(245.7521205,1048.25646131)(246.35567713,1048.19423897)(246.9281226,1048.0697943)
\curveto(247.50056808,1047.94534964)(248.09168025,1047.74001593)(248.70145912,1047.4537932)
\lineto(247.86145761,1045.45645629)
\curveto(247.36367894,1045.66801222)(246.8907892,1045.84223476)(246.4427884,1045.97912389)
\curveto(245.9947876,1046.12845749)(245.54056456,1046.20312429)(245.08011929,1046.20312429)
\curveto(244.25878449,1046.20312429)(243.84811708,1045.97912389)(243.84811708,1045.53112309)
\curveto(243.84811708,1045.36934502)(243.89789495,1045.22001142)(243.99745068,1045.08312228)
\curveto(244.10945089,1044.95867762)(244.31478459,1044.82178848)(244.61345179,1044.67245488)
\curveto(244.92456346,1044.52312128)(245.37878649,1044.32400981)(245.9761209,1044.07512048)
\curveto(246.56101084,1043.83867561)(247.06501174,1043.58978627)(247.48812361,1043.32845247)
\curveto(247.91123548,1043.07956314)(248.23479161,1042.76222923)(248.45879202,1042.37645076)
\curveto(248.69523688,1041.9906723)(248.81345932,1041.49911586)(248.81345932,1040.90178145)
\closepath
}
}
{
\newrgbcolor{curcolor}{0 0 0}
\pscustom[linestyle=none,fillstyle=solid,fillcolor=curcolor]
{
\newpath
\moveto(258.09080486,1040.90178145)
\curveto(258.09080486,1039.86889071)(257.72369309,1039.07244484)(256.98946955,1038.51244384)
\curveto(256.26769048,1037.9648873)(255.18502187,1037.69110903)(253.74146373,1037.69110903)
\curveto(253.03212912,1037.69110903)(252.42235025,1037.7408869)(251.91212712,1037.84044263)
\curveto(251.40190398,1037.9275539)(250.89168084,1038.0768875)(250.38145771,1038.28844344)
\lineto(250.38145771,1040.58444755)
\curveto(250.92901424,1040.33555822)(251.52012641,1040.13022452)(252.15479422,1039.96844645)
\curveto(252.78946202,1039.80666838)(253.34946303,1039.72577935)(253.83479723,1039.72577935)
\curveto(254.3699093,1039.72577935)(254.75568777,1039.80666838)(254.99213264,1039.96844645)
\curveto(255.22857751,1040.13022452)(255.34679994,1040.34178045)(255.34679994,1040.60311425)
\curveto(255.34679994,1040.77733679)(255.29702207,1040.93289262)(255.19746634,1041.06978176)
\curveto(255.11035507,1041.20667089)(254.9112436,1041.36222672)(254.60013193,1041.53644926)
\curveto(254.28902027,1041.71067179)(253.80368606,1041.9346722)(253.14412932,1042.20845046)
\curveto(252.49701705,1042.48222873)(251.96812722,1042.74978477)(251.55745981,1043.01111857)
\curveto(251.15923688,1043.28489684)(250.86056968,1043.60845297)(250.66145821,1043.98178698)
\curveto(250.46234674,1044.36756545)(250.36279101,1044.84667742)(250.36279101,1045.41912289)
\curveto(250.36279101,1046.36490236)(250.72990277,1047.07423696)(251.46412631,1047.5471267)
\curveto(252.19834985,1048.02001644)(253.17524049,1048.25646131)(254.39479823,1048.25646131)
\curveto(255.02946604,1048.25646131)(255.63302267,1048.19423897)(256.20546815,1048.0697943)
\curveto(256.77791362,1047.94534964)(257.36902579,1047.74001593)(257.97880466,1047.4537932)
\lineto(257.13880315,1045.45645629)
\curveto(256.64102448,1045.66801222)(256.16813475,1045.84223476)(255.72013394,1045.97912389)
\curveto(255.27213314,1046.12845749)(254.8179101,1046.20312429)(254.35746483,1046.20312429)
\curveto(253.53613003,1046.20312429)(253.12546262,1045.97912389)(253.12546262,1045.53112309)
\curveto(253.12546262,1045.36934502)(253.17524049,1045.22001142)(253.27479623,1045.08312228)
\curveto(253.38679643,1044.95867762)(253.59213013,1044.82178848)(253.89079733,1044.67245488)
\curveto(254.201909,1044.52312128)(254.65613203,1044.32400981)(255.25346644,1044.07512048)
\curveto(255.83835638,1043.83867561)(256.34235728,1043.58978627)(256.76546915,1043.32845247)
\curveto(257.18858102,1043.07956314)(257.51213715,1042.76222923)(257.73613756,1042.37645076)
\curveto(257.97258242,1041.9906723)(258.09080486,1041.49911586)(258.09080486,1040.90178145)
\closepath
}
}
{
\newrgbcolor{curcolor}{0 0 0}
\pscustom[linestyle=none,fillstyle=solid,fillcolor=curcolor]
{
\newpath
\moveto(264.43748039,1048.27512801)
\curveto(265.80637173,1048.27512801)(266.85170694,1047.9764608)(267.57348601,1047.3791264)
\curveto(268.30770955,1046.79423646)(268.67482132,1045.89201262)(268.67482132,1044.67245488)
\lineto(268.67482132,1037.87777603)
\lineto(266.7334845,1037.87777603)
\lineto(266.1921502,1039.25911184)
\lineto(266.1174834,1039.25911184)
\curveto(265.68192706,1038.71155531)(265.22148179,1038.31333237)(264.73614759,1038.06444304)
\curveto(264.25081338,1037.8155537)(263.58503441,1037.69110903)(262.73881067,1037.69110903)
\curveto(261.8303646,1037.69110903)(261.07747436,1037.95244283)(260.48013996,1038.47511044)
\curveto(259.88280555,1038.99777804)(259.58413835,1039.81289061)(259.58413835,1040.92044815)
\curveto(259.58413835,1042.00311676)(259.96369459,1042.79956263)(260.72280706,1043.30978577)
\curveto(261.48191953,1043.82000891)(262.62058824,1044.10623164)(264.13881318,1044.16845398)
\lineto(265.9121497,1044.22445408)
\lineto(265.9121497,1044.67245488)
\curveto(265.9121497,1045.20756695)(265.76903833,1045.59956765)(265.48281559,1045.84845699)
\curveto(265.20903732,1046.09734632)(264.82325886,1046.22179099)(264.32548019,1046.22179099)
\curveto(263.82770152,1046.22179099)(263.34236731,1046.14712419)(262.86947758,1045.99779059)
\curveto(262.39658784,1045.86090146)(261.9236981,1045.68667892)(261.45080837,1045.47512299)
\lineto(260.53614006,1047.3604597)
\curveto(261.07125213,1047.63423797)(261.67480877,1047.85201614)(262.34680997,1048.0137942)
\curveto(263.01881118,1048.18801674)(263.71570131,1048.27512801)(264.43748039,1048.27512801)
\closepath
\moveto(265.9121497,1042.60045117)
\lineto(264.82948109,1042.56311777)
\curveto(263.93347948,1042.53822883)(263.31125614,1042.37645076)(262.96281108,1042.07778356)
\curveto(262.61436601,1041.77911636)(262.44014347,1041.38711566)(262.44014347,1040.90178145)
\curveto(262.44014347,1040.47866959)(262.56458814,1040.17378015)(262.81347747,1039.98711315)
\curveto(263.06236681,1039.81289061)(263.38592295,1039.72577935)(263.78414588,1039.72577935)
\curveto(264.38148029,1039.72577935)(264.88548119,1039.90000188)(265.29614859,1040.24844695)
\curveto(265.70681599,1040.60933649)(265.9121497,1041.11333739)(265.9121497,1041.76044966)
\closepath
}
}
{
\newrgbcolor{curcolor}{0 0 0}
\pscustom[linestyle=none,fillstyle=solid,fillcolor=curcolor]
{
\newpath
\moveto(274.83483406,1048.25646131)
\curveto(276.0917252,1048.25646131)(277.07483807,1047.75868264)(277.78417268,1046.76312529)
\lineto(277.85883948,1046.76312529)
\lineto(278.08283988,1048.0697943)
\lineto(280.4348441,1048.0697943)
\lineto(280.4348441,1037.85910933)
\curveto(280.4348441,1036.40310672)(280.00550999,1035.29554918)(279.14684179,1034.53643671)
\curveto(278.28817358,1033.77732424)(277.01883797,1033.397768)(275.33883496,1033.397768)
\curveto(274.61705589,1033.397768)(273.94505469,1033.44132364)(273.32283135,1033.5284349)
\curveto(272.71305248,1033.61554617)(272.11571807,1033.77110201)(271.53082814,1033.99510241)
\lineto(271.53082814,1036.21643972)
\curveto(272.77527481,1035.69377212)(274.10061052,1035.43243832)(275.50683526,1035.43243832)
\curveto(276.93794894,1035.43243832)(277.65350578,1036.20399526)(277.65350578,1037.74710913)
\lineto(277.65350578,1037.95244283)
\curveto(277.65350578,1038.1515543)(277.65972801,1038.36311024)(277.67217248,1038.58711064)
\curveto(277.68461694,1038.82355551)(277.70328364,1039.02888921)(277.72817258,1039.20311174)
\lineto(277.65350578,1039.20311174)
\curveto(277.30506071,1038.66799967)(276.88817107,1038.2822212)(276.40283687,1038.04577634)
\curveto(275.91750267,1037.80933147)(275.36994613,1037.69110903)(274.76016726,1037.69110903)
\curveto(273.55305398,1037.69110903)(272.60727451,1038.1515543)(271.92282884,1039.07244484)
\curveto(271.25082763,1040.00577985)(270.91482703,1041.30000439)(270.91482703,1042.95511847)
\curveto(270.91482703,1044.62267701)(271.2632721,1045.92312379)(271.96016224,1046.8564588)
\curveto(272.65705238,1047.7897938)(273.61527632,1048.25646131)(274.83483406,1048.25646131)
\closepath
\moveto(275.71216896,1045.99779059)
\curveto(274.40549996,1045.99779059)(273.75216545,1044.97112208)(273.75216545,1042.91778507)
\curveto(273.75216545,1040.88933699)(274.41794442,1039.87511295)(275.74950236,1039.87511295)
\curveto(276.45883697,1039.87511295)(276.98150457,1040.07422442)(277.31750518,1040.47244735)
\curveto(277.66595024,1040.88311475)(277.84017278,1041.59244936)(277.84017278,1042.60045117)
\lineto(277.84017278,1042.93645177)
\curveto(277.84017278,1044.03156484)(277.67217248,1044.81556625)(277.33617188,1045.28845598)
\curveto(277.00017127,1045.76134572)(276.45883697,1045.99779059)(275.71216896,1045.99779059)
\closepath
}
}
{
\newrgbcolor{curcolor}{0 0 0}
\pscustom[linestyle=none,fillstyle=solid,fillcolor=curcolor]
{
\newpath
\moveto(287.54684504,1048.25646131)
\curveto(288.95306978,1048.25646131)(290.06684955,1047.85201614)(290.88818436,1047.0431258)
\curveto(291.70951917,1046.24667992)(292.12018657,1045.10801122)(292.12018657,1043.62711967)
\lineto(292.12018657,1042.28311726)
\lineto(285.54950812,1042.28311726)
\curveto(285.57439706,1041.49911586)(285.80461969,1040.88311475)(286.24017603,1040.43511395)
\curveto(286.68817683,1039.98711315)(287.30417794,1039.76311275)(288.08817934,1039.76311275)
\curveto(288.73529161,1039.76311275)(289.32640378,1039.82533508)(289.86151585,1039.94977975)
\curveto(290.40907239,1040.08666888)(290.96907339,1040.29200258)(291.54151886,1040.56578085)
\lineto(291.54151886,1038.41911034)
\curveto(291.03129573,1038.170221)(290.50240589,1037.98977623)(289.95484935,1037.87777603)
\curveto(289.40729282,1037.75333137)(288.74151385,1037.69110903)(287.95751244,1037.69110903)
\curveto(286.93706617,1037.69110903)(286.03484233,1037.87777603)(285.25084092,1038.25111004)
\curveto(284.46683952,1038.63688851)(283.85083841,1039.20933398)(283.40283761,1039.96844645)
\curveto(282.95483681,1040.74000339)(282.7308364,1041.71689403)(282.7308364,1042.89911837)
\curveto(282.7308364,1044.08134271)(282.92994787,1045.07067782)(283.32817081,1045.86712369)
\curveto(283.73883821,1046.66356956)(284.30506145,1047.26090396)(285.02684052,1047.6591269)
\curveto(285.74861959,1048.05734984)(286.5886211,1048.25646131)(287.54684504,1048.25646131)
\closepath
\moveto(287.56551174,1046.27779109)
\curveto(287.0179552,1046.27779109)(286.5699544,1046.10356856)(286.22150933,1045.75512349)
\curveto(285.87306426,1045.40667842)(285.66773056,1044.86534412)(285.60550822,1044.13112058)
\lineto(289.50684855,1044.13112058)
\curveto(289.49440408,1044.74089945)(289.32640378,1045.25112258)(289.00284765,1045.66178999)
\curveto(288.69173598,1046.07245739)(288.21262401,1046.27779109)(287.56551174,1046.27779109)
\closepath
}
}
{
\newrgbcolor{curcolor}{0 0 0}
\pscustom[linestyle=none,fillstyle=solid,fillcolor=curcolor]
{
\newpath
\moveto(301.49086306,1040.90178145)
\curveto(301.49086306,1039.86889071)(301.12375129,1039.07244484)(300.38952775,1038.51244384)
\curveto(299.66774868,1037.9648873)(298.58508007,1037.69110903)(297.14152193,1037.69110903)
\curveto(296.43218732,1037.69110903)(295.82240845,1037.7408869)(295.31218532,1037.84044263)
\curveto(294.80196218,1037.9275539)(294.29173904,1038.0768875)(293.7815159,1038.28844344)
\lineto(293.7815159,1040.58444755)
\curveto(294.32907244,1040.33555822)(294.92018461,1040.13022452)(295.55485242,1039.96844645)
\curveto(296.18952022,1039.80666838)(296.74952123,1039.72577935)(297.23485543,1039.72577935)
\curveto(297.7699675,1039.72577935)(298.15574597,1039.80666838)(298.39219084,1039.96844645)
\curveto(298.6286357,1040.13022452)(298.74685814,1040.34178045)(298.74685814,1040.60311425)
\curveto(298.74685814,1040.77733679)(298.69708027,1040.93289262)(298.59752454,1041.06978176)
\curveto(298.51041327,1041.20667089)(298.3113018,1041.36222672)(298.00019013,1041.53644926)
\curveto(297.68907847,1041.71067179)(297.20374426,1041.9346722)(296.54418752,1042.20845046)
\curveto(295.89707525,1042.48222873)(295.36818542,1042.74978477)(294.95751801,1043.01111857)
\curveto(294.55929508,1043.28489684)(294.26062787,1043.60845297)(294.06151641,1043.98178698)
\curveto(293.86240494,1044.36756545)(293.7628492,1044.84667742)(293.7628492,1045.41912289)
\curveto(293.7628492,1046.36490236)(294.12996097,1047.07423696)(294.86418451,1047.5471267)
\curveto(295.59840805,1048.02001644)(296.57529869,1048.25646131)(297.79485643,1048.25646131)
\curveto(298.42952424,1048.25646131)(299.03308087,1048.19423897)(299.60552634,1048.0697943)
\curveto(300.17797182,1047.94534964)(300.76908399,1047.74001593)(301.37886286,1047.4537932)
\lineto(300.53886135,1045.45645629)
\curveto(300.04108268,1045.66801222)(299.56819294,1045.84223476)(299.12019214,1045.97912389)
\curveto(298.67219134,1046.12845749)(298.2179683,1046.20312429)(297.75752303,1046.20312429)
\curveto(296.93618823,1046.20312429)(296.52552082,1045.97912389)(296.52552082,1045.53112309)
\curveto(296.52552082,1045.36934502)(296.57529869,1045.22001142)(296.67485442,1045.08312228)
\curveto(296.78685463,1044.95867762)(296.99218833,1044.82178848)(297.29085553,1044.67245488)
\curveto(297.6019672,1044.52312128)(298.05619023,1044.32400981)(298.65352464,1044.07512048)
\curveto(299.23841458,1043.83867561)(299.74241548,1043.58978627)(300.16552735,1043.32845247)
\curveto(300.58863922,1043.07956314)(300.91219535,1042.76222923)(301.13619576,1042.37645076)
\curveto(301.37264062,1041.9906723)(301.49086306,1041.49911586)(301.49086306,1040.90178145)
\closepath
}
}
{
\newrgbcolor{curcolor}{0 0 0}
\pscustom[linestyle=none,fillstyle=solid,fillcolor=curcolor]
{
\newpath
\moveto(303.26419324,1039.18444504)
\curveto(303.26419324,1039.75689051)(303.41974907,1040.15511345)(303.73086074,1040.37911385)
\curveto(304.04197241,1040.61555872)(304.42152865,1040.73378115)(304.86952945,1040.73378115)
\curveto(305.30508579,1040.73378115)(305.67841979,1040.61555872)(305.98953146,1040.37911385)
\curveto(306.30064313,1040.15511345)(306.45619896,1039.75689051)(306.45619896,1039.18444504)
\curveto(306.45619896,1038.63688851)(306.30064313,1038.23866557)(305.98953146,1037.98977623)
\curveto(305.67841979,1037.75333137)(305.30508579,1037.63510893)(304.86952945,1037.63510893)
\curveto(304.42152865,1037.63510893)(304.04197241,1037.75333137)(303.73086074,1037.98977623)
\curveto(303.41974907,1038.23866557)(303.26419324,1038.63688851)(303.26419324,1039.18444504)
\closepath
}
}
{
\newrgbcolor{curcolor}{0 0 0}
\pscustom[linestyle=none,fillstyle=solid,fillcolor=curcolor]
{
\newpath
\moveto(308.864201,1050.70179902)
\curveto(308.864201,1051.22446663)(309.00731237,1051.57913393)(309.2935351,1051.76580093)
\curveto(309.5922023,1051.9649124)(309.95309184,1052.06446813)(310.37620371,1052.06446813)
\curveto(310.78687111,1052.06446813)(311.14153841,1051.9649124)(311.44020562,1051.76580093)
\curveto(311.73887282,1051.57913393)(311.88820642,1051.22446663)(311.88820642,1050.70179902)
\curveto(311.88820642,1050.19157589)(311.73887282,1049.83690858)(311.44020562,1049.63779711)
\curveto(311.14153841,1049.43868565)(310.78687111,1049.33912991)(310.37620371,1049.33912991)
\curveto(309.95309184,1049.33912991)(309.5922023,1049.43868565)(309.2935351,1049.63779711)
\curveto(309.00731237,1049.83690858)(308.864201,1050.19157589)(308.864201,1050.70179902)
\closepath
\moveto(308.15486639,1033.397768)
\curveto(307.83131026,1033.397768)(307.50153189,1033.42265694)(307.16553129,1033.4724348)
\curveto(306.82953068,1033.5097682)(306.54953018,1033.55954607)(306.32552978,1033.6217684)
\lineto(306.32552978,1035.80577232)
\curveto(306.54953018,1035.74354999)(306.76108612,1035.69999435)(306.96019759,1035.67510542)
\curveto(307.15930905,1035.65021649)(307.38330946,1035.63777202)(307.63219879,1035.63777202)
\curveto(308.00553279,1035.63777202)(308.32286669,1035.74354999)(308.5842005,1035.95510592)
\curveto(308.8455343,1036.16666186)(308.9762012,1036.57732926)(308.9762012,1037.18710813)
\lineto(308.9762012,1048.0697943)
\lineto(311.75753952,1048.0697943)
\lineto(311.75753952,1036.77644073)
\curveto(311.75753952,1036.15421739)(311.63931708,1035.58799415)(311.40287222,1035.07777101)
\curveto(311.16642735,1034.56754788)(310.78064888,1034.16310271)(310.24553681,1033.86443551)
\curveto(309.7228692,1033.55332384)(309.02597907,1033.397768)(308.15486639,1033.397768)
\closepath
}
}
{
\newrgbcolor{curcolor}{0 0 0}
\pscustom[linestyle=none,fillstyle=solid,fillcolor=curcolor]
{
\newpath
\moveto(321.78156562,1040.90178145)
\curveto(321.78156562,1039.86889071)(321.41445385,1039.07244484)(320.68023031,1038.51244384)
\curveto(319.95845124,1037.9648873)(318.87578264,1037.69110903)(317.43222449,1037.69110903)
\curveto(316.72288989,1037.69110903)(316.11311102,1037.7408869)(315.60288788,1037.84044263)
\curveto(315.09266474,1037.9275539)(314.58244161,1038.0768875)(314.07221847,1038.28844344)
\lineto(314.07221847,1040.58444755)
\curveto(314.61977501,1040.33555822)(315.21088718,1040.13022452)(315.84555498,1039.96844645)
\curveto(316.48022279,1039.80666838)(317.04022379,1039.72577935)(317.52555799,1039.72577935)
\curveto(318.06067006,1039.72577935)(318.44644853,1039.80666838)(318.6828934,1039.96844645)
\curveto(318.91933827,1040.13022452)(319.0375607,1040.34178045)(319.0375607,1040.60311425)
\curveto(319.0375607,1040.77733679)(318.98778284,1040.93289262)(318.8882271,1041.06978176)
\curveto(318.80111583,1041.20667089)(318.60200437,1041.36222672)(318.2908927,1041.53644926)
\curveto(317.97978103,1041.71067179)(317.49444683,1041.9346722)(316.83489009,1042.20845046)
\curveto(316.18777782,1042.48222873)(315.65888798,1042.74978477)(315.24822058,1043.01111857)
\curveto(314.84999764,1043.28489684)(314.55133044,1043.60845297)(314.35221897,1043.98178698)
\curveto(314.1531075,1044.36756545)(314.05355177,1044.84667742)(314.05355177,1045.41912289)
\curveto(314.05355177,1046.36490236)(314.42066354,1047.07423696)(315.15488708,1047.5471267)
\curveto(315.88911061,1048.02001644)(316.86600125,1048.25646131)(318.085559,1048.25646131)
\curveto(318.7202268,1048.25646131)(319.32378344,1048.19423897)(319.89622891,1048.0697943)
\curveto(320.46867438,1047.94534964)(321.05978655,1047.74001593)(321.66956542,1047.4537932)
\lineto(320.82956392,1045.45645629)
\curveto(320.33178525,1045.66801222)(319.85889551,1045.84223476)(319.41089471,1045.97912389)
\curveto(318.9628939,1046.12845749)(318.50867087,1046.20312429)(318.0482256,1046.20312429)
\curveto(317.22689079,1046.20312429)(316.81622339,1045.97912389)(316.81622339,1045.53112309)
\curveto(316.81622339,1045.36934502)(316.86600125,1045.22001142)(316.96555699,1045.08312228)
\curveto(317.07755719,1044.95867762)(317.28289089,1044.82178848)(317.58155809,1044.67245488)
\curveto(317.89266976,1044.52312128)(318.3468928,1044.32400981)(318.9442272,1044.07512048)
\curveto(319.52911714,1043.83867561)(320.03311804,1043.58978627)(320.45622991,1043.32845247)
\curveto(320.87934178,1043.07956314)(321.20289792,1042.76222923)(321.42689832,1042.37645076)
\curveto(321.66334319,1041.9906723)(321.78156562,1041.49911586)(321.78156562,1040.90178145)
\closepath
}
}
{
\newrgbcolor{curcolor}{0 0 0}
\pscustom[linestyle=none,fillstyle=solid,fillcolor=curcolor]
{
\newpath
\moveto(217.91556807,1027.10229615)
\curveto(219.06045901,1027.10229615)(219.98757178,1026.65429535)(220.69690639,1025.75829374)
\curveto(221.40624099,1024.87473661)(221.76090829,1023.5680676)(221.76090829,1021.83828672)
\curveto(221.76090829,1020.09606137)(221.39379653,1018.7769479)(220.65957299,1017.88094629)
\curveto(219.92534945,1016.98494468)(218.98579221,1016.53694388)(217.84090127,1016.53694388)
\curveto(217.10667773,1016.53694388)(216.52178779,1016.66761078)(216.08623146,1016.92894458)
\curveto(215.65067512,1017.20272285)(215.29600782,1017.50761229)(215.02222955,1017.84361289)
\lineto(214.87289595,1017.84361289)
\curveto(214.97245168,1017.32094529)(215.02222955,1016.82316662)(215.02222955,1016.35027688)
\lineto(215.02222955,1012.24360285)
\lineto(212.24089123,1012.24360285)
\lineto(212.24089123,1026.91562915)
\lineto(214.49956194,1026.91562915)
\lineto(214.89156265,1025.59029344)
\lineto(215.02222955,1025.59029344)
\curveto(215.29600782,1026.00096085)(215.66311959,1026.35562815)(216.12356486,1026.65429535)
\curveto(216.58401013,1026.95296255)(217.18134453,1027.10229615)(217.91556807,1027.10229615)
\closepath
\moveto(217.01956646,1024.88095884)
\curveto(216.29778739,1024.88095884)(215.78756425,1024.6507362)(215.48889705,1024.19029093)
\curveto(215.19022985,1023.74229013)(215.03467402,1023.06406669)(215.02222955,1022.15562062)
\lineto(215.02222955,1021.85695342)
\curveto(215.02222955,1020.87384054)(215.16534092,1020.11472807)(215.45156365,1019.579616)
\curveto(215.75023085,1019.0569484)(216.28534292,1018.7956146)(217.05689986,1018.7956146)
\curveto(217.69156767,1018.7956146)(218.15823517,1019.0569484)(218.45690237,1019.579616)
\curveto(218.76801404,1020.11472807)(218.92356988,1020.88006278)(218.92356988,1021.87562012)
\curveto(218.92356988,1023.87917926)(218.28890207,1024.88095884)(217.01956646,1024.88095884)
\closepath
}
}
{
\newrgbcolor{curcolor}{0 0 0}
\pscustom[linestyle=none,fillstyle=solid,fillcolor=curcolor]
{
\newpath
\moveto(228.25690908,1027.10229615)
\curveto(229.66313383,1027.10229615)(230.7769136,1026.69785098)(231.59824841,1025.88896065)
\curveto(232.41958321,1025.09251477)(232.83025062,1023.95384607)(232.83025062,1022.47295452)
\lineto(232.83025062,1021.12895211)
\lineto(226.25957217,1021.12895211)
\curveto(226.2844611,1020.34495071)(226.51468374,1019.7289496)(226.95024008,1019.2809488)
\curveto(227.39824088,1018.832948)(228.01424198,1018.6089476)(228.79824339,1018.6089476)
\curveto(229.44535566,1018.6089476)(230.03646783,1018.67116993)(230.5715799,1018.7956146)
\curveto(231.11913644,1018.93250373)(231.67913744,1019.13783743)(232.25158291,1019.4116157)
\lineto(232.25158291,1017.26494519)
\curveto(231.74135977,1017.01605585)(231.21246994,1016.83561108)(230.6649134,1016.72361088)
\curveto(230.11735686,1016.59916622)(229.45157789,1016.53694388)(228.66757649,1016.53694388)
\curveto(227.64713021,1016.53694388)(226.74490637,1016.72361088)(225.96090497,1017.09694489)
\curveto(225.17690356,1017.48272335)(224.56090246,1018.05516883)(224.11290166,1018.8142813)
\curveto(223.66490085,1019.58583824)(223.44090045,1020.56272888)(223.44090045,1021.74495322)
\curveto(223.44090045,1022.92717756)(223.64001192,1023.91651267)(224.03823486,1024.71295854)
\curveto(224.44890226,1025.50940441)(225.0151255,1026.10673881)(225.73690457,1026.50496175)
\curveto(226.45868364,1026.90318469)(227.29868514,1027.10229615)(228.25690908,1027.10229615)
\closepath
\moveto(228.27557578,1025.12362594)
\curveto(227.72801925,1025.12362594)(227.28001844,1024.94940341)(226.93157338,1024.60095834)
\curveto(226.58312831,1024.25251327)(226.3777946,1023.71117896)(226.31557227,1022.97695543)
\lineto(230.2169126,1022.97695543)
\curveto(230.20446813,1023.5867343)(230.03646783,1024.09695743)(229.71291169,1024.50762484)
\curveto(229.40180003,1024.91829224)(228.92268806,1025.12362594)(228.27557578,1025.12362594)
\closepath
}
}
{
\newrgbcolor{curcolor}{0 0 0}
\pscustom[linestyle=none,fillstyle=solid,fillcolor=curcolor]
{
\newpath
\moveto(240.78225694,1027.10229615)
\curveto(240.91914608,1027.10229615)(241.08092414,1027.09607392)(241.26759114,1027.08362945)
\curveto(241.45425815,1027.07118499)(241.60359175,1027.05251829)(241.71559195,1027.02762935)
\lineto(241.51025825,1024.41429134)
\curveto(241.41070251,1024.43918027)(241.28003561,1024.45784697)(241.11825754,1024.47029144)
\curveto(240.95647948,1024.49518037)(240.81336811,1024.50762484)(240.68892344,1024.50762484)
\curveto(240.2160337,1024.50762484)(239.76181067,1024.42051357)(239.32625433,1024.24629103)
\curveto(238.890698,1024.08451297)(238.53603069,1023.81695693)(238.26225242,1023.44362293)
\curveto(238.00091862,1023.07028893)(237.87025172,1022.56006579)(237.87025172,1021.91295352)
\lineto(237.87025172,1016.72361088)
\lineto(235.0889134,1016.72361088)
\lineto(235.0889134,1026.91562915)
\lineto(237.19825052,1026.91562915)
\lineto(237.60891792,1025.19829274)
\lineto(237.73958482,1025.19829274)
\curveto(238.03825202,1025.72096034)(238.44891943,1026.16896115)(238.97158703,1026.54229515)
\curveto(239.49425463,1026.91562915)(240.09781127,1027.10229615)(240.78225694,1027.10229615)
\closepath
}
}
{
\newrgbcolor{curcolor}{0 0 0}
\pscustom[linestyle=none,fillstyle=solid,fillcolor=curcolor]
{
\newpath
\moveto(255.52895947,1027.10229615)
\curveto(256.68629488,1027.10229615)(257.55740755,1026.80362895)(258.14229749,1026.20629455)
\curveto(258.73963189,1025.62140461)(259.03829909,1024.67562514)(259.03829909,1023.36895613)
\lineto(259.03829909,1016.72361088)
\lineto(256.25696077,1016.72361088)
\lineto(256.25696077,1022.67828822)
\curveto(256.25696077,1024.1467353)(255.74673764,1024.88095884)(254.72629136,1024.88095884)
\curveto(253.99206782,1024.88095884)(253.46940022,1024.61962504)(253.15828855,1024.09695743)
\curveto(252.84717688,1023.57428983)(252.69162105,1022.82139959)(252.69162105,1021.83828672)
\lineto(252.69162105,1016.72361088)
\lineto(249.91028273,1016.72361088)
\lineto(249.91028273,1022.67828822)
\curveto(249.91028273,1024.1467353)(249.40005959,1024.88095884)(248.37961332,1024.88095884)
\curveto(247.60805638,1024.88095884)(247.07294431,1024.58851387)(246.77427711,1024.00362393)
\curveto(246.48805437,1023.43117846)(246.34494301,1022.60362142)(246.34494301,1021.52095282)
\lineto(246.34494301,1016.72361088)
\lineto(243.56360469,1016.72361088)
\lineto(243.56360469,1026.91562915)
\lineto(245.6916085,1026.91562915)
\lineto(246.0649425,1025.60896014)
\lineto(246.2142761,1025.60896014)
\curveto(246.52538777,1026.13162775)(246.94849964,1026.51118398)(247.48361171,1026.74762885)
\curveto(248.03116825,1026.98407372)(248.59739149,1027.10229615)(249.18228143,1027.10229615)
\curveto(249.92894943,1027.10229615)(250.557395,1026.97785149)(251.06761814,1026.72896215)
\curveto(251.59028574,1026.49251728)(251.99473091,1026.11918328)(252.28095365,1025.60896014)
\lineto(252.52362075,1025.60896014)
\curveto(252.83473242,1026.13162775)(253.26406652,1026.51118398)(253.81162306,1026.74762885)
\curveto(254.37162406,1026.98407372)(254.94406953,1027.10229615)(255.52895947,1027.10229615)
\closepath
}
}
{
\newrgbcolor{curcolor}{0 0 0}
\pscustom[linestyle=none,fillstyle=solid,fillcolor=curcolor]
{
\newpath
\moveto(263.29431261,1030.91030298)
\curveto(263.70498001,1030.91030298)(264.05964731,1030.81074725)(264.35831452,1030.61163578)
\curveto(264.65698172,1030.42496878)(264.80631532,1030.07030147)(264.80631532,1029.54763387)
\curveto(264.80631532,1029.03741073)(264.65698172,1028.68274343)(264.35831452,1028.48363196)
\curveto(264.05964731,1028.2845205)(263.70498001,1028.18496476)(263.29431261,1028.18496476)
\curveto(262.87120074,1028.18496476)(262.5103112,1028.2845205)(262.211644,1028.48363196)
\curveto(261.92542127,1028.68274343)(261.7823099,1029.03741073)(261.7823099,1029.54763387)
\curveto(261.7823099,1030.07030147)(261.92542127,1030.42496878)(262.211644,1030.61163578)
\curveto(262.5103112,1030.81074725)(262.87120074,1030.91030298)(263.29431261,1030.91030298)
\closepath
\moveto(264.67564842,1026.91562915)
\lineto(264.67564842,1016.72361088)
\lineto(261.8943101,1016.72361088)
\lineto(261.8943101,1026.91562915)
\closepath
}
}
{
\newrgbcolor{curcolor}{0 0 0}
\pscustom[linestyle=none,fillstyle=solid,fillcolor=curcolor]
{
\newpath
\moveto(274.69967071,1019.7476163)
\curveto(274.69967071,1018.71472556)(274.33255894,1017.91827969)(273.5983354,1017.35827869)
\curveto(272.87655633,1016.81072215)(271.79388772,1016.53694388)(270.35032958,1016.53694388)
\curveto(269.64099497,1016.53694388)(269.0312161,1016.58672175)(268.52099296,1016.68627748)
\curveto(268.01076983,1016.77338875)(267.50054669,1016.92272235)(266.99032355,1017.13427829)
\lineto(266.99032355,1019.4302824)
\curveto(267.53788009,1019.18139307)(268.12899226,1018.97605936)(268.76366007,1018.8142813)
\curveto(269.39832787,1018.65250323)(269.95832887,1018.5716142)(270.44366308,1018.5716142)
\curveto(270.97877515,1018.5716142)(271.36455362,1018.65250323)(271.60099849,1018.8142813)
\curveto(271.83744335,1018.97605936)(271.95566579,1019.1876153)(271.95566579,1019.4489491)
\curveto(271.95566579,1019.62317164)(271.90588792,1019.77872747)(271.80633219,1019.9156166)
\curveto(271.71922092,1020.05250574)(271.52010945,1020.20806157)(271.20899778,1020.38228411)
\curveto(270.89788611,1020.55650664)(270.41255191,1020.78050704)(269.75299517,1021.05428531)
\curveto(269.1058829,1021.32806358)(268.57699306,1021.59561962)(268.16632566,1021.85695342)
\curveto(267.76810273,1022.13073169)(267.46943552,1022.45428782)(267.27032406,1022.82762182)
\curveto(267.07121259,1023.21340029)(266.97165685,1023.69251226)(266.97165685,1024.26495773)
\curveto(266.97165685,1025.21073721)(267.33876862,1025.92007181)(268.07299216,1026.39296155)
\curveto(268.8072157,1026.86585129)(269.78410634,1027.10229615)(271.00366408,1027.10229615)
\curveto(271.63833189,1027.10229615)(272.24188852,1027.04007382)(272.81433399,1026.91562915)
\curveto(273.38677946,1026.79118448)(273.97789163,1026.58585078)(274.58767051,1026.29962805)
\lineto(273.747669,1024.30229113)
\curveto(273.24989033,1024.51384707)(272.77700059,1024.6880696)(272.32899979,1024.82495874)
\curveto(271.88099899,1024.97429234)(271.42677595,1025.04895914)(270.96633068,1025.04895914)
\curveto(270.14499588,1025.04895914)(269.73432847,1024.82495874)(269.73432847,1024.37695794)
\curveto(269.73432847,1024.21517987)(269.78410634,1024.06584627)(269.88366207,1023.92895713)
\curveto(269.99566227,1023.80451246)(270.20099598,1023.66762333)(270.49966318,1023.51828973)
\curveto(270.81077485,1023.36895613)(271.26499788,1023.16984466)(271.86233229,1022.92095533)
\curveto(272.44722222,1022.68451046)(272.95122313,1022.43562112)(273.374335,1022.17428732)
\curveto(273.79744687,1021.92539798)(274.121003,1021.60806408)(274.3450034,1021.22228561)
\curveto(274.58144827,1020.83650714)(274.69967071,1020.34495071)(274.69967071,1019.7476163)
\closepath
}
}
{
\newrgbcolor{curcolor}{0 0 0}
\pscustom[linestyle=none,fillstyle=solid,fillcolor=curcolor]
{
\newpath
\moveto(283.97701815,1019.7476163)
\curveto(283.97701815,1018.71472556)(283.60990638,1017.91827969)(282.87568285,1017.35827869)
\curveto(282.15390377,1016.81072215)(281.07123517,1016.53694388)(279.62767702,1016.53694388)
\curveto(278.91834242,1016.53694388)(278.30856355,1016.58672175)(277.79834041,1016.68627748)
\curveto(277.28811727,1016.77338875)(276.77789414,1016.92272235)(276.267671,1017.13427829)
\lineto(276.267671,1019.4302824)
\curveto(276.81522754,1019.18139307)(277.40633971,1018.97605936)(278.04100751,1018.8142813)
\curveto(278.67567532,1018.65250323)(279.23567632,1018.5716142)(279.72101052,1018.5716142)
\curveto(280.2561226,1018.5716142)(280.64190106,1018.65250323)(280.87834593,1018.8142813)
\curveto(281.1147908,1018.97605936)(281.23301324,1019.1876153)(281.23301324,1019.4489491)
\curveto(281.23301324,1019.62317164)(281.18323537,1019.77872747)(281.08367963,1019.9156166)
\curveto(280.99656837,1020.05250574)(280.7974569,1020.20806157)(280.48634523,1020.38228411)
\curveto(280.17523356,1020.55650664)(279.68989936,1020.78050704)(279.03034262,1021.05428531)
\curveto(278.38323035,1021.32806358)(277.85434051,1021.59561962)(277.44367311,1021.85695342)
\curveto(277.04545017,1022.13073169)(276.74678297,1022.45428782)(276.5476715,1022.82762182)
\curveto(276.34856004,1023.21340029)(276.2490043,1023.69251226)(276.2490043,1024.26495773)
\curveto(276.2490043,1025.21073721)(276.61611607,1025.92007181)(277.35033961,1026.39296155)
\curveto(278.08456315,1026.86585129)(279.06145379,1027.10229615)(280.28101153,1027.10229615)
\curveto(280.91567933,1027.10229615)(281.51923597,1027.04007382)(282.09168144,1026.91562915)
\curveto(282.66412691,1026.79118448)(283.25523908,1026.58585078)(283.86501795,1026.29962805)
\lineto(283.02501645,1024.30229113)
\curveto(282.52723778,1024.51384707)(282.05434804,1024.6880696)(281.60634724,1024.82495874)
\curveto(281.15834643,1024.97429234)(280.7041234,1025.04895914)(280.24367813,1025.04895914)
\curveto(279.42234332,1025.04895914)(279.01167592,1024.82495874)(279.01167592,1024.37695794)
\curveto(279.01167592,1024.21517987)(279.06145379,1024.06584627)(279.16100952,1023.92895713)
\curveto(279.27300972,1023.80451246)(279.47834342,1023.66762333)(279.77701063,1023.51828973)
\curveto(280.08812229,1023.36895613)(280.54234533,1023.16984466)(281.13967973,1022.92095533)
\curveto(281.72456967,1022.68451046)(282.22857058,1022.43562112)(282.65168244,1022.17428732)
\curveto(283.07479431,1021.92539798)(283.39835045,1021.60806408)(283.62235085,1021.22228561)
\curveto(283.85879572,1020.83650714)(283.97701815,1020.34495071)(283.97701815,1019.7476163)
\closepath
}
}
{
\newrgbcolor{curcolor}{0 0 0}
\pscustom[linestyle=none,fillstyle=solid,fillcolor=curcolor]
{
\newpath
\moveto(287.54235155,1030.91030298)
\curveto(287.95301895,1030.91030298)(288.30768625,1030.81074725)(288.60635346,1030.61163578)
\curveto(288.90502066,1030.42496878)(289.05435426,1030.07030147)(289.05435426,1029.54763387)
\curveto(289.05435426,1029.03741073)(288.90502066,1028.68274343)(288.60635346,1028.48363196)
\curveto(288.30768625,1028.2845205)(287.95301895,1028.18496476)(287.54235155,1028.18496476)
\curveto(287.11923968,1028.18496476)(286.75835014,1028.2845205)(286.45968294,1028.48363196)
\curveto(286.17346021,1028.68274343)(286.03034884,1029.03741073)(286.03034884,1029.54763387)
\curveto(286.03034884,1030.07030147)(286.17346021,1030.42496878)(286.45968294,1030.61163578)
\curveto(286.75835014,1030.81074725)(287.11923968,1030.91030298)(287.54235155,1030.91030298)
\closepath
\moveto(288.92368736,1026.91562915)
\lineto(288.92368736,1016.72361088)
\lineto(286.14234904,1016.72361088)
\lineto(286.14234904,1026.91562915)
\closepath
}
}
{
\newrgbcolor{curcolor}{0 0 0}
\pscustom[linestyle=none,fillstyle=solid,fillcolor=curcolor]
{
\newpath
\moveto(301.09438398,1021.83828672)
\curveto(301.09438398,1020.14583924)(300.64638317,1018.83917023)(299.75038157,1017.91827969)
\curveto(298.86682443,1016.99738915)(297.65971115,1016.53694388)(296.12904174,1016.53694388)
\curveto(295.18326227,1016.53694388)(294.33703853,1016.74227758)(293.59037052,1017.15294499)
\curveto(292.85614699,1017.56361239)(292.27747928,1018.16094679)(291.85436741,1018.9449482)
\curveto(291.43125554,1019.74139407)(291.21969961,1020.70584024)(291.21969961,1021.83828672)
\curveto(291.21969961,1023.5307342)(291.66147818,1024.83118097)(292.54503532,1025.73962704)
\curveto(293.42859246,1026.64807312)(294.64192796,1027.10229615)(296.18504184,1027.10229615)
\curveto(297.14326578,1027.10229615)(297.98948952,1026.89696245)(298.72371306,1026.48629505)
\curveto(299.4579366,1026.07562765)(300.0366043,1025.47829324)(300.45971617,1024.69429184)
\curveto(300.88282804,1023.91029043)(301.09438398,1022.95828873)(301.09438398,1021.83828672)
\closepath
\moveto(294.05703803,1021.83828672)
\curveto(294.05703803,1020.83028491)(294.2188161,1020.06495021)(294.54237223,1019.5422826)
\curveto(294.87837283,1019.03205947)(295.41970714,1018.7769479)(296.16637514,1018.7769479)
\curveto(296.90059868,1018.7769479)(297.42948852,1019.03205947)(297.75304465,1019.5422826)
\curveto(298.08904526,1020.06495021)(298.25704556,1020.83028491)(298.25704556,1021.83828672)
\curveto(298.25704556,1022.84628852)(298.08904526,1023.59917876)(297.75304465,1024.09695743)
\curveto(297.42948852,1024.60718057)(296.89437645,1024.86229214)(296.14770844,1024.86229214)
\curveto(295.4134849,1024.86229214)(294.87837283,1024.60718057)(294.54237223,1024.09695743)
\curveto(294.2188161,1023.59917876)(294.05703803,1022.84628852)(294.05703803,1021.83828672)
\closepath
}
}
{
\newrgbcolor{curcolor}{0 0 0}
\pscustom[linestyle=none,fillstyle=solid,fillcolor=curcolor]
{
\newpath
\moveto(309.17705748,1027.10229615)
\curveto(310.27217055,1027.10229615)(311.14950546,1026.80362895)(311.8090622,1026.20629455)
\curveto(312.46861893,1025.62140461)(312.7983973,1024.67562514)(312.7983973,1023.36895613)
\lineto(312.7983973,1016.72361088)
\lineto(310.01705898,1016.72361088)
\lineto(310.01705898,1022.67828822)
\curveto(310.01705898,1023.41251176)(309.88639208,1023.9600683)(309.62505828,1024.32095783)
\curveto(309.36372448,1024.69429184)(308.94683484,1024.88095884)(308.37438937,1024.88095884)
\curveto(307.52816563,1024.88095884)(306.94949793,1024.58851387)(306.63838626,1024.00362393)
\curveto(306.32727459,1023.43117846)(306.17171876,1022.60362142)(306.17171876,1021.52095282)
\lineto(306.17171876,1016.72361088)
\lineto(303.39038044,1016.72361088)
\lineto(303.39038044,1026.91562915)
\lineto(305.51838425,1026.91562915)
\lineto(305.89171825,1025.60896014)
\lineto(306.04105186,1025.60896014)
\curveto(306.36460799,1026.13162775)(306.80638656,1026.51118398)(307.36638756,1026.74762885)
\curveto(307.93883304,1026.98407372)(308.54238967,1027.10229615)(309.17705748,1027.10229615)
\closepath
}
}
{
\newrgbcolor{curcolor}{0 0 0}
\pscustom[linestyle=none,fillstyle=solid,fillcolor=curcolor]
{
\newpath
\moveto(315.26238715,1018.03027989)
\curveto(315.26238715,1018.60272536)(315.41794298,1019.0009483)(315.72905465,1019.2249487)
\curveto(316.04016632,1019.46139357)(316.41972256,1019.579616)(316.86772336,1019.579616)
\curveto(317.3032797,1019.579616)(317.6766137,1019.46139357)(317.98772537,1019.2249487)
\curveto(318.29883704,1019.0009483)(318.45439287,1018.60272536)(318.45439287,1018.03027989)
\curveto(318.45439287,1017.48272335)(318.29883704,1017.08450042)(317.98772537,1016.83561108)
\curveto(317.6766137,1016.59916622)(317.3032797,1016.48094378)(316.86772336,1016.48094378)
\curveto(316.41972256,1016.48094378)(316.04016632,1016.59916622)(315.72905465,1016.83561108)
\curveto(315.41794298,1017.08450042)(315.26238715,1017.48272335)(315.26238715,1018.03027989)
\closepath
}
}
{
\newrgbcolor{curcolor}{0 0 0}
\pscustom[linestyle=none,fillstyle=solid,fillcolor=curcolor]
{
\newpath
\moveto(320.86239491,1029.54763387)
\curveto(320.86239491,1030.07030147)(321.00550628,1030.42496878)(321.29172901,1030.61163578)
\curveto(321.59039621,1030.81074725)(321.95128575,1030.91030298)(322.37439762,1030.91030298)
\curveto(322.78506502,1030.91030298)(323.13973232,1030.81074725)(323.43839953,1030.61163578)
\curveto(323.73706673,1030.42496878)(323.88640033,1030.07030147)(323.88640033,1029.54763387)
\curveto(323.88640033,1029.03741073)(323.73706673,1028.68274343)(323.43839953,1028.48363196)
\curveto(323.13973232,1028.2845205)(322.78506502,1028.18496476)(322.37439762,1028.18496476)
\curveto(321.95128575,1028.18496476)(321.59039621,1028.2845205)(321.29172901,1028.48363196)
\curveto(321.00550628,1028.68274343)(320.86239491,1029.03741073)(320.86239491,1029.54763387)
\closepath
\moveto(320.1530603,1012.24360285)
\curveto(319.82950417,1012.24360285)(319.4997258,1012.26849179)(319.1637252,1012.31826965)
\curveto(318.82772459,1012.35560305)(318.54772409,1012.40538092)(318.32372369,1012.46760325)
\lineto(318.32372369,1014.65160717)
\curveto(318.54772409,1014.58938483)(318.75928003,1014.5458292)(318.95839149,1014.52094027)
\curveto(319.15750296,1014.49605133)(319.38150336,1014.48360687)(319.6303927,1014.48360687)
\curveto(320.0037267,1014.48360687)(320.3210606,1014.58938483)(320.58239441,1014.80094077)
\curveto(320.84372821,1015.0124967)(320.97439511,1015.42316411)(320.97439511,1016.03294298)
\lineto(320.97439511,1026.91562915)
\lineto(323.75573343,1026.91562915)
\lineto(323.75573343,1015.62227558)
\curveto(323.75573343,1015.00005224)(323.63751099,1014.433829)(323.40106613,1013.92360586)
\curveto(323.16462126,1013.41338273)(322.77884279,1013.00893756)(322.24373072,1012.71027036)
\curveto(321.72106311,1012.39915869)(321.02417298,1012.24360285)(320.1530603,1012.24360285)
\closepath
}
}
{
\newrgbcolor{curcolor}{0 0 0}
\pscustom[linestyle=none,fillstyle=solid,fillcolor=curcolor]
{
\newpath
\moveto(333.77975953,1019.7476163)
\curveto(333.77975953,1018.71472556)(333.41264776,1017.91827969)(332.67842422,1017.35827869)
\curveto(331.95664515,1016.81072215)(330.87397654,1016.53694388)(329.4304184,1016.53694388)
\curveto(328.7210838,1016.53694388)(328.11130493,1016.58672175)(327.60108179,1016.68627748)
\curveto(327.09085865,1016.77338875)(326.58063552,1016.92272235)(326.07041238,1017.13427829)
\lineto(326.07041238,1019.4302824)
\curveto(326.61796892,1019.18139307)(327.20908109,1018.97605936)(327.84374889,1018.8142813)
\curveto(328.47841669,1018.65250323)(329.0384177,1018.5716142)(329.5237519,1018.5716142)
\curveto(330.05886397,1018.5716142)(330.44464244,1018.65250323)(330.68108731,1018.8142813)
\curveto(330.91753218,1018.97605936)(331.03575461,1019.1876153)(331.03575461,1019.4489491)
\curveto(331.03575461,1019.62317164)(330.98597675,1019.77872747)(330.88642101,1019.9156166)
\curveto(330.79930974,1020.05250574)(330.60019828,1020.20806157)(330.28908661,1020.38228411)
\curveto(329.97797494,1020.55650664)(329.49264073,1020.78050704)(328.833084,1021.05428531)
\curveto(328.18597173,1021.32806358)(327.65708189,1021.59561962)(327.24641449,1021.85695342)
\curveto(326.84819155,1022.13073169)(326.54952435,1022.45428782)(326.35041288,1022.82762182)
\curveto(326.15130141,1023.21340029)(326.05174568,1023.69251226)(326.05174568,1024.26495773)
\curveto(326.05174568,1025.21073721)(326.41885745,1025.92007181)(327.15308099,1026.39296155)
\curveto(327.88730452,1026.86585129)(328.86419516,1027.10229615)(330.08375291,1027.10229615)
\curveto(330.71842071,1027.10229615)(331.32197735,1027.04007382)(331.89442282,1026.91562915)
\curveto(332.46686829,1026.79118448)(333.05798046,1026.58585078)(333.66775933,1026.29962805)
\lineto(332.82775782,1024.30229113)
\curveto(332.32997915,1024.51384707)(331.85708942,1024.6880696)(331.40908861,1024.82495874)
\curveto(330.96108781,1024.97429234)(330.50686478,1025.04895914)(330.04641951,1025.04895914)
\curveto(329.2250847,1025.04895914)(328.8144173,1024.82495874)(328.8144173,1024.37695794)
\curveto(328.8144173,1024.21517987)(328.86419516,1024.06584627)(328.9637509,1023.92895713)
\curveto(329.0757511,1023.80451246)(329.2810848,1023.66762333)(329.579752,1023.51828973)
\curveto(329.89086367,1023.36895613)(330.34508671,1023.16984466)(330.94242111,1022.92095533)
\curveto(331.52731105,1022.68451046)(332.03131195,1022.43562112)(332.45442382,1022.17428732)
\curveto(332.87753569,1021.92539798)(333.20109183,1021.60806408)(333.42509223,1021.22228561)
\curveto(333.6615371,1020.83650714)(333.77975953,1020.34495071)(333.77975953,1019.7476163)
\closepath
}
}
{
\newrgbcolor{curcolor}{0 0 0}
\pscustom[linestyle=none,fillstyle=solid,fillcolor=curcolor]
{
\newpath
\moveto(221.87334738,1005.58263558)
\lineto(221.87334738,995.39061731)
\lineto(219.74534356,995.39061731)
\lineto(219.37200956,996.69728632)
\lineto(219.22267596,996.69728632)
\curveto(218.89911983,996.17461871)(218.45111902,995.79506248)(217.87867355,995.55861761)
\curveto(217.31867255,995.32217274)(216.72133814,995.2039503)(216.08667034,995.2039503)
\curveto(214.99155726,995.2039503)(214.11422236,995.49639527)(213.45466562,996.08128521)
\curveto(212.79510888,996.67861961)(212.46533051,997.63062132)(212.46533051,998.93729033)
\lineto(212.46533051,1005.58263558)
\lineto(215.24666883,1005.58263558)
\lineto(215.24666883,999.62795824)
\curveto(215.24666883,998.8937347)(215.37733573,998.33995593)(215.63866954,997.96662192)
\curveto(215.90000334,997.60573239)(216.31689297,997.42528762)(216.88933844,997.42528762)
\curveto(217.73556218,997.42528762)(218.31422989,997.71151036)(218.62534156,998.28395583)
\curveto(218.93645323,998.86884576)(219.09200906,999.70262504)(219.09200906,1000.78529364)
\lineto(219.09200906,1005.58263558)
\closepath
}
}
{
\newrgbcolor{curcolor}{0 0 0}
\pscustom[linestyle=none,fillstyle=solid,fillcolor=curcolor]
{
\newpath
\moveto(231.89735745,998.41462273)
\curveto(231.89735745,997.38173199)(231.53024568,996.58528611)(230.79602215,996.02528511)
\curveto(230.07424307,995.47772857)(228.99157447,995.2039503)(227.54801632,995.2039503)
\curveto(226.83868172,995.2039503)(226.22890285,995.25372817)(225.71867971,995.35328391)
\curveto(225.20845657,995.44039517)(224.69823344,995.58972877)(224.1880103,995.80128471)
\lineto(224.1880103,998.09728882)
\curveto(224.73556684,997.84839949)(225.32667901,997.64306579)(225.96134681,997.48128772)
\curveto(226.59601462,997.31950965)(227.15601562,997.23862062)(227.64134982,997.23862062)
\curveto(228.17646189,997.23862062)(228.56224036,997.31950965)(228.79868523,997.48128772)
\curveto(229.0351301,997.64306579)(229.15335253,997.85462172)(229.15335253,998.11595552)
\curveto(229.15335253,998.29017806)(229.10357467,998.44573389)(229.00401893,998.58262303)
\curveto(228.91690767,998.71951216)(228.7177962,998.875068)(228.40668453,999.04929053)
\curveto(228.09557286,999.22351307)(227.61023866,999.44751347)(226.95068192,999.72129174)
\curveto(226.30356965,999.99507)(225.77467981,1000.26262604)(225.36401241,1000.52395984)
\curveto(224.96578947,1000.79773811)(224.66712227,1001.12129425)(224.4680108,1001.49462825)
\curveto(224.26889933,1001.88040672)(224.1693436,1002.35951869)(224.1693436,1002.93196416)
\curveto(224.1693436,1003.87774363)(224.53645537,1004.58707824)(225.27067891,1005.05996797)
\curveto(226.00490245,1005.53285771)(226.98179309,1005.76930258)(228.20135083,1005.76930258)
\curveto(228.83601863,1005.76930258)(229.43957527,1005.70708024)(230.01202074,1005.58263558)
\curveto(230.58446621,1005.45819091)(231.17557838,1005.25285721)(231.78535725,1004.96663447)
\lineto(230.94535575,1002.96929756)
\curveto(230.44757708,1003.18085349)(229.97468734,1003.35507603)(229.52668654,1003.49196516)
\curveto(229.07868573,1003.64129876)(228.6244627,1003.71596556)(228.16401743,1003.71596556)
\curveto(227.34268262,1003.71596556)(226.93201522,1003.49196516)(226.93201522,1003.04396436)
\curveto(226.93201522,1002.88218629)(226.98179309,1002.73285269)(227.08134882,1002.59596356)
\curveto(227.19334902,1002.47151889)(227.39868272,1002.33462975)(227.69734992,1002.18529615)
\curveto(228.00846159,1002.03596255)(228.46268463,1001.83685108)(229.06001903,1001.58796175)
\curveto(229.64490897,1001.35151688)(230.14890987,1001.10262755)(230.57202174,1000.84129374)
\curveto(230.99513361,1000.59240441)(231.31868975,1000.27507051)(231.54269015,999.88929204)
\curveto(231.77913502,999.50351357)(231.89735745,999.01195713)(231.89735745,998.41462273)
\closepath
}
}
{
\newrgbcolor{curcolor}{0 0 0}
\pscustom[linestyle=none,fillstyle=solid,fillcolor=curcolor]
{
\newpath
\moveto(238.26269682,1005.76930258)
\curveto(239.66892156,1005.76930258)(240.78270134,1005.36485741)(241.60403614,1004.55596707)
\curveto(242.42537095,1003.7595212)(242.83603835,1002.62085249)(242.83603835,1001.13996095)
\lineto(242.83603835,999.79595854)
\lineto(236.26535991,999.79595854)
\curveto(236.29024884,999.01195713)(236.52047147,998.39595603)(236.95602781,997.94795522)
\curveto(237.40402861,997.49995442)(238.02002972,997.27595402)(238.80403112,997.27595402)
\curveto(239.45114339,997.27595402)(240.04225557,997.33817635)(240.57736764,997.46262102)
\curveto(241.12492417,997.59951015)(241.68492518,997.80484386)(242.25737065,998.07862212)
\lineto(242.25737065,995.93195161)
\curveto(241.74714751,995.68306227)(241.21825767,995.50261751)(240.67070114,995.39061731)
\curveto(240.1231446,995.26617264)(239.45736563,995.2039503)(238.67336422,995.2039503)
\curveto(237.65291795,995.2039503)(236.75069411,995.39061731)(235.9666927,995.76395131)
\curveto(235.1826913,996.14972978)(234.56669019,996.72217525)(234.11868939,997.48128772)
\curveto(233.67068859,998.25284466)(233.44668819,999.2297353)(233.44668819,1000.41195964)
\curveto(233.44668819,1001.59418398)(233.64579965,1002.58351909)(234.04402259,1003.37996496)
\curveto(234.45468999,1004.17641083)(235.02091323,1004.77374524)(235.7426923,1005.17196817)
\curveto(236.46447137,1005.57019111)(237.30447288,1005.76930258)(238.26269682,1005.76930258)
\closepath
\moveto(238.28136352,1003.79063236)
\curveto(237.73380698,1003.79063236)(237.28580618,1003.61640983)(236.93736111,1003.26796476)
\curveto(236.58891604,1002.91951969)(236.38358234,1002.37818539)(236.32136001,1001.64396185)
\lineto(240.22270033,1001.64396185)
\curveto(240.21025587,1002.25374072)(240.04225557,1002.76396386)(239.71869943,1003.17463126)
\curveto(239.40758776,1003.58529866)(238.92847579,1003.79063236)(238.28136352,1003.79063236)
\closepath
}
}
{
\newrgbcolor{curcolor}{0 0 0}
\pscustom[linestyle=none,fillstyle=solid,fillcolor=curcolor]
{
\newpath
\moveto(250.78804563,1005.76930258)
\curveto(250.92493476,1005.76930258)(251.08671283,1005.76308034)(251.27337983,1005.75063588)
\curveto(251.46004684,1005.73819141)(251.60938044,1005.71952471)(251.72138064,1005.69463578)
\lineto(251.51604694,1003.08129776)
\curveto(251.4164912,1003.10618669)(251.2858243,1003.12485339)(251.12404623,1003.13729786)
\curveto(250.96226817,1003.16218679)(250.8191568,1003.17463126)(250.69471213,1003.17463126)
\curveto(250.22182239,1003.17463126)(249.76759936,1003.08751999)(249.33204302,1002.91329746)
\curveto(248.89648668,1002.75151939)(248.54181938,1002.48396335)(248.26804111,1002.11062935)
\curveto(248.00670731,1001.73729535)(247.87604041,1001.22707221)(247.87604041,1000.57995994)
\lineto(247.87604041,995.39061731)
\lineto(245.09470209,995.39061731)
\lineto(245.09470209,1005.58263558)
\lineto(247.20403921,1005.58263558)
\lineto(247.61470661,1003.86529916)
\lineto(247.74537351,1003.86529916)
\curveto(248.04404071,1004.38796677)(248.45470811,1004.83596757)(248.97737572,1005.20930157)
\curveto(249.50004332,1005.58263558)(250.10359996,1005.76930258)(250.78804563,1005.76930258)
\closepath
}
}
{
\newrgbcolor{curcolor}{0 0 0}
\pscustom[linestyle=none,fillstyle=solid,fillcolor=curcolor]
{
\newpath
\moveto(260.68140803,998.41462273)
\curveto(260.68140803,997.38173199)(260.31429626,996.58528611)(259.58007272,996.02528511)
\curveto(258.85829365,995.47772857)(257.77562505,995.2039503)(256.3320669,995.2039503)
\curveto(255.6227323,995.2039503)(255.01295343,995.25372817)(254.50273029,995.35328391)
\curveto(253.99250715,995.44039517)(253.48228402,995.58972877)(252.97206088,995.80128471)
\lineto(252.97206088,998.09728882)
\curveto(253.51961742,997.84839949)(254.11072959,997.64306579)(254.74539739,997.48128772)
\curveto(255.3800652,997.31950965)(255.9400662,997.23862062)(256.4254004,997.23862062)
\curveto(256.96051247,997.23862062)(257.34629094,997.31950965)(257.58273581,997.48128772)
\curveto(257.81918068,997.64306579)(257.93740311,997.85462172)(257.93740311,998.11595552)
\curveto(257.93740311,998.29017806)(257.88762525,998.44573389)(257.78806951,998.58262303)
\curveto(257.70095824,998.71951216)(257.50184678,998.875068)(257.19073511,999.04929053)
\curveto(256.87962344,999.22351307)(256.39428924,999.44751347)(255.7347325,999.72129174)
\curveto(255.08762023,999.99507)(254.55873039,1000.26262604)(254.14806299,1000.52395984)
\curveto(253.74984005,1000.79773811)(253.45117285,1001.12129425)(253.25206138,1001.49462825)
\curveto(253.05294991,1001.88040672)(252.95339418,1002.35951869)(252.95339418,1002.93196416)
\curveto(252.95339418,1003.87774363)(253.32050595,1004.58707824)(254.05472949,1005.05996797)
\curveto(254.78895302,1005.53285771)(255.76584366,1005.76930258)(256.98540141,1005.76930258)
\curveto(257.62006921,1005.76930258)(258.22362585,1005.70708024)(258.79607132,1005.58263558)
\curveto(259.36851679,1005.45819091)(259.95962896,1005.25285721)(260.56940783,1004.96663447)
\lineto(259.72940633,1002.96929756)
\curveto(259.23162766,1003.18085349)(258.75873792,1003.35507603)(258.31073712,1003.49196516)
\curveto(257.86273631,1003.64129876)(257.40851328,1003.71596556)(256.94806801,1003.71596556)
\curveto(256.1267332,1003.71596556)(255.7160658,1003.49196516)(255.7160658,1003.04396436)
\curveto(255.7160658,1002.88218629)(255.76584366,1002.73285269)(255.8653994,1002.59596356)
\curveto(255.9773996,1002.47151889)(256.1827333,1002.33462975)(256.4814005,1002.18529615)
\curveto(256.79251217,1002.03596255)(257.24673521,1001.83685108)(257.84406961,1001.58796175)
\curveto(258.42895955,1001.35151688)(258.93296045,1001.10262755)(259.35607232,1000.84129374)
\curveto(259.77918419,1000.59240441)(260.10274033,1000.27507051)(260.32674073,999.88929204)
\curveto(260.5631856,999.50351357)(260.68140803,999.01195713)(260.68140803,998.41462273)
\closepath
}
}
{
\newrgbcolor{curcolor}{0 0 0}
\pscustom[linestyle=none,fillstyle=solid,fillcolor=curcolor]
{
\newpath
\moveto(269.10008346,992.44127869)
\lineto(261.35340291,992.44127869)
\lineto(261.35340291,993.71061429)
\lineto(269.10008346,993.71061429)
\closepath
}
}
{
\newrgbcolor{curcolor}{0 0 0}
\pscustom[linestyle=none,fillstyle=solid,fillcolor=curcolor]
{
\newpath
\moveto(273.30007802,995.39061731)
\lineto(270.5187397,995.39061731)
\lineto(270.5187397,1009.5773094)
\lineto(273.30007802,1009.5773094)
\closepath
}
}
{
\newrgbcolor{curcolor}{0 0 0}
\pscustom[linestyle=none,fillstyle=solid,fillcolor=curcolor]
{
\newpath
\moveto(277.61209007,1009.5773094)
\curveto(278.02275748,1009.5773094)(278.37742478,1009.47775367)(278.67609198,1009.2786422)
\curveto(278.97475918,1009.0919752)(279.12409278,1008.7373079)(279.12409278,1008.21464029)
\curveto(279.12409278,1007.70441716)(278.97475918,1007.34974985)(278.67609198,1007.15063839)
\curveto(278.37742478,1006.95152692)(278.02275748,1006.85197118)(277.61209007,1006.85197118)
\curveto(277.1889782,1006.85197118)(276.82808867,1006.95152692)(276.52942147,1007.15063839)
\curveto(276.24319873,1007.34974985)(276.10008736,1007.70441716)(276.10008736,1008.21464029)
\curveto(276.10008736,1008.7373079)(276.24319873,1009.0919752)(276.52942147,1009.2786422)
\curveto(276.82808867,1009.47775367)(277.1889782,1009.5773094)(277.61209007,1009.5773094)
\closepath
\moveto(278.99342588,1005.58263558)
\lineto(278.99342588,995.39061731)
\lineto(276.21208756,995.39061731)
\lineto(276.21208756,1005.58263558)
\closepath
}
}
{
\newrgbcolor{curcolor}{0 0 0}
\pscustom[linestyle=none,fillstyle=solid,fillcolor=curcolor]
{
\newpath
\moveto(289.01745199,998.41462273)
\curveto(289.01745199,997.38173199)(288.65034022,996.58528611)(287.91611668,996.02528511)
\curveto(287.19433761,995.47772857)(286.111669,995.2039503)(284.66811086,995.2039503)
\curveto(283.95877625,995.2039503)(283.34899738,995.25372817)(282.83877424,995.35328391)
\curveto(282.32855111,995.44039517)(281.81832797,995.58972877)(281.30810483,995.80128471)
\lineto(281.30810483,998.09728882)
\curveto(281.85566137,997.84839949)(282.44677354,997.64306579)(283.08144135,997.48128772)
\curveto(283.71610915,997.31950965)(284.27611015,997.23862062)(284.76144436,997.23862062)
\curveto(285.29655643,997.23862062)(285.6823349,997.31950965)(285.91877977,997.48128772)
\curveto(286.15522463,997.64306579)(286.27344707,997.85462172)(286.27344707,998.11595552)
\curveto(286.27344707,998.29017806)(286.2236692,998.44573389)(286.12411347,998.58262303)
\curveto(286.0370022,998.71951216)(285.83789073,998.875068)(285.52677906,999.04929053)
\curveto(285.21566739,999.22351307)(284.73033319,999.44751347)(284.07077645,999.72129174)
\curveto(283.42366418,999.99507)(282.89477434,1000.26262604)(282.48410694,1000.52395984)
\curveto(282.08588401,1000.79773811)(281.7872168,1001.12129425)(281.58810534,1001.49462825)
\curveto(281.38899387,1001.88040672)(281.28943813,1002.35951869)(281.28943813,1002.93196416)
\curveto(281.28943813,1003.87774363)(281.6565499,1004.58707824)(282.39077344,1005.05996797)
\curveto(283.12499698,1005.53285771)(284.10188762,1005.76930258)(285.32144536,1005.76930258)
\curveto(285.95611317,1005.76930258)(286.5596698,1005.70708024)(287.13211527,1005.58263558)
\curveto(287.70456074,1005.45819091)(288.29567291,1005.25285721)(288.90545179,1004.96663447)
\lineto(288.06545028,1002.96929756)
\curveto(287.56767161,1003.18085349)(287.09478187,1003.35507603)(286.64678107,1003.49196516)
\curveto(286.19878027,1003.64129876)(285.74455723,1003.71596556)(285.28411196,1003.71596556)
\curveto(284.46277716,1003.71596556)(284.05210975,1003.49196516)(284.05210975,1003.04396436)
\curveto(284.05210975,1002.88218629)(284.10188762,1002.73285269)(284.20144335,1002.59596356)
\curveto(284.31344355,1002.47151889)(284.51877726,1002.33462975)(284.81744446,1002.18529615)
\curveto(285.12855613,1002.03596255)(285.58277916,1001.83685108)(286.18011357,1001.58796175)
\curveto(286.7650035,1001.35151688)(287.26900441,1001.10262755)(287.69211628,1000.84129374)
\curveto(288.11522815,1000.59240441)(288.43878428,1000.27507051)(288.66278468,999.88929204)
\curveto(288.89922955,999.50351357)(289.01745199,999.01195713)(289.01745199,998.41462273)
\closepath
}
}
{
\newrgbcolor{curcolor}{0 0 0}
\pscustom[linestyle=none,fillstyle=solid,fillcolor=curcolor]
{
\newpath
\moveto(295.4761239,997.42528762)
\curveto(295.78723557,997.42528762)(296.08590277,997.45017655)(296.37212551,997.49995442)
\curveto(296.65834824,997.56217675)(296.94457098,997.64306579)(297.23079371,997.74262152)
\lineto(297.23079371,995.67061781)
\curveto(296.93212651,995.53372867)(296.55879251,995.42172847)(296.1107917,995.33461721)
\curveto(295.67523537,995.24750594)(295.1961234,995.2039503)(294.67345579,995.2039503)
\curveto(294.06367692,995.2039503)(293.51612039,995.30350604)(293.03078618,995.50261751)
\curveto(292.55789645,995.70172897)(292.17834021,996.04395181)(291.89211748,996.52928601)
\curveto(291.61833921,997.01462022)(291.48145007,997.69906589)(291.48145007,998.58262303)
\lineto(291.48145007,1003.49196516)
\lineto(290.15611436,1003.49196516)
\lineto(290.15611436,1004.66796727)
\lineto(291.68678377,1005.60130228)
\lineto(292.48945188,1007.74797279)
\lineto(294.26278839,1007.74797279)
\lineto(294.26278839,1005.58263558)
\lineto(297.11879351,1005.58263558)
\lineto(297.11879351,1003.49196516)
\lineto(294.26278839,1003.49196516)
\lineto(294.26278839,998.58262303)
\curveto(294.26278839,998.19684456)(294.37478859,997.90439959)(294.59878899,997.70528812)
\curveto(294.8227894,997.51862112)(295.11523436,997.42528762)(295.4761239,997.42528762)
\closepath
}
}
{
\newrgbcolor{curcolor}{0 0 0}
\pscustom[linestyle=none,fillstyle=solid,fillcolor=curcolor]
{
\newpath
\moveto(298.89213094,996.69728632)
\curveto(298.89213094,997.26973179)(299.04768678,997.66795472)(299.35879845,997.89195512)
\curveto(299.66991011,998.12839999)(300.04946635,998.24662243)(300.49746715,998.24662243)
\curveto(300.93302349,998.24662243)(301.30635749,998.12839999)(301.61746916,997.89195512)
\curveto(301.92858083,997.66795472)(302.08413666,997.26973179)(302.08413666,996.69728632)
\curveto(302.08413666,996.14972978)(301.92858083,995.75150684)(301.61746916,995.50261751)
\curveto(301.30635749,995.26617264)(300.93302349,995.1479502)(300.49746715,995.1479502)
\curveto(300.04946635,995.1479502)(299.66991011,995.26617264)(299.35879845,995.50261751)
\curveto(299.04768678,995.75150684)(298.89213094,996.14972978)(298.89213094,996.69728632)
\closepath
}
}
{
\newrgbcolor{curcolor}{0 0 0}
\pscustom[linestyle=none,fillstyle=solid,fillcolor=curcolor]
{
\newpath
\moveto(304.4921387,1008.21464029)
\curveto(304.4921387,1008.7373079)(304.63525007,1009.0919752)(304.9214728,1009.2786422)
\curveto(305.22014001,1009.47775367)(305.58102954,1009.5773094)(306.00414141,1009.5773094)
\curveto(306.41480882,1009.5773094)(306.76947612,1009.47775367)(307.06814332,1009.2786422)
\curveto(307.36681052,1009.0919752)(307.51614412,1008.7373079)(307.51614412,1008.21464029)
\curveto(307.51614412,1007.70441716)(307.36681052,1007.34974985)(307.06814332,1007.15063839)
\curveto(306.76947612,1006.95152692)(306.41480882,1006.85197118)(306.00414141,1006.85197118)
\curveto(305.58102954,1006.85197118)(305.22014001,1006.95152692)(304.9214728,1007.15063839)
\curveto(304.63525007,1007.34974985)(304.4921387,1007.70441716)(304.4921387,1008.21464029)
\closepath
\moveto(303.7828041,990.91060928)
\curveto(303.45924796,990.91060928)(303.12946959,990.93549821)(302.79346899,990.98527608)
\curveto(302.45746839,991.02260948)(302.17746789,991.07238734)(301.95346748,991.13460968)
\lineto(301.95346748,993.31861359)
\curveto(302.17746789,993.25639126)(302.38902382,993.21283562)(302.58813529,993.18794669)
\curveto(302.78724676,993.16305776)(303.01124716,993.15061329)(303.26013649,993.15061329)
\curveto(303.6334705,993.15061329)(303.9508044,993.25639126)(304.2121382,993.46794719)
\curveto(304.473472,993.67950313)(304.6041389,994.09017053)(304.6041389,994.6999494)
\lineto(304.6041389,1005.58263558)
\lineto(307.38547722,1005.58263558)
\lineto(307.38547722,994.289282)
\curveto(307.38547722,993.66705866)(307.26725479,993.10083542)(307.03080992,992.59061229)
\curveto(306.79436505,992.08038915)(306.40858658,991.67594398)(305.87347451,991.37727678)
\curveto(305.35080691,991.06616511)(304.65391677,990.91060928)(303.7828041,990.91060928)
\closepath
}
}
{
\newrgbcolor{curcolor}{0 0 0}
\pscustom[linestyle=none,fillstyle=solid,fillcolor=curcolor]
{
\newpath
\moveto(317.40950333,998.41462273)
\curveto(317.40950333,997.38173199)(317.04239156,996.58528611)(316.30816802,996.02528511)
\curveto(315.58638895,995.47772857)(314.50372034,995.2039503)(313.0601622,995.2039503)
\curveto(312.35082759,995.2039503)(311.74104872,995.25372817)(311.23082558,995.35328391)
\curveto(310.72060245,995.44039517)(310.21037931,995.58972877)(309.70015617,995.80128471)
\lineto(309.70015617,998.09728882)
\curveto(310.24771271,997.84839949)(310.83882488,997.64306579)(311.47349268,997.48128772)
\curveto(312.10816049,997.31950965)(312.66816149,997.23862062)(313.1534957,997.23862062)
\curveto(313.68860777,997.23862062)(314.07438624,997.31950965)(314.3108311,997.48128772)
\curveto(314.54727597,997.64306579)(314.66549841,997.85462172)(314.66549841,998.11595552)
\curveto(314.66549841,998.29017806)(314.61572054,998.44573389)(314.51616481,998.58262303)
\curveto(314.42905354,998.71951216)(314.22994207,998.875068)(313.9188304,999.04929053)
\curveto(313.60771873,999.22351307)(313.12238453,999.44751347)(312.46282779,999.72129174)
\curveto(311.81571552,999.99507)(311.28682568,1000.26262604)(310.87615828,1000.52395984)
\curveto(310.47793534,1000.79773811)(310.17926814,1001.12129425)(309.98015667,1001.49462825)
\curveto(309.78104521,1001.88040672)(309.68148947,1002.35951869)(309.68148947,1002.93196416)
\curveto(309.68148947,1003.87774363)(310.04860124,1004.58707824)(310.78282478,1005.05996797)
\curveto(311.51704832,1005.53285771)(312.49393896,1005.76930258)(313.7134967,1005.76930258)
\curveto(314.3481645,1005.76930258)(314.95172114,1005.70708024)(315.52416661,1005.58263558)
\curveto(316.09661208,1005.45819091)(316.68772425,1005.25285721)(317.29750312,1004.96663447)
\lineto(316.45750162,1002.96929756)
\curveto(315.95972295,1003.18085349)(315.48683321,1003.35507603)(315.03883241,1003.49196516)
\curveto(314.59083161,1003.64129876)(314.13660857,1003.71596556)(313.6761633,1003.71596556)
\curveto(312.85482849,1003.71596556)(312.44416109,1003.49196516)(312.44416109,1003.04396436)
\curveto(312.44416109,1002.88218629)(312.49393896,1002.73285269)(312.59349469,1002.59596356)
\curveto(312.70549489,1002.47151889)(312.91082859,1002.33462975)(313.2094958,1002.18529615)
\curveto(313.52060747,1002.03596255)(313.9748305,1001.83685108)(314.57216491,1001.58796175)
\curveto(315.15705484,1001.35151688)(315.66105575,1001.10262755)(316.08416762,1000.84129374)
\curveto(316.50727949,1000.59240441)(316.83083562,1000.27507051)(317.05483602,999.88929204)
\curveto(317.29128089,999.50351357)(317.40950333,999.01195713)(317.40950333,998.41462273)
\closepath
}
}
{
\newrgbcolor{curcolor}{0 0 0}
\pscustom[linestyle=none,fillstyle=solid,fillcolor=curcolor]
{
\newpath
\moveto(215.99357178,974.3340232)
\lineto(212.11089815,984.52604147)
\lineto(215.02290337,984.52604147)
\lineto(216.98290688,978.72069773)
\curveto(217.09490708,978.37225266)(217.18201835,978.01136313)(217.24424068,977.63802913)
\curveto(217.30646302,977.26469512)(217.35001865,976.92869452)(217.37490758,976.63002732)
\lineto(217.44957439,976.63002732)
\curveto(217.48690779,977.30202852)(217.62379692,977.99891866)(217.86024179,978.72069773)
\lineto(219.8202453,984.52604147)
\lineto(222.73225052,984.52604147)
\lineto(218.8495769,974.3340232)
\closepath
}
}
{
\newrgbcolor{curcolor}{0 0 0}
\pscustom[linestyle=none,fillstyle=solid,fillcolor=curcolor]
{
\newpath
\moveto(233.44694085,979.44869904)
\curveto(233.44694085,977.75625156)(232.99894005,976.44958255)(232.10293844,975.52869201)
\curveto(231.2193813,974.60780147)(230.01226803,974.1473562)(228.48159862,974.1473562)
\curveto(227.53581915,974.1473562)(226.68959541,974.3526899)(225.9429274,974.76335731)
\curveto(225.20870386,975.17402471)(224.63003616,975.77135911)(224.20692429,976.55536052)
\curveto(223.78381242,977.35180639)(223.57225648,978.31625256)(223.57225648,979.44869904)
\curveto(223.57225648,981.14114652)(224.01403505,982.44159329)(224.89759219,983.35003936)
\curveto(225.78114933,984.25848544)(226.99448484,984.71270847)(228.53759872,984.71270847)
\curveto(229.49582266,984.71270847)(230.3420464,984.50737477)(231.07626994,984.09670737)
\curveto(231.81049347,983.68603997)(232.38916118,983.08870556)(232.81227305,982.30470416)
\curveto(233.23538492,981.52070275)(233.44694085,980.56870105)(233.44694085,979.44869904)
\closepath
\moveto(226.4095949,979.44869904)
\curveto(226.4095949,978.44069723)(226.57137297,977.67536253)(226.89492911,977.15269492)
\curveto(227.23092971,976.64247179)(227.77226401,976.38736022)(228.51893202,976.38736022)
\curveto(229.25315556,976.38736022)(229.78204539,976.64247179)(230.10560153,977.15269492)
\curveto(230.44160213,977.67536253)(230.60960243,978.44069723)(230.60960243,979.44869904)
\curveto(230.60960243,980.45670084)(230.44160213,981.20959108)(230.10560153,981.70736975)
\curveto(229.78204539,982.21759289)(229.24693332,982.47270446)(228.50026532,982.47270446)
\curveto(227.76604178,982.47270446)(227.23092971,982.21759289)(226.89492911,981.70736975)
\curveto(226.57137297,981.20959108)(226.4095949,980.45670084)(226.4095949,979.44869904)
\closepath
}
}
{
\newrgbcolor{curcolor}{0 0 0}
\pscustom[linestyle=none,fillstyle=solid,fillcolor=curcolor]
{
\newpath
\moveto(237.14293792,988.5207153)
\curveto(237.55360532,988.5207153)(237.90827262,988.42115957)(238.20693982,988.2220481)
\curveto(238.50560703,988.0353811)(238.65494063,987.68071379)(238.65494063,987.15804619)
\curveto(238.65494063,986.64782305)(238.50560703,986.29315575)(238.20693982,986.09404428)
\curveto(237.90827262,985.89493282)(237.55360532,985.79537708)(237.14293792,985.79537708)
\curveto(236.71982605,985.79537708)(236.35893651,985.89493282)(236.06026931,986.09404428)
\curveto(235.77404657,986.29315575)(235.63093521,986.64782305)(235.63093521,987.15804619)
\curveto(235.63093521,987.68071379)(235.77404657,988.0353811)(236.06026931,988.2220481)
\curveto(236.35893651,988.42115957)(236.71982605,988.5207153)(237.14293792,988.5207153)
\closepath
\moveto(238.52427373,984.52604147)
\lineto(238.52427373,974.3340232)
\lineto(235.74293541,974.3340232)
\lineto(235.74293541,984.52604147)
\closepath
}
}
{
\newrgbcolor{curcolor}{0 0 0}
\pscustom[linestyle=none,fillstyle=solid,fillcolor=curcolor]
{
\newpath
\moveto(245.5802926,974.1473562)
\curveto(244.06206766,974.1473562)(242.88606555,974.56424584)(242.05228628,975.39802511)
\curveto(241.23095147,976.23180438)(240.82028407,977.55714009)(240.82028407,979.37403224)
\curveto(240.82028407,980.61847891)(241.03184,981.63270295)(241.45495187,982.41670436)
\curveto(241.87806374,983.20070576)(242.46295368,983.77937347)(243.20962169,984.15270747)
\curveto(243.96873416,984.52604147)(244.83984683,984.71270847)(245.8229597,984.71270847)
\curveto(246.51984984,984.71270847)(247.12340648,984.64426391)(247.63362962,984.50737477)
\curveto(248.15629722,984.37048564)(248.61052026,984.20870757)(248.99629873,984.02204057)
\lineto(248.17496392,981.87537005)
\curveto(247.73940758,982.04959259)(247.32874018,982.19270396)(246.94296171,982.30470416)
\curveto(246.56962771,982.41670436)(246.19629371,982.47270446)(245.8229597,982.47270446)
\curveto(244.37940156,982.47270446)(243.65762249,981.44603595)(243.65762249,979.39269894)
\curveto(243.65762249,978.37225266)(243.84428949,977.61936243)(244.21762349,977.13402822)
\curveto(244.60340196,976.64869402)(245.13851403,976.40602692)(245.8229597,976.40602692)
\curveto(246.40784964,976.40602692)(246.92429501,976.48069372)(247.37229581,976.63002732)
\curveto(247.82029662,976.79180539)(248.25585295,977.00958355)(248.67896482,977.28336182)
\lineto(248.67896482,974.91269091)
\curveto(248.25585295,974.63891264)(247.80785215,974.4460234)(247.33496241,974.3340232)
\curveto(246.87451714,974.20957854)(246.28962721,974.1473562)(245.5802926,974.1473562)
\closepath
}
}
{
\newrgbcolor{curcolor}{0 0 0}
\pscustom[linestyle=none,fillstyle=solid,fillcolor=curcolor]
{
\newpath
\moveto(255.23097337,984.71270847)
\curveto(256.63719811,984.71270847)(257.75097788,984.3082633)(258.57231269,983.49937297)
\curveto(259.3936475,982.70292709)(259.8043149,981.56425839)(259.8043149,980.08336684)
\lineto(259.8043149,978.73936443)
\lineto(253.23363645,978.73936443)
\curveto(253.25852539,977.95536303)(253.48874802,977.33936192)(253.92430436,976.89136112)
\curveto(254.37230516,976.44336032)(254.98830627,976.21935992)(255.77230767,976.21935992)
\curveto(256.41941994,976.21935992)(257.01053211,976.28158225)(257.54564418,976.40602692)
\curveto(258.09320072,976.54291605)(258.65320172,976.74824975)(259.22564719,977.02202802)
\lineto(259.22564719,974.87535751)
\curveto(258.71542406,974.62646817)(258.18653422,974.4460234)(257.63897768,974.3340232)
\curveto(257.09142115,974.20957854)(256.42564218,974.1473562)(255.64164077,974.1473562)
\curveto(254.6211945,974.1473562)(253.71897066,974.3340232)(252.93496925,974.70735721)
\curveto(252.15096785,975.09313567)(251.53496674,975.66558115)(251.08696594,976.42469362)
\curveto(250.63896514,977.19625056)(250.41496473,978.1731412)(250.41496473,979.35536554)
\curveto(250.41496473,980.53758988)(250.6140762,981.52692499)(251.01229914,982.32337086)
\curveto(251.42296654,983.11981673)(251.98918978,983.71715113)(252.71096885,984.11537407)
\curveto(253.43274792,984.51359701)(254.27274943,984.71270847)(255.23097337,984.71270847)
\closepath
\moveto(255.24964007,982.73403826)
\curveto(254.70208353,982.73403826)(254.25408273,982.55981573)(253.90563766,982.21137066)
\curveto(253.55719259,981.86292559)(253.35185889,981.32159128)(253.28963655,980.58736775)
\lineto(257.19097688,980.58736775)
\curveto(257.17853241,981.19714662)(257.01053211,981.70736975)(256.68697598,982.11803716)
\curveto(256.37586431,982.52870456)(255.89675234,982.73403826)(255.24964007,982.73403826)
\closepath
}
}
{
\newrgbcolor{curcolor}{0 0 0}
\pscustom[linestyle=none,fillstyle=solid,fillcolor=curcolor]
{
\newpath
\moveto(261.67097794,975.64069221)
\curveto(261.67097794,976.21313768)(261.82653377,976.61136062)(262.13764544,976.83536102)
\curveto(262.44875711,977.07180589)(262.82831334,977.19002832)(263.27631415,977.19002832)
\curveto(263.71187048,977.19002832)(264.08520449,977.07180589)(264.39631616,976.83536102)
\curveto(264.70742782,976.61136062)(264.86298366,976.21313768)(264.86298366,975.64069221)
\curveto(264.86298366,975.09313567)(264.70742782,974.69491274)(264.39631616,974.4460234)
\curveto(264.08520449,974.20957854)(263.71187048,974.0913561)(263.27631415,974.0913561)
\curveto(262.82831334,974.0913561)(262.44875711,974.20957854)(262.13764544,974.4460234)
\curveto(261.82653377,974.69491274)(261.67097794,975.09313567)(261.67097794,975.64069221)
\closepath
}
}
{
\newrgbcolor{curcolor}{0 0 0}
\pscustom[linestyle=none,fillstyle=solid,fillcolor=curcolor]
{
\newpath
\moveto(267.2709857,987.15804619)
\curveto(267.2709857,987.68071379)(267.41409706,988.0353811)(267.7003198,988.2220481)
\curveto(267.998987,988.42115957)(268.35987654,988.5207153)(268.78298841,988.5207153)
\curveto(269.19365581,988.5207153)(269.54832311,988.42115957)(269.84699031,988.2220481)
\curveto(270.14565752,988.0353811)(270.29499112,987.68071379)(270.29499112,987.15804619)
\curveto(270.29499112,986.64782305)(270.14565752,986.29315575)(269.84699031,986.09404428)
\curveto(269.54832311,985.89493282)(269.19365581,985.79537708)(268.78298841,985.79537708)
\curveto(268.35987654,985.79537708)(267.998987,985.89493282)(267.7003198,986.09404428)
\curveto(267.41409706,986.29315575)(267.2709857,986.64782305)(267.2709857,987.15804619)
\closepath
\moveto(266.56165109,969.85401517)
\curveto(266.23809496,969.85401517)(265.90831659,969.87890411)(265.57231598,969.92868197)
\curveto(265.23631538,969.96601537)(264.95631488,970.01579324)(264.73231448,970.07801557)
\lineto(264.73231448,972.26201949)
\curveto(264.95631488,972.19979715)(265.16787081,972.15624152)(265.36698228,972.13135259)
\curveto(265.56609375,972.10646365)(265.79009415,972.09401919)(266.03898349,972.09401919)
\curveto(266.41231749,972.09401919)(266.72965139,972.19979715)(266.99098519,972.41135309)
\curveto(267.252319,972.62290902)(267.3829859,973.03357643)(267.3829859,973.6433553)
\lineto(267.3829859,984.52604147)
\lineto(270.16432422,984.52604147)
\lineto(270.16432422,973.2326879)
\curveto(270.16432422,972.61046456)(270.04610178,972.04424132)(269.80965691,971.53401818)
\curveto(269.57321204,971.02379505)(269.18743358,970.61934988)(268.6523215,970.32068268)
\curveto(268.1296539,970.00957101)(267.43276376,969.85401517)(266.56165109,969.85401517)
\closepath
}
}
{
\newrgbcolor{curcolor}{0 0 0}
\pscustom[linestyle=none,fillstyle=solid,fillcolor=curcolor]
{
\newpath
\moveto(280.18835032,977.35802862)
\curveto(280.18835032,976.32513788)(279.82123855,975.52869201)(279.08701501,974.96869101)
\curveto(278.36523594,974.42113447)(277.28256733,974.1473562)(275.83900919,974.1473562)
\curveto(275.12967458,974.1473562)(274.51989571,974.19713407)(274.00967258,974.2966898)
\curveto(273.49944944,974.38380107)(272.9892263,974.53313467)(272.47900317,974.74469061)
\lineto(272.47900317,977.04069472)
\curveto(273.0265597,976.79180539)(273.61767187,976.58647168)(274.25233968,976.42469362)
\curveto(274.88700748,976.26291555)(275.44700849,976.18202652)(275.93234269,976.18202652)
\curveto(276.46745476,976.18202652)(276.85323323,976.26291555)(277.0896781,976.42469362)
\curveto(277.32612297,976.58647168)(277.4443454,976.79802762)(277.4443454,977.05936142)
\curveto(277.4443454,977.23358396)(277.39456753,977.38913979)(277.2950118,977.52602892)
\curveto(277.20790053,977.66291806)(277.00878906,977.81847389)(276.69767739,977.99269643)
\curveto(276.38656573,978.16691896)(275.90123152,978.39091936)(275.24167479,978.66469763)
\curveto(274.59456251,978.9384759)(274.06567268,979.20603194)(273.65500527,979.46736574)
\curveto(273.25678234,979.74114401)(272.95811514,980.06470014)(272.75900367,980.43803414)
\curveto(272.5598922,980.82381261)(272.46033647,981.30292458)(272.46033647,981.87537005)
\curveto(272.46033647,982.82114953)(272.82744824,983.53048413)(273.56167177,984.00337387)
\curveto(274.29589531,984.47626361)(275.27278595,984.71270847)(276.49234369,984.71270847)
\curveto(277.1270115,984.71270847)(277.73056814,984.65048614)(278.30301361,984.52604147)
\curveto(278.87545908,984.4015968)(279.46657125,984.1962631)(280.07635012,983.91004037)
\lineto(279.23634861,981.91270345)
\curveto(278.73856994,982.12425939)(278.26568021,982.29848192)(277.8176794,982.43537106)
\curveto(277.3696786,982.58470466)(276.91545556,982.65937146)(276.45501029,982.65937146)
\curveto(275.63367549,982.65937146)(275.22300808,982.43537106)(275.22300808,981.98737026)
\curveto(275.22300808,981.82559219)(275.27278595,981.67625859)(275.37234169,981.53936945)
\curveto(275.48434189,981.41492478)(275.68967559,981.27803565)(275.98834279,981.12870205)
\curveto(276.29945446,980.97936845)(276.7536775,980.78025698)(277.3510119,980.53136765)
\curveto(277.93590184,980.29492278)(278.43990274,980.04603344)(278.86301461,979.78469964)
\curveto(279.28612648,979.5358103)(279.60968261,979.2184764)(279.83368302,978.83269793)
\curveto(280.07012788,978.44691946)(280.18835032,977.95536303)(280.18835032,977.35802862)
\closepath
}
}
{
\newrgbcolor{curcolor}{0 0 0}
\pscustom[linestyle=none,fillstyle=solid,fillcolor=curcolor]
{
\newpath
\moveto(215.87648174,957.98762683)
\lineto(212.59114252,962.97163577)
\lineto(215.74581484,962.97163577)
\lineto(217.72448506,959.72362995)
\lineto(219.72182197,962.97163577)
\lineto(222.87649429,962.97163577)
\lineto(219.55382167,957.98762683)
\lineto(223.02582789,952.7796175)
\lineto(219.87115557,952.7796175)
\lineto(217.72448506,956.27029042)
\lineto(215.57781454,952.7796175)
\lineto(212.42314222,952.7796175)
\closepath
}
}
{
\newrgbcolor{curcolor}{0 0 0}
\pscustom[linestyle=none,fillstyle=solid,fillcolor=curcolor]
{
\newpath
\moveto(236.54050218,963.15830277)
\curveto(237.69783759,963.15830277)(238.56895026,962.85963557)(239.1538402,962.26230116)
\curveto(239.7511746,961.67741123)(240.0498418,960.73163175)(240.0498418,959.42496274)
\lineto(240.0498418,952.7796175)
\lineto(237.26850349,952.7796175)
\lineto(237.26850349,958.73429484)
\curveto(237.26850349,960.20274192)(236.75828035,960.93696545)(235.73783408,960.93696545)
\curveto(235.00361054,960.93696545)(234.48094293,960.67563165)(234.16983126,960.15296405)
\curveto(233.8587196,959.63029644)(233.70316376,958.87740621)(233.70316376,957.89429333)
\lineto(233.70316376,952.7796175)
\lineto(230.92182544,952.7796175)
\lineto(230.92182544,958.73429484)
\curveto(230.92182544,960.20274192)(230.41160231,960.93696545)(229.39115603,960.93696545)
\curveto(228.61959909,960.93696545)(228.08448702,960.64452049)(227.78581982,960.05963055)
\curveto(227.49959709,959.48718508)(227.35648572,958.65962804)(227.35648572,957.57695943)
\lineto(227.35648572,952.7796175)
\lineto(224.5751474,952.7796175)
\lineto(224.5751474,962.97163577)
\lineto(226.70315121,962.97163577)
\lineto(227.07648522,961.66496676)
\lineto(227.22581882,961.66496676)
\curveto(227.53693049,962.18763436)(227.96004236,962.5671906)(228.49515443,962.80363547)
\curveto(229.04271096,963.04008033)(229.6089342,963.15830277)(230.19382414,963.15830277)
\curveto(230.94049214,963.15830277)(231.56893771,963.0338581)(232.07916085,962.78496877)
\curveto(232.60182845,962.5485239)(233.00627362,962.1751899)(233.29249636,961.66496676)
\lineto(233.53516346,961.66496676)
\curveto(233.84627513,962.18763436)(234.27560923,962.5671906)(234.82316577,962.80363547)
\curveto(235.38316677,963.04008033)(235.95561224,963.15830277)(236.54050218,963.15830277)
\closepath
}
}
{
\newrgbcolor{curcolor}{0 0 0}
\pscustom[linestyle=none,fillstyle=solid,fillcolor=curcolor]
{
\newpath
\moveto(245.68719018,952.7796175)
\lineto(242.90585186,952.7796175)
\lineto(242.90585186,966.9663096)
\lineto(245.68719018,966.9663096)
\closepath
}
}
{
\newrgbcolor{curcolor}{0 0 0}
\pscustom[linestyle=none,fillstyle=solid,fillcolor=curcolor]
{
\newpath
\moveto(248.20720283,954.08628651)
\curveto(248.20720283,954.65873198)(248.36275866,955.05695491)(248.67387033,955.28095532)
\curveto(248.984982,955.51740018)(249.36453824,955.63562262)(249.81253904,955.63562262)
\curveto(250.24809538,955.63562262)(250.62142938,955.51740018)(250.93254105,955.28095532)
\curveto(251.24365272,955.05695491)(251.39920855,954.65873198)(251.39920855,954.08628651)
\curveto(251.39920855,953.53872997)(251.24365272,953.14050703)(250.93254105,952.8916177)
\curveto(250.62142938,952.65517283)(250.24809538,952.5369504)(249.81253904,952.5369504)
\curveto(249.36453824,952.5369504)(248.984982,952.65517283)(248.67387033,952.8916177)
\curveto(248.36275866,953.14050703)(248.20720283,953.53872997)(248.20720283,954.08628651)
\closepath
}
}
{
\newrgbcolor{curcolor}{0 0 0}
\pscustom[linestyle=none,fillstyle=solid,fillcolor=curcolor]
{
\newpath
\moveto(253.80721059,965.60364049)
\curveto(253.80721059,966.12630809)(253.95032195,966.48097539)(254.23654469,966.66764239)
\curveto(254.53521189,966.86675386)(254.89610143,966.9663096)(255.3192133,966.9663096)
\curveto(255.7298807,966.9663096)(256.084548,966.86675386)(256.3832152,966.66764239)
\curveto(256.68188241,966.48097539)(256.83121601,966.12630809)(256.83121601,965.60364049)
\curveto(256.83121601,965.09341735)(256.68188241,964.73875005)(256.3832152,964.53963858)
\curveto(256.084548,964.34052711)(255.7298807,964.24097138)(255.3192133,964.24097138)
\curveto(254.89610143,964.24097138)(254.53521189,964.34052711)(254.23654469,964.53963858)
\curveto(253.95032195,964.73875005)(253.80721059,965.09341735)(253.80721059,965.60364049)
\closepath
\moveto(253.09787598,948.29960947)
\curveto(252.77431985,948.29960947)(252.44454148,948.3244984)(252.10854088,948.37427627)
\curveto(251.77254027,948.41160967)(251.49253977,948.46138753)(251.26853937,948.52360987)
\lineto(251.26853937,950.70761378)
\curveto(251.49253977,950.64539145)(251.70409571,950.60183582)(251.90320717,950.57694688)
\curveto(252.10231864,950.55205795)(252.32631904,950.53961348)(252.57520838,950.53961348)
\curveto(252.94854238,950.53961348)(253.26587628,950.64539145)(253.52721009,950.85694738)
\curveto(253.78854389,951.06850332)(253.91921079,951.47917072)(253.91921079,952.08894959)
\lineto(253.91921079,962.97163577)
\lineto(256.70054911,962.97163577)
\lineto(256.70054911,951.67828219)
\curveto(256.70054911,951.05605885)(256.58232667,950.48983562)(256.3458818,949.97961248)
\curveto(256.10943694,949.46938934)(255.72365847,949.06494417)(255.1885464,948.76627697)
\curveto(254.66587879,948.4551653)(253.96898865,948.29960947)(253.09787598,948.29960947)
\closepath
}
}
{
\newrgbcolor{curcolor}{0 0 0}
\pscustom[linestyle=none,fillstyle=solid,fillcolor=curcolor]
{
\newpath
\moveto(266.7245714,955.80362292)
\curveto(266.7245714,954.77073218)(266.35745963,953.97428631)(265.62323609,953.4142853)
\curveto(264.90145702,952.86672877)(263.81878841,952.5929505)(262.37523027,952.5929505)
\curveto(261.66589566,952.5929505)(261.05611679,952.64272836)(260.54589365,952.7422841)
\curveto(260.03567052,952.82939536)(259.52544738,952.97872897)(259.01522424,953.1902849)
\lineto(259.01522424,955.48628902)
\curveto(259.56278078,955.23739968)(260.15389295,955.03206598)(260.78856076,954.87028791)
\curveto(261.42322856,954.70850984)(261.98322956,954.62762081)(262.46856377,954.62762081)
\curveto(263.00367584,954.62762081)(263.38945431,954.70850984)(263.62589917,954.87028791)
\curveto(263.86234404,955.03206598)(263.98056648,955.24362191)(263.98056648,955.50495572)
\curveto(263.98056648,955.67917825)(263.93078861,955.83473409)(263.83123288,955.97162322)
\curveto(263.74412161,956.10851235)(263.54501014,956.26406819)(263.23389847,956.43829072)
\curveto(262.9227868,956.61251326)(262.4374526,956.83651366)(261.77789586,957.11029193)
\curveto(261.13078359,957.3840702)(260.60189375,957.65162623)(260.19122635,957.91296003)
\curveto(259.79300342,958.1867383)(259.49433621,958.51029444)(259.29522474,958.88362844)
\curveto(259.09611328,959.26940691)(258.99655754,959.74851888)(258.99655754,960.32096435)
\curveto(258.99655754,961.26674382)(259.36366931,961.97607843)(260.09789285,962.44896816)
\curveto(260.83211639,962.9218579)(261.80900703,963.15830277)(263.02856477,963.15830277)
\curveto(263.66323257,963.15830277)(264.26678921,963.09608044)(264.83923468,962.97163577)
\curveto(265.41168015,962.8471911)(266.00279232,962.6418574)(266.6125712,962.35563466)
\lineto(265.77256969,960.35829775)
\curveto(265.27479102,960.56985368)(264.80190128,960.74407622)(264.35390048,960.88096535)
\curveto(263.90589968,961.03029895)(263.45167664,961.10496575)(262.99123137,961.10496575)
\curveto(262.16989656,961.10496575)(261.75922916,960.88096535)(261.75922916,960.43296455)
\curveto(261.75922916,960.27118648)(261.80900703,960.12185288)(261.90856276,959.98496375)
\curveto(262.02056296,959.86051908)(262.22589667,959.72362995)(262.52456387,959.57429634)
\curveto(262.83567554,959.42496274)(263.28989857,959.22585128)(263.88723298,958.97696194)
\curveto(264.47212291,958.74051707)(264.97612382,958.49162774)(265.39923569,958.23029394)
\curveto(265.82234756,957.9814046)(266.14590369,957.6640707)(266.36990409,957.27829223)
\curveto(266.60634896,956.89251376)(266.7245714,956.40095732)(266.7245714,955.80362292)
\closepath
}
}
{
\newrgbcolor{curcolor}{0 0 0}
\pscustom[linestyle=none,fillstyle=solid,fillcolor=curcolor]
{
\newpath
\moveto(170.81519994,922.53448665)
\curveto(169.68275346,922.53448665)(168.75564069,922.97626522)(168.03386162,923.85982236)
\curveto(167.32452701,924.75582396)(166.96985971,926.06871521)(166.96985971,927.79849608)
\curveto(166.96985971,929.54072143)(167.33074925,930.85983491)(168.05252832,931.75583651)
\curveto(168.77430739,932.65183812)(169.72008686,933.09983892)(170.88986674,933.09983892)
\curveto(171.62409028,933.09983892)(172.22764691,932.95672755)(172.70053665,932.67050482)
\curveto(173.17342639,932.38428208)(173.54676039,932.02961478)(173.82053866,931.60650291)
\lineto(173.91387216,931.60650291)
\curveto(173.87653876,931.80561438)(173.83298312,932.09183711)(173.78320526,932.46517112)
\curveto(173.73342739,932.85094959)(173.70853846,933.24295029)(173.70853846,933.64117322)
\lineto(173.70853846,936.90784575)
\lineto(176.48987678,936.90784575)
\lineto(176.48987678,922.72115365)
\lineto(174.36187296,922.72115365)
\lineto(173.82053866,924.04648936)
\lineto(173.70853846,924.04648936)
\curveto(173.43476019,923.62337749)(173.06764842,923.26248795)(172.60720315,922.96382075)
\curveto(172.14675788,922.67759802)(171.54942348,922.53448665)(170.81519994,922.53448665)
\closepath
\moveto(171.78586834,924.75582396)
\curveto(172.54498082,924.75582396)(173.08009289,924.97982437)(173.39120456,925.42782517)
\curveto(173.70231622,925.88827044)(173.87031652,926.57271611)(173.89520546,927.48116218)
\lineto(173.89520546,927.77982938)
\curveto(173.89520546,928.76294226)(173.73964962,929.5158325)(173.42853796,930.0385001)
\curveto(173.12987075,930.57361217)(172.56986975,930.84116821)(171.74853494,930.84116821)
\curveto(171.13875607,930.84116821)(170.6596441,930.57361217)(170.31119903,930.0385001)
\curveto(169.96275396,929.5158325)(169.78853143,928.75672002)(169.78853143,927.76116268)
\curveto(169.78853143,926.76560534)(169.96275396,926.01271511)(170.31119903,925.50249197)
\curveto(170.6596441,925.0047133)(171.15120054,924.75582396)(171.78586834,924.75582396)
\closepath
}
}
{
\newrgbcolor{curcolor}{0 0 0}
\pscustom[linestyle=none,fillstyle=solid,fillcolor=curcolor]
{
\newpath
\moveto(183.58321197,933.11850562)
\curveto(184.95210331,933.11850562)(185.99743852,932.81983842)(186.71921759,932.22250401)
\curveto(187.45344113,931.63761408)(187.8205529,930.73539024)(187.8205529,929.5158325)
\lineto(187.8205529,922.72115365)
\lineto(185.87921609,922.72115365)
\lineto(185.33788178,924.10248946)
\lineto(185.26321498,924.10248946)
\curveto(184.82765865,923.55493292)(184.36721338,923.15670999)(183.88187917,922.90782065)
\curveto(183.39654497,922.65893132)(182.730766,922.53448665)(181.88454226,922.53448665)
\curveto(180.97609619,922.53448665)(180.22320595,922.79582045)(179.62587154,923.31848805)
\curveto(179.02853714,923.84115566)(178.72986994,924.65626823)(178.72986994,925.76382577)
\curveto(178.72986994,926.84649438)(179.10942617,927.64294025)(179.86853864,928.15316339)
\curveto(180.62765112,928.66338652)(181.76631982,928.94960926)(183.28454477,929.01183159)
\lineto(185.05788128,929.06783169)
\lineto(185.05788128,929.5158325)
\curveto(185.05788128,930.05094457)(184.91476991,930.44294527)(184.62854718,930.6918346)
\curveto(184.35476891,930.94072394)(183.96899044,931.06516861)(183.47121177,931.06516861)
\curveto(182.9734331,931.06516861)(182.4880989,930.99050181)(182.01520916,930.84116821)
\curveto(181.54231942,930.70427907)(181.06942969,930.53005654)(180.59653995,930.3185006)
\lineto(179.68187164,932.20383731)
\curveto(180.21698371,932.47761558)(180.82054035,932.69539375)(181.49254156,932.85717182)
\curveto(182.16454276,933.03139435)(182.8614329,933.11850562)(183.58321197,933.11850562)
\closepath
\moveto(185.05788128,927.44382878)
\lineto(183.97521267,927.40649538)
\curveto(183.07921107,927.38160645)(182.45698773,927.21982838)(182.10854266,926.92116118)
\curveto(181.76009759,926.62249398)(181.58587506,926.23049327)(181.58587506,925.74515907)
\curveto(181.58587506,925.3220472)(181.71031972,925.01715777)(181.95920906,924.83049076)
\curveto(182.20809839,924.65626823)(182.53165453,924.56915696)(182.92987747,924.56915696)
\curveto(183.52721187,924.56915696)(184.03121277,924.7433795)(184.44188018,925.09182457)
\curveto(184.85254758,925.4527141)(185.05788128,925.95671501)(185.05788128,926.60382728)
\closepath
}
}
{
\newrgbcolor{curcolor}{0 0 0}
\pscustom[linestyle=none,fillstyle=solid,fillcolor=curcolor]
{
\newpath
\moveto(194.9698998,924.75582396)
\curveto(195.28101146,924.75582396)(195.57967867,924.7807129)(195.8659014,924.83049076)
\curveto(196.15212414,924.8927131)(196.43834687,924.97360213)(196.72456961,925.07315787)
\lineto(196.72456961,923.00115415)
\curveto(196.42590241,922.86426502)(196.0525684,922.75226482)(195.6045676,922.66515355)
\curveto(195.16901126,922.57804228)(194.68989929,922.53448665)(194.16723169,922.53448665)
\curveto(193.55745282,922.53448665)(193.00989628,922.63404238)(192.52456208,922.83315385)
\curveto(192.05167234,923.03226532)(191.67211611,923.37448815)(191.38589337,923.85982236)
\curveto(191.1121151,924.34515656)(190.97522597,925.02960223)(190.97522597,925.91315937)
\lineto(190.97522597,930.82250151)
\lineto(189.64989026,930.82250151)
\lineto(189.64989026,931.99850361)
\lineto(191.18055967,932.93183862)
\lineto(191.98322778,935.07850913)
\lineto(193.75656429,935.07850913)
\lineto(193.75656429,932.91317192)
\lineto(196.61256941,932.91317192)
\lineto(196.61256941,930.82250151)
\lineto(193.75656429,930.82250151)
\lineto(193.75656429,925.91315937)
\curveto(193.75656429,925.5273809)(193.86856449,925.23493593)(194.09256489,925.03582447)
\curveto(194.31656529,924.84915746)(194.60901026,924.75582396)(194.9698998,924.75582396)
\closepath
}
}
{
\newrgbcolor{curcolor}{0 0 0}
\pscustom[linestyle=none,fillstyle=solid,fillcolor=curcolor]
{
\newpath
\moveto(202.95924837,933.11850562)
\curveto(204.32813971,933.11850562)(205.37347492,932.81983842)(206.09525399,932.22250401)
\curveto(206.82947753,931.63761408)(207.1965893,930.73539024)(207.1965893,929.5158325)
\lineto(207.1965893,922.72115365)
\lineto(205.25525249,922.72115365)
\lineto(204.71391818,924.10248946)
\lineto(204.63925138,924.10248946)
\curveto(204.20369505,923.55493292)(203.74324978,923.15670999)(203.25791557,922.90782065)
\curveto(202.77258137,922.65893132)(202.1068024,922.53448665)(201.26057866,922.53448665)
\curveto(200.35213259,922.53448665)(199.59924235,922.79582045)(199.00190794,923.31848805)
\curveto(198.40457354,923.84115566)(198.10590634,924.65626823)(198.10590634,925.76382577)
\curveto(198.10590634,926.84649438)(198.48546257,927.64294025)(199.24457504,928.15316339)
\curveto(200.00368752,928.66338652)(201.14235622,928.94960926)(202.66058117,929.01183159)
\lineto(204.43391768,929.06783169)
\lineto(204.43391768,929.5158325)
\curveto(204.43391768,930.05094457)(204.29080631,930.44294527)(204.00458358,930.6918346)
\curveto(203.73080531,930.94072394)(203.34502684,931.06516861)(202.84724817,931.06516861)
\curveto(202.3494695,931.06516861)(201.8641353,930.99050181)(201.39124556,930.84116821)
\curveto(200.91835582,930.70427907)(200.44546609,930.53005654)(199.97257635,930.3185006)
\lineto(199.05790804,932.20383731)
\curveto(199.59302011,932.47761558)(200.19657675,932.69539375)(200.86857796,932.85717182)
\curveto(201.54057916,933.03139435)(202.2374693,933.11850562)(202.95924837,933.11850562)
\closepath
\moveto(204.43391768,927.44382878)
\lineto(203.35124907,927.40649538)
\curveto(202.45524747,927.38160645)(201.83302413,927.21982838)(201.48457906,926.92116118)
\curveto(201.13613399,926.62249398)(200.96191146,926.23049327)(200.96191146,925.74515907)
\curveto(200.96191146,925.3220472)(201.08635612,925.01715777)(201.33524546,924.83049076)
\curveto(201.58413479,924.65626823)(201.90769093,924.56915696)(202.30591387,924.56915696)
\curveto(202.90324827,924.56915696)(203.40724917,924.7433795)(203.81791658,925.09182457)
\curveto(204.22858398,925.4527141)(204.43391768,925.95671501)(204.43391768,926.60382728)
\closepath
}
}
{
\newrgbcolor{curcolor}{0 0 0}
\pscustom[linestyle=none,fillstyle=solid,fillcolor=curcolor]
{
\newpath
\moveto(212.83393444,936.90784575)
\lineto(212.83393444,933.60383982)
\curveto(212.83393444,933.21806136)(212.82148997,932.83850512)(212.79660104,932.46517112)
\curveto(212.77171211,932.09183711)(212.74682317,931.79939215)(212.72193424,931.58783621)
\lineto(212.83393444,931.58783621)
\curveto(213.10771271,932.01094808)(213.47482448,932.36561538)(213.93526975,932.65183812)
\curveto(214.39571502,932.95050532)(214.99304942,933.09983892)(215.72727296,933.09983892)
\curveto(216.8721639,933.09983892)(217.79927667,932.65183812)(218.50861128,931.75583651)
\curveto(219.21794588,930.87227937)(219.57261319,929.56561036)(219.57261319,927.83582948)
\curveto(219.57261319,926.09360414)(219.21172365,924.77449066)(218.48994458,923.87848906)
\curveto(217.76816551,922.98248745)(216.82238603,922.53448665)(215.65260616,922.53448665)
\curveto(214.90593815,922.53448665)(214.31482598,922.66515355)(213.87926965,922.92648735)
\curveto(213.45615778,923.20026562)(213.10771271,923.50515506)(212.83393444,923.84115566)
\lineto(212.64726744,923.84115566)
\lineto(212.18059993,922.72115365)
\lineto(210.05259612,922.72115365)
\lineto(210.05259612,936.90784575)
\closepath
\moveto(214.83127135,930.87850161)
\curveto(214.10949228,930.87850161)(213.59926914,930.64827897)(213.30060194,930.1878337)
\curveto(213.00193474,929.7398329)(212.84637891,929.06160946)(212.83393444,928.15316339)
\lineto(212.83393444,927.85449618)
\curveto(212.83393444,926.87138331)(212.97704581,926.11227084)(213.26326854,925.57715877)
\curveto(213.56193574,925.05449117)(214.09704781,924.79315736)(214.86860475,924.79315736)
\curveto(215.44105022,924.79315736)(215.89527326,925.05449117)(216.23127386,925.57715877)
\curveto(216.56727446,926.11227084)(216.73527477,926.87760554)(216.73527477,927.87316289)
\curveto(216.73527477,928.86872023)(216.56105223,929.61538823)(216.21260716,930.1131669)
\curveto(215.87660656,930.62339004)(215.41616129,930.87850161)(214.83127135,930.87850161)
\closepath
}
}
{
\newrgbcolor{curcolor}{0 0 0}
\pscustom[linestyle=none,fillstyle=solid,fillcolor=curcolor]
{
\newpath
\moveto(226.04994632,933.11850562)
\curveto(227.41883766,933.11850562)(228.46417287,932.81983842)(229.18595194,932.22250401)
\curveto(229.92017548,931.63761408)(230.28728725,930.73539024)(230.28728725,929.5158325)
\lineto(230.28728725,922.72115365)
\lineto(228.34595044,922.72115365)
\lineto(227.80461613,924.10248946)
\lineto(227.72994933,924.10248946)
\curveto(227.294393,923.55493292)(226.83394773,923.15670999)(226.34861352,922.90782065)
\curveto(225.86327932,922.65893132)(225.19750035,922.53448665)(224.35127661,922.53448665)
\curveto(223.44283054,922.53448665)(222.6899403,922.79582045)(222.09260589,923.31848805)
\curveto(221.49527149,923.84115566)(221.19660429,924.65626823)(221.19660429,925.76382577)
\curveto(221.19660429,926.84649438)(221.57616052,927.64294025)(222.335273,928.15316339)
\curveto(223.09438547,928.66338652)(224.23305418,928.94960926)(225.75127912,929.01183159)
\lineto(227.52461563,929.06783169)
\lineto(227.52461563,929.5158325)
\curveto(227.52461563,930.05094457)(227.38150426,930.44294527)(227.09528153,930.6918346)
\curveto(226.82150326,930.94072394)(226.43572479,931.06516861)(225.93794612,931.06516861)
\curveto(225.44016745,931.06516861)(224.95483325,930.99050181)(224.48194351,930.84116821)
\curveto(224.00905377,930.70427907)(223.53616404,930.53005654)(223.0632743,930.3185006)
\lineto(222.14860599,932.20383731)
\curveto(222.68371806,932.47761558)(223.2872747,932.69539375)(223.95927591,932.85717182)
\curveto(224.63127711,933.03139435)(225.32816725,933.11850562)(226.04994632,933.11850562)
\closepath
\moveto(227.52461563,927.44382878)
\lineto(226.44194702,927.40649538)
\curveto(225.54594542,927.38160645)(224.92372208,927.21982838)(224.57527701,926.92116118)
\curveto(224.22683194,926.62249398)(224.05260941,926.23049327)(224.05260941,925.74515907)
\curveto(224.05260941,925.3220472)(224.17705408,925.01715777)(224.42594341,924.83049076)
\curveto(224.67483275,924.65626823)(224.99838888,924.56915696)(225.39661182,924.56915696)
\curveto(225.99394622,924.56915696)(226.49794712,924.7433795)(226.90861453,925.09182457)
\curveto(227.31928193,925.4527141)(227.52461563,925.95671501)(227.52461563,926.60382728)
\closepath
}
}
{
\newrgbcolor{curcolor}{0 0 0}
\pscustom[linestyle=none,fillstyle=solid,fillcolor=curcolor]
{
\newpath
\moveto(240.25530682,925.74515907)
\curveto(240.25530682,924.71226833)(239.88819505,923.91582246)(239.15397151,923.35582145)
\curveto(238.43219244,922.80826492)(237.34952383,922.53448665)(235.90596569,922.53448665)
\curveto(235.19663109,922.53448665)(234.58685221,922.58426452)(234.07662908,922.68382025)
\curveto(233.56640594,922.77093152)(233.0561828,922.92026512)(232.54595967,923.13182105)
\lineto(232.54595967,925.42782517)
\curveto(233.0935162,925.17893583)(233.68462838,924.97360213)(234.31929618,924.81182406)
\curveto(234.95396398,924.650046)(235.51396499,924.56915696)(235.99929919,924.56915696)
\curveto(236.53441126,924.56915696)(236.92018973,924.650046)(237.1566346,924.81182406)
\curveto(237.39307947,924.97360213)(237.5113019,925.18515807)(237.5113019,925.44649187)
\curveto(237.5113019,925.6207144)(237.46152403,925.77627024)(237.3619683,925.91315937)
\curveto(237.27485703,926.05004851)(237.07574556,926.20560434)(236.7646339,926.37982687)
\curveto(236.45352223,926.55404941)(235.96818802,926.77804981)(235.30863129,927.05182808)
\curveto(234.66151902,927.32560635)(234.13262918,927.59316238)(233.72196178,927.85449618)
\curveto(233.32373884,928.12827445)(233.02507164,928.45183059)(232.82596017,928.82516459)
\curveto(232.6268487,929.21094306)(232.52729297,929.69005503)(232.52729297,930.2625005)
\curveto(232.52729297,931.20827997)(232.89440474,931.91761458)(233.62862827,932.39050432)
\curveto(234.36285181,932.86339405)(235.33974245,933.09983892)(236.55930019,933.09983892)
\curveto(237.193968,933.09983892)(237.79752464,933.03761659)(238.36997011,932.91317192)
\curveto(238.94241558,932.78872725)(239.53352775,932.58339355)(240.14330662,932.29717082)
\lineto(239.30330511,930.2998339)
\curveto(238.80552644,930.51138984)(238.33263671,930.68561237)(237.8846359,930.82250151)
\curveto(237.4366351,930.97183511)(236.98241206,931.04650191)(236.52196679,931.04650191)
\curveto(235.70063199,931.04650191)(235.28996459,930.82250151)(235.28996459,930.3745007)
\curveto(235.28996459,930.21272263)(235.33974245,930.06338903)(235.43929819,929.9264999)
\curveto(235.55129839,929.80205523)(235.75663209,929.6651661)(236.05529929,929.5158325)
\curveto(236.36641096,929.3664989)(236.820634,929.16738743)(237.4179684,928.91849809)
\curveto(238.00285834,928.68205322)(238.50685924,928.43316389)(238.92997111,928.17183009)
\curveto(239.35308298,927.92294075)(239.67663912,927.60560685)(239.90063952,927.21982838)
\curveto(240.13708439,926.83404991)(240.25530682,926.34249347)(240.25530682,925.74515907)
\closepath
}
}
{
\newrgbcolor{curcolor}{0 0 0}
\pscustom[linestyle=none,fillstyle=solid,fillcolor=curcolor]
{
\newpath
\moveto(246.62064523,933.09983892)
\curveto(248.02686998,933.09983892)(249.14064975,932.69539375)(249.96198456,931.88650341)
\curveto(250.78331936,931.09005754)(251.19398676,929.95138883)(251.19398676,928.47049729)
\lineto(251.19398676,927.12649488)
\lineto(244.62330832,927.12649488)
\curveto(244.64819725,926.34249347)(244.87841989,925.72649237)(245.31397622,925.27849157)
\curveto(245.76197703,924.83049076)(246.37797813,924.60649036)(247.16197954,924.60649036)
\curveto(247.80909181,924.60649036)(248.40020398,924.6687127)(248.93531605,924.79315736)
\curveto(249.48287259,924.9300465)(250.04287359,925.1353802)(250.61531906,925.40915847)
\lineto(250.61531906,923.26248795)
\curveto(250.10509592,923.01359862)(249.57620609,922.83315385)(249.02864955,922.72115365)
\curveto(248.48109301,922.59670898)(247.81531404,922.53448665)(247.03131264,922.53448665)
\curveto(246.01086636,922.53448665)(245.10864252,922.72115365)(244.32464112,923.09448765)
\curveto(243.54063971,923.48026612)(242.92463861,924.05271159)(242.4766378,924.81182406)
\curveto(242.028637,925.583381)(241.8046366,926.56027164)(241.8046366,927.74249598)
\curveto(241.8046366,928.92472033)(242.00374807,929.91405543)(242.401971,930.7105013)
\curveto(242.81263841,931.50694718)(243.37886164,932.10428158)(244.10064072,932.50250452)
\curveto(244.82241979,932.90072745)(245.66242129,933.09983892)(246.62064523,933.09983892)
\closepath
\moveto(246.63931193,931.12116871)
\curveto(246.0917554,931.12116871)(245.64375459,930.94694617)(245.29530952,930.5985011)
\curveto(244.94686446,930.25005603)(244.74153075,929.70872173)(244.67930842,928.97449819)
\lineto(248.58064875,928.97449819)
\curveto(248.56820428,929.58427706)(248.40020398,930.0945002)(248.07664784,930.5051676)
\curveto(247.76553617,930.91583501)(247.2864242,931.12116871)(246.63931193,931.12116871)
\closepath
}
}
{
\newrgbcolor{curcolor}{0 0 0}
\pscustom[linestyle=none,fillstyle=solid,fillcolor=curcolor]
{
\newpath
\moveto(253.0606498,924.02782266)
\curveto(253.0606498,924.60026813)(253.21620564,924.99849107)(253.52731731,925.22249147)
\curveto(253.83842897,925.45893634)(254.21798521,925.57715877)(254.66598601,925.57715877)
\curveto(255.10154235,925.57715877)(255.47487635,925.45893634)(255.78598802,925.22249147)
\curveto(256.09709969,924.99849107)(256.25265552,924.60026813)(256.25265552,924.02782266)
\curveto(256.25265552,923.48026612)(256.09709969,923.08204319)(255.78598802,922.83315385)
\curveto(255.47487635,922.59670898)(255.10154235,922.47848655)(254.66598601,922.47848655)
\curveto(254.21798521,922.47848655)(253.83842897,922.59670898)(253.52731731,922.83315385)
\curveto(253.21620564,923.08204319)(253.0606498,923.48026612)(253.0606498,924.02782266)
\closepath
}
}
{
\newrgbcolor{curcolor}{0 0 0}
\pscustom[linestyle=none,fillstyle=solid,fillcolor=curcolor]
{
\newpath
\moveto(258.66065756,935.54517664)
\curveto(258.66065756,936.06784424)(258.80376893,936.42251154)(259.08999166,936.60917854)
\curveto(259.38865887,936.80829001)(259.7495484,936.90784575)(260.17266027,936.90784575)
\curveto(260.58332767,936.90784575)(260.93799498,936.80829001)(261.23666218,936.60917854)
\curveto(261.53532938,936.42251154)(261.68466298,936.06784424)(261.68466298,935.54517664)
\curveto(261.68466298,935.0349535)(261.53532938,934.6802862)(261.23666218,934.48117473)
\curveto(260.93799498,934.28206326)(260.58332767,934.18250753)(260.17266027,934.18250753)
\curveto(259.7495484,934.18250753)(259.38865887,934.28206326)(259.08999166,934.48117473)
\curveto(258.80376893,934.6802862)(258.66065756,935.0349535)(258.66065756,935.54517664)
\closepath
\moveto(257.95132296,918.24114562)
\curveto(257.62776682,918.24114562)(257.29798845,918.26603455)(256.96198785,918.31581242)
\curveto(256.62598725,918.35314582)(256.34598675,918.40292369)(256.12198634,918.46514602)
\lineto(256.12198634,920.64914994)
\curveto(256.34598675,920.5869276)(256.55754268,920.54337197)(256.75665415,920.51848303)
\curveto(256.95576562,920.4935941)(257.17976602,920.48114963)(257.42865535,920.48114963)
\curveto(257.80198936,920.48114963)(258.11932326,920.5869276)(258.38065706,920.79848354)
\curveto(258.64199086,921.01003947)(258.77265776,921.42070687)(258.77265776,922.03048575)
\lineto(258.77265776,932.91317192)
\lineto(261.55399608,932.91317192)
\lineto(261.55399608,921.61981834)
\curveto(261.55399608,920.997595)(261.43577365,920.43137177)(261.19932878,919.92114863)
\curveto(260.96288391,919.41092549)(260.57710544,919.00648032)(260.04199337,918.70781312)
\curveto(259.51932577,918.39670145)(258.82243563,918.24114562)(257.95132296,918.24114562)
\closepath
}
}
{
\newrgbcolor{curcolor}{0 0 0}
\pscustom[linestyle=none,fillstyle=solid,fillcolor=curcolor]
{
\newpath
\moveto(271.57802218,925.74515907)
\curveto(271.57802218,924.71226833)(271.21091042,923.91582246)(270.47668688,923.35582145)
\curveto(269.75490781,922.80826492)(268.6722392,922.53448665)(267.22868105,922.53448665)
\curveto(266.51934645,922.53448665)(265.90956758,922.58426452)(265.39934444,922.68382025)
\curveto(264.88912131,922.77093152)(264.37889817,922.92026512)(263.86867503,923.13182105)
\lineto(263.86867503,925.42782517)
\curveto(264.41623157,925.17893583)(265.00734374,924.97360213)(265.64201154,924.81182406)
\curveto(266.27667935,924.650046)(266.83668035,924.56915696)(267.32201456,924.56915696)
\curveto(267.85712663,924.56915696)(268.2429051,924.650046)(268.47934996,924.81182406)
\curveto(268.71579483,924.97360213)(268.83401727,925.18515807)(268.83401727,925.44649187)
\curveto(268.83401727,925.6207144)(268.7842394,925.77627024)(268.68468366,925.91315937)
\curveto(268.5975724,926.05004851)(268.39846093,926.20560434)(268.08734926,926.37982687)
\curveto(267.77623759,926.55404941)(267.29090339,926.77804981)(266.63134665,927.05182808)
\curveto(265.98423438,927.32560635)(265.45534454,927.59316238)(265.04467714,927.85449618)
\curveto(264.6464542,928.12827445)(264.347787,928.45183059)(264.14867553,928.82516459)
\curveto(263.94956407,929.21094306)(263.85000833,929.69005503)(263.85000833,930.2625005)
\curveto(263.85000833,931.20827997)(264.2171201,931.91761458)(264.95134364,932.39050432)
\curveto(265.68556718,932.86339405)(266.66245782,933.09983892)(267.88201556,933.09983892)
\curveto(268.51668336,933.09983892)(269.12024,933.03761659)(269.69268547,932.91317192)
\curveto(270.26513094,932.78872725)(270.85624311,932.58339355)(271.46602198,932.29717082)
\lineto(270.62602048,930.2998339)
\curveto(270.12824181,930.51138984)(269.65535207,930.68561237)(269.20735127,930.82250151)
\curveto(268.75935047,930.97183511)(268.30512743,931.04650191)(267.84468216,931.04650191)
\curveto(267.02334735,931.04650191)(266.61267995,930.82250151)(266.61267995,930.3745007)
\curveto(266.61267995,930.21272263)(266.66245782,930.06338903)(266.76201355,929.9264999)
\curveto(266.87401375,929.80205523)(267.07934745,929.6651661)(267.37801466,929.5158325)
\curveto(267.68912632,929.3664989)(268.14334936,929.16738743)(268.74068377,928.91849809)
\curveto(269.3255737,928.68205322)(269.82957461,928.43316389)(270.25268648,928.17183009)
\curveto(270.67579835,927.92294075)(270.99935448,927.60560685)(271.22335488,927.21982838)
\curveto(271.45979975,926.83404991)(271.57802218,926.34249347)(271.57802218,925.74515907)
\closepath
}
}
{
\newrgbcolor{curcolor}{0 0 0}
\pscustom[linestyle=none,fillstyle=solid,fillcolor=curcolor]
{
\newpath
\moveto(170.47173862,901.06915611)
\curveto(169.33929214,901.06915611)(168.41217937,901.51093468)(167.6904003,902.39449182)
\curveto(166.98106569,903.29049343)(166.62639839,904.60338467)(166.62639839,906.33316555)
\curveto(166.62639839,908.07539089)(166.98728793,909.39450437)(167.709067,910.29050598)
\curveto(168.43084607,911.18650758)(169.37662554,911.63450839)(170.54640542,911.63450839)
\curveto(171.28062896,911.63450839)(171.88418559,911.49139702)(172.35707533,911.20517428)
\curveto(172.82996507,910.91895155)(173.20329907,910.56428425)(173.47707734,910.14117238)
\lineto(173.57041084,910.14117238)
\curveto(173.53307744,910.34028384)(173.48952181,910.62650658)(173.43974394,910.99984058)
\curveto(173.38996607,911.38561905)(173.36507714,911.77761975)(173.36507714,912.17584269)
\lineto(173.36507714,915.44251521)
\lineto(176.14641546,915.44251521)
\lineto(176.14641546,901.25582311)
\lineto(174.01841164,901.25582311)
\lineto(173.47707734,902.58115882)
\lineto(173.36507714,902.58115882)
\curveto(173.09129887,902.15804695)(172.7241871,901.79715742)(172.26374183,901.49849022)
\curveto(171.80329656,901.21226748)(171.20596216,901.06915611)(170.47173862,901.06915611)
\closepath
\moveto(171.44240703,903.29049343)
\curveto(172.2015195,903.29049343)(172.73663157,903.51449383)(173.04774324,903.96249463)
\curveto(173.3588549,904.4229399)(173.52685521,905.10738557)(173.55174414,906.01583165)
\lineto(173.55174414,906.31449885)
\curveto(173.55174414,907.29761172)(173.39618831,908.05050196)(173.08507664,908.57316956)
\curveto(172.78640943,909.10828164)(172.22640843,909.37583767)(171.40507362,909.37583767)
\curveto(170.79529475,909.37583767)(170.31618278,909.10828164)(169.96773771,908.57316956)
\curveto(169.61929265,908.05050196)(169.44507011,907.29138949)(169.44507011,906.29583215)
\curveto(169.44507011,905.30027481)(169.61929265,904.54738457)(169.96773771,904.03716143)
\curveto(170.31618278,903.53938276)(170.80773922,903.29049343)(171.44240703,903.29049343)
\closepath
}
}
{
\newrgbcolor{curcolor}{0 0 0}
\pscustom[linestyle=none,fillstyle=solid,fillcolor=curcolor]
{
\newpath
\moveto(181.83974814,915.44251521)
\lineto(181.83974814,912.13850929)
\curveto(181.83974814,911.75273082)(181.82730367,911.37317458)(181.80241474,910.99984058)
\curveto(181.77752581,910.62650658)(181.75263687,910.33406161)(181.72774794,910.12250568)
\lineto(181.83974814,910.12250568)
\curveto(182.11352641,910.54561754)(182.48063818,910.90028485)(182.94108345,911.18650758)
\curveto(183.40152872,911.48517478)(183.99886312,911.63450839)(184.73308666,911.63450839)
\curveto(185.8779776,911.63450839)(186.80509038,911.18650758)(187.51442498,910.29050598)
\curveto(188.22375959,909.40694884)(188.57842689,908.10027983)(188.57842689,906.37049895)
\curveto(188.57842689,904.6282736)(188.21753735,903.30916013)(187.49575828,902.41315852)
\curveto(186.77397921,901.51715692)(185.82819974,901.06915611)(184.65841986,901.06915611)
\curveto(183.91175186,901.06915611)(183.32063969,901.19982301)(182.88508335,901.46115682)
\curveto(182.46197148,901.73493508)(182.11352641,902.03982452)(181.83974814,902.37582512)
\lineto(181.65308114,902.37582512)
\lineto(181.18641364,901.25582311)
\lineto(179.05840982,901.25582311)
\lineto(179.05840982,915.44251521)
\closepath
\moveto(183.83708506,909.41317107)
\curveto(183.11530598,909.41317107)(182.60508285,909.18294844)(182.30641564,908.72250317)
\curveto(182.00774844,908.27450236)(181.85219261,907.59627892)(181.83974814,906.68783285)
\lineto(181.83974814,906.38916565)
\curveto(181.83974814,905.40605278)(181.98285951,904.6469403)(182.26908224,904.11182823)
\curveto(182.56774945,903.58916063)(183.10286152,903.32782683)(183.87441846,903.32782683)
\curveto(184.44686393,903.32782683)(184.90108696,903.58916063)(185.23708756,904.11182823)
\curveto(185.57308817,904.6469403)(185.74108847,905.41227501)(185.74108847,906.40783235)
\curveto(185.74108847,907.40338969)(185.56686593,908.1500577)(185.21842086,908.64783637)
\curveto(184.88242026,909.1580595)(184.42197499,909.41317107)(183.83708506,909.41317107)
\closepath
}
}
{
\newrgbcolor{curcolor}{0 0 0}
\pscustom[linestyle=none,fillstyle=solid,fillcolor=curcolor]
{
\newpath
\moveto(194.98109418,914.58384701)
\curveto(196.79798632,914.58384701)(198.1357665,914.25406864)(198.9944347,913.5945119)
\curveto(199.86554738,912.93495516)(200.30110371,911.93317559)(200.30110371,910.58917318)
\curveto(200.30110371,909.97939431)(200.18288128,909.44428224)(199.94643641,908.98383697)
\curveto(199.72243601,908.53583616)(199.41754657,908.1500577)(199.0317681,907.82650156)
\curveto(198.6584341,907.51538989)(198.25398893,907.26027832)(197.8184326,907.06116685)
\lineto(201.73843962,901.25582311)
\lineto(198.602434,901.25582311)
\lineto(195.42909498,906.37049895)
\lineto(193.91709227,906.37049895)
\lineto(193.91709227,901.25582311)
\lineto(191.09842055,901.25582311)
\lineto(191.09842055,914.58384701)
\closepath
\moveto(194.77576048,912.26917619)
\lineto(193.91709227,912.26917619)
\lineto(193.91709227,908.66650307)
\lineto(194.83176058,908.66650307)
\curveto(195.76509558,908.66650307)(196.43087455,908.8220589)(196.82909749,909.13317057)
\curveto(197.23976489,909.44428224)(197.44509859,909.90472751)(197.44509859,910.51450638)
\curveto(197.44509859,911.14917418)(197.22732043,911.59717499)(196.79176409,911.85850879)
\curveto(196.35620775,912.13228706)(195.68420655,912.26917619)(194.77576048,912.26917619)
\closepath
}
}
{
\newrgbcolor{curcolor}{0 0 0}
\pscustom[linestyle=none,fillstyle=solid,fillcolor=curcolor]
{
\newpath
\moveto(211.57578903,901.25582311)
\lineto(210.60512062,904.42916214)
\lineto(205.75177859,904.42916214)
\lineto(204.78111018,901.25582311)
\lineto(201.73843806,901.25582311)
\lineto(206.44244649,914.63984711)
\lineto(209.89578602,914.63984711)
\lineto(214.61846115,901.25582311)
\closepath
\moveto(209.93311942,906.79983305)
\lineto(208.96245101,909.89850527)
\curveto(208.90022868,910.11006121)(208.81933964,910.37761724)(208.71978391,910.70117338)
\curveto(208.62022817,911.02472951)(208.52067244,911.35450788)(208.42111671,911.69050849)
\curveto(208.32156097,912.02650909)(208.24067194,912.31895406)(208.1784496,912.56784339)
\curveto(208.11622727,912.31895406)(208.029116,912.00784239)(207.9171158,911.63450839)
\curveto(207.81756007,911.27361885)(207.71800433,910.92517378)(207.6184486,910.58917318)
\curveto(207.53133733,910.26561704)(207.46289277,910.03539441)(207.4131149,909.89850527)
\lineto(206.46111319,906.79983305)
\closepath
}
}
{
\newrgbcolor{curcolor}{0 0 0}
\pscustom[linestyle=none,fillstyle=solid,fillcolor=curcolor]
{
\newpath
\moveto(221.93581593,901.25582311)
\lineto(218.72514351,911.70917519)
\lineto(218.65047671,911.70917519)
\lineto(218.70647681,910.58917318)
\curveto(218.73136574,910.09139451)(218.75625468,909.55628244)(218.78114361,908.98383697)
\curveto(218.80603254,908.4113915)(218.81847701,907.90116836)(218.81847701,907.45316756)
\lineto(218.81847701,901.25582311)
\lineto(216.29847249,901.25582311)
\lineto(216.29847249,914.58384701)
\lineto(220.14381272,914.58384701)
\lineto(223.29848504,904.39182874)
\lineto(223.35448514,904.39182874)
\lineto(226.69582447,914.58384701)
\lineto(230.54116469,914.58384701)
\lineto(230.54116469,901.25582311)
\lineto(227.90915997,901.25582311)
\lineto(227.90915997,907.56516776)
\curveto(227.90915997,907.98827963)(227.91538221,908.47361383)(227.92782667,909.02117037)
\curveto(227.95271561,909.5687269)(227.97138231,910.08517228)(227.98382677,910.57050648)
\lineto(228.03982687,911.69050849)
\lineto(227.96516007,911.69050849)
\lineto(224.53048725,901.25582311)
\closepath
}
}
{
\newrgbcolor{curcolor}{0 0 0}
\pscustom[linestyle=none,fillstyle=solid,fillcolor=curcolor]
{
\newpath
\moveto(233.28518189,902.56249212)
\curveto(233.28518189,903.13493759)(233.44073773,903.53316053)(233.7518494,903.75716093)
\curveto(234.06296106,903.9936058)(234.4425173,904.11182823)(234.8905181,904.11182823)
\curveto(235.32607444,904.11182823)(235.69940844,903.9936058)(236.01052011,903.75716093)
\curveto(236.32163178,903.53316053)(236.47718761,903.13493759)(236.47718761,902.56249212)
\curveto(236.47718761,902.01493559)(236.32163178,901.61671265)(236.01052011,901.36782332)
\curveto(235.69940844,901.13137845)(235.32607444,901.01315601)(234.8905181,901.01315601)
\curveto(234.4425173,901.01315601)(234.06296106,901.13137845)(233.7518494,901.36782332)
\curveto(233.44073773,901.61671265)(233.28518189,902.01493559)(233.28518189,902.56249212)
\closepath
}
}
{
\newrgbcolor{curcolor}{0 0 0}
\pscustom[linestyle=none,fillstyle=solid,fillcolor=curcolor]
{
\newpath
\moveto(238.88518965,914.0798461)
\curveto(238.88518965,914.60251371)(239.02830102,914.95718101)(239.31452375,915.14384801)
\curveto(239.61319096,915.34295948)(239.97408049,915.44251521)(240.39719236,915.44251521)
\curveto(240.80785976,915.44251521)(241.16252707,915.34295948)(241.46119427,915.14384801)
\curveto(241.75986147,914.95718101)(241.90919507,914.60251371)(241.90919507,914.0798461)
\curveto(241.90919507,913.56962297)(241.75986147,913.21495566)(241.46119427,913.0158442)
\curveto(241.16252707,912.81673273)(240.80785976,912.71717699)(240.39719236,912.71717699)
\curveto(239.97408049,912.71717699)(239.61319096,912.81673273)(239.31452375,913.0158442)
\curveto(239.02830102,913.21495566)(238.88518965,913.56962297)(238.88518965,914.0798461)
\closepath
\moveto(238.17585505,896.77581508)
\curveto(237.85229891,896.77581508)(237.52252054,896.80070402)(237.18651994,896.85048188)
\curveto(236.85051934,896.88781528)(236.57051884,896.93759315)(236.34651843,896.99981549)
\lineto(236.34651843,899.1838194)
\curveto(236.57051884,899.12159707)(236.78207477,899.07804143)(236.98118624,899.0531525)
\curveto(237.18029771,899.02826357)(237.40429811,899.0158191)(237.65318744,899.0158191)
\curveto(238.02652145,899.0158191)(238.34385535,899.12159707)(238.60518915,899.333153)
\curveto(238.86652295,899.54470894)(238.99718985,899.95537634)(238.99718985,900.56515521)
\lineto(238.99718985,911.44784138)
\lineto(241.77852817,911.44784138)
\lineto(241.77852817,900.15448781)
\curveto(241.77852817,899.53226447)(241.66030574,898.96604123)(241.42386087,898.4558181)
\curveto(241.187416,897.94559496)(240.80163753,897.54114979)(240.26652546,897.24248259)
\curveto(239.74385786,896.93137092)(239.04696772,896.77581508)(238.17585505,896.77581508)
\closepath
}
}
{
\newrgbcolor{curcolor}{0 0 0}
\pscustom[linestyle=none,fillstyle=solid,fillcolor=curcolor]
{
\newpath
\moveto(251.80255427,904.27982854)
\curveto(251.80255427,903.24693779)(251.43544251,902.45049192)(250.70121897,901.89049092)
\curveto(249.9794399,901.34293438)(248.89677129,901.06915611)(247.45321314,901.06915611)
\curveto(246.74387854,901.06915611)(246.13409967,901.11893398)(245.62387653,901.21848971)
\curveto(245.1136534,901.30560098)(244.60343026,901.45493458)(244.09320712,901.66649052)
\lineto(244.09320712,903.96249463)
\curveto(244.64076366,903.7136053)(245.23187583,903.5082716)(245.86654363,903.34649353)
\curveto(246.50121144,903.18471546)(247.06121244,903.10382643)(247.54654665,903.10382643)
\curveto(248.08165872,903.10382643)(248.46743719,903.18471546)(248.70388205,903.34649353)
\curveto(248.94032692,903.5082716)(249.05854936,903.71982753)(249.05854936,903.98116133)
\curveto(249.05854936,904.15538387)(249.00877149,904.3109397)(248.90921575,904.44782884)
\curveto(248.82210449,904.58471797)(248.62299302,904.74027381)(248.31188135,904.91449634)
\curveto(248.00076968,905.08871887)(247.51543548,905.31271928)(246.85587874,905.58649754)
\curveto(246.20876647,905.86027581)(245.67987663,906.12783185)(245.26920923,906.38916565)
\curveto(244.87098629,906.66294392)(244.57231909,906.98650005)(244.37320762,907.35983406)
\curveto(244.17409616,907.74561253)(244.07454042,908.2247245)(244.07454042,908.79716997)
\curveto(244.07454042,909.74294944)(244.44165219,910.45228404)(245.17587573,910.92517378)
\curveto(245.91009927,911.39806352)(246.88698991,911.63450839)(248.10654765,911.63450839)
\curveto(248.74121545,911.63450839)(249.34477209,911.57228605)(249.91721756,911.44784138)
\curveto(250.48966303,911.32339672)(251.0807752,911.11806302)(251.69055407,910.83184028)
\lineto(250.85055257,908.83450337)
\curveto(250.3527739,909.0460593)(249.87988416,909.22028184)(249.43188336,909.35717097)
\curveto(248.98388256,909.50650457)(248.52965952,909.58117137)(248.06921425,909.58117137)
\curveto(247.24787944,909.58117137)(246.83721204,909.35717097)(246.83721204,908.90917017)
\curveto(246.83721204,908.7473921)(246.88698991,908.5980585)(246.98654564,908.46116936)
\curveto(247.09854584,908.3367247)(247.30387954,908.19983556)(247.60254675,908.05050196)
\curveto(247.91365841,907.90116836)(248.36788145,907.70205689)(248.96521586,907.45316756)
\curveto(249.55010579,907.21672269)(250.0541067,906.96783335)(250.47721857,906.70649955)
\curveto(250.90033044,906.45761022)(251.22388657,906.14027631)(251.44788697,905.75449785)
\curveto(251.68433184,905.36871938)(251.80255427,904.87716294)(251.80255427,904.27982854)
\closepath
}
}
{
\newrgbcolor{curcolor}{0 0 0}
\pscustom[linestyle=none,fillstyle=solid,fillcolor=curcolor]
{
\newpath
\moveto(179.62362825,890.94684361)
\curveto(180.78096366,890.94684361)(181.65207633,890.64817641)(182.23696627,890.05084201)
\curveto(182.83430067,889.46595207)(183.13296787,888.5201726)(183.13296787,887.21350359)
\lineto(183.13296787,880.56815834)
\lineto(180.35162956,880.56815834)
\lineto(180.35162956,886.52283568)
\curveto(180.35162956,887.99128276)(179.84140642,888.7255063)(178.82096015,888.7255063)
\curveto(178.08673661,888.7255063)(177.564069,888.4641725)(177.25295733,887.94150489)
\curveto(176.94184567,887.41883729)(176.78628983,886.66594705)(176.78628983,885.68283418)
\lineto(176.78628983,880.56815834)
\lineto(174.00495151,880.56815834)
\lineto(174.00495151,886.52283568)
\curveto(174.00495151,887.99128276)(173.49472838,888.7255063)(172.4742821,888.7255063)
\curveto(171.70272516,888.7255063)(171.16761309,888.43306133)(170.86894589,887.84817139)
\curveto(170.58272316,887.27572592)(170.43961179,886.44816888)(170.43961179,885.36550028)
\lineto(170.43961179,880.56815834)
\lineto(167.65827347,880.56815834)
\lineto(167.65827347,890.76017661)
\lineto(169.78627728,890.76017661)
\lineto(170.15961129,889.4535076)
\lineto(170.30894489,889.4535076)
\curveto(170.62005656,889.97617521)(171.04316843,890.35573144)(171.5782805,890.59217631)
\curveto(172.12583703,890.82862118)(172.69206027,890.94684361)(173.27695021,890.94684361)
\curveto(174.02361821,890.94684361)(174.65206378,890.82239895)(175.16228692,890.57350961)
\curveto(175.68495452,890.33706474)(176.08939969,889.96373074)(176.37562243,889.4535076)
\lineto(176.61828953,889.4535076)
\curveto(176.9294012,889.97617521)(177.3587353,890.35573144)(177.90629184,890.59217631)
\curveto(178.46629284,890.82862118)(179.03873831,890.94684361)(179.62362825,890.94684361)
\closepath
}
}
{
\newrgbcolor{curcolor}{0 0 0}
\pscustom[linestyle=none,fillstyle=solid,fillcolor=curcolor]
{
\newpath
\moveto(190.17031971,890.96551031)
\curveto(191.53921105,890.96551031)(192.58454626,890.66684311)(193.30632533,890.06950871)
\curveto(194.04054887,889.48461877)(194.40766064,888.58239493)(194.40766064,887.36283719)
\lineto(194.40766064,880.56815834)
\lineto(192.46632382,880.56815834)
\lineto(191.92498952,881.94949415)
\lineto(191.85032272,881.94949415)
\curveto(191.41476638,881.40193762)(190.95432111,881.00371468)(190.46898691,880.75482534)
\curveto(189.98365271,880.50593601)(189.31787374,880.38149134)(188.47165,880.38149134)
\curveto(187.56320392,880.38149134)(186.81031369,880.64282514)(186.21297928,881.16549275)
\curveto(185.61564488,881.68816035)(185.31697768,882.50327292)(185.31697768,883.61083046)
\curveto(185.31697768,884.69349907)(185.69653391,885.48994494)(186.45564638,886.00016808)
\curveto(187.21475886,886.51039122)(188.35342756,886.79661395)(189.87165251,886.85883629)
\lineto(191.64498902,886.91483639)
\lineto(191.64498902,887.36283719)
\curveto(191.64498902,887.89794926)(191.50187765,888.28994996)(191.21565492,888.5388393)
\curveto(190.94187665,888.78772863)(190.55609818,888.9121733)(190.05831951,888.9121733)
\curveto(189.56054084,888.9121733)(189.07520663,888.8375065)(188.6023169,888.6881729)
\curveto(188.12942716,888.55128376)(187.65653743,888.37706123)(187.18364769,888.1655053)
\lineto(186.26897938,890.05084201)
\curveto(186.80409145,890.32462028)(187.40764809,890.54239844)(188.07964929,890.70417651)
\curveto(188.7516505,890.87839905)(189.44854064,890.96551031)(190.17031971,890.96551031)
\closepath
\moveto(191.64498902,885.29083348)
\lineto(190.56232041,885.25350008)
\curveto(189.66631881,885.22861114)(189.04409547,885.06683307)(188.6956504,884.76816587)
\curveto(188.34720533,884.46949867)(188.1729828,884.07749797)(188.1729828,883.59216376)
\curveto(188.1729828,883.16905189)(188.29742746,882.86416246)(188.5463168,882.67749546)
\curveto(188.79520613,882.50327292)(189.11876227,882.41616166)(189.5169852,882.41616166)
\curveto(190.11431961,882.41616166)(190.61832051,882.59038419)(191.02898791,882.93882926)
\curveto(191.43965532,883.2997188)(191.64498902,883.8037197)(191.64498902,884.45083197)
\closepath
}
}
{
\newrgbcolor{curcolor}{0 0 0}
\pscustom[linestyle=none,fillstyle=solid,fillcolor=curcolor]
{
\newpath
\moveto(198.66366997,894.75485044)
\curveto(199.07433737,894.75485044)(199.42900467,894.65529471)(199.72767188,894.45618324)
\curveto(200.02633908,894.26951624)(200.17567268,893.91484893)(200.17567268,893.39218133)
\curveto(200.17567268,892.88195819)(200.02633908,892.52729089)(199.72767188,892.32817942)
\curveto(199.42900467,892.12906796)(199.07433737,892.02951222)(198.66366997,892.02951222)
\curveto(198.2405581,892.02951222)(197.87966856,892.12906796)(197.58100136,892.32817942)
\curveto(197.29477863,892.52729089)(197.15166726,892.88195819)(197.15166726,893.39218133)
\curveto(197.15166726,893.91484893)(197.29477863,894.26951624)(197.58100136,894.45618324)
\curveto(197.87966856,894.65529471)(198.2405581,894.75485044)(198.66366997,894.75485044)
\closepath
\moveto(200.04500578,890.76017661)
\lineto(200.04500578,880.56815834)
\lineto(197.26366746,880.56815834)
\lineto(197.26366746,890.76017661)
\closepath
}
}
{
\newrgbcolor{curcolor}{0 0 0}
\pscustom[linestyle=none,fillstyle=solid,fillcolor=curcolor]
{
\newpath
\moveto(208.74369427,890.94684361)
\curveto(209.83880734,890.94684361)(210.71614225,890.64817641)(211.37569898,890.05084201)
\curveto(212.03525572,889.46595207)(212.36503409,888.5201726)(212.36503409,887.21350359)
\lineto(212.36503409,880.56815834)
\lineto(209.58369577,880.56815834)
\lineto(209.58369577,886.52283568)
\curveto(209.58369577,887.25705922)(209.45302887,887.80461576)(209.19169507,888.1655053)
\curveto(208.93036127,888.5388393)(208.51347163,888.7255063)(207.94102616,888.7255063)
\curveto(207.09480242,888.7255063)(206.51613472,888.43306133)(206.20502305,887.84817139)
\curveto(205.89391138,887.27572592)(205.73835554,886.44816888)(205.73835554,885.36550028)
\lineto(205.73835554,880.56815834)
\lineto(202.95701723,880.56815834)
\lineto(202.95701723,890.76017661)
\lineto(205.08502104,890.76017661)
\lineto(205.45835504,889.4535076)
\lineto(205.60768864,889.4535076)
\curveto(205.93124478,889.97617521)(206.37302335,890.35573144)(206.93302435,890.59217631)
\curveto(207.50546982,890.82862118)(208.10902646,890.94684361)(208.74369427,890.94684361)
\closepath
}
}
{
\newrgbcolor{curcolor}{0 0 0}
\pscustom[linestyle=none,fillstyle=solid,fillcolor=curcolor]
{
\newpath
\moveto(214.82903157,881.87482735)
\curveto(214.82903157,882.44727282)(214.9845874,882.84549576)(215.29569907,883.06949616)
\curveto(215.60681074,883.30594103)(215.98636697,883.42416346)(216.43436778,883.42416346)
\curveto(216.86992411,883.42416346)(217.24325812,883.30594103)(217.55436978,883.06949616)
\curveto(217.86548145,882.84549576)(218.02103729,882.44727282)(218.02103729,881.87482735)
\curveto(218.02103729,881.32727081)(217.86548145,880.92904788)(217.55436978,880.68015854)
\curveto(217.24325812,880.44371368)(216.86992411,880.32549124)(216.43436778,880.32549124)
\curveto(215.98636697,880.32549124)(215.60681074,880.44371368)(215.29569907,880.68015854)
\curveto(214.9845874,880.92904788)(214.82903157,881.32727081)(214.82903157,881.87482735)
\closepath
}
}
{
\newrgbcolor{curcolor}{0 0 0}
\pscustom[linestyle=none,fillstyle=solid,fillcolor=curcolor]
{
\newpath
\moveto(220.42903933,893.39218133)
\curveto(220.42903933,893.91484893)(220.57215069,894.26951624)(220.85837343,894.45618324)
\curveto(221.15704063,894.65529471)(221.51793017,894.75485044)(221.94104204,894.75485044)
\curveto(222.35170944,894.75485044)(222.70637674,894.65529471)(223.00504394,894.45618324)
\curveto(223.30371115,894.26951624)(223.45304475,893.91484893)(223.45304475,893.39218133)
\curveto(223.45304475,892.88195819)(223.30371115,892.52729089)(223.00504394,892.32817942)
\curveto(222.70637674,892.12906796)(222.35170944,892.02951222)(221.94104204,892.02951222)
\curveto(221.51793017,892.02951222)(221.15704063,892.12906796)(220.85837343,892.32817942)
\curveto(220.57215069,892.52729089)(220.42903933,892.88195819)(220.42903933,893.39218133)
\closepath
\moveto(219.71970472,876.08815031)
\curveto(219.39614858,876.08815031)(219.06637022,876.11303925)(218.73036961,876.16281711)
\curveto(218.39436901,876.20015051)(218.11436851,876.24992838)(217.89036811,876.31215071)
\lineto(217.89036811,878.49615463)
\curveto(218.11436851,878.4339323)(218.32592444,878.39037666)(218.52503591,878.36548773)
\curveto(218.72414738,878.34059879)(218.94814778,878.32815433)(219.19703712,878.32815433)
\curveto(219.57037112,878.32815433)(219.88770502,878.4339323)(220.14903882,878.64548823)
\curveto(220.41037263,878.85704416)(220.54103953,879.26771157)(220.54103953,879.87749044)
\lineto(220.54103953,890.76017661)
\lineto(223.32237785,890.76017661)
\lineto(223.32237785,879.46682304)
\curveto(223.32237785,878.8445997)(223.20415541,878.27837646)(222.96771054,877.76815332)
\curveto(222.73126567,877.25793019)(222.34548721,876.85348502)(221.81037513,876.55481782)
\curveto(221.28770753,876.24370615)(220.59081739,876.08815031)(219.71970472,876.08815031)
\closepath
}
}
{
\newrgbcolor{curcolor}{0 0 0}
\pscustom[linestyle=none,fillstyle=solid,fillcolor=curcolor]
{
\newpath
\moveto(233.34640013,883.59216376)
\curveto(233.34640013,882.55927302)(232.97928836,881.76282715)(232.24506483,881.20282615)
\curveto(231.52328575,880.65526961)(230.44061715,880.38149134)(228.997059,880.38149134)
\curveto(228.2877244,880.38149134)(227.67794553,880.43126921)(227.16772239,880.53082494)
\curveto(226.65749925,880.61793621)(226.14727612,880.76726981)(225.63705298,880.97882575)
\lineto(225.63705298,883.27482986)
\curveto(226.18460952,883.02594053)(226.77572169,882.82060683)(227.41038949,882.65882876)
\curveto(228.0450573,882.49705069)(228.6050583,882.41616166)(229.0903925,882.41616166)
\curveto(229.62550458,882.41616166)(230.01128304,882.49705069)(230.24772791,882.65882876)
\curveto(230.48417278,882.82060683)(230.60239522,883.03216276)(230.60239522,883.29349656)
\curveto(230.60239522,883.4677191)(230.55261735,883.62327493)(230.45306161,883.76016406)
\curveto(230.36595035,883.8970532)(230.16683888,884.05260903)(229.85572721,884.22683157)
\curveto(229.54461554,884.4010541)(229.05928134,884.6250545)(228.3997246,884.89883277)
\curveto(227.75261233,885.17261104)(227.22372249,885.44016708)(226.81305509,885.70150088)
\curveto(226.41483215,885.97527915)(226.11616495,886.29883528)(225.91705348,886.67216928)
\curveto(225.71794202,887.05794775)(225.61838628,887.53705972)(225.61838628,888.10950519)
\curveto(225.61838628,889.05528467)(225.98549805,889.76461927)(226.71972159,890.23750901)
\curveto(227.45394513,890.71039875)(228.43083577,890.94684361)(229.65039351,890.94684361)
\curveto(230.28506131,890.94684361)(230.88861795,890.88462128)(231.46106342,890.76017661)
\curveto(232.03350889,890.63573195)(232.62462106,890.43039824)(233.23439993,890.14417551)
\lineto(232.39439843,888.14683859)
\curveto(231.89661976,888.35839453)(231.42373002,888.53261706)(230.97572922,888.6695062)
\curveto(230.52772841,888.8188398)(230.07350538,888.8935066)(229.61306011,888.8935066)
\curveto(228.7917253,888.8935066)(228.3810579,888.6695062)(228.3810579,888.2215054)
\curveto(228.3810579,888.05972733)(228.43083577,887.91039373)(228.5303915,887.77350459)
\curveto(228.6423917,887.64905992)(228.8477254,887.51217079)(229.14639261,887.36283719)
\curveto(229.45750427,887.21350359)(229.91172731,887.01439212)(230.50906171,886.76550279)
\curveto(231.09395165,886.52905792)(231.59795256,886.28016858)(232.02106442,886.01883478)
\curveto(232.44417629,885.76994545)(232.76773243,885.45261154)(232.99173283,885.06683307)
\curveto(233.2281777,884.6810546)(233.34640013,884.18949817)(233.34640013,883.59216376)
\closepath
}
}
{
\newrgbcolor{curcolor}{0 0 0}
\pscustom[linestyle=none,fillstyle=solid,fillcolor=curcolor]
{
\newpath
\moveto(172.8635376,869.38525737)
\curveto(173.00042673,869.38525737)(173.1622048,869.37903514)(173.3488718,869.36659067)
\curveto(173.5355388,869.35414621)(173.6848724,869.33547951)(173.7968726,869.31059057)
\lineto(173.5915389,866.69725255)
\curveto(173.49198317,866.72214149)(173.36131627,866.74080819)(173.1995382,866.75325266)
\curveto(173.03776013,866.77814159)(172.89464876,866.79058606)(172.7702041,866.79058606)
\curveto(172.29731436,866.79058606)(171.84309132,866.70347479)(171.40753499,866.52925225)
\curveto(170.97197865,866.36747419)(170.61731135,866.09991815)(170.34353308,865.72658415)
\curveto(170.08219928,865.35325015)(169.95153238,864.84302701)(169.95153238,864.19591474)
\lineto(169.95153238,859.0065721)
\lineto(167.17019406,859.0065721)
\lineto(167.17019406,869.19859037)
\lineto(169.27953117,869.19859037)
\lineto(169.69019858,867.48125396)
\lineto(169.82086548,867.48125396)
\curveto(170.11953268,868.00392156)(170.53020008,868.45192237)(171.05286768,868.82525637)
\curveto(171.57553529,869.19859037)(172.17909193,869.38525737)(172.8635376,869.38525737)
\closepath
}
}
{
\newrgbcolor{curcolor}{0 0 0}
\pscustom[linestyle=none,fillstyle=solid,fillcolor=curcolor]
{
\newpath
\moveto(179.45288271,869.40392407)
\curveto(180.82177405,869.40392407)(181.86710926,869.10525687)(182.58888833,868.50792247)
\curveto(183.32311187,867.92303253)(183.69022364,867.02080869)(183.69022364,865.80125095)
\lineto(183.69022364,859.0065721)
\lineto(181.74888682,859.0065721)
\lineto(181.20755252,860.38790791)
\lineto(181.13288572,860.38790791)
\curveto(180.69732938,859.84035137)(180.23688411,859.44212844)(179.75154991,859.1932391)
\curveto(179.26621571,858.94434977)(178.60043674,858.8199051)(177.754213,858.8199051)
\curveto(176.84576692,858.8199051)(176.09287669,859.0812389)(175.49554228,859.60390651)
\curveto(174.89820788,860.12657411)(174.59954067,860.94168668)(174.59954067,862.04924422)
\curveto(174.59954067,863.13191283)(174.97909691,863.9283587)(175.73820938,864.43858184)
\curveto(176.49732185,864.94880498)(177.63599056,865.23502771)(179.15421551,865.29725005)
\lineto(180.92755202,865.35325015)
\lineto(180.92755202,865.80125095)
\curveto(180.92755202,866.33636302)(180.78444065,866.72836372)(180.49821792,866.97725306)
\curveto(180.22443965,867.22614239)(179.83866118,867.35058706)(179.34088251,867.35058706)
\curveto(178.84310384,867.35058706)(178.35776963,867.27592026)(177.8848799,867.12658666)
\curveto(177.41199016,866.98969752)(176.93910042,866.81547499)(176.46621069,866.60391905)
\lineto(175.55154238,868.48925577)
\curveto(176.08665445,868.76303404)(176.69021109,868.9808122)(177.36221229,869.14259027)
\curveto(178.0342135,869.31681281)(178.73110364,869.40392407)(179.45288271,869.40392407)
\closepath
\moveto(180.92755202,863.72924723)
\lineto(179.84488341,863.69191383)
\curveto(178.9488818,863.6670249)(178.32665847,863.50524683)(177.9782134,863.20657963)
\curveto(177.62976833,862.90791243)(177.45554579,862.51591173)(177.45554579,862.03057752)
\curveto(177.45554579,861.60746565)(177.57999046,861.30257622)(177.8288798,861.11590922)
\curveto(178.07776913,860.94168668)(178.40132527,860.85457541)(178.7995482,860.85457541)
\curveto(179.39688261,860.85457541)(179.90088351,861.02879795)(180.31155091,861.37724302)
\curveto(180.72221832,861.73813255)(180.92755202,862.24213346)(180.92755202,862.88924573)
\closepath
}
}
{
\newrgbcolor{curcolor}{0 0 0}
\pscustom[linestyle=none,fillstyle=solid,fillcolor=curcolor]
{
\newpath
\moveto(192.33290845,869.38525737)
\curveto(193.42802153,869.38525737)(194.30535643,869.08659017)(194.96491317,868.48925577)
\curveto(195.62446991,867.90436583)(195.95424828,866.95858636)(195.95424828,865.65191735)
\lineto(195.95424828,859.0065721)
\lineto(193.17290996,859.0065721)
\lineto(193.17290996,864.96124944)
\curveto(193.17290996,865.69547298)(193.04224306,866.24302952)(192.78090925,866.60391905)
\curveto(192.51957545,866.97725306)(192.10268582,867.16392006)(191.53024035,867.16392006)
\curveto(190.68401661,867.16392006)(190.1053489,866.87147509)(189.79423723,866.28658515)
\curveto(189.48312557,865.71413968)(189.32756973,864.88658264)(189.32756973,863.80391403)
\lineto(189.32756973,859.0065721)
\lineto(186.54623141,859.0065721)
\lineto(186.54623141,869.19859037)
\lineto(188.67423523,869.19859037)
\lineto(189.04756923,867.89192136)
\lineto(189.19690283,867.89192136)
\curveto(189.52045897,868.41458897)(189.96223754,868.7941452)(190.52223854,869.03059007)
\curveto(191.09468401,869.26703494)(191.69824065,869.38525737)(192.33290845,869.38525737)
\closepath
}
}
{
\newrgbcolor{curcolor}{0 0 0}
\pscustom[linestyle=none,fillstyle=solid,fillcolor=curcolor]
{
\newpath
\moveto(202.03958367,858.8199051)
\curveto(200.9071372,858.8199051)(199.98002442,859.26168367)(199.25824535,860.14524081)
\curveto(198.54891075,861.04124242)(198.19424344,862.35413366)(198.19424344,864.08391454)
\curveto(198.19424344,865.82613988)(198.55513298,867.14525336)(199.27691205,868.04125496)
\curveto(199.99869112,868.93725657)(200.9444706,869.38525737)(202.11425047,869.38525737)
\curveto(202.84847401,869.38525737)(203.45203065,869.24214601)(203.92492038,868.95592327)
\curveto(204.39781012,868.66970053)(204.77114412,868.31503323)(205.04492239,867.89192136)
\lineto(205.13825589,867.89192136)
\curveto(205.10092249,868.09103283)(205.05736686,868.37725557)(205.00758899,868.75058957)
\curveto(204.95781112,869.13636804)(204.93292219,869.52836874)(204.93292219,869.92659168)
\lineto(204.93292219,873.1932642)
\lineto(207.71426051,873.1932642)
\lineto(207.71426051,859.0065721)
\lineto(205.58625669,859.0065721)
\lineto(205.04492239,860.33190781)
\lineto(204.93292219,860.33190781)
\curveto(204.65914392,859.90879594)(204.29203215,859.54790641)(203.83158688,859.2492392)
\curveto(203.37114161,858.96301647)(202.77380721,858.8199051)(202.03958367,858.8199051)
\closepath
\moveto(203.01025208,861.04124242)
\curveto(203.76936455,861.04124242)(204.30447662,861.26524282)(204.61558829,861.71324362)
\curveto(204.92669996,862.17368889)(205.09470026,862.85813456)(205.11958919,863.76658063)
\lineto(205.11958919,864.06524784)
\curveto(205.11958919,865.04836071)(204.96403336,865.80125095)(204.65292169,866.32391855)
\curveto(204.35425449,866.85903062)(203.79425348,867.12658666)(202.97291868,867.12658666)
\curveto(202.36313981,867.12658666)(201.88402784,866.85903062)(201.53558277,866.32391855)
\curveto(201.1871377,865.80125095)(201.01291516,865.04213848)(201.01291516,864.04658114)
\curveto(201.01291516,863.0510238)(201.1871377,862.29813356)(201.53558277,861.78791042)
\curveto(201.88402784,861.29013175)(202.37558427,861.04124242)(203.01025208,861.04124242)
\closepath
}
}
{
\newrgbcolor{curcolor}{0 0 0}
\pscustom[linestyle=none,fillstyle=solid,fillcolor=curcolor]
{
\newpath
\moveto(219.88493909,864.12124794)
\curveto(219.88493909,862.42880046)(219.43693829,861.12213145)(218.54093668,860.20124091)
\curveto(217.65737954,859.28035037)(216.45026627,858.8199051)(214.91959686,858.8199051)
\curveto(213.97381738,858.8199051)(213.12759364,859.0252388)(212.38092564,859.4359062)
\curveto(211.6467021,859.84657361)(211.0680344,860.44390801)(210.64492253,861.22790942)
\curveto(210.22181066,862.02435529)(210.01025472,862.98880146)(210.01025472,864.12124794)
\curveto(210.01025472,865.81369542)(210.45203329,867.11414219)(211.33559043,868.02258826)
\curveto(212.21914757,868.93103434)(213.43248308,869.38525737)(214.97559696,869.38525737)
\curveto(215.9338209,869.38525737)(216.78004464,869.17992367)(217.51426818,868.76925627)
\curveto(218.24849171,868.35858887)(218.82715942,867.76125446)(219.25027129,866.97725306)
\curveto(219.67338316,866.19325165)(219.88493909,865.24124994)(219.88493909,864.12124794)
\closepath
\moveto(212.84759314,864.12124794)
\curveto(212.84759314,863.11324613)(213.00937121,862.34791142)(213.33292735,861.82524382)
\curveto(213.66892795,861.31502068)(214.21026225,861.05990912)(214.95693026,861.05990912)
\curveto(215.6911538,861.05990912)(216.22004363,861.31502068)(216.54359977,861.82524382)
\curveto(216.87960037,862.34791142)(217.04760067,863.11324613)(217.04760067,864.12124794)
\curveto(217.04760067,865.12924974)(216.87960037,865.88213998)(216.54359977,866.37991865)
\curveto(216.22004363,866.89014179)(215.68493156,867.14525336)(214.93826356,867.14525336)
\curveto(214.20404002,867.14525336)(213.66892795,866.89014179)(213.33292735,866.37991865)
\curveto(213.00937121,865.88213998)(212.84759314,865.12924974)(212.84759314,864.12124794)
\closepath
}
}
{
\newrgbcolor{curcolor}{0 0 0}
\pscustom[linestyle=none,fillstyle=solid,fillcolor=curcolor]
{
\newpath
\moveto(234.14628652,869.38525737)
\curveto(235.30362193,869.38525737)(236.1747346,869.08659017)(236.75962454,868.48925577)
\curveto(237.35695894,867.90436583)(237.65562614,866.95858636)(237.65562614,865.65191735)
\lineto(237.65562614,859.0065721)
\lineto(234.87428783,859.0065721)
\lineto(234.87428783,864.96124944)
\curveto(234.87428783,866.42969652)(234.36406469,867.16392006)(233.34361841,867.16392006)
\curveto(232.60939488,867.16392006)(232.08672727,866.90258626)(231.7756156,866.37991865)
\curveto(231.46450394,865.85725105)(231.3089481,865.10436081)(231.3089481,864.12124794)
\lineto(231.3089481,859.0065721)
\lineto(228.52760978,859.0065721)
\lineto(228.52760978,864.96124944)
\curveto(228.52760978,866.42969652)(228.01738664,867.16392006)(226.99694037,867.16392006)
\curveto(226.22538343,867.16392006)(225.69027136,866.87147509)(225.39160416,866.28658515)
\curveto(225.10538142,865.71413968)(224.96227006,864.88658264)(224.96227006,863.80391403)
\lineto(224.96227006,859.0065721)
\lineto(222.18093174,859.0065721)
\lineto(222.18093174,869.19859037)
\lineto(224.30893555,869.19859037)
\lineto(224.68226956,867.89192136)
\lineto(224.83160316,867.89192136)
\curveto(225.14271483,868.41458897)(225.56582669,868.7941452)(226.10093876,869.03059007)
\curveto(226.6484953,869.26703494)(227.21471854,869.38525737)(227.79960848,869.38525737)
\curveto(228.54627648,869.38525737)(229.17472205,869.26081271)(229.68494519,869.01192337)
\curveto(230.20761279,868.7754785)(230.61205796,868.4021445)(230.8982807,867.89192136)
\lineto(231.1409478,867.89192136)
\curveto(231.45205947,868.41458897)(231.88139357,868.7941452)(232.42895011,869.03059007)
\curveto(232.98895111,869.26703494)(233.56139658,869.38525737)(234.14628652,869.38525737)
\closepath
}
}
{
\newrgbcolor{curcolor}{0 0 0}
\pscustom[linestyle=none,fillstyle=solid,fillcolor=curcolor]
{
\newpath
\moveto(245.71964839,859.0065721)
\lineto(239.65297085,859.0065721)
\lineto(239.65297085,860.61190831)
\lineto(241.27697376,861.35857632)
\lineto(241.27697376,869.98259178)
\lineto(239.65297085,870.72925978)
\lineto(239.65297085,872.33459599)
\lineto(245.71964839,872.33459599)
\lineto(245.71964839,870.72925978)
\lineto(244.09564548,869.98259178)
\lineto(244.09564548,861.35857632)
\lineto(245.71964839,860.61190831)
\closepath
}
}
{
\newrgbcolor{curcolor}{0 0 0}
\pscustom[linestyle=none,fillstyle=solid,fillcolor=curcolor]
{
\newpath
\moveto(259.04767075,865.80125095)
\curveto(259.04767075,863.54880247)(258.40678071,861.85013275)(257.12500064,860.70524181)
\curveto(255.84322056,859.57279534)(254.05743959,859.0065721)(251.7676577,859.0065721)
\lineto(247.99698428,859.0065721)
\lineto(247.99698428,872.33459599)
\lineto(252.17832511,872.33459599)
\curveto(253.57210538,872.33459599)(254.77921866,872.08570666)(255.79966493,871.58792799)
\curveto(256.83255567,871.09014932)(257.62900154,870.35592578)(258.18900255,869.38525737)
\curveto(258.76144802,868.41458897)(259.04767075,867.21992016)(259.04767075,865.80125095)
\closepath
\moveto(256.11699883,865.72658415)
\curveto(256.11699883,867.19503122)(255.7934427,868.27769983)(255.14633043,868.97458997)
\curveto(254.49921815,869.67148011)(253.55966092,870.01992518)(252.32765871,870.01992518)
\lineto(250.815656,870.01992518)
\lineto(250.815656,861.33990962)
\lineto(252.0289915,861.33990962)
\curveto(254.75432972,861.33990962)(256.11699883,862.80213446)(256.11699883,865.72658415)
\closepath
}
}
{
\newrgbcolor{curcolor}{0 0 0}
\pscustom[linestyle=none,fillstyle=solid,fillcolor=curcolor]
{
\newpath
\moveto(260.44764866,860.31324111)
\curveto(260.44764866,860.88568658)(260.60320449,861.28390952)(260.91431616,861.50790992)
\curveto(261.22542783,861.74435479)(261.60498407,861.86257722)(262.05298487,861.86257722)
\curveto(262.4885412,861.86257722)(262.86187521,861.74435479)(263.17298688,861.50790992)
\curveto(263.48409855,861.28390952)(263.63965438,860.88568658)(263.63965438,860.31324111)
\curveto(263.63965438,859.76568457)(263.48409855,859.36746164)(263.17298688,859.1185723)
\curveto(262.86187521,858.88212743)(262.4885412,858.763905)(262.05298487,858.763905)
\curveto(261.60498407,858.763905)(261.22542783,858.88212743)(260.91431616,859.1185723)
\curveto(260.60320449,859.36746164)(260.44764866,859.76568457)(260.44764866,860.31324111)
\closepath
}
}
{
\newrgbcolor{curcolor}{0 0 0}
\pscustom[linestyle=none,fillstyle=solid,fillcolor=curcolor]
{
\newpath
\moveto(266.04765642,871.83059509)
\curveto(266.04765642,872.35326269)(266.19076778,872.70793)(266.47699052,872.894597)
\curveto(266.77565772,873.09370847)(267.13654726,873.1932642)(267.55965913,873.1932642)
\curveto(267.97032653,873.1932642)(268.32499383,873.09370847)(268.62366103,872.894597)
\curveto(268.92232824,872.70793)(269.07166184,872.35326269)(269.07166184,871.83059509)
\curveto(269.07166184,871.32037195)(268.92232824,870.96570465)(268.62366103,870.76659318)
\curveto(268.32499383,870.56748171)(267.97032653,870.46792598)(267.55965913,870.46792598)
\curveto(267.13654726,870.46792598)(266.77565772,870.56748171)(266.47699052,870.76659318)
\curveto(266.19076778,870.96570465)(266.04765642,871.32037195)(266.04765642,871.83059509)
\closepath
\moveto(265.33832181,854.52656407)
\curveto(265.01476568,854.52656407)(264.68498731,854.551453)(264.3489867,854.60123087)
\curveto(264.0129861,854.63856427)(263.7329856,854.68834214)(263.5089852,854.75056447)
\lineto(263.5089852,856.93456839)
\curveto(263.7329856,856.87234605)(263.94454154,856.82879042)(264.143653,856.80390149)
\curveto(264.34276447,856.77901255)(264.56676487,856.76656809)(264.81565421,856.76656809)
\curveto(265.18898821,856.76656809)(265.50632211,856.87234605)(265.76765591,857.08390199)
\curveto(266.02898972,857.29545792)(266.15965662,857.70612533)(266.15965662,858.3159042)
\lineto(266.15965662,869.19859037)
\lineto(268.94099494,869.19859037)
\lineto(268.94099494,857.90523679)
\curveto(268.94099494,857.28301346)(268.8227725,856.71679022)(268.58632763,856.20656708)
\curveto(268.34988277,855.69634395)(267.9641043,855.29189878)(267.42899223,854.99323157)
\curveto(266.90632462,854.68211991)(266.20943448,854.52656407)(265.33832181,854.52656407)
\closepath
}
}
{
\newrgbcolor{curcolor}{0 0 0}
\pscustom[linestyle=none,fillstyle=solid,fillcolor=curcolor]
{
\newpath
\moveto(278.96502104,862.03057752)
\curveto(278.96502104,860.99768678)(278.59790927,860.20124091)(277.86368573,859.64123991)
\curveto(277.14190666,859.09368337)(276.05923805,858.8199051)(274.61567991,858.8199051)
\curveto(273.90634531,858.8199051)(273.29656643,858.86968297)(272.7863433,858.9692387)
\curveto(272.27612016,859.05634997)(271.76589702,859.20568357)(271.25567389,859.4172395)
\lineto(271.25567389,861.71324362)
\curveto(271.80323042,861.46435429)(272.39434259,861.25902058)(273.0290104,861.09724252)
\curveto(273.6636782,860.93546445)(274.22367921,860.85457541)(274.70901341,860.85457541)
\curveto(275.24412548,860.85457541)(275.62990395,860.93546445)(275.86634882,861.09724252)
\curveto(276.10279369,861.25902058)(276.22101612,861.47057652)(276.22101612,861.73191032)
\curveto(276.22101612,861.90613286)(276.17123825,862.06168869)(276.07168252,862.19857782)
\curveto(275.98457125,862.33546696)(275.78545978,862.49102279)(275.47434812,862.66524533)
\curveto(275.16323645,862.83946786)(274.67790224,863.06346826)(274.01834551,863.33724653)
\curveto(273.37123324,863.6110248)(272.8423434,863.87858084)(272.431676,864.13991464)
\curveto(272.03345306,864.41369291)(271.73478586,864.73724904)(271.53567439,865.11058304)
\curveto(271.33656292,865.49636151)(271.23700719,865.97547348)(271.23700719,866.54791895)
\curveto(271.23700719,867.49369843)(271.60411896,868.20303303)(272.33834249,868.67592277)
\curveto(273.07256603,869.1488125)(274.04945667,869.38525737)(275.26901441,869.38525737)
\curveto(275.90368222,869.38525737)(276.50723886,869.32303504)(277.07968433,869.19859037)
\curveto(277.6521298,869.0741457)(278.24324197,868.868812)(278.85302084,868.58258927)
\lineto(278.01301933,866.58525235)
\curveto(277.51524066,866.79680829)(277.04235093,866.97103082)(276.59435012,867.10791996)
\curveto(276.14634932,867.25725356)(275.69212628,867.33192036)(275.23168101,867.33192036)
\curveto(274.41034621,867.33192036)(273.99967881,867.10791996)(273.99967881,866.65991915)
\curveto(273.99967881,866.49814109)(274.04945667,866.34880749)(274.14901241,866.21191835)
\curveto(274.26101261,866.08747368)(274.46634631,865.95058455)(274.76501351,865.80125095)
\curveto(275.07612518,865.65191735)(275.53034822,865.45280588)(276.12768262,865.20391654)
\curveto(276.71257256,864.96747168)(277.21657346,864.71858234)(277.63968533,864.45724854)
\curveto(278.0627972,864.2083592)(278.38635334,863.8910253)(278.61035374,863.50524683)
\curveto(278.84679861,863.11946836)(278.96502104,862.62791193)(278.96502104,862.03057752)
\closepath
}
}
{
\newrgbcolor{curcolor}{0 0 0}
\pscustom[linestyle=none,fillstyle=solid,fillcolor=curcolor]
{
\newpath
\moveto(172.67764991,847.9639372)
\curveto(172.81453905,847.9639372)(172.97631712,847.95771497)(173.16298412,847.9452705)
\curveto(173.34965112,847.93282603)(173.49898472,847.91415933)(173.61098492,847.8892704)
\lineto(173.40565122,845.27593238)
\curveto(173.30609549,845.30082132)(173.17542858,845.31948802)(173.01365052,845.33193248)
\curveto(172.85187245,845.35682142)(172.70876108,845.36926588)(172.58431641,845.36926588)
\curveto(172.11142668,845.36926588)(171.65720364,845.28215462)(171.2216473,845.10793208)
\curveto(170.78609097,844.94615401)(170.43142367,844.67859798)(170.1576454,844.30526398)
\curveto(169.8963116,843.93192997)(169.76564469,843.42170684)(169.76564469,842.77459457)
\lineto(169.76564469,837.58525193)
\lineto(166.98430638,837.58525193)
\lineto(166.98430638,847.7772702)
\lineto(169.09364349,847.7772702)
\lineto(169.50431089,846.05993379)
\lineto(169.63497779,846.05993379)
\curveto(169.933645,846.58260139)(170.3443124,847.03060219)(170.86698,847.4039362)
\curveto(171.38964761,847.7772702)(171.99320424,847.9639372)(172.67764991,847.9639372)
\closepath
}
}
{
\newrgbcolor{curcolor}{0 0 0}
\pscustom[linestyle=none,fillstyle=solid,fillcolor=curcolor]
{
\newpath
\moveto(184.34433746,842.69992776)
\curveto(184.34433746,841.00748029)(183.89633666,839.70081128)(183.00033505,838.77992074)
\curveto(182.11677791,837.8590302)(180.90966464,837.39858493)(179.37899523,837.39858493)
\curveto(178.43321575,837.39858493)(177.58699201,837.60391863)(176.84032401,838.01458603)
\curveto(176.10610047,838.42525344)(175.52743277,839.02258784)(175.1043209,839.80658924)
\curveto(174.68120903,840.60303512)(174.46965309,841.56748129)(174.46965309,842.69992776)
\curveto(174.46965309,844.39237524)(174.91143166,845.69282202)(175.7949888,846.60126809)
\curveto(176.67854594,847.50971416)(177.89188145,847.9639372)(179.43499533,847.9639372)
\curveto(180.39321927,847.9639372)(181.23944301,847.7586035)(181.97366654,847.3479361)
\curveto(182.70789008,846.93726869)(183.28655779,846.33993429)(183.70966966,845.55593288)
\curveto(184.13278153,844.77193148)(184.34433746,843.81992977)(184.34433746,842.69992776)
\closepath
\moveto(177.30699151,842.69992776)
\curveto(177.30699151,841.69192596)(177.46876958,840.92659125)(177.79232572,840.40392365)
\curveto(178.12832632,839.89370051)(178.66966062,839.63858894)(179.41632863,839.63858894)
\curveto(180.15055216,839.63858894)(180.679442,839.89370051)(181.00299814,840.40392365)
\curveto(181.33899874,840.92659125)(181.50699904,841.69192596)(181.50699904,842.69992776)
\curveto(181.50699904,843.70792957)(181.33899874,844.46081981)(181.00299814,844.95859848)
\curveto(180.679442,845.46882162)(180.14432993,845.72393319)(179.39766193,845.72393319)
\curveto(178.66343839,845.72393319)(178.12832632,845.46882162)(177.79232572,844.95859848)
\curveto(177.46876958,844.46081981)(177.30699151,843.70792957)(177.30699151,842.69992776)
\closepath
}
}
{
\newrgbcolor{curcolor}{0 0 0}
\pscustom[linestyle=none,fillstyle=solid,fillcolor=curcolor]
{
\newpath
\moveto(195.99234782,847.7772702)
\lineto(195.99234782,837.58525193)
\lineto(193.86434401,837.58525193)
\lineto(193.49101001,838.89192094)
\lineto(193.34167641,838.89192094)
\curveto(193.01812027,838.36925333)(192.57011947,837.9896971)(191.997674,837.75325223)
\curveto(191.43767299,837.51680736)(190.84033859,837.39858493)(190.20567079,837.39858493)
\curveto(189.11055771,837.39858493)(188.23322281,837.6910299)(187.57366607,838.27591983)
\curveto(186.91410933,838.87325424)(186.58433096,839.82525594)(186.58433096,841.13192495)
\lineto(186.58433096,847.7772702)
\lineto(189.36566928,847.7772702)
\lineto(189.36566928,841.82259286)
\curveto(189.36566928,841.08836932)(189.49633618,840.53459055)(189.75766998,840.16125655)
\curveto(190.01900378,839.80036701)(190.43589342,839.61992224)(191.00833889,839.61992224)
\curveto(191.85456263,839.61992224)(192.43323033,839.90614498)(192.744342,840.47859045)
\curveto(193.05545367,841.06348039)(193.21100951,841.89725966)(193.21100951,842.97992827)
\lineto(193.21100951,847.7772702)
\closepath
}
}
{
\newrgbcolor{curcolor}{0 0 0}
\pscustom[linestyle=none,fillstyle=solid,fillcolor=curcolor]
{
\newpath
\moveto(203.19768713,839.61992224)
\curveto(203.5087988,839.61992224)(203.807466,839.64481118)(204.09368874,839.69458904)
\curveto(204.37991147,839.75681138)(204.66613421,839.83770041)(204.95235695,839.93725615)
\lineto(204.95235695,837.86525243)
\curveto(204.65368974,837.7283633)(204.28035574,837.6163631)(203.83235494,837.52925183)
\curveto(203.3967986,837.44214056)(202.91768663,837.39858493)(202.39501903,837.39858493)
\curveto(201.78524016,837.39858493)(201.23768362,837.49814066)(200.75234942,837.69725213)
\curveto(200.27945968,837.8963636)(199.89990344,838.23858643)(199.61368071,838.72392064)
\curveto(199.33990244,839.20925484)(199.20301331,839.89370051)(199.20301331,840.77725765)
\lineto(199.20301331,845.68659979)
\lineto(197.8776776,845.68659979)
\lineto(197.8776776,846.86260189)
\lineto(199.40834701,847.7959369)
\lineto(200.21101511,849.94260741)
\lineto(201.98435163,849.94260741)
\lineto(201.98435163,847.7772702)
\lineto(204.84035674,847.7772702)
\lineto(204.84035674,845.68659979)
\lineto(201.98435163,845.68659979)
\lineto(201.98435163,840.77725765)
\curveto(201.98435163,840.39147918)(202.09635183,840.09903421)(202.32035223,839.89992275)
\curveto(202.54435263,839.71325574)(202.8367976,839.61992224)(203.19768713,839.61992224)
\closepath
}
}
{
\newrgbcolor{curcolor}{0 0 0}
\pscustom[linestyle=none,fillstyle=solid,fillcolor=curcolor]
{
\newpath
\moveto(211.20570241,847.9639372)
\curveto(212.61192715,847.9639372)(213.72570692,847.55949203)(214.54704173,846.75060169)
\curveto(215.36837654,845.95415582)(215.77904394,844.81548711)(215.77904394,843.33459557)
\lineto(215.77904394,841.99059316)
\lineto(209.20836549,841.99059316)
\curveto(209.23325443,841.20659175)(209.46347706,840.59059065)(209.8990334,840.14258985)
\curveto(210.3470342,839.69458904)(210.96303531,839.47058864)(211.74703671,839.47058864)
\curveto(212.39414898,839.47058864)(212.98526115,839.53281098)(213.52037322,839.65725564)
\curveto(214.06792976,839.79414478)(214.62793076,839.99947848)(215.20037624,840.27325675)
\lineto(215.20037624,838.12658623)
\curveto(214.6901531,837.8776969)(214.16126326,837.69725213)(213.61370672,837.58525193)
\curveto(213.06615019,837.46080726)(212.40037122,837.39858493)(211.61636981,837.39858493)
\curveto(210.59592354,837.39858493)(209.6936997,837.58525193)(208.90969829,837.95858593)
\curveto(208.12569689,838.3443644)(207.50969578,838.91680987)(207.06169498,839.67592234)
\curveto(206.61369418,840.44747928)(206.38969377,841.42436992)(206.38969377,842.60659426)
\curveto(206.38969377,843.78881861)(206.58880524,844.77815371)(206.98702818,845.57459958)
\curveto(207.39769558,846.37104546)(207.96391882,846.96837986)(208.68569789,847.3666028)
\curveto(209.40747696,847.76482573)(210.24747847,847.9639372)(211.20570241,847.9639372)
\closepath
\moveto(211.22436911,845.98526699)
\curveto(210.67681257,845.98526699)(210.22881177,845.81104445)(209.8803667,845.46259938)
\curveto(209.53192163,845.11415431)(209.32658793,844.57282001)(209.26436559,843.83859647)
\lineto(213.16570592,843.83859647)
\curveto(213.15326145,844.44837534)(212.98526115,844.95859848)(212.66170502,845.36926588)
\curveto(212.35059335,845.77993329)(211.87148138,845.98526699)(211.22436911,845.98526699)
\closepath
}
}
{
\newrgbcolor{curcolor}{0 0 0}
\pscustom[linestyle=none,fillstyle=solid,fillcolor=curcolor]
{
\newpath
\moveto(223.73105122,847.9639372)
\curveto(223.86794035,847.9639372)(224.02971842,847.95771497)(224.21638542,847.9452705)
\curveto(224.40305242,847.93282603)(224.55238602,847.91415933)(224.66438622,847.8892704)
\lineto(224.45905252,845.27593238)
\curveto(224.35949679,845.30082132)(224.22882989,845.31948802)(224.06705182,845.33193248)
\curveto(223.90527375,845.35682142)(223.76216239,845.36926588)(223.63771772,845.36926588)
\curveto(223.16482798,845.36926588)(222.71060494,845.28215462)(222.27504861,845.10793208)
\curveto(221.83949227,844.94615401)(221.48482497,844.67859798)(221.2110467,844.30526398)
\curveto(220.9497129,843.93192997)(220.819046,843.42170684)(220.819046,842.77459457)
\lineto(220.819046,837.58525193)
\lineto(218.03770768,837.58525193)
\lineto(218.03770768,847.7772702)
\lineto(220.14704479,847.7772702)
\lineto(220.5577122,846.05993379)
\lineto(220.6883791,846.05993379)
\curveto(220.9870463,846.58260139)(221.3977137,847.03060219)(221.92038131,847.4039362)
\curveto(222.44304891,847.7772702)(223.04660555,847.9639372)(223.73105122,847.9639372)
\closepath
}
}
{
\newrgbcolor{curcolor}{0 0 0}
\pscustom[linestyle=none,fillstyle=solid,fillcolor=curcolor]
{
\newpath
\moveto(225.00040507,838.89192094)
\curveto(225.00040507,839.46436641)(225.1559609,839.86258935)(225.46707257,840.08658975)
\curveto(225.77818424,840.32303461)(226.15774047,840.44125705)(226.60574128,840.44125705)
\curveto(227.04129761,840.44125705)(227.41463162,840.32303461)(227.72574328,840.08658975)
\curveto(228.03685495,839.86258935)(228.19241079,839.46436641)(228.19241079,838.89192094)
\curveto(228.19241079,838.3443644)(228.03685495,837.94614147)(227.72574328,837.69725213)
\curveto(227.41463162,837.46080726)(227.04129761,837.34258483)(226.60574128,837.34258483)
\curveto(226.15774047,837.34258483)(225.77818424,837.46080726)(225.46707257,837.69725213)
\curveto(225.1559609,837.94614147)(225.00040507,838.3443644)(225.00040507,838.89192094)
\closepath
}
}
{
\newrgbcolor{curcolor}{0 0 0}
\pscustom[linestyle=none,fillstyle=solid,fillcolor=curcolor]
{
\newpath
\moveto(230.60041282,850.40927492)
\curveto(230.60041282,850.93194252)(230.74352419,851.28660982)(231.02974693,851.47327682)
\curveto(231.32841413,851.67238829)(231.68930367,851.77194403)(232.11241553,851.77194403)
\curveto(232.52308294,851.77194403)(232.87775024,851.67238829)(233.17641744,851.47327682)
\curveto(233.47508464,851.28660982)(233.62441825,850.93194252)(233.62441825,850.40927492)
\curveto(233.62441825,849.89905178)(233.47508464,849.54438448)(233.17641744,849.34527301)
\curveto(232.87775024,849.14616154)(232.52308294,849.04660581)(232.11241553,849.04660581)
\curveto(231.68930367,849.04660581)(231.32841413,849.14616154)(231.02974693,849.34527301)
\curveto(230.74352419,849.54438448)(230.60041282,849.89905178)(230.60041282,850.40927492)
\closepath
\moveto(229.89107822,833.1052439)
\curveto(229.56752208,833.1052439)(229.23774371,833.13013283)(228.90174311,833.1799107)
\curveto(228.56574251,833.2172441)(228.28574201,833.26702197)(228.06174161,833.3292443)
\lineto(228.06174161,835.51324822)
\curveto(228.28574201,835.45102588)(228.49729794,835.40747025)(228.69640941,835.38258131)
\curveto(228.89552088,835.35769238)(229.11952128,835.34524791)(229.36841062,835.34524791)
\curveto(229.74174462,835.34524791)(230.05907852,835.45102588)(230.32041232,835.66258182)
\curveto(230.58174612,835.87413775)(230.71241303,836.28480515)(230.71241303,836.89458402)
\lineto(230.71241303,847.7772702)
\lineto(233.49375134,847.7772702)
\lineto(233.49375134,836.48391662)
\curveto(233.49375134,835.86169328)(233.37552891,835.29547005)(233.13908404,834.78524691)
\curveto(232.90263917,834.27502377)(232.5168607,833.8705786)(231.98174863,833.5719114)
\curveto(231.45908103,833.26079973)(230.76219089,833.1052439)(229.89107822,833.1052439)
\closepath
}
}
{
\newrgbcolor{curcolor}{0 0 0}
\pscustom[linestyle=none,fillstyle=solid,fillcolor=curcolor]
{
\newpath
\moveto(243.51777363,840.60925735)
\curveto(243.51777363,839.57636661)(243.15066186,838.77992074)(242.41643833,838.21991973)
\curveto(241.69465925,837.6723632)(240.61199065,837.39858493)(239.1684325,837.39858493)
\curveto(238.4590979,837.39858493)(237.84931903,837.4483628)(237.33909589,837.54791853)
\curveto(236.82887275,837.6350298)(236.31864962,837.7843634)(235.80842648,837.99591933)
\lineto(235.80842648,840.29192345)
\curveto(236.35598302,840.04303411)(236.94709519,839.83770041)(237.58176299,839.67592234)
\curveto(238.2164308,839.51414428)(238.7764318,839.43325524)(239.261766,839.43325524)
\curveto(239.79687807,839.43325524)(240.18265654,839.51414428)(240.41910141,839.67592234)
\curveto(240.65554628,839.83770041)(240.77376871,840.04925635)(240.77376871,840.31059015)
\curveto(240.77376871,840.48481268)(240.72399085,840.64036852)(240.62443511,840.77725765)
\curveto(240.53732385,840.91414679)(240.33821238,841.06970262)(240.02710071,841.24392515)
\curveto(239.71598904,841.41814769)(239.23065484,841.64214809)(238.5710981,841.91592636)
\curveto(237.92398583,842.18970463)(237.39509599,842.45726066)(236.98442859,842.71859446)
\curveto(236.58620565,842.99237273)(236.28753845,843.31592887)(236.08842698,843.68926287)
\curveto(235.88931551,844.07504134)(235.78975978,844.55415331)(235.78975978,845.12659878)
\curveto(235.78975978,846.07237825)(236.15687155,846.78171286)(236.89109509,847.2546026)
\curveto(237.62531863,847.72749233)(238.60220927,847.9639372)(239.82176701,847.9639372)
\curveto(240.45643481,847.9639372)(241.05999145,847.90171487)(241.63243692,847.7772702)
\curveto(242.20488239,847.65282553)(242.79599456,847.44749183)(243.40577343,847.1612691)
\lineto(242.56577193,845.16393218)
\curveto(242.06799326,845.37548812)(241.59510352,845.54971065)(241.14710272,845.68659979)
\curveto(240.69910191,845.83593339)(240.24487888,845.91060019)(239.78443361,845.91060019)
\curveto(238.9630988,845.91060019)(238.5524314,845.68659979)(238.5524314,845.23859898)
\curveto(238.5524314,845.07682091)(238.60220927,844.92748731)(238.701765,844.79059818)
\curveto(238.8137652,844.66615351)(239.0190989,844.52926438)(239.3177661,844.37993078)
\curveto(239.62887777,844.23059718)(240.08310081,844.03148571)(240.68043521,843.78259637)
\curveto(241.26532515,843.5461515)(241.76932605,843.29726217)(242.19243792,843.03592837)
\curveto(242.61554979,842.78703903)(242.93910593,842.46970513)(243.16310633,842.08392666)
\curveto(243.3995512,841.69814819)(243.51777363,841.20659175)(243.51777363,840.60925735)
\closepath
}
}
{
\newrgbcolor{curcolor}{0 0 0}
\pscustom[linestyle=none,fillstyle=solid,fillcolor=curcolor]
{
\newpath
\moveto(217.24399753,754.90755706)
\curveto(218.61288888,754.90755706)(219.65822408,754.60888986)(220.38000316,754.01155545)
\curveto(221.11422669,753.42666552)(221.48133846,752.52444168)(221.48133846,751.30488394)
\lineto(221.48133846,744.51020509)
\lineto(219.54000165,744.51020509)
\lineto(218.99866735,745.8915409)
\lineto(218.92400055,745.8915409)
\curveto(218.48844421,745.34398436)(218.02799894,744.94576143)(217.54266474,744.69687209)
\curveto(217.05733053,744.44798276)(216.39155156,744.32353809)(215.54532782,744.32353809)
\curveto(214.63688175,744.32353809)(213.88399151,744.58487189)(213.28665711,745.10753949)
\curveto(212.6893227,745.6302071)(212.3906555,746.44531967)(212.3906555,747.55287721)
\curveto(212.3906555,748.63554582)(212.77021174,749.43199169)(213.52932421,749.94221483)
\curveto(214.28843668,750.45243796)(215.42710539,750.7386607)(216.94533033,750.80088303)
\lineto(218.71866684,750.85688313)
\lineto(218.71866684,751.30488394)
\curveto(218.71866684,751.83999601)(218.57555548,752.23199671)(218.28933274,752.48088604)
\curveto(218.01555447,752.72977538)(217.629776,752.85422005)(217.13199733,752.85422005)
\curveto(216.63421866,752.85422005)(216.14888446,752.77955325)(215.67599472,752.63021964)
\curveto(215.20310499,752.49333051)(214.73021525,752.31910798)(214.25732551,752.10755204)
\lineto(213.34265721,753.99288875)
\curveto(213.87776928,754.26666702)(214.48132591,754.48444519)(215.15332712,754.64622326)
\curveto(215.82532832,754.82044579)(216.52221846,754.90755706)(217.24399753,754.90755706)
\closepath
\moveto(218.71866684,749.23288022)
\lineto(217.63599824,749.19554682)
\curveto(216.73999663,749.17065789)(216.11777329,749.00887982)(215.76932822,748.71021262)
\curveto(215.42088315,748.41154542)(215.24666062,748.01954471)(215.24666062,747.53421051)
\curveto(215.24666062,747.11109864)(215.37110529,746.80620921)(215.61999462,746.6195422)
\curveto(215.86888396,746.44531967)(216.19244009,746.3582084)(216.59066303,746.3582084)
\curveto(217.18799743,746.3582084)(217.69199834,746.53243094)(218.10266574,746.88087601)
\curveto(218.51333314,747.24176554)(218.71866684,747.74576644)(218.71866684,748.39287872)
\closepath
}
}
{
\newrgbcolor{curcolor}{0 0 0}
\pscustom[linestyle=none,fillstyle=solid,fillcolor=curcolor]
{
\newpath
\moveto(227.56668441,744.32353809)
\curveto(226.43423793,744.32353809)(225.50712516,744.76531666)(224.78534609,745.6488738)
\curveto(224.07601148,746.5448754)(223.72134418,747.85776665)(223.72134418,749.58754752)
\curveto(223.72134418,751.32977287)(224.08223372,752.64888635)(224.80401279,753.54488795)
\curveto(225.52579186,754.44088956)(226.47157133,754.88889036)(227.64135121,754.88889036)
\curveto(228.37557474,754.88889036)(228.97913138,754.74577899)(229.45202112,754.45955626)
\curveto(229.92491086,754.17333352)(230.29824486,753.81866622)(230.57202313,753.39555435)
\lineto(230.66535663,753.39555435)
\curveto(230.62802323,753.59466582)(230.58446759,753.88088855)(230.53468973,754.25422256)
\curveto(230.48491186,754.64000103)(230.46002293,755.03200173)(230.46002293,755.43022466)
\lineto(230.46002293,758.69689719)
\lineto(233.24136124,758.69689719)
\lineto(233.24136124,744.51020509)
\lineto(231.11335743,744.51020509)
\lineto(230.57202313,745.8355408)
\lineto(230.46002293,745.8355408)
\curveto(230.18624466,745.41242893)(229.81913289,745.05153939)(229.35868762,744.75287219)
\curveto(228.89824235,744.46664946)(228.30090794,744.32353809)(227.56668441,744.32353809)
\closepath
\moveto(228.53735281,746.5448754)
\curveto(229.29646528,746.5448754)(229.83157735,746.7688758)(230.14268902,747.21687661)
\curveto(230.45380069,747.67732188)(230.62180099,748.36176755)(230.64668993,749.27021362)
\lineto(230.64668993,749.56888082)
\curveto(230.64668993,750.5519937)(230.49113409,751.30488394)(230.18002242,751.82755154)
\curveto(229.88135522,752.36266361)(229.32135422,752.63021964)(228.50001941,752.63021964)
\curveto(227.89024054,752.63021964)(227.41112857,752.36266361)(227.0626835,751.82755154)
\curveto(226.71423843,751.30488394)(226.5400159,750.54577146)(226.5400159,749.55021412)
\curveto(226.5400159,748.55465678)(226.71423843,747.80176655)(227.0626835,747.29154341)
\curveto(227.41112857,746.79376474)(227.90268501,746.5448754)(228.53735281,746.5448754)
\closepath
}
}
{
\newrgbcolor{curcolor}{0 0 0}
\pscustom[linestyle=none,fillstyle=solid,fillcolor=curcolor]
{
\newpath
\moveto(248.11870944,754.88889036)
\curveto(249.27604485,754.88889036)(250.14715752,754.59022316)(250.73204746,753.99288875)
\curveto(251.32938186,753.40799882)(251.62804906,752.46221934)(251.62804906,751.15555033)
\lineto(251.62804906,744.51020509)
\lineto(248.84671074,744.51020509)
\lineto(248.84671074,750.46488243)
\curveto(248.84671074,751.93332951)(248.33648761,752.66755305)(247.31604133,752.66755305)
\curveto(246.58181779,752.66755305)(246.05915019,752.40621924)(245.74803852,751.88355164)
\curveto(245.43692685,751.36088404)(245.28137102,750.6079938)(245.28137102,749.62488092)
\lineto(245.28137102,744.51020509)
\lineto(242.5000327,744.51020509)
\lineto(242.5000327,750.46488243)
\curveto(242.5000327,751.93332951)(241.98980956,752.66755305)(240.96936329,752.66755305)
\curveto(240.19780635,752.66755305)(239.66269428,752.37510808)(239.36402708,751.79021814)
\curveto(239.07780434,751.21777267)(238.93469298,750.39021563)(238.93469298,749.30754702)
\lineto(238.93469298,744.51020509)
\lineto(236.15335466,744.51020509)
\lineto(236.15335466,754.70222336)
\lineto(238.28135847,754.70222336)
\lineto(238.65469247,753.39555435)
\lineto(238.80402607,753.39555435)
\curveto(239.11513774,753.91822195)(239.53824961,754.29777819)(240.07336168,754.53422306)
\curveto(240.62091822,754.77066793)(241.18714146,754.88889036)(241.77203139,754.88889036)
\curveto(242.5186994,754.88889036)(243.14714497,754.76444569)(243.65736811,754.51555636)
\curveto(244.18003571,754.27911149)(244.58448088,753.90577749)(244.87070362,753.39555435)
\lineto(245.11337072,753.39555435)
\curveto(245.42448239,753.91822195)(245.85381649,754.29777819)(246.40137303,754.53422306)
\curveto(246.96137403,754.77066793)(247.5338195,754.88889036)(248.11870944,754.88889036)
\closepath
}
}
{
\newrgbcolor{curcolor}{0 0 0}
\pscustom[linestyle=none,fillstyle=solid,fillcolor=curcolor]
{
\newpath
\moveto(255.88406448,758.69689719)
\curveto(256.29473189,758.69689719)(256.64939919,758.59734145)(256.94806639,758.39822998)
\curveto(257.24673359,758.21156298)(257.3960672,757.85689568)(257.3960672,757.33422808)
\curveto(257.3960672,756.82400494)(257.24673359,756.46933764)(256.94806639,756.27022617)
\curveto(256.64939919,756.0711147)(256.29473189,755.97155897)(255.88406448,755.97155897)
\curveto(255.46095262,755.97155897)(255.10006308,756.0711147)(254.80139588,756.27022617)
\curveto(254.51517314,756.46933764)(254.37206177,756.82400494)(254.37206177,757.33422808)
\curveto(254.37206177,757.85689568)(254.51517314,758.21156298)(254.80139588,758.39822998)
\curveto(255.10006308,758.59734145)(255.46095262,758.69689719)(255.88406448,758.69689719)
\closepath
\moveto(257.26540029,754.70222336)
\lineto(257.26540029,744.51020509)
\lineto(254.48406198,744.51020509)
\lineto(254.48406198,754.70222336)
\closepath
}
}
{
\newrgbcolor{curcolor}{0 0 0}
\pscustom[linestyle=none,fillstyle=solid,fillcolor=curcolor]
{
\newpath
\moveto(265.96408687,754.88889036)
\curveto(267.05919995,754.88889036)(267.93653485,754.59022316)(268.59609159,753.99288875)
\curveto(269.25564833,753.40799882)(269.5854267,752.46221934)(269.5854267,751.15555033)
\lineto(269.5854267,744.51020509)
\lineto(266.80408838,744.51020509)
\lineto(266.80408838,750.46488243)
\curveto(266.80408838,751.19910597)(266.67342148,751.74666251)(266.41208768,752.10755204)
\curveto(266.15075388,752.48088604)(265.73386424,752.66755305)(265.16141877,752.66755305)
\curveto(264.31519503,752.66755305)(263.73652733,752.37510808)(263.42541566,751.79021814)
\curveto(263.11430399,751.21777267)(262.95874815,750.39021563)(262.95874815,749.30754702)
\lineto(262.95874815,744.51020509)
\lineto(260.17740983,744.51020509)
\lineto(260.17740983,754.70222336)
\lineto(262.30541365,754.70222336)
\lineto(262.67874765,753.39555435)
\lineto(262.82808125,753.39555435)
\curveto(263.15163739,753.91822195)(263.59341596,754.29777819)(264.15341696,754.53422306)
\curveto(264.72586243,754.77066793)(265.32941907,754.88889036)(265.96408687,754.88889036)
\closepath
}
}
{
\newrgbcolor{curcolor}{0 0 0}
\pscustom[linestyle=none,fillstyle=solid,fillcolor=curcolor]
{
\newpath
\moveto(272.04942417,745.8168741)
\curveto(272.04942417,746.38931957)(272.20498001,746.7875425)(272.51609168,747.01154291)
\curveto(272.82720335,747.24798777)(273.20675958,747.36621021)(273.65476039,747.36621021)
\curveto(274.09031672,747.36621021)(274.46365072,747.24798777)(274.77476239,747.01154291)
\curveto(275.08587406,746.7875425)(275.2414299,746.38931957)(275.2414299,745.8168741)
\curveto(275.2414299,745.26931756)(275.08587406,744.87109463)(274.77476239,744.62220529)
\curveto(274.46365072,744.38576042)(274.09031672,744.26753799)(273.65476039,744.26753799)
\curveto(273.20675958,744.26753799)(272.82720335,744.38576042)(272.51609168,744.62220529)
\curveto(272.20498001,744.87109463)(272.04942417,745.26931756)(272.04942417,745.8168741)
\closepath
}
}
{
\newrgbcolor{curcolor}{0 0 0}
\pscustom[linestyle=none,fillstyle=solid,fillcolor=curcolor]
{
\newpath
\moveto(280.54277045,758.69689719)
\lineto(280.54277045,755.80355867)
\curveto(280.54277045,755.29333553)(280.52410375,754.80800133)(280.48677035,754.34755606)
\curveto(280.46188142,753.89955525)(280.43699249,753.58222135)(280.41210355,753.39555435)
\lineto(280.56143715,753.39555435)
\curveto(280.88499329,753.91822195)(281.30188293,754.29777819)(281.81210606,754.53422306)
\curveto(282.3223292,754.77066793)(282.88855244,754.88889036)(283.51077577,754.88889036)
\curveto(284.60588885,754.88889036)(285.48944599,754.59022316)(286.16144719,753.99288875)
\curveto(286.8334484,753.40799882)(287.169449,752.46221934)(287.169449,751.15555033)
\lineto(287.169449,744.51020509)
\lineto(284.38811068,744.51020509)
\lineto(284.38811068,750.46488243)
\curveto(284.38811068,751.93332951)(283.84055414,752.66755305)(282.74544107,752.66755305)
\curveto(281.9116618,752.66755305)(281.33299409,752.37510808)(281.00943796,751.79021814)
\curveto(280.69832629,751.21777267)(280.54277045,750.39021563)(280.54277045,749.30754702)
\lineto(280.54277045,744.51020509)
\lineto(277.76143213,744.51020509)
\lineto(277.76143213,758.69689719)
\closepath
}
}
{
\newrgbcolor{curcolor}{0 0 0}
\pscustom[linestyle=none,fillstyle=solid,fillcolor=curcolor]
{
\newpath
\moveto(294.20678801,754.90755706)
\curveto(295.57567935,754.90755706)(296.62101456,754.60888986)(297.34279363,754.01155545)
\curveto(298.07701717,753.42666552)(298.44412894,752.52444168)(298.44412894,751.30488394)
\lineto(298.44412894,744.51020509)
\lineto(296.50279212,744.51020509)
\lineto(295.96145782,745.8915409)
\lineto(295.88679102,745.8915409)
\curveto(295.45123468,745.34398436)(294.99078941,744.94576143)(294.50545521,744.69687209)
\curveto(294.02012101,744.44798276)(293.35434203,744.32353809)(292.5081183,744.32353809)
\curveto(291.59967222,744.32353809)(290.84678198,744.58487189)(290.24944758,745.10753949)
\curveto(289.65211318,745.6302071)(289.35344597,746.44531967)(289.35344597,747.55287721)
\curveto(289.35344597,748.63554582)(289.73300221,749.43199169)(290.49211468,749.94221483)
\curveto(291.25122715,750.45243796)(292.38989586,750.7386607)(293.9081208,750.80088303)
\lineto(295.68145732,750.85688313)
\lineto(295.68145732,751.30488394)
\curveto(295.68145732,751.83999601)(295.53834595,752.23199671)(295.25212321,752.48088604)
\curveto(294.97834495,752.72977538)(294.59256648,752.85422005)(294.09478781,752.85422005)
\curveto(293.59700914,752.85422005)(293.11167493,752.77955325)(292.6387852,752.63021964)
\curveto(292.16589546,752.49333051)(291.69300572,752.31910798)(291.22011599,752.10755204)
\lineto(290.30544768,753.99288875)
\curveto(290.84055975,754.26666702)(291.44411639,754.48444519)(292.11611759,754.64622326)
\curveto(292.7881188,754.82044579)(293.48500894,754.90755706)(294.20678801,754.90755706)
\closepath
\moveto(295.68145732,749.23288022)
\lineto(294.59878871,749.19554682)
\curveto(293.7027871,749.17065789)(293.08056377,749.00887982)(292.7321187,748.71021262)
\curveto(292.38367363,748.41154542)(292.20945109,748.01954471)(292.20945109,747.53421051)
\curveto(292.20945109,747.11109864)(292.33389576,746.80620921)(292.5827851,746.6195422)
\curveto(292.83167443,746.44531967)(293.15523057,746.3582084)(293.5534535,746.3582084)
\curveto(294.15078791,746.3582084)(294.65478881,746.53243094)(295.06545621,746.88087601)
\curveto(295.47612362,747.24176554)(295.68145732,747.74576644)(295.68145732,748.39287872)
\closepath
}
}
{
\newrgbcolor{curcolor}{0 0 0}
\pscustom[linestyle=none,fillstyle=solid,fillcolor=curcolor]
{
\newpath
\moveto(307.0868128,754.88889036)
\curveto(308.18192587,754.88889036)(309.05926078,754.59022316)(309.71881751,753.99288875)
\curveto(310.37837425,753.40799882)(310.70815262,752.46221934)(310.70815262,751.15555033)
\lineto(310.70815262,744.51020509)
\lineto(307.9268143,744.51020509)
\lineto(307.9268143,750.46488243)
\curveto(307.9268143,751.19910597)(307.7961474,751.74666251)(307.5348136,752.10755204)
\curveto(307.2734798,752.48088604)(306.85659016,752.66755305)(306.28414469,752.66755305)
\curveto(305.43792095,752.66755305)(304.85925325,752.37510808)(304.54814158,751.79021814)
\curveto(304.23702991,751.21777267)(304.08147408,750.39021563)(304.08147408,749.30754702)
\lineto(304.08147408,744.51020509)
\lineto(301.30013576,744.51020509)
\lineto(301.30013576,754.70222336)
\lineto(303.42813957,754.70222336)
\lineto(303.80147357,753.39555435)
\lineto(303.95080717,753.39555435)
\curveto(304.27436331,753.91822195)(304.71614188,754.29777819)(305.27614288,754.53422306)
\curveto(305.84858835,754.77066793)(306.45214499,754.88889036)(307.0868128,754.88889036)
\closepath
}
}
{
\newrgbcolor{curcolor}{0 0 0}
\pscustom[linestyle=none,fillstyle=solid,fillcolor=curcolor]
{
\newpath
\moveto(316.79348992,744.32353809)
\curveto(315.66104345,744.32353809)(314.73393067,744.76531666)(314.0121516,745.6488738)
\curveto(313.302817,746.5448754)(312.9481497,747.85776665)(312.9481497,749.58754752)
\curveto(312.9481497,751.32977287)(313.30903923,752.64888635)(314.0308183,753.54488795)
\curveto(314.75259737,754.44088956)(315.69837685,754.88889036)(316.86815672,754.88889036)
\curveto(317.60238026,754.88889036)(318.2059369,754.74577899)(318.67882664,754.45955626)
\curveto(319.15171637,754.17333352)(319.52505037,753.81866622)(319.79882864,753.39555435)
\lineto(319.89216214,753.39555435)
\curveto(319.85482874,753.59466582)(319.81127311,753.88088855)(319.76149524,754.25422256)
\curveto(319.71171738,754.64000103)(319.68682844,755.03200173)(319.68682844,755.43022466)
\lineto(319.68682844,758.69689719)
\lineto(322.46816676,758.69689719)
\lineto(322.46816676,744.51020509)
\lineto(320.34016295,744.51020509)
\lineto(319.79882864,745.8355408)
\lineto(319.68682844,745.8355408)
\curveto(319.41305017,745.41242893)(319.0459384,745.05153939)(318.58549313,744.75287219)
\curveto(318.12504786,744.46664946)(317.52771346,744.32353809)(316.79348992,744.32353809)
\closepath
\moveto(317.76415833,746.5448754)
\curveto(318.5232708,746.5448754)(319.05838287,746.7688758)(319.36949454,747.21687661)
\curveto(319.68060621,747.67732188)(319.84860651,748.36176755)(319.87349544,749.27021362)
\lineto(319.87349544,749.56888082)
\curveto(319.87349544,750.5519937)(319.71793961,751.30488394)(319.40682794,751.82755154)
\curveto(319.10816074,752.36266361)(318.54815973,752.63021964)(317.72682493,752.63021964)
\curveto(317.11704606,752.63021964)(316.63793409,752.36266361)(316.28948902,751.82755154)
\curveto(315.94104395,751.30488394)(315.76682142,750.54577146)(315.76682142,749.55021412)
\curveto(315.76682142,748.55465678)(315.94104395,747.80176655)(316.28948902,747.29154341)
\curveto(316.63793409,746.79376474)(317.12949052,746.5448754)(317.76415833,746.5448754)
\closepath
}
}
{
\newrgbcolor{curcolor}{0 0 0}
\pscustom[linestyle=none,fillstyle=solid,fillcolor=curcolor]
{
\newpath
\moveto(328.16149849,744.51020509)
\lineto(325.38016017,744.51020509)
\lineto(325.38016017,758.69689719)
\lineto(328.16149849,758.69689719)
\closepath
}
}
{
\newrgbcolor{curcolor}{0 0 0}
\pscustom[linestyle=none,fillstyle=solid,fillcolor=curcolor]
{
\newpath
\moveto(335.27351175,754.88889036)
\curveto(336.67973649,754.88889036)(337.79351626,754.48444519)(338.61485107,753.67555485)
\curveto(339.43618587,752.87910898)(339.84685328,751.74044027)(339.84685328,750.25954873)
\lineto(339.84685328,748.91554632)
\lineto(333.27617483,748.91554632)
\curveto(333.30106377,748.13154491)(333.5312864,747.51554381)(333.96684274,747.06754301)
\curveto(334.41484354,746.6195422)(335.03084464,746.3955418)(335.81484605,746.3955418)
\curveto(336.46195832,746.3955418)(337.05307049,746.45776414)(337.58818256,746.5822088)
\curveto(338.1357391,746.71909794)(338.6957401,746.92443164)(339.26818557,747.19820991)
\lineto(339.26818557,745.05153939)
\curveto(338.75796244,744.80265006)(338.2290726,744.62220529)(337.68151606,744.51020509)
\curveto(337.13395953,744.38576042)(336.46818055,744.32353809)(335.68417915,744.32353809)
\curveto(334.66373288,744.32353809)(333.76150904,744.51020509)(332.97750763,744.88353909)
\curveto(332.19350622,745.26931756)(331.57750512,745.84176303)(331.12950432,746.6008755)
\curveto(330.68150351,747.37243244)(330.45750311,748.34932308)(330.45750311,749.53154742)
\curveto(330.45750311,750.71377177)(330.65661458,751.70310687)(331.05483752,752.49955274)
\curveto(331.46550492,753.29599862)(332.03172816,753.89333302)(332.75350723,754.29155596)
\curveto(333.4752863,754.68977889)(334.31528781,754.88889036)(335.27351175,754.88889036)
\closepath
\moveto(335.29217845,752.91022015)
\curveto(334.74462191,752.91022015)(334.29662111,752.73599761)(333.94817604,752.38755254)
\curveto(333.59973097,752.03910747)(333.39439727,751.49777317)(333.33217493,750.76354963)
\lineto(337.23351526,750.76354963)
\curveto(337.22107079,751.3733285)(337.05307049,751.88355164)(336.72951436,752.29421904)
\curveto(336.41840269,752.70488645)(335.93929072,752.91022015)(335.29217845,752.91022015)
\closepath
}
}
{
\newrgbcolor{curcolor}{0 0 0}
\pscustom[linestyle=none,fillstyle=solid,fillcolor=curcolor]
{
\newpath
\moveto(344.88685534,758.69689719)
\lineto(344.88685534,755.39289126)
\curveto(344.88685534,755.00711279)(344.87441087,754.62755656)(344.84952194,754.25422256)
\curveto(344.824633,753.88088855)(344.79974407,753.58844358)(344.77485514,753.37688765)
\lineto(344.88685534,753.37688765)
\curveto(345.16063361,753.79999952)(345.52774537,754.15466682)(345.98819064,754.44088956)
\curveto(346.44863591,754.73955676)(347.04597032,754.88889036)(347.78019386,754.88889036)
\curveto(348.9250848,754.88889036)(349.85219757,754.44088956)(350.56153218,753.54488795)
\curveto(351.27086678,752.66133081)(351.62553408,751.3546618)(351.62553408,749.62488092)
\curveto(351.62553408,747.88265558)(351.26464455,746.5635421)(350.54286548,745.6675405)
\curveto(349.8210864,744.77153889)(348.87530693,744.32353809)(347.70552706,744.32353809)
\curveto(346.95885905,744.32353809)(346.36774688,744.45420499)(345.93219054,744.71553879)
\curveto(345.50907867,744.98931706)(345.16063361,745.29420649)(344.88685534,745.6302071)
\lineto(344.70018834,745.6302071)
\lineto(344.23352083,744.51020509)
\lineto(342.10551702,744.51020509)
\lineto(342.10551702,758.69689719)
\closepath
\moveto(346.88419225,752.66755305)
\curveto(346.16241318,752.66755305)(345.65219004,752.43733041)(345.35352284,751.97688514)
\curveto(345.05485564,751.52888434)(344.8992998,750.8506609)(344.88685534,749.94221483)
\lineto(344.88685534,749.64354762)
\curveto(344.88685534,748.66043475)(345.0299667,747.90132228)(345.31618944,747.36621021)
\curveto(345.61485664,746.84354261)(346.14996871,746.5822088)(346.92152565,746.5822088)
\curveto(347.49397112,746.5822088)(347.94819416,746.84354261)(348.28419476,747.36621021)
\curveto(348.62019536,747.90132228)(348.78819566,748.66665698)(348.78819566,749.66221432)
\curveto(348.78819566,750.65777166)(348.61397313,751.40443967)(348.26552806,751.90221834)
\curveto(347.92952746,752.41244148)(347.46908219,752.66755305)(346.88419225,752.66755305)
\closepath
}
}
{
\newrgbcolor{curcolor}{0 0 0}
\pscustom[linestyle=none,fillstyle=solid,fillcolor=curcolor]
{
\newpath
\moveto(358.10286722,754.90755706)
\curveto(359.47175856,754.90755706)(360.51709377,754.60888986)(361.23887284,754.01155545)
\curveto(361.97309638,753.42666552)(362.34020815,752.52444168)(362.34020815,751.30488394)
\lineto(362.34020815,744.51020509)
\lineto(360.39887133,744.51020509)
\lineto(359.85753703,745.8915409)
\lineto(359.78287023,745.8915409)
\curveto(359.34731389,745.34398436)(358.88686862,744.94576143)(358.40153442,744.69687209)
\curveto(357.91620022,744.44798276)(357.25042125,744.32353809)(356.40419751,744.32353809)
\curveto(355.49575143,744.32353809)(354.7428612,744.58487189)(354.14552679,745.10753949)
\curveto(353.54819239,745.6302071)(353.24952519,746.44531967)(353.24952519,747.55287721)
\curveto(353.24952519,748.63554582)(353.62908142,749.43199169)(354.38819389,749.94221483)
\curveto(355.14730637,750.45243796)(356.28597507,750.7386607)(357.80420002,750.80088303)
\lineto(359.57753653,750.85688313)
\lineto(359.57753653,751.30488394)
\curveto(359.57753653,751.83999601)(359.43442516,752.23199671)(359.14820243,752.48088604)
\curveto(358.87442416,752.72977538)(358.48864569,752.85422005)(357.99086702,752.85422005)
\curveto(357.49308835,752.85422005)(357.00775414,752.77955325)(356.53486441,752.63021964)
\curveto(356.06197467,752.49333051)(355.58908493,752.31910798)(355.1161952,752.10755204)
\lineto(354.20152689,753.99288875)
\curveto(354.73663896,754.26666702)(355.3401956,754.48444519)(356.0121968,754.64622326)
\curveto(356.68419801,754.82044579)(357.38108815,754.90755706)(358.10286722,754.90755706)
\closepath
\moveto(359.57753653,749.23288022)
\lineto(358.49486792,749.19554682)
\curveto(357.59886632,749.17065789)(356.97664298,749.00887982)(356.62819791,748.71021262)
\curveto(356.27975284,748.41154542)(356.10553031,748.01954471)(356.10553031,747.53421051)
\curveto(356.10553031,747.11109864)(356.22997497,746.80620921)(356.47886431,746.6195422)
\curveto(356.72775364,746.44531967)(357.05130978,746.3582084)(357.44953271,746.3582084)
\curveto(358.04686712,746.3582084)(358.55086802,746.53243094)(358.96153542,746.88087601)
\curveto(359.37220283,747.24176554)(359.57753653,747.74576644)(359.57753653,748.39287872)
\closepath
}
}
{
\newrgbcolor{curcolor}{0 0 0}
\pscustom[linestyle=none,fillstyle=solid,fillcolor=curcolor]
{
\newpath
\moveto(370.88955851,754.88889036)
\curveto(371.02644764,754.88889036)(371.18822571,754.88266813)(371.37489271,754.87022366)
\curveto(371.56155971,754.85777919)(371.71089331,754.83911249)(371.82289351,754.81422356)
\lineto(371.61755981,752.20088554)
\curveto(371.51800408,752.22577448)(371.38733718,752.24444118)(371.22555911,752.25688564)
\curveto(371.06378104,752.28177458)(370.92066967,752.29421904)(370.79622501,752.29421904)
\curveto(370.32333527,752.29421904)(369.86911223,752.20710778)(369.4335559,752.03288524)
\curveto(368.99799956,751.87110717)(368.64333226,751.60355114)(368.36955399,751.23021714)
\curveto(368.10822019,750.85688313)(367.97755329,750.34666)(367.97755329,749.69954772)
\lineto(367.97755329,744.51020509)
\lineto(365.19621497,744.51020509)
\lineto(365.19621497,754.70222336)
\lineto(367.30555208,754.70222336)
\lineto(367.71621949,752.98488695)
\lineto(367.84688639,752.98488695)
\curveto(368.14555359,753.50755455)(368.55622099,753.95555535)(369.0788886,754.32888936)
\curveto(369.6015562,754.70222336)(370.20511284,754.88889036)(370.88955851,754.88889036)
\closepath
}
}
{
\newrgbcolor{curcolor}{0 0 0}
\pscustom[linestyle=none,fillstyle=solid,fillcolor=curcolor]
{
\newpath
\moveto(380.78292854,747.53421051)
\curveto(380.78292854,746.50131977)(380.41581677,745.7048739)(379.68159323,745.14487289)
\curveto(378.95981416,744.59731636)(377.87714555,744.32353809)(376.43358741,744.32353809)
\curveto(375.7242528,744.32353809)(375.11447393,744.37331596)(374.6042508,744.47287169)
\curveto(374.09402766,744.55998296)(373.58380452,744.70931656)(373.07358139,744.92087249)
\lineto(373.07358139,747.21687661)
\curveto(373.62113792,746.96798727)(374.21225009,746.76265357)(374.8469179,746.6008755)
\curveto(375.4815857,746.43909744)(376.04158671,746.3582084)(376.52692091,746.3582084)
\curveto(377.06203298,746.3582084)(377.44781145,746.43909744)(377.68425632,746.6008755)
\curveto(377.92070119,746.76265357)(378.03892362,746.97420951)(378.03892362,747.23554331)
\curveto(378.03892362,747.40976584)(377.98914575,747.56532168)(377.88959002,747.70221081)
\curveto(377.80247875,747.83909995)(377.60336728,747.99465578)(377.29225561,748.16887831)
\curveto(376.98114395,748.34310085)(376.49580974,748.56710125)(375.836253,748.84087952)
\curveto(375.18914073,749.11465779)(374.6602509,749.38221382)(374.24958349,749.64354762)
\curveto(373.85136056,749.91732589)(373.55269336,750.24088203)(373.35358189,750.61421603)
\curveto(373.15447042,750.9999945)(373.05491469,751.47910647)(373.05491469,752.05155194)
\curveto(373.05491469,752.99733141)(373.42202645,753.70666602)(374.15624999,754.17955576)
\curveto(374.89047353,754.65244549)(375.86736417,754.88889036)(377.08692191,754.88889036)
\curveto(377.72158972,754.88889036)(378.32514636,754.82666803)(378.89759183,754.70222336)
\curveto(379.4700373,754.57777869)(380.06114947,754.37244499)(380.67092834,754.08622225)
\lineto(379.83092683,752.08888534)
\curveto(379.33314816,752.30044128)(378.86025843,752.47466381)(378.41225762,752.61155294)
\curveto(377.96425682,752.76088655)(377.51003378,752.83555335)(377.04958851,752.83555335)
\curveto(376.22825371,752.83555335)(375.8175863,752.61155294)(375.8175863,752.16355214)
\curveto(375.8175863,752.00177407)(375.86736417,751.85244047)(375.96691991,751.71555134)
\curveto(376.07892011,751.59110667)(376.28425381,751.45421754)(376.58292101,751.30488394)
\curveto(376.89403268,751.15555033)(377.34825572,750.95643887)(377.94559012,750.70754953)
\curveto(378.53048006,750.47110466)(379.03448096,750.22221533)(379.45759283,749.96088153)
\curveto(379.8807047,749.71199219)(380.20426083,749.39465829)(380.42826124,749.00887982)
\curveto(380.6647061,748.62310135)(380.78292854,748.13154491)(380.78292854,747.53421051)
\closepath
}
}
{
\newrgbcolor{curcolor}{0 0 0}
\pscustom[linestyle=none,fillstyle=solid,fillcolor=curcolor]
{
\newpath
\moveto(217.16677129,733.80309016)
\curveto(218.53566263,733.80309016)(219.58099784,733.50442295)(220.30277691,732.90708855)
\curveto(221.03700045,732.32219861)(221.40411222,731.41997477)(221.40411222,730.20041703)
\lineto(221.40411222,723.40573819)
\lineto(219.46277541,723.40573819)
\lineto(218.9214411,724.787074)
\lineto(218.8467743,724.787074)
\curveto(218.41121797,724.23951746)(217.9507727,723.84129452)(217.46543849,723.59240519)
\curveto(216.98010429,723.34351585)(216.31432532,723.21907118)(215.46810158,723.21907118)
\curveto(214.55965551,723.21907118)(213.80676527,723.48040499)(213.20943086,724.00307259)
\curveto(212.61209646,724.52574019)(212.31342926,725.34085277)(212.31342926,726.44841031)
\curveto(212.31342926,727.53107891)(212.69298549,728.32752479)(213.45209796,728.83774792)
\curveto(214.21121044,729.34797106)(215.34987914,729.63419379)(216.86810409,729.69641613)
\lineto(218.6414406,729.75241623)
\lineto(218.6414406,730.20041703)
\curveto(218.6414406,730.7355291)(218.49832923,731.12752981)(218.2121065,731.37641914)
\curveto(217.93832823,731.62530848)(217.55254976,731.74975314)(217.05477109,731.74975314)
\curveto(216.55699242,731.74975314)(216.07165822,731.67508634)(215.59876848,731.52575274)
\curveto(215.12587874,731.38886361)(214.65298901,731.21464107)(214.18009927,731.00308514)
\lineto(213.26543096,732.88842185)
\curveto(213.80054303,733.16220012)(214.40409967,733.37997829)(215.07610088,733.54175636)
\curveto(215.74810208,733.71597889)(216.44499222,733.80309016)(217.16677129,733.80309016)
\closepath
\moveto(218.6414406,728.12841332)
\lineto(217.55877199,728.09107992)
\curveto(216.66277039,728.06619098)(216.04054705,727.90441292)(215.69210198,727.60574571)
\curveto(215.34365691,727.30707851)(215.16943438,726.91507781)(215.16943438,726.42974361)
\curveto(215.16943438,726.00663174)(215.29387904,725.7017423)(215.54276838,725.5150753)
\curveto(215.79165771,725.34085277)(216.11521385,725.2537415)(216.51343679,725.2537415)
\curveto(217.11077119,725.2537415)(217.61477209,725.42796403)(218.0254395,725.7764091)
\curveto(218.4361069,726.13729864)(218.6414406,726.64129954)(218.6414406,727.28841181)
\closepath
}
}
{
\newrgbcolor{curcolor}{0 0 0}
\pscustom[linestyle=none,fillstyle=solid,fillcolor=curcolor]
{
\newpath
\moveto(227.48945816,723.21907118)
\curveto(226.35701169,723.21907118)(225.42989891,723.66084975)(224.70811984,724.54440689)
\curveto(223.99878524,725.4404085)(223.64411794,726.75329974)(223.64411794,728.48308062)
\curveto(223.64411794,730.22530597)(224.00500747,731.54441944)(224.72678654,732.44042105)
\curveto(225.44856561,733.33642265)(226.39434509,733.78442346)(227.56412496,733.78442346)
\curveto(228.2983485,733.78442346)(228.90190514,733.64131209)(229.37479488,733.35508935)
\curveto(229.84768461,733.06886662)(230.22101861,732.71419932)(230.49479688,732.29108745)
\lineto(230.58813038,732.29108745)
\curveto(230.55079698,732.49019891)(230.50724135,732.77642165)(230.45746348,733.14975565)
\curveto(230.40768562,733.53553412)(230.38279668,733.92753482)(230.38279668,734.32575776)
\lineto(230.38279668,737.59243028)
\lineto(233.164135,737.59243028)
\lineto(233.164135,723.40573819)
\lineto(231.03613119,723.40573819)
\lineto(230.49479688,724.73107389)
\lineto(230.38279668,724.73107389)
\curveto(230.10901841,724.30796203)(229.74190664,723.94707249)(229.28146137,723.64840529)
\curveto(228.8210161,723.36218255)(228.2236817,723.21907118)(227.48945816,723.21907118)
\closepath
\moveto(228.46012657,725.4404085)
\curveto(229.21923904,725.4404085)(229.75435111,725.6644089)(230.06546278,726.1124097)
\curveto(230.37657445,726.57285497)(230.54457475,727.25730065)(230.56946368,728.16574672)
\lineto(230.56946368,728.46441392)
\curveto(230.56946368,729.44752679)(230.41390785,730.20041703)(230.10279618,730.72308464)
\curveto(229.80412898,731.25819671)(229.24412797,731.52575274)(228.42279317,731.52575274)
\curveto(227.8130143,731.52575274)(227.33390233,731.25819671)(226.98545726,730.72308464)
\curveto(226.63701219,730.20041703)(226.46278966,729.44130456)(226.46278966,728.44574722)
\curveto(226.46278966,727.45018988)(226.63701219,726.69729964)(226.98545726,726.1870765)
\curveto(227.33390233,725.68929783)(227.82545876,725.4404085)(228.46012657,725.4404085)
\closepath
}
}
{
\newrgbcolor{curcolor}{0 0 0}
\pscustom[linestyle=none,fillstyle=solid,fillcolor=curcolor]
{
\newpath
\moveto(248.04148319,733.78442346)
\curveto(249.1988186,733.78442346)(250.06993128,733.48575625)(250.65482121,732.88842185)
\curveto(251.25215562,732.30353191)(251.55082282,731.35775244)(251.55082282,730.05108343)
\lineto(251.55082282,723.40573819)
\lineto(248.7694845,723.40573819)
\lineto(248.7694845,729.36041553)
\curveto(248.7694845,730.8288626)(248.25926136,731.56308614)(247.23881509,731.56308614)
\curveto(246.50459155,731.56308614)(245.98192395,731.30175234)(245.67081228,730.77908474)
\curveto(245.35970061,730.25641713)(245.20414478,729.50352689)(245.20414478,728.52041402)
\lineto(245.20414478,723.40573819)
\lineto(242.42280646,723.40573819)
\lineto(242.42280646,729.36041553)
\curveto(242.42280646,730.8288626)(241.91258332,731.56308614)(240.89213705,731.56308614)
\curveto(240.12058011,731.56308614)(239.58546804,731.27064117)(239.28680083,730.68575124)
\curveto(239.0005781,730.11330576)(238.85746673,729.28574873)(238.85746673,728.20308012)
\lineto(238.85746673,723.40573819)
\lineto(236.07612841,723.40573819)
\lineto(236.07612841,733.59775646)
\lineto(238.20413223,733.59775646)
\lineto(238.57746623,732.29108745)
\lineto(238.72679983,732.29108745)
\curveto(239.0379115,732.81375505)(239.46102337,733.19331129)(239.99613544,733.42975615)
\curveto(240.54369198,733.66620102)(241.10991521,733.78442346)(241.69480515,733.78442346)
\curveto(242.44147316,733.78442346)(243.06991873,733.65997879)(243.58014186,733.41108945)
\curveto(244.10280947,733.17464459)(244.50725464,732.80131058)(244.79347737,732.29108745)
\lineto(245.03614447,732.29108745)
\curveto(245.34725614,732.81375505)(245.77659025,733.19331129)(246.32414678,733.42975615)
\curveto(246.88414779,733.66620102)(247.45659326,733.78442346)(248.04148319,733.78442346)
\closepath
}
}
{
\newrgbcolor{curcolor}{0 0 0}
\pscustom[linestyle=none,fillstyle=solid,fillcolor=curcolor]
{
\newpath
\moveto(255.80683824,737.59243028)
\curveto(256.21750564,737.59243028)(256.57217295,737.49287455)(256.87084015,737.29376308)
\curveto(257.16950735,737.10709608)(257.31884095,736.75242878)(257.31884095,736.22976117)
\curveto(257.31884095,735.71953804)(257.16950735,735.36487073)(256.87084015,735.16575927)
\curveto(256.57217295,734.9666478)(256.21750564,734.86709206)(255.80683824,734.86709206)
\curveto(255.38372637,734.86709206)(255.02283684,734.9666478)(254.72416963,735.16575927)
\curveto(254.4379469,735.36487073)(254.29483553,735.71953804)(254.29483553,736.22976117)
\curveto(254.29483553,736.75242878)(254.4379469,737.10709608)(254.72416963,737.29376308)
\curveto(255.02283684,737.49287455)(255.38372637,737.59243028)(255.80683824,737.59243028)
\closepath
\moveto(257.18817405,733.59775646)
\lineto(257.18817405,723.40573819)
\lineto(254.40683573,723.40573819)
\lineto(254.40683573,733.59775646)
\closepath
}
}
{
\newrgbcolor{curcolor}{0 0 0}
\pscustom[linestyle=none,fillstyle=solid,fillcolor=curcolor]
{
\newpath
\moveto(265.88686063,733.78442346)
\curveto(266.9819737,733.78442346)(267.85930861,733.48575625)(268.51886535,732.88842185)
\curveto(269.17842209,732.30353191)(269.50820046,731.35775244)(269.50820046,730.05108343)
\lineto(269.50820046,723.40573819)
\lineto(266.72686214,723.40573819)
\lineto(266.72686214,729.36041553)
\curveto(266.72686214,730.09463906)(266.59619524,730.6421956)(266.33486143,731.00308514)
\curveto(266.07352763,731.37641914)(265.656638,731.56308614)(265.08419253,731.56308614)
\curveto(264.23796879,731.56308614)(263.65930108,731.27064117)(263.34818941,730.68575124)
\curveto(263.03707774,730.11330576)(262.88152191,729.28574873)(262.88152191,728.20308012)
\lineto(262.88152191,723.40573819)
\lineto(260.10018359,723.40573819)
\lineto(260.10018359,733.59775646)
\lineto(262.22818741,733.59775646)
\lineto(262.60152141,732.29108745)
\lineto(262.75085501,732.29108745)
\curveto(263.07441114,732.81375505)(263.51618971,733.19331129)(264.07619072,733.42975615)
\curveto(264.64863619,733.66620102)(265.25219283,733.78442346)(265.88686063,733.78442346)
\closepath
}
}
{
\newrgbcolor{curcolor}{0 0 0}
\pscustom[linestyle=none,fillstyle=solid,fillcolor=curcolor]
{
\newpath
\moveto(276.84420666,736.73376208)
\curveto(278.56154308,736.73376208)(279.81221199,736.36042807)(280.59621339,735.61376007)
\curveto(281.39265926,734.87953653)(281.7908822,733.86531249)(281.7908822,732.57108795)
\curveto(281.7908822,731.78708654)(281.62910413,731.05908524)(281.305548,730.38708403)
\curveto(280.98199186,729.71508283)(280.44065756,729.17374853)(279.68154508,728.76308112)
\curveto(278.93487708,728.35241372)(277.91443081,728.14708002)(276.62020626,728.14708002)
\lineto(275.40687076,728.14708002)
\lineto(275.40687076,723.40573819)
\lineto(272.58819904,723.40573819)
\lineto(272.58819904,736.73376208)
\closepath
\moveto(276.69487306,734.41909126)
\lineto(275.40687076,734.41909126)
\lineto(275.40687076,730.46175083)
\lineto(276.34020576,730.46175083)
\curveto(277.13665163,730.46175083)(277.7650972,730.61730667)(278.22554247,730.92841834)
\curveto(278.69843221,731.25197447)(278.93487708,731.76841984)(278.93487708,732.47775445)
\curveto(278.93487708,733.77197899)(278.18820907,734.41909126)(276.69487306,734.41909126)
\closepath
}
}
{
\newrgbcolor{curcolor}{0 0 0}
\pscustom[linestyle=none,fillstyle=solid,fillcolor=curcolor]
{
\newpath
\moveto(288.26823371,733.80309016)
\curveto(289.63712505,733.80309016)(290.68246026,733.50442295)(291.40423933,732.90708855)
\curveto(292.13846287,732.32219861)(292.50557464,731.41997477)(292.50557464,730.20041703)
\lineto(292.50557464,723.40573819)
\lineto(290.56423783,723.40573819)
\lineto(290.02290352,724.787074)
\lineto(289.94823672,724.787074)
\curveto(289.51268039,724.23951746)(289.05223512,723.84129452)(288.56690091,723.59240519)
\curveto(288.08156671,723.34351585)(287.41578774,723.21907118)(286.569564,723.21907118)
\curveto(285.66111793,723.21907118)(284.90822769,723.48040499)(284.31089329,724.00307259)
\curveto(283.71355888,724.52574019)(283.41489168,725.34085277)(283.41489168,726.44841031)
\curveto(283.41489168,727.53107891)(283.79444791,728.32752479)(284.55356039,728.83774792)
\curveto(285.31267286,729.34797106)(286.45134157,729.63419379)(287.96956651,729.69641613)
\lineto(289.74290302,729.75241623)
\lineto(289.74290302,730.20041703)
\curveto(289.74290302,730.7355291)(289.59979165,731.12752981)(289.31356892,731.37641914)
\curveto(289.03979065,731.62530848)(288.65401218,731.74975314)(288.15623351,731.74975314)
\curveto(287.65845484,731.74975314)(287.17312064,731.67508634)(286.7002309,731.52575274)
\curveto(286.22734116,731.38886361)(285.75445143,731.21464107)(285.28156169,731.00308514)
\lineto(284.36689339,732.88842185)
\curveto(284.90200546,733.16220012)(285.50556209,733.37997829)(286.1775633,733.54175636)
\curveto(286.8495645,733.71597889)(287.54645464,733.80309016)(288.26823371,733.80309016)
\closepath
\moveto(289.74290302,728.12841332)
\lineto(288.66023441,728.09107992)
\curveto(287.76423281,728.06619098)(287.14200947,727.90441292)(286.7935644,727.60574571)
\curveto(286.44511933,727.30707851)(286.2708968,726.91507781)(286.2708968,726.42974361)
\curveto(286.2708968,726.00663174)(286.39534147,725.7017423)(286.6442308,725.5150753)
\curveto(286.89312014,725.34085277)(287.21667627,725.2537415)(287.61489921,725.2537415)
\curveto(288.21223361,725.2537415)(288.71623452,725.42796403)(289.12690192,725.7764091)
\curveto(289.53756932,726.13729864)(289.74290302,726.64129954)(289.74290302,727.28841181)
\closepath
}
}
{
\newrgbcolor{curcolor}{0 0 0}
\pscustom[linestyle=none,fillstyle=solid,fillcolor=curcolor]
{
\newpath
\moveto(301.1482585,733.78442346)
\curveto(302.24337158,733.78442346)(303.12070648,733.48575625)(303.78026322,732.88842185)
\curveto(304.43981996,732.30353191)(304.76959833,731.35775244)(304.76959833,730.05108343)
\lineto(304.76959833,723.40573819)
\lineto(301.98826001,723.40573819)
\lineto(301.98826001,729.36041553)
\curveto(301.98826001,730.09463906)(301.85759311,730.6421956)(301.5962593,731.00308514)
\curveto(301.3349255,731.37641914)(300.91803587,731.56308614)(300.3455904,731.56308614)
\curveto(299.49936666,731.56308614)(298.92069895,731.27064117)(298.60958728,730.68575124)
\curveto(298.29847562,730.11330576)(298.14291978,729.28574873)(298.14291978,728.20308012)
\lineto(298.14291978,723.40573819)
\lineto(295.36158146,723.40573819)
\lineto(295.36158146,733.59775646)
\lineto(297.48958528,733.59775646)
\lineto(297.86291928,732.29108745)
\lineto(298.01225288,732.29108745)
\curveto(298.33580902,732.81375505)(298.77758759,733.19331129)(299.33758859,733.42975615)
\curveto(299.91003406,733.66620102)(300.5135907,733.78442346)(301.1482585,733.78442346)
\closepath
}
}
{
\newrgbcolor{curcolor}{0 0 0}
\pscustom[linestyle=none,fillstyle=solid,fillcolor=curcolor]
{
\newpath
\moveto(311.82560403,733.78442346)
\curveto(313.23182878,733.78442346)(314.34560855,733.37997829)(315.16694336,732.57108795)
\curveto(315.98827816,731.77464208)(316.39894557,730.63597337)(316.39894557,729.15508182)
\lineto(316.39894557,727.81107942)
\lineto(309.82826712,727.81107942)
\curveto(309.85315605,727.02707801)(310.08337869,726.41107691)(310.51893503,725.9630761)
\curveto(310.96693583,725.5150753)(311.58293693,725.2910749)(312.36693834,725.2910749)
\curveto(313.01405061,725.2910749)(313.60516278,725.35329723)(314.14027485,725.4777419)
\curveto(314.68783139,725.61463103)(315.24783239,725.81996474)(315.82027786,726.093743)
\lineto(315.82027786,723.94707249)
\curveto(315.31005472,723.69818315)(314.78116489,723.51773839)(314.23360835,723.40573819)
\curveto(313.68605181,723.28129352)(313.02027284,723.21907118)(312.23627144,723.21907118)
\curveto(311.21582516,723.21907118)(310.31360132,723.40573819)(309.52959992,723.77907219)
\curveto(308.74559851,724.16485066)(308.12959741,724.73729613)(307.68159661,725.4964086)
\curveto(307.2335958,726.26796554)(307.0095954,727.24485618)(307.0095954,728.42708052)
\curveto(307.0095954,729.60930486)(307.20870687,730.59863997)(307.60692981,731.39508584)
\curveto(308.01759721,732.19153171)(308.58382045,732.78886612)(309.30559952,733.18708905)
\curveto(310.02737859,733.58531199)(310.86738009,733.78442346)(311.82560403,733.78442346)
\closepath
\moveto(311.84427073,731.80575324)
\curveto(311.2967142,731.80575324)(310.84871339,731.63153071)(310.50026833,731.28308564)
\curveto(310.15182326,730.93464057)(309.94648955,730.39330627)(309.88426722,729.65908273)
\lineto(313.78560755,729.65908273)
\curveto(313.77316308,730.2688616)(313.60516278,730.77908474)(313.28160664,731.18975214)
\curveto(312.97049498,731.60041954)(312.49138301,731.80575324)(311.84427073,731.80575324)
\closepath
}
}
{
\newrgbcolor{curcolor}{0 0 0}
\pscustom[linestyle=none,fillstyle=solid,fillcolor=curcolor]
{
\newpath
\moveto(321.43894762,723.40573819)
\lineto(318.65760931,723.40573819)
\lineto(318.65760931,737.59243028)
\lineto(321.43894762,737.59243028)
\closepath
}
}
{
\newrgbcolor{curcolor}{0 0 0}
\pscustom[linestyle=none,fillstyle=solid,fillcolor=curcolor]
{
\newpath
\moveto(323.95895265,724.71240719)
\curveto(323.95895265,725.28485267)(324.11450848,725.6830756)(324.42562015,725.907076)
\curveto(324.73673182,726.14352087)(325.11628806,726.26174331)(325.56428886,726.26174331)
\curveto(325.99984519,726.26174331)(326.3731792,726.14352087)(326.68429087,725.907076)
\curveto(326.99540254,725.6830756)(327.15095837,725.28485267)(327.15095837,724.71240719)
\curveto(327.15095837,724.16485066)(326.99540254,723.76662772)(326.68429087,723.51773839)
\curveto(326.3731792,723.28129352)(325.99984519,723.16307108)(325.56428886,723.16307108)
\curveto(325.11628806,723.16307108)(324.73673182,723.28129352)(324.42562015,723.51773839)
\curveto(324.11450848,723.76662772)(323.95895265,724.16485066)(323.95895265,724.71240719)
\closepath
}
}
{
\newrgbcolor{curcolor}{0 0 0}
\pscustom[linestyle=none,fillstyle=solid,fillcolor=curcolor]
{
\newpath
\moveto(332.45229893,737.59243028)
\lineto(332.45229893,734.69909176)
\curveto(332.45229893,734.18886863)(332.43363223,733.70353442)(332.39629883,733.24308915)
\curveto(332.37140989,732.79508835)(332.34652096,732.47775445)(332.32163203,732.29108745)
\lineto(332.47096563,732.29108745)
\curveto(332.79452176,732.81375505)(333.2114114,733.19331129)(333.72163454,733.42975615)
\curveto(334.23185767,733.66620102)(334.79808091,733.78442346)(335.42030425,733.78442346)
\curveto(336.51541732,733.78442346)(337.39897446,733.48575625)(338.07097567,732.88842185)
\curveto(338.74297687,732.30353191)(339.07897747,731.35775244)(339.07897747,730.05108343)
\lineto(339.07897747,723.40573819)
\lineto(336.29763915,723.40573819)
\lineto(336.29763915,729.36041553)
\curveto(336.29763915,730.8288626)(335.75008262,731.56308614)(334.65496954,731.56308614)
\curveto(333.82119027,731.56308614)(333.24252257,731.27064117)(332.91896643,730.68575124)
\curveto(332.60785476,730.11330576)(332.45229893,729.28574873)(332.45229893,728.20308012)
\lineto(332.45229893,723.40573819)
\lineto(329.67096061,723.40573819)
\lineto(329.67096061,737.59243028)
\closepath
}
}
{
\newrgbcolor{curcolor}{0 0 0}
\pscustom[linestyle=none,fillstyle=solid,fillcolor=curcolor]
{
\newpath
\moveto(346.11632411,733.80309016)
\curveto(347.48521545,733.80309016)(348.53055066,733.50442295)(349.25232973,732.90708855)
\curveto(349.98655327,732.32219861)(350.35366504,731.41997477)(350.35366504,730.20041703)
\lineto(350.35366504,723.40573819)
\lineto(348.41232822,723.40573819)
\lineto(347.87099392,724.787074)
\lineto(347.79632712,724.787074)
\curveto(347.36077078,724.23951746)(346.90032551,723.84129452)(346.41499131,723.59240519)
\curveto(345.92965711,723.34351585)(345.26387814,723.21907118)(344.4176544,723.21907118)
\curveto(343.50920832,723.21907118)(342.75631809,723.48040499)(342.15898368,724.00307259)
\curveto(341.56164928,724.52574019)(341.26298208,725.34085277)(341.26298208,726.44841031)
\curveto(341.26298208,727.53107891)(341.64253831,728.32752479)(342.40165078,728.83774792)
\curveto(343.16076326,729.34797106)(344.29943196,729.63419379)(345.81765691,729.69641613)
\lineto(347.59099342,729.75241623)
\lineto(347.59099342,730.20041703)
\curveto(347.59099342,730.7355291)(347.44788205,731.12752981)(347.16165932,731.37641914)
\curveto(346.88788105,731.62530848)(346.50210258,731.74975314)(346.00432391,731.74975314)
\curveto(345.50654524,731.74975314)(345.02121103,731.67508634)(344.5483213,731.52575274)
\curveto(344.07543156,731.38886361)(343.60254182,731.21464107)(343.12965209,731.00308514)
\lineto(342.21498378,732.88842185)
\curveto(342.75009585,733.16220012)(343.35365249,733.37997829)(344.02565369,733.54175636)
\curveto(344.6976549,733.71597889)(345.39454504,733.80309016)(346.11632411,733.80309016)
\closepath
\moveto(347.59099342,728.12841332)
\lineto(346.50832481,728.09107992)
\curveto(345.61232321,728.06619098)(344.99009987,727.90441292)(344.6416548,727.60574571)
\curveto(344.29320973,727.30707851)(344.1189872,726.91507781)(344.1189872,726.42974361)
\curveto(344.1189872,726.00663174)(344.24343186,725.7017423)(344.4923212,725.5150753)
\curveto(344.74121053,725.34085277)(345.06476667,725.2537415)(345.4629896,725.2537415)
\curveto(346.06032401,725.2537415)(346.56432491,725.42796403)(346.97499231,725.7764091)
\curveto(347.38565972,726.13729864)(347.59099342,726.64129954)(347.59099342,727.28841181)
\closepath
}
}
{
\newrgbcolor{curcolor}{0 0 0}
\pscustom[linestyle=none,fillstyle=solid,fillcolor=curcolor]
{
\newpath
\moveto(358.9963489,733.78442346)
\curveto(360.09146197,733.78442346)(360.96879688,733.48575625)(361.62835362,732.88842185)
\curveto(362.28791035,732.30353191)(362.61768872,731.35775244)(362.61768872,730.05108343)
\lineto(362.61768872,723.40573819)
\lineto(359.8363504,723.40573819)
\lineto(359.8363504,729.36041553)
\curveto(359.8363504,730.09463906)(359.7056835,730.6421956)(359.4443497,731.00308514)
\curveto(359.1830159,731.37641914)(358.76612626,731.56308614)(358.19368079,731.56308614)
\curveto(357.34745705,731.56308614)(356.76878935,731.27064117)(356.45767768,730.68575124)
\curveto(356.14656601,730.11330576)(355.99101018,729.28574873)(355.99101018,728.20308012)
\lineto(355.99101018,723.40573819)
\lineto(353.20967186,723.40573819)
\lineto(353.20967186,733.59775646)
\lineto(355.33767567,733.59775646)
\lineto(355.71100968,732.29108745)
\lineto(355.86034328,732.29108745)
\curveto(356.18389941,732.81375505)(356.62567798,733.19331129)(357.18567899,733.42975615)
\curveto(357.75812446,733.66620102)(358.36168109,733.78442346)(358.9963489,733.78442346)
\closepath
}
}
{
\newrgbcolor{curcolor}{0 0 0}
\pscustom[linestyle=none,fillstyle=solid,fillcolor=curcolor]
{
\newpath
\moveto(368.70301839,723.21907118)
\curveto(367.57057192,723.21907118)(366.64345915,723.66084975)(365.92168008,724.54440689)
\curveto(365.21234547,725.4404085)(364.85767817,726.75329974)(364.85767817,728.48308062)
\curveto(364.85767817,730.22530597)(365.2185677,731.54441944)(365.94034678,732.44042105)
\curveto(366.66212585,733.33642265)(367.60790532,733.78442346)(368.7776852,733.78442346)
\curveto(369.51190873,733.78442346)(370.11546537,733.64131209)(370.58835511,733.35508935)
\curveto(371.06124484,733.06886662)(371.43457885,732.71419932)(371.70835712,732.29108745)
\lineto(371.80169062,732.29108745)
\curveto(371.76435722,732.49019891)(371.72080158,732.77642165)(371.67102372,733.14975565)
\curveto(371.62124585,733.53553412)(371.59635691,733.92753482)(371.59635691,734.32575776)
\lineto(371.59635691,737.59243028)
\lineto(374.37769523,737.59243028)
\lineto(374.37769523,723.40573819)
\lineto(372.24969142,723.40573819)
\lineto(371.70835712,724.73107389)
\lineto(371.59635691,724.73107389)
\curveto(371.32257865,724.30796203)(370.95546688,723.94707249)(370.49502161,723.64840529)
\curveto(370.03457634,723.36218255)(369.43724193,723.21907118)(368.70301839,723.21907118)
\closepath
\moveto(369.6736868,725.4404085)
\curveto(370.43279927,725.4404085)(370.96791134,725.6644089)(371.27902301,726.1124097)
\curveto(371.59013468,726.57285497)(371.75813498,727.25730065)(371.78302392,728.16574672)
\lineto(371.78302392,728.46441392)
\curveto(371.78302392,729.44752679)(371.62746808,730.20041703)(371.31635641,730.72308464)
\curveto(371.01768921,731.25819671)(370.45768821,731.52575274)(369.6363534,731.52575274)
\curveto(369.02657453,731.52575274)(368.54746256,731.25819671)(368.19901749,730.72308464)
\curveto(367.85057242,730.20041703)(367.67634989,729.44130456)(367.67634989,728.44574722)
\curveto(367.67634989,727.45018988)(367.85057242,726.69729964)(368.19901749,726.1870765)
\curveto(368.54746256,725.68929783)(369.039019,725.4404085)(369.6736868,725.4404085)
\closepath
}
}
{
\newrgbcolor{curcolor}{0 0 0}
\pscustom[linestyle=none,fillstyle=solid,fillcolor=curcolor]
{
\newpath
\moveto(380.07102696,723.40573819)
\lineto(377.28968865,723.40573819)
\lineto(377.28968865,737.59243028)
\lineto(380.07102696,737.59243028)
\closepath
}
}
{
\newrgbcolor{curcolor}{0 0 0}
\pscustom[linestyle=none,fillstyle=solid,fillcolor=curcolor]
{
\newpath
\moveto(387.18304022,733.78442346)
\curveto(388.58926496,733.78442346)(389.70304474,733.37997829)(390.52437954,732.57108795)
\curveto(391.34571435,731.77464208)(391.75638175,730.63597337)(391.75638175,729.15508182)
\lineto(391.75638175,727.81107942)
\lineto(385.1857033,727.81107942)
\curveto(385.21059224,727.02707801)(385.44081487,726.41107691)(385.87637121,725.9630761)
\curveto(386.32437201,725.5150753)(386.94037312,725.2910749)(387.72437452,725.2910749)
\curveto(388.37148679,725.2910749)(388.96259896,725.35329723)(389.49771103,725.4777419)
\curveto(390.04526757,725.61463103)(390.60526858,725.81996474)(391.17771405,726.093743)
\lineto(391.17771405,723.94707249)
\curveto(390.66749091,723.69818315)(390.13860107,723.51773839)(389.59104454,723.40573819)
\curveto(389.043488,723.28129352)(388.37770903,723.21907118)(387.59370762,723.21907118)
\curveto(386.57326135,723.21907118)(385.67103751,723.40573819)(384.8870361,723.77907219)
\curveto(384.1030347,724.16485066)(383.48703359,724.73729613)(383.03903279,725.4964086)
\curveto(382.59103199,726.26796554)(382.36703159,727.24485618)(382.36703159,728.42708052)
\curveto(382.36703159,729.60930486)(382.56614305,730.59863997)(382.96436599,731.39508584)
\curveto(383.37503339,732.19153171)(383.94125663,732.78886612)(384.6630357,733.18708905)
\curveto(385.38481477,733.58531199)(386.22481628,733.78442346)(387.18304022,733.78442346)
\closepath
\moveto(387.20170692,731.80575324)
\curveto(386.65415038,731.80575324)(386.20614958,731.63153071)(385.85770451,731.28308564)
\curveto(385.50925944,730.93464057)(385.30392574,730.39330627)(385.24170341,729.65908273)
\lineto(389.14304373,729.65908273)
\curveto(389.13059927,730.2688616)(388.96259896,730.77908474)(388.63904283,731.18975214)
\curveto(388.32793116,731.60041954)(387.84881919,731.80575324)(387.20170692,731.80575324)
\closepath
}
}
{
\newrgbcolor{curcolor}{0 0 0}
\pscustom[linestyle=none,fillstyle=solid,fillcolor=curcolor]
{
\newpath
\moveto(396.79638381,737.59243028)
\lineto(396.79638381,734.28842436)
\curveto(396.79638381,733.90264589)(396.78393934,733.52308965)(396.75905041,733.14975565)
\curveto(396.73416148,732.77642165)(396.70927254,732.48397668)(396.68438361,732.27242075)
\lineto(396.79638381,732.27242075)
\curveto(397.07016208,732.69553262)(397.43727385,733.05019992)(397.89771912,733.33642265)
\curveto(398.35816439,733.63508986)(398.95549879,733.78442346)(399.68972233,733.78442346)
\curveto(400.83461327,733.78442346)(401.76172604,733.33642265)(402.47106065,732.44042105)
\curveto(403.18039525,731.55686391)(403.53506256,730.2501949)(403.53506256,728.52041402)
\curveto(403.53506256,726.77818868)(403.17417302,725.4590752)(402.45239395,724.56307359)
\curveto(401.73061488,723.66707199)(400.7848354,723.21907118)(399.61505553,723.21907118)
\curveto(398.86838752,723.21907118)(398.27727535,723.34973809)(397.84171902,723.61107189)
\curveto(397.41860715,723.88485016)(397.07016208,724.18973959)(396.79638381,724.52574019)
\lineto(396.60971681,724.52574019)
\lineto(396.1430493,723.40573819)
\lineto(394.01504549,723.40573819)
\lineto(394.01504549,737.59243028)
\closepath
\moveto(398.79372072,731.56308614)
\curveto(398.07194165,731.56308614)(397.56171851,731.33286351)(397.26305131,730.87241824)
\curveto(396.96438411,730.42441743)(396.80882828,729.746194)(396.79638381,728.83774792)
\lineto(396.79638381,728.53908072)
\curveto(396.79638381,727.55596785)(396.93949518,726.79685538)(397.22571791,726.26174331)
\curveto(397.52438511,725.7390757)(398.05949718,725.4777419)(398.83105412,725.4777419)
\curveto(399.40349959,725.4777419)(399.85772263,725.7390757)(400.19372323,726.26174331)
\curveto(400.52972383,726.79685538)(400.69772414,727.56219008)(400.69772414,728.55774742)
\curveto(400.69772414,729.55330476)(400.5235016,730.29997277)(400.17505653,730.79775144)
\curveto(399.83905593,731.30797457)(399.37861066,731.56308614)(398.79372072,731.56308614)
\closepath
}
}
{
\newrgbcolor{curcolor}{0 0 0}
\pscustom[linestyle=none,fillstyle=solid,fillcolor=curcolor]
{
\newpath
\moveto(410.01239569,733.80309016)
\curveto(411.38128703,733.80309016)(412.42662224,733.50442295)(413.14840131,732.90708855)
\curveto(413.88262485,732.32219861)(414.24973662,731.41997477)(414.24973662,730.20041703)
\lineto(414.24973662,723.40573819)
\lineto(412.30839981,723.40573819)
\lineto(411.7670655,724.787074)
\lineto(411.6923987,724.787074)
\curveto(411.25684237,724.23951746)(410.7963971,723.84129452)(410.31106289,723.59240519)
\curveto(409.82572869,723.34351585)(409.15994972,723.21907118)(408.31372598,723.21907118)
\curveto(407.40527991,723.21907118)(406.65238967,723.48040499)(406.05505526,724.00307259)
\curveto(405.45772086,724.52574019)(405.15905366,725.34085277)(405.15905366,726.44841031)
\curveto(405.15905366,727.53107891)(405.53860989,728.32752479)(406.29772237,728.83774792)
\curveto(407.05683484,729.34797106)(408.19550355,729.63419379)(409.71372849,729.69641613)
\lineto(411.487065,729.75241623)
\lineto(411.487065,730.20041703)
\curveto(411.487065,730.7355291)(411.34395363,731.12752981)(411.0577309,731.37641914)
\curveto(410.78395263,731.62530848)(410.39817416,731.74975314)(409.90039549,731.74975314)
\curveto(409.40261682,731.74975314)(408.91728262,731.67508634)(408.44439288,731.52575274)
\curveto(407.97150314,731.38886361)(407.49861341,731.21464107)(407.02572367,731.00308514)
\lineto(406.11105536,732.88842185)
\curveto(406.64616744,733.16220012)(407.24972407,733.37997829)(407.92172528,733.54175636)
\curveto(408.59372648,733.71597889)(409.29061662,733.80309016)(410.01239569,733.80309016)
\closepath
\moveto(411.487065,728.12841332)
\lineto(410.40439639,728.09107992)
\curveto(409.50839479,728.06619098)(408.88617145,727.90441292)(408.53772638,727.60574571)
\curveto(408.18928131,727.30707851)(408.01505878,726.91507781)(408.01505878,726.42974361)
\curveto(408.01505878,726.00663174)(408.13950345,725.7017423)(408.38839278,725.5150753)
\curveto(408.63728212,725.34085277)(408.96083825,725.2537415)(409.35906119,725.2537415)
\curveto(409.95639559,725.2537415)(410.46039649,725.42796403)(410.8710639,725.7764091)
\curveto(411.2817313,726.13729864)(411.487065,726.64129954)(411.487065,727.28841181)
\closepath
}
}
{
\newrgbcolor{curcolor}{0 0 0}
\pscustom[linestyle=none,fillstyle=solid,fillcolor=curcolor]
{
\newpath
\moveto(422.79908698,733.78442346)
\curveto(422.93597611,733.78442346)(423.09775418,733.77820122)(423.28442118,733.76575676)
\curveto(423.47108819,733.75331229)(423.62042179,733.73464559)(423.73242199,733.70975666)
\lineto(423.52708829,731.09641864)
\curveto(423.42753255,731.12130757)(423.29686565,731.13997427)(423.13508758,731.15241874)
\curveto(422.97330952,731.17730767)(422.83019815,731.18975214)(422.70575348,731.18975214)
\curveto(422.23286374,731.18975214)(421.77864071,731.10264087)(421.34308437,730.92841834)
\curveto(420.90752803,730.76664027)(420.55286073,730.49908423)(420.27908246,730.12575023)
\curveto(420.01774866,729.75241623)(419.88708176,729.24219309)(419.88708176,728.59508082)
\lineto(419.88708176,723.40573819)
\lineto(417.10574344,723.40573819)
\lineto(417.10574344,733.59775646)
\lineto(419.21508056,733.59775646)
\lineto(419.62574796,731.88042004)
\lineto(419.75641486,731.88042004)
\curveto(420.05508206,732.40308765)(420.46574946,732.85108845)(420.98841707,733.22442245)
\curveto(421.51108467,733.59775646)(422.11464131,733.78442346)(422.79908698,733.78442346)
\closepath
}
}
{
\newrgbcolor{curcolor}{0 0 0}
\pscustom[linestyle=none,fillstyle=solid,fillcolor=curcolor]
{
\newpath
\moveto(432.69245701,726.42974361)
\curveto(432.69245701,725.39685287)(432.32534524,724.60040699)(431.5911217,724.04040599)
\curveto(430.86934263,723.49284945)(429.78667402,723.21907118)(428.34311588,723.21907118)
\curveto(427.63378128,723.21907118)(427.02400241,723.26884905)(426.51377927,723.36840479)
\curveto(426.00355613,723.45551605)(425.493333,723.60484965)(424.98310986,723.81640559)
\lineto(424.98310986,726.1124097)
\curveto(425.5306664,725.86352037)(426.12177857,725.65818667)(426.75644637,725.4964086)
\curveto(427.39111417,725.33463053)(427.95111518,725.2537415)(428.43644938,725.2537415)
\curveto(428.97156145,725.2537415)(429.35733992,725.33463053)(429.59378479,725.4964086)
\curveto(429.83022966,725.65818667)(429.94845209,725.8697426)(429.94845209,726.1310764)
\curveto(429.94845209,726.30529894)(429.89867423,726.46085477)(429.79911849,726.59774391)
\curveto(429.71200722,726.73463304)(429.51289576,726.89018888)(429.20178409,727.06441141)
\curveto(428.89067242,727.23863395)(428.40533822,727.46263435)(427.74578148,727.73641262)
\curveto(427.09866921,728.01019088)(426.56977937,728.27774692)(426.15911197,728.53908072)
\curveto(425.76088903,728.81285899)(425.46222183,729.13641512)(425.26311036,729.50974913)
\curveto(425.06399889,729.8955276)(424.96444316,730.37463957)(424.96444316,730.94708504)
\curveto(424.96444316,731.89286451)(425.33155493,732.60219912)(426.06577847,733.07508885)
\curveto(426.800002,733.54797859)(427.77689264,733.78442346)(428.99645039,733.78442346)
\curveto(429.63111819,733.78442346)(430.23467483,733.72220112)(430.8071203,733.59775646)
\curveto(431.37956577,733.47331179)(431.97067794,733.26797809)(432.58045681,732.98175535)
\lineto(431.7404553,730.98441844)
\curveto(431.24267663,731.19597437)(430.7697869,731.37019691)(430.3217861,731.50708604)
\curveto(429.87378529,731.65641964)(429.41956226,731.73108644)(428.95911699,731.73108644)
\curveto(428.13778218,731.73108644)(427.72711478,731.50708604)(427.72711478,731.05908524)
\curveto(427.72711478,730.89730717)(427.77689264,730.74797357)(427.87644838,730.61108443)
\curveto(427.98844858,730.48663977)(428.19378228,730.34975063)(428.49244948,730.20041703)
\curveto(428.80356115,730.05108343)(429.25778419,729.85197196)(429.85511859,729.60308263)
\curveto(430.44000853,729.36663776)(430.94400943,729.11774842)(431.3671213,728.85641462)
\curveto(431.79023317,728.60752529)(432.11378931,728.29019139)(432.33778971,727.90441292)
\curveto(432.57423458,727.51863445)(432.69245701,727.02707801)(432.69245701,726.42974361)
\closepath
}
}
{
\newrgbcolor{curcolor}{0 0 0}
\pscustom[linestyle=none,fillstyle=solid,fillcolor=curcolor]
{
\newpath
\moveto(216.81838279,701.7476171)
\curveto(215.30015784,701.7476171)(214.12415574,702.16450674)(213.29037646,702.99828601)
\curveto(212.46904166,703.83206528)(212.05837425,705.15740099)(212.05837425,706.97429314)
\curveto(212.05837425,708.21873981)(212.26993019,709.23296385)(212.69304206,710.01696526)
\curveto(213.11615393,710.80096666)(213.70104387,711.37963437)(214.44771187,711.75296837)
\curveto(215.20682434,712.12630237)(216.07793702,712.31296937)(217.06104989,712.31296937)
\curveto(217.75794003,712.31296937)(218.36149666,712.24452481)(218.8717198,712.10763567)
\curveto(219.3943874,711.97074654)(219.84861044,711.80896847)(220.23438891,711.62230147)
\lineto(219.4130541,709.47563095)
\curveto(218.97749777,709.64985349)(218.56683037,709.79296486)(218.1810519,709.90496506)
\curveto(217.80771789,710.01696526)(217.43438389,710.07296536)(217.06104989,710.07296536)
\curveto(215.61749175,710.07296536)(214.89571267,709.04629685)(214.89571267,706.99295984)
\curveto(214.89571267,705.97251356)(215.08237967,705.21962333)(215.45571368,704.73428912)
\curveto(215.84149215,704.24895492)(216.37660422,704.00628782)(217.06104989,704.00628782)
\curveto(217.64593983,704.00628782)(218.1623852,704.08095462)(218.610386,704.23028822)
\curveto(219.0583868,704.39206629)(219.49394314,704.60984445)(219.91705501,704.88362272)
\lineto(219.91705501,702.51295181)
\curveto(219.49394314,702.23917354)(219.04594234,702.0462843)(218.5730526,701.9342841)
\curveto(218.11260733,701.80983943)(217.52771739,701.7476171)(216.81838279,701.7476171)
\closepath
}
}
{
\newrgbcolor{curcolor}{0 0 0}
\pscustom[linestyle=none,fillstyle=solid,fillcolor=curcolor]
{
\newpath
\moveto(225.05039625,716.1209762)
\lineto(225.05039625,713.22763768)
\curveto(225.05039625,712.71741454)(225.03172955,712.23208034)(224.99439615,711.77163507)
\curveto(224.96950722,711.32363427)(224.94461828,711.00630036)(224.91972935,710.81963336)
\lineto(225.06906295,710.81963336)
\curveto(225.39261909,711.34230097)(225.80950872,711.7218572)(226.31973186,711.95830207)
\curveto(226.829955,712.19474694)(227.39617823,712.31296937)(228.01840157,712.31296937)
\curveto(229.11351464,712.31296937)(229.99707178,712.01430217)(230.66907299,711.41696777)
\curveto(231.34107419,710.83207783)(231.6770748,709.88629836)(231.6770748,708.57962935)
\lineto(231.6770748,701.9342841)
\lineto(228.89573648,701.9342841)
\lineto(228.89573648,707.88896144)
\curveto(228.89573648,709.35740852)(228.34817994,710.09163206)(227.25306687,710.09163206)
\curveto(226.41928759,710.09163206)(225.84061989,709.79918709)(225.51706375,709.21429715)
\curveto(225.20595208,708.64185168)(225.05039625,707.81429464)(225.05039625,706.73162604)
\lineto(225.05039625,701.9342841)
\lineto(222.26905793,701.9342841)
\lineto(222.26905793,716.1209762)
\closepath
}
}
{
\newrgbcolor{curcolor}{0 0 0}
\pscustom[linestyle=none,fillstyle=solid,fillcolor=curcolor]
{
\newpath
\moveto(243.79175433,707.04895994)
\curveto(243.79175433,705.35651246)(243.34375353,704.04984345)(242.44775192,703.12895291)
\curveto(241.56419478,702.20806237)(240.35708151,701.7476171)(238.8264121,701.7476171)
\curveto(237.88063262,701.7476171)(237.03440888,701.9529508)(236.28774088,702.36361821)
\curveto(235.55351734,702.77428561)(234.97484964,703.37162001)(234.55173777,704.15562142)
\curveto(234.1286259,704.95206729)(233.91706996,705.91651346)(233.91706996,707.04895994)
\curveto(233.91706996,708.74140742)(234.35884853,710.04185419)(235.24240567,710.95030026)
\curveto(236.12596281,711.85874634)(237.33929832,712.31296937)(238.8824122,712.31296937)
\curveto(239.84063614,712.31296937)(240.68685988,712.10763567)(241.42108341,711.69696827)
\curveto(242.15530695,711.28630087)(242.73397466,710.68896646)(243.15708653,709.90496506)
\curveto(243.5801984,709.12096365)(243.79175433,708.16896195)(243.79175433,707.04895994)
\closepath
\moveto(236.75440838,707.04895994)
\curveto(236.75440838,706.04095813)(236.91618645,705.27562343)(237.23974259,704.75295582)
\curveto(237.57574319,704.24273268)(238.11707749,703.98762112)(238.8637455,703.98762112)
\curveto(239.59796903,703.98762112)(240.12685887,704.24273268)(240.45041501,704.75295582)
\curveto(240.78641561,705.27562343)(240.95441591,706.04095813)(240.95441591,707.04895994)
\curveto(240.95441591,708.05696174)(240.78641561,708.80985198)(240.45041501,709.30763065)
\curveto(240.12685887,709.81785379)(239.5917468,710.07296536)(238.8450788,710.07296536)
\curveto(238.11085526,710.07296536)(237.57574319,709.81785379)(237.23974259,709.30763065)
\curveto(236.91618645,708.80985198)(236.75440838,708.05696174)(236.75440838,707.04895994)
\closepath
}
}
{
\newrgbcolor{curcolor}{0 0 0}
\pscustom[linestyle=none,fillstyle=solid,fillcolor=curcolor]
{
\newpath
\moveto(247.4877533,716.1209762)
\curveto(247.8984207,716.1209762)(248.25308801,716.02142047)(248.55175521,715.822309)
\curveto(248.85042241,715.635642)(248.99975601,715.28097469)(248.99975601,714.75830709)
\curveto(248.99975601,714.24808395)(248.85042241,713.89341665)(248.55175521,713.69430518)
\curveto(248.25308801,713.49519372)(247.8984207,713.39563798)(247.4877533,713.39563798)
\curveto(247.06464143,713.39563798)(246.7037519,713.49519372)(246.40508469,713.69430518)
\curveto(246.11886196,713.89341665)(245.97575059,714.24808395)(245.97575059,714.75830709)
\curveto(245.97575059,715.28097469)(246.11886196,715.635642)(246.40508469,715.822309)
\curveto(246.7037519,716.02142047)(247.06464143,716.1209762)(247.4877533,716.1209762)
\closepath
\moveto(248.86908911,712.12630237)
\lineto(248.86908911,701.9342841)
\lineto(246.08775079,701.9342841)
\lineto(246.08775079,712.12630237)
\closepath
}
}
{
\newrgbcolor{curcolor}{0 0 0}
\pscustom[linestyle=none,fillstyle=solid,fillcolor=curcolor]
{
\newpath
\moveto(255.92510608,701.7476171)
\curveto(254.40688114,701.7476171)(253.23087903,702.16450674)(252.39709976,702.99828601)
\curveto(251.57576495,703.83206528)(251.16509755,705.15740099)(251.16509755,706.97429314)
\curveto(251.16509755,708.21873981)(251.37665348,709.23296385)(251.79976535,710.01696526)
\curveto(252.22287722,710.80096666)(252.80776716,711.37963437)(253.55443516,711.75296837)
\curveto(254.31354764,712.12630237)(255.18466031,712.31296937)(256.16777318,712.31296937)
\curveto(256.86466332,712.31296937)(257.46821996,712.24452481)(257.97844309,712.10763567)
\curveto(258.5011107,711.97074654)(258.95533373,711.80896847)(259.3411122,711.62230147)
\lineto(258.5197774,709.47563095)
\curveto(258.08422106,709.64985349)(257.67355366,709.79296486)(257.28777519,709.90496506)
\curveto(256.91444119,710.01696526)(256.54110718,710.07296536)(256.16777318,710.07296536)
\curveto(254.72421504,710.07296536)(254.00243597,709.04629685)(254.00243597,706.99295984)
\curveto(254.00243597,705.97251356)(254.18910297,705.21962333)(254.56243697,704.73428912)
\curveto(254.94821544,704.24895492)(255.48332751,704.00628782)(256.16777318,704.00628782)
\curveto(256.75266312,704.00628782)(257.26910849,704.08095462)(257.71710929,704.23028822)
\curveto(258.16511009,704.39206629)(258.60066643,704.60984445)(259.0237783,704.88362272)
\lineto(259.0237783,702.51295181)
\curveto(258.60066643,702.23917354)(258.15266563,702.0462843)(257.67977589,701.9342841)
\curveto(257.21933062,701.80983943)(256.63444068,701.7476171)(255.92510608,701.7476171)
\closepath
}
}
{
\newrgbcolor{curcolor}{0 0 0}
\pscustom[linestyle=none,fillstyle=solid,fillcolor=curcolor]
{
\newpath
\moveto(265.57579066,712.31296937)
\curveto(266.9820154,712.31296937)(268.09579518,711.9085242)(268.91712998,711.09963387)
\curveto(269.73846479,710.30318799)(270.14913219,709.16451929)(270.14913219,707.68362774)
\lineto(270.14913219,706.33962533)
\lineto(263.57845375,706.33962533)
\curveto(263.60334268,705.55562393)(263.83356531,704.93962282)(264.26912165,704.49162202)
\curveto(264.71712245,704.04362122)(265.33312356,703.81962082)(266.11712496,703.81962082)
\curveto(266.76423723,703.81962082)(267.35534941,703.88184315)(267.89046148,704.00628782)
\curveto(268.43801801,704.14317695)(268.99801902,704.34851065)(269.57046449,704.62228892)
\lineto(269.57046449,702.47561841)
\curveto(269.06024135,702.22672907)(268.53135151,702.0462843)(267.98379498,701.9342841)
\curveto(267.43623844,701.80983943)(266.77045947,701.7476171)(265.98645806,701.7476171)
\curveto(264.96601179,701.7476171)(264.06378795,701.9342841)(263.27978654,702.30761811)
\curveto(262.49578514,702.69339657)(261.87978403,703.26584204)(261.43178323,704.02495452)
\curveto(260.98378243,704.79651146)(260.75978203,705.7734021)(260.75978203,706.95562644)
\curveto(260.75978203,708.13785078)(260.95889349,709.12718589)(261.35711643,709.92363176)
\curveto(261.76778383,710.72007763)(262.33400707,711.31741203)(263.05578614,711.71563497)
\curveto(263.77756521,712.11385791)(264.61756672,712.31296937)(265.57579066,712.31296937)
\closepath
\moveto(265.59445736,710.33429916)
\curveto(265.04690082,710.33429916)(264.59890002,710.16007663)(264.25045495,709.81163156)
\curveto(263.90200988,709.46318649)(263.69667618,708.92185218)(263.63445385,708.18762865)
\lineto(267.53579417,708.18762865)
\curveto(267.52334971,708.79740752)(267.35534941,709.30763065)(267.03179327,709.71829806)
\curveto(266.7206816,710.12896546)(266.24156963,710.33429916)(265.59445736,710.33429916)
\closepath
}
}
{
\newrgbcolor{curcolor}{0 0 0}
\pscustom[linestyle=none,fillstyle=solid,fillcolor=curcolor]
{
\newpath
\moveto(278.28780647,713.09697078)
\curveto(277.20513786,713.09697078)(276.37758083,712.69252561)(275.80513535,711.88363527)
\curveto(275.23268988,711.07474493)(274.94646715,709.96718739)(274.94646715,708.56096265)
\curveto(274.94646715,707.14229344)(275.20780095,706.04095813)(275.73046855,705.25695673)
\curveto(276.26558062,704.48539979)(277.1180266,704.09962132)(278.28780647,704.09962132)
\curveto(278.82291854,704.09962132)(279.36425285,704.16184365)(279.91180938,704.28628832)
\curveto(280.45936592,704.41073299)(281.05047809,704.58495552)(281.6851459,704.80895592)
\lineto(281.6851459,702.43828501)
\curveto(281.10025596,702.20184014)(280.52158825,702.0276176)(279.94914278,701.9156174)
\curveto(279.37669731,701.8036172)(278.73580727,701.7476171)(278.02647267,701.7476171)
\curveto(276.64513686,701.7476171)(275.51269039,702.0276176)(274.62913325,702.58761861)
\curveto(273.74557611,703.16006408)(273.0922416,703.95650995)(272.66912973,704.97695622)
\curveto(272.24601786,706.00984696)(272.03446193,707.21073801)(272.03446193,708.57962935)
\curveto(272.03446193,709.92363176)(272.27712903,711.11207833)(272.76246323,712.14496907)
\curveto(273.24779744,713.17785981)(273.95090981,713.98675015)(274.87180035,714.57164009)
\curveto(275.80513535,715.15653003)(276.94380406,715.448975)(278.28780647,715.448975)
\curveto(278.94736321,715.448975)(279.60691995,715.36186373)(280.26647669,715.18764119)
\curveto(280.93847789,715.02586313)(281.57936793,714.80186272)(282.1891468,714.51563999)
\lineto(281.27447849,712.21963587)
\curveto(280.77669982,712.45608074)(280.27269892,712.66141444)(279.76247578,712.83563698)
\curveto(279.26469711,713.00985951)(278.77314068,713.09697078)(278.28780647,713.09697078)
\closepath
}
}
{
\newrgbcolor{curcolor}{0 0 0}
\pscustom[linestyle=none,fillstyle=solid,fillcolor=curcolor]
{
\newpath
\moveto(289.9918101,712.31296937)
\curveto(290.12869923,712.31296937)(290.2904773,712.30674714)(290.4771443,712.29430267)
\curveto(290.6638113,712.28185821)(290.8131449,712.26319151)(290.9251451,712.23830257)
\lineto(290.7198114,709.62496456)
\curveto(290.62025567,709.64985349)(290.48958877,709.66852019)(290.3278107,709.68096466)
\curveto(290.16603263,709.70585359)(290.02292126,709.71829806)(289.8984766,709.71829806)
\curveto(289.42558686,709.71829806)(288.97136382,709.63118679)(288.53580749,709.45696425)
\curveto(288.10025115,709.29518619)(287.74558385,709.02763015)(287.47180558,708.65429615)
\curveto(287.21047178,708.28096215)(287.07980488,707.77073901)(287.07980488,707.12362674)
\lineto(287.07980488,701.9342841)
\lineto(284.29846656,701.9342841)
\lineto(284.29846656,712.12630237)
\lineto(286.40780367,712.12630237)
\lineto(286.81847108,710.40896596)
\lineto(286.94913798,710.40896596)
\curveto(287.24780518,710.93163356)(287.65847258,711.37963437)(288.18114018,711.75296837)
\curveto(288.70380779,712.12630237)(289.30736443,712.31296937)(289.9918101,712.31296937)
\closepath
}
}
{
\newrgbcolor{curcolor}{0 0 0}
\pscustom[linestyle=none,fillstyle=solid,fillcolor=curcolor]
{
\newpath
\moveto(296.59982286,712.31296937)
\curveto(298.0060476,712.31296937)(299.11982738,711.9085242)(299.94116218,711.09963387)
\curveto(300.76249699,710.30318799)(301.17316439,709.16451929)(301.17316439,707.68362774)
\lineto(301.17316439,706.33962533)
\lineto(294.60248595,706.33962533)
\curveto(294.62737488,705.55562393)(294.85759752,704.93962282)(295.29315385,704.49162202)
\curveto(295.74115466,704.04362122)(296.35715576,703.81962082)(297.14115717,703.81962082)
\curveto(297.78826944,703.81962082)(298.37938161,703.88184315)(298.91449368,704.00628782)
\curveto(299.46205021,704.14317695)(300.02205122,704.34851065)(300.59449669,704.62228892)
\lineto(300.59449669,702.47561841)
\curveto(300.08427355,702.22672907)(299.55538372,702.0462843)(299.00782718,701.9342841)
\curveto(298.46027064,701.80983943)(297.79449167,701.7476171)(297.01049026,701.7476171)
\curveto(295.99004399,701.7476171)(295.08782015,701.9342841)(294.30381875,702.30761811)
\curveto(293.51981734,702.69339657)(292.90381624,703.26584204)(292.45581543,704.02495452)
\curveto(292.00781463,704.79651146)(291.78381423,705.7734021)(291.78381423,706.95562644)
\curveto(291.78381423,708.13785078)(291.9829257,709.12718589)(292.38114863,709.92363176)
\curveto(292.79181604,710.72007763)(293.35803927,711.31741203)(294.07981834,711.71563497)
\curveto(294.80159742,712.11385791)(295.64159892,712.31296937)(296.59982286,712.31296937)
\closepath
\moveto(296.61848956,710.33429916)
\curveto(296.07093302,710.33429916)(295.62293222,710.16007663)(295.27448715,709.81163156)
\curveto(294.92604208,709.46318649)(294.72070838,708.92185218)(294.65848605,708.18762865)
\lineto(298.55982638,708.18762865)
\curveto(298.54738191,708.79740752)(298.37938161,709.30763065)(298.05582547,709.71829806)
\curveto(297.7447138,710.12896546)(297.26560183,710.33429916)(296.61848956,710.33429916)
\closepath
}
}
{
\newrgbcolor{curcolor}{0 0 0}
\pscustom[linestyle=none,fillstyle=solid,fillcolor=curcolor]
{
\newpath
\moveto(307.61316896,712.33163607)
\curveto(308.9820603,712.33163607)(310.02739551,712.03296887)(310.74917458,711.43563447)
\curveto(311.48339812,710.85074453)(311.85050989,709.94852069)(311.85050989,708.72896295)
\lineto(311.85050989,701.9342841)
\lineto(309.90917308,701.9342841)
\lineto(309.36783877,703.31561991)
\lineto(309.29317197,703.31561991)
\curveto(308.85761564,702.76806337)(308.39717037,702.36984044)(307.91183616,702.1209511)
\curveto(307.42650196,701.87206177)(306.76072299,701.7476171)(305.91449925,701.7476171)
\curveto(305.00605318,701.7476171)(304.25316294,702.0089509)(303.65582854,702.53161851)
\curveto(303.05849413,703.05428611)(302.75982693,703.86939868)(302.75982693,704.97695622)
\curveto(302.75982693,706.05962483)(303.13938316,706.8560707)(303.89849564,707.36629384)
\curveto(304.65760811,707.87651698)(305.79627682,708.16273971)(307.31450176,708.22496205)
\lineto(309.08783827,708.28096215)
\lineto(309.08783827,708.72896295)
\curveto(309.08783827,709.26407502)(308.9447269,709.65607572)(308.65850417,709.90496506)
\curveto(308.3847259,710.15385439)(307.99894743,710.27829906)(307.50116876,710.27829906)
\curveto(307.00339009,710.27829906)(306.51805589,710.20363226)(306.04516615,710.05429866)
\curveto(305.57227641,709.91740952)(305.09938668,709.74318699)(304.62649694,709.53163105)
\lineto(303.71182864,711.41696777)
\curveto(304.24694071,711.69074604)(304.85049734,711.9085242)(305.52249855,712.07030227)
\curveto(306.19449975,712.24452481)(306.89138989,712.33163607)(307.61316896,712.33163607)
\closepath
\moveto(309.08783827,706.65695923)
\lineto(308.00516966,706.61962583)
\curveto(307.10916806,706.5947369)(306.48694472,706.43295883)(306.13849965,706.13429163)
\curveto(305.79005458,705.83562443)(305.61583205,705.44362373)(305.61583205,704.95828952)
\curveto(305.61583205,704.53517765)(305.74027672,704.23028822)(305.98916605,704.04362122)
\curveto(306.23805539,703.86939868)(306.56161152,703.78228742)(306.95983446,703.78228742)
\curveto(307.55716886,703.78228742)(308.06116977,703.95650995)(308.47183717,704.30495502)
\curveto(308.88250457,704.66584455)(309.08783827,705.16984546)(309.08783827,705.81695773)
\closepath
}
}
{
\newrgbcolor{curcolor}{0 0 0}
\pscustom[linestyle=none,fillstyle=solid,fillcolor=curcolor]
{
\newpath
\moveto(318.99985774,703.96895442)
\curveto(319.31096941,703.96895442)(319.60963661,703.99384335)(319.89585935,704.04362122)
\curveto(320.18208208,704.10584355)(320.46830482,704.18673258)(320.75452755,704.28628832)
\lineto(320.75452755,702.2142846)
\curveto(320.45586035,702.07739547)(320.08252635,701.96539527)(319.63452555,701.878284)
\curveto(319.19896921,701.79117273)(318.71985724,701.7476171)(318.19718964,701.7476171)
\curveto(317.58741077,701.7476171)(317.03985423,701.84717284)(316.55452002,702.0462843)
\curveto(316.08163029,702.24539577)(315.70207405,702.58761861)(315.41585132,703.07295281)
\curveto(315.14207305,703.55828701)(315.00518391,704.24273268)(315.00518391,705.12628982)
\lineto(315.00518391,710.03563196)
\lineto(313.67984821,710.03563196)
\lineto(313.67984821,711.21163407)
\lineto(315.21051762,712.14496907)
\lineto(316.01318572,714.29163959)
\lineto(317.78652223,714.29163959)
\lineto(317.78652223,712.12630237)
\lineto(320.64252735,712.12630237)
\lineto(320.64252735,710.03563196)
\lineto(317.78652223,710.03563196)
\lineto(317.78652223,705.12628982)
\curveto(317.78652223,704.74051136)(317.89852243,704.44806639)(318.12252284,704.24895492)
\curveto(318.34652324,704.06228792)(318.63896821,703.96895442)(318.99985774,703.96895442)
\closepath
}
}
{
\newrgbcolor{curcolor}{0 0 0}
\pscustom[linestyle=none,fillstyle=solid,fillcolor=curcolor]
{
\newpath
\moveto(327.00787302,712.31296937)
\curveto(328.41409776,712.31296937)(329.52787753,711.9085242)(330.34921234,711.09963387)
\curveto(331.17054714,710.30318799)(331.58121455,709.16451929)(331.58121455,707.68362774)
\lineto(331.58121455,706.33962533)
\lineto(325.0105361,706.33962533)
\curveto(325.03542504,705.55562393)(325.26564767,704.93962282)(325.70120401,704.49162202)
\curveto(326.14920481,704.04362122)(326.76520591,703.81962082)(327.54920732,703.81962082)
\curveto(328.19631959,703.81962082)(328.78743176,703.88184315)(329.32254383,704.00628782)
\curveto(329.87010037,704.14317695)(330.43010137,704.34851065)(331.00254684,704.62228892)
\lineto(331.00254684,702.47561841)
\curveto(330.49232371,702.22672907)(329.96343387,702.0462843)(329.41587733,701.9342841)
\curveto(328.8683208,701.80983943)(328.20254182,701.7476171)(327.41854042,701.7476171)
\curveto(326.39809414,701.7476171)(325.49587031,701.9342841)(324.7118689,702.30761811)
\curveto(323.92786749,702.69339657)(323.31186639,703.26584204)(322.86386559,704.02495452)
\curveto(322.41586478,704.79651146)(322.19186438,705.7734021)(322.19186438,706.95562644)
\curveto(322.19186438,708.13785078)(322.39097585,709.12718589)(322.78919879,709.92363176)
\curveto(323.19986619,710.72007763)(323.76608943,711.31741203)(324.4878685,711.71563497)
\curveto(325.20964757,712.11385791)(326.04964908,712.31296937)(327.00787302,712.31296937)
\closepath
\moveto(327.02653972,710.33429916)
\curveto(326.47898318,710.33429916)(326.03098238,710.16007663)(325.68253731,709.81163156)
\curveto(325.33409224,709.46318649)(325.12875854,708.92185218)(325.0665362,708.18762865)
\lineto(328.96787653,708.18762865)
\curveto(328.95543206,708.79740752)(328.78743176,709.30763065)(328.46387563,709.71829806)
\curveto(328.15276396,710.12896546)(327.67365199,710.33429916)(327.02653972,710.33429916)
\closepath
}
}
{
\newrgbcolor{curcolor}{0 0 0}
\pscustom[linestyle=none,fillstyle=solid,fillcolor=curcolor]
{
\newpath
\moveto(337.94655232,715.26230799)
\curveto(339.76344446,715.26230799)(341.10122464,714.93252962)(341.95989284,714.27297289)
\curveto(342.83100552,713.61341615)(343.26656185,712.61163658)(343.26656185,711.26763417)
\curveto(343.26656185,710.6578553)(343.14833942,710.12274323)(342.91189455,709.66229796)
\curveto(342.68789415,709.21429715)(342.38300471,708.82851868)(341.99722624,708.50496255)
\curveto(341.62389224,708.19385088)(341.21944707,707.93873931)(340.78389074,707.73962784)
\lineto(344.70389776,701.9342841)
\lineto(341.56789214,701.9342841)
\lineto(338.39455312,707.04895994)
\lineto(336.88255041,707.04895994)
\lineto(336.88255041,701.9342841)
\lineto(334.06387869,701.9342841)
\lineto(334.06387869,715.26230799)
\closepath
\moveto(337.74121861,712.94763718)
\lineto(336.88255041,712.94763718)
\lineto(336.88255041,709.34496405)
\lineto(337.79721871,709.34496405)
\curveto(338.73055372,709.34496405)(339.39633269,709.50051989)(339.79455563,709.81163156)
\curveto(340.20522303,710.12274323)(340.41055673,710.5831885)(340.41055673,711.19296737)
\curveto(340.41055673,711.82763517)(340.19277856,712.27563597)(339.75722223,712.53696978)
\curveto(339.32166589,712.81074804)(338.64966469,712.94763718)(337.74121861,712.94763718)
\closepath
}
}
{
\newrgbcolor{curcolor}{0 0 0}
\pscustom[linestyle=none,fillstyle=solid,fillcolor=curcolor]
{
\newpath
\moveto(355.41857635,707.04895994)
\curveto(355.41857635,705.35651246)(354.97057555,704.04984345)(354.07457394,703.12895291)
\curveto(353.1910168,702.20806237)(351.98390353,701.7476171)(350.45323412,701.7476171)
\curveto(349.50745464,701.7476171)(348.6612309,701.9529508)(347.9145629,702.36361821)
\curveto(347.18033936,702.77428561)(346.60167166,703.37162001)(346.17855979,704.15562142)
\curveto(345.75544792,704.95206729)(345.54389198,705.91651346)(345.54389198,707.04895994)
\curveto(345.54389198,708.74140742)(345.98567055,710.04185419)(346.86922769,710.95030026)
\curveto(347.75278483,711.85874634)(348.96612034,712.31296937)(350.50923422,712.31296937)
\curveto(351.46745816,712.31296937)(352.31368189,712.10763567)(353.04790543,711.69696827)
\curveto(353.78212897,711.28630087)(354.36079668,710.68896646)(354.78390855,709.90496506)
\curveto(355.20702041,709.12096365)(355.41857635,708.16896195)(355.41857635,707.04895994)
\closepath
\moveto(348.3812304,707.04895994)
\curveto(348.3812304,706.04095813)(348.54300847,705.27562343)(348.8665646,704.75295582)
\curveto(349.20256521,704.24273268)(349.74389951,703.98762112)(350.49056752,703.98762112)
\curveto(351.22479105,703.98762112)(351.75368089,704.24273268)(352.07723703,704.75295582)
\curveto(352.41323763,705.27562343)(352.58123793,706.04095813)(352.58123793,707.04895994)
\curveto(352.58123793,708.05696174)(352.41323763,708.80985198)(352.07723703,709.30763065)
\curveto(351.75368089,709.81785379)(351.21856882,710.07296536)(350.47190082,710.07296536)
\curveto(349.73767728,710.07296536)(349.20256521,709.81785379)(348.8665646,709.30763065)
\curveto(348.54300847,708.80985198)(348.3812304,708.05696174)(348.3812304,707.04895994)
\closepath
}
}
{
\newrgbcolor{curcolor}{0 0 0}
\pscustom[linestyle=none,fillstyle=solid,fillcolor=curcolor]
{
\newpath
\moveto(366.9732637,707.04895994)
\curveto(366.9732637,705.35651246)(366.5252629,704.04984345)(365.62926129,703.12895291)
\curveto(364.74570416,702.20806237)(363.53859088,701.7476171)(362.00792147,701.7476171)
\curveto(361.062142,701.7476171)(360.21591826,701.9529508)(359.46925025,702.36361821)
\curveto(358.73502671,702.77428561)(358.15635901,703.37162001)(357.73324714,704.15562142)
\curveto(357.31013527,704.95206729)(357.09857934,705.91651346)(357.09857934,707.04895994)
\curveto(357.09857934,708.74140742)(357.54035791,710.04185419)(358.42391505,710.95030026)
\curveto(359.30747218,711.85874634)(360.52080769,712.31296937)(362.06392157,712.31296937)
\curveto(363.02214551,712.31296937)(363.86836925,712.10763567)(364.60259279,711.69696827)
\curveto(365.33681633,711.28630087)(365.91548403,710.68896646)(366.3385959,709.90496506)
\curveto(366.76170777,709.12096365)(366.9732637,708.16896195)(366.9732637,707.04895994)
\closepath
\moveto(359.93591776,707.04895994)
\curveto(359.93591776,706.04095813)(360.09769582,705.27562343)(360.42125196,704.75295582)
\curveto(360.75725256,704.24273268)(361.29858686,703.98762112)(362.04525487,703.98762112)
\curveto(362.77947841,703.98762112)(363.30836825,704.24273268)(363.63192438,704.75295582)
\curveto(363.96792498,705.27562343)(364.13592528,706.04095813)(364.13592528,707.04895994)
\curveto(364.13592528,708.05696174)(363.96792498,708.80985198)(363.63192438,709.30763065)
\curveto(363.30836825,709.81785379)(362.77325618,710.07296536)(362.02658817,710.07296536)
\curveto(361.29236463,710.07296536)(360.75725256,709.81785379)(360.42125196,709.30763065)
\curveto(360.09769582,708.80985198)(359.93591776,708.05696174)(359.93591776,707.04895994)
\closepath
}
}
{
\newrgbcolor{curcolor}{0 0 0}
\pscustom[linestyle=none,fillstyle=solid,fillcolor=curcolor]
{
\newpath
\moveto(381.23460732,712.31296937)
\curveto(382.39194273,712.31296937)(383.2630554,712.01430217)(383.84794534,711.41696777)
\curveto(384.44527974,710.83207783)(384.74394694,709.88629836)(384.74394694,708.57962935)
\lineto(384.74394694,701.9342841)
\lineto(381.96260862,701.9342841)
\lineto(381.96260862,707.88896144)
\curveto(381.96260862,709.35740852)(381.45238549,710.09163206)(380.43193921,710.09163206)
\curveto(379.69771567,710.09163206)(379.17504807,709.83029826)(378.8639364,709.30763065)
\curveto(378.55282473,708.78496305)(378.3972689,708.03207281)(378.3972689,707.04895994)
\lineto(378.3972689,701.9342841)
\lineto(375.61593058,701.9342841)
\lineto(375.61593058,707.88896144)
\curveto(375.61593058,709.35740852)(375.10570744,710.09163206)(374.08526117,710.09163206)
\curveto(373.31370423,710.09163206)(372.77859216,709.79918709)(372.47992496,709.21429715)
\curveto(372.19370222,708.64185168)(372.05059086,707.81429464)(372.05059086,706.73162604)
\lineto(372.05059086,701.9342841)
\lineto(369.26925254,701.9342841)
\lineto(369.26925254,712.12630237)
\lineto(371.39725635,712.12630237)
\lineto(371.77059035,710.81963336)
\lineto(371.91992395,710.81963336)
\curveto(372.23103562,711.34230097)(372.65414749,711.7218572)(373.18925956,711.95830207)
\curveto(373.7368161,712.19474694)(374.30303934,712.31296937)(374.88792927,712.31296937)
\curveto(375.63459728,712.31296937)(376.26304285,712.18852471)(376.77326599,711.93963537)
\curveto(377.29593359,711.7031905)(377.70037876,711.3298565)(377.9866015,710.81963336)
\lineto(378.2292686,710.81963336)
\curveto(378.54038027,711.34230097)(378.96971437,711.7218572)(379.51727091,711.95830207)
\curveto(380.07727191,712.19474694)(380.64971738,712.31296937)(381.23460732,712.31296937)
\closepath
}
}
{
\newrgbcolor{curcolor}{0 0 0}
\pscustom[linestyle=none,fillstyle=solid,fillcolor=curcolor]
{
\newpath
\moveto(387.20795534,703.24095311)
\curveto(387.20795534,703.81339858)(387.36351117,704.21162152)(387.67462284,704.43562192)
\curveto(387.98573451,704.67206679)(388.36529075,704.79028922)(388.81329155,704.79028922)
\curveto(389.24884789,704.79028922)(389.62218189,704.67206679)(389.93329356,704.43562192)
\curveto(390.24440523,704.21162152)(390.39996106,703.81339858)(390.39996106,703.24095311)
\curveto(390.39996106,702.69339657)(390.24440523,702.29517364)(389.93329356,702.0462843)
\curveto(389.62218189,701.80983943)(389.24884789,701.691617)(388.81329155,701.691617)
\curveto(388.36529075,701.691617)(387.98573451,701.80983943)(387.67462284,702.0462843)
\curveto(387.36351117,702.29517364)(387.20795534,702.69339657)(387.20795534,703.24095311)
\closepath
}
}
{
\newrgbcolor{curcolor}{0 0 0}
\pscustom[linestyle=none,fillstyle=solid,fillcolor=curcolor]
{
\newpath
\moveto(395.70130162,716.1209762)
\lineto(395.70130162,713.22763768)
\curveto(395.70130162,712.71741454)(395.68263492,712.23208034)(395.64530152,711.77163507)
\curveto(395.62041258,711.32363427)(395.59552365,711.00630036)(395.57063472,710.81963336)
\lineto(395.71996832,710.81963336)
\curveto(396.04352445,711.34230097)(396.46041409,711.7218572)(396.97063723,711.95830207)
\curveto(397.48086036,712.19474694)(398.0470836,712.31296937)(398.66930694,712.31296937)
\curveto(399.76442001,712.31296937)(400.64797715,712.01430217)(401.31997836,711.41696777)
\curveto(401.99197956,710.83207783)(402.32798016,709.88629836)(402.32798016,708.57962935)
\lineto(402.32798016,701.9342841)
\lineto(399.54664184,701.9342841)
\lineto(399.54664184,707.88896144)
\curveto(399.54664184,709.35740852)(398.99908531,710.09163206)(397.90397223,710.09163206)
\curveto(397.07019296,710.09163206)(396.49152526,709.79918709)(396.16796912,709.21429715)
\curveto(395.85685745,708.64185168)(395.70130162,707.81429464)(395.70130162,706.73162604)
\lineto(395.70130162,701.9342841)
\lineto(392.9199633,701.9342841)
\lineto(392.9199633,716.1209762)
\closepath
}
}
{
\newrgbcolor{curcolor}{0 0 0}
\pscustom[linestyle=none,fillstyle=solid,fillcolor=curcolor]
{
\newpath
\moveto(409.36531154,712.33163607)
\curveto(410.73420288,712.33163607)(411.77953809,712.03296887)(412.50131716,711.43563447)
\curveto(413.2355407,710.85074453)(413.60265247,709.94852069)(413.60265247,708.72896295)
\lineto(413.60265247,701.9342841)
\lineto(411.66131566,701.9342841)
\lineto(411.11998135,703.31561991)
\lineto(411.04531455,703.31561991)
\curveto(410.60975822,702.76806337)(410.14931295,702.36984044)(409.66397874,702.1209511)
\curveto(409.17864454,701.87206177)(408.51286557,701.7476171)(407.66664183,701.7476171)
\curveto(406.75819576,701.7476171)(406.00530552,702.0089509)(405.40797111,702.53161851)
\curveto(404.81063671,703.05428611)(404.51196951,703.86939868)(404.51196951,704.97695622)
\curveto(404.51196951,706.05962483)(404.89152574,706.8560707)(405.65063821,707.36629384)
\curveto(406.40975069,707.87651698)(407.54841939,708.16273971)(409.06664434,708.22496205)
\lineto(410.83998085,708.28096215)
\lineto(410.83998085,708.72896295)
\curveto(410.83998085,709.26407502)(410.69686948,709.65607572)(410.41064675,709.90496506)
\curveto(410.13686848,710.15385439)(409.75109001,710.27829906)(409.25331134,710.27829906)
\curveto(408.75553267,710.27829906)(408.27019847,710.20363226)(407.79730873,710.05429866)
\curveto(407.32441899,709.91740952)(406.85152926,709.74318699)(406.37863952,709.53163105)
\lineto(405.46397121,711.41696777)
\curveto(405.99908328,711.69074604)(406.60263992,711.9085242)(407.27464113,712.07030227)
\curveto(407.94664233,712.24452481)(408.64353247,712.33163607)(409.36531154,712.33163607)
\closepath
\moveto(410.83998085,706.65695923)
\lineto(409.75731224,706.61962583)
\curveto(408.86131064,706.5947369)(408.2390873,706.43295883)(407.89064223,706.13429163)
\curveto(407.54219716,705.83562443)(407.36797463,705.44362373)(407.36797463,704.95828952)
\curveto(407.36797463,704.53517765)(407.49241929,704.23028822)(407.74130863,704.04362122)
\curveto(407.99019796,703.86939868)(408.3137541,703.78228742)(408.71197704,703.78228742)
\curveto(409.30931144,703.78228742)(409.81331234,703.95650995)(410.22397975,704.30495502)
\curveto(410.63464715,704.66584455)(410.83998085,705.16984546)(410.83998085,705.81695773)
\closepath
}
}
{
\newrgbcolor{curcolor}{0 0 0}
\pscustom[linestyle=none,fillstyle=solid,fillcolor=curcolor]
{
\newpath
\moveto(422.24533633,712.31296937)
\curveto(423.3404494,712.31296937)(424.21778431,712.01430217)(424.87734105,711.41696777)
\curveto(425.53689779,710.83207783)(425.86667615,709.88629836)(425.86667615,708.57962935)
\lineto(425.86667615,701.9342841)
\lineto(423.08533784,701.9342841)
\lineto(423.08533784,707.88896144)
\curveto(423.08533784,708.62318498)(422.95467093,709.17074152)(422.69333713,709.53163105)
\curveto(422.43200333,709.90496506)(422.01511369,710.09163206)(421.44266822,710.09163206)
\curveto(420.59644449,710.09163206)(420.01777678,709.79918709)(419.70666511,709.21429715)
\curveto(419.39555344,708.64185168)(419.23999761,707.81429464)(419.23999761,706.73162604)
\lineto(419.23999761,701.9342841)
\lineto(416.45865929,701.9342841)
\lineto(416.45865929,712.12630237)
\lineto(418.5866631,712.12630237)
\lineto(418.95999711,710.81963336)
\lineto(419.10933071,710.81963336)
\curveto(419.43288684,711.34230097)(419.87466541,711.7218572)(420.43466642,711.95830207)
\curveto(421.00711189,712.19474694)(421.61066853,712.31296937)(422.24533633,712.31296937)
\closepath
}
}
{
\newrgbcolor{curcolor}{0 0 0}
\pscustom[linestyle=none,fillstyle=solid,fillcolor=curcolor]
{
\newpath
\moveto(431.95202108,701.7476171)
\curveto(430.81957461,701.7476171)(429.89246184,702.18939567)(429.17068277,703.07295281)
\curveto(428.46134816,703.96895442)(428.10668086,705.28184566)(428.10668086,707.01162654)
\curveto(428.10668086,708.75385188)(428.46757039,710.07296536)(429.18934947,710.96896696)
\curveto(429.91112854,711.86496857)(430.85690801,712.31296937)(432.02668789,712.31296937)
\curveto(432.76091142,712.31296937)(433.36446806,712.16985801)(433.8373578,711.88363527)
\curveto(434.31024753,711.59741254)(434.68358154,711.24274523)(434.95735981,710.81963336)
\lineto(435.05069331,710.81963336)
\curveto(435.01335991,711.01874483)(434.96980427,711.30496757)(434.92002641,711.67830157)
\curveto(434.87024854,712.06408004)(434.8453596,712.45608074)(434.8453596,712.85430368)
\lineto(434.8453596,716.1209762)
\lineto(437.62669792,716.1209762)
\lineto(437.62669792,701.9342841)
\lineto(435.49869411,701.9342841)
\lineto(434.95735981,703.25961981)
\lineto(434.8453596,703.25961981)
\curveto(434.57158134,702.83650794)(434.20446957,702.47561841)(433.7440243,702.1769512)
\curveto(433.28357903,701.89072847)(432.68624462,701.7476171)(431.95202108,701.7476171)
\closepath
\moveto(432.92268949,703.96895442)
\curveto(433.68180196,703.96895442)(434.21691403,704.19295482)(434.5280257,704.64095562)
\curveto(434.83913737,705.10140089)(435.00713767,705.78584656)(435.03202661,706.69429264)
\lineto(435.03202661,706.99295984)
\curveto(435.03202661,707.97607271)(434.87647077,708.72896295)(434.5653591,709.25163055)
\curveto(434.2666919,709.78674262)(433.7066909,710.05429866)(432.88535609,710.05429866)
\curveto(432.27557722,710.05429866)(431.79646525,709.78674262)(431.44802018,709.25163055)
\curveto(431.09957511,708.72896295)(430.92535258,707.96985048)(430.92535258,706.97429314)
\curveto(430.92535258,705.9787358)(431.09957511,705.22584556)(431.44802018,704.71562242)
\curveto(431.79646525,704.21784375)(432.28802169,703.96895442)(432.92268949,703.96895442)
\closepath
}
}
{
\newrgbcolor{curcolor}{0 0 0}
\pscustom[linestyle=none,fillstyle=solid,fillcolor=curcolor]
{
\newpath
\moveto(443.32002965,701.9342841)
\lineto(440.53869134,701.9342841)
\lineto(440.53869134,716.1209762)
\lineto(443.32002965,716.1209762)
\closepath
}
}
{
\newrgbcolor{curcolor}{0 0 0}
\pscustom[linestyle=none,fillstyle=solid,fillcolor=curcolor]
{
\newpath
\moveto(450.43204291,712.31296937)
\curveto(451.83826765,712.31296937)(452.95204743,711.9085242)(453.77338223,711.09963387)
\curveto(454.59471704,710.30318799)(455.00538444,709.16451929)(455.00538444,707.68362774)
\lineto(455.00538444,706.33962533)
\lineto(448.43470599,706.33962533)
\curveto(448.45959493,705.55562393)(448.68981756,704.93962282)(449.1253739,704.49162202)
\curveto(449.5733747,704.04362122)(450.18937581,703.81962082)(450.97337721,703.81962082)
\curveto(451.62048948,703.81962082)(452.21160165,703.88184315)(452.74671372,704.00628782)
\curveto(453.29427026,704.14317695)(453.85427127,704.34851065)(454.42671674,704.62228892)
\lineto(454.42671674,702.47561841)
\curveto(453.9164936,702.22672907)(453.38760376,702.0462843)(452.84004723,701.9342841)
\curveto(452.29249069,701.80983943)(451.62671172,701.7476171)(450.84271031,701.7476171)
\curveto(449.82226404,701.7476171)(448.9200402,701.9342841)(448.13603879,702.30761811)
\curveto(447.35203739,702.69339657)(446.73603628,703.26584204)(446.28803548,704.02495452)
\curveto(445.84003468,704.79651146)(445.61603428,705.7734021)(445.61603428,706.95562644)
\curveto(445.61603428,708.13785078)(445.81514574,709.12718589)(446.21336868,709.92363176)
\curveto(446.62403608,710.72007763)(447.19025932,711.31741203)(447.91203839,711.71563497)
\curveto(448.63381746,712.11385791)(449.47381897,712.31296937)(450.43204291,712.31296937)
\closepath
\moveto(450.45070961,710.33429916)
\curveto(449.90315307,710.33429916)(449.45515227,710.16007663)(449.1067072,709.81163156)
\curveto(448.75826213,709.46318649)(448.55292843,708.92185218)(448.4907061,708.18762865)
\lineto(452.39204642,708.18762865)
\curveto(452.37960196,708.79740752)(452.21160165,709.30763065)(451.88804552,709.71829806)
\curveto(451.57693385,710.12896546)(451.09782188,710.33429916)(450.45070961,710.33429916)
\closepath
}
}
{
\newrgbcolor{curcolor}{0 0 0}
\pscustom[linestyle=none,fillstyle=solid,fillcolor=curcolor]
{
\newpath
\moveto(460.0453865,716.1209762)
\lineto(460.0453865,712.81697028)
\curveto(460.0453865,712.43119181)(460.03294203,712.05163557)(460.0080531,711.67830157)
\curveto(459.98316417,711.30496757)(459.95827523,711.0125226)(459.9333863,710.80096666)
\lineto(460.0453865,710.80096666)
\curveto(460.31916477,711.22407853)(460.68627654,711.57874584)(461.14672181,711.86496857)
\curveto(461.60716708,712.16363577)(462.20450148,712.31296937)(462.93872502,712.31296937)
\curveto(464.08361596,712.31296937)(465.01072873,711.86496857)(465.72006334,710.96896696)
\curveto(466.42939794,710.08540983)(466.78406525,708.77874082)(466.78406525,707.04895994)
\curveto(466.78406525,705.30673459)(466.42317571,703.98762112)(465.70139664,703.09161951)
\curveto(464.97961757,702.1956179)(464.03383809,701.7476171)(462.86405822,701.7476171)
\curveto(462.11739021,701.7476171)(461.52627804,701.878284)(461.09072171,702.1396178)
\curveto(460.66760984,702.41339607)(460.31916477,702.71828551)(460.0453865,703.05428611)
\lineto(459.8587195,703.05428611)
\lineto(459.39205199,701.9342841)
\lineto(457.26404818,701.9342841)
\lineto(457.26404818,716.1209762)
\closepath
\moveto(462.04272341,710.09163206)
\curveto(461.32094434,710.09163206)(460.8107212,709.86140942)(460.512054,709.40096415)
\curveto(460.2133868,708.95296335)(460.05783097,708.27473991)(460.0453865,707.36629384)
\lineto(460.0453865,707.06762664)
\curveto(460.0453865,706.08451376)(460.18849787,705.32540129)(460.4747206,704.79028922)
\curveto(460.7733878,704.26762162)(461.30849987,704.00628782)(462.08005681,704.00628782)
\curveto(462.65250228,704.00628782)(463.10672532,704.26762162)(463.44272592,704.79028922)
\curveto(463.77872652,705.32540129)(463.94672683,706.090736)(463.94672683,707.08629334)
\curveto(463.94672683,708.08185068)(463.77250429,708.82851868)(463.42405922,709.32629735)
\curveto(463.08805862,709.83652049)(462.62761335,710.09163206)(462.04272341,710.09163206)
\closepath
}
}
{
\newrgbcolor{curcolor}{0 0 0}
\pscustom[linestyle=none,fillstyle=solid,fillcolor=curcolor]
{
\newpath
\moveto(473.26141364,712.33163607)
\curveto(474.63030498,712.33163607)(475.67564019,712.03296887)(476.39741926,711.43563447)
\curveto(477.1316428,710.85074453)(477.49875457,709.94852069)(477.49875457,708.72896295)
\lineto(477.49875457,701.9342841)
\lineto(475.55741776,701.9342841)
\lineto(475.01608345,703.31561991)
\lineto(474.94141665,703.31561991)
\curveto(474.50586032,702.76806337)(474.04541505,702.36984044)(473.56008084,702.1209511)
\curveto(473.07474664,701.87206177)(472.40896767,701.7476171)(471.56274393,701.7476171)
\curveto(470.65429786,701.7476171)(469.90140762,702.0089509)(469.30407321,702.53161851)
\curveto(468.70673881,703.05428611)(468.40807161,703.86939868)(468.40807161,704.97695622)
\curveto(468.40807161,706.05962483)(468.78762784,706.8560707)(469.54674031,707.36629384)
\curveto(470.30585279,707.87651698)(471.44452149,708.16273971)(472.96274644,708.22496205)
\lineto(474.73608295,708.28096215)
\lineto(474.73608295,708.72896295)
\curveto(474.73608295,709.26407502)(474.59297158,709.65607572)(474.30674885,709.90496506)
\curveto(474.03297058,710.15385439)(473.64719211,710.27829906)(473.14941344,710.27829906)
\curveto(472.65163477,710.27829906)(472.16630057,710.20363226)(471.69341083,710.05429866)
\curveto(471.22052109,709.91740952)(470.74763136,709.74318699)(470.27474162,709.53163105)
\lineto(469.36007331,711.41696777)
\curveto(469.89518538,711.69074604)(470.49874202,711.9085242)(471.17074323,712.07030227)
\curveto(471.84274443,712.24452481)(472.53963457,712.33163607)(473.26141364,712.33163607)
\closepath
\moveto(474.73608295,706.65695923)
\lineto(473.65341434,706.61962583)
\curveto(472.75741274,706.5947369)(472.1351894,706.43295883)(471.78674433,706.13429163)
\curveto(471.43829926,705.83562443)(471.26407673,705.44362373)(471.26407673,704.95828952)
\curveto(471.26407673,704.53517765)(471.38852139,704.23028822)(471.63741073,704.04362122)
\curveto(471.88630006,703.86939868)(472.2098562,703.78228742)(472.60807914,703.78228742)
\curveto(473.20541354,703.78228742)(473.70941444,703.95650995)(474.12008185,704.30495502)
\curveto(474.53074925,704.66584455)(474.73608295,705.16984546)(474.73608295,705.81695773)
\closepath
}
}
{
\newrgbcolor{curcolor}{0 0 0}
\pscustom[linestyle=none,fillstyle=solid,fillcolor=curcolor]
{
\newpath
\moveto(486.04810493,712.31296937)
\curveto(486.18499406,712.31296937)(486.34677213,712.30674714)(486.53343913,712.29430267)
\curveto(486.72010613,712.28185821)(486.86943974,712.26319151)(486.98143994,712.23830257)
\lineto(486.77610623,709.62496456)
\curveto(486.6765505,709.64985349)(486.5458836,709.66852019)(486.38410553,709.68096466)
\curveto(486.22232746,709.70585359)(486.0792161,709.71829806)(485.95477143,709.71829806)
\curveto(485.48188169,709.71829806)(485.02765866,709.63118679)(484.59210232,709.45696425)
\curveto(484.15654598,709.29518619)(483.80187868,709.02763015)(483.52810041,708.65429615)
\curveto(483.26676661,708.28096215)(483.13609971,707.77073901)(483.13609971,707.12362674)
\lineto(483.13609971,701.9342841)
\lineto(480.35476139,701.9342841)
\lineto(480.35476139,712.12630237)
\lineto(482.4640985,712.12630237)
\lineto(482.87476591,710.40896596)
\lineto(483.00543281,710.40896596)
\curveto(483.30410001,710.93163356)(483.71476741,711.37963437)(484.23743502,711.75296837)
\curveto(484.76010262,712.12630237)(485.36365926,712.31296937)(486.04810493,712.31296937)
\closepath
}
}
{
\newrgbcolor{curcolor}{0 0 0}
\pscustom[linestyle=none,fillstyle=solid,fillcolor=curcolor]
{
\newpath
\moveto(495.94144444,704.95828952)
\curveto(495.94144444,703.92539878)(495.57433267,703.12895291)(494.84010913,702.56895191)
\curveto(494.11833006,702.02139537)(493.03566146,701.7476171)(491.59210331,701.7476171)
\curveto(490.88276871,701.7476171)(490.27298984,701.79739497)(489.7627667,701.8969507)
\curveto(489.25254356,701.98406197)(488.74232043,702.13339557)(488.23209729,702.34495151)
\lineto(488.23209729,704.64095562)
\curveto(488.77965383,704.39206629)(489.370766,704.18673258)(490.0054338,704.02495452)
\curveto(490.64010161,703.86317645)(491.20010261,703.78228742)(491.68543681,703.78228742)
\curveto(492.22054888,703.78228742)(492.60632735,703.86317645)(492.84277222,704.02495452)
\curveto(493.07921709,704.18673258)(493.19743952,704.39828852)(493.19743952,704.65962232)
\curveto(493.19743952,704.83384486)(493.14766166,704.98940069)(493.04810592,705.12628982)
\curveto(492.96099466,705.26317896)(492.76188319,705.41873479)(492.45077152,705.59295733)
\curveto(492.13965985,705.76717986)(491.65432565,705.99118026)(490.99476891,706.26495853)
\curveto(490.34765664,706.5387368)(489.8187668,706.80629284)(489.4080994,707.06762664)
\curveto(489.00987646,707.34140491)(488.71120926,707.66496104)(488.51209779,708.03829504)
\curveto(488.31298632,708.42407351)(488.21343059,708.90318548)(488.21343059,709.47563095)
\curveto(488.21343059,710.42141043)(488.58054236,711.13074503)(489.3147659,711.60363477)
\curveto(490.04898944,712.07652451)(491.02588008,712.31296937)(492.24543782,712.31296937)
\curveto(492.88010562,712.31296937)(493.48366226,712.25074704)(494.05610773,712.12630237)
\curveto(494.6285532,712.0018577)(495.21966537,711.796524)(495.82944424,711.51030127)
\lineto(494.98944274,709.51296435)
\curveto(494.49166407,709.72452029)(494.01877433,709.89874282)(493.57077353,710.03563196)
\curveto(493.12277272,710.18496556)(492.66854969,710.25963236)(492.20810442,710.25963236)
\curveto(491.38676961,710.25963236)(490.97610221,710.03563196)(490.97610221,709.58763115)
\curveto(490.97610221,709.42585309)(491.02588008,709.27651949)(491.12543581,709.13963035)
\curveto(491.23743601,709.01518568)(491.44276971,708.87829655)(491.74143691,708.72896295)
\curveto(492.05254858,708.57962935)(492.50677162,708.38051788)(493.10410602,708.13162854)
\curveto(493.68899596,707.89518368)(494.19299686,707.64629434)(494.61610873,707.38496054)
\curveto(495.0392206,707.1360712)(495.36277674,706.8187373)(495.58677714,706.43295883)
\curveto(495.82322201,706.04718036)(495.94144444,705.55562393)(495.94144444,704.95828952)
\closepath
}
}
{
\newrgbcolor{curcolor}{0 0 0}
\pscustom[linestyle=none,fillstyle=solid,fillcolor=curcolor]
{
\newpath
\moveto(216.8740559,680.64403364)
\curveto(215.35583096,680.64403364)(214.17982885,681.06092328)(213.34604958,681.89470255)
\curveto(212.52471477,682.72848182)(212.11404737,684.05381753)(212.11404737,685.87070967)
\curveto(212.11404737,687.11515635)(212.3256033,688.12938039)(212.74871517,688.9133818)
\curveto(213.17182704,689.6973832)(213.75671698,690.27605091)(214.50338498,690.64938491)
\curveto(215.26249746,691.02271891)(216.13361013,691.20938591)(217.116723,691.20938591)
\curveto(217.81361314,691.20938591)(218.41716978,691.14094134)(218.92739291,691.00405221)
\curveto(219.45006052,690.86716308)(219.90428355,690.70538501)(220.29006202,690.51871801)
\lineto(219.46872722,688.37204749)
\curveto(219.03317088,688.54627003)(218.62250348,688.68938139)(218.23672501,688.80138159)
\curveto(217.86339101,688.9133818)(217.490057,688.9693819)(217.116723,688.9693819)
\curveto(215.67316486,688.9693819)(214.95138579,687.94271339)(214.95138579,685.88937637)
\curveto(214.95138579,684.8689301)(215.13805279,684.11603986)(215.51138679,683.63070566)
\curveto(215.89716526,683.14537146)(216.43227733,682.90270435)(217.116723,682.90270435)
\curveto(217.70161294,682.90270435)(218.21805831,682.97737115)(218.66605911,683.12670476)
\curveto(219.11405992,683.28848282)(219.54961625,683.50626099)(219.97272812,683.78003926)
\lineto(219.97272812,681.40936834)
\curveto(219.54961625,681.13559008)(219.10161545,680.94270084)(218.62872571,680.83070064)
\curveto(218.16828044,680.70625597)(217.58339051,680.64403364)(216.8740559,680.64403364)
\closepath
}
}
{
\newrgbcolor{curcolor}{0 0 0}
\pscustom[linestyle=none,fillstyle=solid,fillcolor=curcolor]
{
\newpath
\moveto(231.58341431,685.94537648)
\curveto(231.58341431,684.252929)(231.13541351,682.94625999)(230.2394119,682.02536945)
\curveto(229.35585476,681.10447891)(228.14874148,680.64403364)(226.61807207,680.64403364)
\curveto(225.6722926,680.64403364)(224.82606886,680.84936734)(224.07940086,681.26003474)
\curveto(223.34517732,681.67070215)(222.76650961,682.26803655)(222.34339774,683.05203796)
\curveto(221.92028588,683.84848383)(221.70872994,684.81293)(221.70872994,685.94537648)
\curveto(221.70872994,687.63782395)(222.15050851,688.93827073)(223.03406565,689.8467168)
\curveto(223.91762279,690.75516287)(225.1309583,691.20938591)(226.67407217,691.20938591)
\curveto(227.63229611,691.20938591)(228.47851985,691.00405221)(229.21274339,690.59338481)
\curveto(229.94696693,690.1827174)(230.52563463,689.585383)(230.9487465,688.80138159)
\curveto(231.37185837,688.01738019)(231.58341431,687.06537848)(231.58341431,685.94537648)
\closepath
\moveto(224.54606836,685.94537648)
\curveto(224.54606836,684.93737467)(224.70784643,684.17203996)(225.03140256,683.64937236)
\curveto(225.36740317,683.13914922)(225.90873747,682.88403765)(226.65540547,682.88403765)
\curveto(227.38962901,682.88403765)(227.91851885,683.13914922)(228.24207499,683.64937236)
\curveto(228.57807559,684.17203996)(228.74607589,684.93737467)(228.74607589,685.94537648)
\curveto(228.74607589,686.95337828)(228.57807559,687.70626852)(228.24207499,688.20404719)
\curveto(227.91851885,688.71427033)(227.38340678,688.9693819)(226.63673877,688.9693819)
\curveto(225.90251524,688.9693819)(225.36740317,688.71427033)(225.03140256,688.20404719)
\curveto(224.70784643,687.70626852)(224.54606836,686.95337828)(224.54606836,685.94537648)
\closepath
}
}
{
\newrgbcolor{curcolor}{0 0 0}
\pscustom[linestyle=none,fillstyle=solid,fillcolor=curcolor]
{
\newpath
\moveto(239.6660859,691.20938591)
\curveto(240.76119898,691.20938591)(241.63853388,690.91071871)(242.29809062,690.31338431)
\curveto(242.95764736,689.72849437)(243.28742573,688.78271489)(243.28742573,687.47604589)
\lineto(243.28742573,680.83070064)
\lineto(240.50608741,680.83070064)
\lineto(240.50608741,686.78537798)
\curveto(240.50608741,687.51960152)(240.37542051,688.06715806)(240.11408671,688.42804759)
\curveto(239.8527529,688.80138159)(239.43586327,688.9880486)(238.8634178,688.9880486)
\curveto(238.01719406,688.9880486)(237.43852635,688.69560363)(237.12741468,688.11071369)
\curveto(236.81630302,687.53826822)(236.66074718,686.71071118)(236.66074718,685.62804257)
\lineto(236.66074718,680.83070064)
\lineto(233.87940886,680.83070064)
\lineto(233.87940886,691.02271891)
\lineto(236.00741268,691.02271891)
\lineto(236.38074668,689.7160499)
\lineto(236.53008028,689.7160499)
\curveto(236.85363642,690.2387175)(237.29541499,690.61827374)(237.85541599,690.85471861)
\curveto(238.42786146,691.09116348)(239.0314181,691.20938591)(239.6660859,691.20938591)
\closepath
}
}
{
\newrgbcolor{curcolor}{0 0 0}
\pscustom[linestyle=none,fillstyle=solid,fillcolor=curcolor]
{
\newpath
\moveto(251.93010095,691.20938591)
\curveto(253.02521402,691.20938591)(253.90254893,690.91071871)(254.56210566,690.31338431)
\curveto(255.2216624,689.72849437)(255.55144077,688.78271489)(255.55144077,687.47604589)
\lineto(255.55144077,680.83070064)
\lineto(252.77010245,680.83070064)
\lineto(252.77010245,686.78537798)
\curveto(252.77010245,687.51960152)(252.63943555,688.06715806)(252.37810175,688.42804759)
\curveto(252.11676795,688.80138159)(251.69987831,688.9880486)(251.12743284,688.9880486)
\curveto(250.2812091,688.9880486)(249.7025414,688.69560363)(249.39142973,688.11071369)
\curveto(249.08031806,687.53826822)(248.92476222,686.71071118)(248.92476222,685.62804257)
\lineto(248.92476222,680.83070064)
\lineto(246.14342391,680.83070064)
\lineto(246.14342391,691.02271891)
\lineto(248.27142772,691.02271891)
\lineto(248.64476172,689.7160499)
\lineto(248.79409532,689.7160499)
\curveto(249.11765146,690.2387175)(249.55943003,690.61827374)(250.11943103,690.85471861)
\curveto(250.6918765,691.09116348)(251.29543314,691.20938591)(251.93010095,691.20938591)
\closepath
}
}
{
\newrgbcolor{curcolor}{0 0 0}
\pscustom[linestyle=none,fillstyle=solid,fillcolor=curcolor]
{
\newpath
\moveto(262.60744648,691.20938591)
\curveto(264.01367122,691.20938591)(265.12745099,690.80494074)(265.9487858,689.9960504)
\curveto(266.77012061,689.19960453)(267.18078801,688.06093582)(267.18078801,686.58004428)
\lineto(267.18078801,685.23604187)
\lineto(260.61010956,685.23604187)
\curveto(260.6349985,684.45204047)(260.86522113,683.83603936)(261.30077747,683.38803856)
\curveto(261.74877827,682.94003775)(262.36477938,682.71603735)(263.14878078,682.71603735)
\curveto(263.79589305,682.71603735)(264.38700522,682.77825969)(264.92211729,682.90270435)
\curveto(265.46967383,683.03959349)(266.02967483,683.24492719)(266.60212031,683.51870546)
\lineto(266.60212031,681.37203494)
\curveto(266.09189717,681.12314561)(265.56300733,680.94270084)(265.01545079,680.83070064)
\curveto(264.46789426,680.70625597)(263.80211529,680.64403364)(263.01811388,680.64403364)
\curveto(261.99766761,680.64403364)(261.09544377,680.83070064)(260.31144236,681.20403464)
\curveto(259.52744096,681.58981311)(258.91143985,682.16225858)(258.46343905,682.92137105)
\curveto(258.01543825,683.69292799)(257.79143784,684.66981863)(257.79143784,685.85204297)
\curveto(257.79143784,687.03426732)(257.99054931,688.02360242)(258.38877225,688.8200483)
\curveto(258.79943965,689.61649417)(259.36566289,690.21382857)(260.08744196,690.61205151)
\curveto(260.80922103,691.01027444)(261.64922254,691.20938591)(262.60744648,691.20938591)
\closepath
\moveto(262.62611318,689.2307157)
\curveto(262.07855664,689.2307157)(261.63055584,689.05649316)(261.28211077,688.70804809)
\curveto(260.9336657,688.35960303)(260.728332,687.81826872)(260.66610966,687.08404518)
\lineto(264.56744999,687.08404518)
\curveto(264.55500552,687.69382405)(264.38700522,688.20404719)(264.06344909,688.61471459)
\curveto(263.75233742,689.025382)(263.27322545,689.2307157)(262.62611318,689.2307157)
\closepath
}
}
{
\newrgbcolor{curcolor}{0 0 0}
\pscustom[linestyle=none,fillstyle=solid,fillcolor=curcolor]
{
\newpath
\moveto(273.58345918,680.64403364)
\curveto(272.06523423,680.64403364)(270.88923213,681.06092328)(270.05545285,681.89470255)
\curveto(269.23411805,682.72848182)(268.82345065,684.05381753)(268.82345065,685.87070967)
\curveto(268.82345065,687.11515635)(269.03500658,688.12938039)(269.45811845,688.9133818)
\curveto(269.88123032,689.6973832)(270.46612026,690.27605091)(271.21278826,690.64938491)
\curveto(271.97190073,691.02271891)(272.84301341,691.20938591)(273.82612628,691.20938591)
\curveto(274.52301642,691.20938591)(275.12657306,691.14094134)(275.63679619,691.00405221)
\curveto(276.1594638,690.86716308)(276.61368683,690.70538501)(276.9994653,690.51871801)
\lineto(276.1781305,688.37204749)
\curveto(275.74257416,688.54627003)(275.33190676,688.68938139)(274.94612829,688.80138159)
\curveto(274.57279428,688.9133818)(274.19946028,688.9693819)(273.82612628,688.9693819)
\curveto(272.38256814,688.9693819)(271.66078906,687.94271339)(271.66078906,685.88937637)
\curveto(271.66078906,684.8689301)(271.84745607,684.11603986)(272.22079007,683.63070566)
\curveto(272.60656854,683.14537146)(273.14168061,682.90270435)(273.82612628,682.90270435)
\curveto(274.41101622,682.90270435)(274.92746159,682.97737115)(275.37546239,683.12670476)
\curveto(275.82346319,683.28848282)(276.25901953,683.50626099)(276.6821314,683.78003926)
\lineto(276.6821314,681.40936834)
\curveto(276.25901953,681.13559008)(275.81101873,680.94270084)(275.33812899,680.83070064)
\curveto(274.87768372,680.70625597)(274.29279378,680.64403364)(273.58345918,680.64403364)
\closepath
}
}
{
\newrgbcolor{curcolor}{0 0 0}
\pscustom[linestyle=none,fillstyle=solid,fillcolor=curcolor]
{
\newpath
\moveto(283.32747344,682.86537095)
\curveto(283.63858511,682.86537095)(283.93725231,682.89025989)(284.22347505,682.94003775)
\curveto(284.50969779,683.00226009)(284.79592052,683.08314912)(285.08214326,683.18270486)
\lineto(285.08214326,681.11070114)
\curveto(284.78347605,680.97381201)(284.41014205,680.86181181)(283.96214125,680.77470054)
\curveto(283.52658491,680.68758927)(283.04747294,680.64403364)(282.52480534,680.64403364)
\curveto(281.91502647,680.64403364)(281.36746993,680.74358937)(280.88213573,680.94270084)
\curveto(280.40924599,681.14181231)(280.02968975,681.48403514)(279.74346702,681.96936935)
\curveto(279.46968875,682.45470355)(279.33279962,683.13914922)(279.33279962,684.02270636)
\lineto(279.33279962,688.9320485)
\lineto(278.00746391,688.9320485)
\lineto(278.00746391,690.1080506)
\lineto(279.53813332,691.04138561)
\lineto(280.34080142,693.18805612)
\lineto(282.11413794,693.18805612)
\lineto(282.11413794,691.02271891)
\lineto(284.97014306,691.02271891)
\lineto(284.97014306,688.9320485)
\lineto(282.11413794,688.9320485)
\lineto(282.11413794,684.02270636)
\curveto(282.11413794,683.63692789)(282.22613814,683.34448292)(282.45013854,683.14537146)
\curveto(282.67413894,682.95870445)(282.96658391,682.86537095)(283.32747344,682.86537095)
\closepath
}
}
{
\newrgbcolor{curcolor}{0 0 0}
\pscustom[linestyle=none,fillstyle=solid,fillcolor=curcolor]
{
\newpath
\moveto(291.24215522,694.15872453)
\curveto(293.05904736,694.15872453)(294.39682754,693.82894616)(295.25549575,693.16938942)
\curveto(296.12660842,692.50983269)(296.56216475,691.50805311)(296.56216475,690.1640507)
\curveto(296.56216475,689.55427183)(296.44394232,689.01915976)(296.20749745,688.55871449)
\curveto(295.98349705,688.11071369)(295.67860761,687.72493522)(295.29282915,687.40137909)
\curveto(294.91949514,687.09026742)(294.51504997,686.83515585)(294.07949364,686.63604438)
\lineto(297.99950066,680.83070064)
\lineto(294.86349504,680.83070064)
\lineto(291.69015602,685.94537648)
\lineto(290.17815331,685.94537648)
\lineto(290.17815331,680.83070064)
\lineto(287.35948159,680.83070064)
\lineto(287.35948159,694.15872453)
\closepath
\moveto(291.03682152,691.84405372)
\lineto(290.17815331,691.84405372)
\lineto(290.17815331,688.24138059)
\lineto(291.09282162,688.24138059)
\curveto(292.02615662,688.24138059)(292.69193559,688.39693643)(293.09015853,688.70804809)
\curveto(293.50082593,689.01915976)(293.70615963,689.47960503)(293.70615963,690.0893839)
\curveto(293.70615963,690.72405171)(293.48838147,691.17205251)(293.05282513,691.43338631)
\curveto(292.61726879,691.70716458)(291.94526759,691.84405372)(291.03682152,691.84405372)
\closepath
}
}
{
\newrgbcolor{curcolor}{0 0 0}
\pscustom[linestyle=none,fillstyle=solid,fillcolor=curcolor]
{
\newpath
\moveto(308.71417925,685.94537648)
\curveto(308.71417925,684.252929)(308.26617845,682.94625999)(307.37017684,682.02536945)
\curveto(306.4866197,681.10447891)(305.27950643,680.64403364)(303.74883702,680.64403364)
\curveto(302.80305754,680.64403364)(301.95683381,680.84936734)(301.2101658,681.26003474)
\curveto(300.47594226,681.67070215)(299.89727456,682.26803655)(299.47416269,683.05203796)
\curveto(299.05105082,683.84848383)(298.83949488,684.81293)(298.83949488,685.94537648)
\curveto(298.83949488,687.63782395)(299.28127345,688.93827073)(300.16483059,689.8467168)
\curveto(301.04838773,690.75516287)(302.26172324,691.20938591)(303.80483712,691.20938591)
\curveto(304.76306106,691.20938591)(305.6092848,691.00405221)(306.34350834,690.59338481)
\curveto(307.07773187,690.1827174)(307.65639958,689.585383)(308.07951145,688.80138159)
\curveto(308.50262332,688.01738019)(308.71417925,687.06537848)(308.71417925,685.94537648)
\closepath
\moveto(301.6768333,685.94537648)
\curveto(301.6768333,684.93737467)(301.83861137,684.17203996)(302.16216751,683.64937236)
\curveto(302.49816811,683.13914922)(303.03950241,682.88403765)(303.78617042,682.88403765)
\curveto(304.52039396,682.88403765)(305.04928379,683.13914922)(305.37283993,683.64937236)
\curveto(305.70884053,684.17203996)(305.87684083,684.93737467)(305.87684083,685.94537648)
\curveto(305.87684083,686.95337828)(305.70884053,687.70626852)(305.37283993,688.20404719)
\curveto(305.04928379,688.71427033)(304.51417172,688.9693819)(303.76750372,688.9693819)
\curveto(303.03328018,688.9693819)(302.49816811,688.71427033)(302.16216751,688.20404719)
\curveto(301.83861137,687.70626852)(301.6768333,686.95337828)(301.6768333,685.94537648)
\closepath
}
}
{
\newrgbcolor{curcolor}{0 0 0}
\pscustom[linestyle=none,fillstyle=solid,fillcolor=curcolor]
{
\newpath
\moveto(320.26885898,685.94537648)
\curveto(320.26885898,684.252929)(319.82085817,682.94625999)(318.92485657,682.02536945)
\curveto(318.04129943,681.10447891)(316.83418615,680.64403364)(315.30351674,680.64403364)
\curveto(314.35773727,680.64403364)(313.51151353,680.84936734)(312.76484552,681.26003474)
\curveto(312.03062199,681.67070215)(311.45195428,682.26803655)(311.02884241,683.05203796)
\curveto(310.60573054,683.84848383)(310.39417461,684.81293)(310.39417461,685.94537648)
\curveto(310.39417461,687.63782395)(310.83595318,688.93827073)(311.71951032,689.8467168)
\curveto(312.60306746,690.75516287)(313.81640297,691.20938591)(315.35951684,691.20938591)
\curveto(316.31774078,691.20938591)(317.16396452,691.00405221)(317.89818806,690.59338481)
\curveto(318.6324116,690.1827174)(319.2110793,689.585383)(319.63419117,688.80138159)
\curveto(320.05730304,688.01738019)(320.26885898,687.06537848)(320.26885898,685.94537648)
\closepath
\moveto(313.23151303,685.94537648)
\curveto(313.23151303,684.93737467)(313.3932911,684.17203996)(313.71684723,683.64937236)
\curveto(314.05284783,683.13914922)(314.59418214,682.88403765)(315.34085014,682.88403765)
\curveto(316.07507368,682.88403765)(316.60396352,683.13914922)(316.92751965,683.64937236)
\curveto(317.26352026,684.17203996)(317.43152056,684.93737467)(317.43152056,685.94537648)
\curveto(317.43152056,686.95337828)(317.26352026,687.70626852)(316.92751965,688.20404719)
\curveto(316.60396352,688.71427033)(316.06885145,688.9693819)(315.32218344,688.9693819)
\curveto(314.5879599,688.9693819)(314.05284783,688.71427033)(313.71684723,688.20404719)
\curveto(313.3932911,687.70626852)(313.23151303,686.95337828)(313.23151303,685.94537648)
\closepath
}
}
{
\newrgbcolor{curcolor}{0 0 0}
\pscustom[linestyle=none,fillstyle=solid,fillcolor=curcolor]
{
\newpath
\moveto(334.53021022,691.20938591)
\curveto(335.68754563,691.20938591)(336.5586583,690.91071871)(337.14354824,690.31338431)
\curveto(337.74088264,689.72849437)(338.03954984,688.78271489)(338.03954984,687.47604589)
\lineto(338.03954984,680.83070064)
\lineto(335.25821153,680.83070064)
\lineto(335.25821153,686.78537798)
\curveto(335.25821153,688.25382506)(334.74798839,688.9880486)(333.72754211,688.9880486)
\curveto(332.99331858,688.9880486)(332.47065097,688.72671479)(332.1595393,688.20404719)
\curveto(331.84842763,687.68137959)(331.6928718,686.92848935)(331.6928718,685.94537648)
\lineto(331.6928718,680.83070064)
\lineto(328.91153348,680.83070064)
\lineto(328.91153348,686.78537798)
\curveto(328.91153348,688.25382506)(328.40131034,688.9880486)(327.38086407,688.9880486)
\curveto(326.60930713,688.9880486)(326.07419506,688.69560363)(325.77552786,688.11071369)
\curveto(325.48930512,687.53826822)(325.34619376,686.71071118)(325.34619376,685.62804257)
\lineto(325.34619376,680.83070064)
\lineto(322.56485544,680.83070064)
\lineto(322.56485544,691.02271891)
\lineto(324.69285925,691.02271891)
\lineto(325.06619326,689.7160499)
\lineto(325.21552686,689.7160499)
\curveto(325.52663852,690.2387175)(325.94975039,690.61827374)(326.48486246,690.85471861)
\curveto(327.032419,691.09116348)(327.59864224,691.20938591)(328.18353218,691.20938591)
\curveto(328.93020018,691.20938591)(329.55864575,691.08494124)(330.06886889,690.83605191)
\curveto(330.59153649,690.59960704)(330.99598166,690.22627304)(331.2822044,689.7160499)
\lineto(331.5248715,689.7160499)
\curveto(331.83598317,690.2387175)(332.26531727,690.61827374)(332.81287381,690.85471861)
\curveto(333.37287481,691.09116348)(333.94532028,691.20938591)(334.53021022,691.20938591)
\closepath
}
}
{
\newrgbcolor{curcolor}{0 0 0}
\pscustom[linestyle=none,fillstyle=solid,fillcolor=curcolor]
{
\newpath
\moveto(340.50355824,682.13736965)
\curveto(340.50355824,682.70981512)(340.65911407,683.10803806)(340.97022574,683.33203846)
\curveto(341.28133741,683.56848333)(341.66089365,683.68670576)(342.10889445,683.68670576)
\curveto(342.54445079,683.68670576)(342.91778479,683.56848333)(343.22889646,683.33203846)
\curveto(343.54000813,683.10803806)(343.69556396,682.70981512)(343.69556396,682.13736965)
\curveto(343.69556396,681.58981311)(343.54000813,681.19159018)(343.22889646,680.94270084)
\curveto(342.91778479,680.70625597)(342.54445079,680.58803354)(342.10889445,680.58803354)
\curveto(341.66089365,680.58803354)(341.28133741,680.70625597)(340.97022574,680.94270084)
\curveto(340.65911407,681.19159018)(340.50355824,681.58981311)(340.50355824,682.13736965)
\closepath
}
}
{
\newrgbcolor{curcolor}{0 0 0}
\pscustom[linestyle=none,fillstyle=solid,fillcolor=curcolor]
{
\newpath
\moveto(348.99690452,695.01739274)
\lineto(348.99690452,692.12405422)
\curveto(348.99690452,691.61383108)(348.97823782,691.12849688)(348.94090442,690.66805161)
\curveto(348.91601548,690.2200508)(348.89112655,689.9027169)(348.86623762,689.7160499)
\lineto(349.01557122,689.7160499)
\curveto(349.33912735,690.2387175)(349.75601699,690.61827374)(350.26624013,690.85471861)
\curveto(350.77646326,691.09116348)(351.3426865,691.20938591)(351.96490984,691.20938591)
\curveto(353.06002291,691.20938591)(353.94358005,690.91071871)(354.61558126,690.31338431)
\curveto(355.28758246,689.72849437)(355.62358306,688.78271489)(355.62358306,687.47604589)
\lineto(355.62358306,680.83070064)
\lineto(352.84224475,680.83070064)
\lineto(352.84224475,686.78537798)
\curveto(352.84224475,688.25382506)(352.29468821,688.9880486)(351.19957513,688.9880486)
\curveto(350.36579586,688.9880486)(349.78712816,688.69560363)(349.46357202,688.11071369)
\curveto(349.15246035,687.53826822)(348.99690452,686.71071118)(348.99690452,685.62804257)
\lineto(348.99690452,680.83070064)
\lineto(346.2155662,680.83070064)
\lineto(346.2155662,695.01739274)
\closepath
}
}
{
\newrgbcolor{curcolor}{0 0 0}
\pscustom[linestyle=none,fillstyle=solid,fillcolor=curcolor]
{
\newpath
\moveto(362.6609297,691.22805261)
\curveto(364.02982104,691.22805261)(365.07515625,690.92938541)(365.79693532,690.33205101)
\curveto(366.53115886,689.74716107)(366.89827063,688.84493723)(366.89827063,687.62537949)
\lineto(366.89827063,680.83070064)
\lineto(364.95693382,680.83070064)
\lineto(364.41559951,682.21203645)
\lineto(364.34093271,682.21203645)
\curveto(363.90537638,681.66447991)(363.44493111,681.26625698)(362.9595969,681.01736764)
\curveto(362.4742627,680.76847831)(361.80848373,680.64403364)(360.96225999,680.64403364)
\curveto(360.05381392,680.64403364)(359.30092368,680.90536744)(358.70358927,681.42803504)
\curveto(358.10625487,681.95070265)(357.80758767,682.76581522)(357.80758767,683.87337276)
\curveto(357.80758767,684.95604137)(358.1871439,685.75248724)(358.94625638,686.26271038)
\curveto(359.70536885,686.77293351)(360.84403756,687.05915625)(362.3622625,687.12137858)
\lineto(364.13559901,687.17737868)
\lineto(364.13559901,687.62537949)
\curveto(364.13559901,688.16049156)(363.99248764,688.55249226)(363.70626491,688.80138159)
\curveto(363.43248664,689.05027093)(363.04670817,689.1747156)(362.5489295,689.1747156)
\curveto(362.05115083,689.1747156)(361.56581663,689.1000488)(361.09292689,688.9507152)
\curveto(360.62003715,688.81382606)(360.14714742,688.63960353)(359.67425768,688.42804759)
\lineto(358.75958937,690.31338431)
\curveto(359.29470144,690.58716257)(359.89825808,690.80494074)(360.57025929,690.96671881)
\curveto(361.24226049,691.14094134)(361.93915063,691.22805261)(362.6609297,691.22805261)
\closepath
\moveto(364.13559901,685.55337577)
\lineto(363.0529304,685.51604237)
\curveto(362.1569288,685.49115344)(361.53470546,685.32937537)(361.18626039,685.03070817)
\curveto(360.83781532,684.73204097)(360.66359279,684.34004026)(360.66359279,683.85470606)
\curveto(360.66359279,683.43159419)(360.78803745,683.12670476)(361.03692679,682.94003775)
\curveto(361.28581612,682.76581522)(361.60937226,682.67870395)(362.0075952,682.67870395)
\curveto(362.6049296,682.67870395)(363.1089305,682.85292649)(363.51959791,683.20137156)
\curveto(363.93026531,683.56226109)(364.13559901,684.066262)(364.13559901,684.71337427)
\closepath
}
}
{
\newrgbcolor{curcolor}{0 0 0}
\pscustom[linestyle=none,fillstyle=solid,fillcolor=curcolor]
{
\newpath
\moveto(375.54095449,691.20938591)
\curveto(376.63606756,691.20938591)(377.51340247,690.91071871)(378.17295921,690.31338431)
\curveto(378.83251595,689.72849437)(379.16229432,688.78271489)(379.16229432,687.47604589)
\lineto(379.16229432,680.83070064)
\lineto(376.380956,680.83070064)
\lineto(376.380956,686.78537798)
\curveto(376.380956,687.51960152)(376.2502891,688.06715806)(375.98895529,688.42804759)
\curveto(375.72762149,688.80138159)(375.31073186,688.9880486)(374.73828639,688.9880486)
\curveto(373.89206265,688.9880486)(373.31339494,688.69560363)(373.00228327,688.11071369)
\curveto(372.6911716,687.53826822)(372.53561577,686.71071118)(372.53561577,685.62804257)
\lineto(372.53561577,680.83070064)
\lineto(369.75427745,680.83070064)
\lineto(369.75427745,691.02271891)
\lineto(371.88228127,691.02271891)
\lineto(372.25561527,689.7160499)
\lineto(372.40494887,689.7160499)
\curveto(372.728505,690.2387175)(373.17028357,690.61827374)(373.73028458,690.85471861)
\curveto(374.30273005,691.09116348)(374.90628669,691.20938591)(375.54095449,691.20938591)
\closepath
}
}
{
\newrgbcolor{curcolor}{0 0 0}
\pscustom[linestyle=none,fillstyle=solid,fillcolor=curcolor]
{
\newpath
\moveto(385.24762399,680.64403364)
\curveto(384.11517751,680.64403364)(383.18806474,681.08581221)(382.46628567,681.96936935)
\curveto(381.75695106,682.86537095)(381.40228376,684.1782622)(381.40228376,685.90804308)
\curveto(381.40228376,687.65026842)(381.7631733,688.9693819)(382.48495237,689.8653835)
\curveto(383.20673144,690.76138511)(384.15251091,691.20938591)(385.32229079,691.20938591)
\curveto(386.05651433,691.20938591)(386.66007096,691.06627454)(387.1329607,690.78005181)
\curveto(387.60585044,690.49382907)(387.97918444,690.13916177)(388.25296271,689.7160499)
\lineto(388.34629621,689.7160499)
\curveto(388.30896281,689.91516137)(388.26540717,690.2013841)(388.21562931,690.57471811)
\curveto(388.16585144,690.96049658)(388.14096251,691.35249728)(388.14096251,691.75072022)
\lineto(388.14096251,695.01739274)
\lineto(390.92230083,695.01739274)
\lineto(390.92230083,680.83070064)
\lineto(388.79429701,680.83070064)
\lineto(388.25296271,682.15603635)
\lineto(388.14096251,682.15603635)
\curveto(387.86718424,681.73292448)(387.50007247,681.37203494)(387.0396272,681.07336774)
\curveto(386.57918193,680.78714501)(385.98184753,680.64403364)(385.24762399,680.64403364)
\closepath
\moveto(386.21829239,682.86537095)
\curveto(386.97740487,682.86537095)(387.51251694,683.08937136)(387.8236286,683.53737216)
\curveto(388.13474027,683.99781743)(388.30274057,684.6822631)(388.32762951,685.59070917)
\lineto(388.32762951,685.88937637)
\curveto(388.32762951,686.87248925)(388.17207367,687.62537949)(387.860962,688.14804709)
\curveto(387.5622948,688.68315916)(387.0022938,688.9507152)(386.18095899,688.9507152)
\curveto(385.57118012,688.9507152)(385.09206815,688.68315916)(384.74362308,688.14804709)
\curveto(384.39517801,687.62537949)(384.22095548,686.86626701)(384.22095548,685.87070967)
\curveto(384.22095548,684.87515233)(384.39517801,684.1222621)(384.74362308,683.61203896)
\curveto(385.09206815,683.11426029)(385.58362459,682.86537095)(386.21829239,682.86537095)
\closepath
}
}
{
\newrgbcolor{curcolor}{0 0 0}
\pscustom[linestyle=none,fillstyle=solid,fillcolor=curcolor]
{
\newpath
\moveto(396.61563256,680.83070064)
\lineto(393.83429424,680.83070064)
\lineto(393.83429424,695.01739274)
\lineto(396.61563256,695.01739274)
\closepath
}
}
{
\newrgbcolor{curcolor}{0 0 0}
\pscustom[linestyle=none,fillstyle=solid,fillcolor=curcolor]
{
\newpath
\moveto(403.72764581,691.20938591)
\curveto(405.13387055,691.20938591)(406.24765033,690.80494074)(407.06898513,689.9960504)
\curveto(407.89031994,689.19960453)(408.30098734,688.06093582)(408.30098734,686.58004428)
\lineto(408.30098734,685.23604187)
\lineto(401.7303089,685.23604187)
\curveto(401.75519783,684.45204047)(401.98542047,683.83603936)(402.4209768,683.38803856)
\curveto(402.8689776,682.94003775)(403.48497871,682.71603735)(404.26898011,682.71603735)
\curveto(404.91609239,682.71603735)(405.50720456,682.77825969)(406.04231663,682.90270435)
\curveto(406.58987316,683.03959349)(407.14987417,683.24492719)(407.72231964,683.51870546)
\lineto(407.72231964,681.37203494)
\curveto(407.2120965,681.12314561)(406.68320666,680.94270084)(406.13565013,680.83070064)
\curveto(405.58809359,680.70625597)(404.92231462,680.64403364)(404.13831321,680.64403364)
\curveto(403.11786694,680.64403364)(402.2156431,680.83070064)(401.43164169,681.20403464)
\curveto(400.64764029,681.58981311)(400.03163919,682.16225858)(399.58363838,682.92137105)
\curveto(399.13563758,683.69292799)(398.91163718,684.66981863)(398.91163718,685.85204297)
\curveto(398.91163718,687.03426732)(399.11074865,688.02360242)(399.50897158,688.8200483)
\curveto(399.91963898,689.61649417)(400.48586222,690.21382857)(401.20764129,690.61205151)
\curveto(401.92942036,691.01027444)(402.76942187,691.20938591)(403.72764581,691.20938591)
\closepath
\moveto(403.74631251,689.2307157)
\curveto(403.19875597,689.2307157)(402.75075517,689.05649316)(402.4023101,688.70804809)
\curveto(402.05386503,688.35960303)(401.84853133,687.81826872)(401.786309,687.08404518)
\lineto(405.68764932,687.08404518)
\curveto(405.67520486,687.69382405)(405.50720456,688.20404719)(405.18364842,688.61471459)
\curveto(404.87253675,689.025382)(404.39342478,689.2307157)(403.74631251,689.2307157)
\closepath
}
}
{
\newrgbcolor{curcolor}{0 0 0}
\pscustom[linestyle=none,fillstyle=solid,fillcolor=curcolor]
{
\newpath
\moveto(413.3409894,695.01739274)
\lineto(413.3409894,691.71338681)
\curveto(413.3409894,691.32760835)(413.32854493,690.94805211)(413.303656,690.57471811)
\curveto(413.27876707,690.2013841)(413.25387813,689.90893914)(413.2289892,689.6973832)
\lineto(413.3409894,689.6973832)
\curveto(413.61476767,690.12049507)(413.98187944,690.47516237)(414.44232471,690.76138511)
\curveto(414.90276998,691.06005231)(415.50010438,691.20938591)(416.23432792,691.20938591)
\curveto(417.37921886,691.20938591)(418.30633164,690.76138511)(419.01566624,689.8653835)
\curveto(419.72500085,688.98182636)(420.07966815,687.67515735)(420.07966815,685.94537648)
\curveto(420.07966815,684.20315113)(419.71877861,682.88403765)(418.99699954,681.98803605)
\curveto(418.27522047,681.09203444)(417.329441,680.64403364)(416.15966112,680.64403364)
\curveto(415.41299312,680.64403364)(414.82188094,680.77470054)(414.38632461,681.03603434)
\curveto(413.96321274,681.30981261)(413.61476767,681.61470205)(413.3409894,681.95070265)
\lineto(413.1543224,681.95070265)
\lineto(412.6876549,680.83070064)
\lineto(410.55965108,680.83070064)
\lineto(410.55965108,695.01739274)
\closepath
\moveto(415.33832632,688.9880486)
\curveto(414.61654724,688.9880486)(414.10632411,688.75782596)(413.8076569,688.29738069)
\curveto(413.5089897,687.84937989)(413.35343387,687.17115645)(413.3409894,686.26271038)
\lineto(413.3409894,685.96404318)
\curveto(413.3409894,684.9809303)(413.48410077,684.22181783)(413.7703235,683.68670576)
\curveto(414.06899071,683.16403816)(414.60410278,682.90270435)(415.37565972,682.90270435)
\curveto(415.94810519,682.90270435)(416.40232822,683.16403816)(416.73832882,683.68670576)
\curveto(417.07432943,684.22181783)(417.24232973,684.98715254)(417.24232973,685.98270988)
\curveto(417.24232973,686.97826722)(417.06810719,687.72493522)(416.71966212,688.22271389)
\curveto(416.38366152,688.73293703)(415.92321625,688.9880486)(415.33832632,688.9880486)
\closepath
}
}
{
\newrgbcolor{curcolor}{0 0 0}
\pscustom[linestyle=none,fillstyle=solid,fillcolor=curcolor]
{
\newpath
\moveto(426.55700128,691.22805261)
\curveto(427.92589263,691.22805261)(428.97122783,690.92938541)(429.69300691,690.33205101)
\curveto(430.42723044,689.74716107)(430.79434221,688.84493723)(430.79434221,687.62537949)
\lineto(430.79434221,680.83070064)
\lineto(428.8530054,680.83070064)
\lineto(428.3116711,682.21203645)
\lineto(428.2370043,682.21203645)
\curveto(427.80144796,681.66447991)(427.34100269,681.26625698)(426.85566849,681.01736764)
\curveto(426.37033428,680.76847831)(425.70455531,680.64403364)(424.85833157,680.64403364)
\curveto(423.9498855,680.64403364)(423.19699526,680.90536744)(422.59966086,681.42803504)
\curveto(422.00232645,681.95070265)(421.70365925,682.76581522)(421.70365925,683.87337276)
\curveto(421.70365925,684.95604137)(422.08321549,685.75248724)(422.84232796,686.26271038)
\curveto(423.60144043,686.77293351)(424.74010914,687.05915625)(426.25833408,687.12137858)
\lineto(428.03167059,687.17737868)
\lineto(428.03167059,687.62537949)
\curveto(428.03167059,688.16049156)(427.88855923,688.55249226)(427.60233649,688.80138159)
\curveto(427.32855822,689.05027093)(426.94277975,689.1747156)(426.44500108,689.1747156)
\curveto(425.94722241,689.1747156)(425.46188821,689.1000488)(424.98899847,688.9507152)
\curveto(424.51610874,688.81382606)(424.043219,688.63960353)(423.57032926,688.42804759)
\lineto(422.65566096,690.31338431)
\curveto(423.19077303,690.58716257)(423.79432966,690.80494074)(424.46633087,690.96671881)
\curveto(425.13833207,691.14094134)(425.83522221,691.22805261)(426.55700128,691.22805261)
\closepath
\moveto(428.03167059,685.55337577)
\lineto(426.94900199,685.51604237)
\curveto(426.05300038,685.49115344)(425.43077704,685.32937537)(425.08233197,685.03070817)
\curveto(424.7338869,684.73204097)(424.55966437,684.34004026)(424.55966437,683.85470606)
\curveto(424.55966437,683.43159419)(424.68410904,683.12670476)(424.93299837,682.94003775)
\curveto(425.18188771,682.76581522)(425.50544384,682.67870395)(425.90366678,682.67870395)
\curveto(426.50100118,682.67870395)(427.00500209,682.85292649)(427.41566949,683.20137156)
\curveto(427.82633689,683.56226109)(428.03167059,684.066262)(428.03167059,684.71337427)
\closepath
}
}
{
\newrgbcolor{curcolor}{0 0 0}
\pscustom[linestyle=none,fillstyle=solid,fillcolor=curcolor]
{
\newpath
\moveto(439.34369257,691.20938591)
\curveto(439.48058171,691.20938591)(439.64235977,691.20316368)(439.82902678,691.19071921)
\curveto(440.01569378,691.17827474)(440.16502738,691.15960804)(440.27702758,691.13471911)
\lineto(440.07169388,688.52138109)
\curveto(439.97213814,688.54627003)(439.84147124,688.56493673)(439.67969317,688.57738119)
\curveto(439.51791511,688.60227013)(439.37480374,688.61471459)(439.25035907,688.61471459)
\curveto(438.77746934,688.61471459)(438.3232463,688.52760333)(437.88768996,688.35338079)
\curveto(437.45213363,688.19160272)(437.09746632,687.92404669)(436.82368806,687.55071269)
\curveto(436.56235425,687.17737868)(436.43168735,686.66715555)(436.43168735,686.02004328)
\lineto(436.43168735,680.83070064)
\lineto(433.65034903,680.83070064)
\lineto(433.65034903,691.02271891)
\lineto(435.75968615,691.02271891)
\lineto(436.17035355,689.3053825)
\lineto(436.30102045,689.3053825)
\curveto(436.59968765,689.8280501)(437.01035506,690.27605091)(437.53302266,690.64938491)
\curveto(438.05569026,691.02271891)(438.6592469,691.20938591)(439.34369257,691.20938591)
\closepath
}
}
{
\newrgbcolor{curcolor}{0 0 0}
\pscustom[linestyle=none,fillstyle=solid,fillcolor=curcolor]
{
\newpath
\moveto(449.2370626,683.85470606)
\curveto(449.2370626,682.82181532)(448.86995083,682.02536945)(448.1357273,681.46536844)
\curveto(447.41394822,680.91781191)(446.33127962,680.64403364)(444.88772147,680.64403364)
\curveto(444.17838687,680.64403364)(443.568608,680.69381151)(443.05838486,680.79336724)
\curveto(442.54816172,680.88047851)(442.03793859,681.02981211)(441.52771545,681.24136804)
\lineto(441.52771545,683.53737216)
\curveto(442.07527199,683.28848282)(442.66638416,683.08314912)(443.30105196,682.92137105)
\curveto(443.93571977,682.75959299)(444.49572077,682.67870395)(444.98105497,682.67870395)
\curveto(445.51616704,682.67870395)(445.90194551,682.75959299)(446.13839038,682.92137105)
\curveto(446.37483525,683.08314912)(446.49305768,683.29470506)(446.49305768,683.55603886)
\curveto(446.49305768,683.73026139)(446.44327982,683.88581723)(446.34372408,684.02270636)
\curveto(446.25661282,684.1595955)(446.05750135,684.31515133)(445.74638968,684.48937387)
\curveto(445.43527801,684.6635964)(444.94994381,684.8875968)(444.29038707,685.16137507)
\curveto(443.6432748,685.43515334)(443.11438496,685.70270937)(442.70371756,685.96404318)
\curveto(442.30549462,686.23782144)(442.00682742,686.56137758)(441.80771595,686.93471158)
\curveto(441.60860448,687.32049005)(441.50904875,687.79960202)(441.50904875,688.37204749)
\curveto(441.50904875,689.31782697)(441.87616052,690.02716157)(442.61038406,690.50005131)
\curveto(443.3446076,690.97294104)(444.32149824,691.20938591)(445.54105598,691.20938591)
\curveto(446.17572378,691.20938591)(446.77928042,691.14716358)(447.35172589,691.02271891)
\curveto(447.92417136,690.89827424)(448.51528353,690.69294054)(449.1250624,690.40671781)
\lineto(448.2850609,688.40938089)
\curveto(447.78728223,688.62093683)(447.31439249,688.79515936)(446.86639169,688.9320485)
\curveto(446.41839088,689.0813821)(445.96416785,689.1560489)(445.50372258,689.1560489)
\curveto(444.68238777,689.1560489)(444.27172037,688.9320485)(444.27172037,688.48404769)
\curveto(444.27172037,688.32226962)(444.32149824,688.17293602)(444.42105397,688.03604689)
\curveto(444.53305417,687.91160222)(444.73838787,687.77471309)(445.03705507,687.62537949)
\curveto(445.34816674,687.47604589)(445.80238978,687.27693442)(446.39972418,687.02804508)
\curveto(446.98461412,686.79160021)(447.48861502,686.54271088)(447.91172689,686.28137708)
\curveto(448.33483876,686.03248774)(448.6583949,685.71515384)(448.8823953,685.32937537)
\curveto(449.11884017,684.9435969)(449.2370626,684.45204047)(449.2370626,683.85470606)
\closepath
}
}
{
\newrgbcolor{curcolor}{0 0 0}
\pscustom[linestyle=none,fillstyle=solid,fillcolor=curcolor]
{
\newpath
\moveto(216.73455065,659.17172351)
\curveto(215.21632571,659.17172351)(214.0403236,659.58861314)(213.20654433,660.42239242)
\curveto(212.38520952,661.25617169)(211.97454212,662.5815074)(211.97454212,664.39839954)
\curveto(211.97454212,665.64284622)(212.18609805,666.65707026)(212.60920992,667.44107166)
\curveto(213.03232179,668.22507307)(213.61721173,668.80374077)(214.36387973,669.17707478)
\curveto(215.12299221,669.55040878)(215.99410488,669.73707578)(216.97721775,669.73707578)
\curveto(217.67410789,669.73707578)(218.27766453,669.66863121)(218.78788767,669.53174208)
\curveto(219.31055527,669.39485294)(219.76477831,669.23307488)(220.15055677,669.04640787)
\lineto(219.32922197,666.89973736)
\curveto(218.89366563,667.07395989)(218.48299823,667.21707126)(218.09721976,667.32907146)
\curveto(217.72388576,667.44107166)(217.35055176,667.49707176)(216.97721775,667.49707176)
\curveto(215.53365961,667.49707176)(214.81188054,666.47040326)(214.81188054,664.41706624)
\curveto(214.81188054,663.39661997)(214.99854754,662.64372973)(215.37188154,662.15839553)
\curveto(215.75766001,661.67306132)(216.29277208,661.43039422)(216.97721775,661.43039422)
\curveto(217.56210769,661.43039422)(218.07855306,661.50506102)(218.52655386,661.65439462)
\curveto(218.97455467,661.81617269)(219.410111,662.03395086)(219.83322287,662.30772913)
\lineto(219.83322287,659.93705821)
\curveto(219.410111,659.66327994)(218.9621102,659.47039071)(218.48922046,659.35839051)
\curveto(218.02877519,659.23394584)(217.44388526,659.17172351)(216.73455065,659.17172351)
\closepath
}
}
{
\newrgbcolor{curcolor}{0 0 0}
\pscustom[linestyle=none,fillstyle=solid,fillcolor=curcolor]
{
\newpath
\moveto(227.87856933,669.73707578)
\curveto(228.01545847,669.73707578)(228.17723654,669.73085355)(228.36390354,669.71840908)
\curveto(228.55057054,669.70596461)(228.69990414,669.68729791)(228.81190434,669.66240898)
\lineto(228.60657064,667.04907096)
\curveto(228.50701491,667.07395989)(228.376348,667.09262659)(228.21456994,667.10507106)
\curveto(228.05279187,667.12995999)(227.9096805,667.14240446)(227.78523583,667.14240446)
\curveto(227.3123461,667.14240446)(226.85812306,667.05529319)(226.42256672,666.88107066)
\curveto(225.98701039,666.71929259)(225.63234309,666.45173656)(225.35856482,666.07840255)
\curveto(225.09723102,665.70506855)(224.96656411,665.19484541)(224.96656411,664.54773314)
\lineto(224.96656411,659.35839051)
\lineto(222.1852258,659.35839051)
\lineto(222.1852258,669.55040878)
\lineto(224.29456291,669.55040878)
\lineto(224.70523031,667.83307237)
\lineto(224.83589721,667.83307237)
\curveto(225.13456442,668.35573997)(225.54523182,668.80374077)(226.06789942,669.17707478)
\curveto(226.59056703,669.55040878)(227.19412366,669.73707578)(227.87856933,669.73707578)
\closepath
}
}
{
\newrgbcolor{curcolor}{0 0 0}
\pscustom[linestyle=none,fillstyle=solid,fillcolor=curcolor]
{
\newpath
\moveto(234.48658019,669.73707578)
\curveto(235.89280493,669.73707578)(237.00658471,669.33263061)(237.82791951,668.52374027)
\curveto(238.64925432,667.7272944)(239.05992172,666.58862569)(239.05992172,665.10773415)
\lineto(239.05992172,663.76373174)
\lineto(232.48924328,663.76373174)
\curveto(232.51413221,662.97973033)(232.74435485,662.36372923)(233.17991118,661.91572843)
\curveto(233.62791199,661.46772762)(234.24391309,661.24372722)(235.0279145,661.24372722)
\curveto(235.67502677,661.24372722)(236.26613894,661.30594955)(236.80125101,661.43039422)
\curveto(237.34880754,661.56728336)(237.90880855,661.77261706)(238.48125402,662.04639533)
\lineto(238.48125402,659.89972481)
\curveto(237.97103088,659.65083548)(237.44214105,659.47039071)(236.89458451,659.35839051)
\curveto(236.34702797,659.23394584)(235.681249,659.17172351)(234.89724759,659.17172351)
\curveto(233.87680132,659.17172351)(232.97457748,659.35839051)(232.19057608,659.73172451)
\curveto(231.40657467,660.11750298)(230.79057357,660.68994845)(230.34257276,661.44906092)
\curveto(229.89457196,662.22061786)(229.67057156,663.1975085)(229.67057156,664.37973284)
\curveto(229.67057156,665.56195718)(229.86968303,666.55129229)(230.26790596,667.34773816)
\curveto(230.67857337,668.14418403)(231.2447966,668.74151844)(231.96657567,669.13974138)
\curveto(232.68835475,669.53796431)(233.52835625,669.73707578)(234.48658019,669.73707578)
\closepath
\moveto(234.50524689,667.75840557)
\curveto(233.95769035,667.75840557)(233.50968955,667.58418303)(233.16124448,667.23573796)
\curveto(232.81279941,666.88729289)(232.60746571,666.34595859)(232.54524338,665.61173505)
\lineto(236.4465837,665.61173505)
\curveto(236.43413924,666.22151392)(236.26613894,666.73173706)(235.9425828,667.14240446)
\curveto(235.63147113,667.55307186)(235.15235916,667.75840557)(234.50524689,667.75840557)
\closepath
}
}
{
\newrgbcolor{curcolor}{0 0 0}
\pscustom[linestyle=none,fillstyle=solid,fillcolor=curcolor]
{
\newpath
\moveto(245.49992629,669.75574248)
\curveto(246.86881763,669.75574248)(247.91415284,669.45707528)(248.63593191,668.85974087)
\curveto(249.37015545,668.27485094)(249.73726722,667.3726271)(249.73726722,666.15306935)
\lineto(249.73726722,659.35839051)
\lineto(247.79593041,659.35839051)
\lineto(247.2545961,660.73972632)
\lineto(247.1799293,660.73972632)
\curveto(246.74437297,660.19216978)(246.2839277,659.79394684)(245.79859349,659.54505751)
\curveto(245.31325929,659.29616817)(244.64748032,659.17172351)(243.80125658,659.17172351)
\curveto(242.89281051,659.17172351)(242.13992027,659.43305731)(241.54258586,659.95572491)
\curveto(240.94525146,660.47839252)(240.64658426,661.29350509)(240.64658426,662.40106263)
\curveto(240.64658426,663.48373124)(241.02614049,664.28017711)(241.78525297,664.79040025)
\curveto(242.54436544,665.30062338)(243.68303415,665.58684612)(245.20125909,665.64906845)
\lineto(246.9745956,665.70506855)
\lineto(246.9745956,666.15306935)
\curveto(246.9745956,666.68818142)(246.83148423,667.08018213)(246.5452615,667.32907146)
\curveto(246.27148323,667.5779608)(245.88570476,667.70240547)(245.38792609,667.70240547)
\curveto(244.89014742,667.70240547)(244.40481322,667.62773866)(243.93192348,667.47840506)
\curveto(243.45903374,667.34151593)(242.98614401,667.16729339)(242.51325427,666.95573746)
\lineto(241.59858597,668.84107417)
\curveto(242.13369804,669.11485244)(242.73725467,669.33263061)(243.40925588,669.49440868)
\curveto(244.08125708,669.66863121)(244.77814722,669.75574248)(245.49992629,669.75574248)
\closepath
\moveto(246.9745956,664.08106564)
\lineto(245.89192699,664.04373224)
\curveto(244.99592539,664.01884331)(244.37370205,663.85706524)(244.02525698,663.55839804)
\curveto(243.67681191,663.25973083)(243.50258938,662.86773013)(243.50258938,662.38239593)
\curveto(243.50258938,661.95928406)(243.62703405,661.65439462)(243.87592338,661.46772762)
\curveto(244.12481272,661.29350509)(244.44836885,661.20639382)(244.84659179,661.20639382)
\curveto(245.44392619,661.20639382)(245.94792709,661.38061636)(246.3585945,661.72906142)
\curveto(246.7692619,662.08995096)(246.9745956,662.59395186)(246.9745956,663.24106413)
\closepath
}
}
{
\newrgbcolor{curcolor}{0 0 0}
\pscustom[linestyle=none,fillstyle=solid,fillcolor=curcolor]
{
\newpath
\moveto(256.88661507,661.39306082)
\curveto(257.19772674,661.39306082)(257.49639394,661.41794976)(257.78261668,661.46772762)
\curveto(258.06883941,661.52994996)(258.35506215,661.61083899)(258.64128488,661.71039472)
\lineto(258.64128488,659.63839101)
\curveto(258.34261768,659.50150188)(257.96928368,659.38950167)(257.52128288,659.30239041)
\curveto(257.08572654,659.21527914)(256.60661457,659.17172351)(256.08394697,659.17172351)
\curveto(255.47416809,659.17172351)(254.92661156,659.27127924)(254.44127735,659.47039071)
\curveto(253.96838762,659.66950218)(253.58883138,660.01172501)(253.30260865,660.49705922)
\curveto(253.02883038,660.98239342)(252.89194124,661.66683909)(252.89194124,662.55039623)
\lineto(252.89194124,667.45973836)
\lineto(251.56660553,667.45973836)
\lineto(251.56660553,668.63574047)
\lineto(253.09727495,669.56907548)
\lineto(253.89994305,671.71574599)
\lineto(255.67327956,671.71574599)
\lineto(255.67327956,669.55040878)
\lineto(258.52928468,669.55040878)
\lineto(258.52928468,667.45973836)
\lineto(255.67327956,667.45973836)
\lineto(255.67327956,662.55039623)
\curveto(255.67327956,662.16461776)(255.78527976,661.87217279)(256.00928017,661.67306132)
\curveto(256.23328057,661.48639432)(256.52572554,661.39306082)(256.88661507,661.39306082)
\closepath
}
}
{
\newrgbcolor{curcolor}{0 0 0}
\pscustom[linestyle=none,fillstyle=solid,fillcolor=curcolor]
{
\newpath
\moveto(264.89463035,669.73707578)
\curveto(266.30085509,669.73707578)(267.41463486,669.33263061)(268.23596967,668.52374027)
\curveto(269.05730447,667.7272944)(269.46797188,666.58862569)(269.46797188,665.10773415)
\lineto(269.46797188,663.76373174)
\lineto(262.89729343,663.76373174)
\curveto(262.92218237,662.97973033)(263.152405,662.36372923)(263.58796134,661.91572843)
\curveto(264.03596214,661.46772762)(264.65196324,661.24372722)(265.43596465,661.24372722)
\curveto(266.08307692,661.24372722)(266.67418909,661.30594955)(267.20930116,661.43039422)
\curveto(267.7568577,661.56728336)(268.3168587,661.77261706)(268.88930417,662.04639533)
\lineto(268.88930417,659.89972481)
\curveto(268.37908104,659.65083548)(267.8501912,659.47039071)(267.30263466,659.35839051)
\curveto(266.75507812,659.23394584)(266.08929915,659.17172351)(265.30529775,659.17172351)
\curveto(264.28485147,659.17172351)(263.38262764,659.35839051)(262.59862623,659.73172451)
\curveto(261.81462482,660.11750298)(261.19862372,660.68994845)(260.75062292,661.44906092)
\curveto(260.30262211,662.22061786)(260.07862171,663.1975085)(260.07862171,664.37973284)
\curveto(260.07862171,665.56195718)(260.27773318,666.55129229)(260.67595612,667.34773816)
\curveto(261.08662352,668.14418403)(261.65284676,668.74151844)(262.37462583,669.13974138)
\curveto(263.0964049,669.53796431)(263.93640641,669.73707578)(264.89463035,669.73707578)
\closepath
\moveto(264.91329705,667.75840557)
\curveto(264.36574051,667.75840557)(263.91773971,667.58418303)(263.56929464,667.23573796)
\curveto(263.22084957,666.88729289)(263.01551587,666.34595859)(262.95329353,665.61173505)
\lineto(266.85463386,665.61173505)
\curveto(266.84218939,666.22151392)(266.67418909,666.73173706)(266.35063296,667.14240446)
\curveto(266.03952129,667.55307186)(265.56040932,667.75840557)(264.91329705,667.75840557)
\closepath
}
}
{
\newrgbcolor{curcolor}{0 0 0}
\pscustom[linestyle=none,fillstyle=solid,fillcolor=curcolor]
{
\newpath
\moveto(275.83330965,672.6864144)
\curveto(277.65020179,672.6864144)(278.98798197,672.35663603)(279.84665017,671.69707929)
\curveto(280.71776285,671.03752255)(281.15331918,670.03574298)(281.15331918,668.69174057)
\curveto(281.15331918,668.0819617)(281.03509675,667.54684963)(280.79865188,667.08640436)
\curveto(280.57465148,666.63840356)(280.26976204,666.25262509)(279.88398357,665.92906895)
\curveto(279.51064957,665.61795728)(279.1062044,665.36284572)(278.67064806,665.16373425)
\lineto(282.59065509,659.35839051)
\lineto(279.45464947,659.35839051)
\lineto(276.28131045,664.47306634)
\lineto(274.76930774,664.47306634)
\lineto(274.76930774,659.35839051)
\lineto(271.95063602,659.35839051)
\lineto(271.95063602,672.6864144)
\closepath
\moveto(275.62797594,670.37174358)
\lineto(274.76930774,670.37174358)
\lineto(274.76930774,666.76907046)
\lineto(275.68397604,666.76907046)
\curveto(276.61731105,666.76907046)(277.28309002,666.92462629)(277.68131296,667.23573796)
\curveto(278.09198036,667.54684963)(278.29731406,668.0072949)(278.29731406,668.61707377)
\curveto(278.29731406,669.25174158)(278.07953589,669.69974238)(277.64397956,669.96107618)
\curveto(277.20842322,670.23485445)(276.53642202,670.37174358)(275.62797594,670.37174358)
\closepath
}
}
{
\newrgbcolor{curcolor}{0 0 0}
\pscustom[linestyle=none,fillstyle=solid,fillcolor=curcolor]
{
\newpath
\moveto(293.30534131,664.47306634)
\curveto(293.30534131,662.78061886)(292.85734051,661.47394986)(291.9613389,660.55305932)
\curveto(291.07778176,659.63216878)(289.87066848,659.17172351)(288.33999907,659.17172351)
\curveto(287.3942196,659.17172351)(286.54799586,659.37705721)(285.80132786,659.78772461)
\curveto(285.06710432,660.19839201)(284.48843661,660.79572642)(284.06532475,661.57972782)
\curveto(283.64221288,662.3761737)(283.43065694,663.34061987)(283.43065694,664.47306634)
\curveto(283.43065694,666.16551382)(283.87243551,667.4659606)(284.75599265,668.37440667)
\curveto(285.63954979,669.28285274)(286.8528853,669.73707578)(288.39599917,669.73707578)
\curveto(289.35422311,669.73707578)(290.20044685,669.53174208)(290.93467039,669.12107468)
\curveto(291.66889393,668.71040727)(292.24756163,668.11307287)(292.6706735,667.32907146)
\curveto(293.09378537,666.54507006)(293.30534131,665.59306835)(293.30534131,664.47306634)
\closepath
\moveto(286.26799536,664.47306634)
\curveto(286.26799536,663.46506454)(286.42977343,662.69972983)(286.75332956,662.17706223)
\curveto(287.08933017,661.66683909)(287.63066447,661.41172752)(288.37733247,661.41172752)
\curveto(289.11155601,661.41172752)(289.64044585,661.66683909)(289.96400199,662.17706223)
\curveto(290.30000259,662.69972983)(290.46800289,663.46506454)(290.46800289,664.47306634)
\curveto(290.46800289,665.48106815)(290.30000259,666.23395839)(289.96400199,666.73173706)
\curveto(289.64044585,667.2419602)(289.10533378,667.49707176)(288.35866577,667.49707176)
\curveto(287.62444224,667.49707176)(287.08933017,667.2419602)(286.75332956,666.73173706)
\curveto(286.42977343,666.23395839)(286.26799536,665.48106815)(286.26799536,664.47306634)
\closepath
}
}
{
\newrgbcolor{curcolor}{0 0 0}
\pscustom[linestyle=none,fillstyle=solid,fillcolor=curcolor]
{
\newpath
\moveto(304.8600134,664.47306634)
\curveto(304.8600134,662.78061886)(304.4120126,661.47394986)(303.51601099,660.55305932)
\curveto(302.63245386,659.63216878)(301.42534058,659.17172351)(299.89467117,659.17172351)
\curveto(298.9488917,659.17172351)(298.10266796,659.37705721)(297.35599995,659.78772461)
\curveto(296.62177641,660.19839201)(296.04310871,660.79572642)(295.61999684,661.57972782)
\curveto(295.19688497,662.3761737)(294.98532904,663.34061987)(294.98532904,664.47306634)
\curveto(294.98532904,666.16551382)(295.42710761,667.4659606)(296.31066475,668.37440667)
\curveto(297.19422188,669.28285274)(298.40755739,669.73707578)(299.95067127,669.73707578)
\curveto(300.90889521,669.73707578)(301.75511895,669.53174208)(302.48934249,669.12107468)
\curveto(303.22356603,668.71040727)(303.80223373,668.11307287)(304.2253456,667.32907146)
\curveto(304.64845747,666.54507006)(304.8600134,665.59306835)(304.8600134,664.47306634)
\closepath
\moveto(297.82266746,664.47306634)
\curveto(297.82266746,663.46506454)(297.98444552,662.69972983)(298.30800166,662.17706223)
\curveto(298.64400226,661.66683909)(299.18533657,661.41172752)(299.93200457,661.41172752)
\curveto(300.66622811,661.41172752)(301.19511795,661.66683909)(301.51867408,662.17706223)
\curveto(301.85467468,662.69972983)(302.02267498,663.46506454)(302.02267498,664.47306634)
\curveto(302.02267498,665.48106815)(301.85467468,666.23395839)(301.51867408,666.73173706)
\curveto(301.19511795,667.2419602)(300.66000588,667.49707176)(299.91333787,667.49707176)
\curveto(299.17911433,667.49707176)(298.64400226,667.2419602)(298.30800166,666.73173706)
\curveto(297.98444552,666.23395839)(297.82266746,665.48106815)(297.82266746,664.47306634)
\closepath
}
}
{
\newrgbcolor{curcolor}{0 0 0}
\pscustom[linestyle=none,fillstyle=solid,fillcolor=curcolor]
{
\newpath
\moveto(319.12136465,669.73707578)
\curveto(320.27870006,669.73707578)(321.14981273,669.43840858)(321.73470267,668.84107417)
\curveto(322.33203707,668.25618424)(322.63070427,667.31040476)(322.63070427,666.00373575)
\lineto(322.63070427,659.35839051)
\lineto(319.84936595,659.35839051)
\lineto(319.84936595,665.31306785)
\curveto(319.84936595,666.78151493)(319.33914282,667.51573846)(318.31869654,667.51573846)
\curveto(317.584473,667.51573846)(317.0618054,667.25440466)(316.75069373,666.73173706)
\curveto(316.43958206,666.20906946)(316.28402623,665.45617922)(316.28402623,664.47306634)
\lineto(316.28402623,659.35839051)
\lineto(313.50268791,659.35839051)
\lineto(313.50268791,665.31306785)
\curveto(313.50268791,666.78151493)(312.99246477,667.51573846)(311.9720185,667.51573846)
\curveto(311.20046156,667.51573846)(310.66534949,667.2232935)(310.36668229,666.63840356)
\curveto(310.08045955,666.06595809)(309.93734818,665.23840105)(309.93734818,664.15573244)
\lineto(309.93734818,659.35839051)
\lineto(307.15600987,659.35839051)
\lineto(307.15600987,669.55040878)
\lineto(309.28401368,669.55040878)
\lineto(309.65734768,668.24373977)
\lineto(309.80668128,668.24373977)
\curveto(310.11779295,668.76640737)(310.54090482,669.14596361)(311.07601689,669.38240848)
\curveto(311.62357343,669.61885335)(312.18979667,669.73707578)(312.7746866,669.73707578)
\curveto(313.52135461,669.73707578)(314.14980018,669.61263111)(314.66002332,669.36374178)
\curveto(315.18269092,669.12729691)(315.58713609,668.75396291)(315.87335883,668.24373977)
\lineto(316.11602593,668.24373977)
\curveto(316.4271376,668.76640737)(316.8564717,669.14596361)(317.40402824,669.38240848)
\curveto(317.96402924,669.61885335)(318.53647471,669.73707578)(319.12136465,669.73707578)
\closepath
}
}
{
\newrgbcolor{curcolor}{0 0 0}
\pscustom[linestyle=none,fillstyle=solid,fillcolor=curcolor]
{
\newpath
\moveto(325.09471267,660.66505952)
\curveto(325.09471267,661.23750499)(325.2502685,661.63572792)(325.56138017,661.85972833)
\curveto(325.87249184,662.09617319)(326.25204808,662.21439563)(326.70004888,662.21439563)
\curveto(327.13560521,662.21439563)(327.50893922,662.09617319)(327.82005089,661.85972833)
\curveto(328.13116255,661.63572792)(328.28671839,661.23750499)(328.28671839,660.66505952)
\curveto(328.28671839,660.11750298)(328.13116255,659.71928004)(327.82005089,659.47039071)
\curveto(327.50893922,659.23394584)(327.13560521,659.11572341)(326.70004888,659.11572341)
\curveto(326.25204808,659.11572341)(325.87249184,659.23394584)(325.56138017,659.47039071)
\curveto(325.2502685,659.71928004)(325.09471267,660.11750298)(325.09471267,660.66505952)
\closepath
}
}
{
\newrgbcolor{curcolor}{0 0 0}
\pscustom[linestyle=none,fillstyle=solid,fillcolor=curcolor]
{
\newpath
\moveto(333.58805895,673.54508261)
\lineto(333.58805895,670.65174409)
\curveto(333.58805895,670.14152095)(333.56939225,669.65618675)(333.53205885,669.19574148)
\curveto(333.50716991,668.74774067)(333.48228098,668.43040677)(333.45739205,668.24373977)
\lineto(333.60672565,668.24373977)
\curveto(333.93028178,668.76640737)(334.34717142,669.14596361)(334.85739456,669.38240848)
\curveto(335.36761769,669.61885335)(335.93384093,669.73707578)(336.55606427,669.73707578)
\curveto(337.65117734,669.73707578)(338.53473448,669.43840858)(339.20673568,668.84107417)
\curveto(339.87873689,668.25618424)(340.21473749,667.31040476)(340.21473749,666.00373575)
\lineto(340.21473749,659.35839051)
\lineto(337.43339917,659.35839051)
\lineto(337.43339917,665.31306785)
\curveto(337.43339917,666.78151493)(336.88584264,667.51573846)(335.79072956,667.51573846)
\curveto(334.95695029,667.51573846)(334.37828259,667.2232935)(334.05472645,666.63840356)
\curveto(333.74361478,666.06595809)(333.58805895,665.23840105)(333.58805895,664.15573244)
\lineto(333.58805895,659.35839051)
\lineto(330.80672063,659.35839051)
\lineto(330.80672063,673.54508261)
\closepath
}
}
{
\newrgbcolor{curcolor}{0 0 0}
\pscustom[linestyle=none,fillstyle=solid,fillcolor=curcolor]
{
\newpath
\moveto(347.2520765,669.75574248)
\curveto(348.62096784,669.75574248)(349.66630305,669.45707528)(350.38808212,668.85974087)
\curveto(351.12230566,668.27485094)(351.48941743,667.3726271)(351.48941743,666.15306935)
\lineto(351.48941743,659.35839051)
\lineto(349.54808061,659.35839051)
\lineto(349.00674631,660.73972632)
\lineto(348.93207951,660.73972632)
\curveto(348.49652317,660.19216978)(348.0360779,659.79394684)(347.5507437,659.54505751)
\curveto(347.0654095,659.29616817)(346.39963053,659.17172351)(345.55340679,659.17172351)
\curveto(344.64496071,659.17172351)(343.89207048,659.43305731)(343.29473607,659.95572491)
\curveto(342.69740167,660.47839252)(342.39873447,661.29350509)(342.39873447,662.40106263)
\curveto(342.39873447,663.48373124)(342.7782907,664.28017711)(343.53740317,664.79040025)
\curveto(344.29651565,665.30062338)(345.43518435,665.58684612)(346.9534093,665.64906845)
\lineto(348.72674581,665.70506855)
\lineto(348.72674581,666.15306935)
\curveto(348.72674581,666.68818142)(348.58363444,667.08018213)(348.29741171,667.32907146)
\curveto(348.02363344,667.5779608)(347.63785497,667.70240547)(347.1400763,667.70240547)
\curveto(346.64229763,667.70240547)(346.15696343,667.62773866)(345.68407369,667.47840506)
\curveto(345.21118395,667.34151593)(344.73829422,667.16729339)(344.26540448,666.95573746)
\lineto(343.35073617,668.84107417)
\curveto(343.88584824,669.11485244)(344.48940488,669.33263061)(345.16140608,669.49440868)
\curveto(345.83340729,669.66863121)(346.53029743,669.75574248)(347.2520765,669.75574248)
\closepath
\moveto(348.72674581,664.08106564)
\lineto(347.6440772,664.04373224)
\curveto(346.7480756,664.01884331)(346.12585226,663.85706524)(345.77740719,663.55839804)
\curveto(345.42896212,663.25973083)(345.25473959,662.86773013)(345.25473959,662.38239593)
\curveto(345.25473959,661.95928406)(345.37918425,661.65439462)(345.62807359,661.46772762)
\curveto(345.87696292,661.29350509)(346.20051906,661.20639382)(346.59874199,661.20639382)
\curveto(347.1960764,661.20639382)(347.7000773,661.38061636)(348.11074471,661.72906142)
\curveto(348.52141211,662.08995096)(348.72674581,662.59395186)(348.72674581,663.24106413)
\closepath
}
}
{
\newrgbcolor{curcolor}{0 0 0}
\pscustom[linestyle=none,fillstyle=solid,fillcolor=curcolor]
{
\newpath
\moveto(360.13210129,669.73707578)
\curveto(361.22721436,669.73707578)(362.10454927,669.43840858)(362.76410601,668.84107417)
\curveto(363.42366274,668.25618424)(363.75344111,667.31040476)(363.75344111,666.00373575)
\lineto(363.75344111,659.35839051)
\lineto(360.97210279,659.35839051)
\lineto(360.97210279,665.31306785)
\curveto(360.97210279,666.04729139)(360.84143589,666.59484792)(360.58010209,666.95573746)
\curveto(360.31876829,667.32907146)(359.90187865,667.51573846)(359.32943318,667.51573846)
\curveto(358.48320944,667.51573846)(357.90454174,667.2232935)(357.59343007,666.63840356)
\curveto(357.2823184,666.06595809)(357.12676257,665.23840105)(357.12676257,664.15573244)
\lineto(357.12676257,659.35839051)
\lineto(354.34542425,659.35839051)
\lineto(354.34542425,669.55040878)
\lineto(356.47342806,669.55040878)
\lineto(356.84676207,668.24373977)
\lineto(356.99609567,668.24373977)
\curveto(357.3196518,668.76640737)(357.76143037,669.14596361)(358.32143138,669.38240848)
\curveto(358.89387685,669.61885335)(359.49743348,669.73707578)(360.13210129,669.73707578)
\closepath
}
}
{
\newrgbcolor{curcolor}{0 0 0}
\pscustom[linestyle=none,fillstyle=solid,fillcolor=curcolor]
{
\newpath
\moveto(369.83878604,659.17172351)
\curveto(368.70633957,659.17172351)(367.7792268,659.61350208)(367.05744772,660.49705922)
\curveto(366.34811312,661.39306082)(365.99344582,662.70595206)(365.99344582,664.43573294)
\curveto(365.99344582,666.17795829)(366.35433535,667.49707176)(367.07611443,668.39307337)
\curveto(367.7978935,669.28907498)(368.74367297,669.73707578)(369.91345284,669.73707578)
\curveto(370.64767638,669.73707578)(371.25123302,669.59396441)(371.72412276,669.30774168)
\curveto(372.19701249,669.02151894)(372.5703465,668.66685164)(372.84412476,668.24373977)
\lineto(372.93745827,668.24373977)
\curveto(372.90012486,668.44285124)(372.85656923,668.72907397)(372.80679136,669.10240797)
\curveto(372.7570135,669.48818644)(372.73212456,669.88018715)(372.73212456,670.27841008)
\lineto(372.73212456,673.54508261)
\lineto(375.51346288,673.54508261)
\lineto(375.51346288,659.35839051)
\lineto(373.38545907,659.35839051)
\lineto(372.84412476,660.68372622)
\lineto(372.73212456,660.68372622)
\curveto(372.4583463,660.26061435)(372.09123453,659.89972481)(371.63078926,659.60105761)
\curveto(371.17034399,659.31483487)(370.57300958,659.17172351)(369.83878604,659.17172351)
\closepath
\moveto(370.80945445,661.39306082)
\curveto(371.56856692,661.39306082)(372.10367899,661.61706122)(372.41479066,662.06506203)
\curveto(372.72590233,662.5255073)(372.89390263,663.20995297)(372.91879157,664.11839904)
\lineto(372.91879157,664.41706624)
\curveto(372.91879157,665.40017912)(372.76323573,666.15306935)(372.45212406,666.67573696)
\curveto(372.15345686,667.21084903)(371.59345586,667.47840506)(370.77212105,667.47840506)
\curveto(370.16234218,667.47840506)(369.68323021,667.21084903)(369.33478514,666.67573696)
\curveto(368.98634007,666.15306935)(368.81211754,665.39395688)(368.81211754,664.39839954)
\curveto(368.81211754,663.4028422)(368.98634007,662.64995196)(369.33478514,662.13972883)
\curveto(369.68323021,661.64195016)(370.17478665,661.39306082)(370.80945445,661.39306082)
\closepath
}
}
{
\newrgbcolor{curcolor}{0 0 0}
\pscustom[linestyle=none,fillstyle=solid,fillcolor=curcolor]
{
\newpath
\moveto(381.20679461,659.35839051)
\lineto(378.42545629,659.35839051)
\lineto(378.42545629,673.54508261)
\lineto(381.20679461,673.54508261)
\closepath
}
}
{
\newrgbcolor{curcolor}{0 0 0}
\pscustom[linestyle=none,fillstyle=solid,fillcolor=curcolor]
{
\newpath
\moveto(388.31880787,669.73707578)
\curveto(389.72503261,669.73707578)(390.83881239,669.33263061)(391.66014719,668.52374027)
\curveto(392.481482,667.7272944)(392.8921494,666.58862569)(392.8921494,665.10773415)
\lineto(392.8921494,663.76373174)
\lineto(386.32147095,663.76373174)
\curveto(386.34635989,662.97973033)(386.57658252,662.36372923)(387.01213886,661.91572843)
\curveto(387.46013966,661.46772762)(388.07614077,661.24372722)(388.86014217,661.24372722)
\curveto(389.50725444,661.24372722)(390.09836661,661.30594955)(390.63347868,661.43039422)
\curveto(391.18103522,661.56728336)(391.74103622,661.77261706)(392.3134817,662.04639533)
\lineto(392.3134817,659.89972481)
\curveto(391.80325856,659.65083548)(391.27436872,659.47039071)(390.72681218,659.35839051)
\curveto(390.17925565,659.23394584)(389.51347668,659.17172351)(388.72947527,659.17172351)
\curveto(387.709029,659.17172351)(386.80680516,659.35839051)(386.02280375,659.73172451)
\curveto(385.23880235,660.11750298)(384.62280124,660.68994845)(384.17480044,661.44906092)
\curveto(383.72679964,662.22061786)(383.50279923,663.1975085)(383.50279923,664.37973284)
\curveto(383.50279923,665.56195718)(383.7019107,666.55129229)(384.10013364,667.34773816)
\curveto(384.51080104,668.14418403)(385.07702428,668.74151844)(385.79880335,669.13974138)
\curveto(386.52058242,669.53796431)(387.36058393,669.73707578)(388.31880787,669.73707578)
\closepath
\moveto(388.33747457,667.75840557)
\curveto(387.78991803,667.75840557)(387.34191723,667.58418303)(386.99347216,667.23573796)
\curveto(386.64502709,666.88729289)(386.43969339,666.34595859)(386.37747105,665.61173505)
\lineto(390.27881138,665.61173505)
\curveto(390.26636691,666.22151392)(390.09836661,666.73173706)(389.77481048,667.14240446)
\curveto(389.46369881,667.55307186)(388.98458684,667.75840557)(388.33747457,667.75840557)
\closepath
}
}
{
\newrgbcolor{curcolor}{0 0 0}
\pscustom[linestyle=none,fillstyle=solid,fillcolor=curcolor]
{
\newpath
\moveto(397.93215146,673.54508261)
\lineto(397.93215146,670.24107668)
\curveto(397.93215146,669.85529821)(397.91970699,669.47574198)(397.89481806,669.10240797)
\curveto(397.86992912,668.72907397)(397.84504019,668.436629)(397.82015126,668.22507307)
\lineto(397.93215146,668.22507307)
\curveto(398.20592973,668.64818494)(398.5730415,669.00285224)(399.03348677,669.28907498)
\curveto(399.49393204,669.58774218)(400.09126644,669.73707578)(400.82548998,669.73707578)
\curveto(401.97038092,669.73707578)(402.89749369,669.28907498)(403.6068283,668.39307337)
\curveto(404.3161629,667.50951623)(404.6708302,666.20284722)(404.6708302,664.47306634)
\curveto(404.6708302,662.730841)(404.30994067,661.41172752)(403.5881616,660.51572592)
\curveto(402.86638253,659.61972431)(401.92060305,659.17172351)(400.75082318,659.17172351)
\curveto(400.00415517,659.17172351)(399.413043,659.30239041)(398.97748667,659.56372421)
\curveto(398.5543748,659.83750248)(398.20592973,660.14239191)(397.93215146,660.47839252)
\lineto(397.74548446,660.47839252)
\lineto(397.27881695,659.35839051)
\lineto(395.15081314,659.35839051)
\lineto(395.15081314,673.54508261)
\closepath
\moveto(399.92948837,667.51573846)
\curveto(399.2077093,667.51573846)(398.69748616,667.28551583)(398.39881896,666.82507056)
\curveto(398.10015176,666.37706976)(397.94459593,665.69884632)(397.93215146,664.79040025)
\lineto(397.93215146,664.49173304)
\curveto(397.93215146,663.50862017)(398.07526283,662.7495077)(398.36148556,662.21439563)
\curveto(398.66015276,661.69172802)(399.19526483,661.43039422)(399.96682177,661.43039422)
\curveto(400.53926724,661.43039422)(400.99349028,661.69172802)(401.32949088,662.21439563)
\curveto(401.66549148,662.7495077)(401.83349179,663.5148424)(401.83349179,664.51039974)
\curveto(401.83349179,665.50595708)(401.65926925,666.25262509)(401.31082418,666.75040376)
\curveto(400.97482358,667.2606269)(400.51437831,667.51573846)(399.92948837,667.51573846)
\closepath
}
}
{
\newrgbcolor{curcolor}{0 0 0}
\pscustom[linestyle=none,fillstyle=solid,fillcolor=curcolor]
{
\newpath
\moveto(411.14816334,669.75574248)
\curveto(412.51705468,669.75574248)(413.56238989,669.45707528)(414.28416896,668.85974087)
\curveto(415.0183925,668.27485094)(415.38550427,667.3726271)(415.38550427,666.15306935)
\lineto(415.38550427,659.35839051)
\lineto(413.44416746,659.35839051)
\lineto(412.90283315,660.73972632)
\lineto(412.82816635,660.73972632)
\curveto(412.39261002,660.19216978)(411.93216475,659.79394684)(411.44683054,659.54505751)
\curveto(410.96149634,659.29616817)(410.29571737,659.17172351)(409.44949363,659.17172351)
\curveto(408.54104756,659.17172351)(407.78815732,659.43305731)(407.19082291,659.95572491)
\curveto(406.59348851,660.47839252)(406.29482131,661.29350509)(406.29482131,662.40106263)
\curveto(406.29482131,663.48373124)(406.67437754,664.28017711)(407.43349002,664.79040025)
\curveto(408.19260249,665.30062338)(409.33127119,665.58684612)(410.84949614,665.64906845)
\lineto(412.62283265,665.70506855)
\lineto(412.62283265,666.15306935)
\curveto(412.62283265,666.68818142)(412.47972128,667.08018213)(412.19349855,667.32907146)
\curveto(411.91972028,667.5779608)(411.53394181,667.70240547)(411.03616314,667.70240547)
\curveto(410.53838447,667.70240547)(410.05305027,667.62773866)(409.58016053,667.47840506)
\curveto(409.10727079,667.34151593)(408.63438106,667.16729339)(408.16149132,666.95573746)
\lineto(407.24682301,668.84107417)
\curveto(407.78193508,669.11485244)(408.38549172,669.33263061)(409.05749293,669.49440868)
\curveto(409.72949413,669.66863121)(410.42638427,669.75574248)(411.14816334,669.75574248)
\closepath
\moveto(412.62283265,664.08106564)
\lineto(411.54016404,664.04373224)
\curveto(410.64416244,664.01884331)(410.0219391,663.85706524)(409.67349403,663.55839804)
\curveto(409.32504896,663.25973083)(409.15082643,662.86773013)(409.15082643,662.38239593)
\curveto(409.15082643,661.95928406)(409.27527109,661.65439462)(409.52416043,661.46772762)
\curveto(409.77304976,661.29350509)(410.0966059,661.20639382)(410.49482884,661.20639382)
\curveto(411.09216324,661.20639382)(411.59616414,661.38061636)(412.00683155,661.72906142)
\curveto(412.41749895,662.08995096)(412.62283265,662.59395186)(412.62283265,663.24106413)
\closepath
}
}
{
\newrgbcolor{curcolor}{0 0 0}
\pscustom[linestyle=none,fillstyle=solid,fillcolor=curcolor]
{
\newpath
\moveto(423.93485463,669.73707578)
\curveto(424.07174376,669.73707578)(424.23352183,669.73085355)(424.42018883,669.71840908)
\curveto(424.60685583,669.70596461)(424.75618944,669.68729791)(424.86818964,669.66240898)
\lineto(424.66285593,667.04907096)
\curveto(424.5633002,667.07395989)(424.4326333,667.09262659)(424.27085523,667.10507106)
\curveto(424.10907716,667.12995999)(423.9659658,667.14240446)(423.84152113,667.14240446)
\curveto(423.36863139,667.14240446)(422.91440836,667.05529319)(422.47885202,666.88107066)
\curveto(422.04329568,666.71929259)(421.68862838,666.45173656)(421.41485011,666.07840255)
\curveto(421.15351631,665.70506855)(421.02284941,665.19484541)(421.02284941,664.54773314)
\lineto(421.02284941,659.35839051)
\lineto(418.24151109,659.35839051)
\lineto(418.24151109,669.55040878)
\lineto(420.35084821,669.55040878)
\lineto(420.76151561,667.83307237)
\lineto(420.89218251,667.83307237)
\curveto(421.19084971,668.35573997)(421.60151711,668.80374077)(422.12418472,669.17707478)
\curveto(422.64685232,669.55040878)(423.25040896,669.73707578)(423.93485463,669.73707578)
\closepath
}
}
{
\newrgbcolor{curcolor}{0 0 0}
\pscustom[linestyle=none,fillstyle=solid,fillcolor=curcolor]
{
\newpath
\moveto(433.8282094,662.38239593)
\curveto(433.8282094,661.34950519)(433.46109763,660.55305932)(432.72687409,659.99305831)
\curveto(432.00509502,659.44550178)(430.92242642,659.17172351)(429.47886827,659.17172351)
\curveto(428.76953367,659.17172351)(428.1597548,659.22150137)(427.64953166,659.32105711)
\curveto(427.13930852,659.40816838)(426.62908539,659.55750198)(426.11886225,659.76905791)
\lineto(426.11886225,662.06506203)
\curveto(426.66641879,661.81617269)(427.25753096,661.61083899)(427.89219876,661.44906092)
\curveto(428.52686657,661.28728285)(429.08686757,661.20639382)(429.57220177,661.20639382)
\curveto(430.10731384,661.20639382)(430.49309231,661.28728285)(430.72953718,661.44906092)
\curveto(430.96598205,661.61083899)(431.08420448,661.82239492)(431.08420448,662.08372873)
\curveto(431.08420448,662.25795126)(431.03442662,662.4135071)(430.93487088,662.55039623)
\curveto(430.84775961,662.68728536)(430.64864815,662.8428412)(430.33753648,663.01706373)
\curveto(430.02642481,663.19128627)(429.54109061,663.41528667)(428.88153387,663.68906494)
\curveto(428.2344216,663.96284321)(427.70553176,664.23039924)(427.29486436,664.49173304)
\curveto(426.89664142,664.76551131)(426.59797422,665.08906745)(426.39886275,665.46240145)
\curveto(426.19975128,665.84817992)(426.10019555,666.32729189)(426.10019555,666.89973736)
\curveto(426.10019555,667.84551683)(426.46730732,668.55485144)(427.20153086,669.02774117)
\curveto(427.93575439,669.50063091)(428.91264503,669.73707578)(430.13220278,669.73707578)
\curveto(430.76687058,669.73707578)(431.37042722,669.67485345)(431.94287269,669.55040878)
\curveto(432.51531816,669.42596411)(433.10643033,669.22063041)(433.7162092,668.93440767)
\lineto(432.8762077,666.93707076)
\curveto(432.37842903,667.14862669)(431.90553929,667.32284923)(431.45753849,667.45973836)
\curveto(431.00953768,667.60907196)(430.55531465,667.68373877)(430.09486938,667.68373877)
\curveto(429.27353457,667.68373877)(428.86286717,667.45973836)(428.86286717,667.01173756)
\curveto(428.86286717,666.84995949)(428.91264503,666.70062589)(429.01220077,666.56373676)
\curveto(429.12420097,666.43929209)(429.32953467,666.30240296)(429.62820187,666.15306935)
\curveto(429.93931354,666.00373575)(430.39353658,665.80462429)(430.99087098,665.55573495)
\curveto(431.57576092,665.31929008)(432.07976182,665.07040075)(432.50287369,664.80906695)
\curveto(432.92598556,664.56017761)(433.2495417,664.24284371)(433.4735421,663.85706524)
\curveto(433.70998697,663.47128677)(433.8282094,662.97973033)(433.8282094,662.38239593)
\closepath
}
}
{
\newrgbcolor{curcolor}{0 0 0}
\pscustom[linestyle=none,fillstyle=solid,fillcolor=curcolor]
{
\newpath
\moveto(214.83925138,652.07124737)
\curveto(215.24991879,652.07124737)(215.60458609,651.97169164)(215.90325329,651.77258017)
\curveto(216.20192049,651.58591317)(216.35125409,651.23124587)(216.35125409,650.70857826)
\curveto(216.35125409,650.19835513)(216.20192049,649.84368782)(215.90325329,649.64457636)
\curveto(215.60458609,649.44546489)(215.24991879,649.34590915)(214.83925138,649.34590915)
\curveto(214.41613951,649.34590915)(214.05524998,649.44546489)(213.75658278,649.64457636)
\curveto(213.47036004,649.84368782)(213.32724867,650.19835513)(213.32724867,650.70857826)
\curveto(213.32724867,651.23124587)(213.47036004,651.58591317)(213.75658278,651.77258017)
\curveto(214.05524998,651.97169164)(214.41613951,652.07124737)(214.83925138,652.07124737)
\closepath
\moveto(216.22058719,648.07657355)
\lineto(216.22058719,637.88455528)
\lineto(213.43924887,637.88455528)
\lineto(213.43924887,648.07657355)
\closepath
}
}
{
\newrgbcolor{curcolor}{0 0 0}
\pscustom[linestyle=none,fillstyle=solid,fillcolor=curcolor]
{
\newpath
\moveto(224.9192752,648.26324055)
\curveto(226.01438828,648.26324055)(226.89172318,647.96457334)(227.55127992,647.36723894)
\curveto(228.21083666,646.782349)(228.54061503,645.83656953)(228.54061503,644.52990052)
\lineto(228.54061503,637.88455528)
\lineto(225.75927671,637.88455528)
\lineto(225.75927671,643.83923262)
\curveto(225.75927671,644.57345615)(225.62860981,645.12101269)(225.36727601,645.48190223)
\curveto(225.1059422,645.85523623)(224.68905257,646.04190323)(224.1166071,646.04190323)
\curveto(223.27038336,646.04190323)(222.69171565,645.74945826)(222.38060399,645.16456833)
\curveto(222.06949232,644.59212285)(221.91393648,643.76456582)(221.91393648,642.68189721)
\lineto(221.91393648,637.88455528)
\lineto(219.13259816,637.88455528)
\lineto(219.13259816,648.07657355)
\lineto(221.26060198,648.07657355)
\lineto(221.63393598,646.76990454)
\lineto(221.78326958,646.76990454)
\curveto(222.10682572,647.29257214)(222.54860429,647.67212838)(223.10860529,647.90857324)
\curveto(223.68105076,648.14501811)(224.2846074,648.26324055)(224.9192752,648.26324055)
\closepath
}
}
{
\newrgbcolor{curcolor}{0 0 0}
\pscustom[linestyle=none,fillstyle=solid,fillcolor=curcolor]
{
\newpath
\moveto(234.6259509,637.69788827)
\curveto(233.49350442,637.69788827)(232.56639165,638.13966684)(231.84461258,639.02322398)
\curveto(231.13527797,639.91922559)(230.78061067,641.23211683)(230.78061067,642.96189771)
\curveto(230.78061067,644.70412306)(231.14150021,646.02323653)(231.86327928,646.91923814)
\curveto(232.58505835,647.81523974)(233.53083782,648.26324055)(234.7006177,648.26324055)
\curveto(235.43484124,648.26324055)(236.03839787,648.12012918)(236.51128761,647.83390644)
\curveto(236.98417735,647.54768371)(237.35751135,647.19301641)(237.63128962,646.76990454)
\lineto(237.72462312,646.76990454)
\curveto(237.68728972,646.969016)(237.64373409,647.25523874)(237.59395622,647.62857274)
\curveto(237.54417835,648.01435121)(237.51928942,648.40635191)(237.51928942,648.80457485)
\lineto(237.51928942,652.07124737)
\lineto(240.30062774,652.07124737)
\lineto(240.30062774,637.88455528)
\lineto(238.17262392,637.88455528)
\lineto(237.63128962,639.20989098)
\lineto(237.51928942,639.20989098)
\curveto(237.24551115,638.78677911)(236.87839938,638.42588958)(236.41795411,638.12722238)
\curveto(235.95750884,637.84099964)(235.36017444,637.69788827)(234.6259509,637.69788827)
\closepath
\moveto(235.5966193,639.91922559)
\curveto(236.35573178,639.91922559)(236.89084385,640.14322599)(237.20195552,640.59122679)
\curveto(237.51306718,641.05167206)(237.68106749,641.73611774)(237.70595642,642.64456381)
\lineto(237.70595642,642.94323101)
\curveto(237.70595642,643.92634388)(237.55040058,644.67923412)(237.23928892,645.20190173)
\curveto(236.94062171,645.7370138)(236.38062071,646.00456983)(235.5592859,646.00456983)
\curveto(234.94950703,646.00456983)(234.47039506,645.7370138)(234.12194999,645.20190173)
\curveto(233.77350493,644.67923412)(233.59928239,643.92012165)(233.59928239,642.92456431)
\curveto(233.59928239,641.92900697)(233.77350493,641.17611673)(234.12194999,640.66589359)
\curveto(234.47039506,640.16811492)(234.9619515,639.91922559)(235.5966193,639.91922559)
\closepath
}
}
{
\newrgbcolor{curcolor}{0 0 0}
\pscustom[linestyle=none,fillstyle=solid,fillcolor=curcolor]
{
\newpath
\moveto(247.41263058,648.26324055)
\curveto(248.81885533,648.26324055)(249.9326351,647.85879538)(250.75396991,647.04990504)
\curveto(251.57530471,646.25345917)(251.98597212,645.11479046)(251.98597212,643.63389891)
\lineto(251.98597212,642.28989651)
\lineto(245.41529367,642.28989651)
\curveto(245.4401826,641.5058951)(245.67040524,640.889894)(246.10596158,640.44189319)
\curveto(246.55396238,639.99389239)(247.16996348,639.76989199)(247.95396489,639.76989199)
\curveto(248.60107716,639.76989199)(249.19218933,639.83211432)(249.7273014,639.95655899)
\curveto(250.27485794,640.09344812)(250.83485894,640.29878183)(251.40730441,640.57256009)
\lineto(251.40730441,638.42588958)
\curveto(250.89708127,638.17700024)(250.36819144,637.99655548)(249.8206349,637.88455528)
\curveto(249.27307836,637.76011061)(248.60729939,637.69788827)(247.82329799,637.69788827)
\curveto(246.80285171,637.69788827)(245.90062787,637.88455528)(245.11662647,638.25788928)
\curveto(244.33262506,638.64366775)(243.71662396,639.21611322)(243.26862316,639.97522569)
\curveto(242.82062235,640.74678263)(242.59662195,641.72367327)(242.59662195,642.90589761)
\curveto(242.59662195,644.08812195)(242.79573342,645.07745706)(243.19395636,645.87390293)
\curveto(243.60462376,646.6703488)(244.170847,647.26768321)(244.89262607,647.66590614)
\curveto(245.61440514,648.06412908)(246.45440664,648.26324055)(247.41263058,648.26324055)
\closepath
\moveto(247.43129728,646.28457033)
\curveto(246.88374075,646.28457033)(246.43573994,646.1103478)(246.08729488,645.76190273)
\curveto(245.73884981,645.41345766)(245.5335161,644.87212336)(245.47129377,644.13789982)
\lineto(249.3726341,644.13789982)
\curveto(249.36018963,644.74767869)(249.19218933,645.25790183)(248.86863319,645.66856923)
\curveto(248.55752153,646.07923663)(248.07840956,646.28457033)(247.43129728,646.28457033)
\closepath
}
}
{
\newrgbcolor{curcolor}{0 0 0}
\pscustom[linestyle=none,fillstyle=solid,fillcolor=curcolor]
{
\newpath
\moveto(255.96196376,643.09256461)
\lineto(252.67662454,648.07657355)
\lineto(255.83129686,648.07657355)
\lineto(257.80996707,644.82856772)
\lineto(259.80730399,648.07657355)
\lineto(262.96197631,648.07657355)
\lineto(259.63930369,643.09256461)
\lineto(263.11130991,637.88455528)
\lineto(259.95663759,637.88455528)
\lineto(257.80996707,641.3752282)
\lineto(255.66329656,637.88455528)
\lineto(252.50862424,637.88455528)
\closepath
}
}
{
\newrgbcolor{curcolor}{0 0 0}
\pscustom[linestyle=none,fillstyle=solid,fillcolor=curcolor]
{
\newpath
\moveto(264.26862967,639.19122428)
\curveto(264.26862967,639.76366976)(264.4241855,640.16189269)(264.73529717,640.38589309)
\curveto(265.04640884,640.62233796)(265.42596507,640.7405604)(265.87396588,640.7405604)
\curveto(266.30952221,640.7405604)(266.68285622,640.62233796)(266.99396788,640.38589309)
\curveto(267.30507955,640.16189269)(267.46063539,639.76366976)(267.46063539,639.19122428)
\curveto(267.46063539,638.64366775)(267.30507955,638.24544481)(266.99396788,637.99655548)
\curveto(266.68285622,637.76011061)(266.30952221,637.64188817)(265.87396588,637.64188817)
\curveto(265.42596507,637.64188817)(265.04640884,637.76011061)(264.73529717,637.99655548)
\curveto(264.4241855,638.24544481)(264.26862967,638.64366775)(264.26862967,639.19122428)
\closepath
}
}
{
\newrgbcolor{curcolor}{0 0 0}
\pscustom[linestyle=none,fillstyle=solid,fillcolor=curcolor]
{
\newpath
\moveto(272.76197595,652.07124737)
\lineto(272.76197595,649.17790885)
\curveto(272.76197595,648.66768572)(272.74330925,648.18235151)(272.70597584,647.72190624)
\curveto(272.68108691,647.27390544)(272.65619798,646.95657154)(272.63130904,646.76990454)
\lineto(272.78064265,646.76990454)
\curveto(273.10419878,647.29257214)(273.52108842,647.67212838)(274.03131155,647.90857324)
\curveto(274.54153469,648.14501811)(275.10775793,648.26324055)(275.72998127,648.26324055)
\curveto(276.82509434,648.26324055)(277.70865148,647.96457334)(278.38065268,647.36723894)
\curveto(279.05265389,646.782349)(279.38865449,645.83656953)(279.38865449,644.52990052)
\lineto(279.38865449,637.88455528)
\lineto(276.60731617,637.88455528)
\lineto(276.60731617,643.83923262)
\curveto(276.60731617,645.30767969)(276.05975963,646.04190323)(274.96464656,646.04190323)
\curveto(274.13086729,646.04190323)(273.55219958,645.74945826)(273.22864345,645.16456833)
\curveto(272.91753178,644.59212285)(272.76197595,643.76456582)(272.76197595,642.68189721)
\lineto(272.76197595,637.88455528)
\lineto(269.98063763,637.88455528)
\lineto(269.98063763,652.07124737)
\closepath
}
}
{
\newrgbcolor{curcolor}{0 0 0}
\pscustom[linestyle=none,fillstyle=solid,fillcolor=curcolor]
{
\newpath
\moveto(286.4259935,648.28190725)
\curveto(287.79488484,648.28190725)(288.84022005,647.98324004)(289.56199912,647.38590564)
\curveto(290.29622266,646.8010157)(290.66333443,645.89879186)(290.66333443,644.67923412)
\lineto(290.66333443,637.88455528)
\lineto(288.72199761,637.88455528)
\lineto(288.18066331,639.26589108)
\lineto(288.10599651,639.26589108)
\curveto(287.67044017,638.71833455)(287.2099949,638.32011161)(286.7246607,638.07122228)
\curveto(286.2393265,637.82233294)(285.57354753,637.69788827)(284.72732379,637.69788827)
\curveto(283.81887771,637.69788827)(283.06598747,637.95922208)(282.46865307,638.48188968)
\curveto(281.87131867,639.00455728)(281.57265146,639.81966986)(281.57265146,640.9272274)
\curveto(281.57265146,642.009896)(281.9522077,642.80634188)(282.71132017,643.31656501)
\curveto(283.47043264,643.82678815)(284.60910135,644.11301088)(286.1273263,644.17523322)
\lineto(287.90066281,644.23123332)
\lineto(287.90066281,644.67923412)
\curveto(287.90066281,645.21434619)(287.75755144,645.6063469)(287.47132871,645.85523623)
\curveto(287.19755044,646.10412557)(286.81177197,646.22857023)(286.3139933,646.22857023)
\curveto(285.81621463,646.22857023)(285.33088042,646.15390343)(284.85799069,646.00456983)
\curveto(284.38510095,645.8676807)(283.91221121,645.69345816)(283.43932148,645.48190223)
\lineto(282.52465317,647.36723894)
\curveto(283.05976524,647.64101721)(283.66332188,647.85879538)(284.33532308,648.02057344)
\curveto(285.00732429,648.19479598)(285.70421443,648.28190725)(286.4259935,648.28190725)
\closepath
\moveto(287.90066281,642.60723041)
\lineto(286.8179942,642.56989701)
\curveto(285.92199259,642.54500807)(285.29976926,642.38323001)(284.95132419,642.0845628)
\curveto(284.60287912,641.7858956)(284.42865658,641.3938949)(284.42865658,640.9085607)
\curveto(284.42865658,640.48544883)(284.55310125,640.18055939)(284.80199059,639.99389239)
\curveto(285.05087992,639.81966986)(285.37443606,639.73255859)(285.77265899,639.73255859)
\curveto(286.3699934,639.73255859)(286.8739943,639.90678112)(287.2846617,640.25522619)
\curveto(287.69532911,640.61611573)(287.90066281,641.12011663)(287.90066281,641.7672289)
\closepath
}
}
{
\newrgbcolor{curcolor}{0 0 0}
\pscustom[linestyle=none,fillstyle=solid,fillcolor=curcolor]
{
\newpath
\moveto(299.30601829,648.26324055)
\curveto(300.40113136,648.26324055)(301.27846627,647.96457334)(301.93802301,647.36723894)
\curveto(302.59757974,646.782349)(302.92735811,645.83656953)(302.92735811,644.52990052)
\lineto(302.92735811,637.88455528)
\lineto(300.14601979,637.88455528)
\lineto(300.14601979,643.83923262)
\curveto(300.14601979,644.57345615)(300.01535289,645.12101269)(299.75401909,645.48190223)
\curveto(299.49268529,645.85523623)(299.07579565,646.04190323)(298.50335018,646.04190323)
\curveto(297.65712644,646.04190323)(297.07845874,645.74945826)(296.76734707,645.16456833)
\curveto(296.4562354,644.59212285)(296.30067957,643.76456582)(296.30067957,642.68189721)
\lineto(296.30067957,637.88455528)
\lineto(293.51934125,637.88455528)
\lineto(293.51934125,648.07657355)
\lineto(295.64734506,648.07657355)
\lineto(296.02067907,646.76990454)
\lineto(296.17001267,646.76990454)
\curveto(296.4935688,647.29257214)(296.93534737,647.67212838)(297.49534838,647.90857324)
\curveto(298.06779385,648.14501811)(298.67135048,648.26324055)(299.30601829,648.26324055)
\closepath
}
}
{
\newrgbcolor{curcolor}{0 0 0}
\pscustom[linestyle=none,fillstyle=solid,fillcolor=curcolor]
{
\newpath
\moveto(309.01268778,637.69788827)
\curveto(307.88024131,637.69788827)(306.95312854,638.13966684)(306.23134946,639.02322398)
\curveto(305.52201486,639.91922559)(305.16734756,641.23211683)(305.16734756,642.96189771)
\curveto(305.16734756,644.70412306)(305.52823709,646.02323653)(306.25001617,646.91923814)
\curveto(306.97179524,647.81523974)(307.91757471,648.26324055)(309.08735458,648.26324055)
\curveto(309.82157812,648.26324055)(310.42513476,648.12012918)(310.8980245,647.83390644)
\curveto(311.37091423,647.54768371)(311.74424824,647.19301641)(312.0180265,646.76990454)
\lineto(312.11136001,646.76990454)
\curveto(312.0740266,646.969016)(312.03047097,647.25523874)(311.9806931,647.62857274)
\curveto(311.93091524,648.01435121)(311.9060263,648.40635191)(311.9060263,648.80457485)
\lineto(311.9060263,652.07124737)
\lineto(314.68736462,652.07124737)
\lineto(314.68736462,637.88455528)
\lineto(312.55936081,637.88455528)
\lineto(312.0180265,639.20989098)
\lineto(311.9060263,639.20989098)
\curveto(311.63224804,638.78677911)(311.26513627,638.42588958)(310.804691,638.12722238)
\curveto(310.34424573,637.84099964)(309.74691132,637.69788827)(309.01268778,637.69788827)
\closepath
\moveto(309.98335619,639.91922559)
\curveto(310.74246866,639.91922559)(311.27758073,640.14322599)(311.5886924,640.59122679)
\curveto(311.89980407,641.05167206)(312.06780437,641.73611774)(312.09269331,642.64456381)
\lineto(312.09269331,642.94323101)
\curveto(312.09269331,643.92634388)(311.93713747,644.67923412)(311.6260258,645.20190173)
\curveto(311.3273586,645.7370138)(310.7673576,646.00456983)(309.94602279,646.00456983)
\curveto(309.33624392,646.00456983)(308.85713195,645.7370138)(308.50868688,645.20190173)
\curveto(308.16024181,644.67923412)(307.98601928,643.92012165)(307.98601928,642.92456431)
\curveto(307.98601928,641.92900697)(308.16024181,641.17611673)(308.50868688,640.66589359)
\curveto(308.85713195,640.16811492)(309.34868839,639.91922559)(309.98335619,639.91922559)
\closepath
}
}
{
\newrgbcolor{curcolor}{0 0 0}
\pscustom[linestyle=none,fillstyle=solid,fillcolor=curcolor]
{
\newpath
\moveto(320.38070398,637.88455528)
\lineto(317.59936566,637.88455528)
\lineto(317.59936566,652.07124737)
\lineto(320.38070398,652.07124737)
\closepath
}
}
{
\newrgbcolor{curcolor}{0 0 0}
\pscustom[linestyle=none,fillstyle=solid,fillcolor=curcolor]
{
\newpath
\moveto(327.49271724,648.26324055)
\curveto(328.89894198,648.26324055)(330.01272175,647.85879538)(330.83405656,647.04990504)
\curveto(331.65539137,646.25345917)(332.06605877,645.11479046)(332.06605877,643.63389891)
\lineto(332.06605877,642.28989651)
\lineto(325.49538032,642.28989651)
\curveto(325.52026926,641.5058951)(325.75049189,640.889894)(326.18604823,640.44189319)
\curveto(326.63404903,639.99389239)(327.25005014,639.76989199)(328.03405154,639.76989199)
\curveto(328.68116381,639.76989199)(329.27227598,639.83211432)(329.80738805,639.95655899)
\curveto(330.35494459,640.09344812)(330.91494559,640.29878183)(331.48739106,640.57256009)
\lineto(331.48739106,638.42588958)
\curveto(330.97716793,638.17700024)(330.44827809,637.99655548)(329.90072155,637.88455528)
\curveto(329.35316502,637.76011061)(328.68738605,637.69788827)(327.90338464,637.69788827)
\curveto(326.88293837,637.69788827)(325.98071453,637.88455528)(325.19671312,638.25788928)
\curveto(324.41271172,638.64366775)(323.79671061,639.21611322)(323.34870981,639.97522569)
\curveto(322.90070901,640.74678263)(322.6767086,641.72367327)(322.6767086,642.90589761)
\curveto(322.6767086,644.08812195)(322.87582007,645.07745706)(323.27404301,645.87390293)
\curveto(323.68471041,646.6703488)(324.25093365,647.26768321)(324.97271272,647.66590614)
\curveto(325.69449179,648.06412908)(326.5344933,648.26324055)(327.49271724,648.26324055)
\closepath
\moveto(327.51138394,646.28457033)
\curveto(326.9638274,646.28457033)(326.5158266,646.1103478)(326.16738153,645.76190273)
\curveto(325.81893646,645.41345766)(325.61360276,644.87212336)(325.55138042,644.13789982)
\lineto(329.45272075,644.13789982)
\curveto(329.44027628,644.74767869)(329.27227598,645.25790183)(328.94871985,645.66856923)
\curveto(328.63760818,646.07923663)(328.15849621,646.28457033)(327.51138394,646.28457033)
\closepath
}
}
{
\newrgbcolor{curcolor}{0 0 0}
\pscustom[linestyle=none,fillstyle=solid,fillcolor=curcolor]
{
\newpath
\moveto(337.10606083,652.07124737)
\lineto(337.10606083,648.76724145)
\curveto(337.10606083,648.38146298)(337.09361636,648.00190674)(337.06872743,647.62857274)
\curveto(337.04383849,647.25523874)(337.01894956,646.96279377)(336.99406063,646.75123784)
\lineto(337.10606083,646.75123784)
\curveto(337.3798391,647.17434971)(337.74695087,647.52901701)(338.20739614,647.81523974)
\curveto(338.66784141,648.11390695)(339.26517581,648.26324055)(339.99939935,648.26324055)
\curveto(341.14429029,648.26324055)(342.07140306,647.81523974)(342.78073767,646.91923814)
\curveto(343.49007227,646.035681)(343.84473957,644.72901199)(343.84473957,642.99923111)
\curveto(343.84473957,641.25700577)(343.48385004,639.93789229)(342.76207097,639.04189068)
\curveto(342.0402919,638.14588908)(341.09451242,637.69788827)(339.92473255,637.69788827)
\curveto(339.17806454,637.69788827)(338.58695237,637.82855518)(338.15139604,638.08988898)
\curveto(337.72828417,638.36366725)(337.3798391,638.66855668)(337.10606083,639.00455728)
\lineto(336.91939383,639.00455728)
\lineto(336.45272632,637.88455528)
\lineto(334.32472251,637.88455528)
\lineto(334.32472251,652.07124737)
\closepath
\moveto(339.10339774,646.04190323)
\curveto(338.38161867,646.04190323)(337.87139553,645.8116806)(337.57272833,645.35123533)
\curveto(337.27406113,644.90323452)(337.11850529,644.22501109)(337.10606083,643.31656501)
\lineto(337.10606083,643.01789781)
\curveto(337.10606083,642.03478494)(337.2491722,641.27567247)(337.53539493,640.7405604)
\curveto(337.83406213,640.21789279)(338.3691742,639.95655899)(339.14073114,639.95655899)
\curveto(339.71317661,639.95655899)(340.16739965,640.21789279)(340.50340025,640.7405604)
\curveto(340.83940085,641.27567247)(341.00740115,642.04100717)(341.00740115,643.03656451)
\curveto(341.00740115,644.03212185)(340.83317862,644.77878986)(340.48473355,645.27656853)
\curveto(340.14873295,645.78679166)(339.68828768,646.04190323)(339.10339774,646.04190323)
\closepath
}
}
{
\newrgbcolor{curcolor}{0 0 0}
\pscustom[linestyle=none,fillstyle=solid,fillcolor=curcolor]
{
\newpath
\moveto(350.32208034,648.28190725)
\curveto(351.69097168,648.28190725)(352.73630689,647.98324004)(353.45808596,647.38590564)
\curveto(354.1923095,646.8010157)(354.55942127,645.89879186)(354.55942127,644.67923412)
\lineto(354.55942127,637.88455528)
\lineto(352.61808446,637.88455528)
\lineto(352.07675015,639.26589108)
\lineto(352.00208335,639.26589108)
\curveto(351.56652701,638.71833455)(351.10608174,638.32011161)(350.62074754,638.07122228)
\curveto(350.13541334,637.82233294)(349.46963437,637.69788827)(348.62341063,637.69788827)
\curveto(347.71496455,637.69788827)(346.96207432,637.95922208)(346.36473991,638.48188968)
\curveto(345.76740551,639.00455728)(345.46873831,639.81966986)(345.46873831,640.9272274)
\curveto(345.46873831,642.009896)(345.84829454,642.80634188)(346.60740701,643.31656501)
\curveto(347.36651949,643.82678815)(348.50518819,644.11301088)(350.02341314,644.17523322)
\lineto(351.79674965,644.23123332)
\lineto(351.79674965,644.67923412)
\curveto(351.79674965,645.21434619)(351.65363828,645.6063469)(351.36741555,645.85523623)
\curveto(351.09363728,646.10412557)(350.70785881,646.22857023)(350.21008014,646.22857023)
\curveto(349.71230147,646.22857023)(349.22696727,646.15390343)(348.75407753,646.00456983)
\curveto(348.28118779,645.8676807)(347.80829806,645.69345816)(347.33540832,645.48190223)
\lineto(346.42074001,647.36723894)
\curveto(346.95585208,647.64101721)(347.55940872,647.85879538)(348.23140993,648.02057344)
\curveto(348.90341113,648.19479598)(349.60030127,648.28190725)(350.32208034,648.28190725)
\closepath
\moveto(351.79674965,642.60723041)
\lineto(350.71408104,642.56989701)
\curveto(349.81807944,642.54500807)(349.1958561,642.38323001)(348.84741103,642.0845628)
\curveto(348.49896596,641.7858956)(348.32474343,641.3938949)(348.32474343,640.9085607)
\curveto(348.32474343,640.48544883)(348.44918809,640.18055939)(348.69807743,639.99389239)
\curveto(348.94696676,639.81966986)(349.2705229,639.73255859)(349.66874584,639.73255859)
\curveto(350.26608024,639.73255859)(350.77008114,639.90678112)(351.18074855,640.25522619)
\curveto(351.59141595,640.61611573)(351.79674965,641.12011663)(351.79674965,641.7672289)
\closepath
}
}
{
\newrgbcolor{curcolor}{0 0 0}
\pscustom[linestyle=none,fillstyle=solid,fillcolor=curcolor]
{
\newpath
\moveto(363.10877163,648.26324055)
\curveto(363.24566076,648.26324055)(363.40743883,648.25701831)(363.59410583,648.24457385)
\curveto(363.78077283,648.23212938)(363.93010643,648.21346268)(364.04210663,648.18857375)
\lineto(363.83677293,645.57523573)
\curveto(363.7372172,645.60012466)(363.6065503,645.61879136)(363.44477223,645.63123583)
\curveto(363.28299416,645.65612476)(363.1398828,645.66856923)(363.01543813,645.66856923)
\curveto(362.54254839,645.66856923)(362.08832535,645.58145796)(361.65276902,645.40723543)
\curveto(361.21721268,645.24545736)(360.86254538,644.97790132)(360.58876711,644.60456732)
\curveto(360.32743331,644.23123332)(360.19676641,643.72101018)(360.19676641,643.07389791)
\lineto(360.19676641,637.88455528)
\lineto(357.41542809,637.88455528)
\lineto(357.41542809,648.07657355)
\lineto(359.5247652,648.07657355)
\lineto(359.93543261,646.35923713)
\lineto(360.06609951,646.35923713)
\curveto(360.36476671,646.88190474)(360.77543411,647.32990554)(361.29810172,647.70323954)
\curveto(361.82076932,648.07657355)(362.42432596,648.26324055)(363.10877163,648.26324055)
\closepath
}
}
{
\newrgbcolor{curcolor}{0 0 0}
\pscustom[linestyle=none,fillstyle=solid,fillcolor=curcolor]
{
\newpath
\moveto(373.0021264,640.9085607)
\curveto(373.0021264,639.87566996)(372.63501463,639.07922408)(371.90079109,638.51922308)
\curveto(371.17901202,637.97166654)(370.09634341,637.69788827)(368.65278527,637.69788827)
\curveto(367.94345067,637.69788827)(367.33367179,637.74766614)(366.82344866,637.84722188)
\curveto(366.31322552,637.93433314)(365.80300238,638.08366674)(365.29277925,638.29522268)
\lineto(365.29277925,640.59122679)
\curveto(365.84033578,640.34233746)(366.43144796,640.13700376)(367.06611576,639.97522569)
\curveto(367.70078356,639.81344762)(368.26078457,639.73255859)(368.74611877,639.73255859)
\curveto(369.28123084,639.73255859)(369.66700931,639.81344762)(369.90345418,639.97522569)
\curveto(370.13989905,640.13700376)(370.25812148,640.34855969)(370.25812148,640.60989349)
\curveto(370.25812148,640.78411603)(370.20834361,640.93967186)(370.10878788,641.076561)
\curveto(370.02167661,641.21345013)(369.82256515,641.36900597)(369.51145348,641.5432285)
\curveto(369.20034181,641.71745104)(368.7150076,641.94145144)(368.05545087,642.21522971)
\curveto(367.4083386,642.48900797)(366.87944876,642.75656401)(366.46878136,643.01789781)
\curveto(366.07055842,643.29167608)(365.77189122,643.61523221)(365.57277975,643.98856622)
\curveto(365.37366828,644.37434469)(365.27411255,644.85345666)(365.27411255,645.42590213)
\curveto(365.27411255,646.3716816)(365.64122432,647.08101621)(366.37544786,647.55390594)
\curveto(367.10967139,648.02679568)(368.08656203,648.26324055)(369.30611978,648.26324055)
\curveto(369.94078758,648.26324055)(370.54434422,648.20101821)(371.11678969,648.07657355)
\curveto(371.68923516,647.95212888)(372.28034733,647.74679518)(372.8901262,647.46057244)
\lineto(372.05012469,645.46323553)
\curveto(371.55234602,645.67479146)(371.07945629,645.849014)(370.63145548,645.98590313)
\curveto(370.18345468,646.13523673)(369.72923164,646.20990353)(369.26878637,646.20990353)
\curveto(368.44745157,646.20990353)(368.03678417,645.98590313)(368.03678417,645.53790233)
\curveto(368.03678417,645.37612426)(368.08656203,645.22679066)(368.18611777,645.08990152)
\curveto(368.29811797,644.96545686)(368.50345167,644.82856772)(368.80211887,644.67923412)
\curveto(369.11323054,644.52990052)(369.56745358,644.33078905)(370.16478798,644.08189972)
\curveto(370.74967792,643.84545485)(371.25367882,643.59656551)(371.67679069,643.33523171)
\curveto(372.09990256,643.08634238)(372.4234587,642.76900848)(372.6474591,642.38323001)
\curveto(372.88390397,641.99745154)(373.0021264,641.5058951)(373.0021264,640.9085607)
\closepath
}
}
{
\newrgbcolor{curcolor}{0 0 0}
\pscustom[linestyle=none,fillstyle=solid,fillcolor=curcolor]
{
\newpath
\moveto(218.86200933,627.15984333)
\curveto(218.99889847,627.15984333)(219.16067654,627.15362109)(219.34734354,627.14117663)
\curveto(219.53401054,627.12873216)(219.68334414,627.11006546)(219.79534434,627.08517653)
\lineto(219.59001064,624.47183851)
\curveto(219.4904549,624.49672744)(219.359788,624.51539414)(219.19800994,624.52783861)
\curveto(219.03623187,624.55272754)(218.8931205,624.56517201)(218.76867583,624.56517201)
\curveto(218.2957861,624.56517201)(217.84156306,624.47806074)(217.40600672,624.30383821)
\curveto(216.97045039,624.14206014)(216.61578308,623.8745041)(216.34200482,623.5011701)
\curveto(216.08067101,623.1278361)(215.95000411,622.61761296)(215.95000411,621.97050069)
\lineto(215.95000411,616.78115806)
\lineto(213.16866579,616.78115806)
\lineto(213.16866579,626.97317633)
\lineto(215.27800291,626.97317633)
\lineto(215.68867031,625.25583991)
\lineto(215.81933721,625.25583991)
\curveto(216.11800441,625.77850752)(216.52867182,626.22650832)(217.05133942,626.59984232)
\curveto(217.57400702,626.97317633)(218.17756366,627.15984333)(218.86200933,627.15984333)
\closepath
}
}
{
\newrgbcolor{curcolor}{0 0 0}
\pscustom[linestyle=none,fillstyle=solid,fillcolor=curcolor]
{
\newpath
\moveto(230.52869688,621.89583389)
\curveto(230.52869688,620.20338641)(230.08069608,618.8967174)(229.18469447,617.97582686)
\curveto(228.30113733,617.05493632)(227.09402406,616.59449105)(225.56335465,616.59449105)
\curveto(224.61757517,616.59449105)(223.77135143,616.79982476)(223.02468343,617.21049216)
\curveto(222.29045989,617.62115956)(221.71179219,618.21849396)(221.28868032,619.00249537)
\curveto(220.86556845,619.79894124)(220.65401251,620.76338742)(220.65401251,621.89583389)
\curveto(220.65401251,623.58828137)(221.09579108,624.88872814)(221.97934822,625.79717422)
\curveto(222.86290536,626.70562029)(224.07624087,627.15984333)(225.61935475,627.15984333)
\curveto(226.57757869,627.15984333)(227.42380242,626.95450962)(228.15802596,626.54384222)
\curveto(228.8922495,626.13317482)(229.47091721,625.53584042)(229.89402907,624.75183901)
\curveto(230.31714094,623.9678376)(230.52869688,623.0158359)(230.52869688,621.89583389)
\closepath
\moveto(223.49135093,621.89583389)
\curveto(223.49135093,620.88783208)(223.653129,620.12249738)(223.97668513,619.59982977)
\curveto(224.31268574,619.08960664)(224.85402004,618.83449507)(225.60068805,618.83449507)
\curveto(226.33491158,618.83449507)(226.86380142,619.08960664)(227.18735756,619.59982977)
\curveto(227.52335816,620.12249738)(227.69135846,620.88783208)(227.69135846,621.89583389)
\curveto(227.69135846,622.9038357)(227.52335816,623.65672594)(227.18735756,624.15450461)
\curveto(226.86380142,624.66472774)(226.32868935,624.91983931)(225.58202135,624.91983931)
\curveto(224.84779781,624.91983931)(224.31268574,624.66472774)(223.97668513,624.15450461)
\curveto(223.653129,623.65672594)(223.49135093,622.9038357)(223.49135093,621.89583389)
\closepath
}
}
{
\newrgbcolor{curcolor}{0 0 0}
\pscustom[linestyle=none,fillstyle=solid,fillcolor=curcolor]
{
\newpath
\moveto(242.08337374,621.89583389)
\curveto(242.08337374,620.20338641)(241.63537294,618.8967174)(240.73937133,617.97582686)
\curveto(239.85581419,617.05493632)(238.64870092,616.59449105)(237.11803151,616.59449105)
\curveto(236.17225204,616.59449105)(235.3260283,616.79982476)(234.57936029,617.21049216)
\curveto(233.84513675,617.62115956)(233.26646905,618.21849396)(232.84335718,619.00249537)
\curveto(232.42024531,619.79894124)(232.20868938,620.76338742)(232.20868938,621.89583389)
\curveto(232.20868938,623.58828137)(232.65046795,624.88872814)(233.53402508,625.79717422)
\curveto(234.41758222,626.70562029)(235.63091773,627.15984333)(237.17403161,627.15984333)
\curveto(238.13225555,627.15984333)(238.97847929,626.95450962)(239.71270283,626.54384222)
\curveto(240.44692637,626.13317482)(241.02559407,625.53584042)(241.44870594,624.75183901)
\curveto(241.87181781,623.9678376)(242.08337374,623.0158359)(242.08337374,621.89583389)
\closepath
\moveto(235.0460278,621.89583389)
\curveto(235.0460278,620.88783208)(235.20780586,620.12249738)(235.531362,619.59982977)
\curveto(235.8673626,619.08960664)(236.4086969,618.83449507)(237.15536491,618.83449507)
\curveto(237.88958845,618.83449507)(238.41847828,619.08960664)(238.74203442,619.59982977)
\curveto(239.07803502,620.12249738)(239.24603532,620.88783208)(239.24603532,621.89583389)
\curveto(239.24603532,622.9038357)(239.07803502,623.65672594)(238.74203442,624.15450461)
\curveto(238.41847828,624.66472774)(237.88336621,624.91983931)(237.13669821,624.91983931)
\curveto(236.40247467,624.91983931)(235.8673626,624.66472774)(235.531362,624.15450461)
\curveto(235.20780586,623.65672594)(235.0460278,622.9038357)(235.0460278,621.89583389)
\closepath
}
}
{
\newrgbcolor{curcolor}{0 0 0}
\pscustom[linestyle=none,fillstyle=solid,fillcolor=curcolor]
{
\newpath
\moveto(256.34472499,627.15984333)
\curveto(257.50206039,627.15984333)(258.37317307,626.86117612)(258.958063,626.26384172)
\curveto(259.55539741,625.67895178)(259.85406461,624.73317231)(259.85406461,623.4265033)
\lineto(259.85406461,616.78115806)
\lineto(257.07272629,616.78115806)
\lineto(257.07272629,622.7358354)
\curveto(257.07272629,624.20428247)(256.56250316,624.93850601)(255.54205688,624.93850601)
\curveto(254.80783334,624.93850601)(254.28516574,624.67717221)(253.97405407,624.15450461)
\curveto(253.6629424,623.631837)(253.50738657,622.87894676)(253.50738657,621.89583389)
\lineto(253.50738657,616.78115806)
\lineto(250.72604825,616.78115806)
\lineto(250.72604825,622.7358354)
\curveto(250.72604825,624.20428247)(250.21582511,624.93850601)(249.19537884,624.93850601)
\curveto(248.4238219,624.93850601)(247.88870983,624.64606104)(247.59004263,624.06117111)
\curveto(247.30381989,623.48872563)(247.16070852,622.6611686)(247.16070852,621.57849999)
\lineto(247.16070852,616.78115806)
\lineto(244.3793702,616.78115806)
\lineto(244.3793702,626.97317633)
\lineto(246.50737402,626.97317633)
\lineto(246.88070802,625.66650732)
\lineto(247.03004162,625.66650732)
\curveto(247.34115329,626.18917492)(247.76426516,626.56873116)(248.29937723,626.80517602)
\curveto(248.84693377,627.04162089)(249.41315701,627.15984333)(249.99804694,627.15984333)
\curveto(250.74471495,627.15984333)(251.37316052,627.03539866)(251.88338366,626.78650932)
\curveto(252.40605126,626.55006446)(252.81049643,626.17673045)(253.09671916,625.66650732)
\lineto(253.33938627,625.66650732)
\curveto(253.65049794,626.18917492)(254.07983204,626.56873116)(254.62738858,626.80517602)
\curveto(255.18738958,627.04162089)(255.75983505,627.15984333)(256.34472499,627.15984333)
\closepath
}
}
{
\newrgbcolor{curcolor}{0 0 0}
\pscustom[linestyle=none,fillstyle=solid,fillcolor=curcolor]
{
\newpath
\moveto(262.31807301,618.08782706)
\curveto(262.31807301,618.66027253)(262.47362884,619.05849547)(262.78474051,619.28249587)
\curveto(263.09585218,619.51894074)(263.47540841,619.63716317)(263.92340922,619.63716317)
\curveto(264.35896555,619.63716317)(264.73229956,619.51894074)(265.04341123,619.28249587)
\curveto(265.35452289,619.05849547)(265.51007873,618.66027253)(265.51007873,618.08782706)
\curveto(265.51007873,617.54027053)(265.35452289,617.14204759)(265.04341123,616.89315826)
\curveto(264.73229956,616.65671339)(264.35896555,616.53849095)(263.92340922,616.53849095)
\curveto(263.47540841,616.53849095)(263.09585218,616.65671339)(262.78474051,616.89315826)
\curveto(262.47362884,617.14204759)(262.31807301,617.54027053)(262.31807301,618.08782706)
\closepath
}
}
{
\newrgbcolor{curcolor}{0 0 0}
\pscustom[linestyle=none,fillstyle=solid,fillcolor=curcolor]
{
\newpath
\moveto(270.81141929,630.96785015)
\lineto(270.81141929,628.07451163)
\curveto(270.81141929,627.5642885)(270.79275259,627.07895429)(270.75541919,626.61850902)
\curveto(270.73053025,626.17050822)(270.70564132,625.85317432)(270.68075238,625.66650732)
\lineto(270.83008599,625.66650732)
\curveto(271.15364212,626.18917492)(271.57053176,626.56873116)(272.08075489,626.80517602)
\curveto(272.59097803,627.04162089)(273.15720127,627.15984333)(273.77942461,627.15984333)
\curveto(274.87453768,627.15984333)(275.75809482,626.86117612)(276.43009602,626.26384172)
\curveto(277.10209723,625.67895178)(277.43809783,624.73317231)(277.43809783,623.4265033)
\lineto(277.43809783,616.78115806)
\lineto(274.65675951,616.78115806)
\lineto(274.65675951,622.7358354)
\curveto(274.65675951,624.20428247)(274.10920297,624.93850601)(273.0140899,624.93850601)
\curveto(272.18031063,624.93850601)(271.60164292,624.64606104)(271.27808679,624.06117111)
\curveto(270.96697512,623.48872563)(270.81141929,622.6611686)(270.81141929,621.57849999)
\lineto(270.81141929,616.78115806)
\lineto(268.03008097,616.78115806)
\lineto(268.03008097,630.96785015)
\closepath
}
}
{
\newrgbcolor{curcolor}{0 0 0}
\pscustom[linestyle=none,fillstyle=solid,fillcolor=curcolor]
{
\newpath
\moveto(284.47543302,627.17851003)
\curveto(285.84432437,627.17851003)(286.88965957,626.87984282)(287.61143865,626.28250842)
\curveto(288.34566218,625.69761848)(288.71277395,624.79539464)(288.71277395,623.5758369)
\lineto(288.71277395,616.78115806)
\lineto(286.77143714,616.78115806)
\lineto(286.23010284,618.16249386)
\lineto(286.15543604,618.16249386)
\curveto(285.7198797,617.61493733)(285.25943443,617.21671439)(284.77410023,616.96782506)
\curveto(284.28876602,616.71893572)(283.62298705,616.59449105)(282.77676331,616.59449105)
\curveto(281.86831724,616.59449105)(281.115427,616.85582486)(280.5180926,617.37849246)
\curveto(279.92075819,617.90116006)(279.62209099,618.71627264)(279.62209099,619.82383018)
\curveto(279.62209099,620.90649878)(280.00164723,621.70294466)(280.7607597,622.21316779)
\curveto(281.51987217,622.72339093)(282.65854088,623.00961366)(284.17676582,623.071836)
\lineto(285.95010233,623.1278361)
\lineto(285.95010233,623.5758369)
\curveto(285.95010233,624.11094897)(285.80699097,624.50294967)(285.52076823,624.75183901)
\curveto(285.24698996,625.00072834)(284.86121149,625.12517301)(284.36343282,625.12517301)
\curveto(283.86565415,625.12517301)(283.38031995,625.05050621)(282.90743021,624.90117261)
\curveto(282.43454048,624.76428348)(281.96165074,624.59006094)(281.488761,624.37850501)
\lineto(280.5740927,626.26384172)
\curveto(281.10920477,626.53761999)(281.7127614,626.75539816)(282.38476261,626.91717622)
\curveto(283.05676381,627.09139876)(283.75365395,627.17851003)(284.47543302,627.17851003)
\closepath
\moveto(285.95010233,621.50383319)
\lineto(284.86743373,621.46649979)
\curveto(283.97143212,621.44161085)(283.34920878,621.27983279)(283.00076371,620.98116558)
\curveto(282.65231864,620.68249838)(282.47809611,620.29049768)(282.47809611,619.80516348)
\curveto(282.47809611,619.38205161)(282.60254078,619.07716217)(282.85143011,618.89049517)
\curveto(283.10031945,618.71627264)(283.42387558,618.62916137)(283.82209852,618.62916137)
\curveto(284.41943292,618.62916137)(284.92343383,618.8033839)(285.33410123,619.15182897)
\curveto(285.74476863,619.51271851)(285.95010233,620.01671941)(285.95010233,620.66383168)
\closepath
}
}
{
\newrgbcolor{curcolor}{0 0 0}
\pscustom[linestyle=none,fillstyle=solid,fillcolor=curcolor]
{
\newpath
\moveto(297.35545781,627.15984333)
\curveto(298.45057089,627.15984333)(299.32790579,626.86117612)(299.98746253,626.26384172)
\curveto(300.64701927,625.67895178)(300.97679764,624.73317231)(300.97679764,623.4265033)
\lineto(300.97679764,616.78115806)
\lineto(298.19545932,616.78115806)
\lineto(298.19545932,622.7358354)
\curveto(298.19545932,623.47005893)(298.06479242,624.01761547)(297.80345862,624.37850501)
\curveto(297.54212481,624.75183901)(297.12523518,624.93850601)(296.55278971,624.93850601)
\curveto(295.70656597,624.93850601)(295.12789826,624.64606104)(294.8167866,624.06117111)
\curveto(294.50567493,623.48872563)(294.35011909,622.6611686)(294.35011909,621.57849999)
\lineto(294.35011909,616.78115806)
\lineto(291.56878077,616.78115806)
\lineto(291.56878077,626.97317633)
\lineto(293.69678459,626.97317633)
\lineto(294.07011859,625.66650732)
\lineto(294.21945219,625.66650732)
\curveto(294.54300833,626.18917492)(294.9847869,626.56873116)(295.5447879,626.80517602)
\curveto(296.11723337,627.04162089)(296.72079001,627.15984333)(297.35545781,627.15984333)
\closepath
}
}
{
\newrgbcolor{curcolor}{0 0 0}
\pscustom[linestyle=none,fillstyle=solid,fillcolor=curcolor]
{
\newpath
\moveto(307.06213494,616.59449105)
\curveto(305.92968846,616.59449105)(305.00257569,617.03626962)(304.28079662,617.91982676)
\curveto(303.57146202,618.81582837)(303.21679471,620.12871961)(303.21679471,621.85850049)
\curveto(303.21679471,623.60072584)(303.57768425,624.91983931)(304.29946332,625.81584092)
\curveto(305.02124239,626.71184252)(305.96702186,627.15984333)(307.13680174,627.15984333)
\curveto(307.87102528,627.15984333)(308.47458192,627.01673196)(308.94747165,626.73050922)
\curveto(309.42036139,626.44428649)(309.79369539,626.08961919)(310.06747366,625.66650732)
\lineto(310.16080716,625.66650732)
\curveto(310.12347376,625.86561878)(310.07991813,626.15184152)(310.03014026,626.52517552)
\curveto(309.98036239,626.91095399)(309.95547346,627.30295469)(309.95547346,627.70117763)
\lineto(309.95547346,630.96785015)
\lineto(312.73681178,630.96785015)
\lineto(312.73681178,616.78115806)
\lineto(310.60880796,616.78115806)
\lineto(310.06747366,618.10649376)
\lineto(309.95547346,618.10649376)
\curveto(309.68169519,617.68338189)(309.31458342,617.32249236)(308.85413815,617.02382516)
\curveto(308.39369288,616.73760242)(307.79635848,616.59449105)(307.06213494,616.59449105)
\closepath
\moveto(308.03280335,618.81582837)
\curveto(308.79191582,618.81582837)(309.32702789,619.03982877)(309.63813956,619.48782957)
\curveto(309.94925123,619.94827484)(310.11725153,620.63272051)(310.14214046,621.54116659)
\lineto(310.14214046,621.83983379)
\curveto(310.14214046,622.82294666)(309.98658463,623.5758369)(309.67547296,624.09850451)
\curveto(309.37680575,624.63361658)(308.81680475,624.90117261)(307.99546995,624.90117261)
\curveto(307.38569107,624.90117261)(306.9065791,624.63361658)(306.55813404,624.09850451)
\curveto(306.20968897,623.5758369)(306.03546643,622.81672443)(306.03546643,621.82116709)
\curveto(306.03546643,620.82560975)(306.20968897,620.07271951)(306.55813404,619.56249637)
\curveto(306.9065791,619.0647177)(307.39813554,618.81582837)(308.03280335,618.81582837)
\closepath
}
}
{
\newrgbcolor{curcolor}{0 0 0}
\pscustom[linestyle=none,fillstyle=solid,fillcolor=curcolor]
{
\newpath
\moveto(318.43014351,616.78115806)
\lineto(315.64880519,616.78115806)
\lineto(315.64880519,630.96785015)
\lineto(318.43014351,630.96785015)
\closepath
}
}
{
\newrgbcolor{curcolor}{0 0 0}
\pscustom[linestyle=none,fillstyle=solid,fillcolor=curcolor]
{
\newpath
\moveto(325.54216439,627.15984333)
\curveto(326.94838914,627.15984333)(328.06216891,626.75539816)(328.88350372,625.94650782)
\curveto(329.70483852,625.15006195)(330.11550592,624.01139324)(330.11550592,622.53050169)
\lineto(330.11550592,621.18649929)
\lineto(323.54482748,621.18649929)
\curveto(323.56971641,620.40249788)(323.79993905,619.78649678)(324.23549538,619.33849597)
\curveto(324.68349619,618.89049517)(325.29949729,618.66649477)(326.0834987,618.66649477)
\curveto(326.73061097,618.66649477)(327.32172314,618.7287171)(327.85683521,618.85316177)
\curveto(328.40439175,618.9900509)(328.96439275,619.19538461)(329.53683822,619.46916287)
\lineto(329.53683822,617.32249236)
\curveto(329.02661508,617.07360302)(328.49772525,616.89315826)(327.95016871,616.78115806)
\curveto(327.40261217,616.65671339)(326.7368332,616.59449105)(325.9528318,616.59449105)
\curveto(324.93238552,616.59449105)(324.03016168,616.78115806)(323.24616028,617.15449206)
\curveto(322.46215887,617.54027053)(321.84615777,618.112716)(321.39815696,618.87182847)
\curveto(320.95015616,619.64338541)(320.72615576,620.62027605)(320.72615576,621.80250039)
\curveto(320.72615576,622.98472473)(320.92526723,623.97405984)(321.32349016,624.77050571)
\curveto(321.73415757,625.56695158)(322.3003808,626.16428599)(323.02215987,626.56250892)
\curveto(323.74393895,626.96073186)(324.58394045,627.15984333)(325.54216439,627.15984333)
\closepath
\moveto(325.56083109,625.18117311)
\curveto(325.01327456,625.18117311)(324.56527375,625.00695058)(324.21682868,624.65850551)
\curveto(323.86838361,624.31006044)(323.66304991,623.76872614)(323.60082758,623.0345026)
\lineto(327.50216791,623.0345026)
\curveto(327.48972344,623.64428147)(327.32172314,624.15450461)(326.998167,624.56517201)
\curveto(326.68705533,624.97583941)(326.20794336,625.18117311)(325.56083109,625.18117311)
\closepath
}
}
{
\newrgbcolor{curcolor}{0 0 0}
\pscustom[linestyle=none,fillstyle=solid,fillcolor=curcolor]
{
\newpath
\moveto(335.15550798,630.96785015)
\lineto(335.15550798,627.66384423)
\curveto(335.15550798,627.27806576)(335.14306352,626.89850952)(335.11817458,626.52517552)
\curveto(335.09328565,626.15184152)(335.06839672,625.85939655)(335.04350778,625.64784062)
\lineto(335.15550798,625.64784062)
\curveto(335.42928625,626.07095249)(335.79639802,626.42561979)(336.25684329,626.71184252)
\curveto(336.71728856,627.01050973)(337.31462296,627.15984333)(338.0488465,627.15984333)
\curveto(339.19373744,627.15984333)(340.12085022,626.71184252)(340.83018482,625.81584092)
\curveto(341.53951943,624.93228378)(341.89418673,623.62561477)(341.89418673,621.89583389)
\curveto(341.89418673,620.15360854)(341.53329719,618.83449507)(340.81151812,617.93849346)
\curveto(340.08973905,617.04249186)(339.14395958,616.59449105)(337.9741797,616.59449105)
\curveto(337.2275117,616.59449105)(336.63639953,616.72515795)(336.20084319,616.98649176)
\curveto(335.77773132,617.26027003)(335.42928625,617.56515946)(335.15550798,617.90116006)
\lineto(334.96884098,617.90116006)
\lineto(334.50217348,616.78115806)
\lineto(332.37416966,616.78115806)
\lineto(332.37416966,630.96785015)
\closepath
\moveto(337.1528449,624.93850601)
\curveto(336.43106583,624.93850601)(335.92084269,624.70828338)(335.62217549,624.24783811)
\curveto(335.32350828,623.7998373)(335.16795245,623.12161387)(335.15550798,622.21316779)
\lineto(335.15550798,621.91450059)
\curveto(335.15550798,620.93138772)(335.29861935,620.17227525)(335.58484209,619.63716317)
\curveto(335.88350929,619.11449557)(336.41862136,618.85316177)(337.1901783,618.85316177)
\curveto(337.76262377,618.85316177)(338.2168468,619.11449557)(338.55284741,619.63716317)
\curveto(338.88884801,620.17227525)(339.05684831,620.93760995)(339.05684831,621.93316729)
\curveto(339.05684831,622.92872463)(338.88262578,623.67539264)(338.53418071,624.17317131)
\curveto(338.1981801,624.68339444)(337.73773483,624.93850601)(337.1528449,624.93850601)
\closepath
}
}
{
\newrgbcolor{curcolor}{0 0 0}
\pscustom[linestyle=none,fillstyle=solid,fillcolor=curcolor]
{
\newpath
\moveto(348.37151987,627.17851003)
\curveto(349.74041121,627.17851003)(350.78574642,626.87984282)(351.50752549,626.28250842)
\curveto(352.24174903,625.69761848)(352.60886079,624.79539464)(352.60886079,623.5758369)
\lineto(352.60886079,616.78115806)
\lineto(350.66752398,616.78115806)
\lineto(350.12618968,618.16249386)
\lineto(350.05152288,618.16249386)
\curveto(349.61596654,617.61493733)(349.15552127,617.21671439)(348.67018707,616.96782506)
\curveto(348.18485286,616.71893572)(347.51907389,616.59449105)(346.67285015,616.59449105)
\curveto(345.76440408,616.59449105)(345.01151384,616.85582486)(344.41417944,617.37849246)
\curveto(343.81684503,617.90116006)(343.51817783,618.71627264)(343.51817783,619.82383018)
\curveto(343.51817783,620.90649878)(343.89773407,621.70294466)(344.65684654,622.21316779)
\curveto(345.41595901,622.72339093)(346.55462772,623.00961366)(348.07285266,623.071836)
\lineto(349.84618918,623.1278361)
\lineto(349.84618918,623.5758369)
\curveto(349.84618918,624.11094897)(349.70307781,624.50294967)(349.41685507,624.75183901)
\curveto(349.1430768,625.00072834)(348.75729833,625.12517301)(348.25951966,625.12517301)
\curveto(347.76174099,625.12517301)(347.27640679,625.05050621)(346.80351705,624.90117261)
\curveto(346.33062732,624.76428348)(345.85773758,624.59006094)(345.38484784,624.37850501)
\lineto(344.47017954,626.26384172)
\curveto(345.00529161,626.53761999)(345.60884825,626.75539816)(346.28084945,626.91717622)
\curveto(346.95285066,627.09139876)(347.64974079,627.17851003)(348.37151987,627.17851003)
\closepath
\moveto(349.84618918,621.50383319)
\lineto(348.76352057,621.46649979)
\curveto(347.86751896,621.44161085)(347.24529562,621.27983279)(346.89685056,620.98116558)
\curveto(346.54840549,620.68249838)(346.37418295,620.29049768)(346.37418295,619.80516348)
\curveto(346.37418295,619.38205161)(346.49862762,619.07716217)(346.74751695,618.89049517)
\curveto(346.99640629,618.71627264)(347.31996242,618.62916137)(347.71818536,618.62916137)
\curveto(348.31551976,618.62916137)(348.81952067,618.8033839)(349.23018807,619.15182897)
\curveto(349.64085547,619.51271851)(349.84618918,620.01671941)(349.84618918,620.66383168)
\closepath
}
}
{
\newrgbcolor{curcolor}{0 0 0}
\pscustom[linestyle=none,fillstyle=solid,fillcolor=curcolor]
{
\newpath
\moveto(361.15821115,627.15984333)
\curveto(361.29510029,627.15984333)(361.45687836,627.15362109)(361.64354536,627.14117663)
\curveto(361.83021236,627.12873216)(361.97954596,627.11006546)(362.09154616,627.08517653)
\lineto(361.88621246,624.47183851)
\curveto(361.78665673,624.49672744)(361.65598982,624.51539414)(361.49421176,624.52783861)
\curveto(361.33243369,624.55272754)(361.18932232,624.56517201)(361.06487765,624.56517201)
\curveto(360.59198792,624.56517201)(360.13776488,624.47806074)(359.70220854,624.30383821)
\curveto(359.26665221,624.14206014)(358.91198491,623.8745041)(358.63820664,623.5011701)
\curveto(358.37687284,623.1278361)(358.24620593,622.61761296)(358.24620593,621.97050069)
\lineto(358.24620593,616.78115806)
\lineto(355.46486762,616.78115806)
\lineto(355.46486762,626.97317633)
\lineto(357.57420473,626.97317633)
\lineto(357.98487213,625.25583991)
\lineto(358.11553903,625.25583991)
\curveto(358.41420624,625.77850752)(358.82487364,626.22650832)(359.34754124,626.59984232)
\curveto(359.87020885,626.97317633)(360.47376548,627.15984333)(361.15821115,627.15984333)
\closepath
}
}
{
\newrgbcolor{curcolor}{0 0 0}
\pscustom[linestyle=none,fillstyle=solid,fillcolor=curcolor]
{
\newpath
\moveto(371.05156593,619.80516348)
\curveto(371.05156593,618.77227274)(370.68445416,617.97582686)(369.95023062,617.41582586)
\curveto(369.22845155,616.86826932)(368.14578294,616.59449105)(366.7022248,616.59449105)
\curveto(365.99289019,616.59449105)(365.38311132,616.64426892)(364.87288818,616.74382465)
\curveto(364.36266505,616.83093592)(363.85244191,616.98026952)(363.34221877,617.19182546)
\lineto(363.34221877,619.48782957)
\curveto(363.88977531,619.23894024)(364.48088748,619.03360654)(365.11555529,618.87182847)
\curveto(365.75022309,618.7100504)(366.31022409,618.62916137)(366.7955583,618.62916137)
\curveto(367.33067037,618.62916137)(367.71644884,618.7100504)(367.9528937,618.87182847)
\curveto(368.18933857,619.03360654)(368.30756101,619.24516247)(368.30756101,619.50649627)
\curveto(368.30756101,619.68071881)(368.25778314,619.83627464)(368.15822741,619.97316378)
\curveto(368.07111614,620.11005291)(367.87200467,620.26560875)(367.560893,620.43983128)
\curveto(367.24978133,620.61405381)(366.76444713,620.83805422)(366.10489039,621.11183248)
\curveto(365.45777812,621.38561075)(364.92888828,621.65316679)(364.51822088,621.91450059)
\curveto(364.11999795,622.18827886)(363.82133074,622.51183499)(363.62221928,622.885169)
\curveto(363.42310781,623.27094747)(363.32355207,623.75005944)(363.32355207,624.32250491)
\curveto(363.32355207,625.26828438)(363.69066384,625.97761898)(364.42488738,626.45050872)
\curveto(365.15911092,626.92339846)(366.13600156,627.15984333)(367.3555593,627.15984333)
\curveto(367.99022711,627.15984333)(368.59378374,627.09762099)(369.16622921,626.97317633)
\curveto(369.73867468,626.84873166)(370.32978685,626.64339796)(370.93956573,626.35717522)
\lineto(370.09956422,624.35983831)
\curveto(369.60178555,624.57139424)(369.12889581,624.74561678)(368.68089501,624.88250591)
\curveto(368.23289421,625.03183951)(367.77867117,625.10650631)(367.3182259,625.10650631)
\curveto(366.49689109,625.10650631)(366.08622369,624.88250591)(366.08622369,624.43450511)
\curveto(366.08622369,624.27272704)(366.13600156,624.12339344)(366.23555729,623.9865043)
\curveto(366.34755749,623.86205964)(366.5528912,623.7251705)(366.8515584,623.5758369)
\curveto(367.16267007,623.4265033)(367.6168931,623.22739183)(368.21422751,622.9785025)
\curveto(368.79911744,622.74205763)(369.30311835,622.49316829)(369.72623022,622.23183449)
\curveto(370.14934209,621.98294516)(370.47289822,621.66561126)(370.69689862,621.27983279)
\curveto(370.93334349,620.89405432)(371.05156593,620.40249788)(371.05156593,619.80516348)
\closepath
}
}
{
\newrgbcolor{curcolor}{0 0 0}
\pscustom[linestyle=none,fillstyle=solid,fillcolor=curcolor]
{
\newpath
\moveto(219.36559804,566.22639378)
\curveto(219.50248717,566.22639378)(219.66426524,566.22017154)(219.85093224,566.20772708)
\curveto(220.03759924,566.19528261)(220.18693284,566.17661591)(220.29893304,566.15172698)
\lineto(220.09359934,563.53838896)
\curveto(219.99404361,563.56327789)(219.86337671,563.58194459)(219.70159864,563.59438906)
\curveto(219.53982057,563.61927799)(219.3967092,563.63172246)(219.27226454,563.63172246)
\curveto(218.7993748,563.63172246)(218.34515176,563.54461119)(217.90959543,563.37038866)
\curveto(217.47403909,563.20861059)(217.11937179,562.94105455)(216.84559352,562.56772055)
\curveto(216.58425972,562.19438655)(216.45359282,561.68416341)(216.45359282,561.03705114)
\lineto(216.45359282,555.8477085)
\lineto(213.6722545,555.8477085)
\lineto(213.6722545,566.03972677)
\lineto(215.78159161,566.03972677)
\lineto(216.19225902,564.32239036)
\lineto(216.32292592,564.32239036)
\curveto(216.62159312,564.84505797)(217.03226052,565.29305877)(217.55492812,565.66639277)
\curveto(218.07759573,566.03972677)(218.68115237,566.22639378)(219.36559804,566.22639378)
\closepath
}
}
{
\newrgbcolor{curcolor}{0 0 0}
\pscustom[linestyle=none,fillstyle=solid,fillcolor=curcolor]
{
\newpath
\moveto(231.03228558,560.96238434)
\curveto(231.03228558,559.26993686)(230.58428478,557.96326785)(229.68828317,557.04237731)
\curveto(228.80472603,556.12148677)(227.59761276,555.6610415)(226.06694335,555.6610415)
\curveto(225.12116388,555.6610415)(224.27494014,555.86637521)(223.52827213,556.27704261)
\curveto(222.79404859,556.68771001)(222.21538089,557.28504441)(221.79226902,558.06904582)
\curveto(221.36915715,558.86549169)(221.15760121,559.82993787)(221.15760121,560.96238434)
\curveto(221.15760121,562.65483182)(221.59937978,563.95527859)(222.48293692,564.86372467)
\curveto(223.36649406,565.77217074)(224.57982957,566.22639378)(226.12294345,566.22639378)
\curveto(227.08116739,566.22639378)(227.92739113,566.02106007)(228.66161467,565.61039267)
\curveto(229.3958382,565.19972527)(229.97450591,564.60239086)(230.39761778,563.81838946)
\curveto(230.82072965,563.03438805)(231.03228558,562.08238635)(231.03228558,560.96238434)
\closepath
\moveto(223.99493963,560.96238434)
\curveto(223.99493963,559.95438253)(224.1567177,559.18904783)(224.48027384,558.66638022)
\curveto(224.81627444,558.15615709)(225.35760874,557.90104552)(226.10427675,557.90104552)
\curveto(226.83850029,557.90104552)(227.36739012,558.15615709)(227.69094626,558.66638022)
\curveto(228.02694686,559.18904783)(228.19494716,559.95438253)(228.19494716,560.96238434)
\curveto(228.19494716,561.97038615)(228.02694686,562.72327639)(227.69094626,563.22105506)
\curveto(227.36739012,563.73127819)(226.83227805,563.98638976)(226.08561005,563.98638976)
\curveto(225.35138651,563.98638976)(224.81627444,563.73127819)(224.48027384,563.22105506)
\curveto(224.1567177,562.72327639)(223.99493963,561.97038615)(223.99493963,560.96238434)
\closepath
}
}
{
\newrgbcolor{curcolor}{0 0 0}
\pscustom[linestyle=none,fillstyle=solid,fillcolor=curcolor]
{
\newpath
\moveto(242.58696245,560.96238434)
\curveto(242.58696245,559.26993686)(242.13896164,557.96326785)(241.24296004,557.04237731)
\curveto(240.3594029,556.12148677)(239.15228962,555.6610415)(237.62162021,555.6610415)
\curveto(236.67584074,555.6610415)(235.829617,555.86637521)(235.082949,556.27704261)
\curveto(234.34872546,556.68771001)(233.77005775,557.28504441)(233.34694588,558.06904582)
\curveto(232.92383401,558.86549169)(232.71227808,559.82993787)(232.71227808,560.96238434)
\curveto(232.71227808,562.65483182)(233.15405665,563.95527859)(234.03761379,564.86372467)
\curveto(234.92117093,565.77217074)(236.13450644,566.22639378)(237.67762031,566.22639378)
\curveto(238.63584425,566.22639378)(239.48206799,566.02106007)(240.21629153,565.61039267)
\curveto(240.95051507,565.19972527)(241.52918277,564.60239086)(241.95229464,563.81838946)
\curveto(242.37540651,563.03438805)(242.58696245,562.08238635)(242.58696245,560.96238434)
\closepath
\moveto(235.5496165,560.96238434)
\curveto(235.5496165,559.95438253)(235.71139457,559.18904783)(236.0349507,558.66638022)
\curveto(236.3709513,558.15615709)(236.91228561,557.90104552)(237.65895361,557.90104552)
\curveto(238.39317715,557.90104552)(238.92206699,558.15615709)(239.24562312,558.66638022)
\curveto(239.58162373,559.18904783)(239.74962403,559.95438253)(239.74962403,560.96238434)
\curveto(239.74962403,561.97038615)(239.58162373,562.72327639)(239.24562312,563.22105506)
\curveto(238.92206699,563.73127819)(238.38695492,563.98638976)(237.64028691,563.98638976)
\curveto(236.90606337,563.98638976)(236.3709513,563.73127819)(236.0349507,563.22105506)
\curveto(235.71139457,562.72327639)(235.5496165,561.97038615)(235.5496165,560.96238434)
\closepath
}
}
{
\newrgbcolor{curcolor}{0 0 0}
\pscustom[linestyle=none,fillstyle=solid,fillcolor=curcolor]
{
\newpath
\moveto(256.84831369,566.22639378)
\curveto(258.0056491,566.22639378)(258.87676177,565.92772657)(259.46165171,565.33039217)
\curveto(260.05898611,564.74550223)(260.35765331,563.79972276)(260.35765331,562.49305375)
\lineto(260.35765331,555.8477085)
\lineto(257.576315,555.8477085)
\lineto(257.576315,561.80238585)
\curveto(257.576315,563.27083292)(257.06609186,564.00505646)(256.04564558,564.00505646)
\curveto(255.31142205,564.00505646)(254.78875444,563.74372266)(254.47764277,563.22105506)
\curveto(254.1665311,562.69838745)(254.01097527,561.94549721)(254.01097527,560.96238434)
\lineto(254.01097527,555.8477085)
\lineto(251.22963695,555.8477085)
\lineto(251.22963695,561.80238585)
\curveto(251.22963695,563.27083292)(250.71941381,564.00505646)(249.69896754,564.00505646)
\curveto(248.9274106,564.00505646)(248.39229853,563.71261149)(248.09363133,563.12772155)
\curveto(247.80740859,562.55527608)(247.66429723,561.72771905)(247.66429723,560.64505044)
\lineto(247.66429723,555.8477085)
\lineto(244.88295891,555.8477085)
\lineto(244.88295891,566.03972677)
\lineto(247.01096272,566.03972677)
\lineto(247.38429673,564.73305777)
\lineto(247.53363033,564.73305777)
\curveto(247.84474199,565.25572537)(248.26785386,565.63528161)(248.80296593,565.87172647)
\curveto(249.35052247,566.10817134)(249.91674571,566.22639378)(250.50163565,566.22639378)
\curveto(251.24830365,566.22639378)(251.87674922,566.10194911)(252.38697236,565.85305977)
\curveto(252.90963996,565.61661491)(253.31408513,565.2432809)(253.60030787,564.73305777)
\lineto(253.84297497,564.73305777)
\curveto(254.15408664,565.25572537)(254.58342074,565.63528161)(255.13097728,565.87172647)
\curveto(255.69097828,566.10817134)(256.26342375,566.22639378)(256.84831369,566.22639378)
\closepath
}
}
{
\newrgbcolor{curcolor}{0 0 0}
\pscustom[linestyle=none,fillstyle=solid,fillcolor=curcolor]
{
\newpath
\moveto(263.08450739,526.04446891)
\curveto(261.56628245,526.04446891)(260.39028034,526.46135854)(259.55650106,527.29513782)
\curveto(258.73516626,528.12891709)(258.32449886,529.4542528)(258.32449886,531.27114494)
\curveto(258.32449886,532.51559162)(258.53605479,533.52981566)(258.95916666,534.31381706)
\curveto(259.38227853,535.09781847)(259.96716847,535.67648617)(260.71383647,536.04982018)
\curveto(261.47294894,536.42315418)(262.34406162,536.60982118)(263.32717449,536.60982118)
\curveto(264.02406463,536.60982118)(264.62762127,536.54137661)(265.1378444,536.40448748)
\curveto(265.66051201,536.26759834)(266.11473504,536.10582028)(266.50051351,535.91915328)
\lineto(265.67917871,533.77248276)
\curveto(265.24362237,533.9467053)(264.83295497,534.08981666)(264.4471765,534.20181686)
\curveto(264.0738425,534.31381706)(263.70050849,534.36981716)(263.32717449,534.36981716)
\curveto(261.88361635,534.36981716)(261.16183728,533.34314866)(261.16183728,531.28981164)
\curveto(261.16183728,530.26936537)(261.34850428,529.51647513)(261.72183828,529.03114093)
\curveto(262.10761675,528.54580672)(262.64272882,528.30313962)(263.32717449,528.30313962)
\curveto(263.91206443,528.30313962)(264.4285098,528.37780642)(264.8765106,528.52714002)
\curveto(265.3245114,528.68891809)(265.76006774,528.90669626)(266.18317961,529.18047453)
\lineto(266.18317961,526.80980361)
\curveto(265.76006774,526.53602534)(265.31206694,526.34313611)(264.8391772,526.23113591)
\curveto(264.37873193,526.10669124)(263.79384199,526.04446891)(263.08450739,526.04446891)
\closepath
}
}
{
\newrgbcolor{curcolor}{0 0 0}
\pscustom[linestyle=none,fillstyle=solid,fillcolor=curcolor]
{
\newpath
\moveto(277.7938658,531.34581174)
\curveto(277.7938658,529.65336427)(277.34586499,528.34669526)(276.44986339,527.42580472)
\curveto(275.56630625,526.50491418)(274.35919297,526.04446891)(272.82852356,526.04446891)
\curveto(271.88274409,526.04446891)(271.03652035,526.24980261)(270.28985235,526.66047001)
\curveto(269.55562881,527.07113741)(268.9769611,527.66847182)(268.55384923,528.45247322)
\curveto(268.13073736,529.2489191)(267.91918143,530.21336527)(267.91918143,531.34581174)
\curveto(267.91918143,533.03825922)(268.36096,534.338706)(269.24451714,535.24715207)
\curveto(270.12807428,536.15559814)(271.34140979,536.60982118)(272.88452366,536.60982118)
\curveto(273.8427476,536.60982118)(274.68897134,536.40448748)(275.42319488,535.99382008)
\curveto(276.15741842,535.58315267)(276.73608612,534.98581827)(277.15919799,534.20181686)
\curveto(277.58230986,533.41781546)(277.7938658,532.46581375)(277.7938658,531.34581174)
\closepath
\moveto(270.75651985,531.34581174)
\curveto(270.75651985,530.33780994)(270.91829792,529.57247523)(271.24185405,529.04980763)
\curveto(271.57785465,528.53958449)(272.11918896,528.28447292)(272.86585696,528.28447292)
\curveto(273.6000805,528.28447292)(274.12897034,528.53958449)(274.45252647,529.04980763)
\curveto(274.78852708,529.57247523)(274.95652738,530.33780994)(274.95652738,531.34581174)
\curveto(274.95652738,532.35381355)(274.78852708,533.10670379)(274.45252647,533.60448246)
\curveto(274.12897034,534.1147056)(273.59385827,534.36981716)(272.84719026,534.36981716)
\curveto(272.11296672,534.36981716)(271.57785465,534.1147056)(271.24185405,533.60448246)
\curveto(270.91829792,533.10670379)(270.75651985,532.35381355)(270.75651985,531.34581174)
\closepath
}
}
{
\newrgbcolor{curcolor}{0 0 0}
\pscustom[linestyle=none,fillstyle=solid,fillcolor=curcolor]
{
\newpath
\moveto(285.87653739,536.60982118)
\curveto(286.97165046,536.60982118)(287.84898537,536.31115398)(288.50854211,535.71381957)
\curveto(289.16809885,535.12892964)(289.49787722,534.18315016)(289.49787722,532.87648115)
\lineto(289.49787722,526.23113591)
\lineto(286.7165389,526.23113591)
\lineto(286.7165389,532.18581325)
\curveto(286.7165389,532.92003679)(286.585872,533.46759333)(286.32453819,533.82848286)
\curveto(286.06320439,534.20181686)(285.64631476,534.38848387)(285.07386928,534.38848387)
\curveto(284.22764555,534.38848387)(283.64897784,534.0960389)(283.33786617,533.51114896)
\curveto(283.0267545,532.93870349)(282.87119867,532.11114645)(282.87119867,531.02847784)
\lineto(282.87119867,526.23113591)
\lineto(280.08986035,526.23113591)
\lineto(280.08986035,536.42315418)
\lineto(282.21786417,536.42315418)
\lineto(282.59119817,535.11648517)
\lineto(282.74053177,535.11648517)
\curveto(283.0640879,535.63915277)(283.50586647,536.01870901)(284.06586748,536.25515388)
\curveto(284.63831295,536.49159875)(285.24186959,536.60982118)(285.87653739,536.60982118)
\closepath
}
}
{
\newrgbcolor{curcolor}{0 0 0}
\pscustom[linestyle=none,fillstyle=solid,fillcolor=curcolor]
{
\newpath
\moveto(298.14055243,536.60982118)
\curveto(299.23566551,536.60982118)(300.11300041,536.31115398)(300.77255715,535.71381957)
\curveto(301.43211389,535.12892964)(301.76189226,534.18315016)(301.76189226,532.87648115)
\lineto(301.76189226,526.23113591)
\lineto(298.98055394,526.23113591)
\lineto(298.98055394,532.18581325)
\curveto(298.98055394,532.92003679)(298.84988704,533.46759333)(298.58855324,533.82848286)
\curveto(298.32721943,534.20181686)(297.9103298,534.38848387)(297.33788433,534.38848387)
\curveto(296.49166059,534.38848387)(295.91299289,534.0960389)(295.60188122,533.51114896)
\curveto(295.29076955,532.93870349)(295.13521371,532.11114645)(295.13521371,531.02847784)
\lineto(295.13521371,526.23113591)
\lineto(292.35387539,526.23113591)
\lineto(292.35387539,536.42315418)
\lineto(294.48187921,536.42315418)
\lineto(294.85521321,535.11648517)
\lineto(295.00454681,535.11648517)
\curveto(295.32810295,535.63915277)(295.76988152,536.01870901)(296.32988252,536.25515388)
\curveto(296.90232799,536.49159875)(297.50588463,536.60982118)(298.14055243,536.60982118)
\closepath
}
}
{
\newrgbcolor{curcolor}{0 0 0}
\pscustom[linestyle=none,fillstyle=solid,fillcolor=curcolor]
{
\newpath
\moveto(308.81789797,536.60982118)
\curveto(310.22412271,536.60982118)(311.33790248,536.20537601)(312.15923729,535.39648567)
\curveto(312.98057209,534.6000398)(313.3912395,533.46137109)(313.3912395,531.98047955)
\lineto(313.3912395,530.63647714)
\lineto(306.82056105,530.63647714)
\curveto(306.84544999,529.85247573)(307.07567262,529.23647463)(307.51122896,528.78847383)
\curveto(307.95922976,528.34047302)(308.57523086,528.11647262)(309.35923227,528.11647262)
\curveto(310.00634454,528.11647262)(310.59745671,528.17869496)(311.13256878,528.30313962)
\curveto(311.68012532,528.44002876)(312.24012632,528.64536246)(312.81257179,528.91914073)
\lineto(312.81257179,526.77247021)
\curveto(312.30234866,526.52358088)(311.77345882,526.34313611)(311.22590228,526.23113591)
\curveto(310.67834575,526.10669124)(310.01256677,526.04446891)(309.22856537,526.04446891)
\curveto(308.2081191,526.04446891)(307.30589526,526.23113591)(306.52189385,526.60446991)
\curveto(305.73789244,526.99024838)(305.12189134,527.56269385)(304.67389054,528.32180632)
\curveto(304.22588973,529.09336326)(304.00188933,530.0702539)(304.00188933,531.25247824)
\curveto(304.00188933,532.43470258)(304.2010008,533.42403769)(304.59922374,534.22048356)
\curveto(305.00989114,535.01692944)(305.57611438,535.61426384)(306.29789345,536.01248678)
\curveto(307.01967252,536.41070971)(307.85967403,536.60982118)(308.81789797,536.60982118)
\closepath
\moveto(308.83656467,534.63115097)
\curveto(308.28900813,534.63115097)(307.84100733,534.45692843)(307.49256226,534.10848336)
\curveto(307.14411719,533.76003829)(306.93878349,533.21870399)(306.87656115,532.48448045)
\lineto(310.77790148,532.48448045)
\curveto(310.76545701,533.09425932)(310.59745671,533.60448246)(310.27390058,534.01514986)
\curveto(309.96278891,534.42581727)(309.48367694,534.63115097)(308.83656467,534.63115097)
\closepath
}
}
{
\newrgbcolor{curcolor}{0 0 0}
\pscustom[linestyle=none,fillstyle=solid,fillcolor=curcolor]
{
\newpath
\moveto(319.79391067,526.04446891)
\curveto(318.27568572,526.04446891)(317.09968361,526.46135854)(316.26590434,527.29513782)
\curveto(315.44456954,528.12891709)(315.03390213,529.4542528)(315.03390213,531.27114494)
\curveto(315.03390213,532.51559162)(315.24545807,533.52981566)(315.66856994,534.31381706)
\curveto(316.09168181,535.09781847)(316.67657174,535.67648617)(317.42323975,536.04982018)
\curveto(318.18235222,536.42315418)(319.05346489,536.60982118)(320.03657777,536.60982118)
\curveto(320.73346791,536.60982118)(321.33702454,536.54137661)(321.84724768,536.40448748)
\curveto(322.36991528,536.26759834)(322.82413832,536.10582028)(323.20991679,535.91915328)
\lineto(322.38858198,533.77248276)
\curveto(321.95302565,533.9467053)(321.54235824,534.08981666)(321.15657978,534.20181686)
\curveto(320.78324577,534.31381706)(320.40991177,534.36981716)(320.03657777,534.36981716)
\curveto(318.59301962,534.36981716)(317.87124055,533.34314866)(317.87124055,531.28981164)
\curveto(317.87124055,530.26936537)(318.05790755,529.51647513)(318.43124156,529.03114093)
\curveto(318.81702003,528.54580672)(319.3521321,528.30313962)(320.03657777,528.30313962)
\curveto(320.62146771,528.30313962)(321.13791308,528.37780642)(321.58591388,528.52714002)
\curveto(322.03391468,528.68891809)(322.46947102,528.90669626)(322.89258289,529.18047453)
\lineto(322.89258289,526.80980361)
\curveto(322.46947102,526.53602534)(322.02147021,526.34313611)(321.54858048,526.23113591)
\curveto(321.08813521,526.10669124)(320.50324527,526.04446891)(319.79391067,526.04446891)
\closepath
}
}
{
\newrgbcolor{curcolor}{0 0 0}
\pscustom[linestyle=none,fillstyle=solid,fillcolor=curcolor]
{
\newpath
\moveto(329.53792493,528.26580622)
\curveto(329.8490366,528.26580622)(330.1477038,528.29069516)(330.43392654,528.34047302)
\curveto(330.72014927,528.40269536)(331.00637201,528.48358439)(331.29259474,528.58314013)
\lineto(331.29259474,526.51113641)
\curveto(330.99392754,526.37424728)(330.62059354,526.26224708)(330.17259274,526.17513581)
\curveto(329.7370364,526.08802454)(329.25792443,526.04446891)(328.73525683,526.04446891)
\curveto(328.12547796,526.04446891)(327.57792142,526.14402464)(327.09258722,526.34313611)
\curveto(326.61969748,526.54224758)(326.24014124,526.88447041)(325.95391851,527.36980462)
\curveto(325.68014024,527.85513882)(325.54325111,528.53958449)(325.54325111,529.42314163)
\lineto(325.54325111,534.33248376)
\lineto(324.2179154,534.33248376)
\lineto(324.2179154,535.50848587)
\lineto(325.74858481,536.44182088)
\lineto(326.55125291,538.58849139)
\lineto(328.32458942,538.58849139)
\lineto(328.32458942,536.42315418)
\lineto(331.18059454,536.42315418)
\lineto(331.18059454,534.33248376)
\lineto(328.32458942,534.33248376)
\lineto(328.32458942,529.42314163)
\curveto(328.32458942,529.03736316)(328.43658962,528.74491819)(328.66059003,528.54580672)
\curveto(328.88459043,528.35913972)(329.1770354,528.26580622)(329.53792493,528.26580622)
\closepath
}
}
{
\newrgbcolor{curcolor}{0 0 0}
\pscustom[linestyle=none,fillstyle=solid,fillcolor=curcolor]
{
\newpath
\moveto(332.95393198,527.53780492)
\curveto(332.95393198,528.11025039)(333.10948781,528.50847332)(333.42059948,528.73247373)
\curveto(333.73171115,528.96891859)(334.11126738,529.08714103)(334.55926819,529.08714103)
\curveto(334.99482452,529.08714103)(335.36815853,528.96891859)(335.67927019,528.73247373)
\curveto(335.99038186,528.50847332)(336.1459377,528.11025039)(336.1459377,527.53780492)
\curveto(336.1459377,526.99024838)(335.99038186,526.59202544)(335.67927019,526.34313611)
\curveto(335.36815853,526.10669124)(334.99482452,525.98846881)(334.55926819,525.98846881)
\curveto(334.11126738,525.98846881)(333.73171115,526.10669124)(333.42059948,526.34313611)
\curveto(333.10948781,526.59202544)(332.95393198,526.99024838)(332.95393198,527.53780492)
\closepath
}
}
{
\newrgbcolor{curcolor}{0 0 0}
\pscustom[linestyle=none,fillstyle=solid,fillcolor=curcolor]
{
\newpath
\moveto(338.55393973,539.0551589)
\curveto(338.55393973,539.5778265)(338.6970511,539.9324938)(338.98327384,540.1191608)
\curveto(339.28194104,540.31827227)(339.64283058,540.41782801)(340.06594244,540.41782801)
\curveto(340.47660985,540.41782801)(340.83127715,540.31827227)(341.12994435,540.1191608)
\curveto(341.42861155,539.9324938)(341.57794516,539.5778265)(341.57794516,539.0551589)
\curveto(341.57794516,538.54493576)(341.42861155,538.19026846)(341.12994435,537.99115699)
\curveto(340.83127715,537.79204552)(340.47660985,537.69248979)(340.06594244,537.69248979)
\curveto(339.64283058,537.69248979)(339.28194104,537.79204552)(338.98327384,537.99115699)
\curveto(338.6970511,538.19026846)(338.55393973,538.54493576)(338.55393973,539.0551589)
\closepath
\moveto(337.84460513,521.75112788)
\curveto(337.52104899,521.75112788)(337.19127062,521.77601681)(336.85527002,521.82579468)
\curveto(336.51926942,521.86312808)(336.23926892,521.91290595)(336.01526852,521.97512828)
\lineto(336.01526852,524.15913219)
\curveto(336.23926892,524.09690986)(336.45082485,524.05335423)(336.64993632,524.02846529)
\curveto(336.84904779,524.00357636)(337.07304819,523.99113189)(337.32193753,523.99113189)
\curveto(337.69527153,523.99113189)(338.01260543,524.09690986)(338.27393923,524.3084658)
\curveto(338.53527303,524.52002173)(338.66593994,524.93068913)(338.66593994,525.540468)
\lineto(338.66593994,536.42315418)
\lineto(341.44727825,536.42315418)
\lineto(341.44727825,525.1298006)
\curveto(341.44727825,524.50757726)(341.32905582,523.94135403)(341.09261095,523.43113089)
\curveto(340.85616608,522.92090775)(340.47038761,522.51646258)(339.93527554,522.21779538)
\curveto(339.41260794,521.90668371)(338.7157178,521.75112788)(337.84460513,521.75112788)
\closepath
}
}
{
\newrgbcolor{curcolor}{0 0 0}
\pscustom[linestyle=none,fillstyle=solid,fillcolor=curcolor]
{
\newpath
\moveto(351.47130436,529.25514133)
\curveto(351.47130436,528.22225059)(351.10419259,527.42580472)(350.36996905,526.86580371)
\curveto(349.64818998,526.31824718)(348.56552137,526.04446891)(347.12196323,526.04446891)
\curveto(346.41262862,526.04446891)(345.80284975,526.09424677)(345.29262662,526.19380251)
\curveto(344.78240348,526.28091378)(344.27218034,526.43024738)(343.7619572,526.64180331)
\lineto(343.7619572,528.93780743)
\curveto(344.30951374,528.68891809)(344.90062591,528.48358439)(345.53529372,528.32180632)
\curveto(346.16996152,528.16002826)(346.72996253,528.07913922)(347.21529673,528.07913922)
\curveto(347.7504088,528.07913922)(348.13618727,528.16002826)(348.37263214,528.32180632)
\curveto(348.609077,528.48358439)(348.72729944,528.69514033)(348.72729944,528.95647413)
\curveto(348.72729944,529.13069666)(348.67752157,529.2862525)(348.57796584,529.42314163)
\curveto(348.49085457,529.56003077)(348.2917431,529.7155866)(347.98063143,529.88980913)
\curveto(347.66951976,530.06403167)(347.18418556,530.28803207)(346.52462882,530.56181034)
\curveto(345.87751655,530.83558861)(345.34862672,531.10314464)(344.93795931,531.36447844)
\curveto(344.53973638,531.63825671)(344.24106917,531.96181285)(344.04195771,532.33514685)
\curveto(343.84284624,532.72092532)(343.7432905,533.20003729)(343.7432905,533.77248276)
\curveto(343.7432905,534.71826223)(344.11040227,535.42759684)(344.84462581,535.90048658)
\curveto(345.57884935,536.37337631)(346.55573999,536.60982118)(347.77529773,536.60982118)
\curveto(348.40996554,536.60982118)(349.01352217,536.54759885)(349.58596764,536.42315418)
\curveto(350.15841312,536.29870951)(350.74952529,536.09337581)(351.35930416,535.80715307)
\lineto(350.51930265,533.80981616)
\curveto(350.02152398,534.0213721)(349.54863424,534.19559463)(349.10063344,534.33248376)
\curveto(348.65263264,534.48181737)(348.1984096,534.55648417)(347.73796433,534.55648417)
\curveto(346.91662953,534.55648417)(346.50596212,534.33248376)(346.50596212,533.88448296)
\curveto(346.50596212,533.72270489)(346.55573999,533.57337129)(346.65529572,533.43648216)
\curveto(346.76729593,533.31203749)(346.97262963,533.17514836)(347.27129683,533.02581476)
\curveto(347.5824085,532.87648115)(348.03663153,532.67736969)(348.63396594,532.42848035)
\curveto(349.21885588,532.19203548)(349.72285678,531.94314615)(350.14596865,531.68181235)
\curveto(350.56908052,531.43292301)(350.89263665,531.11558911)(351.11663706,530.72981064)
\curveto(351.35308192,530.34403217)(351.47130436,529.85247573)(351.47130436,529.25514133)
\closepath
}
}
{
\newrgbcolor{curcolor}{0 0 0}
\pscustom[linestyle=none,fillstyle=solid,fillcolor=curcolor]
{
\newpath
\moveto(262.14165542,504.87843488)
\curveto(260.62343048,504.87843488)(259.44742837,505.29532451)(258.6136491,506.12910379)
\curveto(257.79231429,506.96288306)(257.38164689,508.28821877)(257.38164689,510.10511091)
\curveto(257.38164689,511.34955759)(257.59320282,512.36378163)(258.01631469,513.14778303)
\curveto(258.43942656,513.93178444)(259.0243165,514.51045214)(259.7709845,514.88378615)
\curveto(260.53009698,515.25712015)(261.40120965,515.44378715)(262.38432252,515.44378715)
\curveto(263.08121266,515.44378715)(263.6847693,515.37534258)(264.19499243,515.23845345)
\curveto(264.71766004,515.10156431)(265.17188307,514.93978625)(265.55766154,514.75311924)
\lineto(264.73632674,512.60644873)
\curveto(264.3007704,512.78067126)(263.890103,512.92378263)(263.50432453,513.03578283)
\curveto(263.13099053,513.14778303)(262.75765652,513.20378313)(262.38432252,513.20378313)
\curveto(260.94076438,513.20378313)(260.21898531,512.17711463)(260.21898531,510.12377761)
\curveto(260.21898531,509.10333134)(260.40565231,508.3504411)(260.77898631,507.8651069)
\curveto(261.16476478,507.37977269)(261.69987685,507.13710559)(262.38432252,507.13710559)
\curveto(262.96921246,507.13710559)(263.48565783,507.21177239)(263.93365863,507.36110599)
\curveto(264.38165944,507.52288406)(264.81721577,507.74066223)(265.24032764,508.0144405)
\lineto(265.24032764,505.64376958)
\curveto(264.81721577,505.36999131)(264.36921497,505.17710208)(263.89632523,505.06510188)
\curveto(263.43587996,504.94065721)(262.85099003,504.87843488)(262.14165542,504.87843488)
\closepath
}
}
{
\newrgbcolor{curcolor}{0 0 0}
\pscustom[linestyle=none,fillstyle=solid,fillcolor=curcolor]
{
\newpath
\moveto(276.94434733,515.25712015)
\lineto(276.94434733,505.06510188)
\lineto(274.81634351,505.06510188)
\lineto(274.44300951,506.37177089)
\lineto(274.29367591,506.37177089)
\curveto(273.97011978,505.84910328)(273.52211897,505.46954705)(272.9496735,505.23310218)
\curveto(272.3896725,504.99665731)(271.79233809,504.87843488)(271.15767029,504.87843488)
\curveto(270.06255721,504.87843488)(269.18522231,505.17087985)(268.52566557,505.75576978)
\curveto(267.86610883,506.35310419)(267.53633046,507.30510589)(267.53633046,508.6117749)
\lineto(267.53633046,515.25712015)
\lineto(270.31766878,515.25712015)
\lineto(270.31766878,509.30244281)
\curveto(270.31766878,508.56821927)(270.44833568,508.0144405)(270.70966949,507.6411065)
\curveto(270.97100329,507.28021696)(271.38789292,507.09977219)(271.96033839,507.09977219)
\curveto(272.80656213,507.09977219)(273.38522984,507.38599493)(273.69634151,507.9584404)
\curveto(274.00745318,508.54333034)(274.16300901,509.37710961)(274.16300901,510.45977822)
\lineto(274.16300901,515.25712015)
\closepath
}
}
{
\newrgbcolor{curcolor}{0 0 0}
\pscustom[linestyle=none,fillstyle=solid,fillcolor=curcolor]
{
\newpath
\moveto(285.54968724,515.44378715)
\curveto(285.68657637,515.44378715)(285.84835444,515.43756492)(286.03502144,515.42512045)
\curveto(286.22168844,515.41267598)(286.37102205,515.39400928)(286.48302225,515.36912035)
\lineto(286.27768854,512.75578233)
\curveto(286.17813281,512.78067126)(286.04746591,512.79933796)(285.88568784,512.81178243)
\curveto(285.72390977,512.83667136)(285.58079841,512.84911583)(285.45635374,512.84911583)
\curveto(284.983464,512.84911583)(284.52924097,512.76200456)(284.09368463,512.58778203)
\curveto(283.65812829,512.42600396)(283.30346099,512.15844793)(283.02968272,511.78511392)
\curveto(282.76834892,511.41177992)(282.63768202,510.90155679)(282.63768202,510.25444451)
\lineto(282.63768202,505.06510188)
\lineto(279.8563437,505.06510188)
\lineto(279.8563437,515.25712015)
\lineto(281.96568082,515.25712015)
\lineto(282.37634822,513.53978374)
\lineto(282.50701512,513.53978374)
\curveto(282.80568232,514.06245134)(283.21634972,514.51045214)(283.73901733,514.88378615)
\curveto(284.26168493,515.25712015)(284.86524157,515.44378715)(285.54968724,515.44378715)
\closepath
}
}
{
\newrgbcolor{curcolor}{0 0 0}
\pscustom[linestyle=none,fillstyle=solid,fillcolor=curcolor]
{
\newpath
\moveto(295.44304773,508.0891073)
\curveto(295.44304773,507.05621656)(295.07593596,506.25977069)(294.34171243,505.69976968)
\curveto(293.61993335,505.15221315)(292.53726475,504.87843488)(291.0937066,504.87843488)
\curveto(290.384372,504.87843488)(289.77459313,504.92821274)(289.26436999,505.02776848)
\curveto(288.75414685,505.11487975)(288.24392372,505.26421335)(287.73370058,505.47576928)
\lineto(287.73370058,507.7717734)
\curveto(288.28125712,507.52288406)(288.87236929,507.31755036)(289.50703709,507.15577229)
\curveto(290.1417049,506.99399422)(290.7017059,506.91310519)(291.1870401,506.91310519)
\curveto(291.72215217,506.91310519)(292.10793064,506.99399422)(292.34437551,507.15577229)
\curveto(292.58082038,507.31755036)(292.69904281,507.5291063)(292.69904281,507.7904401)
\curveto(292.69904281,507.96466263)(292.64926495,508.12021847)(292.54970921,508.2571076)
\curveto(292.46259795,508.39399673)(292.26348648,508.54955257)(291.95237481,508.7237751)
\curveto(291.64126314,508.89799764)(291.15592894,509.12199804)(290.4963722,509.39577631)
\curveto(289.84925993,509.66955458)(289.32037009,509.93711061)(288.90970269,510.19844441)
\curveto(288.51147975,510.47222268)(288.21281255,510.79577882)(288.01370108,511.16911282)
\curveto(287.81458961,511.55489129)(287.71503388,512.03400326)(287.71503388,512.60644873)
\curveto(287.71503388,513.5522282)(288.08214565,514.26156281)(288.81636919,514.73445254)
\curveto(289.55059273,515.20734228)(290.52748337,515.44378715)(291.74704111,515.44378715)
\curveto(292.38170891,515.44378715)(292.98526555,515.38156482)(293.55771102,515.25712015)
\curveto(294.13015649,515.13267548)(294.72126866,514.92734178)(295.33104753,514.64111904)
\lineto(294.49104603,512.64378213)
\curveto(293.99326736,512.85533807)(293.52037762,513.0295606)(293.07237682,513.16644973)
\curveto(292.62437601,513.31578333)(292.17015298,513.39045014)(291.70970771,513.39045014)
\curveto(290.8883729,513.39045014)(290.4777055,513.16644973)(290.4777055,512.71844893)
\curveto(290.4777055,512.55667086)(290.52748337,512.40733726)(290.6270391,512.27044813)
\curveto(290.7390393,512.14600346)(290.944373,512.00911433)(291.2430402,511.85978072)
\curveto(291.55415187,511.71044712)(292.00837491,511.51133566)(292.60570931,511.26244632)
\curveto(293.19059925,511.02600145)(293.69460015,510.77711212)(294.11771202,510.51577832)
\curveto(294.54082389,510.26688898)(294.86438003,509.94955508)(295.08838043,509.56377661)
\curveto(295.3248253,509.17799814)(295.44304773,508.6864417)(295.44304773,508.0891073)
\closepath
}
}
{
\newrgbcolor{curcolor}{0 0 0}
\pscustom[linestyle=none,fillstyle=solid,fillcolor=curcolor]
{
\newpath
\moveto(306.86706379,510.17977771)
\curveto(306.86706379,508.48733024)(306.41906299,507.18066123)(305.52306138,506.25977069)
\curveto(304.63950424,505.33888015)(303.43239097,504.87843488)(301.90172155,504.87843488)
\curveto(300.95594208,504.87843488)(300.10971834,505.08376858)(299.36305034,505.49443598)
\curveto(298.6288268,505.90510338)(298.0501591,506.50243779)(297.62704723,507.28643919)
\curveto(297.20393536,508.08288507)(296.99237942,509.04733124)(296.99237942,510.17977771)
\curveto(296.99237942,511.87222519)(297.43415799,513.17267197)(298.31771513,514.08111804)
\curveto(299.20127227,514.98956411)(300.41460778,515.44378715)(301.95772166,515.44378715)
\curveto(302.9159456,515.44378715)(303.76216933,515.23845345)(304.49639287,514.82778605)
\curveto(305.23061641,514.41711864)(305.80928411,513.81978424)(306.23239598,513.03578283)
\curveto(306.65550785,512.25178143)(306.86706379,511.29977972)(306.86706379,510.17977771)
\closepath
\moveto(299.82971784,510.17977771)
\curveto(299.82971784,509.17177591)(299.99149591,508.4064412)(300.31505204,507.8837736)
\curveto(300.65105265,507.37355046)(301.19238695,507.11843889)(301.93905496,507.11843889)
\curveto(302.67327849,507.11843889)(303.20216833,507.37355046)(303.52572447,507.8837736)
\curveto(303.86172507,508.4064412)(304.02972537,509.17177591)(304.02972537,510.17977771)
\curveto(304.02972537,511.18777952)(303.86172507,511.94066976)(303.52572447,512.43844843)
\curveto(303.20216833,512.94867157)(302.66705626,513.20378313)(301.92038825,513.20378313)
\curveto(301.18616472,513.20378313)(300.65105265,512.94867157)(300.31505204,512.43844843)
\curveto(299.99149591,511.94066976)(299.82971784,511.18777952)(299.82971784,510.17977771)
\closepath
}
}
{
\newrgbcolor{curcolor}{0 0 0}
\pscustom[linestyle=none,fillstyle=solid,fillcolor=curcolor]
{
\newpath
\moveto(314.85640379,515.44378715)
\curveto(314.99329292,515.44378715)(315.15507099,515.43756492)(315.34173799,515.42512045)
\curveto(315.52840499,515.41267598)(315.6777386,515.39400928)(315.7897388,515.36912035)
\lineto(315.58440509,512.75578233)
\curveto(315.48484936,512.78067126)(315.35418246,512.79933796)(315.19240439,512.81178243)
\curveto(315.03062632,512.83667136)(314.88751496,512.84911583)(314.76307029,512.84911583)
\curveto(314.29018055,512.84911583)(313.83595752,512.76200456)(313.40040118,512.58778203)
\curveto(312.96484484,512.42600396)(312.61017754,512.15844793)(312.33639927,511.78511392)
\curveto(312.07506547,511.41177992)(311.94439857,510.90155679)(311.94439857,510.25444451)
\lineto(311.94439857,505.06510188)
\lineto(309.16306025,505.06510188)
\lineto(309.16306025,515.25712015)
\lineto(311.27239736,515.25712015)
\lineto(311.68306477,513.53978374)
\lineto(311.81373167,513.53978374)
\curveto(312.11239887,514.06245134)(312.52306627,514.51045214)(313.04573388,514.88378615)
\curveto(313.56840148,515.25712015)(314.17195812,515.44378715)(314.85640379,515.44378715)
\closepath
}
}
{
\newrgbcolor{curcolor}{0 0 0}
\pscustom[linestyle=none,fillstyle=solid,fillcolor=curcolor]
{
\newpath
\moveto(316.12575764,506.37177089)
\curveto(316.12575764,506.94421636)(316.28131347,507.34243929)(316.59242514,507.5664397)
\curveto(316.90353681,507.80288456)(317.28309304,507.921107)(317.73109385,507.921107)
\curveto(318.16665018,507.921107)(318.53998419,507.80288456)(318.85109585,507.5664397)
\curveto(319.16220752,507.34243929)(319.31776336,506.94421636)(319.31776336,506.37177089)
\curveto(319.31776336,505.82421435)(319.16220752,505.42599141)(318.85109585,505.17710208)
\curveto(318.53998419,504.94065721)(318.16665018,504.82243478)(317.73109385,504.82243478)
\curveto(317.28309304,504.82243478)(316.90353681,504.94065721)(316.59242514,505.17710208)
\curveto(316.28131347,505.42599141)(316.12575764,505.82421435)(316.12575764,506.37177089)
\closepath
}
}
{
\newrgbcolor{curcolor}{0 0 0}
\pscustom[linestyle=none,fillstyle=solid,fillcolor=curcolor]
{
\newpath
\moveto(321.7257654,517.88912487)
\curveto(321.7257654,518.41179247)(321.86887676,518.76645977)(322.1550995,518.95312677)
\curveto(322.4537667,519.15223824)(322.81465624,519.25179398)(323.23776811,519.25179398)
\curveto(323.64843551,519.25179398)(324.00310281,519.15223824)(324.30177001,518.95312677)
\curveto(324.60043721,518.76645977)(324.74977082,518.41179247)(324.74977082,517.88912487)
\curveto(324.74977082,517.37890173)(324.60043721,517.02423443)(324.30177001,516.82512296)
\curveto(324.00310281,516.62601149)(323.64843551,516.52645576)(323.23776811,516.52645576)
\curveto(322.81465624,516.52645576)(322.4537667,516.62601149)(322.1550995,516.82512296)
\curveto(321.86887676,517.02423443)(321.7257654,517.37890173)(321.7257654,517.88912487)
\closepath
\moveto(321.01643079,500.58509385)
\curveto(320.69287465,500.58509385)(320.36309629,500.60998278)(320.02709568,500.65976065)
\curveto(319.69109508,500.69709405)(319.41109458,500.74687192)(319.18709418,500.80909425)
\lineto(319.18709418,502.99309816)
\curveto(319.41109458,502.93087583)(319.62265051,502.8873202)(319.82176198,502.86243126)
\curveto(320.02087345,502.83754233)(320.24487385,502.82509786)(320.49376319,502.82509786)
\curveto(320.86709719,502.82509786)(321.18443109,502.93087583)(321.44576489,503.14243177)
\curveto(321.7070987,503.3539877)(321.8377656,503.7646551)(321.8377656,504.37443397)
\lineto(321.8377656,515.25712015)
\lineto(324.61910392,515.25712015)
\lineto(324.61910392,503.96376657)
\curveto(324.61910392,503.34154323)(324.50088148,502.77532)(324.26443661,502.26509686)
\curveto(324.02799174,501.75487372)(323.64221328,501.35042855)(323.1071012,501.05176135)
\curveto(322.5844336,500.74064968)(321.88754346,500.58509385)(321.01643079,500.58509385)
\closepath
}
}
{
\newrgbcolor{curcolor}{0 0 0}
\pscustom[linestyle=none,fillstyle=solid,fillcolor=curcolor]
{
\newpath
\moveto(334.64312239,508.0891073)
\curveto(334.64312239,507.05621656)(334.27601062,506.25977069)(333.54178708,505.69976968)
\curveto(332.82000801,505.15221315)(331.7373394,504.87843488)(330.29378126,504.87843488)
\curveto(329.58444665,504.87843488)(328.97466778,504.92821274)(328.46444465,505.02776848)
\curveto(327.95422151,505.11487975)(327.44399837,505.26421335)(326.93377524,505.47576928)
\lineto(326.93377524,507.7717734)
\curveto(327.48133177,507.52288406)(328.07244394,507.31755036)(328.70711175,507.15577229)
\curveto(329.34177955,506.99399422)(329.90178056,506.91310519)(330.38711476,506.91310519)
\curveto(330.92222683,506.91310519)(331.3080053,506.99399422)(331.54445017,507.15577229)
\curveto(331.78089504,507.31755036)(331.89911747,507.5291063)(331.89911747,507.7904401)
\curveto(331.89911747,507.96466263)(331.8493396,508.12021847)(331.74978387,508.2571076)
\curveto(331.6626726,508.39399673)(331.46356113,508.54955257)(331.15244947,508.7237751)
\curveto(330.8413378,508.89799764)(330.35600359,509.12199804)(329.69644686,509.39577631)
\curveto(329.04933458,509.66955458)(328.52044475,509.93711061)(328.10977734,510.19844441)
\curveto(327.71155441,510.47222268)(327.41288721,510.79577882)(327.21377574,511.16911282)
\curveto(327.01466427,511.55489129)(326.91510854,512.03400326)(326.91510854,512.60644873)
\curveto(326.91510854,513.5522282)(327.28222031,514.26156281)(328.01644384,514.73445254)
\curveto(328.75066738,515.20734228)(329.72755802,515.44378715)(330.94711576,515.44378715)
\curveto(331.58178357,515.44378715)(332.18534021,515.38156482)(332.75778568,515.25712015)
\curveto(333.33023115,515.13267548)(333.92134332,514.92734178)(334.53112219,514.64111904)
\lineto(333.69112068,512.64378213)
\curveto(333.19334201,512.85533807)(332.72045228,513.0295606)(332.27245147,513.16644973)
\curveto(331.82445067,513.31578333)(331.37022763,513.39045014)(330.90978236,513.39045014)
\curveto(330.08844756,513.39045014)(329.67778016,513.16644973)(329.67778016,512.71844893)
\curveto(329.67778016,512.55667086)(329.72755802,512.40733726)(329.82711376,512.27044813)
\curveto(329.93911396,512.14600346)(330.14444766,512.00911433)(330.44311486,511.85978072)
\curveto(330.75422653,511.71044712)(331.20844957,511.51133566)(331.80578397,511.26244632)
\curveto(332.39067391,511.02600145)(332.89467481,510.77711212)(333.31778668,510.51577832)
\curveto(333.74089855,510.26688898)(334.06445469,509.94955508)(334.28845509,509.56377661)
\curveto(334.52489996,509.17799814)(334.64312239,508.6864417)(334.64312239,508.0891073)
\closepath
}
}
{
\newrgbcolor{curcolor}{0 0 0}
\pscustom[linestyle=none,fillstyle=solid,fillcolor=curcolor]
{
\newpath
\moveto(264.07354821,491.47483286)
\lineto(261.66554389,491.47483286)
\lineto(261.66554389,483.37348501)
\lineto(258.88420557,483.37348501)
\lineto(258.88420557,491.47483286)
\lineto(257.35353616,491.47483286)
\lineto(257.35353616,492.81883527)
\lineto(258.88420557,493.56550328)
\lineto(258.88420557,494.31217128)
\curveto(258.88420557,495.18328396)(259.02731694,495.85528516)(259.31353967,496.3281749)
\curveto(259.61220687,496.8135091)(260.02909651,497.15573194)(260.56420858,497.3548434)
\curveto(261.11176512,497.55395487)(261.75265516,497.65351061)(262.48687869,497.65351061)
\curveto(263.02199076,497.65351061)(263.5135472,497.60995497)(263.961548,497.52284371)
\curveto(264.40954881,497.43573244)(264.77043834,497.3361767)(265.04421661,497.2241765)
\lineto(264.33488201,495.18950619)
\curveto(264.12332607,495.25172852)(263.8868812,495.30772862)(263.6255474,495.35750649)
\curveto(263.37665807,495.41972882)(263.09665757,495.45083999)(262.7855459,495.45083999)
\curveto(262.39976743,495.45083999)(262.11354469,495.33261756)(261.92687769,495.09617269)
\curveto(261.75265516,494.85972782)(261.66554389,494.56106062)(261.66554389,494.20017108)
\lineto(261.66554389,493.56550328)
\lineto(264.07354821,493.56550328)
\closepath
}
}
{
\newrgbcolor{curcolor}{0 0 0}
\pscustom[linestyle=none,fillstyle=solid,fillcolor=curcolor]
{
\newpath
\moveto(275.01221808,493.56550328)
\lineto(275.01221808,483.37348501)
\lineto(272.88421426,483.37348501)
\lineto(272.51088026,484.68015402)
\lineto(272.36154666,484.68015402)
\curveto(272.03799053,484.15748641)(271.58998972,483.77793018)(271.01754425,483.54148531)
\curveto(270.45754325,483.30504044)(269.86020884,483.18681801)(269.22554104,483.18681801)
\curveto(268.13042797,483.18681801)(267.25309306,483.47926298)(266.59353632,484.06415291)
\curveto(265.93397958,484.66148732)(265.60420121,485.61348902)(265.60420121,486.92015803)
\lineto(265.60420121,493.56550328)
\lineto(268.38553953,493.56550328)
\lineto(268.38553953,487.61082594)
\curveto(268.38553953,486.8766024)(268.51620643,486.32282363)(268.77754024,485.94948963)
\curveto(269.03887404,485.58860009)(269.45576367,485.40815532)(270.02820915,485.40815532)
\curveto(270.87443288,485.40815532)(271.45310059,485.69437806)(271.76421226,486.26682353)
\curveto(272.07532393,486.85171347)(272.23087976,487.68549274)(272.23087976,488.76816135)
\lineto(272.23087976,493.56550328)
\closepath
}
}
{
\newrgbcolor{curcolor}{0 0 0}
\pscustom[linestyle=none,fillstyle=solid,fillcolor=curcolor]
{
\newpath
\moveto(280.70555325,483.37348501)
\lineto(277.92421493,483.37348501)
\lineto(277.92421493,497.56017711)
\lineto(280.70555325,497.56017711)
\closepath
}
}
{
\newrgbcolor{curcolor}{0 0 0}
\pscustom[linestyle=none,fillstyle=solid,fillcolor=curcolor]
{
\newpath
\moveto(286.39890301,483.37348501)
\lineto(283.61756469,483.37348501)
\lineto(283.61756469,497.56017711)
\lineto(286.39890301,497.56017711)
\closepath
}
}
{
\newrgbcolor{curcolor}{0 0 0}
\pscustom[linestyle=none,fillstyle=solid,fillcolor=curcolor]
{
\newpath
\moveto(297.39359562,487.06949163)
\curveto(297.39359562,485.88726729)(296.96426151,484.94148782)(296.10559331,484.23215321)
\curveto(295.25936957,483.53526308)(294.05225629,483.18681801)(292.48425348,483.18681801)
\curveto(291.07802874,483.18681801)(289.8211376,483.45437404)(288.71358006,483.98948611)
\lineto(288.71358006,486.62149083)
\curveto(289.34824786,486.34771256)(290.00158237,486.09260099)(290.67358357,485.85615613)
\curveto(291.35802924,485.63215572)(292.03625268,485.52015552)(292.70825388,485.52015552)
\curveto(293.40514402,485.52015552)(293.89670046,485.65082242)(294.18292319,485.91215623)
\curveto(294.4815904,486.18593449)(294.630924,486.52815733)(294.630924,486.93882473)
\curveto(294.630924,487.27482534)(294.51270156,487.56104807)(294.27625669,487.79749294)
\curveto(294.05225629,488.03393781)(293.74736686,488.25171598)(293.36158839,488.45082744)
\curveto(292.97580992,488.66238338)(292.53403135,488.88638378)(292.03625268,489.12282865)
\curveto(291.72514101,489.27216225)(291.38914041,489.44638478)(291.02825087,489.64549625)
\curveto(290.66736134,489.85705219)(290.31891627,490.11216375)(289.98291567,490.41083096)
\curveto(289.65935953,490.72194263)(289.39180349,491.09527663)(289.18024756,491.53083296)
\curveto(288.96869163,491.9663893)(288.86291366,492.4890569)(288.86291366,493.09883578)
\curveto(288.86291366,494.29350458)(289.26735883,495.22061736)(290.07624917,495.88017409)
\curveto(290.8851395,496.5521753)(291.98647481,496.8881759)(293.38025509,496.8881759)
\curveto(294.07714523,496.8881759)(294.73670196,496.80728687)(295.3589253,496.6455088)
\curveto(295.98114864,496.48373073)(296.64070538,496.2535081)(297.33759552,495.95484089)
\lineto(296.42292721,493.75217028)
\curveto(295.81314834,494.00105961)(295.2655918,494.19394885)(294.7802576,494.33083798)
\curveto(294.2949234,494.46772712)(293.79714472,494.53617169)(293.28692159,494.53617169)
\curveto(292.75180952,494.53617169)(292.34114211,494.41172702)(292.05491938,494.16283768)
\curveto(291.76869664,493.91394835)(291.62558528,493.59039221)(291.62558528,493.19216928)
\curveto(291.62558528,492.71927954)(291.83714121,492.34594554)(292.26025308,492.07216727)
\curveto(292.68336495,491.798389)(293.31181052,491.4623884)(294.14558979,491.06416546)
\curveto(294.83003547,490.74060933)(295.40870317,490.40460872)(295.88159291,490.05616365)
\curveto(296.36692711,489.70771859)(296.74026111,489.29705118)(297.00159491,488.82416145)
\curveto(297.26292872,488.35127171)(297.39359562,487.76638177)(297.39359562,487.06949163)
\closepath
}
}
{
\newrgbcolor{curcolor}{0 0 0}
\pscustom[linestyle=none,fillstyle=solid,fillcolor=curcolor]
{
\newpath
\moveto(303.740284,483.18681801)
\curveto(302.22205906,483.18681801)(301.04605695,483.60370764)(300.21227768,484.43748692)
\curveto(299.39094287,485.27126619)(298.98027547,486.5966019)(298.98027547,488.41349404)
\curveto(298.98027547,489.65794072)(299.19183141,490.67216476)(299.61494328,491.45616616)
\curveto(300.03805514,492.24016757)(300.62294508,492.81883527)(301.36961309,493.19216928)
\curveto(302.12872556,493.56550328)(302.99983823,493.75217028)(303.98295111,493.75217028)
\curveto(304.67984124,493.75217028)(305.28339788,493.68372571)(305.79362102,493.54683658)
\curveto(306.31628862,493.40994744)(306.77051166,493.24816938)(307.15629013,493.06150238)
\lineto(306.33495532,490.91483186)
\curveto(305.89939898,491.08905439)(305.48873158,491.23216576)(305.10295311,491.34416596)
\curveto(304.72961911,491.45616616)(304.35628511,491.51216626)(303.98295111,491.51216626)
\curveto(302.53939296,491.51216626)(301.81761389,490.48549776)(301.81761389,488.43216074)
\curveto(301.81761389,487.41171447)(302.00428089,486.65882423)(302.37761489,486.17349003)
\curveto(302.76339336,485.68815582)(303.29850543,485.44548872)(303.98295111,485.44548872)
\curveto(304.56784104,485.44548872)(305.08428641,485.52015552)(305.53228722,485.66948912)
\curveto(305.98028802,485.83126719)(306.41584436,486.04904536)(306.83895622,486.32282363)
\lineto(306.83895622,483.95215271)
\curveto(306.41584436,483.67837444)(305.96784355,483.48548521)(305.49495382,483.37348501)
\curveto(305.03450855,483.24904034)(304.44961861,483.18681801)(303.740284,483.18681801)
\closepath
}
}
{
\newrgbcolor{curcolor}{0 0 0}
\pscustom[linestyle=none,fillstyle=solid,fillcolor=curcolor]
{
\newpath
\moveto(314.88430459,493.75217028)
\curveto(315.02119373,493.75217028)(315.1829718,493.74594805)(315.3696388,493.73350358)
\curveto(315.5563058,493.72105911)(315.7056394,493.70239241)(315.8176396,493.67750348)
\lineto(315.6123059,491.06416546)
\curveto(315.51275016,491.08905439)(315.38208326,491.10772109)(315.2203052,491.12016556)
\curveto(315.05852713,491.1450545)(314.91541576,491.15749896)(314.79097109,491.15749896)
\curveto(314.31808136,491.15749896)(313.86385832,491.07038769)(313.42830198,490.89616516)
\curveto(312.99274565,490.73438709)(312.63807835,490.46683106)(312.36430008,490.09349705)
\curveto(312.10296627,489.72016305)(311.97229937,489.20993992)(311.97229937,488.56282764)
\lineto(311.97229937,483.37348501)
\lineto(309.19096105,483.37348501)
\lineto(309.19096105,493.56550328)
\lineto(311.30029817,493.56550328)
\lineto(311.71096557,491.84816687)
\lineto(311.84163247,491.84816687)
\curveto(312.14029968,492.37083447)(312.55096708,492.81883527)(313.07363468,493.19216928)
\curveto(313.59630229,493.56550328)(314.19985892,493.75217028)(314.88430459,493.75217028)
\closepath
}
}
{
\newrgbcolor{curcolor}{0 0 0}
\pscustom[linestyle=none,fillstyle=solid,fillcolor=curcolor]
{
\newpath
\moveto(321.49231354,493.75217028)
\curveto(322.89853829,493.75217028)(324.01231806,493.34772511)(324.83365287,492.53883477)
\curveto(325.65498767,491.7423889)(326.06565508,490.60372019)(326.06565508,489.12282865)
\lineto(326.06565508,487.77882624)
\lineto(319.49497663,487.77882624)
\curveto(319.51986556,486.99482483)(319.7500882,486.37882373)(320.18564453,485.93082293)
\curveto(320.63364534,485.48282212)(321.24964644,485.25882172)(322.03364785,485.25882172)
\curveto(322.68076012,485.25882172)(323.27187229,485.32104406)(323.80698436,485.44548872)
\curveto(324.3545409,485.58237786)(324.9145419,485.78771156)(325.48698737,486.06148983)
\lineto(325.48698737,483.91481931)
\curveto(324.97676423,483.66592998)(324.4478744,483.48548521)(323.90031786,483.37348501)
\curveto(323.35276132,483.24904034)(322.68698235,483.18681801)(321.90298095,483.18681801)
\curveto(320.88253467,483.18681801)(319.98031083,483.37348501)(319.19630943,483.74681901)
\curveto(318.41230802,484.13259748)(317.79630692,484.70504295)(317.34830612,485.46415542)
\curveto(316.90030531,486.23571236)(316.67630491,487.212603)(316.67630491,488.39482734)
\curveto(316.67630491,489.57705168)(316.87541638,490.56638679)(317.27363931,491.36283266)
\curveto(317.68430672,492.15927854)(318.25052995,492.75661294)(318.97230903,493.15483588)
\curveto(319.6940881,493.55305881)(320.5340896,493.75217028)(321.49231354,493.75217028)
\closepath
\moveto(321.51098024,491.77350007)
\curveto(320.96342371,491.77350007)(320.5154229,491.59927753)(320.16697783,491.25083246)
\curveto(319.81853277,490.90238739)(319.61319906,490.36105309)(319.55097673,489.62682955)
\lineto(323.45231706,489.62682955)
\curveto(323.43987259,490.23660842)(323.27187229,490.74683156)(322.94831615,491.15749896)
\curveto(322.63720449,491.56816636)(322.15809252,491.77350007)(321.51098024,491.77350007)
\closepath
}
}
{
\newrgbcolor{curcolor}{0 0 0}
\pscustom[linestyle=none,fillstyle=solid,fillcolor=curcolor]
{
\newpath
\moveto(332.52432634,493.75217028)
\curveto(333.93055109,493.75217028)(335.04433086,493.34772511)(335.86566567,492.53883477)
\curveto(336.68700047,491.7423889)(337.09766788,490.60372019)(337.09766788,489.12282865)
\lineto(337.09766788,487.77882624)
\lineto(330.52698943,487.77882624)
\curveto(330.55187836,486.99482483)(330.782101,486.37882373)(331.21765734,485.93082293)
\curveto(331.66565814,485.48282212)(332.28165924,485.25882172)(333.06566065,485.25882172)
\curveto(333.71277292,485.25882172)(334.30388509,485.32104406)(334.83899716,485.44548872)
\curveto(335.3865537,485.58237786)(335.9465547,485.78771156)(336.51900017,486.06148983)
\lineto(336.51900017,483.91481931)
\curveto(336.00877703,483.66592998)(335.4798872,483.48548521)(334.93233066,483.37348501)
\curveto(334.38477412,483.24904034)(333.71899515,483.18681801)(332.93499375,483.18681801)
\curveto(331.91454747,483.18681801)(331.01232363,483.37348501)(330.22832223,483.74681901)
\curveto(329.44432082,484.13259748)(328.82831972,484.70504295)(328.38031892,485.46415542)
\curveto(327.93231811,486.23571236)(327.70831771,487.212603)(327.70831771,488.39482734)
\curveto(327.70831771,489.57705168)(327.90742918,490.56638679)(328.30565212,491.36283266)
\curveto(328.71631952,492.15927854)(329.28254276,492.75661294)(330.00432183,493.15483588)
\curveto(330.7261009,493.55305881)(331.5661024,493.75217028)(332.52432634,493.75217028)
\closepath
\moveto(332.54299304,491.77350007)
\curveto(331.99543651,491.77350007)(331.5474357,491.59927753)(331.19899064,491.25083246)
\curveto(330.85054557,490.90238739)(330.64521186,490.36105309)(330.58298953,489.62682955)
\lineto(334.48432986,489.62682955)
\curveto(334.47188539,490.23660842)(334.30388509,490.74683156)(333.98032895,491.15749896)
\curveto(333.66921729,491.56816636)(333.19010532,491.77350007)(332.54299304,491.77350007)
\closepath
}
}
{
\newrgbcolor{curcolor}{0 0 0}
\pscustom[linestyle=none,fillstyle=solid,fillcolor=curcolor]
{
\newpath
\moveto(345.14300866,493.75217028)
\curveto(346.23812173,493.75217028)(347.11545664,493.45350308)(347.77501337,492.85616867)
\curveto(348.43457011,492.27127874)(348.76434848,491.32549926)(348.76434848,490.01883025)
\lineto(348.76434848,483.37348501)
\lineto(345.98301016,483.37348501)
\lineto(345.98301016,489.32816235)
\curveto(345.98301016,490.06238589)(345.85234326,490.60994242)(345.59100946,490.97083196)
\curveto(345.32967566,491.34416596)(344.91278602,491.53083296)(344.34034055,491.53083296)
\curveto(343.49411681,491.53083296)(342.91544911,491.238388)(342.60433744,490.65349806)
\curveto(342.29322577,490.08105259)(342.13766994,489.25349555)(342.13766994,488.17082694)
\lineto(342.13766994,483.37348501)
\lineto(339.35633162,483.37348501)
\lineto(339.35633162,493.56550328)
\lineto(341.48433543,493.56550328)
\lineto(341.85766943,492.25883427)
\lineto(342.00700303,492.25883427)
\curveto(342.33055917,492.78150187)(342.77233774,493.16105811)(343.33233874,493.39750298)
\curveto(343.90478421,493.63394785)(344.50834085,493.75217028)(345.14300866,493.75217028)
\closepath
}
}
{
\newrgbcolor{curcolor}{0 0 0}
\pscustom[linestyle=none,fillstyle=solid,fillcolor=curcolor]
{
\newpath
\moveto(351.22834596,484.68015402)
\curveto(351.22834596,485.25259949)(351.38390179,485.65082242)(351.69501346,485.87482283)
\curveto(352.00612513,486.11126769)(352.38568136,486.22949013)(352.83368217,486.22949013)
\curveto(353.2692385,486.22949013)(353.64257251,486.11126769)(353.95368418,485.87482283)
\curveto(354.26479584,485.65082242)(354.42035168,485.25259949)(354.42035168,484.68015402)
\curveto(354.42035168,484.13259748)(354.26479584,483.73437454)(353.95368418,483.48548521)
\curveto(353.64257251,483.24904034)(353.2692385,483.13081791)(352.83368217,483.13081791)
\curveto(352.38568136,483.13081791)(352.00612513,483.24904034)(351.69501346,483.48548521)
\curveto(351.38390179,483.73437454)(351.22834596,484.13259748)(351.22834596,484.68015402)
\closepath
}
}
{
\newrgbcolor{curcolor}{0 0 0}
\pscustom[linestyle=none,fillstyle=solid,fillcolor=curcolor]
{
\newpath
\moveto(356.82835372,496.197508)
\curveto(356.82835372,496.7201756)(356.97146508,497.0748429)(357.25768782,497.2615099)
\curveto(357.55635502,497.46062137)(357.91724456,497.56017711)(358.34035643,497.56017711)
\curveto(358.75102383,497.56017711)(359.10569113,497.46062137)(359.40435833,497.2615099)
\curveto(359.70302554,497.0748429)(359.85235914,496.7201756)(359.85235914,496.197508)
\curveto(359.85235914,495.68728486)(359.70302554,495.33261756)(359.40435833,495.13350609)
\curveto(359.10569113,494.93439462)(358.75102383,494.83483889)(358.34035643,494.83483889)
\curveto(357.91724456,494.83483889)(357.55635502,494.93439462)(357.25768782,495.13350609)
\curveto(356.97146508,495.33261756)(356.82835372,495.68728486)(356.82835372,496.197508)
\closepath
\moveto(356.11901911,478.89347698)
\curveto(355.79546298,478.89347698)(355.46568461,478.91836591)(355.129684,478.96814378)
\curveto(354.7936834,479.00547718)(354.5136829,479.05525505)(354.2896825,479.11747738)
\lineto(354.2896825,481.30148129)
\curveto(354.5136829,481.23925896)(354.72523883,481.19570333)(354.9243503,481.17081439)
\curveto(355.12346177,481.14592546)(355.34746217,481.13348099)(355.59635151,481.13348099)
\curveto(355.96968551,481.13348099)(356.28701941,481.23925896)(356.54835321,481.4508149)
\curveto(356.80968702,481.66237083)(356.94035392,482.07303823)(356.94035392,482.6828171)
\lineto(356.94035392,493.56550328)
\lineto(359.72169224,493.56550328)
\lineto(359.72169224,482.2721497)
\curveto(359.72169224,481.64992636)(359.6034698,481.08370313)(359.36702493,480.57347999)
\curveto(359.13058006,480.06325685)(358.7448016,479.65881168)(358.20968953,479.36014448)
\curveto(357.68702192,479.04903281)(356.99013178,478.89347698)(356.11901911,478.89347698)
\closepath
}
}
{
\newrgbcolor{curcolor}{0 0 0}
\pscustom[linestyle=none,fillstyle=solid,fillcolor=curcolor]
{
\newpath
\moveto(369.74571834,486.39749043)
\curveto(369.74571834,485.36459969)(369.37860657,484.56815382)(368.64438303,484.00815281)
\curveto(367.92260396,483.46059628)(366.83993535,483.18681801)(365.39637721,483.18681801)
\curveto(364.6870426,483.18681801)(364.07726373,483.23659587)(363.5670406,483.33615161)
\curveto(363.05681746,483.42326288)(362.54659432,483.57259648)(362.03637119,483.78415241)
\lineto(362.03637119,486.08015653)
\curveto(362.58392772,485.83126719)(363.17503989,485.62593349)(363.8097077,485.46415542)
\curveto(364.4443755,485.30237736)(365.00437651,485.22148832)(365.48971071,485.22148832)
\curveto(366.02482278,485.22148832)(366.41060125,485.30237736)(366.64704612,485.46415542)
\curveto(366.88349099,485.62593349)(367.00171342,485.83748943)(367.00171342,486.09882323)
\curveto(367.00171342,486.27304576)(366.95193555,486.4286016)(366.85237982,486.56549073)
\curveto(366.76526855,486.70237986)(366.56615708,486.8579357)(366.25504542,487.03215823)
\curveto(365.94393375,487.20638077)(365.45859954,487.43038117)(364.79904281,487.70415944)
\curveto(364.15193053,487.97793771)(363.6230407,488.24549374)(363.21237329,488.50682754)
\curveto(362.81415036,488.78060581)(362.51548316,489.10416195)(362.31637169,489.47749595)
\curveto(362.11726022,489.86327442)(362.01770449,490.34238639)(362.01770449,490.91483186)
\curveto(362.01770449,491.86061133)(362.38481626,492.56994594)(363.11903979,493.04283567)
\curveto(363.85326333,493.51572541)(364.83015397,493.75217028)(366.04971171,493.75217028)
\curveto(366.68437952,493.75217028)(367.28793616,493.68994795)(367.86038163,493.56550328)
\curveto(368.4328271,493.44105861)(369.02393927,493.23572491)(369.63371814,492.94950217)
\lineto(368.79371663,490.95216526)
\curveto(368.29593796,491.1637212)(367.82304823,491.33794373)(367.37504742,491.47483286)
\curveto(366.92704662,491.62416647)(366.47282358,491.69883327)(366.01237831,491.69883327)
\curveto(365.19104351,491.69883327)(364.78037611,491.47483286)(364.78037611,491.02683206)
\curveto(364.78037611,490.86505399)(364.83015397,490.71572039)(364.92970971,490.57883126)
\curveto(365.04170991,490.45438659)(365.24704361,490.31749746)(365.54571081,490.16816386)
\curveto(365.85682248,490.01883025)(366.31104552,489.81971879)(366.90837992,489.57082945)
\curveto(367.49326986,489.33438458)(367.99727076,489.08549525)(368.42038263,488.82416145)
\curveto(368.8434945,488.57527211)(369.16705064,488.25793821)(369.39105104,487.87215974)
\curveto(369.6274959,487.48638127)(369.74571834,486.99482483)(369.74571834,486.39749043)
\closepath
}
}
{
\newrgbcolor{curcolor}{0 0 0}
\pscustom[linestyle=none,fillstyle=solid,fillcolor=curcolor]
{
\newpath
\moveto(262.08498376,472.58698927)
\curveto(263.3418749,472.58698927)(264.32498777,472.0892106)(265.03432238,471.09365326)
\lineto(265.10898918,471.09365326)
\lineto(265.33298958,472.40032227)
\lineto(267.6849938,472.40032227)
\lineto(267.6849938,462.1896373)
\curveto(267.6849938,460.73363469)(267.25565969,459.62607715)(266.39699149,458.86696468)
\curveto(265.53832328,458.10785221)(264.26898767,457.72829597)(262.58898466,457.72829597)
\curveto(261.86720559,457.72829597)(261.19520439,457.77185161)(260.57298105,457.85896287)
\curveto(259.96320218,457.94607414)(259.36586777,458.10162998)(258.78097784,458.32563038)
\lineto(258.78097784,460.54696769)
\curveto(260.02542451,460.02430009)(261.35076022,459.76296629)(262.75698496,459.76296629)
\curveto(264.18809864,459.76296629)(264.90365548,460.53452323)(264.90365548,462.0776371)
\lineto(264.90365548,462.2829708)
\curveto(264.90365548,462.48208227)(264.90987771,462.69363821)(264.92232218,462.91763861)
\curveto(264.93476664,463.15408348)(264.95343334,463.35941718)(264.97832228,463.53363971)
\lineto(264.90365548,463.53363971)
\curveto(264.55521041,462.99852764)(264.13832077,462.61274917)(263.65298657,462.3763043)
\curveto(263.16765237,462.13985944)(262.62009583,462.021637)(262.01031696,462.021637)
\curveto(260.80320368,462.021637)(259.85742421,462.48208227)(259.17297854,463.40297281)
\curveto(258.50097733,464.33630782)(258.16497673,465.63053236)(258.16497673,467.28564644)
\curveto(258.16497673,468.95320498)(258.5134218,470.25365176)(259.21031194,471.18698676)
\curveto(259.90720208,472.12032177)(260.86542602,472.58698927)(262.08498376,472.58698927)
\closepath
\moveto(262.96231866,470.32831856)
\curveto(261.65564966,470.32831856)(261.00231515,469.30165005)(261.00231515,467.24831304)
\curveto(261.00231515,465.21986496)(261.66809412,464.20564092)(262.99965206,464.20564092)
\curveto(263.70898667,464.20564092)(264.23165427,464.40475239)(264.56765488,464.80297532)
\curveto(264.91609994,465.21364272)(265.09032248,465.92297733)(265.09032248,466.93097914)
\lineto(265.09032248,467.26697974)
\curveto(265.09032248,468.36209281)(264.92232218,469.14609422)(264.58632158,469.61898395)
\curveto(264.25032097,470.09187369)(263.70898667,470.32831856)(262.96231866,470.32831856)
\closepath
}
}
{
\newrgbcolor{curcolor}{0 0 0}
\pscustom[linestyle=none,fillstyle=solid,fillcolor=curcolor]
{
\newpath
\moveto(273.90099408,472.58698927)
\curveto(275.15788523,472.58698927)(276.1409981,472.0892106)(276.8503327,471.09365326)
\lineto(276.92499951,471.09365326)
\lineto(277.14899991,472.40032227)
\lineto(279.50100412,472.40032227)
\lineto(279.50100412,462.1896373)
\curveto(279.50100412,460.73363469)(279.07167002,459.62607715)(278.21300181,458.86696468)
\curveto(277.35433361,458.10785221)(276.084998,457.72829597)(274.40499499,457.72829597)
\curveto(273.68321592,457.72829597)(273.01121471,457.77185161)(272.38899137,457.85896287)
\curveto(271.7792125,457.94607414)(271.1818781,458.10162998)(270.59698816,458.32563038)
\lineto(270.59698816,460.54696769)
\curveto(271.84143484,460.02430009)(273.16677055,459.76296629)(274.57299529,459.76296629)
\curveto(276.00410897,459.76296629)(276.7196658,460.53452323)(276.7196658,462.0776371)
\lineto(276.7196658,462.2829708)
\curveto(276.7196658,462.48208227)(276.72588804,462.69363821)(276.7383325,462.91763861)
\curveto(276.75077697,463.15408348)(276.76944367,463.35941718)(276.7943326,463.53363971)
\lineto(276.7196658,463.53363971)
\curveto(276.37122073,462.99852764)(275.9543311,462.61274917)(275.4689969,462.3763043)
\curveto(274.98366269,462.13985944)(274.43610615,462.021637)(273.82632728,462.021637)
\curveto(272.61921401,462.021637)(271.67343454,462.48208227)(270.98898886,463.40297281)
\curveto(270.31698766,464.33630782)(269.98098706,465.63053236)(269.98098706,467.28564644)
\curveto(269.98098706,468.95320498)(270.32943213,470.25365176)(271.02632226,471.18698676)
\curveto(271.7232124,472.12032177)(272.68143634,472.58698927)(273.90099408,472.58698927)
\closepath
\moveto(274.77832899,470.32831856)
\curveto(273.47165998,470.32831856)(272.81832548,469.30165005)(272.81832548,467.24831304)
\curveto(272.81832548,465.21986496)(273.48410445,464.20564092)(274.81566239,464.20564092)
\curveto(275.524997,464.20564092)(276.0476646,464.40475239)(276.3836652,464.80297532)
\curveto(276.73211027,465.21364272)(276.90633281,465.92297733)(276.90633281,466.93097914)
\lineto(276.90633281,467.26697974)
\curveto(276.90633281,468.36209281)(276.7383325,469.14609422)(276.4023319,469.61898395)
\curveto(276.0663313,470.09187369)(275.524997,470.32831856)(274.77832899,470.32831856)
\closepath
}
}
{
\newrgbcolor{curcolor}{0 0 0}
\pscustom[linestyle=none,fillstyle=solid,fillcolor=curcolor]
{
\newpath
\moveto(285.19433681,476.3949961)
\lineto(285.19433681,473.09099018)
\curveto(285.19433681,472.70521171)(285.18189234,472.32565547)(285.15700341,471.95232147)
\curveto(285.13211447,471.57898747)(285.10722554,471.2865425)(285.08233661,471.07498656)
\lineto(285.19433681,471.07498656)
\curveto(285.46811508,471.49809843)(285.83522684,471.85276574)(286.29567211,472.13898847)
\curveto(286.75611738,472.43765567)(287.35345179,472.58698927)(288.08767533,472.58698927)
\curveto(289.23256627,472.58698927)(290.15967904,472.13898847)(290.86901365,471.24298687)
\curveto(291.57834825,470.35942973)(291.93301555,469.05276072)(291.93301555,467.32297984)
\curveto(291.93301555,465.58075449)(291.57212602,464.26164102)(290.85034695,463.36563941)
\curveto(290.12856787,462.46963781)(289.1827884,462.021637)(288.01300853,462.021637)
\curveto(287.26634052,462.021637)(286.67522835,462.1523039)(286.23967201,462.4136377)
\curveto(285.81656014,462.68741597)(285.46811508,462.99230541)(285.19433681,463.32830601)
\lineto(285.00766981,463.32830601)
\lineto(284.5410023,462.208304)
\lineto(282.41299849,462.208304)
\lineto(282.41299849,476.3949961)
\closepath
\moveto(287.19167372,470.36565196)
\curveto(286.46989465,470.36565196)(285.95967151,470.13542932)(285.66100431,469.67498405)
\curveto(285.36233711,469.22698325)(285.20678127,468.54875981)(285.19433681,467.64031374)
\lineto(285.19433681,467.34164654)
\curveto(285.19433681,466.35853367)(285.33744817,465.59942119)(285.62367091,465.06430912)
\curveto(285.92233811,464.54164152)(286.45745018,464.28030772)(287.22900712,464.28030772)
\curveto(287.80145259,464.28030772)(288.25567563,464.54164152)(288.59167623,465.06430912)
\curveto(288.92767683,465.59942119)(289.09567713,466.3647559)(289.09567713,467.36031324)
\curveto(289.09567713,468.35587058)(288.9214546,469.10253858)(288.57300953,469.60031725)
\curveto(288.23700893,470.11054039)(287.77656366,470.36565196)(287.19167372,470.36565196)
\closepath
}
}
{
\newrgbcolor{curcolor}{0 0 0}
\pscustom[linestyle=none,fillstyle=solid,fillcolor=curcolor]
{
\newpath
\moveto(306.54903185,468.89098265)
\curveto(306.54903185,467.50964684)(306.31880922,466.30253356)(305.85836395,465.26964282)
\curveto(305.41036314,464.24919655)(304.71347301,463.45275068)(303.76769353,462.88030521)
\curveto(302.83435853,462.30785974)(301.64591195,462.021637)(300.20235381,462.021637)
\curveto(298.75879567,462.021637)(297.56412686,462.30785974)(296.61834738,462.88030521)
\curveto(295.68501238,463.45275068)(294.98812224,464.25541878)(294.52767697,465.28830952)
\curveto(294.07967617,466.32120026)(293.85567577,467.52831354)(293.85567577,468.90964935)
\curveto(293.85567577,470.29098516)(294.07967617,471.4918762)(294.52767697,472.51232247)
\curveto(294.98812224,473.53276875)(295.68501238,474.32299239)(296.61834738,474.88299339)
\curveto(297.56412686,475.45543886)(298.7650179,475.7416616)(300.22102051,475.7416616)
\curveto(301.66457865,475.7416616)(302.85302523,475.45543886)(303.78636023,474.88299339)
\curveto(304.71969524,474.32299239)(305.41036314,473.52654651)(305.85836395,472.49365577)
\curveto(306.31880922,471.4732095)(306.54903185,470.27231846)(306.54903185,468.89098265)
\closepath
\moveto(296.82368109,468.89098265)
\curveto(296.82368109,467.49720237)(297.09123712,466.39586707)(297.62634919,465.58697673)
\curveto(298.16146126,464.79053085)(299.02012947,464.39230792)(300.20235381,464.39230792)
\curveto(301.40946708,464.39230792)(302.27435752,464.79053085)(302.79702513,465.58697673)
\curveto(303.31969273,466.39586707)(303.58102653,467.49720237)(303.58102653,468.89098265)
\curveto(303.58102653,470.28476293)(303.31969273,471.379876)(302.79702513,472.17632187)
\curveto(302.27435752,472.98521221)(301.41568932,473.38965738)(300.22102051,473.38965738)
\curveto(299.0263517,473.38965738)(298.16146126,472.98521221)(297.62634919,472.17632187)
\curveto(297.09123712,471.379876)(296.82368109,470.28476293)(296.82368109,468.89098265)
\closepath
}
}
{
\newrgbcolor{curcolor}{0 0 0}
\pscustom[linestyle=none,fillstyle=solid,fillcolor=curcolor]
{
\newpath
\moveto(314.87437116,472.58698927)
\curveto(315.96948423,472.58698927)(316.84681914,472.28832207)(317.50637588,471.69098767)
\curveto(318.16593262,471.10609773)(318.49571099,470.16031826)(318.49571099,468.85364925)
\lineto(318.49571099,462.208304)
\lineto(315.71437267,462.208304)
\lineto(315.71437267,468.16298134)
\curveto(315.71437267,468.89720488)(315.58370577,469.44476142)(315.32237196,469.80565096)
\curveto(315.06103816,470.17898496)(314.64414853,470.36565196)(314.07170305,470.36565196)
\curveto(313.22547932,470.36565196)(312.64681161,470.07320699)(312.33569994,469.48831705)
\curveto(312.02458827,468.91587158)(311.86903244,468.08831454)(311.86903244,467.00564594)
\lineto(311.86903244,462.208304)
\lineto(309.08769412,462.208304)
\lineto(309.08769412,472.40032227)
\lineto(311.21569794,472.40032227)
\lineto(311.58903194,471.09365326)
\lineto(311.73836554,471.09365326)
\curveto(312.06192167,471.61632087)(312.50370024,471.9958771)(313.06370125,472.23232197)
\curveto(313.63614672,472.46876684)(314.23970336,472.58698927)(314.87437116,472.58698927)
\closepath
}
}
{
\newrgbcolor{curcolor}{0 0 0}
\pscustom[linestyle=none,fillstyle=solid,fillcolor=curcolor]
{
\newpath
\moveto(326.5597185,462.208304)
\lineto(320.49304096,462.208304)
\lineto(320.49304096,463.81364021)
\lineto(322.11704387,464.56030822)
\lineto(322.11704387,473.18432368)
\lineto(320.49304096,473.93099168)
\lineto(320.49304096,475.53632789)
\lineto(326.5597185,475.53632789)
\lineto(326.5597185,473.93099168)
\lineto(324.93571559,473.18432368)
\lineto(324.93571559,464.56030822)
\lineto(326.5597185,463.81364021)
\closepath
}
}
{
\newrgbcolor{curcolor}{0 0 0}
\pscustom[linestyle=none,fillstyle=solid,fillcolor=curcolor]
{
\newpath
\moveto(334.39973102,472.58698927)
\curveto(335.4948441,472.58698927)(336.372179,472.28832207)(337.03173574,471.69098767)
\curveto(337.69129248,471.10609773)(338.02107085,470.16031826)(338.02107085,468.85364925)
\lineto(338.02107085,462.208304)
\lineto(335.23973253,462.208304)
\lineto(335.23973253,468.16298134)
\curveto(335.23973253,468.89720488)(335.10906563,469.44476142)(334.84773182,469.80565096)
\curveto(334.58639802,470.17898496)(334.16950839,470.36565196)(333.59706292,470.36565196)
\curveto(332.75083918,470.36565196)(332.17217147,470.07320699)(331.8610598,469.48831705)
\curveto(331.54994814,468.91587158)(331.3943923,468.08831454)(331.3943923,467.00564594)
\lineto(331.3943923,462.208304)
\lineto(328.61305398,462.208304)
\lineto(328.61305398,472.40032227)
\lineto(330.7410578,472.40032227)
\lineto(331.1143918,471.09365326)
\lineto(331.2637254,471.09365326)
\curveto(331.58728154,471.61632087)(332.02906011,471.9958771)(332.58906111,472.23232197)
\curveto(333.16150658,472.46876684)(333.76506322,472.58698927)(334.39973102,472.58698927)
\closepath
}
}
{
\newrgbcolor{curcolor}{0 0 0}
\pscustom[linestyle=none,fillstyle=solid,fillcolor=curcolor]
{
\newpath
\moveto(342.27707153,476.3949961)
\curveto(342.68773894,476.3949961)(343.04240624,476.29544037)(343.34107344,476.0963289)
\curveto(343.63974064,475.9096619)(343.78907425,475.55499459)(343.78907425,475.03232699)
\curveto(343.78907425,474.52210385)(343.63974064,474.16743655)(343.34107344,473.96832508)
\curveto(343.04240624,473.76921362)(342.68773894,473.66965788)(342.27707153,473.66965788)
\curveto(341.85395967,473.66965788)(341.49307013,473.76921362)(341.19440293,473.96832508)
\curveto(340.90818019,474.16743655)(340.76506882,474.52210385)(340.76506882,475.03232699)
\curveto(340.76506882,475.55499459)(340.90818019,475.9096619)(341.19440293,476.0963289)
\curveto(341.49307013,476.29544037)(341.85395967,476.3949961)(342.27707153,476.3949961)
\closepath
\moveto(343.65840734,472.40032227)
\lineto(343.65840734,462.208304)
\lineto(340.87706903,462.208304)
\lineto(340.87706903,472.40032227)
\closepath
}
}
{
\newrgbcolor{curcolor}{0 0 0}
\pscustom[linestyle=none,fillstyle=solid,fillcolor=curcolor]
{
\newpath
\moveto(350.8637541,464.24297432)
\curveto(351.17486577,464.24297432)(351.47353297,464.26786325)(351.75975571,464.31764112)
\curveto(352.04597844,464.37986345)(352.33220118,464.46075249)(352.61842391,464.56030822)
\lineto(352.61842391,462.48830451)
\curveto(352.31975671,462.35141537)(351.94642271,462.23941517)(351.4984219,462.1523039)
\curveto(351.06286557,462.06519264)(350.5837536,462.021637)(350.06108599,462.021637)
\curveto(349.45130712,462.021637)(348.90375059,462.12119274)(348.41841638,462.3203042)
\curveto(347.94552665,462.51941567)(347.56597041,462.86163851)(347.27974767,463.34697271)
\curveto(347.00596941,463.83230691)(346.86908027,464.51675259)(346.86908027,465.40030973)
\lineto(346.86908027,470.30965186)
\lineto(345.54374456,470.30965186)
\lineto(345.54374456,471.48565397)
\lineto(347.07441397,472.41898897)
\lineto(347.87708208,474.56565949)
\lineto(349.65041859,474.56565949)
\lineto(349.65041859,472.40032227)
\lineto(352.50642371,472.40032227)
\lineto(352.50642371,470.30965186)
\lineto(349.65041859,470.30965186)
\lineto(349.65041859,465.40030973)
\curveto(349.65041859,465.01453126)(349.76241879,464.72208629)(349.98641919,464.52297482)
\curveto(350.21041959,464.33630782)(350.50286456,464.24297432)(350.8637541,464.24297432)
\closepath
}
}
{
\newrgbcolor{curcolor}{0 0 0}
\pscustom[linestyle=none,fillstyle=solid,fillcolor=curcolor]
{
\newpath
\moveto(354.27976114,463.51497301)
\curveto(354.27976114,464.08741848)(354.43531698,464.48564142)(354.74642865,464.70964182)
\curveto(355.05754031,464.94608669)(355.43709655,465.06430912)(355.88509735,465.06430912)
\curveto(356.32065369,465.06430912)(356.69398769,464.94608669)(357.00509936,464.70964182)
\curveto(357.31621103,464.48564142)(357.47176686,464.08741848)(357.47176686,463.51497301)
\curveto(357.47176686,462.96741648)(357.31621103,462.56919354)(357.00509936,462.3203042)
\curveto(356.69398769,462.08385934)(356.32065369,461.9656369)(355.88509735,461.9656369)
\curveto(355.43709655,461.9656369)(355.05754031,462.08385934)(354.74642865,462.3203042)
\curveto(354.43531698,462.56919354)(354.27976114,462.96741648)(354.27976114,463.51497301)
\closepath
}
}
{
\newrgbcolor{curcolor}{0 0 0}
\pscustom[linestyle=none,fillstyle=solid,fillcolor=curcolor]
{
\newpath
\moveto(359.8797689,475.03232699)
\curveto(359.8797689,475.55499459)(360.02288027,475.9096619)(360.309103,476.0963289)
\curveto(360.60777021,476.29544037)(360.96865974,476.3949961)(361.39177161,476.3949961)
\curveto(361.80243901,476.3949961)(362.15710632,476.29544037)(362.45577352,476.0963289)
\curveto(362.75444072,475.9096619)(362.90377432,475.55499459)(362.90377432,475.03232699)
\curveto(362.90377432,474.52210385)(362.75444072,474.16743655)(362.45577352,473.96832508)
\curveto(362.15710632,473.76921362)(361.80243901,473.66965788)(361.39177161,473.66965788)
\curveto(360.96865974,473.66965788)(360.60777021,473.76921362)(360.309103,473.96832508)
\curveto(360.02288027,474.16743655)(359.8797689,474.52210385)(359.8797689,475.03232699)
\closepath
\moveto(359.1704343,457.72829597)
\curveto(358.84687816,457.72829597)(358.51709979,457.75318491)(358.18109919,457.80296277)
\curveto(357.84509859,457.84029617)(357.56509809,457.89007404)(357.34109768,457.95229637)
\lineto(357.34109768,460.13630029)
\curveto(357.56509809,460.07407796)(357.77665402,460.03052232)(357.97576549,460.00563339)
\curveto(358.17487696,459.98074445)(358.39887736,459.96829999)(358.64776669,459.96829999)
\curveto(359.0211007,459.96829999)(359.3384346,460.07407796)(359.5997684,460.28563389)
\curveto(359.8611022,460.49718982)(359.9917691,460.90785723)(359.9917691,461.5176361)
\lineto(359.9917691,472.40032227)
\lineto(362.77310742,472.40032227)
\lineto(362.77310742,461.1069687)
\curveto(362.77310742,460.48474536)(362.65488499,459.91852212)(362.41844012,459.40829898)
\curveto(362.18199525,458.89807585)(361.79621678,458.49363068)(361.26110471,458.19496348)
\curveto(360.73843711,457.88385181)(360.04154697,457.72829597)(359.1704343,457.72829597)
\closepath
}
}
{
\newrgbcolor{curcolor}{0 0 0}
\pscustom[linestyle=none,fillstyle=solid,fillcolor=curcolor]
{
\newpath
\moveto(372.79713352,465.23230942)
\curveto(372.79713352,464.19941868)(372.43002176,463.40297281)(371.69579822,462.84297181)
\curveto(370.97401915,462.29541527)(369.89135054,462.021637)(368.44779239,462.021637)
\curveto(367.73845779,462.021637)(367.12867892,462.07141487)(366.61845578,462.1709706)
\curveto(366.10823265,462.25808187)(365.59800951,462.40741547)(365.08778637,462.61897141)
\lineto(365.08778637,464.91497552)
\curveto(365.63534291,464.66608619)(366.22645508,464.46075249)(366.86112288,464.29897442)
\curveto(367.49579069,464.13719635)(368.05579169,464.05630732)(368.5411259,464.05630732)
\curveto(369.07623797,464.05630732)(369.46201644,464.13719635)(369.6984613,464.29897442)
\curveto(369.93490617,464.46075249)(370.05312861,464.67230842)(370.05312861,464.93364222)
\curveto(370.05312861,465.10786476)(370.00335074,465.26342059)(369.903795,465.40030973)
\curveto(369.81668374,465.53719886)(369.61757227,465.69275469)(369.3064606,465.86697723)
\curveto(368.99534893,466.04119976)(368.51001473,466.26520016)(367.85045799,466.53897843)
\curveto(367.20334572,466.8127567)(366.67445588,467.08031274)(366.26378848,467.34164654)
\curveto(365.86556554,467.61542481)(365.56689834,467.93898094)(365.36778687,468.31231495)
\curveto(365.16867541,468.69809341)(365.06911967,469.17720538)(365.06911967,469.74965086)
\curveto(365.06911967,470.69543033)(365.43623144,471.40476493)(366.17045498,471.87765467)
\curveto(366.90467852,472.35054441)(367.88156916,472.58698927)(369.1011269,472.58698927)
\curveto(369.7357947,472.58698927)(370.33935134,472.52476694)(370.91179681,472.40032227)
\curveto(371.48424228,472.27587761)(372.07535445,472.0705439)(372.68513332,471.78432117)
\lineto(371.84513182,469.78698426)
\curveto(371.34735315,469.99854019)(370.87446341,470.17276272)(370.42646261,470.30965186)
\curveto(369.97846181,470.45898546)(369.52423877,470.53365226)(369.0637935,470.53365226)
\curveto(368.24245869,470.53365226)(367.83179129,470.30965186)(367.83179129,469.86165106)
\curveto(367.83179129,469.69987299)(367.88156916,469.55053939)(367.98112489,469.41365025)
\curveto(368.09312509,469.28920559)(368.29845879,469.15231645)(368.597126,469.00298285)
\curveto(368.90823766,468.85364925)(369.3624607,468.65453778)(369.95979511,468.40564845)
\curveto(370.54468504,468.16920358)(371.04868595,467.92031424)(371.47179782,467.65898044)
\curveto(371.89490969,467.41009111)(372.21846582,467.0927572)(372.44246622,466.70697873)
\curveto(372.67891109,466.32120026)(372.79713352,465.82964383)(372.79713352,465.23230942)
\closepath
}
}
{
\newrgbcolor{curcolor}{0 0 0}
\pscustom[linestyle=none,fillstyle=solid,fillcolor=curcolor]
{
\newpath
\moveto(260.3584958,454.27110265)
\curveto(260.7691632,454.27110265)(261.12383051,454.17154691)(261.42249771,453.97243545)
\curveto(261.72116491,453.78576845)(261.87049851,453.43110114)(261.87049851,452.90843354)
\curveto(261.87049851,452.3982104)(261.72116491,452.0435431)(261.42249771,451.84443163)
\curveto(261.12383051,451.64532016)(260.7691632,451.54576443)(260.3584958,451.54576443)
\curveto(259.93538393,451.54576443)(259.57449439,451.64532016)(259.27582719,451.84443163)
\curveto(258.98960446,452.0435431)(258.84649309,452.3982104)(258.84649309,452.90843354)
\curveto(258.84649309,453.43110114)(258.98960446,453.78576845)(259.27582719,453.97243545)
\curveto(259.57449439,454.17154691)(259.93538393,454.27110265)(260.3584958,454.27110265)
\closepath
\moveto(261.73983161,450.27642882)
\lineto(261.73983161,440.08441055)
\lineto(258.95849329,440.08441055)
\lineto(258.95849329,450.27642882)
\closepath
}
}
{
\newrgbcolor{curcolor}{0 0 0}
\pscustom[linestyle=none,fillstyle=solid,fillcolor=curcolor]
{
\newpath
\moveto(270.43851962,450.46309582)
\curveto(271.53363269,450.46309582)(272.4109676,450.16442862)(273.07052434,449.56709422)
\curveto(273.73008108,448.98220428)(274.05985944,448.03642481)(274.05985944,446.7297558)
\lineto(274.05985944,440.08441055)
\lineto(271.27852113,440.08441055)
\lineto(271.27852113,446.03908789)
\curveto(271.27852113,446.77331143)(271.14785422,447.32086797)(270.88652042,447.6817575)
\curveto(270.62518662,448.05509151)(270.20829699,448.24175851)(269.63585151,448.24175851)
\curveto(268.78962778,448.24175851)(268.21096007,447.94931354)(267.8998484,447.3644236)
\curveto(267.58873673,446.79197813)(267.4331809,445.96442109)(267.4331809,444.88175248)
\lineto(267.4331809,440.08441055)
\lineto(264.65184258,440.08441055)
\lineto(264.65184258,450.27642882)
\lineto(266.77984639,450.27642882)
\lineto(267.1531804,448.96975981)
\lineto(267.302514,448.96975981)
\curveto(267.62607013,449.49242742)(268.0678487,449.87198365)(268.62784971,450.10842852)
\curveto(269.20029518,450.34487339)(269.80385182,450.46309582)(270.43851962,450.46309582)
\closepath
}
}
{
\newrgbcolor{curcolor}{0 0 0}
\pscustom[linestyle=none,fillstyle=solid,fillcolor=curcolor]
{
\newpath
\moveto(278.3158587,454.27110265)
\curveto(278.72652611,454.27110265)(279.08119341,454.17154691)(279.37986061,453.97243545)
\curveto(279.67852781,453.78576845)(279.82786141,453.43110114)(279.82786141,452.90843354)
\curveto(279.82786141,452.3982104)(279.67852781,452.0435431)(279.37986061,451.84443163)
\curveto(279.08119341,451.64532016)(278.72652611,451.54576443)(278.3158587,451.54576443)
\curveto(277.89274683,451.54576443)(277.5318573,451.64532016)(277.2331901,451.84443163)
\curveto(276.94696736,452.0435431)(276.80385599,452.3982104)(276.80385599,452.90843354)
\curveto(276.80385599,453.43110114)(276.94696736,453.78576845)(277.2331901,453.97243545)
\curveto(277.5318573,454.17154691)(277.89274683,454.27110265)(278.3158587,454.27110265)
\closepath
\moveto(279.69719451,450.27642882)
\lineto(279.69719451,440.08441055)
\lineto(276.91585619,440.08441055)
\lineto(276.91585619,450.27642882)
\closepath
}
}
{
\newrgbcolor{curcolor}{0 0 0}
\pscustom[linestyle=none,fillstyle=solid,fillcolor=curcolor]
{
\newpath
\moveto(286.90254699,442.11908087)
\curveto(287.21365866,442.11908087)(287.51232586,442.1439698)(287.7985486,442.19374767)
\curveto(288.08477133,442.25597)(288.37099407,442.33685903)(288.6572168,442.43641477)
\lineto(288.6572168,440.36441105)
\curveto(288.3585496,440.22752192)(287.9852156,440.11552172)(287.53721479,440.02841045)
\curveto(287.10165846,439.94129918)(286.62254649,439.89774355)(286.09987888,439.89774355)
\curveto(285.49010001,439.89774355)(284.94254348,439.99729928)(284.45720927,440.19641075)
\curveto(283.98431954,440.39552222)(283.6047633,440.73774506)(283.31854056,441.22307926)
\curveto(283.0447623,441.70841346)(282.90787316,442.39285913)(282.90787316,443.27641627)
\lineto(282.90787316,448.18575841)
\lineto(281.58253745,448.18575841)
\lineto(281.58253745,449.36176052)
\lineto(283.11320686,450.29509552)
\lineto(283.91587497,452.44176604)
\lineto(285.68921148,452.44176604)
\lineto(285.68921148,450.27642882)
\lineto(288.5452166,450.27642882)
\lineto(288.5452166,448.18575841)
\lineto(285.68921148,448.18575841)
\lineto(285.68921148,443.27641627)
\curveto(285.68921148,442.8906378)(285.80121168,442.59819284)(286.02521208,442.39908137)
\curveto(286.24921248,442.21241437)(286.54165745,442.11908087)(286.90254699,442.11908087)
\closepath
}
}
{
\newrgbcolor{curcolor}{0 0 0}
\pscustom[linestyle=none,fillstyle=solid,fillcolor=curcolor]
{
\newpath
\moveto(290.31855212,441.39107956)
\curveto(290.31855212,441.96352503)(290.47410796,442.36174797)(290.78521963,442.58574837)
\curveto(291.0963313,442.82219324)(291.47588753,442.94041567)(291.92388834,442.94041567)
\curveto(292.35944467,442.94041567)(292.73277867,442.82219324)(293.04389034,442.58574837)
\curveto(293.35500201,442.36174797)(293.51055785,441.96352503)(293.51055785,441.39107956)
\curveto(293.51055785,440.84352302)(293.35500201,440.44530009)(293.04389034,440.19641075)
\curveto(292.73277867,439.95996588)(292.35944467,439.84174345)(291.92388834,439.84174345)
\curveto(291.47588753,439.84174345)(291.0963313,439.95996588)(290.78521963,440.19641075)
\curveto(290.47410796,440.44530009)(290.31855212,440.84352302)(290.31855212,441.39107956)
\closepath
}
}
{
\newrgbcolor{curcolor}{0 0 0}
\pscustom[linestyle=none,fillstyle=solid,fillcolor=curcolor]
{
\newpath
\moveto(295.91855988,452.90843354)
\curveto(295.91855988,453.43110114)(296.06167125,453.78576845)(296.34789399,453.97243545)
\curveto(296.64656119,454.17154691)(297.00745072,454.27110265)(297.43056259,454.27110265)
\curveto(297.84123,454.27110265)(298.1958973,454.17154691)(298.4945645,453.97243545)
\curveto(298.7932317,453.78576845)(298.9425653,453.43110114)(298.9425653,452.90843354)
\curveto(298.9425653,452.3982104)(298.7932317,452.0435431)(298.4945645,451.84443163)
\curveto(298.1958973,451.64532016)(297.84123,451.54576443)(297.43056259,451.54576443)
\curveto(297.00745072,451.54576443)(296.64656119,451.64532016)(296.34789399,451.84443163)
\curveto(296.06167125,452.0435431)(295.91855988,452.3982104)(295.91855988,452.90843354)
\closepath
\moveto(295.20922528,435.60440252)
\curveto(294.88566914,435.60440252)(294.55589077,435.62929145)(294.21989017,435.67906932)
\curveto(293.88388957,435.71640272)(293.60388907,435.76618059)(293.37988867,435.82840292)
\lineto(293.37988867,438.01240684)
\curveto(293.60388907,437.9501845)(293.815445,437.90662887)(294.01455647,437.88173994)
\curveto(294.21366794,437.856851)(294.43766834,437.84440654)(294.68655768,437.84440654)
\curveto(295.05989168,437.84440654)(295.37722558,437.9501845)(295.63855938,438.16174044)
\curveto(295.89989318,438.37329637)(296.03056008,438.78396378)(296.03056008,439.39374265)
\lineto(296.03056008,450.27642882)
\lineto(298.8118984,450.27642882)
\lineto(298.8118984,438.98307524)
\curveto(298.8118984,438.36085191)(298.69367597,437.79462867)(298.4572311,437.28440553)
\curveto(298.22078623,436.7741824)(297.83500776,436.36973723)(297.29989569,436.07107002)
\curveto(296.77722809,435.75995836)(296.08033795,435.60440252)(295.20922528,435.60440252)
\closepath
}
}
{
\newrgbcolor{curcolor}{0 0 0}
\pscustom[linestyle=none,fillstyle=solid,fillcolor=curcolor]
{
\newpath
\moveto(308.83592451,443.10841597)
\curveto(308.83592451,442.07552523)(308.46881274,441.27907936)(307.7345892,440.71907836)
\curveto(307.01281013,440.17152182)(305.93014152,439.89774355)(304.48658338,439.89774355)
\curveto(303.77724877,439.89774355)(303.1674699,439.94752142)(302.65724676,440.04707715)
\curveto(302.14702363,440.13418842)(301.63680049,440.28352202)(301.12657735,440.49507795)
\lineto(301.12657735,442.79108207)
\curveto(301.67413389,442.54219274)(302.26524606,442.33685903)(302.89991387,442.17508097)
\curveto(303.53458167,442.0133029)(304.09458267,441.93241386)(304.57991688,441.93241386)
\curveto(305.11502895,441.93241386)(305.50080742,442.0133029)(305.73725229,442.17508097)
\curveto(305.97369715,442.33685903)(306.09191959,442.54841497)(306.09191959,442.80974877)
\curveto(306.09191959,442.9839713)(306.04214172,443.13952714)(305.94258599,443.27641627)
\curveto(305.85547472,443.41330541)(305.65636325,443.56886124)(305.34525158,443.74308378)
\curveto(305.03413991,443.91730631)(304.54880571,444.14130671)(303.88924897,444.41508498)
\curveto(303.2421367,444.68886325)(302.71324686,444.95641928)(302.30257946,445.21775309)
\curveto(301.90435653,445.49153136)(301.60568932,445.81508749)(301.40657786,446.18842149)
\curveto(301.20746639,446.57419996)(301.10791065,447.05331193)(301.10791065,447.6257574)
\curveto(301.10791065,448.57153688)(301.47502242,449.28087148)(302.20924596,449.75376122)
\curveto(302.9434695,450.22665095)(303.92036014,450.46309582)(305.13991788,450.46309582)
\curveto(305.77458569,450.46309582)(306.37814232,450.40087349)(306.95058779,450.27642882)
\curveto(307.52303326,450.15198415)(308.11414544,449.94665045)(308.72392431,449.66042772)
\lineto(307.8839228,447.6630908)
\curveto(307.38614413,447.87464674)(306.91325439,448.04886927)(306.46525359,448.18575841)
\curveto(306.01725279,448.33509201)(305.56302975,448.40975881)(305.10258448,448.40975881)
\curveto(304.28124968,448.40975881)(303.87058227,448.18575841)(303.87058227,447.7377576)
\curveto(303.87058227,447.57597954)(303.92036014,447.42664594)(304.01991587,447.2897568)
\curveto(304.13191607,447.16531213)(304.33724978,447.028423)(304.63591698,446.8790894)
\curveto(304.94702865,446.7297558)(305.40125168,446.53064433)(305.99858609,446.28175499)
\curveto(306.58347602,446.04531013)(307.08747693,445.79642079)(307.5105888,445.53508699)
\curveto(307.93370067,445.28619765)(308.2572568,444.96886375)(308.4812572,444.58308528)
\curveto(308.71770207,444.19730681)(308.83592451,443.70575038)(308.83592451,443.10841597)
\closepath
}
}
{
\newrgbcolor{curcolor}{0 0 0}
\pscustom[linestyle=none,fillstyle=solid,fillcolor=curcolor]
{
\newpath
\moveto(261.56009811,418.73890141)
\lineto(258.77875979,418.73890141)
\lineto(258.77875979,432.92559351)
\lineto(261.56009811,432.92559351)
\closepath
}
}
{
\newrgbcolor{curcolor}{0 0 0}
\pscustom[linestyle=none,fillstyle=solid,fillcolor=curcolor]
{
\newpath
\moveto(268.65344991,429.13625338)
\curveto(270.02234125,429.13625338)(271.06767646,428.83758618)(271.78945553,428.24025177)
\curveto(272.52367907,427.65536184)(272.89079083,426.753138)(272.89079083,425.53358025)
\lineto(272.89079083,418.73890141)
\lineto(270.94945402,418.73890141)
\lineto(270.40811972,420.12023722)
\lineto(270.33345292,420.12023722)
\curveto(269.89789658,419.57268068)(269.43745131,419.17445774)(268.95211711,418.92556841)
\curveto(268.4667829,418.67667907)(267.80100393,418.55223441)(266.95478019,418.55223441)
\curveto(266.04633412,418.55223441)(265.29344388,418.81356821)(264.69610948,419.33623581)
\curveto(264.09877507,419.85890342)(263.80010787,420.67401599)(263.80010787,421.78157353)
\curveto(263.80010787,422.86424214)(264.17966411,423.66068801)(264.93877658,424.17091115)
\curveto(265.69788905,424.68113428)(266.83655776,424.96735702)(268.3547827,425.02957935)
\lineto(270.12811922,425.08557945)
\lineto(270.12811922,425.53358025)
\curveto(270.12811922,426.06869233)(269.98500785,426.46069303)(269.69878511,426.70958236)
\curveto(269.42500684,426.9584717)(269.03922837,427.08291637)(268.5414497,427.08291637)
\curveto(268.04367103,427.08291637)(267.55833683,427.00824957)(267.08544709,426.85891596)
\curveto(266.61255736,426.72202683)(266.13966762,426.5478043)(265.66677788,426.33624836)
\lineto(264.75210958,428.22158507)
\curveto(265.28722165,428.49536334)(265.89077829,428.71314151)(266.56277949,428.87491958)
\curveto(267.2347807,429.04914211)(267.93167083,429.13625338)(268.65344991,429.13625338)
\closepath
\moveto(270.12811922,423.46157654)
\lineto(269.04545061,423.42424314)
\curveto(268.149449,423.39935421)(267.52722566,423.23757614)(267.1787806,422.93890894)
\curveto(266.83033553,422.64024174)(266.65611299,422.24824103)(266.65611299,421.76290683)
\curveto(266.65611299,421.33979496)(266.78055766,421.03490552)(267.02944699,420.84823852)
\curveto(267.27833633,420.67401599)(267.60189246,420.58690472)(268.0001154,420.58690472)
\curveto(268.5974498,420.58690472)(269.10145071,420.76112726)(269.51211811,421.10957232)
\curveto(269.92278551,421.47046186)(270.12811922,421.97446276)(270.12811922,422.62157504)
\closepath
}
}
{
\newrgbcolor{curcolor}{0 0 0}
\pscustom[linestyle=none,fillstyle=solid,fillcolor=curcolor]
{
\newpath
\moveto(280.04013916,420.77357172)
\curveto(280.35125083,420.77357172)(280.64991803,420.79846066)(280.93614077,420.84823852)
\curveto(281.2223635,420.91046086)(281.50858624,420.99134989)(281.79480897,421.09090562)
\lineto(281.79480897,419.01890191)
\curveto(281.49614177,418.88201278)(281.12280777,418.77001258)(280.67480697,418.68290131)
\curveto(280.23925063,418.59579004)(279.76013866,418.55223441)(279.23747106,418.55223441)
\curveto(278.62769219,418.55223441)(278.08013565,418.65179014)(277.59480144,418.85090161)
\curveto(277.12191171,419.05001308)(276.74235547,419.39223591)(276.45613274,419.87757012)
\curveto(276.18235447,420.36290432)(276.04546533,421.04734999)(276.04546533,421.93090713)
\lineto(276.04546533,426.84024926)
\lineto(274.72012962,426.84024926)
\lineto(274.72012962,428.01625137)
\lineto(276.25079904,428.94958638)
\lineto(277.05346714,431.09625689)
\lineto(278.82680365,431.09625689)
\lineto(278.82680365,428.93091968)
\lineto(281.68280877,428.93091968)
\lineto(281.68280877,426.84024926)
\lineto(278.82680365,426.84024926)
\lineto(278.82680365,421.93090713)
\curveto(278.82680365,421.54512866)(278.93880385,421.25268369)(279.16280426,421.05357222)
\curveto(279.38680466,420.86690522)(279.67924963,420.77357172)(280.04013916,420.77357172)
\closepath
}
}
{
\newrgbcolor{curcolor}{0 0 0}
\pscustom[linestyle=none,fillstyle=solid,fillcolor=curcolor]
{
\newpath
\moveto(288.04815253,429.11758668)
\curveto(289.45437727,429.11758668)(290.56815705,428.71314151)(291.38949185,427.90425117)
\curveto(292.21082666,427.1078053)(292.62149406,425.96913659)(292.62149406,424.48824505)
\lineto(292.62149406,423.14424264)
\lineto(286.05081561,423.14424264)
\curveto(286.07570455,422.36024123)(286.30592718,421.74424013)(286.74148352,421.29623933)
\curveto(287.18948432,420.84823852)(287.80548543,420.62423812)(288.58948683,420.62423812)
\curveto(289.2365991,420.62423812)(289.82771127,420.68646046)(290.36282334,420.81090512)
\curveto(290.91037988,420.94779426)(291.47038089,421.15312796)(292.04282636,421.42690623)
\lineto(292.04282636,419.28023571)
\curveto(291.53260322,419.03134638)(291.00371338,418.85090161)(290.45615684,418.73890141)
\curveto(289.90860031,418.61445674)(289.24282134,418.55223441)(288.45881993,418.55223441)
\curveto(287.43837366,418.55223441)(286.53614982,418.73890141)(285.75214841,419.11223541)
\curveto(284.96814701,419.49801388)(284.3521459,420.07045935)(283.9041451,420.82957182)
\curveto(283.4561443,421.60112876)(283.2321439,422.5780194)(283.2321439,423.76024374)
\curveto(283.2321439,424.94246808)(283.43125536,425.93180319)(283.8294783,426.72824906)
\curveto(284.2401457,427.52469494)(284.80636894,428.12202934)(285.52814801,428.52025228)
\curveto(286.24992708,428.91847521)(287.08992859,429.11758668)(288.04815253,429.11758668)
\closepath
\moveto(288.06681923,427.13891647)
\curveto(287.51926269,427.13891647)(287.07126189,426.96469393)(286.72281682,426.61624886)
\curveto(286.37437175,426.26780379)(286.16903805,425.72646949)(286.10681572,424.99224595)
\lineto(290.00815604,424.99224595)
\curveto(289.99571158,425.60202482)(289.82771127,426.11224796)(289.50415514,426.52291536)
\curveto(289.19304347,426.93358276)(288.7139315,427.13891647)(288.06681923,427.13891647)
\closepath
}
}
{
\newrgbcolor{curcolor}{0 0 0}
\pscustom[linestyle=none,fillstyle=solid,fillcolor=curcolor]
{
\newpath
\moveto(296.59748761,423.94691074)
\lineto(293.31214839,428.93091968)
\lineto(296.46682071,428.93091968)
\lineto(298.44549093,425.68291386)
\lineto(300.44282784,428.93091968)
\lineto(303.59750016,428.93091968)
\lineto(300.27482754,423.94691074)
\lineto(303.74683376,418.73890141)
\lineto(300.59216144,418.73890141)
\lineto(298.44549093,422.22957433)
\lineto(296.29882041,418.73890141)
\lineto(293.14414809,418.73890141)
\closepath
}
}
{
\newrgbcolor{curcolor}{0 0 0}
\pscustom[linestyle=none,fillstyle=solid,fillcolor=curcolor]
{
\newpath
\moveto(304.9041497,420.04557042)
\curveto(304.9041497,420.61801589)(305.05970554,421.01623882)(305.37081721,421.24023923)
\curveto(305.68192887,421.47668409)(306.06148511,421.59490653)(306.50948591,421.59490653)
\curveto(306.94504225,421.59490653)(307.31837625,421.47668409)(307.62948792,421.24023923)
\curveto(307.94059959,421.01623882)(308.09615542,420.61801589)(308.09615542,420.04557042)
\curveto(308.09615542,419.49801388)(307.94059959,419.09979094)(307.62948792,418.85090161)
\curveto(307.31837625,418.61445674)(306.94504225,418.49623431)(306.50948591,418.49623431)
\curveto(306.06148511,418.49623431)(305.68192887,418.61445674)(305.37081721,418.85090161)
\curveto(305.05970554,419.09979094)(304.9041497,419.49801388)(304.9041497,420.04557042)
\closepath
}
}
{
\newrgbcolor{curcolor}{0 0 0}
\pscustom[linestyle=none,fillstyle=solid,fillcolor=curcolor]
{
\newpath
\moveto(310.50415746,431.5629244)
\curveto(310.50415746,432.085592)(310.64726883,432.4402593)(310.93349156,432.6269263)
\curveto(311.23215877,432.82603777)(311.5930483,432.92559351)(312.01616017,432.92559351)
\curveto(312.42682758,432.92559351)(312.78149488,432.82603777)(313.08016208,432.6269263)
\curveto(313.37882928,432.4402593)(313.52816288,432.085592)(313.52816288,431.5629244)
\curveto(313.52816288,431.05270126)(313.37882928,430.69803396)(313.08016208,430.49892249)
\curveto(312.78149488,430.29981102)(312.42682758,430.20025529)(312.01616017,430.20025529)
\curveto(311.5930483,430.20025529)(311.23215877,430.29981102)(310.93349156,430.49892249)
\curveto(310.64726883,430.69803396)(310.50415746,431.05270126)(310.50415746,431.5629244)
\closepath
\moveto(309.79482286,414.25889338)
\curveto(309.47126672,414.25889338)(309.14148835,414.28378231)(308.80548775,414.33356018)
\curveto(308.46948715,414.37089358)(308.18948665,414.42067145)(307.96548624,414.48289378)
\lineto(307.96548624,416.66689769)
\curveto(308.18948665,416.60467536)(308.40104258,416.56111973)(308.60015405,416.53623079)
\curveto(308.79926552,416.51134186)(309.02326592,416.49889739)(309.27215525,416.49889739)
\curveto(309.64548926,416.49889739)(309.96282316,416.60467536)(310.22415696,416.8162313)
\curveto(310.48549076,417.02778723)(310.61615766,417.43845463)(310.61615766,418.0482335)
\lineto(310.61615766,428.93091968)
\lineto(313.39749598,428.93091968)
\lineto(313.39749598,417.6375661)
\curveto(313.39749598,417.01534276)(313.27927355,416.44911953)(313.04282868,415.93889639)
\curveto(312.80638381,415.42867325)(312.42060534,415.02422808)(311.88549327,414.72556088)
\curveto(311.36282567,414.41444921)(310.66593553,414.25889338)(309.79482286,414.25889338)
\closepath
}
}
{
\newrgbcolor{curcolor}{0 0 0}
\pscustom[linestyle=none,fillstyle=solid,fillcolor=curcolor]
{
\newpath
\moveto(323.42152209,421.76290683)
\curveto(323.42152209,420.73001609)(323.05441032,419.93357022)(322.32018678,419.37356921)
\curveto(321.59840771,418.82601268)(320.5157391,418.55223441)(319.07218096,418.55223441)
\curveto(318.36284635,418.55223441)(317.75306748,418.60201227)(317.24284434,418.70156801)
\curveto(316.73262121,418.78867928)(316.22239807,418.93801288)(315.71217493,419.14956881)
\lineto(315.71217493,421.44557293)
\curveto(316.25973147,421.19668359)(316.85084364,420.99134989)(317.48551144,420.82957182)
\curveto(318.12017925,420.66779375)(318.68018025,420.58690472)(319.16551446,420.58690472)
\curveto(319.70062653,420.58690472)(320.086405,420.66779375)(320.32284986,420.82957182)
\curveto(320.55929473,420.99134989)(320.67751717,421.20290583)(320.67751717,421.46423963)
\curveto(320.67751717,421.63846216)(320.6277393,421.794018)(320.52818357,421.93090713)
\curveto(320.4410723,422.06779626)(320.24196083,422.2233521)(319.93084916,422.39757463)
\curveto(319.61973749,422.57179717)(319.13440329,422.79579757)(318.47484655,423.06957584)
\curveto(317.82773428,423.34335411)(317.29884444,423.61091014)(316.88817704,423.87224394)
\curveto(316.4899541,424.14602221)(316.1912869,424.46957835)(315.99217543,424.84291235)
\curveto(315.79306397,425.22869082)(315.69350823,425.70780279)(315.69350823,426.28024826)
\curveto(315.69350823,427.22602773)(316.06062,427.93536234)(316.79484354,428.40825207)
\curveto(317.52906708,428.88114181)(318.50595772,429.11758668)(319.72551546,429.11758668)
\curveto(320.36018326,429.11758668)(320.9637399,429.05536435)(321.53618537,428.93091968)
\curveto(322.10863084,428.80647501)(322.69974301,428.60114131)(323.30952188,428.31491857)
\lineto(322.46952038,426.31758166)
\curveto(321.97174171,426.5291376)(321.49885197,426.70336013)(321.05085117,426.84024926)
\curveto(320.60285037,426.98958286)(320.14862733,427.06424967)(319.68818206,427.06424967)
\curveto(318.86684725,427.06424967)(318.45617985,426.84024926)(318.45617985,426.39224846)
\curveto(318.45617985,426.23047039)(318.50595772,426.08113679)(318.60551345,425.94424766)
\curveto(318.71751365,425.81980299)(318.92284735,425.68291386)(319.22151456,425.53358025)
\curveto(319.53262623,425.38424665)(319.98684926,425.18513519)(320.58418367,424.93624585)
\curveto(321.1690736,424.69980098)(321.67307451,424.45091165)(322.09618638,424.18957785)
\curveto(322.51929825,423.94068851)(322.84285438,423.62335461)(323.06685478,423.23757614)
\curveto(323.30329965,422.85179767)(323.42152209,422.36024123)(323.42152209,421.76290683)
\closepath
}
}
{
\newrgbcolor{curcolor}{0 0 0}
\pscustom[linestyle=none,fillstyle=solid,fillcolor=curcolor]
{
\newpath
\moveto(269.9960995,386.3347593)
\curveto(271.15343491,386.3347593)(272.02454758,386.0360921)(272.60943752,385.4387577)
\curveto(273.20677192,384.85386776)(273.50543912,383.90808829)(273.50543912,382.60141928)
\lineto(273.50543912,375.95607403)
\lineto(270.7241008,375.95607403)
\lineto(270.7241008,381.91075137)
\curveto(270.7241008,383.37919845)(270.21387767,384.11342199)(269.19343139,384.11342199)
\curveto(268.45920785,384.11342199)(267.93654025,383.85208819)(267.62542858,383.32942058)
\curveto(267.31431691,382.80675298)(267.15876108,382.05386274)(267.15876108,381.07074987)
\lineto(267.15876108,375.95607403)
\lineto(264.37742276,375.95607403)
\lineto(264.37742276,381.91075137)
\curveto(264.37742276,383.37919845)(263.86719962,384.11342199)(262.84675335,384.11342199)
\curveto(262.07519641,384.11342199)(261.54008434,383.82097702)(261.24141714,383.23608708)
\curveto(260.9551944,382.66364161)(260.81208303,381.83608457)(260.81208303,380.75341596)
\lineto(260.81208303,375.95607403)
\lineto(258.03074472,375.95607403)
\lineto(258.03074472,386.1480923)
\lineto(260.15874853,386.1480923)
\lineto(260.53208253,384.84142329)
\lineto(260.68141613,384.84142329)
\curveto(260.9925278,385.3640909)(261.41563967,385.74364713)(261.95075174,385.980092)
\curveto(262.49830828,386.21653687)(263.06453152,386.3347593)(263.64942145,386.3347593)
\curveto(264.39608946,386.3347593)(265.02453503,386.21031463)(265.53475817,385.9614253)
\curveto(266.05742577,385.72498043)(266.46187094,385.35164643)(266.74809368,384.84142329)
\lineto(266.99076078,384.84142329)
\curveto(267.30187245,385.3640909)(267.73120655,385.74364713)(268.27876309,385.980092)
\curveto(268.83876409,386.21653687)(269.41120956,386.3347593)(269.9960995,386.3347593)
\closepath
}
}
{
\newrgbcolor{curcolor}{0 0 0}
\pscustom[linestyle=none,fillstyle=solid,fillcolor=curcolor]
{
\newpath
\moveto(280.56145766,386.3347593)
\curveto(281.9676824,386.3347593)(283.08146217,385.93031413)(283.90279698,385.12142379)
\curveto(284.72413178,384.32497792)(285.13479919,383.18630921)(285.13479919,381.70541767)
\lineto(285.13479919,380.36141526)
\lineto(278.56412074,380.36141526)
\curveto(278.58900968,379.57741386)(278.81923231,378.96141275)(279.25478865,378.51341195)
\curveto(279.70278945,378.06541115)(280.31879055,377.84141074)(281.10279196,377.84141074)
\curveto(281.74990423,377.84141074)(282.3410164,377.90363308)(282.87612847,378.02807775)
\curveto(283.42368501,378.16496688)(283.98368601,378.37030058)(284.55613148,378.64407885)
\lineto(284.55613148,376.49740833)
\curveto(284.04590835,376.248519)(283.51701851,376.06807423)(282.96946197,375.95607403)
\curveto(282.42190544,375.83162936)(281.75612646,375.76940703)(280.97212506,375.76940703)
\curveto(279.95167879,375.76940703)(279.04945495,375.95607403)(278.26545354,376.32940803)
\curveto(277.48145214,376.7151865)(276.86545103,377.28763197)(276.41745023,378.04674445)
\curveto(275.96944942,378.81830138)(275.74544902,379.79519202)(275.74544902,380.97741637)
\curveto(275.74544902,382.15964071)(275.94456049,383.14897581)(276.34278343,383.94542169)
\curveto(276.75345083,384.74186756)(277.31967407,385.33920196)(278.04145314,385.7374249)
\curveto(278.76323221,386.13564783)(279.60323372,386.3347593)(280.56145766,386.3347593)
\closepath
\moveto(280.58012436,384.35608909)
\curveto(280.03256782,384.35608909)(279.58456702,384.18186655)(279.23612195,383.83342148)
\curveto(278.88767688,383.48497642)(278.68234318,382.94364211)(278.62012084,382.20941857)
\lineto(282.52146117,382.20941857)
\curveto(282.5090167,382.81919744)(282.3410164,383.32942058)(282.01746027,383.74008798)
\curveto(281.7063486,384.15075539)(281.22723663,384.35608909)(280.58012436,384.35608909)
\closepath
}
}
{
\newrgbcolor{curcolor}{0 0 0}
\pscustom[linestyle=none,fillstyle=solid,fillcolor=curcolor]
{
\newpath
\moveto(294.50547568,378.98007945)
\curveto(294.50547568,377.94718871)(294.13836391,377.15074284)(293.40414037,376.59074184)
\curveto(292.6823613,376.0431853)(291.59969269,375.76940703)(290.15613455,375.76940703)
\curveto(289.44679994,375.76940703)(288.83702107,375.8191849)(288.32679793,375.91874063)
\curveto(287.8165748,376.0058519)(287.30635166,376.1551855)(286.79612852,376.36674143)
\lineto(286.79612852,378.66274555)
\curveto(287.34368506,378.41385621)(287.93479723,378.20852251)(288.56946504,378.04674445)
\curveto(289.20413284,377.88496638)(289.76413384,377.80407734)(290.24946805,377.80407734)
\curveto(290.78458012,377.80407734)(291.17035859,377.88496638)(291.40680346,378.04674445)
\curveto(291.64324832,378.20852251)(291.76147076,378.42007845)(291.76147076,378.68141225)
\curveto(291.76147076,378.85563478)(291.71169289,379.01119062)(291.61213716,379.14807975)
\curveto(291.52502589,379.28496889)(291.32591442,379.44052472)(291.01480275,379.61474726)
\curveto(290.70369108,379.78896979)(290.21835688,380.01297019)(289.55880014,380.28674846)
\curveto(288.91168787,380.56052673)(288.38279803,380.82808276)(287.97213063,381.08941657)
\curveto(287.5739077,381.36319483)(287.27524049,381.68675097)(287.07612903,382.06008497)
\curveto(286.87701756,382.44586344)(286.77746182,382.92497541)(286.77746182,383.49742088)
\curveto(286.77746182,384.44320036)(287.14457359,385.15253496)(287.87879713,385.6254247)
\curveto(288.61302067,386.09831443)(289.58991131,386.3347593)(290.80946905,386.3347593)
\curveto(291.44413686,386.3347593)(292.04769349,386.27253697)(292.62013896,386.1480923)
\curveto(293.19258443,386.02364763)(293.78369661,385.81831393)(294.39347548,385.5320912)
\lineto(293.55347397,383.53475428)
\curveto(293.0556953,383.74631022)(292.58280556,383.92053275)(292.13480476,384.05742189)
\curveto(291.68680396,384.20675549)(291.23258092,384.28142229)(290.77213565,384.28142229)
\curveto(289.95080085,384.28142229)(289.54013344,384.05742189)(289.54013344,383.60942108)
\curveto(289.54013344,383.44764302)(289.58991131,383.29830941)(289.68946704,383.16142028)
\curveto(289.80146724,383.03697561)(290.00680095,382.90008648)(290.30546815,382.75075288)
\curveto(290.61657982,382.60141928)(291.07080285,382.40230781)(291.66813726,382.15341847)
\curveto(292.25302719,381.91697361)(292.7570281,381.66808427)(293.18013997,381.40675047)
\curveto(293.60325184,381.15786113)(293.92680797,380.84052723)(294.15080837,380.45474876)
\curveto(294.38725324,380.06897029)(294.50547568,379.57741386)(294.50547568,378.98007945)
\closepath
}
}
{
\newrgbcolor{curcolor}{0 0 0}
\pscustom[linestyle=none,fillstyle=solid,fillcolor=curcolor]
{
\newpath
\moveto(303.78282122,378.98007945)
\curveto(303.78282122,377.94718871)(303.41570945,377.15074284)(302.68148591,376.59074184)
\curveto(301.95970684,376.0431853)(300.87703823,375.76940703)(299.43348009,375.76940703)
\curveto(298.72414548,375.76940703)(298.11436661,375.8191849)(297.60414347,375.91874063)
\curveto(297.09392034,376.0058519)(296.5836972,376.1551855)(296.07347406,376.36674143)
\lineto(296.07347406,378.66274555)
\curveto(296.6210306,378.41385621)(297.21214277,378.20852251)(297.84681058,378.04674445)
\curveto(298.48147838,377.88496638)(299.04147938,377.80407734)(299.52681359,377.80407734)
\curveto(300.06192566,377.80407734)(300.44770413,377.88496638)(300.684149,378.04674445)
\curveto(300.92059386,378.20852251)(301.0388163,378.42007845)(301.0388163,378.68141225)
\curveto(301.0388163,378.85563478)(300.98903843,379.01119062)(300.8894827,379.14807975)
\curveto(300.80237143,379.28496889)(300.60325996,379.44052472)(300.29214829,379.61474726)
\curveto(299.98103662,379.78896979)(299.49570242,380.01297019)(298.83614568,380.28674846)
\curveto(298.18903341,380.56052673)(297.66014358,380.82808276)(297.24947617,381.08941657)
\curveto(296.85125324,381.36319483)(296.55258603,381.68675097)(296.35347457,382.06008497)
\curveto(296.1543631,382.44586344)(296.05480736,382.92497541)(296.05480736,383.49742088)
\curveto(296.05480736,384.44320036)(296.42191913,385.15253496)(297.15614267,385.6254247)
\curveto(297.89036621,386.09831443)(298.86725685,386.3347593)(300.08681459,386.3347593)
\curveto(300.7214824,386.3347593)(301.32503903,386.27253697)(301.8974845,386.1480923)
\curveto(302.46992997,386.02364763)(303.06104215,385.81831393)(303.67082102,385.5320912)
\lineto(302.83081951,383.53475428)
\curveto(302.33304084,383.74631022)(301.8601511,383.92053275)(301.4121503,384.05742189)
\curveto(300.9641495,384.20675549)(300.50992646,384.28142229)(300.04948119,384.28142229)
\curveto(299.22814639,384.28142229)(298.81747898,384.05742189)(298.81747898,383.60942108)
\curveto(298.81747898,383.44764302)(298.86725685,383.29830941)(298.96681258,383.16142028)
\curveto(299.07881279,383.03697561)(299.28414649,382.90008648)(299.58281369,382.75075288)
\curveto(299.89392536,382.60141928)(300.34814839,382.40230781)(300.9454828,382.15341847)
\curveto(301.53037274,381.91697361)(302.03437364,381.66808427)(302.45748551,381.40675047)
\curveto(302.88059738,381.15786113)(303.20415351,380.84052723)(303.42815391,380.45474876)
\curveto(303.66459878,380.06897029)(303.78282122,379.57741386)(303.78282122,378.98007945)
\closepath
}
}
{
\newrgbcolor{curcolor}{0 0 0}
\pscustom[linestyle=none,fillstyle=solid,fillcolor=curcolor]
{
\newpath
\moveto(310.12949675,386.353426)
\curveto(311.49838809,386.353426)(312.54372329,386.0547588)(313.26550237,385.4574244)
\curveto(313.9997259,384.87253446)(314.36683767,383.97031062)(314.36683767,382.75075288)
\lineto(314.36683767,375.95607403)
\lineto(312.42550086,375.95607403)
\lineto(311.88416656,377.33740984)
\lineto(311.80949976,377.33740984)
\curveto(311.37394342,376.7898533)(310.91349815,376.39163037)(310.42816395,376.14274103)
\curveto(309.94282974,375.8938517)(309.27705077,375.76940703)(308.43082703,375.76940703)
\curveto(307.52238096,375.76940703)(306.76949072,376.03074083)(306.17215632,376.55340844)
\curveto(305.57482191,377.07607604)(305.27615471,377.89118861)(305.27615471,378.99874615)
\curveto(305.27615471,380.08141476)(305.65571095,380.87786063)(306.41482342,381.38808377)
\curveto(307.17393589,381.89830691)(308.3126046,382.18452964)(309.83082954,382.24675197)
\lineto(311.60416606,382.30275207)
\lineto(311.60416606,382.75075288)
\curveto(311.60416606,383.28586495)(311.46105469,383.67786565)(311.17483195,383.92675499)
\curveto(310.90105368,384.17564432)(310.51527521,384.30008899)(310.01749654,384.30008899)
\curveto(309.51971787,384.30008899)(309.03438367,384.22542219)(308.56149393,384.07608859)
\curveto(308.0886042,383.93919945)(307.61571446,383.76497692)(307.14282472,383.55342098)
\lineto(306.22815642,385.4387577)
\curveto(306.76326849,385.71253596)(307.36682513,385.93031413)(308.03882633,386.0920922)
\curveto(308.71082754,386.26631473)(309.40771767,386.353426)(310.12949675,386.353426)
\closepath
\moveto(311.60416606,380.67874916)
\lineto(310.52149745,380.64141576)
\curveto(309.62549584,380.61652683)(309.0032725,380.45474876)(308.65482743,380.15608156)
\curveto(308.30638237,379.85741436)(308.13215983,379.46541365)(308.13215983,378.98007945)
\curveto(308.13215983,378.55696758)(308.2566045,378.25207815)(308.50549383,378.06541115)
\curveto(308.75438317,377.89118861)(309.0779393,377.80407734)(309.47616224,377.80407734)
\curveto(310.07349664,377.80407734)(310.57749755,377.97829988)(310.98816495,378.32674495)
\curveto(311.39883235,378.68763448)(311.60416606,379.19163539)(311.60416606,379.83874766)
\closepath
}
}
{
\newrgbcolor{curcolor}{0 0 0}
\pscustom[linestyle=none,fillstyle=solid,fillcolor=curcolor]
{
\newpath
\moveto(320.52685042,386.3347593)
\curveto(321.78374156,386.3347593)(322.76685443,385.83698063)(323.47618904,384.84142329)
\lineto(323.55085584,384.84142329)
\lineto(323.77485624,386.1480923)
\lineto(326.12686046,386.1480923)
\lineto(326.12686046,375.93740733)
\curveto(326.12686046,374.48140472)(325.69752635,373.37384718)(324.83885815,372.61473471)
\curveto(323.98018994,371.85562224)(322.71085433,371.476066)(321.03085132,371.476066)
\curveto(320.30907225,371.476066)(319.63707104,371.51962163)(319.01484771,371.6067329)
\curveto(318.40506884,371.69384417)(317.80773443,371.8494)(317.2228445,372.0734004)
\lineto(317.2228445,374.29473772)
\curveto(318.46729117,373.77207012)(319.79262688,373.51073631)(321.19885162,373.51073631)
\curveto(322.6299653,373.51073631)(323.34552214,374.28229325)(323.34552214,375.82540713)
\lineto(323.34552214,376.03074083)
\curveto(323.34552214,376.2298523)(323.35174437,376.44140823)(323.36418884,376.66540864)
\curveto(323.3766333,376.9018535)(323.3953,377.10718721)(323.42018894,377.28140974)
\lineto(323.34552214,377.28140974)
\curveto(322.99707707,376.74629767)(322.58018743,376.3605192)(322.09485323,376.12407433)
\curveto(321.60951903,375.88762946)(321.06196249,375.76940703)(320.45218362,375.76940703)
\curveto(319.24507034,375.76940703)(318.29929087,376.2298523)(317.6148452,377.15074284)
\curveto(316.94284399,378.08407785)(316.60684339,379.37830239)(316.60684339,381.03341647)
\curveto(316.60684339,382.70097501)(316.95528846,384.00142179)(317.6521786,384.93475679)
\curveto(318.34906874,385.8680918)(319.30729268,386.3347593)(320.52685042,386.3347593)
\closepath
\moveto(321.40418532,384.07608859)
\curveto(320.09751631,384.07608859)(319.44418181,383.04942008)(319.44418181,380.99608307)
\curveto(319.44418181,378.96763498)(320.10996078,377.95341094)(321.44151872,377.95341094)
\curveto(322.15085333,377.95341094)(322.67352093,378.15252241)(323.00952153,378.55074535)
\curveto(323.3579666,378.96141275)(323.53218914,379.67074736)(323.53218914,380.67874916)
\lineto(323.53218914,381.01474977)
\curveto(323.53218914,382.10986284)(323.36418884,382.89386425)(323.02818823,383.36675398)
\curveto(322.69218763,383.83964372)(322.15085333,384.07608859)(321.40418532,384.07608859)
\closepath
}
}
{
\newrgbcolor{curcolor}{0 0 0}
\pscustom[linestyle=none,fillstyle=solid,fillcolor=curcolor]
{
\newpath
\moveto(333.2388614,386.3347593)
\curveto(334.64508614,386.3347593)(335.75886591,385.93031413)(336.58020072,385.12142379)
\curveto(337.40153552,384.32497792)(337.81220293,383.18630921)(337.81220293,381.70541767)
\lineto(337.81220293,380.36141526)
\lineto(331.24152448,380.36141526)
\curveto(331.26641342,379.57741386)(331.49663605,378.96141275)(331.93219239,378.51341195)
\curveto(332.38019319,378.06541115)(332.99619429,377.84141074)(333.7801957,377.84141074)
\curveto(334.42730797,377.84141074)(335.01842014,377.90363308)(335.55353221,378.02807775)
\curveto(336.10108875,378.16496688)(336.66108975,378.37030058)(337.23353522,378.64407885)
\lineto(337.23353522,376.49740833)
\curveto(336.72331209,376.248519)(336.19442225,376.06807423)(335.64686571,375.95607403)
\curveto(335.09930918,375.83162936)(334.4335302,375.76940703)(333.6495288,375.76940703)
\curveto(332.62908253,375.76940703)(331.72685869,375.95607403)(330.94285728,376.32940803)
\curveto(330.15885588,376.7151865)(329.54285477,377.28763197)(329.09485397,378.04674445)
\curveto(328.64685316,378.81830138)(328.42285276,379.79519202)(328.42285276,380.97741637)
\curveto(328.42285276,382.15964071)(328.62196423,383.14897581)(329.02018717,383.94542169)
\curveto(329.43085457,384.74186756)(329.99707781,385.33920196)(330.71885688,385.7374249)
\curveto(331.44063595,386.13564783)(332.28063746,386.3347593)(333.2388614,386.3347593)
\closepath
\moveto(333.2575281,384.35608909)
\curveto(332.70997156,384.35608909)(332.26197076,384.18186655)(331.91352569,383.83342148)
\curveto(331.56508062,383.48497642)(331.35974692,382.94364211)(331.29752458,382.20941857)
\lineto(335.19886491,382.20941857)
\curveto(335.18642044,382.81919744)(335.01842014,383.32942058)(334.69486401,383.74008798)
\curveto(334.38375234,384.15075539)(333.90464037,384.35608909)(333.2575281,384.35608909)
\closepath
}
}
{
\newrgbcolor{curcolor}{0 0 0}
\pscustom[linestyle=none,fillstyle=solid,fillcolor=curcolor]
{
\newpath
\moveto(347.18287942,378.98007945)
\curveto(347.18287942,377.94718871)(346.81576765,377.15074284)(346.08154411,376.59074184)
\curveto(345.35976504,376.0431853)(344.27709643,375.76940703)(342.83353829,375.76940703)
\curveto(342.12420368,375.76940703)(341.51442481,375.8191849)(341.00420167,375.91874063)
\curveto(340.49397854,376.0058519)(339.9837554,376.1551855)(339.47353226,376.36674143)
\lineto(339.47353226,378.66274555)
\curveto(340.0210888,378.41385621)(340.61220097,378.20852251)(341.24686878,378.04674445)
\curveto(341.88153658,377.88496638)(342.44153758,377.80407734)(342.92687179,377.80407734)
\curveto(343.46198386,377.80407734)(343.84776233,377.88496638)(344.0842072,378.04674445)
\curveto(344.32065206,378.20852251)(344.4388745,378.42007845)(344.4388745,378.68141225)
\curveto(344.4388745,378.85563478)(344.38909663,379.01119062)(344.2895409,379.14807975)
\curveto(344.20242963,379.28496889)(344.00331816,379.44052472)(343.69220649,379.61474726)
\curveto(343.38109482,379.78896979)(342.89576062,380.01297019)(342.23620388,380.28674846)
\curveto(341.58909161,380.56052673)(341.06020177,380.82808276)(340.64953437,381.08941657)
\curveto(340.25131144,381.36319483)(339.95264423,381.68675097)(339.75353277,382.06008497)
\curveto(339.5544213,382.44586344)(339.45486556,382.92497541)(339.45486556,383.49742088)
\curveto(339.45486556,384.44320036)(339.82197733,385.15253496)(340.55620087,385.6254247)
\curveto(341.29042441,386.09831443)(342.26731505,386.3347593)(343.48687279,386.3347593)
\curveto(344.1215406,386.3347593)(344.72509723,386.27253697)(345.2975427,386.1480923)
\curveto(345.86998817,386.02364763)(346.46110035,385.81831393)(347.07087922,385.5320912)
\lineto(346.23087771,383.53475428)
\curveto(345.73309904,383.74631022)(345.2602093,383.92053275)(344.8122085,384.05742189)
\curveto(344.3642077,384.20675549)(343.90998466,384.28142229)(343.44953939,384.28142229)
\curveto(342.62820459,384.28142229)(342.21753718,384.05742189)(342.21753718,383.60942108)
\curveto(342.21753718,383.44764302)(342.26731505,383.29830941)(342.36687078,383.16142028)
\curveto(342.47887098,383.03697561)(342.68420469,382.90008648)(342.98287189,382.75075288)
\curveto(343.29398356,382.60141928)(343.74820659,382.40230781)(344.345541,382.15341847)
\curveto(344.93043093,381.91697361)(345.43443184,381.66808427)(345.85754371,381.40675047)
\curveto(346.28065558,381.15786113)(346.60421171,380.84052723)(346.82821211,380.45474876)
\curveto(347.06465698,380.06897029)(347.18287942,379.57741386)(347.18287942,378.98007945)
\closepath
}
}
{
\newrgbcolor{curcolor}{0 0 0}
\pscustom[linestyle=none,fillstyle=solid,fillcolor=curcolor]
{
\newpath
\moveto(348.9562096,377.26274304)
\curveto(348.9562096,377.83518851)(349.11176543,378.23341145)(349.4228771,378.45741185)
\curveto(349.73398877,378.69385672)(350.11354501,378.81207915)(350.56154581,378.81207915)
\curveto(350.99710215,378.81207915)(351.37043615,378.69385672)(351.68154782,378.45741185)
\curveto(351.99265949,378.23341145)(352.14821532,377.83518851)(352.14821532,377.26274304)
\curveto(352.14821532,376.7151865)(351.99265949,376.31696357)(351.68154782,376.06807423)
\curveto(351.37043615,375.83162936)(350.99710215,375.71340693)(350.56154581,375.71340693)
\curveto(350.11354501,375.71340693)(349.73398877,375.83162936)(349.4228771,376.06807423)
\curveto(349.11176543,376.31696357)(348.9562096,376.7151865)(348.9562096,377.26274304)
\closepath
}
}
{
\newrgbcolor{curcolor}{0 0 0}
\pscustom[linestyle=none,fillstyle=solid,fillcolor=curcolor]
{
\newpath
\moveto(354.55621736,388.78009702)
\curveto(354.55621736,389.30276462)(354.69932873,389.65743192)(354.98555146,389.84409893)
\curveto(355.28421866,390.04321039)(355.6451082,390.14276613)(356.06822007,390.14276613)
\curveto(356.47888747,390.14276613)(356.83355477,390.04321039)(357.13222198,389.84409893)
\curveto(357.43088918,389.65743192)(357.58022278,389.30276462)(357.58022278,388.78009702)
\curveto(357.58022278,388.26987388)(357.43088918,387.91520658)(357.13222198,387.71609511)
\curveto(356.83355477,387.51698364)(356.47888747,387.41742791)(356.06822007,387.41742791)
\curveto(355.6451082,387.41742791)(355.28421866,387.51698364)(354.98555146,387.71609511)
\curveto(354.69932873,387.91520658)(354.55621736,388.26987388)(354.55621736,388.78009702)
\closepath
\moveto(353.84688275,371.476066)
\curveto(353.52332662,371.476066)(353.19354825,371.50095493)(352.85754765,371.5507328)
\curveto(352.52154704,371.5880662)(352.24154654,371.63784407)(352.01754614,371.7000664)
\lineto(352.01754614,373.88407032)
\curveto(352.24154654,373.82184798)(352.45310248,373.77829235)(352.65221394,373.75340342)
\curveto(352.85132541,373.72851448)(353.07532581,373.71607002)(353.32421515,373.71607002)
\curveto(353.69754915,373.71607002)(354.01488305,373.82184798)(354.27621686,374.03340392)
\curveto(354.53755066,374.24495985)(354.66821756,374.65562726)(354.66821756,375.26540613)
\lineto(354.66821756,386.1480923)
\lineto(357.44955588,386.1480923)
\lineto(357.44955588,374.85473872)
\curveto(357.44955588,374.23251539)(357.33133344,373.66629215)(357.09488858,373.15606901)
\curveto(356.85844371,372.64584587)(356.47266524,372.24140071)(355.93755317,371.9427335)
\curveto(355.41488556,371.63162183)(354.71799543,371.476066)(353.84688275,371.476066)
\closepath
}
}
{
\newrgbcolor{curcolor}{0 0 0}
\pscustom[linestyle=none,fillstyle=solid,fillcolor=curcolor]
{
\newpath
\moveto(367.47358198,378.98007945)
\curveto(367.47358198,377.94718871)(367.10647021,377.15074284)(366.37224667,376.59074184)
\curveto(365.6504676,376.0431853)(364.56779899,375.76940703)(363.12424085,375.76940703)
\curveto(362.41490625,375.76940703)(361.80512738,375.8191849)(361.29490424,375.91874063)
\curveto(360.7846811,376.0058519)(360.27445797,376.1551855)(359.76423483,376.36674143)
\lineto(359.76423483,378.66274555)
\curveto(360.31179137,378.41385621)(360.90290354,378.20852251)(361.53757134,378.04674445)
\curveto(362.17223914,377.88496638)(362.73224015,377.80407734)(363.21757435,377.80407734)
\curveto(363.75268642,377.80407734)(364.13846489,377.88496638)(364.37490976,378.04674445)
\curveto(364.61135463,378.20852251)(364.72957706,378.42007845)(364.72957706,378.68141225)
\curveto(364.72957706,378.85563478)(364.6797992,379.01119062)(364.58024346,379.14807975)
\curveto(364.49313219,379.28496889)(364.29402073,379.44052472)(363.98290906,379.61474726)
\curveto(363.67179739,379.78896979)(363.18646319,380.01297019)(362.52690645,380.28674846)
\curveto(361.87979418,380.56052673)(361.35090434,380.82808276)(360.94023694,381.08941657)
\curveto(360.542014,381.36319483)(360.2433468,381.68675097)(360.04423533,382.06008497)
\curveto(359.84512386,382.44586344)(359.74556813,382.92497541)(359.74556813,383.49742088)
\curveto(359.74556813,384.44320036)(360.1126799,385.15253496)(360.84690344,385.6254247)
\curveto(361.58112697,386.09831443)(362.55801761,386.3347593)(363.77757536,386.3347593)
\curveto(364.41224316,386.3347593)(365.0157998,386.27253697)(365.58824527,386.1480923)
\curveto(366.16069074,386.02364763)(366.75180291,385.81831393)(367.36158178,385.5320912)
\lineto(366.52158027,383.53475428)
\curveto(366.0238016,383.74631022)(365.55091187,383.92053275)(365.10291106,384.05742189)
\curveto(364.65491026,384.20675549)(364.20068723,384.28142229)(363.74024196,384.28142229)
\curveto(362.91890715,384.28142229)(362.50823975,384.05742189)(362.50823975,383.60942108)
\curveto(362.50823975,383.44764302)(362.55801761,383.29830941)(362.65757335,383.16142028)
\curveto(362.76957355,383.03697561)(362.97490725,382.90008648)(363.27357445,382.75075288)
\curveto(363.58468612,382.60141928)(364.03890916,382.40230781)(364.63624356,382.15341847)
\curveto(365.2211335,381.91697361)(365.7251344,381.66808427)(366.14824627,381.40675047)
\curveto(366.57135814,381.15786113)(366.89491428,380.84052723)(367.11891468,380.45474876)
\curveto(367.35535955,380.06897029)(367.47358198,379.57741386)(367.47358198,378.98007945)
\closepath
}
}
{
\newrgbcolor{curcolor}{0 0 0}
\pscustom[linestyle=none,fillstyle=solid,fillcolor=curcolor]
{
\newpath
\moveto(263.87648498,365.60308568)
\curveto(264.01337411,365.60308568)(264.17515218,365.59686345)(264.36181918,365.58441898)
\curveto(264.54848618,365.57197451)(264.69781978,365.55330781)(264.80981998,365.52841888)
\lineto(264.60448628,362.91508086)
\curveto(264.50493055,362.93996979)(264.37426365,362.95863649)(264.21248558,362.97108096)
\curveto(264.05070751,362.99596989)(263.90759614,363.00841436)(263.78315148,363.00841436)
\curveto(263.31026174,363.00841436)(262.8560387,362.92130309)(262.42048237,362.74708056)
\curveto(261.98492603,362.58530249)(261.63025873,362.31774646)(261.35648046,361.94441245)
\curveto(261.09514666,361.57107845)(260.96447976,361.06085532)(260.96447976,360.41374304)
\lineto(260.96447976,355.22440041)
\lineto(258.18314144,355.22440041)
\lineto(258.18314144,365.41641868)
\lineto(260.29247855,365.41641868)
\lineto(260.70314595,363.69908227)
\lineto(260.83381286,363.69908227)
\curveto(261.13248006,364.22174987)(261.54314746,364.66975067)(262.06581506,365.04308468)
\curveto(262.58848267,365.41641868)(263.19203931,365.60308568)(263.87648498,365.60308568)
\closepath
}
}
{
\newrgbcolor{curcolor}{0 0 0}
\pscustom[linestyle=none,fillstyle=solid,fillcolor=curcolor]
{
\newpath
\moveto(270.48449679,365.60308568)
\curveto(271.89072153,365.60308568)(273.0045013,365.19864051)(273.82583611,364.38975017)
\curveto(274.64717092,363.5933043)(275.05783832,362.45463559)(275.05783832,360.97374405)
\lineto(275.05783832,359.62974164)
\lineto(268.48715987,359.62974164)
\curveto(268.51204881,358.84574023)(268.74227144,358.22973913)(269.17782778,357.78173833)
\curveto(269.62582858,357.33373752)(270.24182969,357.10973712)(271.02583109,357.10973712)
\curveto(271.67294336,357.10973712)(272.26405553,357.17195946)(272.7991676,357.29640412)
\curveto(273.34672414,357.43329326)(273.90672514,357.63862696)(274.47917061,357.91240523)
\lineto(274.47917061,355.76573471)
\curveto(273.96894748,355.51684538)(273.44005764,355.33640061)(272.8925011,355.22440041)
\curveto(272.34494457,355.09995574)(271.6791656,355.03773341)(270.89516419,355.03773341)
\curveto(269.87471792,355.03773341)(268.97249408,355.22440041)(268.18849267,355.59773441)
\curveto(267.40449127,355.98351288)(266.78849016,356.55595835)(266.34048936,357.31507082)
\curveto(265.89248856,358.08662776)(265.66848815,359.0635184)(265.66848815,360.24574274)
\curveto(265.66848815,361.42796708)(265.86759962,362.41730219)(266.26582256,363.21374806)
\curveto(266.67648996,364.01019394)(267.2427132,364.60752834)(267.96449227,365.00575128)
\curveto(268.68627134,365.40397421)(269.52627285,365.60308568)(270.48449679,365.60308568)
\closepath
\moveto(270.50316349,363.62441547)
\curveto(269.95560695,363.62441547)(269.50760615,363.45019293)(269.15916108,363.10174786)
\curveto(268.81071601,362.75330279)(268.60538231,362.21196849)(268.54315997,361.47774495)
\lineto(272.4445003,361.47774495)
\curveto(272.43205583,362.08752382)(272.26405553,362.59774696)(271.9404994,363.00841436)
\curveto(271.62938773,363.41908176)(271.15027576,363.62441547)(270.50316349,363.62441547)
\closepath
}
}
{
\newrgbcolor{curcolor}{0 0 0}
\pscustom[linestyle=none,fillstyle=solid,fillcolor=curcolor]
{
\newpath
\moveto(284.42851576,358.24840583)
\curveto(284.42851576,357.21551509)(284.06140399,356.41906922)(283.32718045,355.85906821)
\curveto(282.60540138,355.31151168)(281.52273278,355.03773341)(280.07917463,355.03773341)
\curveto(279.36984003,355.03773341)(278.76006116,355.08751127)(278.24983802,355.18706701)
\curveto(277.73961488,355.27417828)(277.22939175,355.42351188)(276.71916861,355.63506781)
\lineto(276.71916861,357.93107193)
\curveto(277.26672515,357.68218259)(277.85783732,357.47684889)(278.49250512,357.31507082)
\curveto(279.12717293,357.15329275)(279.68717393,357.07240372)(280.17250813,357.07240372)
\curveto(280.7076202,357.07240372)(281.09339867,357.15329275)(281.32984354,357.31507082)
\curveto(281.56628841,357.47684889)(281.68451084,357.68840483)(281.68451084,357.94973863)
\curveto(281.68451084,358.12396116)(281.63473298,358.279517)(281.53517724,358.41640613)
\curveto(281.44806597,358.55329526)(281.24895451,358.7088511)(280.93784284,358.88307363)
\curveto(280.62673117,359.05729617)(280.14139697,359.28129657)(279.48184023,359.55507484)
\curveto(278.83472796,359.82885311)(278.30583812,360.09640914)(277.89517072,360.35774294)
\curveto(277.49694778,360.63152121)(277.19828058,360.95507735)(276.99916911,361.32841135)
\curveto(276.80005764,361.71418982)(276.70050191,362.19330179)(276.70050191,362.76574726)
\curveto(276.70050191,363.71152673)(277.06761368,364.42086134)(277.80183722,364.89375107)
\curveto(278.53606075,365.36664081)(279.51295139,365.60308568)(280.73250914,365.60308568)
\curveto(281.36717694,365.60308568)(281.97073358,365.54086335)(282.54317905,365.41641868)
\curveto(283.11562452,365.29197401)(283.70673669,365.08664031)(284.31651556,364.80041757)
\lineto(283.47651406,362.80308066)
\curveto(282.97873539,363.0146366)(282.50584565,363.18885913)(282.05784485,363.32574826)
\curveto(281.60984404,363.47508186)(281.15562101,363.54974867)(280.69517574,363.54974867)
\curveto(279.87384093,363.54974867)(279.46317353,363.32574826)(279.46317353,362.87774746)
\curveto(279.46317353,362.71596939)(279.51295139,362.56663579)(279.61250713,362.42974666)
\curveto(279.72450733,362.30530199)(279.92984103,362.16841286)(280.22850823,362.01907925)
\curveto(280.5396199,361.86974565)(280.99384294,361.67063419)(281.59117734,361.42174485)
\curveto(282.17606728,361.18529998)(282.68006818,360.93641065)(283.10318005,360.67507685)
\curveto(283.52629192,360.42618751)(283.84984806,360.10885361)(284.07384846,359.72307514)
\curveto(284.31029333,359.33729667)(284.42851576,358.84574023)(284.42851576,358.24840583)
\closepath
}
}
{
\newrgbcolor{curcolor}{0 0 0}
\pscustom[linestyle=none,fillstyle=solid,fillcolor=curcolor]
{
\newpath
\moveto(287.99384916,369.41109251)
\curveto(288.40451656,369.41109251)(288.75918386,369.31153677)(289.05785106,369.1124253)
\curveto(289.35651826,368.9257583)(289.50585187,368.571091)(289.50585187,368.0484234)
\curveto(289.50585187,367.53820026)(289.35651826,367.18353296)(289.05785106,366.98442149)
\curveto(288.75918386,366.78531002)(288.40451656,366.68575429)(287.99384916,366.68575429)
\curveto(287.57073729,366.68575429)(287.20984775,366.78531002)(286.91118055,366.98442149)
\curveto(286.62495781,367.18353296)(286.48184645,367.53820026)(286.48184645,368.0484234)
\curveto(286.48184645,368.571091)(286.62495781,368.9257583)(286.91118055,369.1124253)
\curveto(287.20984775,369.31153677)(287.57073729,369.41109251)(287.99384916,369.41109251)
\closepath
\moveto(289.37518497,365.41641868)
\lineto(289.37518497,355.22440041)
\lineto(286.59384665,355.22440041)
\lineto(286.59384665,365.41641868)
\closepath
}
}
{
\newrgbcolor{curcolor}{0 0 0}
\pscustom[linestyle=none,fillstyle=solid,fillcolor=curcolor]
{
\newpath
\moveto(299.38054437,355.22440041)
\lineto(291.33519661,355.22440041)
\lineto(291.33519661,356.86707002)
\lineto(296.05787175,363.28841486)
\lineto(291.61519712,363.28841486)
\lineto(291.61519712,365.41641868)
\lineto(299.21254407,365.41641868)
\lineto(299.21254407,363.60574877)
\lineto(294.62053584,357.35240422)
\lineto(299.38054437,357.35240422)
\closepath
}
}
{
\newrgbcolor{curcolor}{0 0 0}
\pscustom[linestyle=none,fillstyle=solid,fillcolor=curcolor]
{
\newpath
\moveto(305.59656929,365.60308568)
\curveto(307.00279403,365.60308568)(308.11657381,365.19864051)(308.93790861,364.38975017)
\curveto(309.75924342,363.5933043)(310.16991082,362.45463559)(310.16991082,360.97374405)
\lineto(310.16991082,359.62974164)
\lineto(303.59923238,359.62974164)
\curveto(303.62412131,358.84574023)(303.85434394,358.22973913)(304.28990028,357.78173833)
\curveto(304.73790108,357.33373752)(305.35390219,357.10973712)(306.13790359,357.10973712)
\curveto(306.78501586,357.10973712)(307.37612804,357.17195946)(307.91124011,357.29640412)
\curveto(308.45879664,357.43329326)(309.01879765,357.63862696)(309.59124312,357.91240523)
\lineto(309.59124312,355.76573471)
\curveto(309.08101998,355.51684538)(308.55213014,355.33640061)(308.00457361,355.22440041)
\curveto(307.45701707,355.09995574)(306.7912381,355.03773341)(306.00723669,355.03773341)
\curveto(304.98679042,355.03773341)(304.08456658,355.22440041)(303.30056517,355.59773441)
\curveto(302.51656377,355.98351288)(301.90056266,356.55595835)(301.45256186,357.31507082)
\curveto(301.00456106,358.08662776)(300.78056066,359.0635184)(300.78056066,360.24574274)
\curveto(300.78056066,361.42796708)(300.97967212,362.41730219)(301.37789506,363.21374806)
\curveto(301.78856246,364.01019394)(302.3547857,364.60752834)(303.07656477,365.00575128)
\curveto(303.79834384,365.40397421)(304.63834535,365.60308568)(305.59656929,365.60308568)
\closepath
\moveto(305.61523599,363.62441547)
\curveto(305.06767945,363.62441547)(304.61967865,363.45019293)(304.27123358,363.10174786)
\curveto(303.92278851,362.75330279)(303.71745481,362.21196849)(303.65523248,361.47774495)
\lineto(307.5565728,361.47774495)
\curveto(307.54412834,362.08752382)(307.37612804,362.59774696)(307.0525719,363.00841436)
\curveto(306.74146023,363.41908176)(306.26234826,363.62441547)(305.61523599,363.62441547)
\closepath
}
}
{
\newrgbcolor{curcolor}{0 0 0}
\pscustom[linestyle=none,fillstyle=solid,fillcolor=curcolor]
{
\newpath
\moveto(312.03657386,356.53106942)
\curveto(312.03657386,357.10351489)(312.19212969,357.50173782)(312.50324136,357.72573823)
\curveto(312.81435303,357.96218309)(313.19390927,358.08040553)(313.64191007,358.08040553)
\curveto(314.07746641,358.08040553)(314.45080041,357.96218309)(314.76191208,357.72573823)
\curveto(315.07302375,357.50173782)(315.22857958,357.10351489)(315.22857958,356.53106942)
\curveto(315.22857958,355.98351288)(315.07302375,355.58528994)(314.76191208,355.33640061)
\curveto(314.45080041,355.09995574)(314.07746641,354.98173331)(313.64191007,354.98173331)
\curveto(313.19390927,354.98173331)(312.81435303,355.09995574)(312.50324136,355.33640061)
\curveto(312.19212969,355.58528994)(312.03657386,355.98351288)(312.03657386,356.53106942)
\closepath
}
}
{
\newrgbcolor{curcolor}{0 0 0}
\pscustom[linestyle=none,fillstyle=solid,fillcolor=curcolor]
{
\newpath
\moveto(317.63658162,368.0484234)
\curveto(317.63658162,368.571091)(317.77969299,368.9257583)(318.06591572,369.1124253)
\curveto(318.36458292,369.31153677)(318.72547246,369.41109251)(319.14858433,369.41109251)
\curveto(319.55925173,369.41109251)(319.91391903,369.31153677)(320.21258624,369.1124253)
\curveto(320.51125344,368.9257583)(320.66058704,368.571091)(320.66058704,368.0484234)
\curveto(320.66058704,367.53820026)(320.51125344,367.18353296)(320.21258624,366.98442149)
\curveto(319.91391903,366.78531002)(319.55925173,366.68575429)(319.14858433,366.68575429)
\curveto(318.72547246,366.68575429)(318.36458292,366.78531002)(318.06591572,366.98442149)
\curveto(317.77969299,367.18353296)(317.63658162,367.53820026)(317.63658162,368.0484234)
\closepath
\moveto(316.92724701,350.74439238)
\curveto(316.60369088,350.74439238)(316.27391251,350.76928131)(315.93791191,350.81905918)
\curveto(315.6019113,350.85639258)(315.3219108,350.90617045)(315.0979104,350.96839278)
\lineto(315.0979104,353.15239669)
\curveto(315.3219108,353.09017436)(315.53346674,353.04661873)(315.7325782,353.02172979)
\curveto(315.93168967,352.99684086)(316.15569007,352.98439639)(316.40457941,352.98439639)
\curveto(316.77791341,352.98439639)(317.09524731,353.09017436)(317.35658112,353.3017303)
\curveto(317.61791492,353.51328623)(317.74858182,353.92395363)(317.74858182,354.5337325)
\lineto(317.74858182,365.41641868)
\lineto(320.52992014,365.41641868)
\lineto(320.52992014,354.1230651)
\curveto(320.52992014,353.50084176)(320.4116977,352.93461853)(320.17525284,352.42439539)
\curveto(319.93880797,351.91417225)(319.5530295,351.50972708)(319.01791743,351.21105988)
\curveto(318.49524982,350.89994821)(317.79835969,350.74439238)(316.92724701,350.74439238)
\closepath
}
}
{
\newrgbcolor{curcolor}{0 0 0}
\pscustom[linestyle=none,fillstyle=solid,fillcolor=curcolor]
{
\newpath
\moveto(330.55394243,358.24840583)
\curveto(330.55394243,357.21551509)(330.18683066,356.41906922)(329.45260712,355.85906821)
\curveto(328.73082805,355.31151168)(327.64815944,355.03773341)(326.2046013,355.03773341)
\curveto(325.49526669,355.03773341)(324.88548782,355.08751127)(324.37526468,355.18706701)
\curveto(323.86504155,355.27417828)(323.35481841,355.42351188)(322.84459527,355.63506781)
\lineto(322.84459527,357.93107193)
\curveto(323.39215181,357.68218259)(323.98326398,357.47684889)(324.61793179,357.31507082)
\curveto(325.25259959,357.15329275)(325.81260059,357.07240372)(326.2979348,357.07240372)
\curveto(326.83304687,357.07240372)(327.21882534,357.15329275)(327.45527021,357.31507082)
\curveto(327.69171507,357.47684889)(327.80993751,357.68840483)(327.80993751,357.94973863)
\curveto(327.80993751,358.12396116)(327.76015964,358.279517)(327.66060391,358.41640613)
\curveto(327.57349264,358.55329526)(327.37438117,358.7088511)(327.0632695,358.88307363)
\curveto(326.75215783,359.05729617)(326.26682363,359.28129657)(325.60726689,359.55507484)
\curveto(324.96015462,359.82885311)(324.43126478,360.09640914)(324.02059738,360.35774294)
\curveto(323.62237445,360.63152121)(323.32370724,360.95507735)(323.12459578,361.32841135)
\curveto(322.92548431,361.71418982)(322.82592857,362.19330179)(322.82592857,362.76574726)
\curveto(322.82592857,363.71152673)(323.19304034,364.42086134)(323.92726388,364.89375107)
\curveto(324.66148742,365.36664081)(325.63837806,365.60308568)(326.8579358,365.60308568)
\curveto(327.49260361,365.60308568)(328.09616024,365.54086335)(328.66860571,365.41641868)
\curveto(329.24105118,365.29197401)(329.83216336,365.08664031)(330.44194223,364.80041757)
\lineto(329.60194072,362.80308066)
\curveto(329.10416205,363.0146366)(328.63127231,363.18885913)(328.18327151,363.32574826)
\curveto(327.73527071,363.47508186)(327.28104767,363.54974867)(326.8206024,363.54974867)
\curveto(325.9992676,363.54974867)(325.58860019,363.32574826)(325.58860019,362.87774746)
\curveto(325.58860019,362.71596939)(325.63837806,362.56663579)(325.73793379,362.42974666)
\curveto(325.84993399,362.30530199)(326.0552677,362.16841286)(326.3539349,362.01907925)
\curveto(326.66504657,361.86974565)(327.1192696,361.67063419)(327.71660401,361.42174485)
\curveto(328.30149394,361.18529998)(328.80549485,360.93641065)(329.22860672,360.67507685)
\curveto(329.65171859,360.42618751)(329.97527472,360.10885361)(330.19927512,359.72307514)
\curveto(330.43571999,359.33729667)(330.55394243,358.84574023)(330.55394243,358.24840583)
\closepath
}
}
{
\newrgbcolor{curcolor}{0 0 0}
\pscustom[linestyle=none,fillstyle=solid,fillcolor=curcolor]
{
\newpath
\moveto(267.65641118,343.20293595)
\lineto(267.65641118,333.01091768)
\lineto(265.52840736,333.01091768)
\lineto(265.15507336,334.31758669)
\lineto(265.00573976,334.31758669)
\curveto(264.68218362,333.79491909)(264.23418282,333.41536285)(263.66173735,333.17891799)
\curveto(263.10173635,332.94247312)(262.50440194,332.82425068)(261.86973414,332.82425068)
\curveto(260.77462106,332.82425068)(259.89728616,333.11669565)(259.23772942,333.70158559)
\curveto(258.57817268,334.29891999)(258.24839431,335.2509217)(258.24839431,336.55759071)
\lineto(258.24839431,343.20293595)
\lineto(261.02973263,343.20293595)
\lineto(261.02973263,337.24825861)
\curveto(261.02973263,336.51403507)(261.16039953,335.9602563)(261.42173333,335.5869223)
\curveto(261.68306714,335.22603277)(262.09995677,335.045588)(262.67240224,335.045588)
\curveto(263.51862598,335.045588)(264.09729369,335.33181073)(264.40840535,335.9042562)
\curveto(264.71951702,336.48914614)(264.87507286,337.32292541)(264.87507286,338.40559402)
\lineto(264.87507286,343.20293595)
\closepath
}
}
{
\newrgbcolor{curcolor}{0 0 0}
\pscustom[linestyle=none,fillstyle=solid,fillcolor=curcolor]
{
\newpath
\moveto(277.68042125,336.0349231)
\curveto(277.68042125,335.00203236)(277.31330948,334.20558649)(276.57908594,333.64558549)
\curveto(275.85730687,333.09802895)(274.77463826,332.82425068)(273.33108012,332.82425068)
\curveto(272.62174552,332.82425068)(272.01196665,332.87402855)(271.50174351,332.97358428)
\curveto(270.99152037,333.06069555)(270.48129724,333.21002915)(269.9710741,333.42158509)
\lineto(269.9710741,335.7175892)
\curveto(270.51863064,335.46869987)(271.10974281,335.26336617)(271.74441061,335.1015881)
\curveto(272.37907842,334.93981003)(272.93907942,334.858921)(273.42441362,334.858921)
\curveto(273.95952569,334.858921)(274.34530416,334.93981003)(274.58174903,335.1015881)
\curveto(274.8181939,335.26336617)(274.93641633,335.4749221)(274.93641633,335.7362559)
\curveto(274.93641633,335.91047844)(274.88663847,336.06603427)(274.78708273,336.20292341)
\curveto(274.69997146,336.33981254)(274.50086,336.49536837)(274.18974833,336.66959091)
\curveto(273.87863666,336.84381344)(273.39330246,337.06781385)(272.73374572,337.34159211)
\curveto(272.08663345,337.61537038)(271.55774361,337.88292642)(271.14707621,338.14426022)
\curveto(270.74885327,338.41803849)(270.45018607,338.74159462)(270.2510746,339.11492863)
\curveto(270.05196313,339.5007071)(269.9524074,339.97981907)(269.9524074,340.55226454)
\curveto(269.9524074,341.49804401)(270.31951917,342.20737861)(271.05374271,342.68026835)
\curveto(271.78796624,343.15315809)(272.76485688,343.38960296)(273.98441463,343.38960296)
\curveto(274.61908243,343.38960296)(275.22263907,343.32738062)(275.79508454,343.20293595)
\curveto(276.36753001,343.07849129)(276.95864218,342.87315759)(277.56842105,342.58693485)
\lineto(276.72841954,340.58959794)
\curveto(276.23064087,340.80115387)(275.75775114,340.97537641)(275.30975034,341.11226554)
\curveto(274.86174953,341.26159914)(274.4075265,341.33626594)(273.94708123,341.33626594)
\curveto(273.12574642,341.33626594)(272.71507902,341.11226554)(272.71507902,340.66426474)
\curveto(272.71507902,340.50248667)(272.76485688,340.35315307)(272.86441262,340.21626393)
\curveto(272.97641282,340.09181927)(273.18174652,339.95493013)(273.48041372,339.80559653)
\curveto(273.79152539,339.65626293)(274.24574843,339.45715146)(274.84308283,339.20826213)
\curveto(275.42797277,338.97181726)(275.93197367,338.72292792)(276.35508554,338.46159412)
\curveto(276.77819741,338.21270479)(277.10175355,337.89537088)(277.32575395,337.50959241)
\curveto(277.56219882,337.12381395)(277.68042125,336.63225751)(277.68042125,336.0349231)
\closepath
}
}
{
\newrgbcolor{curcolor}{0 0 0}
\pscustom[linestyle=none,fillstyle=solid,fillcolor=curcolor]
{
\newpath
\moveto(284.04576062,343.38960296)
\curveto(285.45198536,343.38960296)(286.56576514,342.98515779)(287.38709994,342.17626745)
\curveto(288.20843475,341.37982157)(288.61910215,340.24115287)(288.61910215,338.76026132)
\lineto(288.61910215,337.41625891)
\lineto(282.0484237,337.41625891)
\curveto(282.07331264,336.63225751)(282.30353527,336.0162564)(282.73909161,335.5682556)
\curveto(283.18709241,335.1202548)(283.80309352,334.8962544)(284.58709492,334.8962544)
\curveto(285.23420719,334.8962544)(285.82531936,334.95847673)(286.36043143,335.0829214)
\curveto(286.90798797,335.21981053)(287.46798898,335.42514423)(288.04043445,335.6989225)
\lineto(288.04043445,333.55225199)
\curveto(287.53021131,333.30336265)(287.00132147,333.12291788)(286.45376493,333.01091768)
\curveto(285.9062084,332.88647302)(285.24042943,332.82425068)(284.45642802,332.82425068)
\curveto(283.43598175,332.82425068)(282.53375791,333.01091768)(281.7497565,333.38425169)
\curveto(280.9657551,333.77003016)(280.34975399,334.34247563)(279.90175319,335.1015881)
\curveto(279.45375239,335.87314504)(279.22975199,336.85003568)(279.22975199,338.03226002)
\curveto(279.22975199,339.21448436)(279.42886345,340.20381947)(279.82708639,341.00026534)
\curveto(280.23775379,341.79671121)(280.80397703,342.39404562)(281.5257561,342.79226855)
\curveto(282.24753517,343.19049149)(283.08753668,343.38960296)(284.04576062,343.38960296)
\closepath
\moveto(284.06442732,341.41093274)
\curveto(283.51687078,341.41093274)(283.06886998,341.23671021)(282.72042491,340.88826514)
\curveto(282.37197984,340.53982007)(282.16664614,339.99848577)(282.1044238,339.26426223)
\lineto(286.00576413,339.26426223)
\curveto(285.99331966,339.8740411)(285.82531936,340.38426423)(285.50176323,340.79493164)
\curveto(285.19065156,341.20559904)(284.71153959,341.41093274)(284.06442732,341.41093274)
\closepath
}
}
{
\newrgbcolor{curcolor}{0 0 0}
\pscustom[linestyle=none,fillstyle=solid,fillcolor=curcolor]
{
\newpath
\moveto(296.57110943,343.38960296)
\curveto(296.70799856,343.38960296)(296.86977663,343.38338072)(297.05644363,343.37093626)
\curveto(297.24311063,343.35849179)(297.39244423,343.33982509)(297.50444444,343.31493615)
\lineto(297.29911073,340.70159814)
\curveto(297.199555,340.72648707)(297.0688881,340.74515377)(296.90711003,340.75759824)
\curveto(296.74533196,340.78248717)(296.6022206,340.79493164)(296.47777593,340.79493164)
\curveto(296.00488619,340.79493164)(295.55066316,340.70782037)(295.11510682,340.53359784)
\curveto(294.67955048,340.37181977)(294.32488318,340.10426373)(294.05110491,339.73092973)
\curveto(293.78977111,339.35759573)(293.65910421,338.84737259)(293.65910421,338.20026032)
\lineto(293.65910421,333.01091768)
\lineto(290.87776589,333.01091768)
\lineto(290.87776589,343.20293595)
\lineto(292.987103,343.20293595)
\lineto(293.39777041,341.48559954)
\lineto(293.52843731,341.48559954)
\curveto(293.82710451,342.00826715)(294.23777191,342.45626795)(294.76043952,342.82960195)
\curveto(295.28310712,343.20293595)(295.88666376,343.38960296)(296.57110943,343.38960296)
\closepath
}
}
{
\newrgbcolor{curcolor}{0 0 0}
\pscustom[linestyle=none,fillstyle=solid,fillcolor=curcolor]
{
\newpath
\moveto(306.46447183,336.0349231)
\curveto(306.46447183,335.00203236)(306.09736006,334.20558649)(305.36313652,333.64558549)
\curveto(304.64135745,333.09802895)(303.55868884,332.82425068)(302.1151307,332.82425068)
\curveto(301.4057961,332.82425068)(300.79601722,332.87402855)(300.28579409,332.97358428)
\curveto(299.77557095,333.06069555)(299.26534781,333.21002915)(298.75512468,333.42158509)
\lineto(298.75512468,335.7175892)
\curveto(299.30268121,335.46869987)(299.89379339,335.26336617)(300.52846119,335.1015881)
\curveto(301.16312899,334.93981003)(301.72313,334.858921)(302.2084642,334.858921)
\curveto(302.74357627,334.858921)(303.12935474,334.93981003)(303.36579961,335.1015881)
\curveto(303.60224448,335.26336617)(303.72046691,335.4749221)(303.72046691,335.7362559)
\curveto(303.72046691,335.91047844)(303.67068904,336.06603427)(303.57113331,336.20292341)
\curveto(303.48402204,336.33981254)(303.28491058,336.49536837)(302.97379891,336.66959091)
\curveto(302.66268724,336.84381344)(302.17735303,337.06781385)(301.5177963,337.34159211)
\curveto(300.87068403,337.61537038)(300.34179419,337.88292642)(299.93112679,338.14426022)
\curveto(299.53290385,338.41803849)(299.23423665,338.74159462)(299.03512518,339.11492863)
\curveto(298.83601371,339.5007071)(298.73645798,339.97981907)(298.73645798,340.55226454)
\curveto(298.73645798,341.49804401)(299.10356975,342.20737861)(299.83779328,342.68026835)
\curveto(300.57201682,343.15315809)(301.54890746,343.38960296)(302.7684652,343.38960296)
\curveto(303.40313301,343.38960296)(304.00668965,343.32738062)(304.57913512,343.20293595)
\curveto(305.15158059,343.07849129)(305.74269276,342.87315759)(306.35247163,342.58693485)
\lineto(305.51247012,340.58959794)
\curveto(305.01469145,340.80115387)(304.54180172,340.97537641)(304.09380091,341.11226554)
\curveto(303.64580011,341.26159914)(303.19157707,341.33626594)(302.7311318,341.33626594)
\curveto(301.909797,341.33626594)(301.4991296,341.11226554)(301.4991296,340.66426474)
\curveto(301.4991296,340.50248667)(301.54890746,340.35315307)(301.6484632,340.21626393)
\curveto(301.7604634,340.09181927)(301.9657971,339.95493013)(302.2644643,339.80559653)
\curveto(302.57557597,339.65626293)(303.02979901,339.45715146)(303.62713341,339.20826213)
\curveto(304.21202335,338.97181726)(304.71602425,338.72292792)(305.13913612,338.46159412)
\curveto(305.56224799,338.21270479)(305.88580413,337.89537088)(306.10980453,337.50959241)
\curveto(306.3462494,337.12381395)(306.46447183,336.63225751)(306.46447183,336.0349231)
\closepath
}
}
{
\newrgbcolor{curcolor}{0 0 0}
\pscustom[linestyle=none,fillstyle=solid,fillcolor=curcolor]
{
\newpath
\moveto(314.88314726,330.06157906)
\lineto(307.1364667,330.06157906)
\lineto(307.1364667,331.33091467)
\lineto(314.88314726,331.33091467)
\closepath
}
}
{
\newrgbcolor{curcolor}{0 0 0}
\pscustom[linestyle=none,fillstyle=solid,fillcolor=curcolor]
{
\newpath
\moveto(319.08314182,333.01091768)
\lineto(316.3018035,333.01091768)
\lineto(316.3018035,347.19760978)
\lineto(319.08314182,347.19760978)
\closepath
}
}
{
\newrgbcolor{curcolor}{0 0 0}
\pscustom[linestyle=none,fillstyle=solid,fillcolor=curcolor]
{
\newpath
\moveto(323.39515387,347.19760978)
\curveto(323.80582127,347.19760978)(324.16048858,347.09805405)(324.45915578,346.89894258)
\curveto(324.75782298,346.71227558)(324.90715658,346.35760828)(324.90715658,345.83494067)
\curveto(324.90715658,345.32471754)(324.75782298,344.97005023)(324.45915578,344.77093876)
\curveto(324.16048858,344.5718273)(323.80582127,344.47227156)(323.39515387,344.47227156)
\curveto(322.972042,344.47227156)(322.61115247,344.5718273)(322.31248526,344.77093876)
\curveto(322.02626253,344.97005023)(321.88315116,345.32471754)(321.88315116,345.83494067)
\curveto(321.88315116,346.35760828)(322.02626253,346.71227558)(322.31248526,346.89894258)
\curveto(322.61115247,347.09805405)(322.972042,347.19760978)(323.39515387,347.19760978)
\closepath
\moveto(324.77648968,343.20293595)
\lineto(324.77648968,333.01091768)
\lineto(321.99515136,333.01091768)
\lineto(321.99515136,343.20293595)
\closepath
}
}
{
\newrgbcolor{curcolor}{0 0 0}
\pscustom[linestyle=none,fillstyle=solid,fillcolor=curcolor]
{
\newpath
\moveto(334.80051578,336.0349231)
\curveto(334.80051578,335.00203236)(334.43340402,334.20558649)(333.69918048,333.64558549)
\curveto(332.97740141,333.09802895)(331.8947328,332.82425068)(330.45117465,332.82425068)
\curveto(329.74184005,332.82425068)(329.13206118,332.87402855)(328.62183804,332.97358428)
\curveto(328.11161491,333.06069555)(327.60139177,333.21002915)(327.09116863,333.42158509)
\lineto(327.09116863,335.7175892)
\curveto(327.63872517,335.46869987)(328.22983734,335.26336617)(328.86450514,335.1015881)
\curveto(329.49917295,334.93981003)(330.05917395,334.858921)(330.54450816,334.858921)
\curveto(331.07962023,334.858921)(331.4653987,334.93981003)(331.70184356,335.1015881)
\curveto(331.93828843,335.26336617)(332.05651087,335.4749221)(332.05651087,335.7362559)
\curveto(332.05651087,335.91047844)(332.006733,336.06603427)(331.90717726,336.20292341)
\curveto(331.820066,336.33981254)(331.62095453,336.49536837)(331.30984286,336.66959091)
\curveto(330.99873119,336.84381344)(330.51339699,337.06781385)(329.85384025,337.34159211)
\curveto(329.20672798,337.61537038)(328.67783814,337.88292642)(328.26717074,338.14426022)
\curveto(327.8689478,338.41803849)(327.5702806,338.74159462)(327.37116913,339.11492863)
\curveto(327.17205767,339.5007071)(327.07250193,339.97981907)(327.07250193,340.55226454)
\curveto(327.07250193,341.49804401)(327.4396137,342.20737861)(328.17383724,342.68026835)
\curveto(328.90806078,343.15315809)(329.88495142,343.38960296)(331.10450916,343.38960296)
\curveto(331.73917696,343.38960296)(332.3427336,343.32738062)(332.91517907,343.20293595)
\curveto(333.48762454,343.07849129)(334.07873671,342.87315759)(334.68851558,342.58693485)
\lineto(333.84851408,340.58959794)
\curveto(333.35073541,340.80115387)(332.87784567,340.97537641)(332.42984487,341.11226554)
\curveto(331.98184407,341.26159914)(331.52762103,341.33626594)(331.06717576,341.33626594)
\curveto(330.24584095,341.33626594)(329.83517355,341.11226554)(329.83517355,340.66426474)
\curveto(329.83517355,340.50248667)(329.88495142,340.35315307)(329.98450715,340.21626393)
\curveto(330.09650735,340.09181927)(330.30184105,339.95493013)(330.60050826,339.80559653)
\curveto(330.91161992,339.65626293)(331.36584296,339.45715146)(331.96317737,339.20826213)
\curveto(332.5480673,338.97181726)(333.05206821,338.72292792)(333.47518008,338.46159412)
\curveto(333.89829195,338.21270479)(334.22184808,337.89537088)(334.44584848,337.50959241)
\curveto(334.68229335,337.12381395)(334.80051578,336.63225751)(334.80051578,336.0349231)
\closepath
}
}
{
\newrgbcolor{curcolor}{0 0 0}
\pscustom[linestyle=none,fillstyle=solid,fillcolor=curcolor]
{
\newpath
\moveto(341.2591877,335.045588)
\curveto(341.57029937,335.045588)(341.86896657,335.07047693)(342.1551893,335.1202548)
\curveto(342.44141204,335.18247713)(342.72763478,335.26336617)(343.01385751,335.3629219)
\lineto(343.01385751,333.29091819)
\curveto(342.71519031,333.15402905)(342.34185631,333.04202885)(341.8938555,332.95491758)
\curveto(341.45829917,332.86780632)(340.9791872,332.82425068)(340.45651959,332.82425068)
\curveto(339.84674072,332.82425068)(339.29918419,332.92380642)(338.81384998,333.12291788)
\curveto(338.34096025,333.32202935)(337.96140401,333.66425219)(337.67518127,334.14958639)
\curveto(337.40140301,334.6349206)(337.26451387,335.31936627)(337.26451387,336.20292341)
\lineto(337.26451387,341.11226554)
\lineto(335.93917816,341.11226554)
\lineto(335.93917816,342.28826765)
\lineto(337.46984757,343.22160265)
\lineto(338.27251568,345.36827317)
\lineto(340.04585219,345.36827317)
\lineto(340.04585219,343.20293595)
\lineto(342.90185731,343.20293595)
\lineto(342.90185731,341.11226554)
\lineto(340.04585219,341.11226554)
\lineto(340.04585219,336.20292341)
\curveto(340.04585219,335.81714494)(340.15785239,335.52469997)(340.38185279,335.3255885)
\curveto(340.60585319,335.1389215)(340.89829816,335.045588)(341.2591877,335.045588)
\closepath
}
}
{
\newrgbcolor{curcolor}{0 0 0}
\pscustom[linestyle=none,fillstyle=solid,fillcolor=curcolor]
{
\newpath
\moveto(344.67519474,334.31758669)
\curveto(344.67519474,334.89003216)(344.83075058,335.2882551)(345.14186224,335.5122555)
\curveto(345.45297391,335.74870037)(345.83253015,335.8669228)(346.28053095,335.8669228)
\curveto(346.71608729,335.8669228)(347.08942129,335.74870037)(347.40053296,335.5122555)
\curveto(347.71164463,335.2882551)(347.86720046,334.89003216)(347.86720046,334.31758669)
\curveto(347.86720046,333.77003016)(347.71164463,333.37180722)(347.40053296,333.12291788)
\curveto(347.08942129,332.88647302)(346.71608729,332.76825058)(346.28053095,332.76825058)
\curveto(345.83253015,332.76825058)(345.45297391,332.88647302)(345.14186224,333.12291788)
\curveto(344.83075058,333.37180722)(344.67519474,333.77003016)(344.67519474,334.31758669)
\closepath
}
}
{
\newrgbcolor{curcolor}{0 0 0}
\pscustom[linestyle=none,fillstyle=solid,fillcolor=curcolor]
{
\newpath
\moveto(350.2752025,345.83494067)
\curveto(350.2752025,346.35760828)(350.41831387,346.71227558)(350.7045366,346.89894258)
\curveto(351.00320381,347.09805405)(351.36409334,347.19760978)(351.78720521,347.19760978)
\curveto(352.19787261,347.19760978)(352.55253992,347.09805405)(352.85120712,346.89894258)
\curveto(353.14987432,346.71227558)(353.29920792,346.35760828)(353.29920792,345.83494067)
\curveto(353.29920792,345.32471754)(353.14987432,344.97005023)(352.85120712,344.77093876)
\curveto(352.55253992,344.5718273)(352.19787261,344.47227156)(351.78720521,344.47227156)
\curveto(351.36409334,344.47227156)(351.00320381,344.5718273)(350.7045366,344.77093876)
\curveto(350.41831387,344.97005023)(350.2752025,345.32471754)(350.2752025,345.83494067)
\closepath
\moveto(349.5658679,328.53090965)
\curveto(349.24231176,328.53090965)(348.91253339,328.55579859)(348.57653279,328.60557645)
\curveto(348.24053219,328.64290985)(347.96053168,328.69268772)(347.73653128,328.75491005)
\lineto(347.73653128,330.93891397)
\curveto(347.96053168,330.87669164)(348.17208762,330.833136)(348.37119909,330.80824707)
\curveto(348.57031056,330.78335814)(348.79431096,330.77091367)(349.04320029,330.77091367)
\curveto(349.41653429,330.77091367)(349.7338682,330.87669164)(349.995202,331.08824757)
\curveto(350.2565358,331.29980351)(350.3872027,331.71047091)(350.3872027,332.32024978)
\lineto(350.3872027,343.20293595)
\lineto(353.16854102,343.20293595)
\lineto(353.16854102,331.90958238)
\curveto(353.16854102,331.28735904)(353.05031859,330.7211358)(352.81387372,330.21091266)
\curveto(352.57742885,329.70068953)(352.19165038,329.29624436)(351.65653831,328.99757716)
\curveto(351.13387071,328.68646549)(350.43698057,328.53090965)(349.5658679,328.53090965)
\closepath
}
}
{
\newrgbcolor{curcolor}{0 0 0}
\pscustom[linestyle=none,fillstyle=solid,fillcolor=curcolor]
{
\newpath
\moveto(363.19256712,336.0349231)
\curveto(363.19256712,335.00203236)(362.82545535,334.20558649)(362.09123182,333.64558549)
\curveto(361.36945274,333.09802895)(360.28678414,332.82425068)(358.84322599,332.82425068)
\curveto(358.13389139,332.82425068)(357.52411252,332.87402855)(357.01388938,332.97358428)
\curveto(356.50366624,333.06069555)(355.99344311,333.21002915)(355.48321997,333.42158509)
\lineto(355.48321997,335.7175892)
\curveto(356.03077651,335.46869987)(356.62188868,335.26336617)(357.25655648,335.1015881)
\curveto(357.89122429,334.93981003)(358.45122529,334.858921)(358.93655949,334.858921)
\curveto(359.47167157,334.858921)(359.85745003,334.93981003)(360.0938949,335.1015881)
\curveto(360.33033977,335.26336617)(360.44856221,335.4749221)(360.44856221,335.7362559)
\curveto(360.44856221,335.91047844)(360.39878434,336.06603427)(360.2992286,336.20292341)
\curveto(360.21211734,336.33981254)(360.01300587,336.49536837)(359.7018942,336.66959091)
\curveto(359.39078253,336.84381344)(358.90544833,337.06781385)(358.24589159,337.34159211)
\curveto(357.59877932,337.61537038)(357.06988948,337.88292642)(356.65922208,338.14426022)
\curveto(356.26099914,338.41803849)(355.96233194,338.74159462)(355.76322047,339.11492863)
\curveto(355.564109,339.5007071)(355.46455327,339.97981907)(355.46455327,340.55226454)
\curveto(355.46455327,341.49804401)(355.83166504,342.20737861)(356.56588858,342.68026835)
\curveto(357.30011212,343.15315809)(358.27700276,343.38960296)(359.4965605,343.38960296)
\curveto(360.1312283,343.38960296)(360.73478494,343.32738062)(361.30723041,343.20293595)
\curveto(361.87967588,343.07849129)(362.47078805,342.87315759)(363.08056692,342.58693485)
\lineto(362.24056542,340.58959794)
\curveto(361.74278675,340.80115387)(361.26989701,340.97537641)(360.82189621,341.11226554)
\curveto(360.3738954,341.26159914)(359.91967237,341.33626594)(359.4592271,341.33626594)
\curveto(358.63789229,341.33626594)(358.22722489,341.11226554)(358.22722489,340.66426474)
\curveto(358.22722489,340.50248667)(358.27700276,340.35315307)(358.37655849,340.21626393)
\curveto(358.48855869,340.09181927)(358.69389239,339.95493013)(358.9925596,339.80559653)
\curveto(359.30367126,339.65626293)(359.7578943,339.45715146)(360.3552287,339.20826213)
\curveto(360.94011864,338.97181726)(361.44411955,338.72292792)(361.86723141,338.46159412)
\curveto(362.29034328,338.21270479)(362.61389942,337.89537088)(362.83789982,337.50959241)
\curveto(363.07434469,337.12381395)(363.19256712,336.63225751)(363.19256712,336.0349231)
\closepath
}
}
{
\newrgbcolor{curcolor}{0 0 0}
\pscustom[linestyle=none,fillstyle=solid,fillcolor=curcolor]
{
\newpath
\moveto(261.87657968,312.36773928)
\lineto(257.99390605,322.55975755)
\lineto(260.90591127,322.55975755)
\lineto(262.86591478,316.75441381)
\curveto(262.97791498,316.40596874)(263.06502625,316.04507921)(263.12724858,315.6717452)
\curveto(263.18947092,315.2984112)(263.23302655,314.9624106)(263.25791549,314.6637434)
\lineto(263.33258229,314.6637434)
\curveto(263.36991569,315.3357446)(263.50680482,316.03263474)(263.74324969,316.75441381)
\lineto(265.7032532,322.55975755)
\lineto(268.61525842,322.55975755)
\lineto(264.7325848,312.36773928)
\closepath
}
}
{
\newrgbcolor{curcolor}{0 0 0}
\pscustom[linestyle=none,fillstyle=solid,fillcolor=curcolor]
{
\newpath
\moveto(279.32994875,317.48241512)
\curveto(279.32994875,315.78996764)(278.88194795,314.48329863)(277.98594634,313.56240809)
\curveto(277.1023892,312.64151755)(275.89527593,312.18107228)(274.36460652,312.18107228)
\curveto(273.41882705,312.18107228)(272.57260331,312.38640598)(271.8259353,312.79707338)
\curveto(271.09171176,313.20774079)(270.51304406,313.80507519)(270.08993219,314.5890766)
\curveto(269.66682032,315.38552247)(269.45526439,316.34996864)(269.45526439,317.48241512)
\curveto(269.45526439,319.1748626)(269.89704296,320.47530937)(270.78060009,321.38375544)
\curveto(271.66415723,322.29220152)(272.87749274,322.74642455)(274.42060662,322.74642455)
\curveto(275.37883056,322.74642455)(276.2250543,322.54109085)(276.95927784,322.13042345)
\curveto(277.69350138,321.71975605)(278.27216908,321.12242164)(278.69528095,320.33842024)
\curveto(279.11839282,319.55441883)(279.32994875,318.60241712)(279.32994875,317.48241512)
\closepath
\moveto(272.2926028,317.48241512)
\curveto(272.2926028,316.47441331)(272.45438087,315.7090786)(272.77793701,315.186411)
\curveto(273.11393761,314.67618786)(273.65527191,314.4210763)(274.40193992,314.4210763)
\curveto(275.13616346,314.4210763)(275.66505329,314.67618786)(275.98860943,315.186411)
\curveto(276.32461003,315.7090786)(276.49261033,316.47441331)(276.49261033,317.48241512)
\curveto(276.49261033,318.49041692)(276.32461003,319.24330716)(275.98860943,319.74108583)
\curveto(275.66505329,320.25130897)(275.12994122,320.50642054)(274.38327322,320.50642054)
\curveto(273.64904968,320.50642054)(273.11393761,320.25130897)(272.77793701,319.74108583)
\curveto(272.45438087,319.24330716)(272.2926028,318.49041692)(272.2926028,317.48241512)
\closepath
}
}
{
\newrgbcolor{curcolor}{0 0 0}
\pscustom[linestyle=none,fillstyle=solid,fillcolor=curcolor]
{
\newpath
\moveto(283.02594582,326.55443138)
\curveto(283.43661322,326.55443138)(283.79128052,326.45487565)(284.08994772,326.25576418)
\curveto(284.38861493,326.06909718)(284.53794853,325.71442987)(284.53794853,325.19176227)
\curveto(284.53794853,324.68153913)(284.38861493,324.32687183)(284.08994772,324.12776036)
\curveto(283.79128052,323.92864889)(283.43661322,323.82909316)(283.02594582,323.82909316)
\curveto(282.60283395,323.82909316)(282.24194441,323.92864889)(281.94327721,324.12776036)
\curveto(281.65705447,324.32687183)(281.51394311,324.68153913)(281.51394311,325.19176227)
\curveto(281.51394311,325.71442987)(281.65705447,326.06909718)(281.94327721,326.25576418)
\curveto(282.24194441,326.45487565)(282.60283395,326.55443138)(283.02594582,326.55443138)
\closepath
\moveto(284.40728163,322.55975755)
\lineto(284.40728163,312.36773928)
\lineto(281.62594331,312.36773928)
\lineto(281.62594331,322.55975755)
\closepath
}
}
{
\newrgbcolor{curcolor}{0 0 0}
\pscustom[linestyle=none,fillstyle=solid,fillcolor=curcolor]
{
\newpath
\moveto(291.4633005,312.18107228)
\curveto(289.94507556,312.18107228)(288.76907345,312.59796192)(287.93529418,313.43174119)
\curveto(287.11395937,314.26552046)(286.70329197,315.59085617)(286.70329197,317.40774832)
\curveto(286.70329197,318.65219499)(286.9148479,319.66641903)(287.33795977,320.45042044)
\curveto(287.76107164,321.23442184)(288.34596158,321.81308955)(289.09262959,322.18642355)
\curveto(289.85174206,322.55975755)(290.72285473,322.74642455)(291.7059676,322.74642455)
\curveto(292.40285774,322.74642455)(293.00641438,322.67797999)(293.51663752,322.54109085)
\curveto(294.03930512,322.40420172)(294.49352816,322.24242365)(294.87930663,322.05575665)
\lineto(294.05797182,319.90908613)
\curveto(293.62241548,320.08330867)(293.21174808,320.22642004)(292.82596961,320.33842024)
\curveto(292.45263561,320.45042044)(292.07930161,320.50642054)(291.7059676,320.50642054)
\curveto(290.26240946,320.50642054)(289.54063039,319.47975203)(289.54063039,317.42641502)
\curveto(289.54063039,316.40596874)(289.72729739,315.6530785)(290.10063139,315.1677443)
\curveto(290.48640986,314.6824101)(291.02152193,314.439743)(291.7059676,314.439743)
\curveto(292.29085754,314.439743)(292.80730291,314.5144098)(293.25530371,314.6637434)
\curveto(293.70330452,314.82552147)(294.13886085,315.04329963)(294.56197272,315.3170779)
\lineto(294.56197272,312.94640699)
\curveto(294.13886085,312.67262872)(293.69086005,312.47973948)(293.21797031,312.36773928)
\curveto(292.75752504,312.24329461)(292.17263511,312.18107228)(291.4633005,312.18107228)
\closepath
}
}
{
\newrgbcolor{curcolor}{0 0 0}
\pscustom[linestyle=none,fillstyle=solid,fillcolor=curcolor]
{
\newpath
\moveto(301.11398127,322.74642455)
\curveto(302.52020601,322.74642455)(303.63398579,322.34197938)(304.45532059,321.53308904)
\curveto(305.2766554,320.73664317)(305.6873228,319.59797446)(305.6873228,318.11708292)
\lineto(305.6873228,316.77308051)
\lineto(299.11664435,316.77308051)
\curveto(299.14153329,315.98907911)(299.37175592,315.373078)(299.80731226,314.9250772)
\curveto(300.25531306,314.4770764)(300.87131417,314.25307599)(301.65531557,314.25307599)
\curveto(302.30242784,314.25307599)(302.89354001,314.31529833)(303.42865208,314.439743)
\curveto(303.97620862,314.57663213)(304.53620962,314.78196583)(305.1086551,315.0557441)
\lineto(305.1086551,312.90907359)
\curveto(304.59843196,312.66018425)(304.06954212,312.47973948)(303.52198558,312.36773928)
\curveto(302.97442905,312.24329461)(302.30865008,312.18107228)(301.52464867,312.18107228)
\curveto(300.5042024,312.18107228)(299.60197856,312.36773928)(298.81797715,312.74107328)
\curveto(298.03397575,313.12685175)(297.41797464,313.69929722)(296.96997384,314.4584097)
\curveto(296.52197304,315.22996663)(296.29797263,316.20685727)(296.29797263,317.38908162)
\curveto(296.29797263,318.57130596)(296.4970841,319.56064106)(296.89530704,320.35708694)
\curveto(297.30597444,321.15353281)(297.87219768,321.75086721)(298.59397675,322.14909015)
\curveto(299.31575582,322.54731308)(300.15575733,322.74642455)(301.11398127,322.74642455)
\closepath
\moveto(301.13264797,320.76775434)
\curveto(300.58509143,320.76775434)(300.13709063,320.5935318)(299.78864556,320.24508674)
\curveto(299.44020049,319.89664167)(299.23486679,319.35530736)(299.17264445,318.62108382)
\lineto(303.07398478,318.62108382)
\curveto(303.06154031,319.2308627)(302.89354001,319.74108583)(302.56998388,320.15175324)
\curveto(302.25887221,320.56242064)(301.77976024,320.76775434)(301.13264797,320.76775434)
\closepath
}
}
{
\newrgbcolor{curcolor}{0 0 0}
\pscustom[linestyle=none,fillstyle=solid,fillcolor=curcolor]
{
\newpath
\moveto(307.55398584,313.67440829)
\curveto(307.55398584,314.24685376)(307.70954167,314.6450767)(308.02065334,314.8690771)
\curveto(308.33176501,315.10552197)(308.71132124,315.2237444)(309.15932205,315.2237444)
\curveto(309.59487838,315.2237444)(309.96821239,315.10552197)(310.27932406,314.8690771)
\curveto(310.59043572,314.6450767)(310.74599156,314.24685376)(310.74599156,313.67440829)
\curveto(310.74599156,313.12685175)(310.59043572,312.72862882)(310.27932406,312.47973948)
\curveto(309.96821239,312.24329461)(309.59487838,312.12507218)(309.15932205,312.12507218)
\curveto(308.71132124,312.12507218)(308.33176501,312.24329461)(308.02065334,312.47973948)
\curveto(307.70954167,312.72862882)(307.55398584,313.12685175)(307.55398584,313.67440829)
\closepath
}
}
{
\newrgbcolor{curcolor}{0 0 0}
\pscustom[linestyle=none,fillstyle=solid,fillcolor=curcolor]
{
\newpath
\moveto(313.1539936,325.19176227)
\curveto(313.1539936,325.71442987)(313.29710496,326.06909718)(313.5833277,326.25576418)
\curveto(313.8819949,326.45487565)(314.24288444,326.55443138)(314.66599631,326.55443138)
\curveto(315.07666371,326.55443138)(315.43133101,326.45487565)(315.72999821,326.25576418)
\curveto(316.02866542,326.06909718)(316.17799902,325.71442987)(316.17799902,325.19176227)
\curveto(316.17799902,324.68153913)(316.02866542,324.32687183)(315.72999821,324.12776036)
\curveto(315.43133101,323.92864889)(315.07666371,323.82909316)(314.66599631,323.82909316)
\curveto(314.24288444,323.82909316)(313.8819949,323.92864889)(313.5833277,324.12776036)
\curveto(313.29710496,324.32687183)(313.1539936,324.68153913)(313.1539936,325.19176227)
\closepath
\moveto(312.44465899,307.88773125)
\curveto(312.12110286,307.88773125)(311.79132449,307.91262018)(311.45532388,307.96239805)
\curveto(311.11932328,307.99973145)(310.83932278,308.04950932)(310.61532238,308.11173165)
\lineto(310.61532238,310.29573557)
\curveto(310.83932278,310.23351323)(311.05087871,310.1899576)(311.24999018,310.16506867)
\curveto(311.44910165,310.14017973)(311.67310205,310.12773527)(311.92199139,310.12773527)
\curveto(312.29532539,310.12773527)(312.61265929,310.23351323)(312.87399309,310.44506917)
\curveto(313.1353269,310.6566251)(313.2659938,311.06729251)(313.2659938,311.67707138)
\lineto(313.2659938,322.55975755)
\lineto(316.04733212,322.55975755)
\lineto(316.04733212,311.26640397)
\curveto(316.04733212,310.64418064)(315.92910968,310.0779574)(315.69266481,309.56773426)
\curveto(315.45621995,309.05751113)(315.07044148,308.65306596)(314.53532941,308.35439875)
\curveto(314.0126618,308.04328709)(313.31577166,307.88773125)(312.44465899,307.88773125)
\closepath
}
}
{
\newrgbcolor{curcolor}{0 0 0}
\pscustom[linestyle=none,fillstyle=solid,fillcolor=curcolor]
{
\newpath
\moveto(326.07135822,315.3917447)
\curveto(326.07135822,314.35885396)(325.70424645,313.56240809)(324.97002291,313.00240709)
\curveto(324.24824384,312.45485055)(323.16557523,312.18107228)(321.72201709,312.18107228)
\curveto(321.01268248,312.18107228)(320.40290361,312.23085015)(319.89268048,312.33040588)
\curveto(319.38245734,312.41751715)(318.8722342,312.56685075)(318.36201107,312.77840668)
\lineto(318.36201107,315.0744108)
\curveto(318.9095676,314.82552147)(319.50067977,314.62018776)(320.13534758,314.4584097)
\curveto(320.77001538,314.29663163)(321.33001639,314.21574259)(321.81535059,314.21574259)
\curveto(322.35046266,314.21574259)(322.73624113,314.29663163)(322.972686,314.4584097)
\curveto(323.20913087,314.62018776)(323.3273533,314.8317437)(323.3273533,315.0930775)
\curveto(323.3273533,315.26730004)(323.27757543,315.42285587)(323.1780197,315.559745)
\curveto(323.09090843,315.69663414)(322.89179696,315.85218997)(322.5806853,316.02641251)
\curveto(322.26957363,316.20063504)(321.78423942,316.42463544)(321.12468269,316.69841371)
\curveto(320.47757041,316.97219198)(319.94868058,317.23974802)(319.53801317,317.50108182)
\curveto(319.13979024,317.77486009)(318.84112304,318.09841622)(318.64201157,318.47175022)
\curveto(318.4429001,318.85752869)(318.34334437,319.33664066)(318.34334437,319.90908613)
\curveto(318.34334437,320.85486561)(318.71045614,321.56420021)(319.44467967,322.03708995)
\curveto(320.17890321,322.50997968)(321.15579385,322.74642455)(322.37535159,322.74642455)
\curveto(323.0100194,322.74642455)(323.61357604,322.68420222)(324.18602151,322.55975755)
\curveto(324.75846698,322.43531288)(325.34957915,322.22997918)(325.95935802,321.94375645)
\lineto(325.11935651,319.94641953)
\curveto(324.62157784,320.15797547)(324.14868811,320.332198)(323.7006873,320.46908714)
\curveto(323.2526865,320.61842074)(322.79846346,320.69308754)(322.33801819,320.69308754)
\curveto(321.51668339,320.69308754)(321.10601599,320.46908714)(321.10601599,320.02108633)
\curveto(321.10601599,319.85930827)(321.15579385,319.70997467)(321.25534959,319.57308553)
\curveto(321.36734979,319.44864086)(321.57268349,319.31175173)(321.87135069,319.16241813)
\curveto(322.18246236,319.01308453)(322.6366854,318.81397306)(323.2340198,318.56508372)
\curveto(323.81890974,318.32863886)(324.32291064,318.07974952)(324.74602251,317.81841572)
\curveto(325.16913438,317.56952638)(325.49269052,317.25219248)(325.71669092,316.86641401)
\curveto(325.95313579,316.48063554)(326.07135822,315.98907911)(326.07135822,315.3917447)
\closepath
}
}
{
\newrgbcolor{curcolor}{0 0 0}
\pscustom[linestyle=none,fillstyle=solid,fillcolor=curcolor]
{
\newpath
\moveto(261.23121707,295.45276222)
\lineto(257.94587785,300.43677115)
\lineto(261.10055017,300.43677115)
\lineto(263.07922038,297.18876533)
\lineto(265.0765573,300.43677115)
\lineto(268.23122962,300.43677115)
\lineto(264.908557,295.45276222)
\lineto(268.38056322,290.24475288)
\lineto(265.2258909,290.24475288)
\lineto(263.07922038,293.73542581)
\lineto(260.93254987,290.24475288)
\lineto(257.77787755,290.24475288)
\closepath
}
}
{
\newrgbcolor{curcolor}{0 0 0}
\pscustom[linestyle=none,fillstyle=solid,fillcolor=curcolor]
{
\newpath
\moveto(281.89523751,300.62343816)
\curveto(283.05257292,300.62343816)(283.92368559,300.32477095)(284.50857553,299.72743655)
\curveto(285.10590993,299.14254661)(285.40457713,298.19676714)(285.40457713,296.89009813)
\lineto(285.40457713,290.24475288)
\lineto(282.62323881,290.24475288)
\lineto(282.62323881,296.19943023)
\curveto(282.62323881,297.6678773)(282.11301568,298.40210084)(281.0925694,298.40210084)
\curveto(280.35834586,298.40210084)(279.83567826,298.14076704)(279.52456659,297.61809944)
\curveto(279.21345492,297.09543183)(279.05789909,296.34254159)(279.05789909,295.35942872)
\lineto(279.05789909,290.24475288)
\lineto(276.27656077,290.24475288)
\lineto(276.27656077,296.19943023)
\curveto(276.27656077,297.6678773)(275.76633763,298.40210084)(274.74589136,298.40210084)
\curveto(273.97433442,298.40210084)(273.43922235,298.10965587)(273.14055515,297.52476593)
\curveto(272.85433241,296.95232046)(272.71122105,296.12476342)(272.71122105,295.04209482)
\lineto(272.71122105,290.24475288)
\lineto(269.92988273,290.24475288)
\lineto(269.92988273,300.43677115)
\lineto(272.05788654,300.43677115)
\lineto(272.43122054,299.13010215)
\lineto(272.58055414,299.13010215)
\curveto(272.89166581,299.65276975)(273.31477768,300.03232598)(273.84988975,300.26877085)
\curveto(274.39744629,300.50521572)(274.96366953,300.62343816)(275.54855946,300.62343816)
\curveto(276.29522747,300.62343816)(276.92367304,300.49899349)(277.43389618,300.25010415)
\curveto(277.95656378,300.01365928)(278.36100895,299.64032528)(278.64723169,299.13010215)
\lineto(278.88989879,299.13010215)
\curveto(279.20101046,299.65276975)(279.63034456,300.03232598)(280.1779011,300.26877085)
\curveto(280.7379021,300.50521572)(281.31034757,300.62343816)(281.89523751,300.62343816)
\closepath
}
}
{
\newrgbcolor{curcolor}{0 0 0}
\pscustom[linestyle=none,fillstyle=solid,fillcolor=curcolor]
{
\newpath
\moveto(291.0419255,290.24475288)
\lineto(288.26058718,290.24475288)
\lineto(288.26058718,304.43144498)
\lineto(291.0419255,304.43144498)
\closepath
}
}
{
\newrgbcolor{curcolor}{0 0 0}
\pscustom[linestyle=none,fillstyle=solid,fillcolor=curcolor]
{
\newpath
\moveto(293.56193816,291.55142189)
\curveto(293.56193816,292.12386736)(293.71749399,292.5220903)(294.02860566,292.7460907)
\curveto(294.33971733,292.98253557)(294.71927356,293.100758)(295.16727437,293.100758)
\curveto(295.6028307,293.100758)(295.97616471,292.98253557)(296.28727637,292.7460907)
\curveto(296.59838804,292.5220903)(296.75394388,292.12386736)(296.75394388,291.55142189)
\curveto(296.75394388,291.00386536)(296.59838804,290.60564242)(296.28727637,290.35675309)
\curveto(295.97616471,290.12030822)(295.6028307,290.00208578)(295.16727437,290.00208578)
\curveto(294.71927356,290.00208578)(294.33971733,290.12030822)(294.02860566,290.35675309)
\curveto(293.71749399,290.60564242)(293.56193816,291.00386536)(293.56193816,291.55142189)
\closepath
}
}
{
\newrgbcolor{curcolor}{0 0 0}
\pscustom[linestyle=none,fillstyle=solid,fillcolor=curcolor]
{
\newpath
\moveto(299.16194591,303.06877587)
\curveto(299.16194591,303.59144348)(299.30505728,303.94611078)(299.59128002,304.13277778)
\curveto(299.88994722,304.33188925)(300.25083676,304.43144498)(300.67394862,304.43144498)
\curveto(301.08461603,304.43144498)(301.43928333,304.33188925)(301.73795053,304.13277778)
\curveto(302.03661773,303.94611078)(302.18595134,303.59144348)(302.18595134,303.06877587)
\curveto(302.18595134,302.55855274)(302.03661773,302.20388543)(301.73795053,302.00477397)
\curveto(301.43928333,301.8056625)(301.08461603,301.70610676)(300.67394862,301.70610676)
\curveto(300.25083676,301.70610676)(299.88994722,301.8056625)(299.59128002,302.00477397)
\curveto(299.30505728,302.20388543)(299.16194591,302.55855274)(299.16194591,303.06877587)
\closepath
\moveto(298.45261131,285.76474485)
\curveto(298.12905517,285.76474485)(297.7992768,285.78963379)(297.4632762,285.83941165)
\curveto(297.1272756,285.87674505)(296.8472751,285.92652292)(296.6232747,285.98874526)
\lineto(296.6232747,288.17274917)
\curveto(296.8472751,288.11052684)(297.05883103,288.0669712)(297.2579425,288.04208227)
\curveto(297.45705397,288.01719334)(297.68105437,288.00474887)(297.92994371,288.00474887)
\curveto(298.30327771,288.00474887)(298.62061161,288.11052684)(298.88194541,288.32208277)
\curveto(299.14327921,288.53363871)(299.27394612,288.94430611)(299.27394612,289.55408498)
\lineto(299.27394612,300.43677115)
\lineto(302.05528443,300.43677115)
\lineto(302.05528443,289.14341758)
\curveto(302.05528443,288.52119424)(301.937062,287.954971)(301.70061713,287.44474787)
\curveto(301.46417226,286.93452473)(301.07839379,286.53007956)(300.54328172,286.23141236)
\curveto(300.02061412,285.92030069)(299.32372398,285.76474485)(298.45261131,285.76474485)
\closepath
}
}
{
\newrgbcolor{curcolor}{0 0 0}
\pscustom[linestyle=none,fillstyle=solid,fillcolor=curcolor]
{
\newpath
\moveto(312.07930672,293.26875831)
\curveto(312.07930672,292.23586756)(311.71219495,291.43942169)(310.97797142,290.87942069)
\curveto(310.25619234,290.33186415)(309.17352374,290.05808588)(307.72996559,290.05808588)
\curveto(307.02063099,290.05808588)(306.41085212,290.10786375)(305.90062898,290.20741948)
\curveto(305.39040584,290.29453075)(304.88018271,290.44386435)(304.36995957,290.65542029)
\lineto(304.36995957,292.9514244)
\curveto(304.91751611,292.70253507)(305.50862828,292.49720137)(306.14329608,292.3354233)
\curveto(306.77796389,292.17364523)(307.33796489,292.0927562)(307.82329909,292.0927562)
\curveto(308.35841116,292.0927562)(308.74418963,292.17364523)(308.9806345,292.3354233)
\curveto(309.21707937,292.49720137)(309.3353018,292.7087573)(309.3353018,292.9700911)
\curveto(309.3353018,293.14431364)(309.28552394,293.29986947)(309.1859682,293.43675861)
\curveto(309.09885694,293.57364774)(308.89974547,293.72920358)(308.5886338,293.90342611)
\curveto(308.27752213,294.07764864)(307.79218793,294.30164905)(307.13263119,294.57542731)
\curveto(306.48551892,294.84920558)(305.95662908,295.11676162)(305.54596168,295.37809542)
\curveto(305.14773874,295.65187369)(304.84907154,295.97542982)(304.64996007,296.34876383)
\curveto(304.4508486,296.7345423)(304.35129287,297.21365427)(304.35129287,297.78609974)
\curveto(304.35129287,298.73187921)(304.71840464,299.44121381)(305.45262818,299.91410355)
\curveto(306.18685172,300.38699329)(307.16374236,300.62343816)(308.3833001,300.62343816)
\curveto(309.0179679,300.62343816)(309.62152454,300.56121582)(310.19397001,300.43677115)
\curveto(310.76641548,300.31232649)(311.35752765,300.10699279)(311.96730652,299.82077005)
\lineto(311.12730502,297.82343314)
\curveto(310.62952635,298.03498907)(310.15663661,298.20921161)(309.70863581,298.34610074)
\curveto(309.260635,298.49543434)(308.80641197,298.57010114)(308.3459667,298.57010114)
\curveto(307.52463189,298.57010114)(307.11396449,298.34610074)(307.11396449,297.89809994)
\curveto(307.11396449,297.73632187)(307.16374236,297.58698827)(307.26329809,297.45009913)
\curveto(307.37529829,297.32565447)(307.58063199,297.18876533)(307.87929919,297.03943173)
\curveto(308.19041086,296.89009813)(308.6446339,296.69098666)(309.2419683,296.44209733)
\curveto(309.82685824,296.20565246)(310.33085914,295.95676312)(310.75397101,295.69542932)
\curveto(311.17708288,295.44653999)(311.50063902,295.12920608)(311.72463942,294.74342762)
\curveto(311.96108429,294.35764915)(312.07930672,293.86609271)(312.07930672,293.26875831)
\closepath
}
}
{
\newrgbcolor{curcolor}{0 0 0}
\pscustom[linestyle=none,fillstyle=solid,fillcolor=curcolor]
{
\newpath
\moveto(263.92556197,408.1154805)
\curveto(265.07045291,408.1154805)(265.99756568,407.6674797)(266.70690029,406.77147809)
\curveto(267.41623489,405.88792095)(267.7709022,404.58125194)(267.7709022,402.85147106)
\curveto(267.7709022,401.10924572)(267.40379043,399.79013224)(266.66956689,398.89413064)
\curveto(265.93534335,397.99812903)(264.99578611,397.55012823)(263.85089517,397.55012823)
\curveto(263.11667163,397.55012823)(262.53178169,397.68079513)(262.09622536,397.94212893)
\curveto(261.66066902,398.2159072)(261.30600172,398.52079663)(261.03222345,398.85679724)
\lineto(260.88288985,398.85679724)
\curveto(260.98244558,398.33412963)(261.03222345,397.83635096)(261.03222345,397.36346123)
\lineto(261.03222345,393.2567872)
\lineto(258.25088513,393.2567872)
\lineto(258.25088513,407.9288135)
\lineto(260.50955585,407.9288135)
\lineto(260.90155655,406.60347779)
\lineto(261.03222345,406.60347779)
\curveto(261.30600172,407.01414519)(261.67311349,407.36881249)(262.13355876,407.6674797)
\curveto(262.59400403,407.9661469)(263.19133843,408.1154805)(263.92556197,408.1154805)
\closepath
\moveto(263.02956036,405.89414318)
\curveto(262.30778129,405.89414318)(261.79755816,405.66392055)(261.49889095,405.20347528)
\curveto(261.20022375,404.75547448)(261.04466792,404.07725104)(261.03222345,403.16880497)
\lineto(261.03222345,402.87013776)
\curveto(261.03222345,401.88702489)(261.17533482,401.12791242)(261.46155755,400.59280035)
\curveto(261.76022476,400.07013274)(262.29533683,399.80879894)(263.06689376,399.80879894)
\curveto(263.70156157,399.80879894)(264.16822907,400.07013274)(264.46689627,400.59280035)
\curveto(264.77800794,401.12791242)(264.93356378,401.89324712)(264.93356378,402.88880446)
\curveto(264.93356378,404.89236361)(264.29889597,405.89414318)(263.02956036,405.89414318)
\closepath
}
}
{
\newrgbcolor{curcolor}{0 0 0}
\pscustom[linestyle=none,fillstyle=solid,fillcolor=curcolor]
{
\newpath
\moveto(274.26690299,408.1154805)
\curveto(275.67312773,408.1154805)(276.7869075,407.71103533)(277.60824231,406.90214499)
\curveto(278.42957711,406.10569912)(278.84024452,404.96703041)(278.84024452,403.48613887)
\lineto(278.84024452,402.14213646)
\lineto(272.26956607,402.14213646)
\curveto(272.29445501,401.35813505)(272.52467764,400.74213395)(272.96023398,400.29413315)
\curveto(273.40823478,399.84613234)(274.02423588,399.62213194)(274.80823729,399.62213194)
\curveto(275.45534956,399.62213194)(276.04646173,399.68435428)(276.5815738,399.80879894)
\curveto(277.12913034,399.94568808)(277.68913134,400.15102178)(278.26157681,400.42480005)
\lineto(278.26157681,398.27812953)
\curveto(277.75135368,398.0292402)(277.22246384,397.84879543)(276.6749073,397.73679523)
\curveto(276.12735077,397.61235056)(275.46157179,397.55012823)(274.67757039,397.55012823)
\curveto(273.65712412,397.55012823)(272.75490028,397.73679523)(271.97089887,398.11012923)
\curveto(271.18689746,398.4959077)(270.57089636,399.06835317)(270.12289556,399.82746564)
\curveto(269.67489475,400.59902258)(269.45089435,401.57591322)(269.45089435,402.75813756)
\curveto(269.45089435,403.9403619)(269.65000582,404.92969701)(270.04822876,405.72614288)
\curveto(270.45889616,406.52258876)(271.0251194,407.11992316)(271.74689847,407.5181461)
\curveto(272.46867754,407.91636903)(273.30867905,408.1154805)(274.26690299,408.1154805)
\closepath
\moveto(274.28556969,406.13681029)
\curveto(273.73801315,406.13681029)(273.29001235,405.96258775)(272.94156728,405.61414268)
\curveto(272.59312221,405.26569761)(272.38778851,404.72436331)(272.32556617,403.99013977)
\lineto(276.2269065,403.99013977)
\curveto(276.21446203,404.59991864)(276.04646173,405.11014178)(275.7229056,405.52080918)
\curveto(275.41179393,405.93147658)(274.93268196,406.13681029)(274.28556969,406.13681029)
\closepath
}
}
{
\newrgbcolor{curcolor}{0 0 0}
\pscustom[linestyle=none,fillstyle=solid,fillcolor=curcolor]
{
\newpath
\moveto(286.79225084,408.1154805)
\curveto(286.92913998,408.1154805)(287.09091805,408.10925827)(287.27758505,408.0968138)
\curveto(287.46425205,408.08436933)(287.61358565,408.06570263)(287.72558585,408.0408137)
\lineto(287.52025215,405.42747568)
\curveto(287.42069641,405.45236462)(287.29002951,405.47103132)(287.12825145,405.48347578)
\curveto(286.96647338,405.50836472)(286.82336201,405.52080918)(286.69891734,405.52080918)
\curveto(286.22602761,405.52080918)(285.77180457,405.43369791)(285.33624823,405.25947538)
\curveto(284.9006919,405.09769731)(284.54602459,404.83014128)(284.27224633,404.45680727)
\curveto(284.01091252,404.08347327)(283.88024562,403.57325014)(283.88024562,402.92613786)
\lineto(283.88024562,397.73679523)
\lineto(281.0989073,397.73679523)
\lineto(281.0989073,407.9288135)
\lineto(283.20824442,407.9288135)
\lineto(283.61891182,406.21147709)
\lineto(283.74957872,406.21147709)
\curveto(284.04824592,406.73414469)(284.45891333,407.18214549)(284.98158093,407.5554795)
\curveto(285.50424853,407.9288135)(286.10780517,408.1154805)(286.79225084,408.1154805)
\closepath
}
}
{
\newrgbcolor{curcolor}{0 0 0}
\pscustom[linestyle=none,fillstyle=solid,fillcolor=curcolor]
{
\newpath
\moveto(301.53895337,408.1154805)
\curveto(302.69628878,408.1154805)(303.56740145,407.8168133)(304.15229139,407.21947889)
\curveto(304.74962579,406.63458896)(305.04829299,405.68880948)(305.04829299,404.38214047)
\lineto(305.04829299,397.73679523)
\lineto(302.26695468,397.73679523)
\lineto(302.26695468,403.69147257)
\curveto(302.26695468,405.15991965)(301.75673154,405.89414318)(300.73628526,405.89414318)
\curveto(300.00206173,405.89414318)(299.47939412,405.63280938)(299.16828245,405.11014178)
\curveto(298.85717079,404.58747418)(298.70161495,403.83458394)(298.70161495,402.85147106)
\lineto(298.70161495,397.73679523)
\lineto(295.92027663,397.73679523)
\lineto(295.92027663,403.69147257)
\curveto(295.92027663,405.15991965)(295.41005349,405.89414318)(294.38960722,405.89414318)
\curveto(293.61805028,405.89414318)(293.08293821,405.60169822)(292.78427101,405.01680828)
\curveto(292.49804828,404.44436281)(292.35493691,403.61680577)(292.35493691,402.53413716)
\lineto(292.35493691,397.73679523)
\lineto(289.57359859,397.73679523)
\lineto(289.57359859,407.9288135)
\lineto(291.7016024,407.9288135)
\lineto(292.07493641,406.62214449)
\lineto(292.22427001,406.62214449)
\curveto(292.53538168,407.14481209)(292.95849354,407.52436833)(293.49360562,407.7608132)
\curveto(294.04116215,407.99725807)(294.60738539,408.1154805)(295.19227533,408.1154805)
\curveto(295.93894333,408.1154805)(296.5673889,407.99103583)(297.07761204,407.7421465)
\curveto(297.60027964,407.50570163)(298.00472481,407.13236763)(298.29094755,406.62214449)
\lineto(298.53361465,406.62214449)
\curveto(298.84472632,407.14481209)(299.27406042,407.52436833)(299.82161696,407.7608132)
\curveto(300.38161796,407.99725807)(300.95406343,408.1154805)(301.53895337,408.1154805)
\closepath
}
}
{
\newrgbcolor{curcolor}{0 0 0}
\pscustom[linestyle=none,fillstyle=solid,fillcolor=curcolor]
{
\newpath
\moveto(309.30430651,411.92348733)
\curveto(309.71497391,411.92348733)(310.06964121,411.82393159)(310.36830842,411.62482012)
\curveto(310.66697562,411.43815312)(310.81630922,411.08348582)(310.81630922,410.56081822)
\curveto(310.81630922,410.05059508)(310.66697562,409.69592778)(310.36830842,409.49681631)
\curveto(310.06964121,409.29770484)(309.71497391,409.19814911)(309.30430651,409.19814911)
\curveto(308.88119464,409.19814911)(308.5203051,409.29770484)(308.2216379,409.49681631)
\curveto(307.93541517,409.69592778)(307.7923038,410.05059508)(307.7923038,410.56081822)
\curveto(307.7923038,411.08348582)(307.93541517,411.43815312)(308.2216379,411.62482012)
\curveto(308.5203051,411.82393159)(308.88119464,411.92348733)(309.30430651,411.92348733)
\closepath
\moveto(310.68564232,407.9288135)
\lineto(310.68564232,397.73679523)
\lineto(307.904304,397.73679523)
\lineto(307.904304,407.9288135)
\closepath
}
}
{
\newrgbcolor{curcolor}{0 0 0}
\pscustom[linestyle=none,fillstyle=solid,fillcolor=curcolor]
{
\newpath
\moveto(320.70966461,400.76080065)
\curveto(320.70966461,399.72790991)(320.34255284,398.93146404)(319.6083293,398.37146303)
\curveto(318.88655023,397.8239065)(317.80388162,397.55012823)(316.36032348,397.55012823)
\curveto(315.65098887,397.55012823)(315.04121,397.59990609)(314.53098687,397.69946183)
\curveto(314.02076373,397.7865731)(313.51054059,397.9359067)(313.00031746,398.14746263)
\lineto(313.00031746,400.44346675)
\curveto(313.54787399,400.19457741)(314.13898616,399.98924371)(314.77365397,399.82746564)
\curveto(315.40832177,399.66568758)(315.96832278,399.58479854)(316.45365698,399.58479854)
\curveto(316.98876905,399.58479854)(317.37454752,399.66568758)(317.61099239,399.82746564)
\curveto(317.84743726,399.98924371)(317.96565969,400.20079965)(317.96565969,400.46213345)
\curveto(317.96565969,400.63635598)(317.91588182,400.79191182)(317.81632609,400.92880095)
\curveto(317.72921482,401.06569008)(317.53010335,401.22124592)(317.21899168,401.39546845)
\curveto(316.90788002,401.56969099)(316.42254581,401.79369139)(315.76298907,402.06746966)
\curveto(315.1158768,402.34124793)(314.58698697,402.60880396)(314.17631956,402.87013776)
\curveto(313.77809663,403.14391603)(313.47942943,403.46747217)(313.28031796,403.84080617)
\curveto(313.08120649,404.22658464)(312.98165076,404.70569661)(312.98165076,405.27814208)
\curveto(312.98165076,406.22392155)(313.34876252,406.93325616)(314.08298606,407.4061459)
\curveto(314.8172096,407.87903563)(315.79410024,408.1154805)(317.01365798,408.1154805)
\curveto(317.64832579,408.1154805)(318.25188242,408.05325817)(318.8243279,407.9288135)
\curveto(319.39677337,407.80436883)(319.98788554,407.59903513)(320.59766441,407.31281239)
\lineto(319.7576629,405.31547548)
\curveto(319.25988423,405.52703142)(318.78699449,405.70125395)(318.33899369,405.83814308)
\curveto(317.89099289,405.98747669)(317.43676985,406.06214349)(316.97632458,406.06214349)
\curveto(316.15498978,406.06214349)(315.74432237,405.83814308)(315.74432237,405.39014228)
\curveto(315.74432237,405.22836421)(315.79410024,405.07903061)(315.89365597,404.94214148)
\curveto(316.00565618,404.81769681)(316.21098988,404.68080768)(316.50965708,404.53147408)
\curveto(316.82076875,404.38214047)(317.27499178,404.18302901)(317.87232619,403.93413967)
\curveto(318.45721613,403.6976948)(318.96121703,403.44880547)(319.3843289,403.18747167)
\curveto(319.80744077,402.93858233)(320.1309969,402.62124843)(320.35499731,402.23546996)
\curveto(320.59144217,401.84969149)(320.70966461,401.35813505)(320.70966461,400.76080065)
\closepath
}
}
{
\newrgbcolor{curcolor}{0 0 0}
\pscustom[linestyle=none,fillstyle=solid,fillcolor=curcolor]
{
\newpath
\moveto(329.98701206,400.76080065)
\curveto(329.98701206,399.72790991)(329.61990029,398.93146404)(328.88567675,398.37146303)
\curveto(328.16389768,397.8239065)(327.08122907,397.55012823)(325.63767093,397.55012823)
\curveto(324.92833632,397.55012823)(324.31855745,397.59990609)(323.80833431,397.69946183)
\curveto(323.29811118,397.7865731)(322.78788804,397.9359067)(322.2776649,398.14746263)
\lineto(322.2776649,400.44346675)
\curveto(322.82522144,400.19457741)(323.41633361,399.98924371)(324.05100142,399.82746564)
\curveto(324.68566922,399.66568758)(325.24567022,399.58479854)(325.73100443,399.58479854)
\curveto(326.2661165,399.58479854)(326.65189497,399.66568758)(326.88833983,399.82746564)
\curveto(327.1247847,399.98924371)(327.24300714,400.20079965)(327.24300714,400.46213345)
\curveto(327.24300714,400.63635598)(327.19322927,400.79191182)(327.09367354,400.92880095)
\curveto(327.00656227,401.06569008)(326.8074508,401.22124592)(326.49633913,401.39546845)
\curveto(326.18522746,401.56969099)(325.69989326,401.79369139)(325.04033652,402.06746966)
\curveto(324.39322425,402.34124793)(323.86433441,402.60880396)(323.45366701,402.87013776)
\curveto(323.05544407,403.14391603)(322.75677687,403.46747217)(322.5576654,403.84080617)
\curveto(322.35855394,404.22658464)(322.2589982,404.70569661)(322.2589982,405.27814208)
\curveto(322.2589982,406.22392155)(322.62610997,406.93325616)(323.36033351,407.4061459)
\curveto(324.09455705,407.87903563)(325.07144769,408.1154805)(326.29100543,408.1154805)
\curveto(326.92567323,408.1154805)(327.52922987,408.05325817)(328.10167534,407.9288135)
\curveto(328.67412081,407.80436883)(329.26523298,407.59903513)(329.87501186,407.31281239)
\lineto(329.03501035,405.31547548)
\curveto(328.53723168,405.52703142)(328.06434194,405.70125395)(327.61634114,405.83814308)
\curveto(327.16834034,405.98747669)(326.7141173,406.06214349)(326.25367203,406.06214349)
\curveto(325.43233722,406.06214349)(325.02166982,405.83814308)(325.02166982,405.39014228)
\curveto(325.02166982,405.22836421)(325.07144769,405.07903061)(325.17100342,404.94214148)
\curveto(325.28300362,404.81769681)(325.48833732,404.68080768)(325.78700453,404.53147408)
\curveto(326.0981162,404.38214047)(326.55233923,404.18302901)(327.14967364,403.93413967)
\curveto(327.73456357,403.6976948)(328.23856448,403.44880547)(328.66167635,403.18747167)
\curveto(329.08478822,402.93858233)(329.40834435,402.62124843)(329.63234475,402.23546996)
\curveto(329.86878962,401.84969149)(329.98701206,401.35813505)(329.98701206,400.76080065)
\closepath
}
}
{
\newrgbcolor{curcolor}{0 0 0}
\pscustom[linestyle=none,fillstyle=solid,fillcolor=curcolor]
{
\newpath
\moveto(333.55234545,411.92348733)
\curveto(333.96301285,411.92348733)(334.31768016,411.82393159)(334.61634736,411.62482012)
\curveto(334.91501456,411.43815312)(335.06434816,411.08348582)(335.06434816,410.56081822)
\curveto(335.06434816,410.05059508)(334.91501456,409.69592778)(334.61634736,409.49681631)
\curveto(334.31768016,409.29770484)(333.96301285,409.19814911)(333.55234545,409.19814911)
\curveto(333.12923358,409.19814911)(332.76834404,409.29770484)(332.46967684,409.49681631)
\curveto(332.18345411,409.69592778)(332.04034274,410.05059508)(332.04034274,410.56081822)
\curveto(332.04034274,411.08348582)(332.18345411,411.43815312)(332.46967684,411.62482012)
\curveto(332.76834404,411.82393159)(333.12923358,411.92348733)(333.55234545,411.92348733)
\closepath
\moveto(334.93368126,407.9288135)
\lineto(334.93368126,397.73679523)
\lineto(332.15234294,397.73679523)
\lineto(332.15234294,407.9288135)
\closepath
}
}
{
\newrgbcolor{curcolor}{0 0 0}
\pscustom[linestyle=none,fillstyle=solid,fillcolor=curcolor]
{
\newpath
\moveto(347.10437788,402.85147106)
\curveto(347.10437788,401.15902359)(346.65637707,399.85235458)(345.76037547,398.93146404)
\curveto(344.87681833,398.0105735)(343.66970505,397.55012823)(342.13903564,397.55012823)
\curveto(341.19325617,397.55012823)(340.34703243,397.75546193)(339.60036443,398.16612933)
\curveto(338.86614089,398.57679673)(338.28747318,399.17413114)(337.86436131,399.95813254)
\curveto(337.44124944,400.75457842)(337.22969351,401.71902459)(337.22969351,402.85147106)
\curveto(337.22969351,404.54391854)(337.67147208,405.84436532)(338.55502922,406.75281139)
\curveto(339.43858636,407.66125746)(340.65192187,408.1154805)(342.19503574,408.1154805)
\curveto(343.15325968,408.1154805)(343.99948342,407.9101468)(344.73370696,407.4994794)
\curveto(345.4679305,407.08881199)(346.0465982,406.49147759)(346.46971007,405.70747618)
\curveto(346.89282194,404.92347478)(347.10437788,403.97147307)(347.10437788,402.85147106)
\closepath
\moveto(340.06703193,402.85147106)
\curveto(340.06703193,401.84346926)(340.22881,401.07813455)(340.55236613,400.55546695)
\curveto(340.88836673,400.04524381)(341.42970104,399.79013224)(342.17636904,399.79013224)
\curveto(342.91059258,399.79013224)(343.43948242,400.04524381)(343.76303855,400.55546695)
\curveto(344.09903916,401.07813455)(344.26703946,401.84346926)(344.26703946,402.85147106)
\curveto(344.26703946,403.85947287)(344.09903916,404.61236311)(343.76303855,405.11014178)
\curveto(343.43948242,405.62036492)(342.90437035,405.87547648)(342.15770234,405.87547648)
\curveto(341.42347881,405.87547648)(340.88836673,405.62036492)(340.55236613,405.11014178)
\curveto(340.22881,404.61236311)(340.06703193,403.85947287)(340.06703193,402.85147106)
\closepath
}
}
{
\newrgbcolor{curcolor}{0 0 0}
\pscustom[linestyle=none,fillstyle=solid,fillcolor=curcolor]
{
\newpath
\moveto(355.18705138,408.1154805)
\curveto(356.28216445,408.1154805)(357.15949936,407.8168133)(357.8190561,407.21947889)
\curveto(358.47861283,406.63458896)(358.8083912,405.68880948)(358.8083912,404.38214047)
\lineto(358.8083912,397.73679523)
\lineto(356.02705288,397.73679523)
\lineto(356.02705288,403.69147257)
\curveto(356.02705288,404.42569611)(355.89638598,404.97325265)(355.63505218,405.33414218)
\curveto(355.37371838,405.70747618)(354.95682874,405.89414318)(354.38438327,405.89414318)
\curveto(353.53815953,405.89414318)(352.95949183,405.60169822)(352.64838016,405.01680828)
\curveto(352.33726849,404.44436281)(352.18171266,403.61680577)(352.18171266,402.53413716)
\lineto(352.18171266,397.73679523)
\lineto(349.40037434,397.73679523)
\lineto(349.40037434,407.9288135)
\lineto(351.52837815,407.9288135)
\lineto(351.90171216,406.62214449)
\lineto(352.05104576,406.62214449)
\curveto(352.37460189,407.14481209)(352.81638046,407.52436833)(353.37638147,407.7608132)
\curveto(353.94882694,407.99725807)(354.55238357,408.1154805)(355.18705138,408.1154805)
\closepath
}
}
{
\newrgbcolor{curcolor}{0 0 0}
\pscustom[linestyle=none,fillstyle=solid,fillcolor=curcolor]
{
\newpath
\moveto(361.27238105,399.04346424)
\curveto(361.27238105,399.61590971)(361.42793688,400.01413264)(361.73904855,400.23813305)
\curveto(362.05016022,400.47457791)(362.42971646,400.59280035)(362.87771726,400.59280035)
\curveto(363.3132736,400.59280035)(363.6866076,400.47457791)(363.99771927,400.23813305)
\curveto(364.30883094,400.01413264)(364.46438677,399.61590971)(364.46438677,399.04346424)
\curveto(364.46438677,398.4959077)(364.30883094,398.09768476)(363.99771927,397.84879543)
\curveto(363.6866076,397.61235056)(363.3132736,397.49412813)(362.87771726,397.49412813)
\curveto(362.42971646,397.49412813)(362.05016022,397.61235056)(361.73904855,397.84879543)
\curveto(361.42793688,398.09768476)(361.27238105,398.4959077)(361.27238105,399.04346424)
\closepath
}
}
{
\newrgbcolor{curcolor}{0 0 0}
\pscustom[linestyle=none,fillstyle=solid,fillcolor=curcolor]
{
\newpath
\moveto(366.87238881,410.56081822)
\curveto(366.87238881,411.08348582)(367.01550018,411.43815312)(367.30172291,411.62482012)
\curveto(367.60039011,411.82393159)(367.96127965,411.92348733)(368.38439152,411.92348733)
\curveto(368.79505892,411.92348733)(369.14972622,411.82393159)(369.44839343,411.62482012)
\curveto(369.74706063,411.43815312)(369.89639423,411.08348582)(369.89639423,410.56081822)
\curveto(369.89639423,410.05059508)(369.74706063,409.69592778)(369.44839343,409.49681631)
\curveto(369.14972622,409.29770484)(368.79505892,409.19814911)(368.38439152,409.19814911)
\curveto(367.96127965,409.19814911)(367.60039011,409.29770484)(367.30172291,409.49681631)
\curveto(367.01550018,409.69592778)(366.87238881,410.05059508)(366.87238881,410.56081822)
\closepath
\moveto(366.1630542,393.2567872)
\curveto(365.83949807,393.2567872)(365.5097197,393.28167613)(365.1737191,393.331454)
\curveto(364.8377185,393.3687874)(364.55771799,393.41856527)(364.33371759,393.4807876)
\lineto(364.33371759,395.66479151)
\curveto(364.55771799,395.60256918)(364.76927393,395.55901355)(364.9683854,395.53412461)
\curveto(365.16749686,395.50923568)(365.39149727,395.49679121)(365.6403866,395.49679121)
\curveto(366.0137206,395.49679121)(366.33105451,395.60256918)(366.59238831,395.81412512)
\curveto(366.85372211,396.02568105)(366.98438901,396.43634845)(366.98438901,397.04612732)
\lineto(366.98438901,407.9288135)
\lineto(369.76572733,407.9288135)
\lineto(369.76572733,396.63545992)
\curveto(369.76572733,396.01323658)(369.6475049,395.44701335)(369.41106003,394.93679021)
\curveto(369.17461516,394.42656707)(368.78883669,394.0221219)(368.25372462,393.7234547)
\curveto(367.73105702,393.41234303)(367.03416688,393.2567872)(366.1630542,393.2567872)
\closepath
}
}
{
\newrgbcolor{curcolor}{0 0 0}
\pscustom[linestyle=none,fillstyle=solid,fillcolor=curcolor]
{
\newpath
\moveto(379.78975343,400.76080065)
\curveto(379.78975343,399.72790991)(379.42264166,398.93146404)(378.68841813,398.37146303)
\curveto(377.96663905,397.8239065)(376.88397045,397.55012823)(375.4404123,397.55012823)
\curveto(374.7310777,397.55012823)(374.12129883,397.59990609)(373.61107569,397.69946183)
\curveto(373.10085255,397.7865731)(372.59062942,397.9359067)(372.08040628,398.14746263)
\lineto(372.08040628,400.44346675)
\curveto(372.62796282,400.19457741)(373.21907499,399.98924371)(373.85374279,399.82746564)
\curveto(374.4884106,399.66568758)(375.0484116,399.58479854)(375.5337458,399.58479854)
\curveto(376.06885787,399.58479854)(376.45463634,399.66568758)(376.69108121,399.82746564)
\curveto(376.92752608,399.98924371)(377.04574851,400.20079965)(377.04574851,400.46213345)
\curveto(377.04574851,400.63635598)(376.99597065,400.79191182)(376.89641491,400.92880095)
\curveto(376.80930365,401.06569008)(376.61019218,401.22124592)(376.29908051,401.39546845)
\curveto(375.98796884,401.56969099)(375.50263464,401.79369139)(374.8430779,402.06746966)
\curveto(374.19596563,402.34124793)(373.66707579,402.60880396)(373.25640839,402.87013776)
\curveto(372.85818545,403.14391603)(372.55951825,403.46747217)(372.36040678,403.84080617)
\curveto(372.16129531,404.22658464)(372.06173958,404.70569661)(372.06173958,405.27814208)
\curveto(372.06173958,406.22392155)(372.42885135,406.93325616)(373.16307489,407.4061459)
\curveto(373.89729843,407.87903563)(374.87418907,408.1154805)(376.09374681,408.1154805)
\curveto(376.72841461,408.1154805)(377.33197125,408.05325817)(377.90441672,407.9288135)
\curveto(378.47686219,407.80436883)(379.06797436,407.59903513)(379.67775323,407.31281239)
\lineto(378.83775173,405.31547548)
\curveto(378.33997306,405.52703142)(377.86708332,405.70125395)(377.41908252,405.83814308)
\curveto(376.97108171,405.98747669)(376.51685868,406.06214349)(376.05641341,406.06214349)
\curveto(375.2350786,406.06214349)(374.8244112,405.83814308)(374.8244112,405.39014228)
\curveto(374.8244112,405.22836421)(374.87418907,405.07903061)(374.9737448,404.94214148)
\curveto(375.085745,404.81769681)(375.2910787,404.68080768)(375.5897459,404.53147408)
\curveto(375.90085757,404.38214047)(376.35508061,404.18302901)(376.95241501,403.93413967)
\curveto(377.53730495,403.6976948)(378.04130585,403.44880547)(378.46441772,403.18747167)
\curveto(378.88752959,402.93858233)(379.21108573,402.62124843)(379.43508613,402.23546996)
\curveto(379.671531,401.84969149)(379.78975343,401.35813505)(379.78975343,400.76080065)
\closepath
}
}
{
\newrgbcolor{curcolor}{0 0 0}
\pscustom[linestyle=none,fillstyle=solid,fillcolor=curcolor]
{
\newpath
\moveto(219.38197597,228.85737847)
\curveto(217.86375102,228.85737847)(216.68774891,229.27426811)(215.85396964,230.10804738)
\curveto(215.03263484,230.94182665)(214.62196743,232.26716236)(214.62196743,234.08405451)
\curveto(214.62196743,235.32850118)(214.83352337,236.34272522)(215.25663524,237.12672663)
\curveto(215.67974711,237.91072803)(216.26463704,238.48939574)(217.01130505,238.86272974)
\curveto(217.77041752,239.23606374)(218.64153019,239.42273074)(219.62464307,239.42273074)
\curveto(220.32153321,239.42273074)(220.92508984,239.35428618)(221.43531298,239.21739704)
\curveto(221.95798058,239.08050791)(222.41220362,238.91872984)(222.79798209,238.73206284)
\lineto(221.97664728,236.58539232)
\curveto(221.54109095,236.75961486)(221.13042355,236.90272623)(220.74464508,237.01472643)
\curveto(220.37131107,237.12672663)(219.99797707,237.18272673)(219.62464307,237.18272673)
\curveto(218.18108492,237.18272673)(217.45930585,236.15605822)(217.45930585,234.10272121)
\curveto(217.45930585,233.08227493)(217.64597285,232.32938469)(218.01930686,231.84405049)
\curveto(218.40508533,231.35871629)(218.9401974,231.11604919)(219.62464307,231.11604919)
\curveto(220.20953301,231.11604919)(220.72597838,231.19071599)(221.17397918,231.34004959)
\curveto(221.62197998,231.50182766)(222.05753632,231.71960582)(222.48064819,231.99338409)
\lineto(222.48064819,229.62271318)
\curveto(222.05753632,229.34893491)(221.60953551,229.15604567)(221.13664578,229.04404547)
\curveto(220.67620051,228.9196008)(220.09131057,228.85737847)(219.38197597,228.85737847)
\closepath
}
}
{
\newrgbcolor{curcolor}{0 0 0}
\pscustom[linestyle=none,fillstyle=solid,fillcolor=curcolor]
{
\newpath
\moveto(227.61398943,243.23073757)
\lineto(227.61398943,240.33739905)
\curveto(227.61398943,239.82717591)(227.59532273,239.34184171)(227.55798933,238.88139644)
\curveto(227.5331004,238.43339564)(227.50821146,238.11606173)(227.48332253,237.92939473)
\lineto(227.63265613,237.92939473)
\curveto(227.95621227,238.45206234)(228.3731019,238.83161857)(228.88332504,239.06806344)
\curveto(229.39354818,239.30450831)(229.95977141,239.42273074)(230.58199475,239.42273074)
\curveto(231.67710782,239.42273074)(232.56066496,239.12406354)(233.23266617,238.52672914)
\curveto(233.90466737,237.9418392)(234.24066798,236.99605973)(234.24066798,235.68939072)
\lineto(234.24066798,229.04404547)
\lineto(231.45932966,229.04404547)
\lineto(231.45932966,234.99872281)
\curveto(231.45932966,236.46716989)(230.91177312,237.20139343)(229.81666004,237.20139343)
\curveto(228.98288077,237.20139343)(228.40421307,236.90894846)(228.08065693,236.32405852)
\curveto(227.76954526,235.75161305)(227.61398943,234.92405601)(227.61398943,233.8413874)
\lineto(227.61398943,229.04404547)
\lineto(224.83265111,229.04404547)
\lineto(224.83265111,243.23073757)
\closepath
}
}
{
\newrgbcolor{curcolor}{0 0 0}
\pscustom[linestyle=none,fillstyle=solid,fillcolor=curcolor]
{
\newpath
\moveto(246.35534751,234.15872131)
\curveto(246.35534751,232.46627383)(245.90734671,231.15960482)(245.0113451,230.23871428)
\curveto(244.12778796,229.31782374)(242.92067469,228.85737847)(241.39000528,228.85737847)
\curveto(240.4442258,228.85737847)(239.59800206,229.06271217)(238.85133406,229.47337957)
\curveto(238.11711052,229.88404698)(237.53844282,230.48138138)(237.11533095,231.26538279)
\curveto(236.69221908,232.06182866)(236.48066314,233.02627483)(236.48066314,234.15872131)
\curveto(236.48066314,235.85116879)(236.92244171,237.15161556)(237.80599885,238.06006163)
\curveto(238.68955599,238.96850771)(239.9028915,239.42273074)(241.44600538,239.42273074)
\curveto(242.40422932,239.42273074)(243.25045306,239.21739704)(243.98467659,238.80672964)
\curveto(244.71890013,238.39606224)(245.29756784,237.79872783)(245.72067971,237.01472643)
\curveto(246.14379158,236.23072502)(246.35534751,235.27872331)(246.35534751,234.15872131)
\closepath
\moveto(239.31800156,234.15872131)
\curveto(239.31800156,233.1507195)(239.47977963,232.38538479)(239.80333576,231.86271719)
\curveto(240.13933637,231.35249405)(240.68067067,231.09738249)(241.42733868,231.09738249)
\curveto(242.16156221,231.09738249)(242.69045205,231.35249405)(243.01400819,231.86271719)
\curveto(243.35000879,232.38538479)(243.51800909,233.1507195)(243.51800909,234.15872131)
\curveto(243.51800909,235.16672311)(243.35000879,235.91961335)(243.01400819,236.41739202)
\curveto(242.69045205,236.92761516)(242.15533998,237.18272673)(241.40867198,237.18272673)
\curveto(240.67444844,237.18272673)(240.13933637,236.92761516)(239.80333576,236.41739202)
\curveto(239.47977963,235.91961335)(239.31800156,235.16672311)(239.31800156,234.15872131)
\closepath
}
}
{
\newrgbcolor{curcolor}{0 0 0}
\pscustom[linestyle=none,fillstyle=solid,fillcolor=curcolor]
{
\newpath
\moveto(250.05134648,243.23073757)
\curveto(250.46201388,243.23073757)(250.81668119,243.13118184)(251.11534839,242.93207037)
\curveto(251.41401559,242.74540337)(251.56334919,242.39073606)(251.56334919,241.86806846)
\curveto(251.56334919,241.35784532)(251.41401559,241.00317802)(251.11534839,240.80406655)
\curveto(250.81668119,240.60495508)(250.46201388,240.50539935)(250.05134648,240.50539935)
\curveto(249.62823461,240.50539935)(249.26734508,240.60495508)(248.96867787,240.80406655)
\curveto(248.68245514,241.00317802)(248.53934377,241.35784532)(248.53934377,241.86806846)
\curveto(248.53934377,242.39073606)(248.68245514,242.74540337)(248.96867787,242.93207037)
\curveto(249.26734508,243.13118184)(249.62823461,243.23073757)(250.05134648,243.23073757)
\closepath
\moveto(251.43268229,239.23606374)
\lineto(251.43268229,229.04404547)
\lineto(248.65134397,229.04404547)
\lineto(248.65134397,239.23606374)
\closepath
}
}
{
\newrgbcolor{curcolor}{0 0 0}
\pscustom[linestyle=none,fillstyle=solid,fillcolor=curcolor]
{
\newpath
\moveto(258.48869926,228.85737847)
\curveto(256.97047432,228.85737847)(255.79447221,229.27426811)(254.96069293,230.10804738)
\curveto(254.13935813,230.94182665)(253.72869073,232.26716236)(253.72869073,234.08405451)
\curveto(253.72869073,235.32850118)(253.94024666,236.34272522)(254.36335853,237.12672663)
\curveto(254.7864704,237.91072803)(255.37136034,238.48939574)(256.11802834,238.86272974)
\curveto(256.87714081,239.23606374)(257.74825349,239.42273074)(258.73136636,239.42273074)
\curveto(259.4282565,239.42273074)(260.03181314,239.35428618)(260.54203627,239.21739704)
\curveto(261.06470388,239.08050791)(261.51892691,238.91872984)(261.90470538,238.73206284)
\lineto(261.08337058,236.58539232)
\curveto(260.64781424,236.75961486)(260.23714684,236.90272623)(259.85136837,237.01472643)
\curveto(259.47803437,237.12672663)(259.10470036,237.18272673)(258.73136636,237.18272673)
\curveto(257.28780822,237.18272673)(256.56602915,236.15605822)(256.56602915,234.10272121)
\curveto(256.56602915,233.08227493)(256.75269615,232.32938469)(257.12603015,231.84405049)
\curveto(257.51180862,231.35871629)(258.04692069,231.11604919)(258.73136636,231.11604919)
\curveto(259.3162563,231.11604919)(259.83270167,231.19071599)(260.28070247,231.34004959)
\curveto(260.72870327,231.50182766)(261.16425961,231.71960582)(261.58737148,231.99338409)
\lineto(261.58737148,229.62271318)
\curveto(261.16425961,229.34893491)(260.71625881,229.15604567)(260.24336907,229.04404547)
\curveto(259.7829238,228.9196008)(259.19803386,228.85737847)(258.48869926,228.85737847)
\closepath
}
}
{
\newrgbcolor{curcolor}{0 0 0}
\pscustom[linestyle=none,fillstyle=solid,fillcolor=curcolor]
{
\newpath
\moveto(268.13938384,239.42273074)
\curveto(269.54560858,239.42273074)(270.65938836,239.01828557)(271.48072316,238.20939523)
\curveto(272.30205797,237.41294936)(272.71272537,236.27428065)(272.71272537,234.79338911)
\lineto(272.71272537,233.4493867)
\lineto(266.14204693,233.4493867)
\curveto(266.16693586,232.6653853)(266.39715849,232.04938419)(266.83271483,231.60138339)
\curveto(267.28071563,231.15338259)(267.89671674,230.92938218)(268.68071814,230.92938218)
\curveto(269.32783041,230.92938218)(269.91894258,230.99160452)(270.45405466,231.11604919)
\curveto(271.00161119,231.25293832)(271.5616122,231.45827202)(272.13405767,231.73205029)
\lineto(272.13405767,229.58537978)
\curveto(271.62383453,229.33649044)(271.09494469,229.15604567)(270.54738816,229.04404547)
\curveto(269.99983162,228.9196008)(269.33405265,228.85737847)(268.55005124,228.85737847)
\curveto(267.52960497,228.85737847)(266.62738113,229.04404547)(265.84337972,229.41737947)
\curveto(265.05937832,229.80315794)(264.44337721,230.37560341)(263.99537641,231.13471589)
\curveto(263.54737561,231.90627282)(263.32337521,232.88316346)(263.32337521,234.06538781)
\curveto(263.32337521,235.24761215)(263.52248667,236.23694725)(263.92070961,237.03339313)
\curveto(264.33137701,237.829839)(264.89760025,238.4271734)(265.61937932,238.82539634)
\curveto(266.34115839,239.22361928)(267.1811599,239.42273074)(268.13938384,239.42273074)
\closepath
\moveto(268.15805054,237.44406053)
\curveto(267.610494,237.44406053)(267.1624932,237.269838)(266.81404813,236.92139293)
\curveto(266.46560306,236.57294786)(266.26026936,236.03161355)(266.19804703,235.29739001)
\lineto(270.09938735,235.29739001)
\curveto(270.08694289,235.90716889)(269.91894258,236.41739202)(269.59538645,236.82805943)
\curveto(269.28427478,237.23872683)(268.80516281,237.44406053)(268.15805054,237.44406053)
\closepath
}
}
{
\newrgbcolor{curcolor}{0 0 0}
\pscustom[linestyle=none,fillstyle=solid,fillcolor=curcolor]
{
\newpath
\moveto(280.85139965,240.20673215)
\curveto(279.76873104,240.20673215)(278.941174,239.80228698)(278.36872853,238.99339664)
\curveto(277.79628306,238.1845063)(277.51006033,237.07694876)(277.51006033,235.67072402)
\curveto(277.51006033,234.25205481)(277.77139413,233.1507195)(278.29406173,232.36671809)
\curveto(278.8291738,231.59516116)(279.68161978,231.20938269)(280.85139965,231.20938269)
\curveto(281.38651172,231.20938269)(281.92784603,231.27160502)(282.47540256,231.39604969)
\curveto(283.0229591,231.52049436)(283.61407127,231.69471689)(284.24873907,231.91871729)
\lineto(284.24873907,229.54804638)
\curveto(283.66384914,229.31160151)(283.08518143,229.13737897)(282.51273596,229.02537877)
\curveto(281.94029049,228.91337857)(281.29940045,228.85737847)(280.59006585,228.85737847)
\curveto(279.20873004,228.85737847)(278.07628357,229.13737897)(277.19272643,229.69737998)
\curveto(276.30916929,230.26982545)(275.65583478,231.06627132)(275.23272291,232.08671759)
\curveto(274.80961104,233.11960833)(274.59805511,234.32049937)(274.59805511,235.68939072)
\curveto(274.59805511,237.03339313)(274.84072221,238.2218397)(275.32605641,239.25473044)
\curveto(275.81139062,240.28762118)(276.51450299,241.09651152)(277.43539353,241.68140146)
\curveto(278.36872853,242.2662914)(279.50739724,242.55873636)(280.85139965,242.55873636)
\curveto(281.51095639,242.55873636)(282.17051313,242.4716251)(282.83006986,242.29740256)
\curveto(283.50207107,242.1356245)(284.14296111,241.91162409)(284.75273998,241.62540136)
\lineto(283.83807167,239.32939724)
\curveto(283.340293,239.56584211)(282.8362921,239.77117581)(282.32606896,239.94539835)
\curveto(281.82829029,240.11962088)(281.33673385,240.20673215)(280.85139965,240.20673215)
\closepath
}
}
{
\newrgbcolor{curcolor}{0 0 0}
\pscustom[linestyle=none,fillstyle=solid,fillcolor=curcolor]
{
\newpath
\moveto(292.55540328,239.42273074)
\curveto(292.69229241,239.42273074)(292.85407048,239.41650851)(293.04073748,239.40406404)
\curveto(293.22740448,239.39161958)(293.37673808,239.37295288)(293.48873828,239.34806394)
\lineto(293.28340458,236.73472592)
\curveto(293.18384885,236.75961486)(293.05318195,236.77828156)(292.89140388,236.79072603)
\curveto(292.72962581,236.81561496)(292.58651444,236.82805943)(292.46206978,236.82805943)
\curveto(291.98918004,236.82805943)(291.534957,236.74094816)(291.09940067,236.56672562)
\curveto(290.66384433,236.40494756)(290.30917703,236.13739152)(290.03539876,235.76405752)
\curveto(289.77406496,235.39072352)(289.64339806,234.88050038)(289.64339806,234.23338811)
\lineto(289.64339806,229.04404547)
\lineto(286.86205974,229.04404547)
\lineto(286.86205974,239.23606374)
\lineto(288.97139685,239.23606374)
\lineto(289.38206425,237.51872733)
\lineto(289.51273116,237.51872733)
\curveto(289.81139836,238.04139493)(290.22206576,238.48939574)(290.74473336,238.86272974)
\curveto(291.26740097,239.23606374)(291.87095761,239.42273074)(292.55540328,239.42273074)
\closepath
}
}
{
\newrgbcolor{curcolor}{0 0 0}
\pscustom[linestyle=none,fillstyle=solid,fillcolor=curcolor]
{
\newpath
\moveto(299.16341604,239.42273074)
\curveto(300.56964078,239.42273074)(301.68342056,239.01828557)(302.50475536,238.20939523)
\curveto(303.32609017,237.41294936)(303.73675757,236.27428065)(303.73675757,234.79338911)
\lineto(303.73675757,233.4493867)
\lineto(297.16607913,233.4493867)
\curveto(297.19096806,232.6653853)(297.4211907,232.04938419)(297.85674703,231.60138339)
\curveto(298.30474784,231.15338259)(298.92074894,230.92938218)(299.70475035,230.92938218)
\curveto(300.35186262,230.92938218)(300.94297479,230.99160452)(301.47808686,231.11604919)
\curveto(302.02564339,231.25293832)(302.5856444,231.45827202)(303.15808987,231.73205029)
\lineto(303.15808987,229.58537978)
\curveto(302.64786673,229.33649044)(302.11897689,229.15604567)(301.57142036,229.04404547)
\curveto(301.02386382,228.9196008)(300.35808485,228.85737847)(299.57408344,228.85737847)
\curveto(298.55363717,228.85737847)(297.65141333,229.04404547)(296.86741193,229.41737947)
\curveto(296.08341052,229.80315794)(295.46740942,230.37560341)(295.01940861,231.13471589)
\curveto(294.57140781,231.90627282)(294.34740741,232.88316346)(294.34740741,234.06538781)
\curveto(294.34740741,235.24761215)(294.54651888,236.23694725)(294.94474181,237.03339313)
\curveto(295.35540922,237.829839)(295.92163245,238.4271734)(296.64341152,238.82539634)
\curveto(297.3651906,239.22361928)(298.2051921,239.42273074)(299.16341604,239.42273074)
\closepath
\moveto(299.18208274,237.44406053)
\curveto(298.6345262,237.44406053)(298.1865254,237.269838)(297.83808033,236.92139293)
\curveto(297.48963526,236.57294786)(297.28430156,236.03161355)(297.22207923,235.29739001)
\lineto(301.12341955,235.29739001)
\curveto(301.11097509,235.90716889)(300.94297479,236.41739202)(300.61941865,236.82805943)
\curveto(300.30830698,237.23872683)(299.82919501,237.44406053)(299.18208274,237.44406053)
\closepath
}
}
{
\newrgbcolor{curcolor}{0 0 0}
\pscustom[linestyle=none,fillstyle=solid,fillcolor=curcolor]
{
\newpath
\moveto(310.17676214,239.44139744)
\curveto(311.54565348,239.44139744)(312.59098869,239.14273024)(313.31276776,238.54539584)
\curveto(314.0469913,237.9605059)(314.41410307,237.05828206)(314.41410307,235.83872432)
\lineto(314.41410307,229.04404547)
\lineto(312.47276626,229.04404547)
\lineto(311.93143195,230.42538128)
\lineto(311.85676515,230.42538128)
\curveto(311.42120882,229.87782474)(310.96076355,229.47960181)(310.47542934,229.23071247)
\curveto(309.99009514,228.98182314)(309.32431617,228.85737847)(308.47809243,228.85737847)
\curveto(307.56964636,228.85737847)(306.81675612,229.11871227)(306.21942171,229.64137988)
\curveto(305.62208731,230.16404748)(305.32342011,230.97916005)(305.32342011,232.08671759)
\curveto(305.32342011,233.1693862)(305.70297634,233.96583207)(306.46208882,234.47605521)
\curveto(307.22120129,234.98627835)(308.35987,235.27250108)(309.87809494,235.33472342)
\lineto(311.65143145,235.39072352)
\lineto(311.65143145,235.83872432)
\curveto(311.65143145,236.37383639)(311.50832008,236.76583709)(311.22209735,237.01472643)
\curveto(310.94831908,237.26361576)(310.56254061,237.38806043)(310.06476194,237.38806043)
\curveto(309.56698327,237.38806043)(309.08164907,237.31339363)(308.60875933,237.16406003)
\curveto(308.13586959,237.02717089)(307.66297986,236.85294836)(307.19009012,236.64139242)
\lineto(306.27542181,238.52672914)
\curveto(306.81053389,238.80050741)(307.41409052,239.01828557)(308.08609173,239.18006364)
\curveto(308.75809293,239.35428618)(309.45498307,239.44139744)(310.17676214,239.44139744)
\closepath
\moveto(311.65143145,233.7667206)
\lineto(310.56876284,233.7293872)
\curveto(309.67276124,233.70449827)(309.0505379,233.5427202)(308.70209283,233.244053)
\curveto(308.35364776,232.9453858)(308.17942523,232.5533851)(308.17942523,232.06805089)
\curveto(308.17942523,231.64493902)(308.3038699,231.34004959)(308.55275923,231.15338259)
\curveto(308.80164857,230.97916005)(309.1252047,230.89204878)(309.52342764,230.89204878)
\curveto(310.12076204,230.89204878)(310.62476294,231.06627132)(311.03543035,231.41471639)
\curveto(311.44609775,231.77560592)(311.65143145,232.27960683)(311.65143145,232.9267191)
\closepath
}
}
{
\newrgbcolor{curcolor}{0 0 0}
\pscustom[linestyle=none,fillstyle=solid,fillcolor=curcolor]
{
\newpath
\moveto(321.56345092,231.07871579)
\curveto(321.87456259,231.07871579)(322.17322979,231.10360472)(322.45945253,231.15338259)
\curveto(322.74567526,231.21560492)(323.031898,231.29649395)(323.31812073,231.39604969)
\lineto(323.31812073,229.32404597)
\curveto(323.01945353,229.18715684)(322.64611953,229.07515664)(322.19811873,228.98804537)
\curveto(321.76256239,228.9009341)(321.28345042,228.85737847)(320.76078282,228.85737847)
\curveto(320.15100394,228.85737847)(319.60344741,228.9569342)(319.1181132,229.15604567)
\curveto(318.64522347,229.35515714)(318.26566723,229.69737998)(317.9794445,230.18271418)
\curveto(317.70566623,230.66804838)(317.56877709,231.35249405)(317.56877709,232.23605119)
\lineto(317.56877709,237.14539333)
\lineto(316.24344138,237.14539333)
\lineto(316.24344138,238.32139544)
\lineto(317.7741108,239.25473044)
\lineto(318.5767789,241.40140096)
\lineto(320.35011541,241.40140096)
\lineto(320.35011541,239.23606374)
\lineto(323.20612053,239.23606374)
\lineto(323.20612053,237.14539333)
\lineto(320.35011541,237.14539333)
\lineto(320.35011541,232.23605119)
\curveto(320.35011541,231.85027272)(320.46211561,231.55782776)(320.68611602,231.35871629)
\curveto(320.91011642,231.17204929)(321.20256139,231.07871579)(321.56345092,231.07871579)
\closepath
}
}
{
\newrgbcolor{curcolor}{0 0 0}
\pscustom[linestyle=none,fillstyle=solid,fillcolor=curcolor]
{
\newpath
\moveto(329.5714662,239.42273074)
\curveto(330.97769094,239.42273074)(332.09147071,239.01828557)(332.91280552,238.20939523)
\curveto(333.73414032,237.41294936)(334.14480773,236.27428065)(334.14480773,234.79338911)
\lineto(334.14480773,233.4493867)
\lineto(327.57412928,233.4493867)
\curveto(327.59901822,232.6653853)(327.82924085,232.04938419)(328.26479719,231.60138339)
\curveto(328.71279799,231.15338259)(329.32879909,230.92938218)(330.1128005,230.92938218)
\curveto(330.75991277,230.92938218)(331.35102494,230.99160452)(331.88613701,231.11604919)
\curveto(332.43369355,231.25293832)(332.99369455,231.45827202)(333.56614002,231.73205029)
\lineto(333.56614002,229.58537978)
\curveto(333.05591689,229.33649044)(332.52702705,229.15604567)(331.97947051,229.04404547)
\curveto(331.43191397,228.9196008)(330.766135,228.85737847)(329.9821336,228.85737847)
\curveto(328.96168732,228.85737847)(328.05946348,229.04404547)(327.27546208,229.41737947)
\curveto(326.49146067,229.80315794)(325.87545957,230.37560341)(325.42745877,231.13471589)
\curveto(324.97945796,231.90627282)(324.75545756,232.88316346)(324.75545756,234.06538781)
\curveto(324.75545756,235.24761215)(324.95456903,236.23694725)(325.35279197,237.03339313)
\curveto(325.76345937,237.829839)(326.32968261,238.4271734)(327.05146168,238.82539634)
\curveto(327.77324075,239.22361928)(328.61324226,239.42273074)(329.5714662,239.42273074)
\closepath
\moveto(329.5901329,237.44406053)
\curveto(329.04257636,237.44406053)(328.59457556,237.269838)(328.24613049,236.92139293)
\curveto(327.89768542,236.57294786)(327.69235172,236.03161355)(327.63012938,235.29739001)
\lineto(331.53146971,235.29739001)
\curveto(331.51902524,235.90716889)(331.35102494,236.41739202)(331.02746881,236.82805943)
\curveto(330.71635714,237.23872683)(330.23724517,237.44406053)(329.5901329,237.44406053)
\closepath
}
}
{
\newrgbcolor{curcolor}{0 0 0}
\pscustom[linestyle=none,fillstyle=solid,fillcolor=curcolor]
{
\newpath
\moveto(340.5101455,242.37206936)
\curveto(342.32703764,242.37206936)(343.66481782,242.04229099)(344.52348602,241.38273426)
\curveto(345.3945987,240.72317752)(345.83015503,239.72139795)(345.83015503,238.37739554)
\curveto(345.83015503,237.76761667)(345.7119326,237.23250459)(345.47548773,236.77205932)
\curveto(345.25148733,236.32405852)(344.94659789,235.93828005)(344.56081942,235.61472392)
\curveto(344.18748542,235.30361225)(343.78304025,235.04850068)(343.34748391,234.84938921)
\lineto(347.26749094,229.04404547)
\lineto(344.13148532,229.04404547)
\lineto(340.9581463,234.15872131)
\lineto(339.44614359,234.15872131)
\lineto(339.44614359,229.04404547)
\lineto(336.62747187,229.04404547)
\lineto(336.62747187,242.37206936)
\closepath
\moveto(340.30481179,240.05739855)
\lineto(339.44614359,240.05739855)
\lineto(339.44614359,236.45472542)
\lineto(340.36081189,236.45472542)
\curveto(341.2941469,236.45472542)(341.95992587,236.61028126)(342.35814881,236.92139293)
\curveto(342.76881621,237.23250459)(342.97414991,237.69294986)(342.97414991,238.30272874)
\curveto(342.97414991,238.93739654)(342.75637174,239.38539734)(342.32081541,239.64673114)
\curveto(341.88525907,239.92050941)(341.21325787,240.05739855)(340.30481179,240.05739855)
\closepath
}
}
{
\newrgbcolor{curcolor}{0 0 0}
\pscustom[linestyle=none,fillstyle=solid,fillcolor=curcolor]
{
\newpath
\moveto(357.98216953,234.15872131)
\curveto(357.98216953,232.46627383)(357.53416873,231.15960482)(356.63816712,230.23871428)
\curveto(355.75460998,229.31782374)(354.54749671,228.85737847)(353.01682729,228.85737847)
\curveto(352.07104782,228.85737847)(351.22482408,229.06271217)(350.47815608,229.47337957)
\curveto(349.74393254,229.88404698)(349.16526484,230.48138138)(348.74215297,231.26538279)
\curveto(348.3190411,232.06182866)(348.10748516,233.02627483)(348.10748516,234.15872131)
\curveto(348.10748516,235.85116879)(348.54926373,237.15161556)(349.43282087,238.06006163)
\curveto(350.31637801,238.96850771)(351.52971352,239.42273074)(353.0728274,239.42273074)
\curveto(354.03105134,239.42273074)(354.87727507,239.21739704)(355.61149861,238.80672964)
\curveto(356.34572215,238.39606224)(356.92438985,237.79872783)(357.34750172,237.01472643)
\curveto(357.77061359,236.23072502)(357.98216953,235.27872331)(357.98216953,234.15872131)
\closepath
\moveto(350.94482358,234.15872131)
\curveto(350.94482358,233.1507195)(351.10660165,232.38538479)(351.43015778,231.86271719)
\curveto(351.76615839,231.35249405)(352.30749269,231.09738249)(353.0541607,231.09738249)
\curveto(353.78838423,231.09738249)(354.31727407,231.35249405)(354.64083021,231.86271719)
\curveto(354.97683081,232.38538479)(355.14483111,233.1507195)(355.14483111,234.15872131)
\curveto(355.14483111,235.16672311)(354.97683081,235.91961335)(354.64083021,236.41739202)
\curveto(354.31727407,236.92761516)(353.782162,237.18272673)(353.03549399,237.18272673)
\curveto(352.30127046,237.18272673)(351.76615839,236.92761516)(351.43015778,236.41739202)
\curveto(351.10660165,235.91961335)(350.94482358,235.16672311)(350.94482358,234.15872131)
\closepath
}
}
{
\newrgbcolor{curcolor}{0 0 0}
\pscustom[linestyle=none,fillstyle=solid,fillcolor=curcolor]
{
\newpath
\moveto(369.53685688,234.15872131)
\curveto(369.53685688,232.46627383)(369.08885608,231.15960482)(368.19285447,230.23871428)
\curveto(367.30929733,229.31782374)(366.10218406,228.85737847)(364.57151465,228.85737847)
\curveto(363.62573518,228.85737847)(362.77951144,229.06271217)(362.03284343,229.47337957)
\curveto(361.29861989,229.88404698)(360.71995219,230.48138138)(360.29684032,231.26538279)
\curveto(359.87372845,232.06182866)(359.66217252,233.02627483)(359.66217252,234.15872131)
\curveto(359.66217252,235.85116879)(360.10395109,237.15161556)(360.98750822,238.06006163)
\curveto(361.87106536,238.96850771)(363.08440087,239.42273074)(364.62751475,239.42273074)
\curveto(365.58573869,239.42273074)(366.43196243,239.21739704)(367.16618597,238.80672964)
\curveto(367.90040951,238.39606224)(368.47907721,237.79872783)(368.90218908,237.01472643)
\curveto(369.32530095,236.23072502)(369.53685688,235.27872331)(369.53685688,234.15872131)
\closepath
\moveto(362.49951094,234.15872131)
\curveto(362.49951094,233.1507195)(362.661289,232.38538479)(362.98484514,231.86271719)
\curveto(363.32084574,231.35249405)(363.86218004,231.09738249)(364.60884805,231.09738249)
\curveto(365.34307159,231.09738249)(365.87196142,231.35249405)(366.19551756,231.86271719)
\curveto(366.53151816,232.38538479)(366.69951846,233.1507195)(366.69951846,234.15872131)
\curveto(366.69951846,235.16672311)(366.53151816,235.91961335)(366.19551756,236.41739202)
\curveto(365.87196142,236.92761516)(365.33684935,237.18272673)(364.59018135,237.18272673)
\curveto(363.85595781,237.18272673)(363.32084574,236.92761516)(362.98484514,236.41739202)
\curveto(362.661289,235.91961335)(362.49951094,235.16672311)(362.49951094,234.15872131)
\closepath
}
}
{
\newrgbcolor{curcolor}{0 0 0}
\pscustom[linestyle=none,fillstyle=solid,fillcolor=curcolor]
{
\newpath
\moveto(383.7982005,239.42273074)
\curveto(384.95553591,239.42273074)(385.82664858,239.12406354)(386.41153852,238.52672914)
\curveto(387.00887292,237.9418392)(387.30754012,236.99605973)(387.30754012,235.68939072)
\lineto(387.30754012,229.04404547)
\lineto(384.5262018,229.04404547)
\lineto(384.5262018,234.99872281)
\curveto(384.5262018,236.46716989)(384.01597867,237.20139343)(382.99553239,237.20139343)
\curveto(382.26130885,237.20139343)(381.73864125,236.94005963)(381.42752958,236.41739202)
\curveto(381.11641791,235.89472442)(380.96086208,235.14183418)(380.96086208,234.15872131)
\lineto(380.96086208,229.04404547)
\lineto(378.17952376,229.04404547)
\lineto(378.17952376,234.99872281)
\curveto(378.17952376,236.46716989)(377.66930062,237.20139343)(376.64885435,237.20139343)
\curveto(375.87729741,237.20139343)(375.34218534,236.90894846)(375.04351814,236.32405852)
\curveto(374.7572954,235.75161305)(374.61418403,234.92405601)(374.61418403,233.8413874)
\lineto(374.61418403,229.04404547)
\lineto(371.83284572,229.04404547)
\lineto(371.83284572,239.23606374)
\lineto(373.96084953,239.23606374)
\lineto(374.33418353,237.92939473)
\lineto(374.48351713,237.92939473)
\curveto(374.7946288,238.45206234)(375.21774067,238.83161857)(375.75285274,239.06806344)
\curveto(376.30040928,239.30450831)(376.86663252,239.42273074)(377.45152245,239.42273074)
\curveto(378.19819046,239.42273074)(378.82663603,239.29828608)(379.33685917,239.04939674)
\curveto(379.85952677,238.81295187)(380.26397194,238.43961787)(380.55019468,237.92939473)
\lineto(380.79286178,237.92939473)
\curveto(381.10397345,238.45206234)(381.53330755,238.83161857)(382.08086409,239.06806344)
\curveto(382.64086509,239.30450831)(383.21331056,239.42273074)(383.7982005,239.42273074)
\closepath
}
}
{
\newrgbcolor{curcolor}{0 0 0}
\pscustom[linestyle=none,fillstyle=solid,fillcolor=curcolor]
{
\newpath
\moveto(389.77154852,230.35071448)
\curveto(389.77154852,230.92315995)(389.92710435,231.32138289)(390.23821602,231.54538329)
\curveto(390.54932769,231.78182816)(390.92888393,231.90005059)(391.37688473,231.90005059)
\curveto(391.81244106,231.90005059)(392.18577507,231.78182816)(392.49688674,231.54538329)
\curveto(392.8079984,231.32138289)(392.96355424,230.92315995)(392.96355424,230.35071448)
\curveto(392.96355424,229.80315794)(392.8079984,229.40493501)(392.49688674,229.15604567)
\curveto(392.18577507,228.9196008)(391.81244106,228.80137837)(391.37688473,228.80137837)
\curveto(390.92888393,228.80137837)(390.54932769,228.9196008)(390.23821602,229.15604567)
\curveto(389.92710435,229.40493501)(389.77154852,229.80315794)(389.77154852,230.35071448)
\closepath
}
}
{
\newrgbcolor{curcolor}{0 0 0}
\pscustom[linestyle=none,fillstyle=solid,fillcolor=curcolor]
{
\newpath
\moveto(399.62756391,228.85737847)
\curveto(398.10933896,228.85737847)(396.93333685,229.27426811)(396.09955758,230.10804738)
\curveto(395.27822278,230.94182665)(394.86755537,232.26716236)(394.86755537,234.08405451)
\curveto(394.86755537,235.32850118)(395.07911131,236.34272522)(395.50222318,237.12672663)
\curveto(395.92533505,237.91072803)(396.51022498,238.48939574)(397.25689299,238.86272974)
\curveto(398.01600546,239.23606374)(398.88711813,239.42273074)(399.87023101,239.42273074)
\curveto(400.56712115,239.42273074)(401.17067778,239.35428618)(401.68090092,239.21739704)
\curveto(402.20356852,239.08050791)(402.65779156,238.91872984)(403.04357003,238.73206284)
\lineto(402.22223522,236.58539232)
\curveto(401.78667889,236.75961486)(401.37601148,236.90272623)(400.99023301,237.01472643)
\curveto(400.61689901,237.12672663)(400.24356501,237.18272673)(399.87023101,237.18272673)
\curveto(398.42667286,237.18272673)(397.70489379,236.15605822)(397.70489379,234.10272121)
\curveto(397.70489379,233.08227493)(397.89156079,232.32938469)(398.2648948,231.84405049)
\curveto(398.65067327,231.35871629)(399.18578534,231.11604919)(399.87023101,231.11604919)
\curveto(400.45512094,231.11604919)(400.97156631,231.19071599)(401.41956712,231.34004959)
\curveto(401.86756792,231.50182766)(402.30312426,231.71960582)(402.72623613,231.99338409)
\lineto(402.72623613,229.62271318)
\curveto(402.30312426,229.34893491)(401.85512345,229.15604567)(401.38223372,229.04404547)
\curveto(400.92178845,228.9196008)(400.33689851,228.85737847)(399.62756391,228.85737847)
\closepath
}
}
{
\newrgbcolor{curcolor}{0 0 0}
\pscustom[linestyle=none,fillstyle=solid,fillcolor=curcolor]
{
\newpath
\moveto(412.19024989,232.06805089)
\curveto(412.19024989,231.03516015)(411.82313812,230.23871428)(411.08891458,229.67871328)
\curveto(410.36713551,229.13115674)(409.2844669,228.85737847)(407.84090876,228.85737847)
\curveto(407.13157416,228.85737847)(406.52179529,228.90715634)(406.01157215,229.00671207)
\curveto(405.50134901,229.09382334)(404.99112588,229.24315694)(404.48090274,229.45471287)
\lineto(404.48090274,231.75071699)
\curveto(405.02845928,231.50182766)(405.61957145,231.29649395)(406.25423925,231.13471589)
\curveto(406.88890705,230.97293782)(407.44890806,230.89204878)(407.93424226,230.89204878)
\curveto(408.46935433,230.89204878)(408.8551328,230.97293782)(409.09157767,231.13471589)
\curveto(409.32802254,231.29649395)(409.44624497,231.50804989)(409.44624497,231.76938369)
\curveto(409.44624497,231.94360623)(409.39646711,232.09916206)(409.29691137,232.23605119)
\curveto(409.2098001,232.37294033)(409.01068864,232.52849616)(408.69957697,232.7027187)
\curveto(408.3884653,232.87694123)(407.9031311,233.10094163)(407.24357436,233.3747199)
\curveto(406.59646209,233.64849817)(406.06757225,233.91605421)(405.65690485,234.17738801)
\curveto(405.25868191,234.45116628)(404.96001471,234.77472241)(404.76090324,235.14805641)
\curveto(404.56179177,235.53383488)(404.46223604,236.01294685)(404.46223604,236.58539232)
\curveto(404.46223604,237.5311718)(404.82934781,238.2405064)(405.56357135,238.71339614)
\curveto(406.29779488,239.18628587)(407.27468552,239.42273074)(408.49424327,239.42273074)
\curveto(409.12891107,239.42273074)(409.73246771,239.36050841)(410.30491318,239.23606374)
\curveto(410.87735865,239.11161907)(411.46847082,238.90628537)(412.07824969,238.62006264)
\lineto(411.23824818,236.62272572)
\curveto(410.74046951,236.83428166)(410.26757978,237.00850419)(409.81957897,237.14539333)
\curveto(409.37157817,237.29472693)(408.91735514,237.36939373)(408.45690987,237.36939373)
\curveto(407.63557506,237.36939373)(407.22490766,237.14539333)(407.22490766,236.69739252)
\curveto(407.22490766,236.53561446)(407.27468552,236.38628086)(407.37424126,236.24939172)
\curveto(407.48624146,236.12494705)(407.69157516,235.98805792)(407.99024236,235.83872432)
\curveto(408.30135403,235.68939072)(408.75557707,235.49027925)(409.35291147,235.24138991)
\curveto(409.93780141,235.00494505)(410.44180231,234.75605571)(410.86491418,234.49472191)
\curveto(411.28802605,234.24583257)(411.61158219,233.92849867)(411.83558259,233.5427202)
\curveto(412.07202746,233.15694173)(412.19024989,232.6653853)(412.19024989,232.06805089)
\closepath
}
}
{
\newrgbcolor{curcolor}{0 0 0}
\pscustom[linestyle=none,fillstyle=solid,fillcolor=curcolor]
{
\newpath
\moveto(421.46759352,232.06805089)
\curveto(421.46759352,231.03516015)(421.10048176,230.23871428)(420.36625822,229.67871328)
\curveto(419.64447915,229.13115674)(418.56181054,228.85737847)(417.11825239,228.85737847)
\curveto(416.40891779,228.85737847)(415.79913892,228.90715634)(415.28891578,229.00671207)
\curveto(414.77869265,229.09382334)(414.26846951,229.24315694)(413.75824637,229.45471287)
\lineto(413.75824637,231.75071699)
\curveto(414.30580291,231.50182766)(414.89691508,231.29649395)(415.53158288,231.13471589)
\curveto(416.16625069,230.97293782)(416.72625169,230.89204878)(417.2115859,230.89204878)
\curveto(417.74669797,230.89204878)(418.13247643,230.97293782)(418.3689213,231.13471589)
\curveto(418.60536617,231.29649395)(418.72358861,231.50804989)(418.72358861,231.76938369)
\curveto(418.72358861,231.94360623)(418.67381074,232.09916206)(418.574255,232.23605119)
\curveto(418.48714374,232.37294033)(418.28803227,232.52849616)(417.9769206,232.7027187)
\curveto(417.66580893,232.87694123)(417.18047473,233.10094163)(416.52091799,233.3747199)
\curveto(415.87380572,233.64849817)(415.34491588,233.91605421)(414.93424848,234.17738801)
\curveto(414.53602554,234.45116628)(414.23735834,234.77472241)(414.03824687,235.14805641)
\curveto(413.83913541,235.53383488)(413.73957967,236.01294685)(413.73957967,236.58539232)
\curveto(413.73957967,237.5311718)(414.10669144,238.2405064)(414.84091498,238.71339614)
\curveto(415.57513852,239.18628587)(416.55202916,239.42273074)(417.7715869,239.42273074)
\curveto(418.4062547,239.42273074)(419.00981134,239.36050841)(419.58225681,239.23606374)
\curveto(420.15470228,239.11161907)(420.74581445,238.90628537)(421.35559332,238.62006264)
\lineto(420.51559182,236.62272572)
\curveto(420.01781315,236.83428166)(419.54492341,237.00850419)(419.09692261,237.14539333)
\curveto(418.6489218,237.29472693)(418.19469877,237.36939373)(417.7342535,237.36939373)
\curveto(416.91291869,237.36939373)(416.50225129,237.14539333)(416.50225129,236.69739252)
\curveto(416.50225129,236.53561446)(416.55202916,236.38628086)(416.65158489,236.24939172)
\curveto(416.76358509,236.12494705)(416.96891879,235.98805792)(417.267586,235.83872432)
\curveto(417.57869766,235.68939072)(418.0329207,235.49027925)(418.6302551,235.24138991)
\curveto(419.21514504,235.00494505)(419.71914595,234.75605571)(420.14225782,234.49472191)
\curveto(420.56536968,234.24583257)(420.88892582,233.92849867)(421.11292622,233.5427202)
\curveto(421.34937109,233.15694173)(421.46759352,232.6653853)(421.46759352,232.06805089)
\closepath
}
}
{
\newrgbcolor{curcolor}{0 0 0}
\pscustom[linestyle=none,fillstyle=solid,fillcolor=curcolor]
{
\newpath
\moveto(219.72469804,207.82442486)
\curveto(218.2064731,207.82442486)(217.03047099,208.24131449)(216.19669171,209.07509377)
\curveto(215.37535691,209.90887304)(214.96468951,211.23420875)(214.96468951,213.05110089)
\curveto(214.96468951,214.29554757)(215.17624544,215.30977161)(215.59935731,216.09377301)
\curveto(216.02246918,216.87777442)(216.60735912,217.45644212)(217.35402712,217.82977613)
\curveto(218.11313959,218.20311013)(218.98425227,218.38977713)(219.96736514,218.38977713)
\curveto(220.66425528,218.38977713)(221.26781192,218.32133256)(221.77803505,218.18444343)
\curveto(222.30070266,218.04755429)(222.75492569,217.88577623)(223.14070416,217.69910923)
\lineto(222.31936936,215.55243871)
\curveto(221.88381302,215.72666124)(221.47314562,215.86977261)(221.08736715,215.98177281)
\curveto(220.71403315,216.09377301)(220.34069914,216.14977311)(219.96736514,216.14977311)
\curveto(218.523807,216.14977311)(217.80202793,215.12310461)(217.80202793,213.06976759)
\curveto(217.80202793,212.04932132)(217.98869493,211.29643108)(218.36202893,210.81109688)
\curveto(218.7478074,210.32576267)(219.28291947,210.08309557)(219.96736514,210.08309557)
\curveto(220.55225508,210.08309557)(221.06870045,210.15776237)(221.51670125,210.30709597)
\curveto(221.96470205,210.46887404)(222.40025839,210.68665221)(222.82337026,210.96043048)
\lineto(222.82337026,208.58975956)
\curveto(222.40025839,208.31598129)(221.95225759,208.12309206)(221.47936785,208.01109186)
\curveto(221.01892258,207.88664719)(220.43403264,207.82442486)(219.72469804,207.82442486)
\closepath
}
}
{
\newrgbcolor{curcolor}{0 0 0}
\pscustom[linestyle=none,fillstyle=solid,fillcolor=curcolor]
{
\newpath
\moveto(234.43405645,213.12576769)
\curveto(234.43405645,211.43332022)(233.98605564,210.12665121)(233.09005404,209.20576067)
\curveto(232.2064969,208.28487013)(230.99938362,207.82442486)(229.46871421,207.82442486)
\curveto(228.52293474,207.82442486)(227.676711,208.02975856)(226.930043,208.44042596)
\curveto(226.19581946,208.85109336)(225.61715175,209.44842777)(225.19403988,210.23242917)
\curveto(224.77092801,211.02887505)(224.55937208,211.99332122)(224.55937208,213.12576769)
\curveto(224.55937208,214.81821517)(225.00115065,216.11866195)(225.88470779,217.02710802)
\curveto(226.76826493,217.93555409)(227.98160044,218.38977713)(229.52471431,218.38977713)
\curveto(230.48293825,218.38977713)(231.32916199,218.18444343)(232.06338553,217.77377603)
\curveto(232.79760907,217.36310862)(233.37627677,216.76577422)(233.79938864,215.98177281)
\curveto(234.22250051,215.19777141)(234.43405645,214.2457697)(234.43405645,213.12576769)
\closepath
\moveto(227.3967105,213.12576769)
\curveto(227.3967105,212.11776589)(227.55848857,211.35243118)(227.8820447,210.82976358)
\curveto(228.2180453,210.31954044)(228.75937961,210.06442887)(229.50604761,210.06442887)
\curveto(230.24027115,210.06442887)(230.76916099,210.31954044)(231.09271712,210.82976358)
\curveto(231.42871773,211.35243118)(231.59671803,212.11776589)(231.59671803,213.12576769)
\curveto(231.59671803,214.1337695)(231.42871773,214.88665974)(231.09271712,215.38443841)
\curveto(230.76916099,215.89466155)(230.23404892,216.14977311)(229.48738091,216.14977311)
\curveto(228.75315737,216.14977311)(228.2180453,215.89466155)(227.8820447,215.38443841)
\curveto(227.55848857,214.88665974)(227.3967105,214.1337695)(227.3967105,213.12576769)
\closepath
}
}
{
\newrgbcolor{curcolor}{0 0 0}
\pscustom[linestyle=none,fillstyle=solid,fillcolor=curcolor]
{
\newpath
\moveto(242.51672804,218.38977713)
\curveto(243.61184111,218.38977713)(244.48917602,218.09110993)(245.14873276,217.49377552)
\curveto(245.8082895,216.90888559)(246.13806787,215.96310611)(246.13806787,214.6564371)
\lineto(246.13806787,208.01109186)
\lineto(243.35672955,208.01109186)
\lineto(243.35672955,213.9657692)
\curveto(243.35672955,214.69999274)(243.22606265,215.24754928)(242.96472884,215.60843881)
\curveto(242.70339504,215.98177281)(242.28650541,216.16843981)(241.71405993,216.16843981)
\curveto(240.8678362,216.16843981)(240.28916849,215.87599485)(239.97805682,215.29110491)
\curveto(239.66694515,214.71865944)(239.51138932,213.8911024)(239.51138932,212.80843379)
\lineto(239.51138932,208.01109186)
\lineto(236.730051,208.01109186)
\lineto(236.730051,218.20311013)
\lineto(238.85805482,218.20311013)
\lineto(239.23138882,216.89644112)
\lineto(239.38072242,216.89644112)
\curveto(239.70427855,217.41910872)(240.14605712,217.79866496)(240.70605813,218.03510983)
\curveto(241.2785036,218.2715547)(241.88206024,218.38977713)(242.51672804,218.38977713)
\closepath
}
}
{
\newrgbcolor{curcolor}{0 0 0}
\pscustom[linestyle=none,fillstyle=solid,fillcolor=curcolor]
{
\newpath
\moveto(254.78074308,218.38977713)
\curveto(255.87585616,218.38977713)(256.75319106,218.09110993)(257.4127478,217.49377552)
\curveto(258.07230454,216.90888559)(258.40208291,215.96310611)(258.40208291,214.6564371)
\lineto(258.40208291,208.01109186)
\lineto(255.62074459,208.01109186)
\lineto(255.62074459,213.9657692)
\curveto(255.62074459,214.69999274)(255.49007769,215.24754928)(255.22874389,215.60843881)
\curveto(254.96741008,215.98177281)(254.55052045,216.16843981)(253.97807498,216.16843981)
\curveto(253.13185124,216.16843981)(252.55318354,215.87599485)(252.24207187,215.29110491)
\curveto(251.9309602,214.71865944)(251.77540436,213.8911024)(251.77540436,212.80843379)
\lineto(251.77540436,208.01109186)
\lineto(248.99406604,208.01109186)
\lineto(248.99406604,218.20311013)
\lineto(251.12206986,218.20311013)
\lineto(251.49540386,216.89644112)
\lineto(251.64473746,216.89644112)
\curveto(251.9682936,217.41910872)(252.41007217,217.79866496)(252.97007317,218.03510983)
\curveto(253.54251864,218.2715547)(254.14607528,218.38977713)(254.78074308,218.38977713)
\closepath
}
}
{
\newrgbcolor{curcolor}{0 0 0}
\pscustom[linestyle=none,fillstyle=solid,fillcolor=curcolor]
{
\newpath
\moveto(265.45808862,218.38977713)
\curveto(266.86431336,218.38977713)(267.97809313,217.98533196)(268.79942794,217.17644162)
\curveto(269.62076274,216.37999575)(270.03143015,215.24132704)(270.03143015,213.7604355)
\lineto(270.03143015,212.41643309)
\lineto(263.4607517,212.41643309)
\curveto(263.48564064,211.63243168)(263.71586327,211.01643058)(264.15141961,210.56842978)
\curveto(264.59942041,210.12042897)(265.21542151,209.89642857)(265.99942292,209.89642857)
\curveto(266.64653519,209.89642857)(267.23764736,209.95865091)(267.77275943,210.08309557)
\curveto(268.32031597,210.21998471)(268.88031697,210.42531841)(269.45276244,210.69909668)
\lineto(269.45276244,208.55242616)
\curveto(268.94253931,208.30353683)(268.41364947,208.12309206)(267.86609293,208.01109186)
\curveto(267.3185364,207.88664719)(266.65275742,207.82442486)(265.86875602,207.82442486)
\curveto(264.84830975,207.82442486)(263.94608591,208.01109186)(263.1620845,208.38442586)
\curveto(262.37808309,208.77020433)(261.76208199,209.3426498)(261.31408119,210.10176227)
\curveto(260.86608038,210.87331921)(260.64207998,211.85020985)(260.64207998,213.03243419)
\curveto(260.64207998,214.21465853)(260.84119145,215.20399364)(261.23941439,216.00043951)
\curveto(261.65008179,216.79688539)(262.21630503,217.39421979)(262.9380841,217.79244273)
\curveto(263.65986317,218.19066566)(264.49986468,218.38977713)(265.45808862,218.38977713)
\closepath
\moveto(265.47675532,216.41110692)
\curveto(264.92919878,216.41110692)(264.48119798,216.23688438)(264.13275291,215.88843931)
\curveto(263.78430784,215.53999424)(263.57897414,214.99865994)(263.5167518,214.2644364)
\lineto(267.41809213,214.2644364)
\curveto(267.40564766,214.87421527)(267.23764736,215.38443841)(266.91409123,215.79510581)
\curveto(266.60297956,216.20577321)(266.12386759,216.41110692)(265.47675532,216.41110692)
\closepath
}
}
{
\newrgbcolor{curcolor}{0 0 0}
\pscustom[linestyle=none,fillstyle=solid,fillcolor=curcolor]
{
\newpath
\moveto(276.43410132,207.82442486)
\curveto(274.91587637,207.82442486)(273.73987426,208.24131449)(272.90609499,209.07509377)
\curveto(272.08476019,209.90887304)(271.67409278,211.23420875)(271.67409278,213.05110089)
\curveto(271.67409278,214.29554757)(271.88564872,215.30977161)(272.30876059,216.09377301)
\curveto(272.73187246,216.87777442)(273.31676239,217.45644212)(274.0634304,217.82977613)
\curveto(274.82254287,218.20311013)(275.69365554,218.38977713)(276.67676842,218.38977713)
\curveto(277.37365856,218.38977713)(277.97721519,218.32133256)(278.48743833,218.18444343)
\curveto(279.01010593,218.04755429)(279.46432897,217.88577623)(279.85010744,217.69910923)
\lineto(279.02877263,215.55243871)
\curveto(278.5932163,215.72666124)(278.18254889,215.86977261)(277.79677043,215.98177281)
\curveto(277.42343642,216.09377301)(277.05010242,216.14977311)(276.67676842,216.14977311)
\curveto(275.23321027,216.14977311)(274.5114312,215.12310461)(274.5114312,213.06976759)
\curveto(274.5114312,212.04932132)(274.6980982,211.29643108)(275.07143221,210.81109688)
\curveto(275.45721068,210.32576267)(275.99232275,210.08309557)(276.67676842,210.08309557)
\curveto(277.26165836,210.08309557)(277.77810373,210.15776237)(278.22610453,210.30709597)
\curveto(278.67410533,210.46887404)(279.10966167,210.68665221)(279.53277354,210.96043048)
\lineto(279.53277354,208.58975956)
\curveto(279.10966167,208.31598129)(278.66166086,208.12309206)(278.18877113,208.01109186)
\curveto(277.72832586,207.88664719)(277.14343592,207.82442486)(276.43410132,207.82442486)
\closepath
}
}
{
\newrgbcolor{curcolor}{0 0 0}
\pscustom[linestyle=none,fillstyle=solid,fillcolor=curcolor]
{
\newpath
\moveto(286.17811558,210.04576217)
\curveto(286.48922725,210.04576217)(286.78789445,210.07065111)(287.07411719,210.12042897)
\curveto(287.36033992,210.18265131)(287.64656266,210.26354034)(287.93278539,210.36309607)
\lineto(287.93278539,208.29109236)
\curveto(287.63411819,208.15420323)(287.26078419,208.04220303)(286.81278339,207.95509176)
\curveto(286.37722705,207.86798049)(285.89811508,207.82442486)(285.37544748,207.82442486)
\curveto(284.76566861,207.82442486)(284.21811207,207.92398059)(283.73277787,208.12309206)
\curveto(283.25988813,208.32220353)(282.88033189,208.66442636)(282.59410916,209.14976057)
\curveto(282.32033089,209.63509477)(282.18344175,210.31954044)(282.18344175,211.20309758)
\lineto(282.18344175,216.11243971)
\lineto(280.85810605,216.11243971)
\lineto(280.85810605,217.28844182)
\lineto(282.38877546,218.22177683)
\lineto(283.19144356,220.36844734)
\lineto(284.96478007,220.36844734)
\lineto(284.96478007,218.20311013)
\lineto(287.82078519,218.20311013)
\lineto(287.82078519,216.11243971)
\lineto(284.96478007,216.11243971)
\lineto(284.96478007,211.20309758)
\curveto(284.96478007,210.81731911)(285.07678027,210.52487414)(285.30078068,210.32576267)
\curveto(285.52478108,210.13909567)(285.81722605,210.04576217)(286.17811558,210.04576217)
\closepath
}
}
{
\newrgbcolor{curcolor}{0 0 0}
\pscustom[linestyle=none,fillstyle=solid,fillcolor=curcolor]
{
\newpath
\moveto(294.09279736,221.33911575)
\curveto(295.9096895,221.33911575)(297.24746968,221.00933738)(298.10613788,220.34978064)
\curveto(298.97725056,219.69022391)(299.41280689,218.68844433)(299.41280689,217.34444192)
\curveto(299.41280689,216.73466305)(299.29458446,216.19955098)(299.05813959,215.73910571)
\curveto(298.83413919,215.29110491)(298.52924975,214.90532644)(298.14347128,214.5817703)
\curveto(297.77013728,214.27065863)(297.36569211,214.01554707)(296.93013578,213.8164356)
\lineto(300.8501428,208.01109186)
\lineto(297.71413718,208.01109186)
\lineto(294.54079816,213.12576769)
\lineto(293.02879545,213.12576769)
\lineto(293.02879545,208.01109186)
\lineto(290.21012373,208.01109186)
\lineto(290.21012373,221.33911575)
\closepath
\moveto(293.88746365,219.02444493)
\lineto(293.02879545,219.02444493)
\lineto(293.02879545,215.42177181)
\lineto(293.94346375,215.42177181)
\curveto(294.87679876,215.42177181)(295.54257773,215.57732764)(295.94080067,215.88843931)
\curveto(296.35146807,216.19955098)(296.55680177,216.65999625)(296.55680177,217.26977512)
\curveto(296.55680177,217.90444293)(296.3390236,218.35244373)(295.90346727,218.61377753)
\curveto(295.46791093,218.8875558)(294.79590973,219.02444493)(293.88746365,219.02444493)
\closepath
}
}
{
\newrgbcolor{curcolor}{0 0 0}
\pscustom[linestyle=none,fillstyle=solid,fillcolor=curcolor]
{
\newpath
\moveto(311.56482139,213.12576769)
\curveto(311.56482139,211.43332022)(311.11682059,210.12665121)(310.22081898,209.20576067)
\curveto(309.33726184,208.28487013)(308.13014857,207.82442486)(306.59947916,207.82442486)
\curveto(305.65369968,207.82442486)(304.80747594,208.02975856)(304.06080794,208.44042596)
\curveto(303.3265844,208.85109336)(302.7479167,209.44842777)(302.32480483,210.23242917)
\curveto(301.90169296,211.02887505)(301.69013702,211.99332122)(301.69013702,213.12576769)
\curveto(301.69013702,214.81821517)(302.13191559,216.11866195)(303.01547273,217.02710802)
\curveto(303.89902987,217.93555409)(305.11236538,218.38977713)(306.65547926,218.38977713)
\curveto(307.6137032,218.38977713)(308.45992694,218.18444343)(309.19415047,217.77377603)
\curveto(309.92837401,217.36310862)(310.50704172,216.76577422)(310.93015359,215.98177281)
\curveto(311.35326545,215.19777141)(311.56482139,214.2457697)(311.56482139,213.12576769)
\closepath
\moveto(304.52747544,213.12576769)
\curveto(304.52747544,212.11776589)(304.68925351,211.35243118)(305.01280964,210.82976358)
\curveto(305.34881025,210.31954044)(305.89014455,210.06442887)(306.63681256,210.06442887)
\curveto(307.37103609,210.06442887)(307.89992593,210.31954044)(308.22348207,210.82976358)
\curveto(308.55948267,211.35243118)(308.72748297,212.11776589)(308.72748297,213.12576769)
\curveto(308.72748297,214.1337695)(308.55948267,214.88665974)(308.22348207,215.38443841)
\curveto(307.89992593,215.89466155)(307.36481386,216.14977311)(306.61814586,216.14977311)
\curveto(305.88392232,216.14977311)(305.34881025,215.89466155)(305.01280964,215.38443841)
\curveto(304.68925351,214.88665974)(304.52747544,214.1337695)(304.52747544,213.12576769)
\closepath
}
}
{
\newrgbcolor{curcolor}{0 0 0}
\pscustom[linestyle=none,fillstyle=solid,fillcolor=curcolor]
{
\newpath
\moveto(323.11950111,213.12576769)
\curveto(323.11950111,211.43332022)(322.67150031,210.12665121)(321.77549871,209.20576067)
\curveto(320.89194157,208.28487013)(319.68482829,207.82442486)(318.15415888,207.82442486)
\curveto(317.20837941,207.82442486)(316.36215567,208.02975856)(315.61548766,208.44042596)
\curveto(314.88126412,208.85109336)(314.30259642,209.44842777)(313.87948455,210.23242917)
\curveto(313.45637268,211.02887505)(313.24481675,211.99332122)(313.24481675,213.12576769)
\curveto(313.24481675,214.81821517)(313.68659532,216.11866195)(314.57015246,217.02710802)
\curveto(315.4537096,217.93555409)(316.6670451,218.38977713)(318.21015898,218.38977713)
\curveto(319.16838292,218.38977713)(320.01460666,218.18444343)(320.7488302,217.77377603)
\curveto(321.48305374,217.36310862)(322.06172144,216.76577422)(322.48483331,215.98177281)
\curveto(322.90794518,215.19777141)(323.11950111,214.2457697)(323.11950111,213.12576769)
\closepath
\moveto(316.08215517,213.12576769)
\curveto(316.08215517,212.11776589)(316.24393323,211.35243118)(316.56748937,210.82976358)
\curveto(316.90348997,210.31954044)(317.44482428,210.06442887)(318.19149228,210.06442887)
\curveto(318.92571582,210.06442887)(319.45460566,210.31954044)(319.77816179,210.82976358)
\curveto(320.11416239,211.35243118)(320.2821627,212.11776589)(320.2821627,213.12576769)
\curveto(320.2821627,214.1337695)(320.11416239,214.88665974)(319.77816179,215.38443841)
\curveto(319.45460566,215.89466155)(318.91949359,216.14977311)(318.17282558,216.14977311)
\curveto(317.43860204,216.14977311)(316.90348997,215.89466155)(316.56748937,215.38443841)
\curveto(316.24393323,214.88665974)(316.08215517,214.1337695)(316.08215517,213.12576769)
\closepath
}
}
{
\newrgbcolor{curcolor}{0 0 0}
\pscustom[linestyle=none,fillstyle=solid,fillcolor=curcolor]
{
\newpath
\moveto(337.38085236,218.38977713)
\curveto(338.53818777,218.38977713)(339.40930044,218.09110993)(339.99419038,217.49377552)
\curveto(340.59152478,216.90888559)(340.89019198,215.96310611)(340.89019198,214.6564371)
\lineto(340.89019198,208.01109186)
\lineto(338.10885366,208.01109186)
\lineto(338.10885366,213.9657692)
\curveto(338.10885366,215.43421628)(337.59863053,216.16843981)(336.57818425,216.16843981)
\curveto(335.84396071,216.16843981)(335.32129311,215.90710601)(335.01018144,215.38443841)
\curveto(334.69906977,214.86177081)(334.54351394,214.10888057)(334.54351394,213.12576769)
\lineto(334.54351394,208.01109186)
\lineto(331.76217562,208.01109186)
\lineto(331.76217562,213.9657692)
\curveto(331.76217562,215.43421628)(331.25195248,216.16843981)(330.23150621,216.16843981)
\curveto(329.45994927,216.16843981)(328.9248372,215.87599485)(328.62617,215.29110491)
\curveto(328.33994726,214.71865944)(328.1968359,213.8911024)(328.1968359,212.80843379)
\lineto(328.1968359,208.01109186)
\lineto(325.41549758,208.01109186)
\lineto(325.41549758,218.20311013)
\lineto(327.54350139,218.20311013)
\lineto(327.91683539,216.89644112)
\lineto(328.06616899,216.89644112)
\curveto(328.37728066,217.41910872)(328.80039253,217.79866496)(329.3355046,218.03510983)
\curveto(329.88306114,218.2715547)(330.44928438,218.38977713)(331.03417431,218.38977713)
\curveto(331.78084232,218.38977713)(332.40928789,218.26533246)(332.91951103,218.01644313)
\curveto(333.44217863,217.77999826)(333.8466238,217.40666426)(334.13284654,216.89644112)
\lineto(334.37551364,216.89644112)
\curveto(334.68662531,217.41910872)(335.11595941,217.79866496)(335.66351595,218.03510983)
\curveto(336.22351695,218.2715547)(336.79596242,218.38977713)(337.38085236,218.38977713)
\closepath
}
}
{
\newrgbcolor{curcolor}{0 0 0}
\pscustom[linestyle=none,fillstyle=solid,fillcolor=curcolor]
{
\newpath
\moveto(343.35420038,209.31776087)
\curveto(343.35420038,209.89020634)(343.50975621,210.28842927)(343.82086788,210.51242968)
\curveto(344.13197955,210.74887454)(344.51153579,210.86709698)(344.95953659,210.86709698)
\curveto(345.39509293,210.86709698)(345.76842693,210.74887454)(346.0795386,210.51242968)
\curveto(346.39065027,210.28842927)(346.5462061,209.89020634)(346.5462061,209.31776087)
\curveto(346.5462061,208.77020433)(346.39065027,208.37198139)(346.0795386,208.12309206)
\curveto(345.76842693,207.88664719)(345.39509293,207.76842476)(344.95953659,207.76842476)
\curveto(344.51153579,207.76842476)(344.13197955,207.88664719)(343.82086788,208.12309206)
\curveto(343.50975621,208.37198139)(343.35420038,208.77020433)(343.35420038,209.31776087)
\closepath
}
}
{
\newrgbcolor{curcolor}{0 0 0}
\pscustom[linestyle=none,fillstyle=solid,fillcolor=curcolor]
{
\newpath
\moveto(353.21021577,207.82442486)
\curveto(351.69199082,207.82442486)(350.51598871,208.24131449)(349.68220944,209.07509377)
\curveto(348.86087464,209.90887304)(348.45020723,211.23420875)(348.45020723,213.05110089)
\curveto(348.45020723,214.29554757)(348.66176317,215.30977161)(349.08487504,216.09377301)
\curveto(349.50798691,216.87777442)(350.09287684,217.45644212)(350.83954485,217.82977613)
\curveto(351.59865732,218.20311013)(352.46976999,218.38977713)(353.45288287,218.38977713)
\curveto(354.14977301,218.38977713)(354.75332964,218.32133256)(355.26355278,218.18444343)
\curveto(355.78622038,218.04755429)(356.24044342,217.88577623)(356.62622189,217.69910923)
\lineto(355.80488708,215.55243871)
\curveto(355.36933075,215.72666124)(354.95866334,215.86977261)(354.57288488,215.98177281)
\curveto(354.19955087,216.09377301)(353.82621687,216.14977311)(353.45288287,216.14977311)
\curveto(352.00932472,216.14977311)(351.28754565,215.12310461)(351.28754565,213.06976759)
\curveto(351.28754565,212.04932132)(351.47421265,211.29643108)(351.84754666,210.81109688)
\curveto(352.23332513,210.32576267)(352.7684372,210.08309557)(353.45288287,210.08309557)
\curveto(354.03777281,210.08309557)(354.55421818,210.15776237)(355.00221898,210.30709597)
\curveto(355.45021978,210.46887404)(355.88577612,210.68665221)(356.30888799,210.96043048)
\lineto(356.30888799,208.58975956)
\curveto(355.88577612,208.31598129)(355.43777531,208.12309206)(354.96488558,208.01109186)
\curveto(354.50444031,207.88664719)(353.91955037,207.82442486)(353.21021577,207.82442486)
\closepath
}
}
{
\newrgbcolor{curcolor}{0 0 0}
\pscustom[linestyle=none,fillstyle=solid,fillcolor=curcolor]
{
\newpath
\moveto(365.77290175,211.03509728)
\curveto(365.77290175,210.00220654)(365.40578998,209.20576067)(364.67156644,208.64575966)
\curveto(363.94978737,208.09820313)(362.86711877,207.82442486)(361.42356062,207.82442486)
\curveto(360.71422602,207.82442486)(360.10444715,207.87420272)(359.59422401,207.97375846)
\curveto(359.08400087,208.06086973)(358.57377774,208.21020333)(358.0635546,208.42175926)
\lineto(358.0635546,210.71776338)
\curveto(358.61111114,210.46887404)(359.20222331,210.26354034)(359.83689111,210.10176227)
\curveto(360.47155892,209.93998421)(361.03155992,209.85909517)(361.51689412,209.85909517)
\curveto(362.05200619,209.85909517)(362.43778466,209.93998421)(362.67422953,210.10176227)
\curveto(362.9106744,210.26354034)(363.02889683,210.47509628)(363.02889683,210.73643008)
\curveto(363.02889683,210.91065261)(362.97911897,211.06620845)(362.87956323,211.20309758)
\curveto(362.79245196,211.33998671)(362.5933405,211.49554255)(362.28222883,211.66976508)
\curveto(361.97111716,211.84398762)(361.48578296,212.06798802)(360.82622622,212.34176629)
\curveto(360.17911395,212.61554456)(359.65022411,212.88310059)(359.23955671,213.14443439)
\curveto(358.84133377,213.41821266)(358.54266657,213.7417688)(358.3435551,214.1151028)
\curveto(358.14444363,214.50088127)(358.0448879,214.97999324)(358.0448879,215.55243871)
\curveto(358.0448879,216.49821818)(358.41199967,217.20755279)(359.14622321,217.68044253)
\curveto(359.88044674,218.15333226)(360.85733738,218.38977713)(362.07689513,218.38977713)
\curveto(362.71156293,218.38977713)(363.31511957,218.3275548)(363.88756504,218.20311013)
\curveto(364.46001051,218.07866546)(365.05112268,217.87333176)(365.66090155,217.58710902)
\lineto(364.82090005,215.58977211)
\curveto(364.32312138,215.80132805)(363.85023164,215.97555058)(363.40223084,216.11243971)
\curveto(362.95423003,216.26177332)(362.500007,216.33644012)(362.03956173,216.33644012)
\curveto(361.21822692,216.33644012)(360.80755952,216.11243971)(360.80755952,215.66443891)
\curveto(360.80755952,215.50266084)(360.85733738,215.35332724)(360.95689312,215.21643811)
\curveto(361.06889332,215.09199344)(361.27422702,214.95510431)(361.57289422,214.80577071)
\curveto(361.88400589,214.6564371)(362.33822893,214.45732564)(362.93556333,214.2084363)
\curveto(363.52045327,213.97199143)(364.02445417,213.7231021)(364.44756604,213.4617683)
\curveto(364.87067791,213.21287896)(365.19423405,212.89554506)(365.41823445,212.50976659)
\curveto(365.65467932,212.12398812)(365.77290175,211.63243168)(365.77290175,211.03509728)
\closepath
}
}
{
\newrgbcolor{curcolor}{0 0 0}
\pscustom[linestyle=none,fillstyle=solid,fillcolor=curcolor]
{
\newpath
\moveto(375.05024539,211.03509728)
\curveto(375.05024539,210.00220654)(374.68313362,209.20576067)(373.94891008,208.64575966)
\curveto(373.22713101,208.09820313)(372.1444624,207.82442486)(370.70090426,207.82442486)
\curveto(369.99156965,207.82442486)(369.38179078,207.87420272)(368.87156764,207.97375846)
\curveto(368.36134451,208.06086973)(367.85112137,208.21020333)(367.34089823,208.42175926)
\lineto(367.34089823,210.71776338)
\curveto(367.88845477,210.46887404)(368.47956694,210.26354034)(369.11423474,210.10176227)
\curveto(369.74890255,209.93998421)(370.30890355,209.85909517)(370.79423776,209.85909517)
\curveto(371.32934983,209.85909517)(371.7151283,209.93998421)(371.95157316,210.10176227)
\curveto(372.18801803,210.26354034)(372.30624047,210.47509628)(372.30624047,210.73643008)
\curveto(372.30624047,210.91065261)(372.2564626,211.06620845)(372.15690687,211.20309758)
\curveto(372.0697956,211.33998671)(371.87068413,211.49554255)(371.55957246,211.66976508)
\curveto(371.24846079,211.84398762)(370.76312659,212.06798802)(370.10356985,212.34176629)
\curveto(369.45645758,212.61554456)(368.92756774,212.88310059)(368.51690034,213.14443439)
\curveto(368.1186774,213.41821266)(367.8200102,213.7417688)(367.62089873,214.1151028)
\curveto(367.42178727,214.50088127)(367.32223153,214.97999324)(367.32223153,215.55243871)
\curveto(367.32223153,216.49821818)(367.6893433,217.20755279)(368.42356684,217.68044253)
\curveto(369.15779038,218.15333226)(370.13468102,218.38977713)(371.35423876,218.38977713)
\curveto(371.98890656,218.38977713)(372.5924632,218.3275548)(373.16490867,218.20311013)
\curveto(373.73735414,218.07866546)(374.32846631,217.87333176)(374.93824518,217.58710902)
\lineto(374.09824368,215.58977211)
\curveto(373.60046501,215.80132805)(373.12757527,215.97555058)(372.67957447,216.11243971)
\curveto(372.23157367,216.26177332)(371.77735063,216.33644012)(371.31690536,216.33644012)
\curveto(370.49557055,216.33644012)(370.08490315,216.11243971)(370.08490315,215.66443891)
\curveto(370.08490315,215.50266084)(370.13468102,215.35332724)(370.23423675,215.21643811)
\curveto(370.34623695,215.09199344)(370.55157065,214.95510431)(370.85023786,214.80577071)
\curveto(371.16134953,214.6564371)(371.61557256,214.45732564)(372.21290697,214.2084363)
\curveto(372.7977969,213.97199143)(373.30179781,213.7231021)(373.72490968,213.4617683)
\curveto(374.14802155,213.21287896)(374.47157768,212.89554506)(374.69557808,212.50976659)
\curveto(374.93202295,212.12398812)(375.05024539,211.63243168)(375.05024539,211.03509728)
\closepath
}
}
{
\newrgbcolor{curcolor}{0 0 0}
\pscustom[linestyle=none,fillstyle=solid,fillcolor=curcolor]
{
\newpath
\moveto(219.78586991,186.51160363)
\curveto(218.26764497,186.51160363)(217.09164286,186.92849327)(216.25786359,187.76227254)
\curveto(215.43652878,188.59605181)(215.02586138,189.92138752)(215.02586138,191.73827967)
\curveto(215.02586138,192.98272634)(215.23741732,193.99695038)(215.66052918,194.78095179)
\curveto(216.08364105,195.56495319)(216.66853099,196.1436209)(217.415199,196.5169549)
\curveto(218.17431147,196.8902889)(219.04542414,197.0769559)(220.02853701,197.0769559)
\curveto(220.72542715,197.0769559)(221.32898379,197.00851134)(221.83920693,196.8716222)
\curveto(222.36187453,196.73473307)(222.81609757,196.572955)(223.20187604,196.386288)
\lineto(222.38054123,194.23961748)
\curveto(221.94498489,194.41384002)(221.53431749,194.55695139)(221.14853902,194.66895159)
\curveto(220.77520502,194.78095179)(220.40187102,194.83695189)(220.02853701,194.83695189)
\curveto(218.58497887,194.83695189)(217.8631998,193.81028338)(217.8631998,191.75694637)
\curveto(217.8631998,190.73650009)(218.0498668,189.98360985)(218.4232008,189.49827565)
\curveto(218.80897927,189.01294145)(219.34409134,188.77027435)(220.02853701,188.77027435)
\curveto(220.61342695,188.77027435)(221.12987232,188.84494115)(221.57787313,188.99427475)
\curveto(222.02587393,189.15605282)(222.46143026,189.37383098)(222.88454213,189.64760925)
\lineto(222.88454213,187.27693834)
\curveto(222.46143026,187.00316007)(222.01342946,186.81027083)(221.54053973,186.69827063)
\curveto(221.08009446,186.57382596)(220.49520452,186.51160363)(219.78586991,186.51160363)
\closepath
}
}
{
\newrgbcolor{curcolor}{0 0 0}
\pscustom[linestyle=none,fillstyle=solid,fillcolor=curcolor]
{
\newpath
\moveto(230.9298886,197.0769559)
\curveto(231.06677773,197.0769559)(231.2285558,197.07073367)(231.4152228,197.0582892)
\curveto(231.6018898,197.04584474)(231.7512234,197.02717804)(231.8632236,197.0022891)
\lineto(231.6578899,194.38895108)
\curveto(231.55833417,194.41384002)(231.42766727,194.43250672)(231.2658892,194.44495118)
\curveto(231.10411113,194.46984012)(230.96099976,194.48228459)(230.8365551,194.48228459)
\curveto(230.36366536,194.48228459)(229.90944232,194.39517332)(229.47388599,194.22095078)
\curveto(229.03832965,194.05917272)(228.68366235,193.79161668)(228.40988408,193.41828268)
\curveto(228.14855028,193.04494868)(228.01788338,192.53472554)(228.01788338,191.88761327)
\lineto(228.01788338,186.69827063)
\lineto(225.23654506,186.69827063)
\lineto(225.23654506,196.8902889)
\lineto(227.34588217,196.8902889)
\lineto(227.75654957,195.17295249)
\lineto(227.88721648,195.17295249)
\curveto(228.18588368,195.69562009)(228.59655108,196.1436209)(229.11921868,196.5169549)
\curveto(229.64188629,196.8902889)(230.24544292,197.0769559)(230.9298886,197.0769559)
\closepath
}
}
{
\newrgbcolor{curcolor}{0 0 0}
\pscustom[linestyle=none,fillstyle=solid,fillcolor=curcolor]
{
\newpath
\moveto(237.53789945,197.0769559)
\curveto(238.9441242,197.0769559)(240.05790397,196.67251073)(240.87923878,195.86362039)
\curveto(241.70057358,195.06717452)(242.11124098,193.92850581)(242.11124098,192.44761427)
\lineto(242.11124098,191.10361186)
\lineto(235.54056254,191.10361186)
\curveto(235.56545147,190.31961046)(235.79567411,189.70360935)(236.23123044,189.25560855)
\curveto(236.67923125,188.80760775)(237.29523235,188.58360734)(238.07923376,188.58360734)
\curveto(238.72634603,188.58360734)(239.3174582,188.64582968)(239.85257027,188.77027435)
\curveto(240.40012681,188.90716348)(240.96012781,189.11249718)(241.53257328,189.38627545)
\lineto(241.53257328,187.23960494)
\curveto(241.02235014,186.9907156)(240.49346031,186.81027083)(239.94590377,186.69827063)
\curveto(239.39834723,186.57382596)(238.73256826,186.51160363)(237.94856686,186.51160363)
\curveto(236.92812058,186.51160363)(236.02589674,186.69827063)(235.24189534,187.07160463)
\curveto(234.45789393,187.4573831)(233.84189283,188.02982857)(233.39389203,188.78894105)
\curveto(232.94589122,189.56049798)(232.72189082,190.53738862)(232.72189082,191.71961297)
\curveto(232.72189082,192.90183731)(232.92100229,193.89117241)(233.31922522,194.68761829)
\curveto(233.72989263,195.48406416)(234.29611586,196.08139856)(235.01789494,196.4796215)
\curveto(235.73967401,196.87784443)(236.57967551,197.0769559)(237.53789945,197.0769559)
\closepath
\moveto(237.55656615,195.09828569)
\curveto(237.00900962,195.09828569)(236.56100881,194.92406315)(236.21256374,194.57561809)
\curveto(235.86411868,194.22717302)(235.65878497,193.68583871)(235.59656264,192.95161517)
\lineto(239.49790297,192.95161517)
\curveto(239.4854585,193.56139405)(239.3174582,194.07161718)(238.99390206,194.48228459)
\curveto(238.68279039,194.89295199)(238.20367842,195.09828569)(237.55656615,195.09828569)
\closepath
}
}
{
\newrgbcolor{curcolor}{0 0 0}
\pscustom[linestyle=none,fillstyle=solid,fillcolor=curcolor]
{
\newpath
\moveto(248.55124555,197.0956226)
\curveto(249.9201369,197.0956226)(250.9654721,196.7969554)(251.68725118,196.199621)
\curveto(252.42147471,195.61473106)(252.78858648,194.71250722)(252.78858648,193.49294948)
\lineto(252.78858648,186.69827063)
\lineto(250.84724967,186.69827063)
\lineto(250.30591537,188.07960644)
\lineto(250.23124857,188.07960644)
\curveto(249.79569223,187.5320499)(249.33524696,187.13382697)(248.84991276,186.88493763)
\curveto(248.36457855,186.6360483)(247.69879958,186.51160363)(246.85257584,186.51160363)
\curveto(245.94412977,186.51160363)(245.19123953,186.77293743)(244.59390513,187.29560504)
\curveto(243.99657072,187.81827264)(243.69790352,188.63338521)(243.69790352,189.74094275)
\curveto(243.69790352,190.82361136)(244.07745976,191.62005723)(244.83657223,192.13028037)
\curveto(245.5956847,192.64050351)(246.73435341,192.92672624)(248.25257835,192.98894857)
\lineto(250.02591486,193.04494868)
\lineto(250.02591486,193.49294948)
\curveto(250.02591486,194.02806155)(249.8828035,194.42006225)(249.59658076,194.66895159)
\curveto(249.32280249,194.91784092)(248.93702402,195.04228559)(248.43924535,195.04228559)
\curveto(247.94146668,195.04228559)(247.45613248,194.96761879)(246.98324274,194.81828519)
\curveto(246.51035301,194.68139605)(246.03746327,194.50717352)(245.56457353,194.29561758)
\lineto(244.64990523,196.1809543)
\curveto(245.1850173,196.45473257)(245.78857394,196.67251073)(246.46057514,196.8342888)
\curveto(247.13257634,197.00851134)(247.82946648,197.0956226)(248.55124555,197.0956226)
\closepath
\moveto(250.02591486,191.42094576)
\lineto(248.94324626,191.38361236)
\curveto(248.04724465,191.35872343)(247.42502131,191.19694536)(247.07657624,190.89827816)
\curveto(246.72813117,190.59961096)(246.55390864,190.20761026)(246.55390864,189.72227605)
\curveto(246.55390864,189.29916418)(246.67835331,188.99427475)(246.92724264,188.80760775)
\curveto(247.17613198,188.63338521)(247.49968811,188.54627394)(247.89791105,188.54627394)
\curveto(248.49524545,188.54627394)(248.99924636,188.72049648)(249.40991376,189.06894155)
\curveto(249.82058116,189.42983108)(250.02591486,189.93383199)(250.02591486,190.58094426)
\closepath
}
}
{
\newrgbcolor{curcolor}{0 0 0}
\pscustom[linestyle=none,fillstyle=solid,fillcolor=curcolor]
{
\newpath
\moveto(259.93793433,188.73294095)
\curveto(260.249046,188.73294095)(260.5477132,188.75782988)(260.83393594,188.80760775)
\curveto(261.12015867,188.86983008)(261.40638141,188.95071911)(261.69260415,189.05027485)
\lineto(261.69260415,186.97827113)
\curveto(261.39393694,186.841382)(261.02060294,186.7293818)(260.57260214,186.64227053)
\curveto(260.1370458,186.55515926)(259.65793383,186.51160363)(259.13526623,186.51160363)
\curveto(258.52548736,186.51160363)(257.97793082,186.61115936)(257.49259662,186.81027083)
\curveto(257.01970688,187.0093823)(256.64015064,187.35160514)(256.35392791,187.83693934)
\curveto(256.08014964,188.32227354)(255.94326051,189.00671921)(255.94326051,189.89027635)
\lineto(255.94326051,194.79961849)
\lineto(254.6179248,194.79961849)
\lineto(254.6179248,195.9756206)
\lineto(256.14859421,196.9089556)
\lineto(256.95126231,199.05562612)
\lineto(258.72459883,199.05562612)
\lineto(258.72459883,196.8902889)
\lineto(261.58060394,196.8902889)
\lineto(261.58060394,194.79961849)
\lineto(258.72459883,194.79961849)
\lineto(258.72459883,189.89027635)
\curveto(258.72459883,189.50449788)(258.83659903,189.21205292)(259.06059943,189.01294145)
\curveto(259.28459983,188.82627445)(259.5770448,188.73294095)(259.93793433,188.73294095)
\closepath
}
}
{
\newrgbcolor{curcolor}{0 0 0}
\pscustom[linestyle=none,fillstyle=solid,fillcolor=curcolor]
{
\newpath
\moveto(267.94594961,197.0769559)
\curveto(269.35217435,197.0769559)(270.46595412,196.67251073)(271.28728893,195.86362039)
\curveto(272.10862374,195.06717452)(272.51929114,193.92850581)(272.51929114,192.44761427)
\lineto(272.51929114,191.10361186)
\lineto(265.94861269,191.10361186)
\curveto(265.97350163,190.31961046)(266.20372426,189.70360935)(266.6392806,189.25560855)
\curveto(267.0872814,188.80760775)(267.70328251,188.58360734)(268.48728391,188.58360734)
\curveto(269.13439618,188.58360734)(269.72550835,188.64582968)(270.26062042,188.77027435)
\curveto(270.80817696,188.90716348)(271.36817796,189.11249718)(271.94062344,189.38627545)
\lineto(271.94062344,187.23960494)
\curveto(271.4304003,186.9907156)(270.90151046,186.81027083)(270.35395392,186.69827063)
\curveto(269.80639739,186.57382596)(269.14061842,186.51160363)(268.35661701,186.51160363)
\curveto(267.33617074,186.51160363)(266.4339469,186.69827063)(265.64994549,187.07160463)
\curveto(264.86594409,187.4573831)(264.24994298,188.02982857)(263.80194218,188.78894105)
\curveto(263.35394138,189.56049798)(263.12994097,190.53738862)(263.12994097,191.71961297)
\curveto(263.12994097,192.90183731)(263.32905244,193.89117241)(263.72727538,194.68761829)
\curveto(264.13794278,195.48406416)(264.70416602,196.08139856)(265.42594509,196.4796215)
\curveto(266.14772416,196.87784443)(266.98772567,197.0769559)(267.94594961,197.0769559)
\closepath
\moveto(267.96461631,195.09828569)
\curveto(267.41705977,195.09828569)(266.96905897,194.92406315)(266.6206139,194.57561809)
\curveto(266.27216883,194.22717302)(266.06683513,193.68583871)(266.00461279,192.95161517)
\lineto(269.90595312,192.95161517)
\curveto(269.89350865,193.56139405)(269.72550835,194.07161718)(269.40195222,194.48228459)
\curveto(269.09084055,194.89295199)(268.61172858,195.09828569)(267.96461631,195.09828569)
\closepath
}
}
{
\newrgbcolor{curcolor}{0 0 0}
\pscustom[linestyle=none,fillstyle=solid,fillcolor=curcolor]
{
\newpath
\moveto(278.88462891,200.02629452)
\curveto(280.70152105,200.02629452)(282.03930123,199.69651615)(282.89796943,199.03695942)
\curveto(283.76908211,198.37740268)(284.20463844,197.3756231)(284.20463844,196.0316207)
\curveto(284.20463844,195.42184182)(284.08641601,194.88672975)(283.84997114,194.42628448)
\curveto(283.62597074,193.97828368)(283.3210813,193.59250521)(282.93530284,193.26894908)
\curveto(282.56196883,192.95783741)(282.15752366,192.70272584)(281.72196733,192.50361437)
\lineto(285.64197435,186.69827063)
\lineto(282.50596873,186.69827063)
\lineto(279.33262971,191.81294647)
\lineto(277.820627,191.81294647)
\lineto(277.820627,186.69827063)
\lineto(275.00195528,186.69827063)
\lineto(275.00195528,200.02629452)
\closepath
\moveto(278.67929521,197.71162371)
\lineto(277.820627,197.71162371)
\lineto(277.820627,194.10895058)
\lineto(278.73529531,194.10895058)
\curveto(279.66863031,194.10895058)(280.33440928,194.26450642)(280.73263222,194.57561809)
\curveto(281.14329962,194.88672975)(281.34863332,195.34717502)(281.34863332,195.9569539)
\curveto(281.34863332,196.5916217)(281.13085516,197.0396225)(280.69529882,197.3009563)
\curveto(280.25974248,197.57473457)(279.58774128,197.71162371)(278.67929521,197.71162371)
\closepath
}
}
{
\newrgbcolor{curcolor}{0 0 0}
\pscustom[linestyle=none,fillstyle=solid,fillcolor=curcolor]
{
\newpath
\moveto(296.35666057,191.81294647)
\curveto(296.35666057,190.12049899)(295.90865977,188.81382998)(295.01265816,187.89293944)
\curveto(294.12910102,186.9720489)(292.92198775,186.51160363)(291.39131834,186.51160363)
\curveto(290.44553886,186.51160363)(289.59931512,186.71693733)(288.85264712,187.12760473)
\curveto(288.11842358,187.53827214)(287.53975588,188.13560654)(287.11664401,188.91960795)
\curveto(286.69353214,189.71605382)(286.4819762,190.68049999)(286.4819762,191.81294647)
\curveto(286.4819762,193.50539395)(286.92375477,194.80584072)(287.80731191,195.71428679)
\curveto(288.69086905,196.62273287)(289.90420456,197.0769559)(291.44731844,197.0769559)
\curveto(292.40554238,197.0769559)(293.25176612,196.8716222)(293.98598965,196.4609548)
\curveto(294.72021319,196.0502874)(295.2988809,195.45295299)(295.72199277,194.66895159)
\curveto(296.14510464,193.88495018)(296.35666057,192.93294847)(296.35666057,191.81294647)
\closepath
\moveto(289.31931462,191.81294647)
\curveto(289.31931462,190.80494466)(289.48109269,190.03960995)(289.80464883,189.51694235)
\curveto(290.14064943,189.00671921)(290.68198373,188.75160765)(291.42865174,188.75160765)
\curveto(292.16287528,188.75160765)(292.69176511,189.00671921)(293.01532125,189.51694235)
\curveto(293.35132185,190.03960995)(293.51932215,190.80494466)(293.51932215,191.81294647)
\curveto(293.51932215,192.82094827)(293.35132185,193.57383851)(293.01532125,194.07161718)
\curveto(292.69176511,194.58184032)(292.15665304,194.83695189)(291.40998504,194.83695189)
\curveto(290.6757615,194.83695189)(290.14064943,194.58184032)(289.80464883,194.07161718)
\curveto(289.48109269,193.57383851)(289.31931462,192.82094827)(289.31931462,191.81294647)
\closepath
}
}
{
\newrgbcolor{curcolor}{0 0 0}
\pscustom[linestyle=none,fillstyle=solid,fillcolor=curcolor]
{
\newpath
\moveto(307.91133267,191.81294647)
\curveto(307.91133267,190.12049899)(307.46333186,188.81382998)(306.56733026,187.89293944)
\curveto(305.68377312,186.9720489)(304.47665984,186.51160363)(302.94599043,186.51160363)
\curveto(302.00021096,186.51160363)(301.15398722,186.71693733)(300.40731921,187.12760473)
\curveto(299.67309568,187.53827214)(299.09442797,188.13560654)(298.6713161,188.91960795)
\curveto(298.24820423,189.71605382)(298.0366483,190.68049999)(298.0366483,191.81294647)
\curveto(298.0366483,193.50539395)(298.47842687,194.80584072)(299.36198401,195.71428679)
\curveto(300.24554115,196.62273287)(301.45887666,197.0769559)(303.00199053,197.0769559)
\curveto(303.96021447,197.0769559)(304.80643821,196.8716222)(305.54066175,196.4609548)
\curveto(306.27488529,196.0502874)(306.85355299,195.45295299)(307.27666486,194.66895159)
\curveto(307.69977673,193.88495018)(307.91133267,192.93294847)(307.91133267,191.81294647)
\closepath
\moveto(300.87398672,191.81294647)
\curveto(300.87398672,190.80494466)(301.03576479,190.03960995)(301.35932092,189.51694235)
\curveto(301.69532152,189.00671921)(302.23665583,188.75160765)(302.98332383,188.75160765)
\curveto(303.71754737,188.75160765)(304.24643721,189.00671921)(304.56999334,189.51694235)
\curveto(304.90599395,190.03960995)(305.07399425,190.80494466)(305.07399425,191.81294647)
\curveto(305.07399425,192.82094827)(304.90599395,193.57383851)(304.56999334,194.07161718)
\curveto(304.24643721,194.58184032)(303.71132514,194.83695189)(302.96465713,194.83695189)
\curveto(302.23043359,194.83695189)(301.69532152,194.58184032)(301.35932092,194.07161718)
\curveto(301.03576479,193.57383851)(300.87398672,192.82094827)(300.87398672,191.81294647)
\closepath
}
}
{
\newrgbcolor{curcolor}{0 0 0}
\pscustom[linestyle=none,fillstyle=solid,fillcolor=curcolor]
{
\newpath
\moveto(322.17268391,197.0769559)
\curveto(323.33001932,197.0769559)(324.20113199,196.7782887)(324.78602193,196.1809543)
\curveto(325.38335633,195.59606436)(325.68202353,194.65028489)(325.68202353,193.34361588)
\lineto(325.68202353,186.69827063)
\lineto(322.90068521,186.69827063)
\lineto(322.90068521,192.65294797)
\curveto(322.90068521,194.12139505)(322.39046208,194.85561859)(321.3700158,194.85561859)
\curveto(320.63579227,194.85561859)(320.11312466,194.59428479)(319.80201299,194.07161718)
\curveto(319.49090132,193.54894958)(319.33534549,192.79605934)(319.33534549,191.81294647)
\lineto(319.33534549,186.69827063)
\lineto(316.55400717,186.69827063)
\lineto(316.55400717,192.65294797)
\curveto(316.55400717,194.12139505)(316.04378403,194.85561859)(315.02333776,194.85561859)
\curveto(314.25178082,194.85561859)(313.71666875,194.56317362)(313.41800155,193.97828368)
\curveto(313.13177881,193.40583821)(312.98866745,192.57828117)(312.98866745,191.49561256)
\lineto(312.98866745,186.69827063)
\lineto(310.20732913,186.69827063)
\lineto(310.20732913,196.8902889)
\lineto(312.33533294,196.8902889)
\lineto(312.70866694,195.58361989)
\lineto(312.85800055,195.58361989)
\curveto(313.16911221,196.1062875)(313.59222408,196.48584373)(314.12733615,196.7222886)
\curveto(314.67489269,196.95873347)(315.24111593,197.0769559)(315.82600587,197.0769559)
\curveto(316.57267387,197.0769559)(317.20111944,196.95251124)(317.71134258,196.7036219)
\curveto(318.23401018,196.46717703)(318.63845535,196.09384303)(318.92467809,195.58361989)
\lineto(319.16734519,195.58361989)
\curveto(319.47845686,196.1062875)(319.90779096,196.48584373)(320.4553475,196.7222886)
\curveto(321.0153485,196.95873347)(321.58779397,197.0769559)(322.17268391,197.0769559)
\closepath
}
}
{
\newrgbcolor{curcolor}{0 0 0}
\pscustom[linestyle=none,fillstyle=solid,fillcolor=curcolor]
{
\newpath
\moveto(328.14603193,188.00493964)
\curveto(328.14603193,188.57738511)(328.30158776,188.97560805)(328.61269943,189.19960845)
\curveto(328.9238111,189.43605332)(329.30336734,189.55427575)(329.75136814,189.55427575)
\curveto(330.18692448,189.55427575)(330.56025848,189.43605332)(330.87137015,189.19960845)
\curveto(331.18248182,188.97560805)(331.33803765,188.57738511)(331.33803765,188.00493964)
\curveto(331.33803765,187.4573831)(331.18248182,187.05916017)(330.87137015,186.81027083)
\curveto(330.56025848,186.57382596)(330.18692448,186.45560353)(329.75136814,186.45560353)
\curveto(329.30336734,186.45560353)(328.9238111,186.57382596)(328.61269943,186.81027083)
\curveto(328.30158776,187.05916017)(328.14603193,187.4573831)(328.14603193,188.00493964)
\closepath
}
}
{
\newrgbcolor{curcolor}{0 0 0}
\pscustom[linestyle=none,fillstyle=solid,fillcolor=curcolor]
{
\newpath
\moveto(338.00204732,186.51160363)
\curveto(336.48382237,186.51160363)(335.30782027,186.92849327)(334.47404099,187.76227254)
\curveto(333.65270619,188.59605181)(333.24203879,189.92138752)(333.24203879,191.73827967)
\curveto(333.24203879,192.98272634)(333.45359472,193.99695038)(333.87670659,194.78095179)
\curveto(334.29981846,195.56495319)(334.8847084,196.1436209)(335.6313764,196.5169549)
\curveto(336.39048887,196.8902889)(337.26160155,197.0769559)(338.24471442,197.0769559)
\curveto(338.94160456,197.0769559)(339.5451612,197.00851134)(340.05538433,196.8716222)
\curveto(340.57805194,196.73473307)(341.03227497,196.572955)(341.41805344,196.386288)
\lineto(340.59671864,194.23961748)
\curveto(340.1611623,194.41384002)(339.7504949,194.55695139)(339.36471643,194.66895159)
\curveto(338.99138242,194.78095179)(338.61804842,194.83695189)(338.24471442,194.83695189)
\curveto(336.80115628,194.83695189)(336.0793772,193.81028338)(336.0793772,191.75694637)
\curveto(336.0793772,190.73650009)(336.26604421,189.98360985)(336.63937821,189.49827565)
\curveto(337.02515668,189.01294145)(337.56026875,188.77027435)(338.24471442,188.77027435)
\curveto(338.82960436,188.77027435)(339.34604973,188.84494115)(339.79405053,188.99427475)
\curveto(340.24205133,189.15605282)(340.67760767,189.37383098)(341.10071954,189.64760925)
\lineto(341.10071954,187.27693834)
\curveto(340.67760767,187.00316007)(340.22960687,186.81027083)(339.75671713,186.69827063)
\curveto(339.29627186,186.57382596)(338.71138192,186.51160363)(338.00204732,186.51160363)
\closepath
}
}
{
\newrgbcolor{curcolor}{0 0 0}
\pscustom[linestyle=none,fillstyle=solid,fillcolor=curcolor]
{
\newpath
\moveto(350.56474093,189.72227605)
\curveto(350.56474093,188.68938531)(350.19762916,187.89293944)(349.46340563,187.33293844)
\curveto(348.74162655,186.7853819)(347.65895795,186.51160363)(346.2153998,186.51160363)
\curveto(345.5060652,186.51160363)(344.89628633,186.5613815)(344.38606319,186.66093723)
\curveto(343.87584005,186.7480485)(343.36561692,186.8973821)(342.85539378,187.10893803)
\lineto(342.85539378,189.40494215)
\curveto(343.40295032,189.15605282)(343.99406249,188.95071911)(344.62873029,188.78894105)
\curveto(345.2633981,188.62716298)(345.8233991,188.54627394)(346.3087333,188.54627394)
\curveto(346.84384537,188.54627394)(347.22962384,188.62716298)(347.46606871,188.78894105)
\curveto(347.70251358,188.95071911)(347.82073601,189.16227505)(347.82073601,189.42360885)
\curveto(347.82073601,189.59783138)(347.77095815,189.75338722)(347.67140241,189.89027635)
\curveto(347.58429115,190.02716549)(347.38517968,190.18272132)(347.07406801,190.35694386)
\curveto(346.76295634,190.53116639)(346.27762214,190.75516679)(345.6180654,191.02894506)
\curveto(344.97095313,191.30272333)(344.44206329,191.57027937)(344.03139589,191.83161317)
\curveto(343.63317295,192.10539144)(343.33450575,192.42894757)(343.13539428,192.80228157)
\curveto(342.93628281,193.18806004)(342.83672708,193.66717201)(342.83672708,194.23961748)
\curveto(342.83672708,195.18539696)(343.20383885,195.89473156)(343.93806239,196.3676213)
\curveto(344.67228593,196.84051103)(345.64917657,197.0769559)(346.86873431,197.0769559)
\curveto(347.50340211,197.0769559)(348.10695875,197.01473357)(348.67940422,196.8902889)
\curveto(349.25184969,196.76584423)(349.84296186,196.56051053)(350.45274073,196.2742878)
\lineto(349.61273923,194.27695088)
\curveto(349.11496056,194.48850682)(348.64207082,194.66272935)(348.19407002,194.79961849)
\curveto(347.74606921,194.94895209)(347.29184618,195.02361889)(346.83140091,195.02361889)
\curveto(346.0100661,195.02361889)(345.5993987,194.79961849)(345.5993987,194.35161768)
\curveto(345.5993987,194.18983962)(345.64917657,194.04050602)(345.7487323,193.90361688)
\curveto(345.8607325,193.77917221)(346.0660662,193.64228308)(346.3647334,193.49294948)
\curveto(346.67584507,193.34361588)(347.13006811,193.14450441)(347.72740251,192.89561507)
\curveto(348.31229245,192.65917021)(348.81629335,192.41028087)(349.23940522,192.14894707)
\curveto(349.66251709,191.90005773)(349.98607323,191.58272383)(350.21007363,191.19694536)
\curveto(350.4465185,190.81116689)(350.56474093,190.31961046)(350.56474093,189.72227605)
\closepath
}
}
{
\newrgbcolor{curcolor}{0 0 0}
\pscustom[linestyle=none,fillstyle=solid,fillcolor=curcolor]
{
\newpath
\moveto(359.84208457,189.72227605)
\curveto(359.84208457,188.68938531)(359.4749728,187.89293944)(358.74074926,187.33293844)
\curveto(358.01897019,186.7853819)(356.93630158,186.51160363)(355.49274344,186.51160363)
\curveto(354.78340883,186.51160363)(354.17362996,186.5613815)(353.66340682,186.66093723)
\curveto(353.15318369,186.7480485)(352.64296055,186.8973821)(352.13273741,187.10893803)
\lineto(352.13273741,189.40494215)
\curveto(352.68029395,189.15605282)(353.27140612,188.95071911)(353.90607393,188.78894105)
\curveto(354.54074173,188.62716298)(355.10074273,188.54627394)(355.58607694,188.54627394)
\curveto(356.12118901,188.54627394)(356.50696748,188.62716298)(356.74341234,188.78894105)
\curveto(356.97985721,188.95071911)(357.09807965,189.16227505)(357.09807965,189.42360885)
\curveto(357.09807965,189.59783138)(357.04830178,189.75338722)(356.94874605,189.89027635)
\curveto(356.86163478,190.02716549)(356.66252331,190.18272132)(356.35141164,190.35694386)
\curveto(356.04029997,190.53116639)(355.55496577,190.75516679)(354.89540903,191.02894506)
\curveto(354.24829676,191.30272333)(353.71940692,191.57027937)(353.30873952,191.83161317)
\curveto(352.91051659,192.10539144)(352.61184938,192.42894757)(352.41273791,192.80228157)
\curveto(352.21362645,193.18806004)(352.11407071,193.66717201)(352.11407071,194.23961748)
\curveto(352.11407071,195.18539696)(352.48118248,195.89473156)(353.21540602,196.3676213)
\curveto(353.94962956,196.84051103)(354.9265202,197.0769559)(356.14607794,197.0769559)
\curveto(356.78074574,197.0769559)(357.38430238,197.01473357)(357.95674785,196.8902889)
\curveto(358.52919332,196.76584423)(359.12030549,196.56051053)(359.73008437,196.2742878)
\lineto(358.89008286,194.27695088)
\curveto(358.39230419,194.48850682)(357.91941445,194.66272935)(357.47141365,194.79961849)
\curveto(357.02341285,194.94895209)(356.56918981,195.02361889)(356.10874454,195.02361889)
\curveto(355.28740973,195.02361889)(354.87674233,194.79961849)(354.87674233,194.35161768)
\curveto(354.87674233,194.18983962)(354.9265202,194.04050602)(355.02607593,193.90361688)
\curveto(355.13807613,193.77917221)(355.34340984,193.64228308)(355.64207704,193.49294948)
\curveto(355.95318871,193.34361588)(356.40741174,193.14450441)(357.00474615,192.89561507)
\curveto(357.58963608,192.65917021)(358.09363699,192.41028087)(358.51674886,192.14894707)
\curveto(358.93986073,191.90005773)(359.26341686,191.58272383)(359.48741726,191.19694536)
\curveto(359.72386213,190.81116689)(359.84208457,190.31961046)(359.84208457,189.72227605)
\closepath
}
}
{
\newrgbcolor{curcolor}{0 0 0}
\pscustom[linestyle=none,fillstyle=solid,fillcolor=curcolor]
{
\newpath
\moveto(218.16237191,179.56990224)
\curveto(218.57303932,179.56990224)(218.92770662,179.47034651)(219.22637382,179.27123504)
\curveto(219.52504102,179.08456804)(219.67437462,178.72990074)(219.67437462,178.20723314)
\curveto(219.67437462,177.69701)(219.52504102,177.3423427)(219.22637382,177.14323123)
\curveto(218.92770662,176.94411976)(218.57303932,176.84456403)(218.16237191,176.84456403)
\curveto(217.73926004,176.84456403)(217.37837051,176.94411976)(217.07970331,177.14323123)
\curveto(216.79348057,177.3423427)(216.6503692,177.69701)(216.6503692,178.20723314)
\curveto(216.6503692,178.72990074)(216.79348057,179.08456804)(217.07970331,179.27123504)
\curveto(217.37837051,179.47034651)(217.73926004,179.56990224)(218.16237191,179.56990224)
\closepath
\moveto(219.54370772,175.57522842)
\lineto(219.54370772,165.38321015)
\lineto(216.7623694,165.38321015)
\lineto(216.7623694,175.57522842)
\closepath
}
}
{
\newrgbcolor{curcolor}{0 0 0}
\pscustom[linestyle=none,fillstyle=solid,fillcolor=curcolor]
{
\newpath
\moveto(228.24239573,175.76189542)
\curveto(229.33750881,175.76189542)(230.21484371,175.46322822)(230.87440045,174.86589381)
\curveto(231.53395719,174.28100388)(231.86373556,173.3352244)(231.86373556,172.02855539)
\lineto(231.86373556,165.38321015)
\lineto(229.08239724,165.38321015)
\lineto(229.08239724,171.33788749)
\curveto(229.08239724,172.07211103)(228.95173034,172.61966756)(228.69039654,172.9805571)
\curveto(228.42906273,173.3538911)(228.0121731,173.5405581)(227.43972763,173.5405581)
\curveto(226.59350389,173.5405581)(226.01483618,173.24811313)(225.70372451,172.6632232)
\curveto(225.39261285,172.09077773)(225.23705701,171.26322069)(225.23705701,170.18055208)
\lineto(225.23705701,165.38321015)
\lineto(222.45571869,165.38321015)
\lineto(222.45571869,175.57522842)
\lineto(224.58372251,175.57522842)
\lineto(224.95705651,174.26855941)
\lineto(225.10639011,174.26855941)
\curveto(225.42994625,174.79122701)(225.87172482,175.17078325)(226.43172582,175.40722812)
\curveto(227.00417129,175.64367298)(227.60772793,175.76189542)(228.24239573,175.76189542)
\closepath
}
}
{
\newrgbcolor{curcolor}{0 0 0}
\pscustom[linestyle=none,fillstyle=solid,fillcolor=curcolor]
{
\newpath
\moveto(237.94907143,165.19654315)
\curveto(236.81662495,165.19654315)(235.88951218,165.63832172)(235.16773311,166.52187886)
\curveto(234.4583985,167.41788046)(234.1037312,168.7307717)(234.1037312,170.46055258)
\curveto(234.1037312,172.20277793)(234.46462074,173.5218914)(235.18639981,174.41789301)
\curveto(235.90817888,175.31389462)(236.85395835,175.76189542)(238.02373823,175.76189542)
\curveto(238.75796177,175.76189542)(239.3615184,175.61878405)(239.83440814,175.33256132)
\curveto(240.30729788,175.04633858)(240.68063188,174.69167128)(240.95441015,174.26855941)
\lineto(241.04774365,174.26855941)
\curveto(241.01041025,174.46767088)(240.96685461,174.75389361)(240.91707675,175.12722761)
\curveto(240.86729888,175.51300608)(240.84240995,175.90500679)(240.84240995,176.30322972)
\lineto(240.84240995,179.56990224)
\lineto(243.62374827,179.56990224)
\lineto(243.62374827,165.38321015)
\lineto(241.49574445,165.38321015)
\lineto(240.95441015,166.70854586)
\lineto(240.84240995,166.70854586)
\curveto(240.56863168,166.28543399)(240.20151991,165.92454445)(239.74107464,165.62587725)
\curveto(239.28062937,165.33965451)(238.68329497,165.19654315)(237.94907143,165.19654315)
\closepath
\moveto(238.91973983,167.41788046)
\curveto(239.67885231,167.41788046)(240.21396438,167.64188086)(240.52507605,168.08988167)
\curveto(240.83618771,168.55032694)(241.00418802,169.23477261)(241.02907695,170.14321868)
\lineto(241.02907695,170.44188588)
\curveto(241.02907695,171.42499876)(240.87352111,172.17788899)(240.56240945,172.7005566)
\curveto(240.26374224,173.23566867)(239.70374124,173.5032247)(238.88240643,173.5032247)
\curveto(238.27262756,173.5032247)(237.79351559,173.23566867)(237.44507052,172.7005566)
\curveto(237.09662545,172.17788899)(236.92240292,171.41877652)(236.92240292,170.42321918)
\curveto(236.92240292,169.42766184)(237.09662545,168.6747716)(237.44507052,168.16454847)
\curveto(237.79351559,167.6667698)(238.28507203,167.41788046)(238.91973983,167.41788046)
\closepath
}
}
{
\newrgbcolor{curcolor}{0 0 0}
\pscustom[linestyle=none,fillstyle=solid,fillcolor=curcolor]
{
\newpath
\moveto(250.73575111,175.76189542)
\curveto(252.14197586,175.76189542)(253.25575563,175.35745025)(254.07709044,174.54855991)
\curveto(254.89842524,173.75211404)(255.30909265,172.61344533)(255.30909265,171.13255379)
\lineto(255.30909265,169.78855138)
\lineto(248.7384142,169.78855138)
\curveto(248.76330313,169.00454997)(248.99352577,168.38854887)(249.42908211,167.94054806)
\curveto(249.87708291,167.49254726)(250.49308401,167.26854686)(251.27708542,167.26854686)
\curveto(251.92419769,167.26854686)(252.51530986,167.33076919)(253.05042193,167.45521386)
\curveto(253.59797847,167.592103)(254.15797947,167.7974367)(254.73042494,168.07121497)
\lineto(254.73042494,165.92454445)
\curveto(254.2202018,165.67565512)(253.69131197,165.49521035)(253.14375543,165.38321015)
\curveto(252.59619889,165.25876548)(251.93041992,165.19654315)(251.14641852,165.19654315)
\curveto(250.12597224,165.19654315)(249.2237484,165.38321015)(248.439747,165.75654415)
\curveto(247.65574559,166.14232262)(247.03974449,166.71476809)(246.59174369,167.47388056)
\curveto(246.14374288,168.2454375)(245.91974248,169.22232814)(245.91974248,170.40455248)
\curveto(245.91974248,171.58677682)(246.11885395,172.57611193)(246.51707689,173.3725578)
\curveto(246.92774429,174.16900367)(247.49396753,174.76633808)(248.2157466,175.16456101)
\curveto(248.93752567,175.56278395)(249.77752717,175.76189542)(250.73575111,175.76189542)
\closepath
\moveto(250.75441781,173.7832252)
\curveto(250.20686128,173.7832252)(249.75886047,173.60900267)(249.41041541,173.2605576)
\curveto(249.06197034,172.91211253)(248.85663663,172.37077823)(248.7944143,171.63655469)
\lineto(252.69575463,171.63655469)
\curveto(252.68331016,172.24633356)(252.51530986,172.7565567)(252.19175372,173.1672241)
\curveto(251.88064206,173.5778915)(251.40153009,173.7832252)(250.75441781,173.7832252)
\closepath
}
}
{
\newrgbcolor{curcolor}{0 0 0}
\pscustom[linestyle=none,fillstyle=solid,fillcolor=curcolor]
{
\newpath
\moveto(259.28508429,170.59121948)
\lineto(255.99974507,175.57522842)
\lineto(259.15441739,175.57522842)
\lineto(261.1330876,172.32722259)
\lineto(263.13042452,175.57522842)
\lineto(266.28509684,175.57522842)
\lineto(262.96242422,170.59121948)
\lineto(266.43443044,165.38321015)
\lineto(263.27975812,165.38321015)
\lineto(261.1330876,168.87388307)
\lineto(258.98641709,165.38321015)
\lineto(255.83174477,165.38321015)
\closepath
}
}
{
\newrgbcolor{curcolor}{0 0 0}
\pscustom[linestyle=none,fillstyle=solid,fillcolor=curcolor]
{
\newpath
\moveto(267.5917502,166.68987916)
\curveto(267.5917502,167.26232463)(267.74730603,167.66054756)(268.0584177,167.88454796)
\curveto(268.36952937,168.12099283)(268.7490856,168.23921527)(269.19708641,168.23921527)
\curveto(269.63264274,168.23921527)(270.00597675,168.12099283)(270.31708841,167.88454796)
\curveto(270.62820008,167.66054756)(270.78375592,167.26232463)(270.78375592,166.68987916)
\curveto(270.78375592,166.14232262)(270.62820008,165.74409968)(270.31708841,165.49521035)
\curveto(270.00597675,165.25876548)(269.63264274,165.14054305)(269.19708641,165.14054305)
\curveto(268.7490856,165.14054305)(268.36952937,165.25876548)(268.0584177,165.49521035)
\curveto(267.74730603,165.74409968)(267.5917502,166.14232262)(267.5917502,166.68987916)
\closepath
}
}
{
\newrgbcolor{curcolor}{0 0 0}
\pscustom[linestyle=none,fillstyle=solid,fillcolor=curcolor]
{
\newpath
\moveto(277.44776558,165.19654315)
\curveto(275.92954064,165.19654315)(274.75353853,165.61343278)(273.91975926,166.44721205)
\curveto(273.09842445,167.28099133)(272.68775705,168.60632704)(272.68775705,170.42321918)
\curveto(272.68775705,171.66766586)(272.89931299,172.6818899)(273.32242486,173.4658913)
\curveto(273.74553673,174.24989271)(274.33042666,174.82856041)(275.07709467,175.20189441)
\curveto(275.83620714,175.57522842)(276.70731981,175.76189542)(277.69043269,175.76189542)
\curveto(278.38732282,175.76189542)(278.99087946,175.69345085)(279.5011026,175.55656172)
\curveto(280.0237702,175.41967258)(280.47799324,175.25789452)(280.86377171,175.07122751)
\lineto(280.0424369,172.924557)
\curveto(279.60688057,173.09877953)(279.19621316,173.2418909)(278.81043469,173.3538911)
\curveto(278.43710069,173.4658913)(278.06376669,173.5218914)(277.69043269,173.5218914)
\curveto(276.24687454,173.5218914)(275.52509547,172.4952229)(275.52509547,170.44188588)
\curveto(275.52509547,169.42143961)(275.71176247,168.66854937)(276.08509647,168.18321517)
\curveto(276.47087494,167.69788096)(277.00598701,167.45521386)(277.69043269,167.45521386)
\curveto(278.27532262,167.45521386)(278.79176799,167.52988066)(279.2397688,167.67921426)
\curveto(279.6877696,167.84099233)(280.12332594,168.0587705)(280.54643781,168.33254877)
\lineto(280.54643781,165.96187785)
\curveto(280.12332594,165.68809958)(279.67532513,165.49521035)(279.2024354,165.38321015)
\curveto(278.74199013,165.25876548)(278.15710019,165.19654315)(277.44776558,165.19654315)
\closepath
}
}
{
\newrgbcolor{curcolor}{0 0 0}
\pscustom[linestyle=none,fillstyle=solid,fillcolor=curcolor]
{
\newpath
\moveto(290.01045157,168.40721557)
\curveto(290.01045157,167.37432483)(289.6433398,166.57787896)(288.90911626,166.01787795)
\curveto(288.18733719,165.47032141)(287.10466858,165.19654315)(285.66111044,165.19654315)
\curveto(284.95177584,165.19654315)(284.34199696,165.24632101)(283.83177383,165.34587675)
\curveto(283.32155069,165.43298801)(282.81132755,165.58232162)(282.30110442,165.79387755)
\lineto(282.30110442,168.08988167)
\curveto(282.84866095,167.84099233)(283.43977312,167.63565863)(284.07444093,167.47388056)
\curveto(284.70910873,167.31210249)(285.26910974,167.23121346)(285.75444394,167.23121346)
\curveto(286.28955601,167.23121346)(286.67533448,167.31210249)(286.91177935,167.47388056)
\curveto(287.14822422,167.63565863)(287.26644665,167.84721456)(287.26644665,168.10854837)
\curveto(287.26644665,168.2827709)(287.21666878,168.43832673)(287.11711305,168.57521587)
\curveto(287.03000178,168.712105)(286.83089031,168.86766084)(286.51977865,169.04188337)
\curveto(286.20866698,169.21610591)(285.72333277,169.44010631)(285.06377604,169.71388458)
\curveto(284.41666376,169.98766285)(283.88777393,170.25521888)(283.47710653,170.51655268)
\curveto(283.07888359,170.79033095)(282.78021639,171.11388709)(282.58110492,171.48722109)
\curveto(282.38199345,171.87299956)(282.28243772,172.35211153)(282.28243772,172.924557)
\curveto(282.28243772,173.87033647)(282.64954949,174.57967108)(283.38377302,175.05256081)
\curveto(284.11799656,175.52545055)(285.0948872,175.76189542)(286.31444494,175.76189542)
\curveto(286.94911275,175.76189542)(287.55266939,175.69967308)(288.12511486,175.57522842)
\curveto(288.69756033,175.45078375)(289.2886725,175.24545005)(289.89845137,174.95922731)
\lineto(289.05844986,172.9618904)
\curveto(288.56067119,173.17344633)(288.08778146,173.34766887)(287.63978065,173.484558)
\curveto(287.19177985,173.6338916)(286.73755681,173.7085584)(286.27711154,173.7085584)
\curveto(285.45577674,173.7085584)(285.04510934,173.484558)(285.04510934,173.0365572)
\curveto(285.04510934,172.87477913)(285.0948872,172.72544553)(285.19444294,172.5885564)
\curveto(285.30644314,172.46411173)(285.51177684,172.32722259)(285.81044404,172.17788899)
\curveto(286.12155571,172.02855539)(286.57577875,171.82944392)(287.17311315,171.58055459)
\curveto(287.75800309,171.34410972)(288.26200399,171.09522039)(288.68511586,170.83388658)
\curveto(289.10822773,170.58499725)(289.43178387,170.26766335)(289.65578427,169.88188488)
\curveto(289.89222914,169.49610641)(290.01045157,169.00454997)(290.01045157,168.40721557)
\closepath
}
}
{
\newrgbcolor{curcolor}{0 0 0}
\pscustom[linestyle=none,fillstyle=solid,fillcolor=curcolor]
{
\newpath
\moveto(299.2877952,168.40721557)
\curveto(299.2877952,167.37432483)(298.92068343,166.57787896)(298.1864599,166.01787795)
\curveto(297.46468082,165.47032141)(296.38201222,165.19654315)(294.93845407,165.19654315)
\curveto(294.22911947,165.19654315)(293.6193406,165.24632101)(293.10911746,165.34587675)
\curveto(292.59889432,165.43298801)(292.08867119,165.58232162)(291.57844805,165.79387755)
\lineto(291.57844805,168.08988167)
\curveto(292.12600459,167.84099233)(292.71711676,167.63565863)(293.35178456,167.47388056)
\curveto(293.98645237,167.31210249)(294.54645337,167.23121346)(295.03178757,167.23121346)
\curveto(295.56689964,167.23121346)(295.95267811,167.31210249)(296.18912298,167.47388056)
\curveto(296.42556785,167.63565863)(296.54379028,167.84721456)(296.54379028,168.10854837)
\curveto(296.54379028,168.2827709)(296.49401242,168.43832673)(296.39445668,168.57521587)
\curveto(296.30734542,168.712105)(296.10823395,168.86766084)(295.79712228,169.04188337)
\curveto(295.48601061,169.21610591)(295.00067641,169.44010631)(294.34111967,169.71388458)
\curveto(293.6940074,169.98766285)(293.16511756,170.25521888)(292.75445016,170.51655268)
\curveto(292.35622722,170.79033095)(292.05756002,171.11388709)(291.85844855,171.48722109)
\curveto(291.65933708,171.87299956)(291.55978135,172.35211153)(291.55978135,172.924557)
\curveto(291.55978135,173.87033647)(291.92689312,174.57967108)(292.66111666,175.05256081)
\curveto(293.3953402,175.52545055)(294.37223084,175.76189542)(295.59178858,175.76189542)
\curveto(296.22645638,175.76189542)(296.83001302,175.69967308)(297.40245849,175.57522842)
\curveto(297.97490396,175.45078375)(298.56601613,175.24545005)(299.175795,174.95922731)
\lineto(298.3357935,172.9618904)
\curveto(297.83801483,173.17344633)(297.36512509,173.34766887)(296.91712429,173.484558)
\curveto(296.46912348,173.6338916)(296.01490045,173.7085584)(295.55445518,173.7085584)
\curveto(294.73312037,173.7085584)(294.32245297,173.484558)(294.32245297,173.0365572)
\curveto(294.32245297,172.87477913)(294.37223084,172.72544553)(294.47178657,172.5885564)
\curveto(294.58378677,172.46411173)(294.78912047,172.32722259)(295.08778767,172.17788899)
\curveto(295.39889934,172.02855539)(295.85312238,171.82944392)(296.45045678,171.58055459)
\curveto(297.03534672,171.34410972)(297.53934762,171.09522039)(297.96245949,170.83388658)
\curveto(298.38557136,170.58499725)(298.7091275,170.26766335)(298.9331279,169.88188488)
\curveto(299.16957277,169.49610641)(299.2877952,169.00454997)(299.2877952,168.40721557)
\closepath
}
}
{
\newrgbcolor{curcolor}{0 0 0}
\pscustom[linestyle=none,fillstyle=solid,fillcolor=curcolor]
{
\newpath
\moveto(221.78878549,154.38552575)
\curveto(221.92567462,154.38552575)(222.08745269,154.37930351)(222.27411969,154.36685905)
\curveto(222.46078669,154.35441458)(222.61012029,154.33574788)(222.72212049,154.31085895)
\lineto(222.51678679,151.69752093)
\curveto(222.41723106,151.72240986)(222.28656416,151.74107656)(222.12478609,151.75352103)
\curveto(221.96300802,151.77840996)(221.81989665,151.79085443)(221.69545199,151.79085443)
\curveto(221.22256225,151.79085443)(220.76833921,151.70374316)(220.33278288,151.52952063)
\curveto(219.89722654,151.36774256)(219.54255924,151.10018653)(219.26878097,150.72685252)
\curveto(219.00744717,150.35351852)(218.87678027,149.84329538)(218.87678027,149.19618311)
\lineto(218.87678027,144.00684048)
\lineto(216.09544195,144.00684048)
\lineto(216.09544195,154.19885875)
\lineto(218.20477906,154.19885875)
\lineto(218.61544646,152.48152234)
\lineto(218.74611337,152.48152234)
\curveto(219.04478057,153.00418994)(219.45544797,153.45219074)(219.97811557,153.82552474)
\curveto(220.50078318,154.19885875)(221.10433982,154.38552575)(221.78878549,154.38552575)
\closepath
}
}
{
\newrgbcolor{curcolor}{0 0 0}
\pscustom[linestyle=none,fillstyle=solid,fillcolor=curcolor]
{
\newpath
\moveto(233.45547303,149.12151631)
\curveto(233.45547303,147.42906883)(233.00747223,146.12239982)(232.11147062,145.20150929)
\curveto(231.22791348,144.28061875)(230.02080021,143.82017348)(228.4901308,143.82017348)
\curveto(227.54435132,143.82017348)(226.69812759,144.02550718)(225.95145958,144.43617458)
\curveto(225.21723604,144.84684198)(224.63856834,145.44417639)(224.21545647,146.22817779)
\curveto(223.7923446,147.02462366)(223.58078866,147.98906984)(223.58078866,149.12151631)
\curveto(223.58078866,150.81396379)(224.02256723,152.11441057)(224.90612437,153.02285664)
\curveto(225.78968151,153.93130271)(227.00301702,154.38552575)(228.5461309,154.38552575)
\curveto(229.50435484,154.38552575)(230.35057858,154.18019205)(231.08480212,153.76952464)
\curveto(231.81902565,153.35885724)(232.39769336,152.76152284)(232.82080523,151.97752143)
\curveto(233.2439171,151.19352003)(233.45547303,150.24151832)(233.45547303,149.12151631)
\closepath
\moveto(226.41812708,149.12151631)
\curveto(226.41812708,148.11351451)(226.57990515,147.3481798)(226.90346129,146.8255122)
\curveto(227.23946189,146.31528906)(227.78079619,146.06017749)(228.5274642,146.06017749)
\curveto(229.26168774,146.06017749)(229.79057757,146.31528906)(230.11413371,146.8255122)
\curveto(230.45013431,147.3481798)(230.61813461,148.11351451)(230.61813461,149.12151631)
\curveto(230.61813461,150.12951812)(230.45013431,150.88240836)(230.11413371,151.38018703)
\curveto(229.79057757,151.89041016)(229.2554655,152.14552173)(228.5087975,152.14552173)
\curveto(227.77457396,152.14552173)(227.23946189,151.89041016)(226.90346129,151.38018703)
\curveto(226.57990515,150.88240836)(226.41812708,150.12951812)(226.41812708,149.12151631)
\closepath
}
}
{
\newrgbcolor{curcolor}{0 0 0}
\pscustom[linestyle=none,fillstyle=solid,fillcolor=curcolor]
{
\newpath
\moveto(245.0101499,149.12151631)
\curveto(245.0101499,147.42906883)(244.56214909,146.12239982)(243.66614749,145.20150929)
\curveto(242.78259035,144.28061875)(241.57547707,143.82017348)(240.04480766,143.82017348)
\curveto(239.09902819,143.82017348)(238.25280445,144.02550718)(237.50613644,144.43617458)
\curveto(236.77191291,144.84684198)(236.1932452,145.44417639)(235.77013333,146.22817779)
\curveto(235.34702146,147.02462366)(235.13546553,147.98906984)(235.13546553,149.12151631)
\curveto(235.13546553,150.81396379)(235.5772441,152.11441057)(236.46080124,153.02285664)
\curveto(237.34435838,153.93130271)(238.55769389,154.38552575)(240.10080776,154.38552575)
\curveto(241.0590317,154.38552575)(241.90525544,154.18019205)(242.63947898,153.76952464)
\curveto(243.37370252,153.35885724)(243.95237022,152.76152284)(244.37548209,151.97752143)
\curveto(244.79859396,151.19352003)(245.0101499,150.24151832)(245.0101499,149.12151631)
\closepath
\moveto(237.97280395,149.12151631)
\curveto(237.97280395,148.11351451)(238.13458202,147.3481798)(238.45813815,146.8255122)
\curveto(238.79413875,146.31528906)(239.33547306,146.06017749)(240.08214106,146.06017749)
\curveto(240.8163646,146.06017749)(241.34525444,146.31528906)(241.66881057,146.8255122)
\curveto(242.00481118,147.3481798)(242.17281148,148.11351451)(242.17281148,149.12151631)
\curveto(242.17281148,150.12951812)(242.00481118,150.88240836)(241.66881057,151.38018703)
\curveto(241.34525444,151.89041016)(240.81014237,152.14552173)(240.06347436,152.14552173)
\curveto(239.32925082,152.14552173)(238.79413875,151.89041016)(238.45813815,151.38018703)
\curveto(238.13458202,150.88240836)(237.97280395,150.12951812)(237.97280395,149.12151631)
\closepath
}
}
{
\newrgbcolor{curcolor}{0 0 0}
\pscustom[linestyle=none,fillstyle=solid,fillcolor=curcolor]
{
\newpath
\moveto(259.27150114,154.38552575)
\curveto(260.42883655,154.38552575)(261.29994922,154.08685855)(261.88483916,153.48952414)
\curveto(262.48217356,152.9046342)(262.78084076,151.95885473)(262.78084076,150.65218572)
\lineto(262.78084076,144.00684048)
\lineto(259.99950244,144.00684048)
\lineto(259.99950244,149.96151782)
\curveto(259.99950244,151.42996489)(259.48927931,152.16418843)(258.46883303,152.16418843)
\curveto(257.7346095,152.16418843)(257.21194189,151.90285463)(256.90083022,151.38018703)
\curveto(256.58971855,150.85751942)(256.43416272,150.10462919)(256.43416272,149.12151631)
\lineto(256.43416272,144.00684048)
\lineto(253.6528244,144.00684048)
\lineto(253.6528244,149.96151782)
\curveto(253.6528244,151.42996489)(253.14260126,152.16418843)(252.12215499,152.16418843)
\curveto(251.35059805,152.16418843)(250.81548598,151.87174346)(250.51681878,151.28685353)
\curveto(250.23059604,150.71440806)(250.08748468,149.88685102)(250.08748468,148.80418241)
\lineto(250.08748468,144.00684048)
\lineto(247.30614636,144.00684048)
\lineto(247.30614636,154.19885875)
\lineto(249.43415017,154.19885875)
\lineto(249.80748417,152.89218974)
\lineto(249.95681778,152.89218974)
\curveto(250.26792944,153.41485734)(250.69104131,153.79441358)(251.22615338,154.03085845)
\curveto(251.77370992,154.26730331)(252.33993316,154.38552575)(252.9248231,154.38552575)
\curveto(253.6714911,154.38552575)(254.29993667,154.26108108)(254.81015981,154.01219175)
\curveto(255.33282741,153.77574688)(255.73727258,153.40241287)(256.02349532,152.89218974)
\lineto(256.26616242,152.89218974)
\curveto(256.57727409,153.41485734)(257.00660819,153.79441358)(257.55416473,154.03085845)
\curveto(258.11416573,154.26730331)(258.6866112,154.38552575)(259.27150114,154.38552575)
\closepath
}
}
{
\newrgbcolor{curcolor}{0 0 0}
\pscustom[linestyle=none,fillstyle=solid,fillcolor=curcolor]
{
\newpath
\moveto(265.24484916,145.31350949)
\curveto(265.24484916,145.88595496)(265.40040499,146.28417789)(265.71151666,146.50817829)
\curveto(266.02262833,146.74462316)(266.40218457,146.8628456)(266.85018537,146.8628456)
\curveto(267.28574171,146.8628456)(267.65907571,146.74462316)(267.97018738,146.50817829)
\curveto(268.28129905,146.28417789)(268.43685488,145.88595496)(268.43685488,145.31350949)
\curveto(268.43685488,144.76595295)(268.28129905,144.36773001)(267.97018738,144.11884068)
\curveto(267.65907571,143.88239581)(267.28574171,143.76417338)(266.85018537,143.76417338)
\curveto(266.40218457,143.76417338)(266.02262833,143.88239581)(265.71151666,144.11884068)
\curveto(265.40040499,144.36773001)(265.24484916,144.76595295)(265.24484916,145.31350949)
\closepath
}
}
{
\newrgbcolor{curcolor}{0 0 0}
\pscustom[linestyle=none,fillstyle=solid,fillcolor=curcolor]
{
\newpath
\moveto(275.10086455,143.82017348)
\curveto(273.5826396,143.82017348)(272.4066375,144.23706311)(271.57285822,145.07084238)
\curveto(270.75152342,145.90462166)(270.34085602,147.22995737)(270.34085602,149.04684951)
\curveto(270.34085602,150.29129619)(270.55241195,151.30552023)(270.97552382,152.08952163)
\curveto(271.39863569,152.87352304)(271.98352563,153.45219074)(272.73019363,153.82552474)
\curveto(273.4893061,154.19885875)(274.36041878,154.38552575)(275.34353165,154.38552575)
\curveto(276.04042179,154.38552575)(276.64397842,154.31708118)(277.15420156,154.18019205)
\curveto(277.67686917,154.04330291)(278.1310922,153.88152484)(278.51687067,153.69485784)
\lineto(277.69553587,151.54818733)
\curveto(277.25997953,151.72240986)(276.84931213,151.86552123)(276.46353366,151.97752143)
\curveto(276.09019965,152.08952163)(275.71686565,152.14552173)(275.34353165,152.14552173)
\curveto(273.89997351,152.14552173)(273.17819443,151.11885323)(273.17819443,149.06551621)
\curveto(273.17819443,148.04506994)(273.36486144,147.2921797)(273.73819544,146.8068455)
\curveto(274.12397391,146.32151129)(274.65908598,146.07884419)(275.34353165,146.07884419)
\curveto(275.92842159,146.07884419)(276.44486696,146.15351099)(276.89286776,146.30284459)
\curveto(277.34086856,146.46462266)(277.7764249,146.68240083)(278.19953677,146.9561791)
\lineto(278.19953677,144.58550818)
\curveto(277.7764249,144.31172991)(277.3284241,144.11884068)(276.85553436,144.00684048)
\curveto(276.39508909,143.88239581)(275.81019915,143.82017348)(275.10086455,143.82017348)
\closepath
}
}
{
\newrgbcolor{curcolor}{0 0 0}
\pscustom[linestyle=none,fillstyle=solid,fillcolor=curcolor]
{
\newpath
\moveto(287.66355435,147.0308459)
\curveto(287.66355435,145.99795516)(287.29644258,145.20150929)(286.56221904,144.64150828)
\curveto(285.84043997,144.09395174)(284.75777136,143.82017348)(283.31421322,143.82017348)
\curveto(282.60487861,143.82017348)(281.99509974,143.86995134)(281.48487661,143.96950708)
\curveto(280.97465347,144.05661834)(280.46443033,144.20595195)(279.9542072,144.41750788)
\lineto(279.9542072,146.713512)
\curveto(280.50176373,146.46462266)(281.0928759,146.25928896)(281.72754371,146.09751089)
\curveto(282.36221151,145.93573282)(282.92221252,145.85484379)(283.40754672,145.85484379)
\curveto(283.94265879,145.85484379)(284.32843726,145.93573282)(284.56488213,146.09751089)
\curveto(284.80132699,146.25928896)(284.91954943,146.47084489)(284.91954943,146.7321787)
\curveto(284.91954943,146.90640123)(284.86977156,147.06195706)(284.77021583,147.1988462)
\curveto(284.68310456,147.33573533)(284.48399309,147.49129117)(284.17288142,147.6655137)
\curveto(283.86176976,147.83973624)(283.37643555,148.06373664)(282.71687881,148.33751491)
\curveto(282.06976654,148.61129318)(281.54087671,148.87884921)(281.1302093,149.14018301)
\curveto(280.73198637,149.41396128)(280.43331917,149.73751742)(280.2342077,150.11085142)
\curveto(280.03509623,150.49662989)(279.93554049,150.97574186)(279.93554049,151.54818733)
\curveto(279.93554049,152.4939668)(280.30265226,153.20330141)(281.0368758,153.67619114)
\curveto(281.77109934,154.14908088)(282.74798998,154.38552575)(283.96754772,154.38552575)
\curveto(284.60221553,154.38552575)(285.20577216,154.32330341)(285.77821764,154.19885875)
\curveto(286.35066311,154.07441408)(286.94177528,153.86908038)(287.55155415,153.58285764)
\lineto(286.71155264,151.58552073)
\curveto(286.21377397,151.79707666)(285.74088423,151.9712992)(285.29288343,152.10818833)
\curveto(284.84488263,152.25752193)(284.39065959,152.33218873)(283.93021432,152.33218873)
\curveto(283.10887952,152.33218873)(282.69821211,152.10818833)(282.69821211,151.66018753)
\curveto(282.69821211,151.49840946)(282.74798998,151.34907586)(282.84754571,151.21218673)
\curveto(282.95954592,151.08774206)(283.16487962,150.95085292)(283.46354682,150.80151932)
\curveto(283.77465849,150.65218572)(284.22888152,150.45307425)(284.82621593,150.20418492)
\curveto(285.41110587,149.96774005)(285.91510677,149.71885072)(286.33821864,149.45751691)
\curveto(286.76133051,149.20862758)(287.08488664,148.89129368)(287.30888705,148.50551521)
\curveto(287.54533191,148.11973674)(287.66355435,147.6281803)(287.66355435,147.0308459)
\closepath
}
}
{
\newrgbcolor{curcolor}{0 0 0}
\pscustom[linestyle=none,fillstyle=solid,fillcolor=curcolor]
{
\newpath
\moveto(296.94089798,147.0308459)
\curveto(296.94089798,145.99795516)(296.57378621,145.20150929)(295.83956267,144.64150828)
\curveto(295.1177836,144.09395174)(294.03511499,143.82017348)(292.59155685,143.82017348)
\curveto(291.88222225,143.82017348)(291.27244338,143.86995134)(290.76222024,143.96950708)
\curveto(290.2519971,144.05661834)(289.74177397,144.20595195)(289.23155083,144.41750788)
\lineto(289.23155083,146.713512)
\curveto(289.77910737,146.46462266)(290.37021954,146.25928896)(291.00488734,146.09751089)
\curveto(291.63955514,145.93573282)(292.19955615,145.85484379)(292.68489035,145.85484379)
\curveto(293.22000242,145.85484379)(293.60578089,145.93573282)(293.84222576,146.09751089)
\curveto(294.07867063,146.25928896)(294.19689306,146.47084489)(294.19689306,146.7321787)
\curveto(294.19689306,146.90640123)(294.1471152,147.06195706)(294.04755946,147.1988462)
\curveto(293.96044819,147.33573533)(293.76133673,147.49129117)(293.45022506,147.6655137)
\curveto(293.13911339,147.83973624)(292.65377919,148.06373664)(291.99422245,148.33751491)
\curveto(291.34711018,148.61129318)(290.81822034,148.87884921)(290.40755294,149.14018301)
\curveto(290.00933,149.41396128)(289.7106628,149.73751742)(289.51155133,150.11085142)
\curveto(289.31243986,150.49662989)(289.21288413,150.97574186)(289.21288413,151.54818733)
\curveto(289.21288413,152.4939668)(289.5799959,153.20330141)(290.31421944,153.67619114)
\curveto(291.04844297,154.14908088)(292.02533361,154.38552575)(293.24489136,154.38552575)
\curveto(293.87955916,154.38552575)(294.4831158,154.32330341)(295.05556127,154.19885875)
\curveto(295.62800674,154.07441408)(296.21911891,153.86908038)(296.82889778,153.58285764)
\lineto(295.98889627,151.58552073)
\curveto(295.4911176,151.79707666)(295.01822787,151.9712992)(294.57022706,152.10818833)
\curveto(294.12222626,152.25752193)(293.66800323,152.33218873)(293.20755796,152.33218873)
\curveto(292.38622315,152.33218873)(291.97555575,152.10818833)(291.97555575,151.66018753)
\curveto(291.97555575,151.49840946)(292.02533361,151.34907586)(292.12488935,151.21218673)
\curveto(292.23688955,151.08774206)(292.44222325,150.95085292)(292.74089045,150.80151932)
\curveto(293.05200212,150.65218572)(293.50622516,150.45307425)(294.10355956,150.20418492)
\curveto(294.6884495,149.96774005)(295.1924504,149.71885072)(295.61556227,149.45751691)
\curveto(296.03867414,149.20862758)(296.36223028,148.89129368)(296.58623068,148.50551521)
\curveto(296.82267555,148.11973674)(296.94089798,147.6281803)(296.94089798,147.0308459)
\closepath
}
}
{
\newrgbcolor{curcolor}{0 0 0}
\pscustom[linestyle=none,fillstyle=solid,fillcolor=curcolor]
{
\newpath
\moveto(121.90837221,107.30731804)
\curveto(121.90837221,107.87976351)(122.06392804,108.27798645)(122.37503971,108.50198685)
\curveto(122.68615138,108.73843172)(123.06570761,108.85665415)(123.51370842,108.85665415)
\curveto(123.94926475,108.85665415)(124.32259876,108.73843172)(124.63371042,108.50198685)
\curveto(124.94482209,108.27798645)(125.10037793,107.87976351)(125.10037793,107.30731804)
\curveto(125.10037793,106.75976151)(124.94482209,106.36153857)(124.63371042,106.11264924)
\curveto(124.32259876,105.87620437)(123.94926475,105.75798193)(123.51370842,105.75798193)
\curveto(123.06570761,105.75798193)(122.68615138,105.87620437)(122.37503971,106.11264924)
\curveto(122.06392804,106.36153857)(121.90837221,106.75976151)(121.90837221,107.30731804)
\closepath
}
}
{
\newrgbcolor{curcolor}{0 0 0}
\pscustom[linestyle=none,fillstyle=solid,fillcolor=curcolor]
{
\newpath
\moveto(130.92438561,116.37933431)
\curveto(132.18127675,116.37933431)(133.16438963,115.88155564)(133.87372423,114.8859983)
\lineto(133.94839103,114.8859983)
\lineto(134.17239143,116.1926673)
\lineto(136.52439565,116.1926673)
\lineto(136.52439565,105.98198233)
\curveto(136.52439565,104.52597972)(136.09506155,103.41842218)(135.23639334,102.65930971)
\curveto(134.37772514,101.90019724)(133.10838953,101.520641)(131.42838651,101.520641)
\curveto(130.70660744,101.520641)(130.03460624,101.56419664)(129.4123829,101.6513079)
\curveto(128.80260403,101.73841917)(128.20526963,101.89397501)(127.62037969,102.11797541)
\lineto(127.62037969,104.33931272)
\curveto(128.86482636,103.81664512)(130.19016207,103.55531132)(131.59638682,103.55531132)
\curveto(133.02750049,103.55531132)(133.74305733,104.32686826)(133.74305733,105.86998213)
\lineto(133.74305733,106.07531584)
\curveto(133.74305733,106.2744273)(133.74927956,106.48598324)(133.76172403,106.70998364)
\curveto(133.7741685,106.94642851)(133.7928352,107.15176221)(133.81772413,107.32598474)
\lineto(133.74305733,107.32598474)
\curveto(133.39461226,106.79087267)(132.97772263,106.4050942)(132.49238842,106.16864934)
\curveto(132.00705422,105.93220447)(131.45949768,105.81398203)(130.84971881,105.81398203)
\curveto(129.64260554,105.81398203)(128.69682606,106.2744273)(128.01238039,107.19531784)
\curveto(127.34037919,108.12865285)(127.00437858,109.42287739)(127.00437858,111.07799147)
\curveto(127.00437858,112.74555001)(127.35282365,114.04599679)(128.04971379,114.9793318)
\curveto(128.74660393,115.9126668)(129.70482787,116.37933431)(130.92438561,116.37933431)
\closepath
\moveto(131.80172052,114.12066359)
\curveto(130.49505151,114.12066359)(129.841717,113.09399508)(129.841717,111.04065807)
\curveto(129.841717,109.01220999)(130.50749598,107.99798595)(131.83905392,107.99798595)
\curveto(132.54838852,107.99798595)(133.07105613,108.19709742)(133.40705673,108.59532035)
\curveto(133.7555018,109.00598776)(133.92972433,109.71532236)(133.92972433,110.72332417)
\lineto(133.92972433,111.05932477)
\curveto(133.92972433,112.15443784)(133.76172403,112.93843925)(133.42572343,113.41132899)
\curveto(133.08972283,113.88421872)(132.54838852,114.12066359)(131.80172052,114.12066359)
\closepath
}
}
{
\newrgbcolor{curcolor}{0 0 0}
\pscustom[linestyle=none,fillstyle=solid,fillcolor=curcolor]
{
\newpath
\moveto(140.83639205,120.18734113)
\curveto(141.24705945,120.18734113)(141.60172675,120.0877854)(141.90039396,119.88867393)
\curveto(142.19906116,119.70200693)(142.34839476,119.34733963)(142.34839476,118.82467202)
\curveto(142.34839476,118.31444889)(142.19906116,117.95978158)(141.90039396,117.76067012)
\curveto(141.60172675,117.56155865)(141.24705945,117.46200291)(140.83639205,117.46200291)
\curveto(140.41328018,117.46200291)(140.05239064,117.56155865)(139.75372344,117.76067012)
\curveto(139.46750071,117.95978158)(139.32438934,118.31444889)(139.32438934,118.82467202)
\curveto(139.32438934,119.34733963)(139.46750071,119.70200693)(139.75372344,119.88867393)
\curveto(140.05239064,120.0877854)(140.41328018,120.18734113)(140.83639205,120.18734113)
\closepath
\moveto(142.21772786,116.1926673)
\lineto(142.21772786,106.00064903)
\lineto(139.43638954,106.00064903)
\lineto(139.43638954,116.1926673)
\closepath
}
}
{
\newrgbcolor{curcolor}{0 0 0}
\pscustom[linestyle=none,fillstyle=solid,fillcolor=curcolor]
{
\newpath
\moveto(149.42308033,108.03531935)
\curveto(149.734192,108.03531935)(150.0328592,108.06020828)(150.31908194,108.10998615)
\curveto(150.60530468,108.17220848)(150.89152741,108.25309752)(151.17775015,108.35265325)
\lineto(151.17775015,106.28064954)
\curveto(150.87908294,106.1437604)(150.50574894,106.0317602)(150.05774814,105.94464893)
\curveto(149.6221918,105.85753767)(149.14307983,105.81398203)(148.62041223,105.81398203)
\curveto(148.01063336,105.81398203)(147.46307682,105.91353777)(146.97774262,106.11264924)
\curveto(146.50485288,106.3117607)(146.12529664,106.65398354)(145.83907391,107.13931774)
\curveto(145.56529564,107.62465195)(145.42840651,108.30909762)(145.42840651,109.19265476)
\lineto(145.42840651,114.10199689)
\lineto(144.1030708,114.10199689)
\lineto(144.1030708,115.277999)
\lineto(145.63374021,116.211334)
\lineto(146.43640831,118.35800452)
\lineto(148.20974483,118.35800452)
\lineto(148.20974483,116.1926673)
\lineto(151.06574995,116.1926673)
\lineto(151.06574995,114.10199689)
\lineto(148.20974483,114.10199689)
\lineto(148.20974483,109.19265476)
\curveto(148.20974483,108.80687629)(148.32174503,108.51443132)(148.54574543,108.31531985)
\curveto(148.76974583,108.12865285)(149.0621908,108.03531935)(149.42308033,108.03531935)
\closepath
}
}
{
\newrgbcolor{curcolor}{0 0 0}
\pscustom[linestyle=none,fillstyle=solid,fillcolor=curcolor]
{
\newpath
\moveto(154.63108868,120.18734113)
\curveto(155.04175608,120.18734113)(155.39642339,120.0877854)(155.69509059,119.88867393)
\curveto(155.99375779,119.70200693)(156.14309139,119.34733963)(156.14309139,118.82467202)
\curveto(156.14309139,118.31444889)(155.99375779,117.95978158)(155.69509059,117.76067012)
\curveto(155.39642339,117.56155865)(155.04175608,117.46200291)(154.63108868,117.46200291)
\curveto(154.20797681,117.46200291)(153.84708728,117.56155865)(153.54842007,117.76067012)
\curveto(153.26219734,117.95978158)(153.11908597,118.31444889)(153.11908597,118.82467202)
\curveto(153.11908597,119.34733963)(153.26219734,119.70200693)(153.54842007,119.88867393)
\curveto(153.84708728,120.0877854)(154.20797681,120.18734113)(154.63108868,120.18734113)
\closepath
\moveto(156.01242449,116.1926673)
\lineto(156.01242449,106.00064903)
\lineto(153.23108617,106.00064903)
\lineto(153.23108617,116.1926673)
\closepath
}
}
{
\newrgbcolor{curcolor}{0 0 0}
\pscustom[linestyle=none,fillstyle=solid,fillcolor=curcolor]
{
\newpath
\moveto(162.22844377,116.37933431)
\curveto(163.48533491,116.37933431)(164.46844778,115.88155564)(165.17778239,114.8859983)
\lineto(165.25244919,114.8859983)
\lineto(165.47644959,116.1926673)
\lineto(167.82845381,116.1926673)
\lineto(167.82845381,105.98198233)
\curveto(167.82845381,104.52597972)(167.3991197,103.41842218)(166.5404515,102.65930971)
\curveto(165.68178329,101.90019724)(164.41244768,101.520641)(162.73244467,101.520641)
\curveto(162.0106656,101.520641)(161.3386644,101.56419664)(160.71644106,101.6513079)
\curveto(160.10666219,101.73841917)(159.50932778,101.89397501)(158.92443785,102.11797541)
\lineto(158.92443785,104.33931272)
\curveto(160.16888452,103.81664512)(161.49422023,103.55531132)(162.90044497,103.55531132)
\curveto(164.33155865,103.55531132)(165.04711549,104.32686826)(165.04711549,105.86998213)
\lineto(165.04711549,106.07531584)
\curveto(165.04711549,106.2744273)(165.05333772,106.48598324)(165.06578219,106.70998364)
\curveto(165.07822665,106.94642851)(165.09689336,107.15176221)(165.12178229,107.32598474)
\lineto(165.04711549,107.32598474)
\curveto(164.69867042,106.79087267)(164.28178078,106.4050942)(163.79644658,106.16864934)
\curveto(163.31111238,105.93220447)(162.76355584,105.81398203)(162.15377697,105.81398203)
\curveto(160.94666369,105.81398203)(160.00088422,106.2744273)(159.31643855,107.19531784)
\curveto(158.64443734,108.12865285)(158.30843674,109.42287739)(158.30843674,111.07799147)
\curveto(158.30843674,112.74555001)(158.65688181,114.04599679)(159.35377195,114.9793318)
\curveto(160.05066209,115.9126668)(161.00888603,116.37933431)(162.22844377,116.37933431)
\closepath
\moveto(163.10577867,114.12066359)
\curveto(161.79910967,114.12066359)(161.14577516,113.09399508)(161.14577516,111.04065807)
\curveto(161.14577516,109.01220999)(161.81155413,107.99798595)(163.14311208,107.99798595)
\curveto(163.85244668,107.99798595)(164.37511428,108.19709742)(164.71111489,108.59532035)
\curveto(165.05955995,109.00598776)(165.23378249,109.71532236)(165.23378249,110.72332417)
\lineto(165.23378249,111.05932477)
\curveto(165.23378249,112.15443784)(165.06578219,112.93843925)(164.72978159,113.41132899)
\curveto(164.39378098,113.88421872)(163.85244668,114.12066359)(163.10577867,114.12066359)
\closepath
}
}
{
\newrgbcolor{curcolor}{0 0 0}
\pscustom[linestyle=none,fillstyle=solid,fillcolor=curcolor]
{
\newpath
\moveto(176.52712426,116.37933431)
\curveto(177.62223733,116.37933431)(178.49957224,116.0806671)(179.15912898,115.4833327)
\curveto(179.81868571,114.89844276)(180.14846408,113.95266329)(180.14846408,112.64599428)
\lineto(180.14846408,106.00064903)
\lineto(177.36712576,106.00064903)
\lineto(177.36712576,111.95532638)
\curveto(177.36712576,112.68954991)(177.23645886,113.23710645)(176.97512506,113.59799599)
\curveto(176.71379126,113.97132999)(176.29690162,114.15799699)(175.72445615,114.15799699)
\curveto(174.87823241,114.15799699)(174.29956471,113.86555202)(173.98845304,113.28066208)
\curveto(173.67734137,112.70821661)(173.52178554,111.88065957)(173.52178554,110.79799097)
\lineto(173.52178554,106.00064903)
\lineto(170.74044722,106.00064903)
\lineto(170.74044722,116.1926673)
\lineto(172.86845103,116.1926673)
\lineto(173.24178504,114.8859983)
\lineto(173.39111864,114.8859983)
\curveto(173.71467477,115.4086659)(174.15645334,115.78822214)(174.71645435,116.024667)
\curveto(175.28889982,116.26111187)(175.89245645,116.37933431)(176.52712426,116.37933431)
\closepath
}
}
{
\newrgbcolor{curcolor}{0 0 0}
\pscustom[linestyle=none,fillstyle=solid,fillcolor=curcolor]
{
\newpath
\moveto(192.26314553,111.11532487)
\curveto(192.26314553,109.42287739)(191.81514472,108.11620838)(190.91914312,107.19531784)
\curveto(190.03558598,106.2744273)(188.8284727,105.81398203)(187.29780329,105.81398203)
\curveto(186.35202382,105.81398203)(185.50580008,106.01931573)(184.75913207,106.42998314)
\curveto(184.02490854,106.84065054)(183.44624083,107.43798494)(183.02312896,108.22198635)
\curveto(182.60001709,109.01843222)(182.38846116,109.9828784)(182.38846116,111.11532487)
\curveto(182.38846116,112.80777235)(182.83023973,114.10821912)(183.71379687,115.0166652)
\curveto(184.59735401,115.92511127)(185.81068951,116.37933431)(187.35380339,116.37933431)
\curveto(188.31202733,116.37933431)(189.15825107,116.1740006)(189.89247461,115.7633332)
\curveto(190.62669815,115.3526658)(191.20536585,114.75533139)(191.62847772,113.97132999)
\curveto(192.05158959,113.18732858)(192.26314553,112.23532688)(192.26314553,111.11532487)
\closepath
\moveto(185.22579958,111.11532487)
\curveto(185.22579958,110.10732306)(185.38757764,109.34198836)(185.71113378,108.81932075)
\curveto(186.04713438,108.30909762)(186.58846869,108.05398605)(187.33513669,108.05398605)
\curveto(188.06936023,108.05398605)(188.59825007,108.30909762)(188.9218062,108.81932075)
\curveto(189.2578068,109.34198836)(189.42580711,110.10732306)(189.42580711,111.11532487)
\curveto(189.42580711,112.12332668)(189.2578068,112.87621692)(188.9218062,113.37399559)
\curveto(188.59825007,113.88421872)(188.063138,114.13933029)(187.31646999,114.13933029)
\curveto(186.58224645,114.13933029)(186.04713438,113.88421872)(185.71113378,113.37399559)
\curveto(185.38757764,112.87621692)(185.22579958,112.12332668)(185.22579958,111.11532487)
\closepath
}
}
{
\newrgbcolor{curcolor}{0 0 0}
\pscustom[linestyle=none,fillstyle=solid,fillcolor=curcolor]
{
\newpath
\moveto(200.25248171,116.37933431)
\curveto(200.38937085,116.37933431)(200.55114891,116.37311207)(200.73781591,116.36066761)
\curveto(200.92448292,116.34822314)(201.07381652,116.32955644)(201.18581672,116.30466751)
\lineto(200.98048302,113.69132949)
\curveto(200.88092728,113.71621842)(200.75026038,113.73488512)(200.58848231,113.74732959)
\curveto(200.42670425,113.77221852)(200.28359288,113.78466299)(200.15914821,113.78466299)
\curveto(199.68625847,113.78466299)(199.23203544,113.69755172)(198.7964791,113.52332919)
\curveto(198.36092277,113.36155112)(198.00625546,113.09399508)(197.73247719,112.72066108)
\curveto(197.47114339,112.34732708)(197.34047649,111.83710394)(197.34047649,111.18999167)
\lineto(197.34047649,106.00064903)
\lineto(194.55913817,106.00064903)
\lineto(194.55913817,116.1926673)
\lineto(196.66847529,116.1926673)
\lineto(197.07914269,114.47533089)
\lineto(197.20980959,114.47533089)
\curveto(197.50847679,114.9979985)(197.9191442,115.4459993)(198.4418118,115.8193333)
\curveto(198.9644794,116.1926673)(199.56803604,116.37933431)(200.25248171,116.37933431)
\closepath
}
}
{
\newrgbcolor{curcolor}{0 0 0}
\pscustom[linestyle=none,fillstyle=solid,fillcolor=curcolor]
{
\newpath
\moveto(206.86049448,116.37933431)
\curveto(208.26671922,116.37933431)(209.38049899,115.97488914)(210.2018338,115.1659988)
\curveto(211.0231686,114.36955293)(211.43383601,113.23088422)(211.43383601,111.74999267)
\lineto(211.43383601,110.40599026)
\lineto(204.86315756,110.40599026)
\curveto(204.8880465,109.62198886)(205.11826913,109.00598776)(205.55382547,108.55798695)
\curveto(206.00182627,108.10998615)(206.61782737,107.88598575)(207.40182878,107.88598575)
\curveto(208.04894105,107.88598575)(208.64005322,107.94820808)(209.17516529,108.07265275)
\curveto(209.72272183,108.20954188)(210.28272283,108.41487558)(210.8551683,108.68865385)
\lineto(210.8551683,106.54198334)
\curveto(210.34494517,106.293094)(209.81605533,106.11264924)(209.26849879,106.00064903)
\curveto(208.72094226,105.87620437)(208.05516328,105.81398203)(207.27116188,105.81398203)
\curveto(206.25071561,105.81398203)(205.34849177,106.00064903)(204.56449036,106.37398304)
\curveto(203.78048895,106.75976151)(203.16448785,107.33220698)(202.71648705,108.09131945)
\curveto(202.26848624,108.86287639)(202.04448584,109.83976703)(202.04448584,111.02199137)
\curveto(202.04448584,112.20421571)(202.24359731,113.19355082)(202.64182025,113.98999669)
\curveto(203.05248765,114.78644256)(203.61871089,115.38377697)(204.34048996,115.7819999)
\curveto(205.06226903,116.18022284)(205.90227054,116.37933431)(206.86049448,116.37933431)
\closepath
\moveto(206.87916118,114.40066409)
\curveto(206.33160464,114.40066409)(205.88360384,114.22644156)(205.53515877,113.87799649)
\curveto(205.1867137,113.52955142)(204.98138,112.98821712)(204.91915766,112.25399358)
\lineto(208.82049799,112.25399358)
\curveto(208.80805352,112.86377245)(208.64005322,113.37399559)(208.31649709,113.78466299)
\curveto(208.00538542,114.19533039)(207.52627345,114.40066409)(206.87916118,114.40066409)
\closepath
}
}
{
\newrgbcolor{curcolor}{0 0 0}
\pscustom[linestyle=none,fillstyle=solid,fillcolor=curcolor]
{
\newpath
\moveto(122.79310295,84.00265049)
\lineto(122.79310295,97.33067438)
\lineto(125.61177467,97.33067438)
\lineto(125.61177467,86.335988)
\lineto(131.02511771,86.335988)
\lineto(131.02511771,84.00265049)
\closepath
}
}
{
\newrgbcolor{curcolor}{0 0 0}
\pscustom[linestyle=none,fillstyle=solid,fillcolor=curcolor]
{
\newpath
\moveto(138.32380746,84.00265049)
\lineto(132.25712992,84.00265049)
\lineto(132.25712992,85.6079867)
\lineto(133.88113283,86.3546547)
\lineto(133.88113283,94.97867016)
\lineto(132.25712992,95.72533817)
\lineto(132.25712992,97.33067438)
\lineto(138.32380746,97.33067438)
\lineto(138.32380746,95.72533817)
\lineto(136.69980455,94.97867016)
\lineto(136.69980455,86.3546547)
\lineto(138.32380746,85.6079867)
\closepath
}
}
{
\newrgbcolor{curcolor}{0 0 0}
\pscustom[linestyle=none,fillstyle=solid,fillcolor=curcolor]
{
\newpath
\moveto(146.25715348,95.16533716)
\curveto(145.17448488,95.16533716)(144.34692784,94.76089199)(143.77448237,93.95200166)
\curveto(143.2020369,93.14311132)(142.91581416,92.03555378)(142.91581416,90.62932903)
\curveto(142.91581416,89.21065982)(143.17714796,88.10932452)(143.69981557,87.32532311)
\curveto(144.23492764,86.55376617)(145.08737361,86.1679877)(146.25715348,86.1679877)
\curveto(146.79226555,86.1679877)(147.33359986,86.23021004)(147.88115639,86.3546547)
\curveto(148.42871293,86.47909937)(149.0198251,86.65332191)(149.65449291,86.87732231)
\lineto(149.65449291,84.50665139)
\curveto(149.06960297,84.27020652)(148.49093526,84.09598399)(147.91848979,83.98398379)
\curveto(147.34604432,83.87198359)(146.70515429,83.81598349)(145.99581968,83.81598349)
\curveto(144.61448387,83.81598349)(143.4820374,84.09598399)(142.59848026,84.65598499)
\curveto(141.71492312,85.22843046)(141.06158861,86.02487633)(140.63847674,87.04532261)
\curveto(140.21536487,88.07821335)(140.00380894,89.27910439)(140.00380894,90.64799573)
\curveto(140.00380894,91.99199814)(140.24647604,93.18044472)(140.73181024,94.21333546)
\curveto(141.21714445,95.2462262)(141.92025682,96.05511654)(142.84114736,96.64000647)
\curveto(143.77448237,97.22489641)(144.91315107,97.51734138)(146.25715348,97.51734138)
\curveto(146.91671022,97.51734138)(147.57626696,97.43023011)(148.2358237,97.25600758)
\curveto(148.9078249,97.09422951)(149.54871494,96.87022911)(150.15849381,96.58400637)
\lineto(149.2438255,94.28800226)
\curveto(148.74604683,94.52444713)(148.24204593,94.72978083)(147.73182279,94.90400336)
\curveto(147.23404412,95.0782259)(146.74248769,95.16533716)(146.25715348,95.16533716)
\closepath
}
}
{
\newrgbcolor{curcolor}{0 0 0}
\pscustom[linestyle=none,fillstyle=solid,fillcolor=curcolor]
{
\newpath
\moveto(160.16382963,84.00265049)
\lineto(152.49181588,84.00265049)
\lineto(152.49181588,97.33067438)
\lineto(160.16382963,97.33067438)
\lineto(160.16382963,95.01600356)
\lineto(155.3104876,95.01600356)
\lineto(155.3104876,92.08533164)
\lineto(159.82782903,92.08533164)
\lineto(159.82782903,89.77066083)
\lineto(155.3104876,89.77066083)
\lineto(155.3104876,86.335988)
\lineto(160.16382963,86.335988)
\closepath
}
}
{
\newrgbcolor{curcolor}{0 0 0}
\pscustom[linestyle=none,fillstyle=solid,fillcolor=curcolor]
{
\newpath
\moveto(174.76117746,84.00265049)
\lineto(171.17717103,84.00265049)
\lineto(165.37182729,94.08266856)
\lineto(165.29716049,94.08266856)
\lineto(165.37182729,92.17866514)
\curveto(165.40916069,91.54399734)(165.44027186,90.90932953)(165.46516079,90.27466173)
\lineto(165.46516079,84.00265049)
\lineto(162.94515628,84.00265049)
\lineto(162.94515628,97.33067438)
\lineto(166.510496,97.33067438)
\lineto(172.29717304,87.34398981)
\lineto(172.35317314,87.34398981)
\lineto(172.27850634,89.17332642)
\curveto(172.25361741,89.78310529)(172.23495071,90.3991064)(172.22250624,91.02132974)
\lineto(172.22250624,97.33067438)
\lineto(174.76117746,97.33067438)
\closepath
}
}
{
\newrgbcolor{curcolor}{0 0 0}
\pscustom[linestyle=none,fillstyle=solid,fillcolor=curcolor]
{
\newpath
\moveto(185.97986223,87.69865711)
\curveto(185.97986223,86.51643277)(185.55052813,85.5706533)(184.69185992,84.86131869)
\curveto(183.84563618,84.16442855)(182.63852291,83.81598349)(181.0705201,83.81598349)
\curveto(179.66429535,83.81598349)(178.40740421,84.08353952)(177.29984667,84.61865159)
\lineto(177.29984667,87.25065631)
\curveto(177.93451447,86.97687804)(178.58784898,86.72176647)(179.25985018,86.4853216)
\curveto(179.94429586,86.2613212)(180.62251929,86.149321)(181.2945205,86.149321)
\curveto(181.99141064,86.149321)(182.48296707,86.2799879)(182.76918981,86.5413217)
\curveto(183.06785701,86.81509997)(183.21719061,87.15732281)(183.21719061,87.56799021)
\curveto(183.21719061,87.90399081)(183.09896818,88.19021355)(182.86252331,88.42665842)
\curveto(182.63852291,88.66310329)(182.33363347,88.88088145)(181.947855,89.07999292)
\curveto(181.56207653,89.29154886)(181.12029796,89.51554926)(180.62251929,89.75199413)
\curveto(180.31140762,89.90132773)(179.97540702,90.07555026)(179.61451749,90.27466173)
\curveto(179.25362795,90.48621766)(178.90518288,90.74132923)(178.56918228,91.03999644)
\curveto(178.24562614,91.3511081)(177.97807011,91.72444211)(177.76651417,92.15999844)
\curveto(177.55495824,92.59555478)(177.44918027,93.11822238)(177.44918027,93.72800125)
\curveto(177.44918027,94.92267006)(177.85362544,95.84978283)(178.66251578,96.50933957)
\curveto(179.47140612,97.18134078)(180.57274143,97.51734138)(181.9665217,97.51734138)
\curveto(182.66341184,97.51734138)(183.32296858,97.43645235)(183.94519192,97.27467428)
\curveto(184.56741525,97.11289621)(185.22697199,96.88267358)(185.92386213,96.58400637)
\lineto(185.00919382,94.38133576)
\curveto(184.39941495,94.63022509)(183.85185842,94.82311433)(183.36652421,94.96000346)
\curveto(182.88119001,95.0968926)(182.38341134,95.16533716)(181.8731882,95.16533716)
\curveto(181.33807613,95.16533716)(180.92740873,95.0408925)(180.64118599,94.79200316)
\curveto(180.35496326,94.54311383)(180.21185189,94.21955769)(180.21185189,93.82133475)
\curveto(180.21185189,93.34844502)(180.42340783,92.97511102)(180.84651969,92.70133275)
\curveto(181.26963156,92.42755448)(181.89807714,92.09155388)(182.73185641,91.69333094)
\curveto(183.41630208,91.3697748)(183.99496978,91.0337742)(184.46785952,90.68532913)
\curveto(184.95319372,90.33688406)(185.32652773,89.92621666)(185.58786153,89.45332692)
\curveto(185.84919533,88.98043719)(185.97986223,88.39554725)(185.97986223,87.69865711)
\closepath
}
}
{
\newrgbcolor{curcolor}{0 0 0}
\pscustom[linestyle=none,fillstyle=solid,fillcolor=curcolor]
{
\newpath
\moveto(196.07855734,84.00265049)
\lineto(188.40654359,84.00265049)
\lineto(188.40654359,97.33067438)
\lineto(196.07855734,97.33067438)
\lineto(196.07855734,95.01600356)
\lineto(191.22521531,95.01600356)
\lineto(191.22521531,92.08533164)
\lineto(195.74255674,92.08533164)
\lineto(195.74255674,89.77066083)
\lineto(191.22521531,89.77066083)
\lineto(191.22521531,86.335988)
\lineto(196.07855734,86.335988)
\closepath
}
}
{
\newrgbcolor{curcolor}{0 0 0}
\pscustom[linestyle=none,fillstyle=solid,fillcolor=curcolor]
{
\newpath
\moveto(126.88723437,75.47644071)
\curveto(128.70412652,75.47644071)(130.04190669,75.14666235)(130.9005749,74.48710561)
\curveto(131.77168757,73.82754887)(132.20724391,72.8257693)(132.20724391,71.48176689)
\curveto(132.20724391,70.87198802)(132.08902147,70.33687595)(131.8525766,69.87643068)
\curveto(131.6285762,69.42842987)(131.32368677,69.0426514)(130.9379083,68.71909527)
\curveto(130.5645743,68.4079836)(130.16012913,68.15287203)(129.72457279,67.95376056)
\lineto(133.64457982,62.14841682)
\lineto(130.5085742,62.14841682)
\lineto(127.33523517,67.26309266)
\lineto(125.82323246,67.26309266)
\lineto(125.82323246,62.14841682)
\lineto(123.00456074,62.14841682)
\lineto(123.00456074,75.47644071)
\closepath
\moveto(126.68190067,73.1617699)
\lineto(125.82323246,73.1617699)
\lineto(125.82323246,69.55909677)
\lineto(126.73790077,69.55909677)
\curveto(127.67123578,69.55909677)(128.33701475,69.71465261)(128.73523768,70.02576428)
\curveto(129.14590509,70.33687595)(129.35123879,70.79732122)(129.35123879,71.40710009)
\curveto(129.35123879,72.04176789)(129.13346062,72.48976869)(128.69790428,72.7511025)
\curveto(128.26234795,73.02488076)(127.59034674,73.1617699)(126.68190067,73.1617699)
\closepath
}
}
{
\newrgbcolor{curcolor}{0 0 0}
\pscustom[linestyle=none,fillstyle=solid,fillcolor=curcolor]
{
\newpath
\moveto(142.99659406,62.14841682)
\lineto(135.32458031,62.14841682)
\lineto(135.32458031,75.47644071)
\lineto(142.99659406,75.47644071)
\lineto(142.99659406,73.1617699)
\lineto(138.14325203,73.1617699)
\lineto(138.14325203,70.23109798)
\lineto(142.66059346,70.23109798)
\lineto(142.66059346,67.91642716)
\lineto(138.14325203,67.91642716)
\lineto(138.14325203,64.48175434)
\lineto(142.99659406,64.48175434)
\closepath
}
}
{
\newrgbcolor{curcolor}{0 0 0}
\pscustom[linestyle=none,fillstyle=solid,fillcolor=curcolor]
{
\newpath
\moveto(153.93526962,62.14841682)
\lineto(152.96460121,65.32175584)
\lineto(148.11125918,65.32175584)
\lineto(147.14059077,62.14841682)
\lineto(144.09791865,62.14841682)
\lineto(148.80192708,75.53244081)
\lineto(152.25526661,75.53244081)
\lineto(156.97794174,62.14841682)
\closepath
\moveto(152.29260001,67.69242676)
\lineto(151.3219316,70.79109898)
\curveto(151.25970927,71.00265492)(151.17882023,71.27021095)(151.0792645,71.59376709)
\curveto(150.97970877,71.91732322)(150.88015303,72.24710159)(150.7805973,72.58310219)
\curveto(150.68104156,72.9191028)(150.60015253,73.21154777)(150.5379302,73.4604371)
\curveto(150.47570786,73.21154777)(150.38859659,72.9004361)(150.27659639,72.52710209)
\curveto(150.17704066,72.16621256)(150.07748493,71.81776749)(149.97792919,71.48176689)
\curveto(149.89081792,71.15821075)(149.82237336,70.92798812)(149.77259549,70.79109898)
\lineto(148.82059378,67.69242676)
\closepath
}
}
{
\newrgbcolor{curcolor}{0 0 0}
\pscustom[linestyle=none,fillstyle=solid,fillcolor=curcolor]
{
\newpath
\moveto(169.70864338,68.94309567)
\curveto(169.70864338,66.69064719)(169.06775334,64.99197748)(167.78597326,63.84708653)
\curveto(166.50419319,62.71464006)(164.71841221,62.14841682)(162.42863033,62.14841682)
\lineto(158.6579569,62.14841682)
\lineto(158.6579569,75.47644071)
\lineto(162.83929773,75.47644071)
\curveto(164.23307801,75.47644071)(165.44019128,75.22755138)(166.46063755,74.72977271)
\curveto(167.4935283,74.23199404)(168.28997417,73.4977705)(168.84997517,72.52710209)
\curveto(169.42242064,71.55643369)(169.70864338,70.36176488)(169.70864338,68.94309567)
\closepath
\moveto(166.77797146,68.86842887)
\curveto(166.77797146,70.33687595)(166.45441532,71.41954455)(165.80730305,72.11643469)
\curveto(165.16019078,72.81332483)(164.22063354,73.1617699)(162.98863133,73.1617699)
\lineto(161.47662862,73.1617699)
\lineto(161.47662862,64.48175434)
\lineto(162.68996413,64.48175434)
\curveto(165.41530235,64.48175434)(166.77797146,65.94397918)(166.77797146,68.86842887)
\closepath
}
}
{
\newrgbcolor{curcolor}{0 0 0}
\pscustom[linestyle=none,fillstyle=solid,fillcolor=curcolor]
{
\newpath
\moveto(178.10865466,62.14841682)
\lineto(174.89798224,72.60176889)
\lineto(174.82331544,72.60176889)
\lineto(174.87931554,71.48176689)
\curveto(174.90420447,70.98398822)(174.9290934,70.44887615)(174.95398234,69.87643068)
\curveto(174.97887127,69.30398521)(174.99131574,68.79376207)(174.99131574,68.34576127)
\lineto(174.99131574,62.14841682)
\lineto(172.47131122,62.14841682)
\lineto(172.47131122,75.47644071)
\lineto(176.31665145,75.47644071)
\lineto(179.47132377,65.28442244)
\lineto(179.52732387,65.28442244)
\lineto(182.86866319,75.47644071)
\lineto(186.71400342,75.47644071)
\lineto(186.71400342,62.14841682)
\lineto(184.0819987,62.14841682)
\lineto(184.0819987,68.45776147)
\curveto(184.0819987,68.88087334)(184.08822093,69.36620754)(184.1006654,69.91376408)
\curveto(184.12555433,70.46132061)(184.14422103,70.97776598)(184.1566655,71.46310019)
\lineto(184.2126656,72.58310219)
\lineto(184.1379988,72.58310219)
\lineto(180.70332598,62.14841682)
\closepath
}
}
{
\newrgbcolor{curcolor}{0 0 0}
\pscustom[linestyle=none,fillstyle=solid,fillcolor=curcolor]
{
\newpath
\moveto(197.74603166,62.14841682)
\lineto(190.07401791,62.14841682)
\lineto(190.07401791,75.47644071)
\lineto(197.74603166,75.47644071)
\lineto(197.74603166,73.1617699)
\lineto(192.89268963,73.1617699)
\lineto(192.89268963,70.23109798)
\lineto(197.41003106,70.23109798)
\lineto(197.41003106,67.91642716)
\lineto(192.89268963,67.91642716)
\lineto(192.89268963,64.48175434)
\lineto(197.74603166,64.48175434)
\closepath
}
}
{
\newrgbcolor{curcolor}{0 0 0}
\pscustom[linestyle=none,fillstyle=solid,fillcolor=curcolor]
{
\newpath
\moveto(199.91136292,63.45508583)
\curveto(199.91136292,64.0275313)(200.06691876,64.42575424)(200.37803043,64.64975464)
\curveto(200.6891421,64.88619951)(201.06869833,65.00442194)(201.51669913,65.00442194)
\curveto(201.95225547,65.00442194)(202.32558947,64.88619951)(202.63670114,64.64975464)
\curveto(202.94781281,64.42575424)(203.10336865,64.0275313)(203.10336865,63.45508583)
\curveto(203.10336865,62.90752929)(202.94781281,62.50930636)(202.63670114,62.26041702)
\curveto(202.32558947,62.02397216)(201.95225547,61.90574972)(201.51669913,61.90574972)
\curveto(201.06869833,61.90574972)(200.6891421,62.02397216)(200.37803043,62.26041702)
\curveto(200.06691876,62.50930636)(199.91136292,62.90752929)(199.91136292,63.45508583)
\closepath
}
}
{
\newrgbcolor{curcolor}{0 0 0}
\pscustom[linestyle=none,fillstyle=solid,fillcolor=curcolor]
{
\newpath
\moveto(217.58872567,72.52710209)
\curveto(218.74606107,72.52710209)(219.61717375,72.22843489)(220.20206368,71.63110049)
\curveto(220.79939809,71.04621055)(221.09806529,70.10043108)(221.09806529,68.79376207)
\lineto(221.09806529,62.14841682)
\lineto(218.31672697,62.14841682)
\lineto(218.31672697,68.10309416)
\curveto(218.31672697,69.57154124)(217.80650383,70.30576478)(216.78605756,70.30576478)
\curveto(216.05183402,70.30576478)(215.52916642,70.04443098)(215.21805475,69.52176337)
\curveto(214.90694308,68.99909577)(214.75138725,68.24620553)(214.75138725,67.26309266)
\lineto(214.75138725,62.14841682)
\lineto(211.97004893,62.14841682)
\lineto(211.97004893,68.10309416)
\curveto(211.97004893,69.57154124)(211.45982579,70.30576478)(210.43937952,70.30576478)
\curveto(209.66782258,70.30576478)(209.13271051,70.01331981)(208.83404331,69.42842987)
\curveto(208.54782057,68.8559844)(208.4047092,68.02842736)(208.4047092,66.94575876)
\lineto(208.4047092,62.14841682)
\lineto(205.62337088,62.14841682)
\lineto(205.62337088,72.34043509)
\lineto(207.7513747,72.34043509)
\lineto(208.1247087,71.03376608)
\lineto(208.2740423,71.03376608)
\curveto(208.58515397,71.55643369)(209.00826584,71.93598992)(209.54337791,72.17243479)
\curveto(210.09093445,72.40887966)(210.65715768,72.52710209)(211.24204762,72.52710209)
\curveto(211.98871563,72.52710209)(212.6171612,72.40265743)(213.12738433,72.15376809)
\curveto(213.65005194,71.91732322)(214.05449711,71.54398922)(214.34071984,71.03376608)
\lineto(214.58338694,71.03376608)
\curveto(214.89449861,71.55643369)(215.32383272,71.93598992)(215.87138925,72.17243479)
\curveto(216.43139026,72.40887966)(217.00383573,72.52710209)(217.58872567,72.52710209)
\closepath
}
}
{
\newrgbcolor{curcolor}{0 0 0}
\pscustom[linestyle=none,fillstyle=solid,fillcolor=curcolor]
{
\newpath
\moveto(227.18341351,61.96174982)
\curveto(226.05096704,61.96174982)(225.12385426,62.40352839)(224.40207519,63.28708553)
\curveto(223.69274059,64.18308714)(223.33807328,65.49597838)(223.33807328,67.22575926)
\curveto(223.33807328,68.9679846)(223.69896282,70.28709808)(224.42074189,71.18309968)
\curveto(225.14252096,72.07910129)(226.08830044,72.52710209)(227.25808031,72.52710209)
\curveto(227.99230385,72.52710209)(228.59586049,72.38399073)(229.06875022,72.09776799)
\curveto(229.54163996,71.81154526)(229.91497396,71.45687795)(230.18875223,71.03376608)
\lineto(230.28208573,71.03376608)
\curveto(230.24475233,71.23287755)(230.2011967,71.51910029)(230.15141883,71.89243429)
\curveto(230.10164096,72.27821276)(230.07675203,72.67021346)(230.07675203,73.0684364)
\lineto(230.07675203,76.33510892)
\lineto(232.85809035,76.33510892)
\lineto(232.85809035,62.14841682)
\lineto(230.73008653,62.14841682)
\lineto(230.18875223,63.47375253)
\lineto(230.07675203,63.47375253)
\curveto(229.80297376,63.05064066)(229.43586199,62.68975113)(228.97541672,62.39108392)
\curveto(228.51497145,62.10486119)(227.91763705,61.96174982)(227.18341351,61.96174982)
\closepath
\moveto(228.15408192,64.18308714)
\curveto(228.91319439,64.18308714)(229.44830646,64.40708754)(229.75941813,64.85508834)
\curveto(230.0705298,65.31553361)(230.2385301,65.99997928)(230.26341903,66.90842536)
\lineto(230.26341903,67.20709256)
\curveto(230.26341903,68.19020543)(230.1078632,68.94309567)(229.79675153,69.46576327)
\curveto(229.49808433,70.00087534)(228.93808332,70.26843138)(228.11674852,70.26843138)
\curveto(227.50696965,70.26843138)(227.02785768,70.00087534)(226.67941261,69.46576327)
\curveto(226.33096754,68.94309567)(226.156745,68.1839832)(226.156745,67.18842586)
\curveto(226.156745,66.19286852)(226.33096754,65.43997828)(226.67941261,64.92975514)
\curveto(227.02785768,64.43197647)(227.51941411,64.18308714)(228.15408192,64.18308714)
\closepath
}
}
{
\newrgbcolor{curcolor}{0 0 0}
\pscustom[linestyle=none,fillstyle=solid,fillcolor=curcolor]
{
\newpath
\moveto(126.62982869,42.09719071)
\curveto(125.11160375,42.09719071)(123.93560164,42.51408034)(123.10182237,43.34785962)
\curveto(122.28048756,44.18163889)(121.86982016,45.5069746)(121.86982016,47.32386674)
\curveto(121.86982016,48.56831342)(122.0813761,49.58253746)(122.50448797,50.36653886)
\curveto(122.92759984,51.15054027)(123.51248977,51.72920797)(124.25915778,52.10254198)
\curveto(125.01827025,52.47587598)(125.88938292,52.66254298)(126.8724958,52.66254298)
\curveto(127.56938593,52.66254298)(128.17294257,52.59409841)(128.68316571,52.45720928)
\curveto(129.20583331,52.32032014)(129.66005635,52.15854208)(130.04583482,51.97187508)
\lineto(129.22450001,49.82520456)
\curveto(128.78894368,49.99942709)(128.37827627,50.14253846)(127.9924978,50.25453866)
\curveto(127.6191638,50.36653886)(127.2458298,50.42253896)(126.8724958,50.42253896)
\curveto(125.42893765,50.42253896)(124.70715858,49.39587046)(124.70715858,47.34253344)
\curveto(124.70715858,46.32208717)(124.89382558,45.56919693)(125.26715958,45.08386273)
\curveto(125.65293805,44.59852852)(126.18805012,44.35586142)(126.8724958,44.35586142)
\curveto(127.45738573,44.35586142)(127.9738311,44.43052822)(128.42183191,44.57986182)
\curveto(128.86983271,44.74163989)(129.30538905,44.95941806)(129.72850092,45.23319633)
\lineto(129.72850092,42.86252541)
\curveto(129.30538905,42.58874714)(128.85738824,42.39585791)(128.38449851,42.28385771)
\curveto(127.92405324,42.15941304)(127.3391633,42.09719071)(126.62982869,42.09719071)
\closepath
}
}
{
\newrgbcolor{curcolor}{0 0 0}
\pscustom[linestyle=none,fillstyle=solid,fillcolor=curcolor]
{
\newpath
\moveto(141.3391871,47.39853354)
\curveto(141.3391871,45.70608607)(140.8911863,44.39941706)(139.99518469,43.47852652)
\curveto(139.11162755,42.55763598)(137.90451428,42.09719071)(136.37384487,42.09719071)
\curveto(135.42806539,42.09719071)(134.58184166,42.30252441)(133.83517365,42.71319181)
\curveto(133.10095011,43.12385921)(132.52228241,43.72119362)(132.09917054,44.50519502)
\curveto(131.67605867,45.3016409)(131.46450273,46.26608707)(131.46450273,47.39853354)
\curveto(131.46450273,49.09098102)(131.9062813,50.3914278)(132.78983844,51.29987387)
\curveto(133.67339558,52.20831994)(134.88673109,52.66254298)(136.42984497,52.66254298)
\curveto(137.38806891,52.66254298)(138.23429265,52.45720928)(138.96851619,52.04654188)
\curveto(139.70273972,51.63587447)(140.28140743,51.03854007)(140.7045193,50.25453866)
\curveto(141.12763117,49.47053726)(141.3391871,48.51853555)(141.3391871,47.39853354)
\closepath
\moveto(134.30184115,47.39853354)
\curveto(134.30184115,46.39053174)(134.46361922,45.62519703)(134.78717536,45.10252943)
\curveto(135.12317596,44.59230629)(135.66451026,44.33719472)(136.41117827,44.33719472)
\curveto(137.14540181,44.33719472)(137.67429164,44.59230629)(137.99784778,45.10252943)
\curveto(138.33384838,45.62519703)(138.50184868,46.39053174)(138.50184868,47.39853354)
\curveto(138.50184868,48.40653535)(138.33384838,49.15942559)(137.99784778,49.65720426)
\curveto(137.67429164,50.1674274)(137.13917957,50.42253896)(136.39251157,50.42253896)
\curveto(135.65828803,50.42253896)(135.12317596,50.1674274)(134.78717536,49.65720426)
\curveto(134.46361922,49.15942559)(134.30184115,48.40653535)(134.30184115,47.39853354)
\closepath
}
}
{
\newrgbcolor{curcolor}{0 0 0}
\pscustom[linestyle=none,fillstyle=solid,fillcolor=curcolor]
{
\newpath
\moveto(149.4218587,52.66254298)
\curveto(150.51697177,52.66254298)(151.39430668,52.36387578)(152.05386341,51.76654137)
\curveto(152.71342015,51.18165144)(153.04319852,50.23587196)(153.04319852,48.92920295)
\lineto(153.04319852,42.28385771)
\lineto(150.2618602,42.28385771)
\lineto(150.2618602,48.23853505)
\curveto(150.2618602,48.97275859)(150.1311933,49.52031512)(149.8698595,49.88120466)
\curveto(149.6085257,50.25453866)(149.19163606,50.44120566)(148.61919059,50.44120566)
\curveto(147.77296685,50.44120566)(147.19429915,50.1487607)(146.88318748,49.56387076)
\curveto(146.57207581,48.99142529)(146.41651997,48.16386825)(146.41651997,47.08119964)
\lineto(146.41651997,42.28385771)
\lineto(143.63518166,42.28385771)
\lineto(143.63518166,52.47587598)
\lineto(145.76318547,52.47587598)
\lineto(146.13651947,51.16920697)
\lineto(146.28585307,51.16920697)
\curveto(146.60940921,51.69187457)(147.05118778,52.07143081)(147.61118878,52.30787568)
\curveto(148.18363425,52.54432055)(148.78719089,52.66254298)(149.4218587,52.66254298)
\closepath
}
}
{
\newrgbcolor{curcolor}{0 0 0}
\pscustom[linestyle=none,fillstyle=solid,fillcolor=curcolor]
{
\newpath
\moveto(161.53654014,50.38520556)
\lineto(159.12853582,50.38520556)
\lineto(159.12853582,42.28385771)
\lineto(156.3471975,42.28385771)
\lineto(156.3471975,50.38520556)
\lineto(154.81652809,50.38520556)
\lineto(154.81652809,51.72920797)
\lineto(156.3471975,52.47587598)
\lineto(156.3471975,53.22254398)
\curveto(156.3471975,54.09365666)(156.49030887,54.76565786)(156.77653161,55.2385476)
\curveto(157.07519881,55.7238818)(157.49208844,56.06610464)(158.02720051,56.2652161)
\curveto(158.57475705,56.46432757)(159.21564709,56.56388331)(159.94987063,56.56388331)
\curveto(160.4849827,56.56388331)(160.97653913,56.52032767)(161.42453994,56.43321641)
\curveto(161.87254074,56.34610514)(162.23343028,56.2465494)(162.50720854,56.1345492)
\lineto(161.79787394,54.09987889)
\curveto(161.586318,54.16210122)(161.34987314,54.21810132)(161.08853933,54.26787919)
\curveto(160.83965,54.33010152)(160.5596495,54.36121269)(160.24853783,54.36121269)
\curveto(159.86275936,54.36121269)(159.57653662,54.24299026)(159.38986962,54.00654539)
\curveto(159.21564709,53.77010052)(159.12853582,53.47143332)(159.12853582,53.11054378)
\lineto(159.12853582,52.47587598)
\lineto(161.53654014,52.47587598)
\closepath
}
}
{
\newrgbcolor{curcolor}{0 0 0}
\pscustom[linestyle=none,fillstyle=solid,fillcolor=curcolor]
{
\newpath
\moveto(164.52319433,56.47054981)
\curveto(164.93386173,56.47054981)(165.28852903,56.37099407)(165.58719623,56.1718826)
\curveto(165.88586344,55.9852156)(166.03519704,55.6305483)(166.03519704,55.1078807)
\curveto(166.03519704,54.59765756)(165.88586344,54.24299026)(165.58719623,54.04387879)
\curveto(165.28852903,53.84476732)(164.93386173,53.74521159)(164.52319433,53.74521159)
\curveto(164.10008246,53.74521159)(163.73919292,53.84476732)(163.44052572,54.04387879)
\curveto(163.15430298,54.24299026)(163.01119162,54.59765756)(163.01119162,55.1078807)
\curveto(163.01119162,55.6305483)(163.15430298,55.9852156)(163.44052572,56.1718826)
\curveto(163.73919292,56.37099407)(164.10008246,56.47054981)(164.52319433,56.47054981)
\closepath
\moveto(165.90453014,52.47587598)
\lineto(165.90453014,42.28385771)
\lineto(163.12319182,42.28385771)
\lineto(163.12319182,52.47587598)
\closepath
}
}
{
\newrgbcolor{curcolor}{0 0 0}
\pscustom[linestyle=none,fillstyle=solid,fillcolor=curcolor]
{
\newpath
\moveto(172.1205456,52.66254298)
\curveto(173.37743674,52.66254298)(174.36054961,52.16476431)(175.06988422,51.16920697)
\lineto(175.14455102,51.16920697)
\lineto(175.36855142,52.47587598)
\lineto(177.72055564,52.47587598)
\lineto(177.72055564,42.26519101)
\curveto(177.72055564,40.8091884)(177.29122153,39.70163086)(176.43255333,38.94251839)
\curveto(175.57388512,38.18340591)(174.30454951,37.80384968)(172.6245465,37.80384968)
\curveto(171.90276743,37.80384968)(171.23076623,37.84740531)(170.60854289,37.93451658)
\curveto(169.99876402,38.02162785)(169.40142961,38.17718368)(168.81653968,38.40118408)
\lineto(168.81653968,40.6225214)
\curveto(170.06098635,40.09985379)(171.38632206,39.83851999)(172.7925468,39.83851999)
\curveto(174.22366048,39.83851999)(174.93921732,40.61007693)(174.93921732,42.15319081)
\lineto(174.93921732,42.35852451)
\curveto(174.93921732,42.55763598)(174.94543955,42.76919191)(174.95788402,42.99319231)
\curveto(174.97032848,43.22963718)(174.98899518,43.43497088)(175.01388412,43.60919342)
\lineto(174.93921732,43.60919342)
\curveto(174.59077225,43.07408135)(174.17388261,42.68830288)(173.68854841,42.45185801)
\curveto(173.20321421,42.21541314)(172.65565767,42.09719071)(172.0458788,42.09719071)
\curveto(170.83876552,42.09719071)(169.89298605,42.55763598)(169.20854038,43.47852652)
\curveto(168.53653917,44.41186152)(168.20053857,45.70608607)(168.20053857,47.36120014)
\curveto(168.20053857,49.02875869)(168.54898364,50.32920546)(169.24587378,51.26254047)
\curveto(169.94276392,52.19587548)(170.90098786,52.66254298)(172.1205456,52.66254298)
\closepath
\moveto(172.9978805,50.40387226)
\curveto(171.6912115,50.40387226)(171.03787699,49.37720376)(171.03787699,47.32386674)
\curveto(171.03787699,45.29541866)(171.70365596,44.28119462)(173.0352139,44.28119462)
\curveto(173.74454851,44.28119462)(174.26721611,44.48030609)(174.60321672,44.87852903)
\curveto(174.95166178,45.28919643)(175.12588432,45.99853103)(175.12588432,47.00653284)
\lineto(175.12588432,47.34253344)
\curveto(175.12588432,48.43764652)(174.95788402,49.22164792)(174.62188342,49.69453766)
\curveto(174.28588281,50.1674274)(173.74454851,50.40387226)(172.9978805,50.40387226)
\closepath
}
}
{
\newrgbcolor{curcolor}{0 0 0}
\pscustom[linestyle=none,fillstyle=solid,fillcolor=curcolor]
{
\newpath
\moveto(180.24055216,43.59052672)
\curveto(180.24055216,44.16297219)(180.39610799,44.56119512)(180.70721966,44.78519553)
\curveto(181.01833133,45.02164039)(181.39788757,45.13986283)(181.84588837,45.13986283)
\curveto(182.28144471,45.13986283)(182.65477871,45.02164039)(182.96589038,44.78519553)
\curveto(183.27700205,44.56119512)(183.43255788,44.16297219)(183.43255788,43.59052672)
\curveto(183.43255788,43.04297018)(183.27700205,42.64474724)(182.96589038,42.39585791)
\curveto(182.65477871,42.15941304)(182.28144471,42.04119061)(181.84588837,42.04119061)
\curveto(181.39788757,42.04119061)(181.01833133,42.15941304)(180.70721966,42.39585791)
\curveto(180.39610799,42.64474724)(180.24055216,43.04297018)(180.24055216,43.59052672)
\closepath
}
}
{
\newrgbcolor{curcolor}{0 0 0}
\pscustom[linestyle=none,fillstyle=solid,fillcolor=curcolor]
{
\newpath
\moveto(185.84055992,55.1078807)
\curveto(185.84055992,55.6305483)(185.98367129,55.9852156)(186.26989402,56.1718826)
\curveto(186.56856122,56.37099407)(186.92945076,56.47054981)(187.35256263,56.47054981)
\curveto(187.76323003,56.47054981)(188.11789734,56.37099407)(188.41656454,56.1718826)
\curveto(188.71523174,55.9852156)(188.86456534,55.6305483)(188.86456534,55.1078807)
\curveto(188.86456534,54.59765756)(188.71523174,54.24299026)(188.41656454,54.04387879)
\curveto(188.11789734,53.84476732)(187.76323003,53.74521159)(187.35256263,53.74521159)
\curveto(186.92945076,53.74521159)(186.56856122,53.84476732)(186.26989402,54.04387879)
\curveto(185.98367129,54.24299026)(185.84055992,54.59765756)(185.84055992,55.1078807)
\closepath
\moveto(185.13122531,37.80384968)
\curveto(184.80766918,37.80384968)(184.47789081,37.82873861)(184.14189021,37.87851648)
\curveto(183.80588961,37.91584988)(183.5258891,37.96562775)(183.3018887,38.02785008)
\lineto(183.3018887,40.21185399)
\curveto(183.5258891,40.14963166)(183.73744504,40.10607603)(183.93655651,40.08118709)
\curveto(184.13566797,40.05629816)(184.35966838,40.04385369)(184.60855771,40.04385369)
\curveto(184.98189171,40.04385369)(185.29922562,40.14963166)(185.56055942,40.3611876)
\curveto(185.82189322,40.57274353)(185.95256012,40.98341093)(185.95256012,41.5931898)
\lineto(185.95256012,52.47587598)
\lineto(188.73389844,52.47587598)
\lineto(188.73389844,41.1825224)
\curveto(188.73389844,40.56029906)(188.61567601,39.99407583)(188.37923114,39.48385269)
\curveto(188.14278627,38.97362955)(187.7570078,38.56918438)(187.22189573,38.27051718)
\curveto(186.69922813,37.95940551)(186.00233799,37.80384968)(185.13122531,37.80384968)
\closepath
}
}
{
\newrgbcolor{curcolor}{0 0 0}
\pscustom[linestyle=none,fillstyle=solid,fillcolor=curcolor]
{
\newpath
\moveto(198.75792073,45.30786313)
\curveto(198.75792073,44.27497239)(198.39080896,43.47852652)(197.65658542,42.91852551)
\curveto(196.93480635,42.37096898)(195.85213774,42.09719071)(194.4085796,42.09719071)
\curveto(193.69924499,42.09719071)(193.08946612,42.14696857)(192.57924299,42.24652431)
\curveto(192.06901985,42.33363558)(191.55879671,42.48296918)(191.04857358,42.69452511)
\lineto(191.04857358,44.99052923)
\curveto(191.59613011,44.74163989)(192.18724228,44.53630619)(192.82191009,44.37452812)
\curveto(193.45657789,44.21275006)(194.0165789,44.13186102)(194.5019131,44.13186102)
\curveto(195.03702517,44.13186102)(195.42280364,44.21275006)(195.65924851,44.37452812)
\curveto(195.89569338,44.53630619)(196.01391581,44.74786213)(196.01391581,45.00919593)
\curveto(196.01391581,45.18341846)(195.96413794,45.3389743)(195.86458221,45.47586343)
\curveto(195.77747094,45.61275256)(195.57835947,45.7683084)(195.2672478,45.94253093)
\curveto(194.95613614,46.11675347)(194.47080193,46.34075387)(193.81124519,46.61453214)
\curveto(193.16413292,46.88831041)(192.63524309,47.15586644)(192.22457568,47.41720024)
\curveto(191.82635275,47.69097851)(191.52768555,48.01453465)(191.32857408,48.38786865)
\curveto(191.12946261,48.77364712)(191.02990688,49.25275909)(191.02990688,49.82520456)
\curveto(191.02990688,50.77098403)(191.39701864,51.48031864)(192.13124218,51.95320837)
\curveto(192.86546572,52.42609811)(193.84235636,52.66254298)(195.0619141,52.66254298)
\curveto(195.69658191,52.66254298)(196.30013854,52.60032065)(196.87258402,52.47587598)
\curveto(197.44502949,52.35143131)(198.03614166,52.14609761)(198.64592053,51.85987487)
\lineto(197.80591902,49.86253796)
\curveto(197.30814035,50.0740939)(196.83525062,50.24831643)(196.38724981,50.38520556)
\curveto(195.93924901,50.53453917)(195.48502597,50.60920597)(195.0245807,50.60920597)
\curveto(194.2032459,50.60920597)(193.79257849,50.38520556)(193.79257849,49.93720476)
\curveto(193.79257849,49.77542669)(193.84235636,49.62609309)(193.9419121,49.48920396)
\curveto(194.0539123,49.36475929)(194.259246,49.22787016)(194.5579132,49.07853656)
\curveto(194.86902487,48.92920295)(195.3232479,48.73009149)(195.92058231,48.48120215)
\curveto(196.50547225,48.24475728)(197.00947315,47.99586795)(197.43258502,47.73453415)
\curveto(197.85569689,47.48564481)(198.17925302,47.16831091)(198.40325343,46.78253244)
\curveto(198.63969829,46.39675397)(198.75792073,45.90519753)(198.75792073,45.30786313)
\closepath
}
}
{
\newrgbcolor{curcolor}{0 0 0}
\pscustom[linestyle=none,fillstyle=solid,fillcolor=curcolor]
{
\newpath
\moveto(210.18193488,47.39853354)
\curveto(210.18193488,45.70608607)(209.73393407,44.39941706)(208.83793247,43.47852652)
\curveto(207.95437533,42.55763598)(206.74726205,42.09719071)(205.21659264,42.09719071)
\curveto(204.27081317,42.09719071)(203.42458943,42.30252441)(202.67792142,42.71319181)
\curveto(201.94369789,43.12385921)(201.36503018,43.72119362)(200.94191831,44.50519502)
\curveto(200.51880644,45.3016409)(200.30725051,46.26608707)(200.30725051,47.39853354)
\curveto(200.30725051,49.09098102)(200.74902908,50.3914278)(201.63258622,51.29987387)
\curveto(202.51614336,52.20831994)(203.72947887,52.66254298)(205.27259274,52.66254298)
\curveto(206.23081668,52.66254298)(207.07704042,52.45720928)(207.81126396,52.04654188)
\curveto(208.5454875,51.63587447)(209.1241552,51.03854007)(209.54726707,50.25453866)
\curveto(209.97037894,49.47053726)(210.18193488,48.51853555)(210.18193488,47.39853354)
\closepath
\moveto(203.14458893,47.39853354)
\curveto(203.14458893,46.39053174)(203.306367,45.62519703)(203.62992313,45.10252943)
\curveto(203.96592373,44.59230629)(204.50725804,44.33719472)(205.25392604,44.33719472)
\curveto(205.98814958,44.33719472)(206.51703942,44.59230629)(206.84059555,45.10252943)
\curveto(207.17659616,45.62519703)(207.34459646,46.39053174)(207.34459646,47.39853354)
\curveto(207.34459646,48.40653535)(207.17659616,49.15942559)(206.84059555,49.65720426)
\curveto(206.51703942,50.1674274)(205.98192735,50.42253896)(205.23525934,50.42253896)
\curveto(204.5010358,50.42253896)(203.96592373,50.1674274)(203.62992313,49.65720426)
\curveto(203.306367,49.15942559)(203.14458893,48.40653535)(203.14458893,47.39853354)
\closepath
}
}
{
\newrgbcolor{curcolor}{0 0 0}
\pscustom[linestyle=none,fillstyle=solid,fillcolor=curcolor]
{
\newpath
\moveto(218.26460838,52.66254298)
\curveto(219.35972145,52.66254298)(220.23705636,52.36387578)(220.8966131,51.76654137)
\curveto(221.55616983,51.18165144)(221.8859482,50.23587196)(221.8859482,48.92920295)
\lineto(221.8859482,42.28385771)
\lineto(219.10460988,42.28385771)
\lineto(219.10460988,48.23853505)
\curveto(219.10460988,48.97275859)(218.97394298,49.52031512)(218.71260918,49.88120466)
\curveto(218.45127538,50.25453866)(218.03438574,50.44120566)(217.46194027,50.44120566)
\curveto(216.61571653,50.44120566)(216.03704883,50.1487607)(215.72593716,49.56387076)
\curveto(215.41482549,48.99142529)(215.25926966,48.16386825)(215.25926966,47.08119964)
\lineto(215.25926966,42.28385771)
\lineto(212.47793134,42.28385771)
\lineto(212.47793134,52.47587598)
\lineto(214.60593515,52.47587598)
\lineto(214.97926915,51.16920697)
\lineto(215.12860276,51.16920697)
\curveto(215.45215889,51.69187457)(215.89393746,52.07143081)(216.45393847,52.30787568)
\curveto(217.02638394,52.54432055)(217.62994057,52.66254298)(218.26460838,52.66254298)
\closepath
}
}
{
\newrgbcolor{curcolor}{0 0 0}
\pscustom[linestyle=none,fillstyle=solid,fillcolor=curcolor]
{
\newpath
\moveto(127.01858865,31.50399875)
\curveto(128.16347959,31.50399875)(129.09059236,31.05599794)(129.79992697,30.15999634)
\curveto(130.50926157,29.2764392)(130.86392887,27.96977019)(130.86392887,26.23998931)
\curveto(130.86392887,24.49776397)(130.49681711,23.17865049)(129.76259357,22.28264888)
\curveto(129.02837003,21.38664728)(128.08881279,20.93864648)(126.94392185,20.93864648)
\curveto(126.20969831,20.93864648)(125.62480837,21.06931338)(125.18925204,21.33064718)
\curveto(124.7536957,21.60442545)(124.3990284,21.90931488)(124.12525013,22.24531548)
\lineto(123.97591653,22.24531548)
\curveto(124.07547226,21.72264788)(124.12525013,21.22486921)(124.12525013,20.75197947)
\lineto(124.12525013,16.64530545)
\lineto(121.34391181,16.64530545)
\lineto(121.34391181,31.31733175)
\lineto(123.60258252,31.31733175)
\lineto(123.99458323,29.99199604)
\lineto(124.12525013,29.99199604)
\curveto(124.3990284,30.40266344)(124.76614017,30.75733074)(125.22658544,31.05599794)
\curveto(125.68703071,31.35466515)(126.28436511,31.50399875)(127.01858865,31.50399875)
\closepath
\moveto(126.12258704,29.28266143)
\curveto(125.40080797,29.28266143)(124.89058483,29.0524388)(124.59191763,28.59199353)
\curveto(124.29325043,28.14399272)(124.13769459,27.46576929)(124.12525013,26.55732321)
\lineto(124.12525013,26.25865601)
\curveto(124.12525013,25.27554314)(124.2683615,24.51643067)(124.55458423,23.9813186)
\curveto(124.85325143,23.45865099)(125.3883635,23.19731719)(126.15992044,23.19731719)
\curveto(126.79458825,23.19731719)(127.26125575,23.45865099)(127.55992295,23.9813186)
\curveto(127.87103462,24.51643067)(128.02659045,25.28176537)(128.02659045,26.27732271)
\curveto(128.02659045,28.28088186)(127.39192265,29.28266143)(126.12258704,29.28266143)
\closepath
}
}
{
\newrgbcolor{curcolor}{0 0 0}
\pscustom[linestyle=none,fillstyle=solid,fillcolor=curcolor]
{
\newpath
\moveto(137.34126296,31.52266545)
\curveto(138.71015431,31.52266545)(139.75548951,31.22399825)(140.47726859,30.62666384)
\curveto(141.21149212,30.0417739)(141.57860389,29.13955006)(141.57860389,27.91999232)
\lineto(141.57860389,21.12531348)
\lineto(139.63726708,21.12531348)
\lineto(139.09593278,22.50664929)
\lineto(139.02126598,22.50664929)
\curveto(138.58570964,21.95909275)(138.12526437,21.56086981)(137.63993017,21.31198048)
\curveto(137.15459596,21.06309114)(136.48881699,20.93864648)(135.64259325,20.93864648)
\curveto(134.73414718,20.93864648)(133.98125694,21.19998028)(133.38392254,21.72264788)
\curveto(132.78658813,22.24531548)(132.48792093,23.06042806)(132.48792093,24.1679856)
\curveto(132.48792093,25.2506542)(132.86747717,26.04710008)(133.62658964,26.55732321)
\curveto(134.38570211,27.06754635)(135.52437082,27.35376909)(137.04259576,27.41599142)
\lineto(138.81593227,27.47199152)
\lineto(138.81593227,27.91999232)
\curveto(138.81593227,28.45510439)(138.67282091,28.8471051)(138.38659817,29.09599443)
\curveto(138.1128199,29.34488377)(137.72704143,29.46932843)(137.22926276,29.46932843)
\curveto(136.73148409,29.46932843)(136.24614989,29.39466163)(135.77326015,29.24532803)
\curveto(135.30037042,29.1084389)(134.82748068,28.93421636)(134.35459094,28.72266043)
\lineto(133.43992264,30.60799714)
\curveto(133.97503471,30.88177541)(134.57859135,31.09955358)(135.25059255,31.26133165)
\curveto(135.92259375,31.43555418)(136.61948389,31.52266545)(137.34126296,31.52266545)
\closepath
\moveto(138.81593227,25.84798861)
\lineto(137.73326367,25.81065521)
\curveto(136.83726206,25.78576628)(136.21503872,25.62398821)(135.86659365,25.32532101)
\curveto(135.51814858,25.0266538)(135.34392605,24.6346531)(135.34392605,24.1493189)
\curveto(135.34392605,23.72620703)(135.46837072,23.42131759)(135.71726005,23.23465059)
\curveto(135.96614939,23.06042806)(136.28970552,22.97331679)(136.68792846,22.97331679)
\curveto(137.28526286,22.97331679)(137.78926377,23.14753932)(138.19993117,23.49598439)
\curveto(138.61059857,23.85687393)(138.81593227,24.36087483)(138.81593227,25.0079871)
\closepath
}
}
{
\newrgbcolor{curcolor}{0 0 0}
\pscustom[linestyle=none,fillstyle=solid,fillcolor=curcolor]
{
\newpath
\moveto(148.57861719,20.93864648)
\curveto(147.06039224,20.93864648)(145.88439014,21.35553611)(145.05061086,22.18931538)
\curveto(144.22927606,23.02309466)(143.81860866,24.34843037)(143.81860866,26.16532251)
\curveto(143.81860866,27.40976919)(144.03016459,28.42399323)(144.45327646,29.20799463)
\curveto(144.87638833,29.99199604)(145.46127827,30.57066374)(146.20794627,30.94399774)
\curveto(146.96705874,31.31733175)(147.83817142,31.50399875)(148.82128429,31.50399875)
\curveto(149.51817443,31.50399875)(150.12173107,31.43555418)(150.6319542,31.29866505)
\curveto(151.15462181,31.16177591)(151.60884484,30.99999784)(151.99462331,30.81333084)
\lineto(151.17328851,28.66666033)
\curveto(150.73773217,28.84088286)(150.32706477,28.98399423)(149.9412863,29.09599443)
\curveto(149.5679523,29.20799463)(149.19461829,29.26399473)(148.82128429,29.26399473)
\curveto(147.37772615,29.26399473)(146.65594708,28.23732623)(146.65594708,26.18398921)
\curveto(146.65594708,25.16354294)(146.84261408,24.4106527)(147.21594808,23.9253185)
\curveto(147.60172655,23.43998429)(148.13683862,23.19731719)(148.82128429,23.19731719)
\curveto(149.40617423,23.19731719)(149.9226196,23.27198399)(150.3706204,23.42131759)
\curveto(150.8186212,23.58309566)(151.25417754,23.80087383)(151.67728941,24.0746521)
\lineto(151.67728941,21.70398118)
\curveto(151.25417754,21.43020291)(150.80617674,21.23731368)(150.333287,21.12531348)
\curveto(149.87284173,21.00086881)(149.28795179,20.93864648)(148.57861719,20.93864648)
\closepath
}
}
{
\newrgbcolor{curcolor}{0 0 0}
\pscustom[linestyle=none,fillstyle=solid,fillcolor=curcolor]
{
\newpath
\moveto(156.81063256,35.31200557)
\lineto(156.81063256,28.96532753)
\curveto(156.81063256,28.57954906)(156.79196586,28.19377059)(156.75463246,27.80799212)
\curveto(156.72974353,27.43465812)(156.69863236,27.05510188)(156.66129896,26.66932341)
\lineto(156.69863236,26.66932341)
\curveto(156.88529936,26.93065722)(157.07818859,27.19199102)(157.27730006,27.45332482)
\curveto(157.47641153,27.71465862)(157.68796747,27.96977019)(157.91196787,28.21865953)
\lineto(160.76797299,31.31733175)
\lineto(163.90397861,31.31733175)
\lineto(159.85330468,26.89332382)
\lineto(164.14664571,21.12531348)
\lineto(160.93597329,21.12531348)
\lineto(158.00530137,25.2506542)
\lineto(156.81063256,24.2986525)
\lineto(156.81063256,21.12531348)
\lineto(154.02929424,21.12531348)
\lineto(154.02929424,35.31200557)
\closepath
}
}
{
\newrgbcolor{curcolor}{0 0 0}
\pscustom[linestyle=none,fillstyle=solid,fillcolor=curcolor]
{
\newpath
\moveto(169.784003,31.52266545)
\curveto(171.15289434,31.52266545)(172.19822955,31.22399825)(172.92000862,30.62666384)
\curveto(173.65423216,30.0417739)(174.02134392,29.13955006)(174.02134392,27.91999232)
\lineto(174.02134392,21.12531348)
\lineto(172.08000711,21.12531348)
\lineto(171.53867281,22.50664929)
\lineto(171.46400601,22.50664929)
\curveto(171.02844967,21.95909275)(170.5680044,21.56086981)(170.0826702,21.31198048)
\curveto(169.59733599,21.06309114)(168.93155702,20.93864648)(168.08533328,20.93864648)
\curveto(167.17688721,20.93864648)(166.42399697,21.19998028)(165.82666257,21.72264788)
\curveto(165.22932816,22.24531548)(164.93066096,23.06042806)(164.93066096,24.1679856)
\curveto(164.93066096,25.2506542)(165.3102172,26.04710008)(166.06932967,26.55732321)
\curveto(166.82844214,27.06754635)(167.96711085,27.35376909)(169.48533579,27.41599142)
\lineto(171.25867231,27.47199152)
\lineto(171.25867231,27.91999232)
\curveto(171.25867231,28.45510439)(171.11556094,28.8471051)(170.8293382,29.09599443)
\curveto(170.55555993,29.34488377)(170.16978146,29.46932843)(169.67200279,29.46932843)
\curveto(169.17422412,29.46932843)(168.68888992,29.39466163)(168.21600018,29.24532803)
\curveto(167.74311045,29.1084389)(167.27022071,28.93421636)(166.79733098,28.72266043)
\lineto(165.88266267,30.60799714)
\curveto(166.41777474,30.88177541)(167.02133138,31.09955358)(167.69333258,31.26133165)
\curveto(168.36533379,31.43555418)(169.06222392,31.52266545)(169.784003,31.52266545)
\closepath
\moveto(171.25867231,25.84798861)
\lineto(170.1760037,25.81065521)
\curveto(169.28000209,25.78576628)(168.65777875,25.62398821)(168.30933369,25.32532101)
\curveto(167.96088862,25.0266538)(167.78666608,24.6346531)(167.78666608,24.1493189)
\curveto(167.78666608,23.72620703)(167.91111075,23.42131759)(168.16000008,23.23465059)
\curveto(168.40888942,23.06042806)(168.73244556,22.97331679)(169.13066849,22.97331679)
\curveto(169.7280029,22.97331679)(170.2320038,23.14753932)(170.6426712,23.49598439)
\curveto(171.0533386,23.85687393)(171.25867231,24.36087483)(171.25867231,25.0079871)
\closepath
}
}
{
\newrgbcolor{curcolor}{0 0 0}
\pscustom[linestyle=none,fillstyle=solid,fillcolor=curcolor]
{
\newpath
\moveto(180.18135667,31.50399875)
\curveto(181.43824781,31.50399875)(182.42136068,31.00622008)(183.13069529,30.01066274)
\lineto(183.20536209,30.01066274)
\lineto(183.42936249,31.31733175)
\lineto(185.78136671,31.31733175)
\lineto(185.78136671,21.10664678)
\curveto(185.78136671,19.65064417)(185.3520326,18.54308663)(184.4933644,17.78397415)
\curveto(183.63469619,17.02486168)(182.36536058,16.64530545)(180.68535757,16.64530545)
\curveto(179.9635785,16.64530545)(179.2915773,16.68886108)(178.66935396,16.77597235)
\curveto(178.05957509,16.86308361)(177.46224068,17.01863945)(176.87735075,17.24263985)
\lineto(176.87735075,19.46397717)
\curveto(178.12179742,18.94130956)(179.44713313,18.67997576)(180.85335787,18.67997576)
\curveto(182.28447155,18.67997576)(183.00002839,19.4515327)(183.00002839,20.99464658)
\lineto(183.00002839,21.19998028)
\curveto(183.00002839,21.39909175)(183.00625062,21.61064768)(183.01869509,21.83464808)
\curveto(183.03113955,22.07109295)(183.04980625,22.27642665)(183.07469519,22.45064919)
\lineto(183.00002839,22.45064919)
\curveto(182.65158332,21.91553712)(182.23469368,21.52975865)(181.74935948,21.29331378)
\curveto(181.26402528,21.05686891)(180.71646874,20.93864648)(180.10668987,20.93864648)
\curveto(178.89957659,20.93864648)(177.95379712,21.39909175)(177.26935145,22.31998228)
\curveto(176.59735024,23.25331729)(176.26134964,24.54754183)(176.26134964,26.20265591)
\curveto(176.26134964,27.87021446)(176.60979471,29.17066123)(177.30668485,30.10399624)
\curveto(178.00357499,31.03733124)(178.96179893,31.50399875)(180.18135667,31.50399875)
\closepath
\moveto(181.05869157,29.24532803)
\curveto(179.75202257,29.24532803)(179.09868806,28.21865953)(179.09868806,26.16532251)
\curveto(179.09868806,24.13687443)(179.76446703,23.12265039)(181.09602497,23.12265039)
\curveto(181.80535958,23.12265039)(182.32802718,23.32176186)(182.66402779,23.71998479)
\curveto(183.01247285,24.1306522)(183.18669539,24.8399868)(183.18669539,25.84798861)
\lineto(183.18669539,26.18398921)
\curveto(183.18669539,27.27910229)(183.01869509,28.06310369)(182.68269449,28.53599343)
\curveto(182.34669388,29.00888316)(181.80535958,29.24532803)(181.05869157,29.24532803)
\closepath
}
}
{
\newrgbcolor{curcolor}{0 0 0}
\pscustom[linestyle=none,fillstyle=solid,fillcolor=curcolor]
{
\newpath
\moveto(192.89336765,31.50399875)
\curveto(194.29959239,31.50399875)(195.41337216,31.09955358)(196.23470697,30.29066324)
\curveto(197.05604178,29.49421737)(197.46670918,28.35554866)(197.46670918,26.87465712)
\lineto(197.46670918,25.53065471)
\lineto(190.89603073,25.53065471)
\curveto(190.92091967,24.7466533)(191.1511423,24.1306522)(191.58669864,23.68265139)
\curveto(192.03469944,23.23465059)(192.65070055,23.01065019)(193.43470195,23.01065019)
\curveto(194.08181422,23.01065019)(194.67292639,23.07287252)(195.20803846,23.19731719)
\curveto(195.755595,23.33420633)(196.315596,23.53954003)(196.88804147,23.8133183)
\lineto(196.88804147,21.66664778)
\curveto(196.37781834,21.41775845)(195.8489285,21.23731368)(195.30137196,21.12531348)
\curveto(194.75381543,21.00086881)(194.08803646,20.93864648)(193.30403505,20.93864648)
\curveto(192.28358878,20.93864648)(191.38136494,21.12531348)(190.59736353,21.49864748)
\curveto(189.81336213,21.88442595)(189.19736102,22.45687142)(188.74936022,23.21598389)
\curveto(188.30135942,23.98754083)(188.07735901,24.96443147)(188.07735901,26.14665581)
\curveto(188.07735901,27.32888015)(188.27647048,28.31821526)(188.67469342,29.11466113)
\curveto(189.08536082,29.911107)(189.65158406,30.50844141)(190.37336313,30.90666434)
\curveto(191.0951422,31.30488728)(191.93514371,31.50399875)(192.89336765,31.50399875)
\closepath
\moveto(192.91203435,29.52532853)
\curveto(192.36447781,29.52532853)(191.91647701,29.351106)(191.56803194,29.00266093)
\curveto(191.21958687,28.65421586)(191.01425317,28.11288156)(190.95203083,27.37865802)
\lineto(194.85337116,27.37865802)
\curveto(194.84092669,27.98843689)(194.67292639,28.49866003)(194.34937026,28.90932743)
\curveto(194.03825859,29.31999483)(193.55914662,29.52532853)(192.91203435,29.52532853)
\closepath
}
}
{
\newrgbcolor{curcolor}{0 0 0}
\pscustom[linestyle=none,fillstyle=solid,fillcolor=curcolor]
{
\newpath
\moveto(199.33337222,22.43198249)
\curveto(199.33337222,23.00442796)(199.48892805,23.40265089)(199.80003972,23.62665129)
\curveto(200.11115139,23.86309616)(200.49070762,23.9813186)(200.93870843,23.9813186)
\curveto(201.37426476,23.9813186)(201.74759877,23.86309616)(202.05871043,23.62665129)
\curveto(202.3698221,23.40265089)(202.52537794,23.00442796)(202.52537794,22.43198249)
\curveto(202.52537794,21.88442595)(202.3698221,21.48620301)(202.05871043,21.23731368)
\curveto(201.74759877,21.00086881)(201.37426476,20.88264637)(200.93870843,20.88264637)
\curveto(200.49070762,20.88264637)(200.11115139,21.00086881)(199.80003972,21.23731368)
\curveto(199.48892805,21.48620301)(199.33337222,21.88442595)(199.33337222,22.43198249)
\closepath
}
}
{
\newrgbcolor{curcolor}{0 0 0}
\pscustom[linestyle=none,fillstyle=solid,fillcolor=curcolor]
{
\newpath
\moveto(204.93337998,33.94933646)
\curveto(204.93337998,34.47200407)(205.07649134,34.82667137)(205.36271408,35.01333837)
\curveto(205.66138128,35.21244984)(206.02227082,35.31200557)(206.44538269,35.31200557)
\curveto(206.85605009,35.31200557)(207.21071739,35.21244984)(207.50938459,35.01333837)
\curveto(207.80805179,34.82667137)(207.9573854,34.47200407)(207.9573854,33.94933646)
\curveto(207.9573854,33.43911333)(207.80805179,33.08444603)(207.50938459,32.88533456)
\curveto(207.21071739,32.68622309)(206.85605009,32.58666736)(206.44538269,32.58666736)
\curveto(206.02227082,32.58666736)(205.66138128,32.68622309)(205.36271408,32.88533456)
\curveto(205.07649134,33.08444603)(204.93337998,33.43911333)(204.93337998,33.94933646)
\closepath
\moveto(204.22404537,16.64530545)
\curveto(203.90048923,16.64530545)(203.57071087,16.67019438)(203.23471026,16.71997225)
\curveto(202.89870966,16.75730565)(202.61870916,16.80708351)(202.39470876,16.86930585)
\lineto(202.39470876,19.05330976)
\curveto(202.61870916,18.99108743)(202.83026509,18.9475318)(203.02937656,18.92264286)
\curveto(203.22848803,18.89775393)(203.45248843,18.88530946)(203.70137777,18.88530946)
\curveto(204.07471177,18.88530946)(204.39204567,18.99108743)(204.65337947,19.20264336)
\curveto(204.91471327,19.4141993)(205.04538018,19.8248667)(205.04538018,20.43464557)
\lineto(205.04538018,31.31733175)
\lineto(207.82671849,31.31733175)
\lineto(207.82671849,20.02397817)
\curveto(207.82671849,19.40175483)(207.70849606,18.83553159)(207.47205119,18.32530846)
\curveto(207.23560632,17.81508532)(206.84982785,17.41064015)(206.31471578,17.11197295)
\curveto(205.79204818,16.80086128)(205.09515804,16.64530545)(204.22404537,16.64530545)
\closepath
}
}
{
\newrgbcolor{curcolor}{0 0 0}
\pscustom[linestyle=none,fillstyle=solid,fillcolor=curcolor]
{
\newpath
\moveto(217.8507446,24.1493189)
\curveto(217.8507446,23.11642816)(217.48363283,22.31998228)(216.74940929,21.75998128)
\curveto(216.02763022,21.21242474)(214.94496161,20.93864648)(213.50140347,20.93864648)
\curveto(212.79206886,20.93864648)(212.18228999,20.98842434)(211.67206686,21.08798008)
\curveto(211.16184372,21.17509134)(210.65162058,21.32442494)(210.14139745,21.53598088)
\lineto(210.14139745,23.831985)
\curveto(210.68895398,23.58309566)(211.28006615,23.37776196)(211.91473396,23.21598389)
\curveto(212.54940176,23.05420582)(213.10940277,22.97331679)(213.59473697,22.97331679)
\curveto(214.12984904,22.97331679)(214.51562751,23.05420582)(214.75207238,23.21598389)
\curveto(214.98851725,23.37776196)(215.10673968,23.58931789)(215.10673968,23.8506517)
\curveto(215.10673968,24.02487423)(215.05696181,24.18043006)(214.95740608,24.3173192)
\curveto(214.87029481,24.45420833)(214.67118334,24.60976417)(214.36007167,24.7839867)
\curveto(214.04896001,24.95820924)(213.5636258,25.18220964)(212.90406906,25.45598791)
\curveto(212.25695679,25.72976617)(211.72806696,25.99732221)(211.31739955,26.25865601)
\curveto(210.91917662,26.53243428)(210.62050942,26.85599042)(210.42139795,27.22932442)
\curveto(210.22228648,27.61510289)(210.12273075,28.09421486)(210.12273075,28.66666033)
\curveto(210.12273075,29.6124398)(210.48984251,30.32177441)(211.22406605,30.79466414)
\curveto(211.95828959,31.26755388)(212.93518023,31.50399875)(214.15473797,31.50399875)
\curveto(214.78940578,31.50399875)(215.39296241,31.44177641)(215.96540789,31.31733175)
\curveto(216.53785336,31.19288708)(217.12896553,30.98755338)(217.7387444,30.70133064)
\lineto(216.89874289,28.70399373)
\curveto(216.40096422,28.91554966)(215.92807449,29.0897722)(215.48007368,29.22666133)
\curveto(215.03207288,29.37599493)(214.57784984,29.45066173)(214.11740457,29.45066173)
\curveto(213.29606977,29.45066173)(212.88540236,29.22666133)(212.88540236,28.77866053)
\curveto(212.88540236,28.61688246)(212.93518023,28.46754886)(213.03473597,28.33065973)
\curveto(213.14673617,28.20621506)(213.35206987,28.06932592)(213.65073707,27.91999232)
\curveto(213.96184874,27.77065872)(214.41607177,27.57154725)(215.01340618,27.32265792)
\curveto(215.59829612,27.08621305)(216.10229702,26.83732372)(216.52540889,26.57598991)
\curveto(216.94852076,26.32710058)(217.27207689,26.00976668)(217.4960773,25.62398821)
\curveto(217.73252216,25.23820974)(217.8507446,24.7466533)(217.8507446,24.1493189)
\closepath
}
}
{
\newrgbcolor{curcolor}{0 0 0}
\pscustom[linestyle=none,fillstyle=solid,fillcolor=curcolor]
{
\newpath
\moveto(229.27475875,26.23998931)
\curveto(229.27475875,24.54754183)(228.82675794,23.24087282)(227.93075634,22.31998228)
\curveto(227.0471992,21.39909175)(225.84008592,20.93864648)(224.30941651,20.93864648)
\curveto(223.36363704,20.93864648)(222.5174133,21.14398018)(221.77074529,21.55464758)
\curveto(221.03652176,21.96531498)(220.45785405,22.56264939)(220.03474218,23.34665079)
\curveto(219.61163031,24.14309666)(219.40007438,25.10754284)(219.40007438,26.23998931)
\curveto(219.40007438,27.93243679)(219.84185295,29.23288357)(220.72541009,30.14132964)
\curveto(221.60896723,31.04977571)(222.82230274,31.50399875)(224.36541661,31.50399875)
\curveto(225.32364055,31.50399875)(226.16986429,31.29866505)(226.90408783,30.88799764)
\curveto(227.63831137,30.47733024)(228.21697907,29.87999584)(228.64009094,29.09599443)
\curveto(229.06320281,28.31199303)(229.27475875,27.35999132)(229.27475875,26.23998931)
\closepath
\moveto(222.2374128,26.23998931)
\curveto(222.2374128,25.2319875)(222.39919087,24.4666528)(222.722747,23.9439852)
\curveto(223.0587476,23.43376206)(223.60008191,23.17865049)(224.34674991,23.17865049)
\curveto(225.08097345,23.17865049)(225.60986329,23.43376206)(225.93341942,23.9439852)
\curveto(226.26942003,24.4666528)(226.43742033,25.2319875)(226.43742033,26.23998931)
\curveto(226.43742033,27.24799112)(226.26942003,28.00088136)(225.93341942,28.49866003)
\curveto(225.60986329,29.00888316)(225.07475122,29.26399473)(224.32808321,29.26399473)
\curveto(223.59385967,29.26399473)(223.0587476,29.00888316)(222.722747,28.49866003)
\curveto(222.39919087,28.00088136)(222.2374128,27.24799112)(222.2374128,26.23998931)
\closepath
}
}
{
\newrgbcolor{curcolor}{0 0 0}
\pscustom[linestyle=none,fillstyle=solid,fillcolor=curcolor]
{
\newpath
\moveto(237.35743225,31.50399875)
\curveto(238.45254532,31.50399875)(239.32988023,31.20533155)(239.98943697,30.60799714)
\curveto(240.6489937,30.0231072)(240.97877207,29.07732773)(240.97877207,27.77065872)
\lineto(240.97877207,21.12531348)
\lineto(238.19743375,21.12531348)
\lineto(238.19743375,27.07999082)
\curveto(238.19743375,27.81421436)(238.06676685,28.36177089)(237.80543305,28.72266043)
\curveto(237.54409925,29.09599443)(237.12720961,29.28266143)(236.55476414,29.28266143)
\curveto(235.7085404,29.28266143)(235.1298727,28.99021646)(234.81876103,28.40532653)
\curveto(234.50764936,27.83288106)(234.35209353,27.00532402)(234.35209353,25.92265541)
\lineto(234.35209353,21.12531348)
\lineto(231.57075521,21.12531348)
\lineto(231.57075521,31.31733175)
\lineto(233.69875902,31.31733175)
\lineto(234.07209303,30.01066274)
\lineto(234.22142663,30.01066274)
\curveto(234.54498276,30.53333034)(234.98676133,30.91288658)(235.54676234,31.14933145)
\curveto(236.11920781,31.38577631)(236.72276444,31.50399875)(237.35743225,31.50399875)
\closepath
}
}
\end{pspicture}
}
		\label{fig:img5}
	\end{center}
\end{figure}

	
	\hfill \break

	\section{\centering Исходный код}
	
	\input{source_code/sc1_s_main.tex}
\input{source_code/sc2_s_router.tex}
\input{source_code/sc3_s_dbRAM.tex}
\input{source_code/sc4_s_database.tex}
\input{source_code/sc5_s_randomID.tex}
\input{source_code/sc6_s_skt_connect.tex}
\input{source_code/sc7_s_skt_disconnect.tex}
\input{source_code/sc8_s_skt_permission.tex}
\begin{lstlisting}[label=sc9_s_skt_users_list,caption={Скрипт сокета передачи списка пользователей клиенту на стороне сервера: users\_list.js}]
    const dbRAM = require('../dbRAM.js');

    module.exports = function (io, socket, rooms) {
        try {
            socket.on('users_list', function (msg) {
                io.to(msg).emit('users_list', dbRAM.getUsers(msg, rooms));
            });
        }
        catch (e) {
            console.log(e);
        }
    };
    
\end{lstlisting}

\begin{lstlisting}[label=sc10_s_skt_messages,caption={Скрипт сокета чата на стороне сервера: messages.js}]
    const dbRAM = require('../dbRAM.js');

    module.exports = function (io, socket, rooms) {
        socket.on('message', function (msg) {
            try {
                dbRAM.addChat(msg.idRoom, msg.userName + ': ' + msg.body, rooms);
                io.to(msg.idRoom).emit('message', msg.userName + ': ' + msg.body);
            }
            catch (e) {
                console.log(e);
            }
        });
    };
    
\end{lstlisting}

\input{source_code/sc11_s_skt_voice.tex}
\begin{lstlisting}[label=sc12_s_skt_xml,caption={Скрипт сокета обмена XML-кода на стороне сервера: xml.js}]
    const dbRAM = require('../dbRAM.js');

    module.exports = function (socket, rooms) {
        socket.on('xml', function (msg) {
            try {
                if (socket.id == dbRAM.getPermissionUserId(msg.idRoom, rooms)) {
                    socket.to(msg.idRoom).emit('xml', msg.data);
                    dbRAM.setXML(msg.idRoom, msg.data, rooms);
                }
            }
            catch (e) {
                console.log(e);
            }
        });
    }
    
\end{lstlisting}

\begin{lstlisting}[label=sc13_s_skt_cursor,caption={Скрипт сокета трансляции координат курсора мыши: cursor.js}]
    const dbRAM = require('../dbRAM.js');

    module.exports = function (socket, rooms) {
        socket.on('cursor', function (msg) {
            try {
                if(socket.id = dbRAM.getPermissionUserId(msg.idRoom, rooms))
                    socket.to(msg.idRoom).emit('cursor', msg);
            }
            catch (e) {
                console.log(e);
            }
        });
    }
    
\end{lstlisting}

\input{source_code/sc14_k_hb_index.tex}
\input{source_code/sc15_k_hb_choiceCreateRoom.tex}
\input{source_code/sc16_k_hb_createRoom.tex}
\input{source_code/sc17_k_hb_connectRoom.tex}
\input{source_code/sc18_k_hb_room.tex}


	\newpage
	\section*{\centering Лицензия проекта}
	\addcontentsline{toc}{section}{Лицензия проекта}
	
	Исходный код проекта распространяется под стандартной общественной лицензией AGPL-3.0.

\noindent
При этом, использование кода, загружаемого с серверов GeoGebra, может накладывать некоторые дополнительные ограничения.

	
	\hfill \break

	\section*{\centering Заключение}
	\addcontentsline{toc}{section}{Заключение}
	
	В ходе выполнения дипломной работы я углубил свои знания в разработке программного
обеспечения, особенностях преподавания математики, особенностях
проведения численных и векторных расчетов на ЭВМ, в частности
преобразование координат.

\noindent
С поставленными задачами помогли справиться следующие дисциплины:
линейная алгебра, аналитическая геометрия, технологии программирования,
численные методы и дискретная математика.


	\newpage
	\addcontentsline{toc}{section}{Список литературы}

	\cite[1]{2022javascript}
	\cite[2]{ларин2022методика}
	\cite[3]{бабичев2022распределенные}
	\cite[4]{kovacs2018using}
	\cite[5]{tamam2021use}
	\cite[6]{peters2017building}
	\cite[7]{hohenwarter2009introducing}
	\cite[7]{львовский2022набор}
	\printbibliography
\end{document}