В качестве основы сервера была выбрана кроссплатформенная среда выполнения JavaScript
с открытым исходным кодом — \emph{Node.js}.

\emph{Node.js} — программная платформа, основанная на движке V8 (компилирующем
JavaScript в машинный код), превращающая JavaScript из узкоспециализированного языка
в язык общего назначения. Node.js добавляет возможность JavaScript взаимодействовать
с устройствами ввода-вывода через свой API, написанный на C++, подключать другие внешние
библиотеки, написанные на разных языках, обеспечивая вызовы к ним из JavaScript-кода.
Node.js применяется преимущественно на сервере, выполняя роль веб-сервера.
В основе Node.js лежит событийно-ориентированное и асинхронное (или реактивное)
программирование с неблокирующим вводом/выводом.

Событиийно-ориентиированное программирование (СОП) — парадигма программирования, в которой
выполнение программы определяется событиями — действиями пользователя (клавиатура, мышь, сенсорный экран),
сообщениями других программ и потоков, событиями операционной системы (например, поступлением сетевого пакета).

Асинхронное программирование — концепция программирования, которая заключается в том,
что результат выполнения функции доступен не сразу, а через некоторое время в виде
некоторого асинхронного (нарушающего обычный порядок выполнения) вызова.

В отличие от синхронного программирования, где компьютер выполняет инструкции последовательно
и ожидает завершения системных операций (обращение к устройствам ввода-вывода, жесткому диску,
сетевой запрос) блокируя следующие операции в потоке выполнения, в асинхронном программировании
длительные операции запускаются без ожидания их завершения и не блокируя дальнейшее выполнение программы.

Асинхронный ввод/вывод является формой неблокирующей обработки ввода/вывода, который позволяет
процессу продолжить выполнение, не дожидаясь окончания передачи данных.

Входные и выходные (I/O) операции на компьютере могут быть весьма медленными, по сравнению с обработкой данных.
Устройство ввода/вывода может быть на несколько порядков медленнее, чем оперативная память.
Например, во время дисковой операции, которой требуется десять миллисекунд для выполнения, процессор,
который работает на частоте один гигагерц, может выполнить десять миллионов циклов команд обработки.

На платформе \emph{Node.js} будет базироваться HTTP-сервер — основа клиент-серверного приложения.

HTTP (англ. HyperText Transfer Protocol) — протокол прикладного уровня передачи данных,
изначально — в виде гипертекстовых документов в формате HTML, в настоящее время используется
для передачи произвольных данных.

Так как стандартный функционал \emph{Node.js} «из коробки» не имеет удобных методов
маршрутизации сервера (процесс определения оптимального маршрута данных в сетях связи),
то было принято решение использовать фреймворк \emph{Express.js} [Исходный код \ref{sc1_s_main}]
[Исходный код \ref{sc2_s_router}].

\emph{Express.js} — это минимальная и гибкая платформа веб-приложений \newline \emph{Node.js},
которая предоставляет надежный и удобный в использовании набор функций для веб-приложений.

Связка \emph{Node.js} и \emph{Express.js} является очень популярной. Её используют многие
крупные IT-компании [Рис. \ref{g4_img1}].

\newpage
\begin{figure}[H]
	\begin{center}
		\includegraphics[width=\linewidth]{g4/img1.png}
		\caption{Компании, использующие связку \emph{Node.js} и \emph{Express.js}}
		\label{g4_img1}
	\end{center}
\end{figure}

Задачей сервера, в случае проекта «\emph{Geometry Room}», является обмен данными между пользователями и их хранение
без каких-либо серьезных вычислений, согласно концепции «сильный клиент — слабый сервер».

В частности, сервер будет обрабатывать данные комнат и находящихся в них пользователей, сохранять их в базу данных и выполнять
маршрутизацию. 

Выбранная платформа является лучшим решение поставленной задачи.
