В первую очередь стоит поговорить об инструменте для работы
с геометрическими объектами. Выбор пал на программу, написанную
Маркусом Хохенвартером с названием «\emph{GeoGebra}».

«\emph{GeoGebra}» — это бесплатная кроссплатформенная динамическая математическая
программа для всех уровней образования, включающая в себя геометрию, алгебру,
таблицы, графы, статистику и арифметику, в одном пакете.

Программа предусматривает возможность работы с функциями (построение графиков,
вычисление корней, экстремумов, интегралов и т. д.) за счёт команд встроенного
языка (который также позволяет управлять и геометрическими построениями).

Данная программа, помимо своей большой популярности, является стандартом
в этой сфере.

К примеру: во время пандемии COVID-19 на математическом факультете ВГУ дистанционные занятия
по аналитической геометрии успешно проводились с использованием программы
«\emph{GeoGebra}», только транслировалась она через сторонний мессенджер,
используя демонстрацию экрана.

\begin{figure}[h]
	\begin{center}
		\includegraphics[width=\linewidth]{g1/img1.png}
		\caption{Пример работы программы «\emph{GeoGebra}»}
		\label{g1_img1}
	\end{center}
\end{figure}

Возможности «\emph{GeoGebra}»:
\begin{enumerate} 
    \item Построение кривых:
        \begin{itemize}
            \item Графиков функций $y=f(x)$.
            \item Кривых, заданных параметрически в декартовой системе координат: $x=f(t); \quad y=g(t)$.
            \item Конических сечений:
                \begin{itemize}
                    \item Коника произвольного вида — по пяти точкам.
                    \item Окружность:
                        \begin{itemize}
                            \item — по центру и точке на ней;
                            \item — по центру и радиусу;
                            \item — по трем точкам;
                        \end{itemize}
                    \item Эллипс — по двум фокусам и точке на кривой.
                    \item Парабола — по фокусу и директрисе.
                    \item Гипербола — по двум фокусам и точке на кривой.
                \end{itemize}
            \item Построение геометрического места точек, зависящих от положения некоторой другой точки,
                принадлежащей какой-либо кривой или многоугольнику.
        \end{itemize}
    \item Вычисления:
        \begin{itemize}
            \item Действия с матрицами:
                \begin{itemize}
                    \item Сложение, умножение;
                    \item Транспонирование, инвертирование;
                    \item Вычисление определителя;
                \end{itemize}
            \item Вычисления с комплексными числами;
            \item Нахождение точек пересечения кривых;
            \item Статистические функции:
                \begin{itemize}
                    \item Вычисление математического ожидания, дисперсии;
                    \item Вычисление коэффициента корреляции;
                \end{itemize}
            \item Аппроксимация множества точек кривой заданного вида:
                \begin{itemize}
                    \item полином;
                    \item экспонента;
                    \item логарифм;
                    \item синусоида;
                \end{itemize}
        \end{itemize}
\end{enumerate}

Помимо всего вышеизложенного, данная программа была включена в проект из-за того,
что имеет специальное и открытое API (\emph{GeoGebra Apps API}) для работы в браузере.
API предполагает создание объекта «\emph{GeoGebra}» — апплета, который представляет
собой отдельное окно программы, работающее на странице в браузере. На официальном сайте в документации
подробно описано API апплетов.

Именно функционал программы «\emph{GeoGebra}» и наличие открытого API для работы в браузере
выделяет ее на фоне всех остальных.
