Немаловажной задачей стала возможность записи результатов работы с комнатами.
Это связано с тем, чтобы после выхода всех участников из комнаты была
возможность вернуться в эту комнату и не потерять всю историю действий.
Было рассмотрено несколько библиотечных баз данных, взаимодействующих
с платформой \emph{Node.js}, но все они оказались слишком тяжелыми
для такой простой задачи. Пример: БД \emph{MySQL} требует отдельный
сервер для работы и большое количество оперативной памяти, что
является недопустимым для проекта, т.к. он должен
работать на маломощном сервере. Поэтому решено было написать собственную
базу данных.

Структура базы данных представляла систему, состоящую из двух блоков.
Первый блок [Исходный код \ref{sc4_s_database}] должен был выполнять функцию хранения данных на локальном
диске с помощью \emph{json}-файлов. Второй блок [Исходный код \ref{sc3_s_dbRAM}] хранил данные в
оперативной памяти. Взаимодействуют они следующим образом: пока
в комнате есть участники — данные комнаты хранятся в оперативной
памяти для быстрого доступа к ним. Как только все участники покидают
комнату — данные записываются на локальный диск в \emph{json}-файл и
освобождаются из оперативной памяти. Соответственно, если в пустую комнату
заходит хотя бы один участник — данные этой комнаты считываются с
\emph{json}-файла и помещаются в оперативную память для последующего использования.
