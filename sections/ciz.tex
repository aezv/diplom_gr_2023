

Цель дипломной работы: создание удобного веб-сервиса для многопользовательской работы с
геометрическими объектами.

Поставленная цель включает в себя выполнение определенного ряда задач. Сервис должен быть
удобным и быстрым для работы. Подходить как для проведения онлайн-занятий в учебных заведениях,
так и для простых онлайн-встреч, где могут на примере обсуждаться разные вопросы по геометрии.

Задачи дипломной работы:
\begin{enumerate} 
    \item Выбор и анализ инструмента для работы с геометрическими объектами. Он должен быть популярным и
    простым в использовании. Кроме того, он должен иметь открытый исходный код и свободную лицензию для
    возможности включить его в проект.
    \item Создание структуры проекта. Структура определяет тип приложения, его архитектуру и платформу.
    \item Определение схемы взаимодействия между пользователями. То есть собрать данные, как лучше организовать
    механизм многопользовательской работы и составить его схему.
    \item Выбор платформы проекта (сервера). Так как предполагается создание онлайн-сервиса,
    то он должен базироваться на клиент-серверной архитектуре. На этом основании подбирается платформа.
    \item Создание веб-интерфейса (клиента). Он должен быть функциональным и понятным пользователю.
    \item Настройка механизма непрерывного обмена данными между клиентом и сервером.
    \item Тестирование и отладка приложения.
\end{enumerate}

Представленные задачи требуют внимательного исследования и определенный подход к решению.
