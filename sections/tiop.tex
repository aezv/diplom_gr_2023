На момент написания дипломной работы произведена отладка и тестирование проекта
«\emph{Geometry Room}». Были выявлены следующие проблемы в порядке важности, требующие решения:
\begin{enumerate}
    \item Мерцание апплета «\emph{GeoGebra}» при его обновлении,
    что происходит довольно часто (при любых масштабных изменениях данных апплета).
    Данная проблема крайне заметна и мешает долгому использованию приложения.
    Проявляется она по причине того, что API апплетов не предназначено для постоянного
    редактирования данных и требует постоянного обновления окна апплета.

    Решение данной проблемы найдено, но еще не реализовано. Оно заключается в
    создании механизма двойной буферизации. Создаются два апплета: один
    отображается у пользователя, другой принимает данные и подготавливается к отображению.
    Далее они меняются местами. Смена апплетов будет осуществляться средствами
    \emph{JavaScript} и \emph{CSS}, в таком варианте мерцание
    на специально подготовленном для тестирования апплете не проявляется.

    \item Потеря некоторых функций, таких как рисование пером на апплете и вставка изображений.
    Это связано с использованием XML-кода, который, в рамках данного случая, не поддерживает эти
    операции.
    
    Решение проблемы заключается в большем использовании возможностей JSON-файлов при обмене данных между
    апплетами. Это требует многих оптимизаций, так как файлы данного формата имеют большой вес.

    \item Масштабируемость веб-интерфейса работает не на всех разрешениях и не на всех форматах экрана,
    а только на тех, которые были доступны для продолжительного тестирования: разрешение не ниже FHD c 
    форматом экрана 16:9.

    Решение данной проблемы в изучении и тестировании средств \emph{CSS} для адаптивной верстки.

    \item Проблема в работе голосового канала: замечены случаи потери пакетов данных.

    Решение на данный момент пока не найдено.
\end{enumerate}
