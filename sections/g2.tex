Первоочередная задача проекта - создание многопользовательского сервиса, включающего
в себя инструмент для работы с геометрическими объектами. Логичным решением поставленной
задачи будет приложения с клиент-серверной архитектурой.

<<Клиент -- сервер>> (англ. client -- server) — вычислительная или сетевая архитектура,
в которой задания или сетевая нагрузка распределены между поставщиками услуг, называемыми
серверами, и заказчиками услуг, называемыми клиентами.

\noindent
Фактически клиент и сервер — это программное обеспечение. Обычно программы расположены на
разных вычислительных машинах и взаимодействуют между собой через вычислительную сеть
посредством сетевых протоколов, но они могут быть расположены также и на одной машине.

\noindent
Программы-серверы ожидают от клиентских программ запросы и предоставляют им свои ресурсы
в виде данных (например, передача файлов посредством HTTP, FTP, BitTorrent, потоковое
мультимедиа или работа с базами данных) или в виде сервисных функций (например, работа
с электронной почтой, общение посредством систем мгновенного обмена сообщениями или
просмотр web-страниц во всемирной паутине).

\noindent
Поскольку одна программа-~сервер может
выполнять запросы от множества программ-клиентов, её размещают на специально выделенной
вычислительной машине, настроенной особым образом, как правило, совместно с другими
программами-серверами, поэтому производительность этой машины должна быть высокой.
Из-за особой роли такой машины в сети, специфики её оборудования и программного обеспечения,
её также называют сервером, а машины, выполняющие клиентские программы, соответственно, клиентами.

\newpage
\begin{figure}[h]
	\begin{center}
		\includegraphics[width=\linewidth]{g2/img1.png}
		\caption{<<клиент -- сервер>>}
		\label{g2_img1}
	\end{center}
\end{figure}

Клиенты и серверы обмениваются сообщениями в шаблоне запрос-ответ. Клиент отправляет запрос,
а сервер возвращает ответ. Этот обмен сообщениями является примером межпроцессного взаимодействия.
Для взаимодействия компьютеры должны иметь общий язык, и они должны следовать правилам,
чтобы и клиент, и сервер знали, чего ожидать.

\newpage
\begin{figure}[h]
	\begin{center}
		\includegraphics[width=\linewidth]{g2/img2.png}
		\caption{<<запрос-ответ>>}
		\label{g2_img2}
	\end{center}
\end{figure}

В нашем случае клиентом будет считаться страница в браузере пользователя, на которой будет располагаться
веб-интерфейс и инструмент <<GeoGebra>>. Сервер же будет средством обмена данными между клиентами.
Данная схема соответствует концепции <<сильный клиент>>, где большая часть обработки информации поручается
клиенту (пользователю). У такой концепции, в рамках проекта и поставленной задачи, есть существенный плюс: вычислительная
нагрузка распределяется на всю группу пользователей, что позволяет использовать менее производительный сервер и
обеспечить более высокую скорость обмена данными.
