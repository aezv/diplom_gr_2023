Помимо основной маршрутизации запрос/ответ, необходимо организовать схему
непрерывного обмена данными, в частности синхронизации апплетов всех пользователей.
Для этой работы используются веб-сокеты.

\emph{WebSocket} — протокол связи поверх TCP-соединения, предназначенный для обмена сообщениями
между браузером и веб-сервером, используя постоянное соединение.

\noindent
В настоящее время в W3C (Консорциум Всемирной паутины) осуществляется стандартизация API Web Sockets.
Черновой вариант стандарта этого протокола утверждён IETF.

\noindent
\emph{WebSocket} разработан для воплощения в веб-браузерах и веб-серверах, но он может быть
использован для любого клиентского или серверного приложения.
Протокол \emph{WebSocket} — это независимый протокол, основанный на протоколе TCP.
Он делает возможным более тесное взаимодействие между браузером и веб-сайтом,
способствуя распространению интерактивного содержимого и созданию приложений реального времени.

Для работы с веб-сокетами была использована библиотека \emph{socket.io},
специально созданная для платформы \emph{Node.js}.

В проекте веб-сокеты необходимы для взаимодействия веб-интерфейса с
сервером (Рис.~\ref{g6_ink1}), помимо основной маршрутизации фреймворка \emph{Express}.

\begin{figure}[h]
	\begin{center}
		\scalebox{0.6}{%LaTeX with PSTricks extensions
%%Creator: Inkscape 1.2 (dc2aedaf03, 2022-05-15)
%%Please note this file requires PSTricks extensions
\psset{xunit=.5pt,yunit=.5pt,runit=.5pt}
\begin{pspicture}(1024,768)
{
\newrgbcolor{curcolor}{0.50196081 0.50196081 0.50196081}
\pscustom[linestyle=none,fillstyle=solid,fillcolor=curcolor]
{
\newpath
\moveto(742.67016602,374.88122559)
\lineto(988.87597656,374.88122559)
\lineto(988.87597656,232.02107239)
\lineto(742.67016602,232.02107239)
\closepath
}
}
{
\newrgbcolor{curcolor}{0.50196081 0.50196081 0.50196081}
\pscustom[linestyle=none,fillstyle=solid,fillcolor=curcolor]
{
\newpath
\moveto(592.21105957,635.77835083)
\lineto(838.41687012,635.77835083)
\lineto(838.41687012,492.91819763)
\lineto(592.21105957,492.91819763)
\closepath
}
}
{
\newrgbcolor{curcolor}{0.50196081 0.50196081 0.50196081}
\pscustom[linestyle=none,fillstyle=solid,fillcolor=curcolor]
{
\newpath
\moveto(408.82321167,277.10818481)
\lineto(655.02902222,277.10818481)
\lineto(655.02902222,134.24803162)
\lineto(408.82321167,134.24803162)
\closepath
}
}
{
\newrgbcolor{curcolor}{0.50196081 0.50196081 0.50196081}
\pscustom[linestyle=none,fillstyle=solid,fillcolor=curcolor]
{
\newpath
\moveto(164.64379883,635.77836609)
\lineto(410.84960938,635.77836609)
\lineto(410.84960938,492.91821289)
\lineto(164.64379883,492.91821289)
\closepath
}
}
{
\newrgbcolor{curcolor}{0.50196081 0.50196081 0.50196081}
\pscustom[linestyle=none,fillstyle=solid,fillcolor=curcolor]
{
\newpath
\moveto(48.12664413,277.10818481)
\lineto(294.33245468,277.10818481)
\lineto(294.33245468,134.24803162)
\lineto(48.12664413,134.24803162)
\closepath
}
}
{
\newrgbcolor{curcolor}{0.50196081 0.50196081 0.50196081}
\pscustom[linestyle=none,fillstyle=solid,fillcolor=curcolor]
{
\newpath
\moveto(256.70527257,543.55182169)
\lineto(266.24469283,540.13782162)
\lineto(158.70369581,239.64609834)
\lineto(149.16427556,243.06009841)
\closepath
}
}
{
\newrgbcolor{curcolor}{0.50196081 0.50196081 0.50196081}
\pscustom[linestyle=none,fillstyle=solid,fillcolor=curcolor]
{
\newpath
\moveto(316.78465992,510.45248412)
\lineto(325.30889862,515.92909712)
\lineto(497.82219956,247.41559131)
\lineto(489.29796086,241.93897831)
\closepath
}
}
{
\newrgbcolor{curcolor}{0.50196081 0.50196081 0.50196081}
\pscustom[linestyle=none,fillstyle=solid,fillcolor=curcolor]
{
\newpath
\moveto(698.15065346,511.80182843)
\lineto(707.00958234,506.88498854)
\lineto(552.12913365,227.82874258)
\lineto(543.27020477,232.74558246)
\closepath
}
}
{
\newrgbcolor{curcolor}{0.50196081 0.50196081 0.50196081}
\pscustom[linestyle=none,fillstyle=solid,fillcolor=curcolor]
{
\newpath
\moveto(705.04066337,551.46836852)
\lineto(713.16396437,557.5237724)
\lineto(903.90917705,301.6398033)
\lineto(895.78587605,295.58439943)
\closepath
}
}
{
\newrgbcolor{curcolor}{0 0 0}
\pscustom[linestyle=none,fillstyle=solid,fillcolor=curcolor]
{
\newpath
\moveto(238.73911644,593.88350288)
\curveto(236.88311644,593.88350288)(235.46444978,593.19016954)(234.48311644,591.80350288)
\curveto(233.50178311,590.41683621)(233.01111644,588.51816954)(233.01111644,586.10750288)
\curveto(233.01111644,583.67550288)(233.45911644,581.78750288)(234.35511644,580.44350288)
\curveto(235.27244978,579.12083621)(236.73378311,578.45950288)(238.73911644,578.45950288)
\curveto(239.65644978,578.45950288)(240.58444978,578.56616954)(241.52311644,578.77950288)
\curveto(242.46178311,578.99283621)(243.47511644,579.29150288)(244.56311644,579.67550288)
\lineto(244.56311644,575.61150288)
\curveto(243.56044978,575.20616954)(242.56844978,574.90750288)(241.58711644,574.71550288)
\curveto(240.60578311,574.52350288)(239.50711644,574.42750288)(238.29111644,574.42750288)
\curveto(235.92311644,574.42750288)(233.98178311,574.90750288)(232.46711644,575.86750288)
\curveto(230.95244978,576.84883621)(229.83244978,578.21416954)(229.10711644,579.96350288)
\curveto(228.38178311,581.73416954)(228.01911644,583.79283621)(228.01911644,586.13950288)
\curveto(228.01911644,588.44350288)(228.43511644,590.48083621)(229.26711644,592.25150288)
\curveto(230.09911644,594.02216954)(231.30444978,595.40883621)(232.88311644,596.41150288)
\curveto(234.48311644,597.41416954)(236.43511644,597.91550288)(238.73911644,597.91550288)
\curveto(239.86978311,597.91550288)(241.00044978,597.76616954)(242.13111644,597.46750288)
\curveto(243.28311644,597.19016954)(244.38178311,596.80616954)(245.42711644,596.31550288)
\lineto(243.85911644,592.37950288)
\curveto(243.00578311,592.78483621)(242.14178311,593.13683621)(241.26711644,593.43550288)
\curveto(240.41378311,593.73416954)(239.57111644,593.88350288)(238.73911644,593.88350288)
\closepath
}
}
{
\newrgbcolor{curcolor}{0 0 0}
\pscustom[linestyle=none,fillstyle=solid,fillcolor=curcolor]
{
\newpath
\moveto(264.91508861,583.51550288)
\curveto(264.91508861,580.61416954)(264.14708861,578.37416954)(262.61108861,576.79550288)
\curveto(261.09642194,575.21683621)(259.02708861,574.42750288)(256.40308861,574.42750288)
\curveto(254.78175528,574.42750288)(253.33108861,574.77950288)(252.05108861,575.48350288)
\curveto(250.79242194,576.18750288)(249.80042194,577.21150288)(249.07508861,578.55550288)
\curveto(248.34975528,579.92083621)(247.98708861,581.57416954)(247.98708861,583.51550288)
\curveto(247.98708861,586.41683621)(248.74442194,588.64616954)(250.25908861,590.20350288)
\curveto(251.77375528,591.76083621)(253.85375528,592.53950288)(256.49908861,592.53950288)
\curveto(258.14175528,592.53950288)(259.59242194,592.18750288)(260.85108861,591.48350288)
\curveto(262.10975528,590.77950288)(263.10175528,589.75550288)(263.82708861,588.41150288)
\curveto(264.55242194,587.06750288)(264.91508861,585.43550288)(264.91508861,583.51550288)
\closepath
\moveto(252.85108861,583.51550288)
\curveto(252.85108861,581.78750288)(253.12842194,580.47550288)(253.68308861,579.57950288)
\curveto(254.25908861,578.70483621)(255.18708861,578.26750288)(256.46708861,578.26750288)
\curveto(257.72575528,578.26750288)(258.63242194,578.70483621)(259.18708861,579.57950288)
\curveto(259.76308861,580.47550288)(260.05108861,581.78750288)(260.05108861,583.51550288)
\curveto(260.05108861,585.24350288)(259.76308861,586.53416954)(259.18708861,587.38750288)
\curveto(258.63242194,588.26216954)(257.71508861,588.69950288)(256.43508861,588.69950288)
\curveto(255.17642194,588.69950288)(254.25908861,588.26216954)(253.68308861,587.38750288)
\curveto(253.12842194,586.53416954)(252.85108861,585.24350288)(252.85108861,583.51550288)
\closepath
}
}
{
\newrgbcolor{curcolor}{0 0 0}
\pscustom[linestyle=none,fillstyle=solid,fillcolor=curcolor]
{
\newpath
\moveto(280.30707201,592.21950288)
\lineto(285.55507201,592.21950288)
\lineto(278.64307201,583.83550288)
\lineto(286.16307201,574.74750288)
\lineto(280.75507201,574.74750288)
\lineto(273.61907201,583.61150288)
\lineto(273.61907201,574.74750288)
\lineto(268.85107201,574.74750288)
\lineto(268.85107201,592.21950288)
\lineto(273.61907201,592.21950288)
\lineto(273.61907201,583.73950288)
\closepath
}
}
{
\newrgbcolor{curcolor}{0 0 0}
\pscustom[linestyle=none,fillstyle=solid,fillcolor=curcolor]
{
\newpath
\moveto(295.21904076,592.53950288)
\curveto(297.62970743,592.53950288)(299.53904076,591.84616954)(300.94704076,590.45950288)
\curveto(302.35504076,589.09416954)(303.05904076,587.14216954)(303.05904076,584.60350288)
\lineto(303.05904076,582.29950288)
\lineto(291.79504076,582.29950288)
\curveto(291.83770743,580.95550288)(292.23237409,579.89950288)(292.97904076,579.13150288)
\curveto(293.74704076,578.36350288)(294.80304076,577.97950288)(296.14704076,577.97950288)
\curveto(297.25637409,577.97950288)(298.26970743,578.08616954)(299.18704076,578.29950288)
\curveto(300.12570743,578.53416954)(301.08570743,578.88616954)(302.06704076,579.35550288)
\lineto(302.06704076,575.67550288)
\curveto(301.19237409,575.24883621)(300.28570743,574.93950288)(299.34704076,574.74750288)
\curveto(298.40837409,574.53416954)(297.26704076,574.42750288)(295.92304076,574.42750288)
\curveto(294.17370743,574.42750288)(292.62704076,574.74750288)(291.28304076,575.38750288)
\curveto(289.93904076,576.04883621)(288.88304076,577.03016954)(288.11504076,578.33150288)
\curveto(287.34704076,579.65416954)(286.96304076,581.32883621)(286.96304076,583.35550288)
\curveto(286.96304076,585.38216954)(287.30437409,587.07816954)(287.98704076,588.44350288)
\curveto(288.69104076,589.80883621)(289.66170743,590.83283621)(290.89904076,591.51550288)
\curveto(292.13637409,592.19816954)(293.57637409,592.53950288)(295.21904076,592.53950288)
\closepath
\moveto(295.25104076,589.14750288)
\curveto(294.31237409,589.14750288)(293.54437409,588.84883621)(292.94704076,588.25150288)
\curveto(292.34970743,587.65416954)(291.99770743,586.72616954)(291.89104076,585.46750288)
\lineto(298.57904076,585.46750288)
\curveto(298.55770743,586.51283621)(298.26970743,587.38750288)(297.71504076,588.09150288)
\curveto(297.18170743,588.79550288)(296.36037409,589.14750288)(295.25104076,589.14750288)
\closepath
}
}
{
\newrgbcolor{curcolor}{0 0 0}
\pscustom[linestyle=none,fillstyle=solid,fillcolor=curcolor]
{
\newpath
\moveto(321.39502806,588.63550288)
\lineto(315.66702806,588.63550288)
\lineto(315.66702806,574.74750288)
\lineto(310.89902806,574.74750288)
\lineto(310.89902806,588.63550288)
\lineto(305.17102806,588.63550288)
\lineto(305.17102806,592.21950288)
\lineto(321.39502806,592.21950288)
\closepath
}
}
{
\newrgbcolor{curcolor}{0 0 0}
\pscustom[linestyle=none,fillstyle=solid,fillcolor=curcolor]
{
\newpath
\moveto(324.62701195,574.74750288)
\lineto(324.62701195,592.21950288)
\lineto(329.39501195,592.21950288)
\lineto(329.39501195,585.46750288)
\lineto(331.69901195,585.46750288)
\curveto(334.36567862,585.46750288)(336.33901195,585.04083621)(337.61901195,584.18750288)
\curveto(338.89901195,583.33416954)(339.53901195,582.04350288)(339.53901195,580.31550288)
\curveto(339.53901195,578.60883621)(338.94167862,577.25416954)(337.74701195,576.25150288)
\curveto(336.55234528,575.24883621)(334.58967862,574.74750288)(331.85901195,574.74750288)
\closepath
\moveto(342.06701195,574.74750288)
\lineto(342.06701195,592.21950288)
\lineto(346.83501195,592.21950288)
\lineto(346.83501195,574.74750288)
\closepath
\moveto(329.39501195,578.04350288)
\lineto(331.60301195,578.04350288)
\curveto(332.54167862,578.04350288)(333.29901195,578.20350288)(333.87501195,578.52350288)
\curveto(334.47234528,578.86483621)(334.77101195,579.44083621)(334.77101195,580.25150288)
\curveto(334.77101195,581.53150288)(333.69367862,582.17150288)(331.53901195,582.17150288)
\lineto(329.39501195,582.17150288)
\closepath
}
}
{
\newrgbcolor{curcolor}{0 0 0}
\pscustom[linestyle=none,fillstyle=solid,fillcolor=curcolor]
{
\newpath
\moveto(205.57104613,543.51550288)
\curveto(205.57104613,546.11816954)(205.94437946,548.62483621)(206.69104613,551.03550288)
\curveto(207.45904613,553.46750288)(208.6537128,555.65416954)(210.27504613,557.59550288)
\lineto(214.17904613,557.59550288)
\curveto(212.72837946,555.59016954)(211.61904613,553.38216954)(210.85104613,550.97150288)
\curveto(210.10437946,548.56083621)(209.73104613,546.08616954)(209.73104613,543.54750288)
\curveto(209.73104613,541.07283621)(210.10437946,538.63016954)(210.85104613,536.21950288)
\curveto(211.5977128,533.83016954)(212.69637946,531.65416954)(214.14704613,529.69150288)
\lineto(210.27504613,529.69150288)
\curveto(208.6537128,531.56883621)(207.45904613,533.69150288)(206.69104613,536.05950288)
\curveto(205.94437946,538.44883621)(205.57104613,540.93416954)(205.57104613,543.51550288)
\closepath
}
}
{
\newrgbcolor{curcolor}{0 0 0}
\pscustom[linestyle=none,fillstyle=solid,fillcolor=curcolor]
{
\newpath
\moveto(231.49106859,541.08350288)
\curveto(231.49106859,539.05683621)(230.75506859,537.43550288)(229.28306859,536.21950288)
\curveto(227.83240192,535.02483621)(225.76306859,534.42750288)(223.07506859,534.42750288)
\curveto(220.66440192,534.42750288)(218.50973526,534.88616954)(216.61106859,535.80350288)
\lineto(216.61106859,540.31550288)
\curveto(217.69906859,539.84616954)(218.81906859,539.40883621)(219.97106859,539.00350288)
\curveto(221.14440192,538.61950288)(222.30706859,538.42750288)(223.45906859,538.42750288)
\curveto(224.65373526,538.42750288)(225.49640192,538.65150288)(225.98706859,539.09950288)
\curveto(226.49906859,539.56883621)(226.75506859,540.15550288)(226.75506859,540.85950288)
\curveto(226.75506859,541.43550288)(226.55240192,541.92616954)(226.14706859,542.33150288)
\curveto(225.76306859,542.73683621)(225.24040192,543.11016954)(224.57906859,543.45150288)
\curveto(223.91773526,543.81416954)(223.16040192,544.19816954)(222.30706859,544.60350288)
\curveto(221.77373526,544.85950288)(221.19773526,545.15816954)(220.57906859,545.49950288)
\curveto(219.96040192,545.86216954)(219.36306859,546.29950288)(218.78706859,546.81150288)
\curveto(218.23240192,547.34483621)(217.77373526,547.98483621)(217.41106859,548.73150288)
\curveto(217.04840192,549.47816954)(216.86706859,550.37416954)(216.86706859,551.41950288)
\curveto(216.86706859,553.46750288)(217.56040192,555.05683621)(218.94706859,556.18750288)
\curveto(220.33373526,557.33950288)(222.22173526,557.91550288)(224.61106859,557.91550288)
\curveto(225.80573526,557.91550288)(226.93640192,557.77683621)(228.00306859,557.49950288)
\curveto(229.06973526,557.22216954)(230.20040192,556.82750288)(231.39506859,556.31550288)
\lineto(229.82706859,552.53950288)
\curveto(228.78173526,552.96616954)(227.84306859,553.29683621)(227.01106859,553.53150288)
\curveto(226.17906859,553.76616954)(225.32573526,553.88350288)(224.45106859,553.88350288)
\curveto(223.53373526,553.88350288)(222.82973526,553.67016954)(222.33906859,553.24350288)
\curveto(221.84840192,552.81683621)(221.60306859,552.26216954)(221.60306859,551.57950288)
\curveto(221.60306859,550.76883621)(221.96573526,550.12883621)(222.69106859,549.65950288)
\curveto(223.41640192,549.19016954)(224.49373526,548.61416954)(225.92306859,547.93150288)
\curveto(227.09640192,547.37683621)(228.08840192,546.80083621)(228.89906859,546.20350288)
\curveto(229.73106859,545.60616954)(230.37106859,544.90216954)(230.81906859,544.09150288)
\curveto(231.26706859,543.28083621)(231.49106859,542.27816954)(231.49106859,541.08350288)
\closepath
}
}
{
\newrgbcolor{curcolor}{0 0 0}
\pscustom[linestyle=none,fillstyle=solid,fillcolor=curcolor]
{
\newpath
\moveto(251.13908763,543.51550288)
\curveto(251.13908763,540.61416954)(250.37108763,538.37416954)(248.83508763,536.79550288)
\curveto(247.32042097,535.21683621)(245.25108763,534.42750288)(242.62708763,534.42750288)
\curveto(241.0057543,534.42750288)(239.55508763,534.77950288)(238.27508763,535.48350288)
\curveto(237.01642097,536.18750288)(236.02442097,537.21150288)(235.29908763,538.55550288)
\curveto(234.5737543,539.92083621)(234.21108763,541.57416954)(234.21108763,543.51550288)
\curveto(234.21108763,546.41683621)(234.96842097,548.64616954)(236.48308763,550.20350288)
\curveto(237.9977543,551.76083621)(240.0777543,552.53950288)(242.72308763,552.53950288)
\curveto(244.3657543,552.53950288)(245.81642097,552.18750288)(247.07508763,551.48350288)
\curveto(248.3337543,550.77950288)(249.3257543,549.75550288)(250.05108763,548.41150288)
\curveto(250.77642097,547.06750288)(251.13908763,545.43550288)(251.13908763,543.51550288)
\closepath
\moveto(239.07508763,543.51550288)
\curveto(239.07508763,541.78750288)(239.35242097,540.47550288)(239.90708763,539.57950288)
\curveto(240.48308763,538.70483621)(241.41108763,538.26750288)(242.69108763,538.26750288)
\curveto(243.9497543,538.26750288)(244.85642097,538.70483621)(245.41108763,539.57950288)
\curveto(245.98708763,540.47550288)(246.27508763,541.78750288)(246.27508763,543.51550288)
\curveto(246.27508763,545.24350288)(245.98708763,546.53416954)(245.41108763,547.38750288)
\curveto(244.85642097,548.26216954)(243.93908763,548.69950288)(242.65908763,548.69950288)
\curveto(241.40042097,548.69950288)(240.48308763,548.26216954)(239.90708763,547.38750288)
\curveto(239.35242097,546.53416954)(239.07508763,545.24350288)(239.07508763,543.51550288)
\closepath
}
}
{
\newrgbcolor{curcolor}{0 0 0}
\pscustom[linestyle=none,fillstyle=solid,fillcolor=curcolor]
{
\newpath
\moveto(262.17907103,534.42750288)
\curveto(259.57640437,534.42750288)(257.56040437,535.14216954)(256.13107103,536.57150288)
\curveto(254.72307103,538.00083621)(254.01907103,540.27283621)(254.01907103,543.38750288)
\curveto(254.01907103,545.52083621)(254.3817377,547.25950288)(255.10707103,548.60350288)
\curveto(255.83240437,549.94750288)(256.83507103,550.93950288)(258.11507103,551.57950288)
\curveto(259.41640437,552.21950288)(260.9097377,552.53950288)(262.59507103,552.53950288)
\curveto(263.7897377,552.53950288)(264.82440437,552.42216954)(265.69907103,552.18750288)
\curveto(266.59507103,551.95283621)(267.3737377,551.67550288)(268.03507103,551.35550288)
\lineto(266.62707103,547.67550288)
\curveto(265.88040437,547.97416954)(265.17640437,548.21950288)(264.51507103,548.41150288)
\curveto(263.87507103,548.60350288)(263.23507103,548.69950288)(262.59507103,548.69950288)
\curveto(260.12040437,548.69950288)(258.88307103,546.93950288)(258.88307103,543.41950288)
\curveto(258.88307103,541.67016954)(259.20307103,540.37950288)(259.84307103,539.54750288)
\curveto(260.50440437,538.71550288)(261.4217377,538.29950288)(262.59507103,538.29950288)
\curveto(263.5977377,538.29950288)(264.48307103,538.42750288)(265.25107103,538.68350288)
\curveto(266.01907103,538.96083621)(266.7657377,539.33416954)(267.49107103,539.80350288)
\lineto(267.49107103,535.73950288)
\curveto(266.7657377,535.27016954)(265.9977377,534.93950288)(265.18707103,534.74750288)
\curveto(264.3977377,534.53416954)(263.39507103,534.42750288)(262.17907103,534.42750288)
\closepath
}
}
{
\newrgbcolor{curcolor}{0 0 0}
\pscustom[linestyle=none,fillstyle=solid,fillcolor=curcolor]
{
\newpath
\moveto(276.29106908,559.06750288)
\lineto(276.29106908,548.18750288)
\curveto(276.29106908,547.52616954)(276.25906908,546.86483621)(276.19506908,546.20350288)
\curveto(276.15240241,545.56350288)(276.09906908,544.91283621)(276.03506908,544.25150288)
\lineto(276.09906908,544.25150288)
\curveto(276.41906908,544.69950288)(276.74973575,545.14750288)(277.09106908,545.59550288)
\curveto(277.43240241,546.04350288)(277.79506908,546.48083621)(278.17906908,546.90750288)
\lineto(283.07506908,552.21950288)
\lineto(288.45106908,552.21950288)
\lineto(281.50706908,544.63550288)
\lineto(288.86706908,534.74750288)
\lineto(283.36306908,534.74750288)
\lineto(278.33906908,541.81950288)
\lineto(276.29106908,540.18750288)
\lineto(276.29106908,534.74750288)
\lineto(271.52306908,534.74750288)
\lineto(271.52306908,559.06750288)
\closepath
}
}
{
\newrgbcolor{curcolor}{0 0 0}
\pscustom[linestyle=none,fillstyle=solid,fillcolor=curcolor]
{
\newpath
\moveto(298.56309594,552.53950288)
\curveto(300.9737626,552.53950288)(302.88309594,551.84616954)(304.29109594,550.45950288)
\curveto(305.69909594,549.09416954)(306.40309594,547.14216954)(306.40309594,544.60350288)
\lineto(306.40309594,542.29950288)
\lineto(295.13909594,542.29950288)
\curveto(295.1817626,540.95550288)(295.57642927,539.89950288)(296.32309594,539.13150288)
\curveto(297.09109594,538.36350288)(298.14709594,537.97950288)(299.49109594,537.97950288)
\curveto(300.60042927,537.97950288)(301.6137626,538.08616954)(302.53109594,538.29950288)
\curveto(303.4697626,538.53416954)(304.4297626,538.88616954)(305.41109594,539.35550288)
\lineto(305.41109594,535.67550288)
\curveto(304.53642927,535.24883621)(303.6297626,534.93950288)(302.69109594,534.74750288)
\curveto(301.75242927,534.53416954)(300.61109594,534.42750288)(299.26709594,534.42750288)
\curveto(297.5177626,534.42750288)(295.97109594,534.74750288)(294.62709594,535.38750288)
\curveto(293.28309594,536.04883621)(292.22709594,537.03016954)(291.45909594,538.33150288)
\curveto(290.69109594,539.65416954)(290.30709594,541.32883621)(290.30709594,543.35550288)
\curveto(290.30709594,545.38216954)(290.64842927,547.07816954)(291.33109594,548.44350288)
\curveto(292.03509594,549.80883621)(293.0057626,550.83283621)(294.24309594,551.51550288)
\curveto(295.48042927,552.19816954)(296.92042927,552.53950288)(298.56309594,552.53950288)
\closepath
\moveto(298.59509594,549.14750288)
\curveto(297.65642927,549.14750288)(296.88842927,548.84883621)(296.29109594,548.25150288)
\curveto(295.6937626,547.65416954)(295.3417626,546.72616954)(295.23509594,545.46750288)
\lineto(301.92309594,545.46750288)
\curveto(301.9017626,546.51283621)(301.6137626,547.38750288)(301.05909594,548.09150288)
\curveto(300.5257626,548.79550288)(299.70442927,549.14750288)(298.59509594,549.14750288)
\closepath
}
}
{
\newrgbcolor{curcolor}{0 0 0}
\pscustom[linestyle=none,fillstyle=solid,fillcolor=curcolor]
{
\newpath
\moveto(317.63508324,538.23550288)
\curveto(318.16841657,538.23550288)(318.68041657,538.27816954)(319.17108324,538.36350288)
\curveto(319.66174991,538.47016954)(320.15241657,538.60883621)(320.64308324,538.77950288)
\lineto(320.64308324,535.22750288)
\curveto(320.13108324,534.99283621)(319.49108324,534.80083621)(318.72308324,534.65150288)
\curveto(317.97641657,534.50216954)(317.15508324,534.42750288)(316.25908324,534.42750288)
\curveto(315.21374991,534.42750288)(314.27508324,534.59816954)(313.44308324,534.93950288)
\curveto(312.63241657,535.28083621)(311.98174991,535.86750288)(311.49108324,536.69950288)
\curveto(311.02174991,537.53150288)(310.78708324,538.70483621)(310.78708324,540.21950288)
\lineto(310.78708324,548.63550288)
\lineto(308.51508324,548.63550288)
\lineto(308.51508324,550.65150288)
\lineto(311.13908324,552.25150288)
\lineto(312.51508324,555.93150288)
\lineto(315.55508324,555.93150288)
\lineto(315.55508324,552.21950288)
\lineto(320.45108324,552.21950288)
\lineto(320.45108324,548.63550288)
\lineto(315.55508324,548.63550288)
\lineto(315.55508324,540.21950288)
\curveto(315.55508324,539.55816954)(315.74708324,539.05683621)(316.13108324,538.71550288)
\curveto(316.51508324,538.39550288)(317.01641657,538.23550288)(317.63508324,538.23550288)
\closepath
}
}
{
\newrgbcolor{curcolor}{0 0 0}
\pscustom[linestyle=none,fillstyle=solid,fillcolor=curcolor]
{
\newpath
\moveto(323.49108373,536.98750288)
\curveto(323.49108373,537.96883621)(323.75775039,538.65150288)(324.29108373,539.03550288)
\curveto(324.82441706,539.44083621)(325.47508373,539.64350288)(326.24308373,539.64350288)
\curveto(326.98975039,539.64350288)(327.62975039,539.44083621)(328.16308373,539.03550288)
\curveto(328.69641706,538.65150288)(328.96308373,537.96883621)(328.96308373,536.98750288)
\curveto(328.96308373,536.04883621)(328.69641706,535.36616954)(328.16308373,534.93950288)
\curveto(327.62975039,534.53416954)(326.98975039,534.33150288)(326.24308373,534.33150288)
\curveto(325.47508373,534.33150288)(324.82441706,534.53416954)(324.29108373,534.93950288)
\curveto(323.75775039,535.36616954)(323.49108373,536.04883621)(323.49108373,536.98750288)
\closepath
}
}
{
\newrgbcolor{curcolor}{0 0 0}
\pscustom[linestyle=none,fillstyle=solid,fillcolor=curcolor]
{
\newpath
\moveto(335.68307885,559.06750288)
\curveto(336.38707885,559.06750288)(336.99507885,558.89683621)(337.50707885,558.55550288)
\curveto(338.01907885,558.23550288)(338.27507885,557.62750288)(338.27507885,556.73150288)
\curveto(338.27507885,555.85683621)(338.01907885,555.24883621)(337.50707885,554.90750288)
\curveto(336.99507885,554.56616954)(336.38707885,554.39550288)(335.68307885,554.39550288)
\curveto(334.95774551,554.39550288)(334.33907885,554.56616954)(333.82707885,554.90750288)
\curveto(333.33641218,555.24883621)(333.09107885,555.85683621)(333.09107885,556.73150288)
\curveto(333.09107885,557.62750288)(333.33641218,558.23550288)(333.82707885,558.55550288)
\curveto(334.33907885,558.89683621)(334.95774551,559.06750288)(335.68307885,559.06750288)
\closepath
\moveto(338.05107885,552.21950288)
\lineto(338.05107885,534.74750288)
\lineto(333.28307885,534.74750288)
\lineto(333.28307885,552.21950288)
\closepath
}
}
{
\newrgbcolor{curcolor}{0 0 0}
\pscustom[linestyle=none,fillstyle=solid,fillcolor=curcolor]
{
\newpath
\moveto(358.91508861,543.51550288)
\curveto(358.91508861,540.61416954)(358.14708861,538.37416954)(356.61108861,536.79550288)
\curveto(355.09642194,535.21683621)(353.02708861,534.42750288)(350.40308861,534.42750288)
\curveto(348.78175528,534.42750288)(347.33108861,534.77950288)(346.05108861,535.48350288)
\curveto(344.79242194,536.18750288)(343.80042194,537.21150288)(343.07508861,538.55550288)
\curveto(342.34975528,539.92083621)(341.98708861,541.57416954)(341.98708861,543.51550288)
\curveto(341.98708861,546.41683621)(342.74442194,548.64616954)(344.25908861,550.20350288)
\curveto(345.77375528,551.76083621)(347.85375528,552.53950288)(350.49908861,552.53950288)
\curveto(352.14175528,552.53950288)(353.59242194,552.18750288)(354.85108861,551.48350288)
\curveto(356.10975528,550.77950288)(357.10175528,549.75550288)(357.82708861,548.41150288)
\curveto(358.55242194,547.06750288)(358.91508861,545.43550288)(358.91508861,543.51550288)
\closepath
\moveto(346.85108861,543.51550288)
\curveto(346.85108861,541.78750288)(347.12842194,540.47550288)(347.68308861,539.57950288)
\curveto(348.25908861,538.70483621)(349.18708861,538.26750288)(350.46708861,538.26750288)
\curveto(351.72575528,538.26750288)(352.63242194,538.70483621)(353.18708861,539.57950288)
\curveto(353.76308861,540.47550288)(354.05108861,541.78750288)(354.05108861,543.51550288)
\curveto(354.05108861,545.24350288)(353.76308861,546.53416954)(353.18708861,547.38750288)
\curveto(352.63242194,548.26216954)(351.71508861,548.69950288)(350.43508861,548.69950288)
\curveto(349.17642194,548.69950288)(348.25908861,548.26216954)(347.68308861,547.38750288)
\curveto(347.12842194,546.53416954)(346.85108861,545.24350288)(346.85108861,543.51550288)
\closepath
}
}
{
\newrgbcolor{curcolor}{0 0 0}
\pscustom[linestyle=none,fillstyle=solid,fillcolor=curcolor]
{
\newpath
\moveto(369.92307201,543.51550288)
\curveto(369.92307201,540.93416954)(369.53907201,538.44883621)(368.77107201,536.05950288)
\curveto(368.02440534,533.69150288)(366.84040534,531.56883621)(365.21907201,529.69150288)
\lineto(361.34707201,529.69150288)
\curveto(362.77640534,531.65416954)(363.86440534,533.83016954)(364.61107201,536.21950288)
\curveto(365.37907201,538.63016954)(365.76307201,541.07283621)(365.76307201,543.54750288)
\curveto(365.76307201,546.08616954)(365.37907201,548.56083621)(364.61107201,550.97150288)
\curveto(363.86440534,553.38216954)(362.76573868,555.59016954)(361.31507201,557.59550288)
\lineto(365.21907201,557.59550288)
\curveto(366.84040534,555.65416954)(368.02440534,553.46750288)(368.77107201,551.03550288)
\curveto(369.53907201,548.62483621)(369.92307201,546.11816954)(369.92307201,543.51550288)
\closepath
}
}
{
\newrgbcolor{curcolor}{0 0 0}
\pscustom[linestyle=none,fillstyle=solid,fillcolor=curcolor]
{
\newpath
\moveto(129.2191004,216.07562688)
\lineto(123.7151004,216.07562688)
\lineto(115.4271004,227.65962688)
\lineto(115.4271004,216.07562688)
\lineto(110.5951004,216.07562688)
\lineto(110.5951004,238.92362688)
\lineto(115.4271004,238.92362688)
\lineto(115.4271004,227.85162688)
\lineto(123.6191004,238.92362688)
\lineto(128.7711004,238.92362688)
\lineto(120.4511004,227.94762688)
\closepath
}
}
{
\newrgbcolor{curcolor}{0 0 0}
\pscustom[linestyle=none,fillstyle=solid,fillcolor=curcolor]
{
\newpath
\moveto(147.36312872,216.07562688)
\lineto(142.59512872,216.07562688)
\lineto(142.59512872,229.96362688)
\lineto(138.21112872,229.96362688)
\curveto(137.93379538,226.55029354)(137.56046205,223.79829354)(137.09112872,221.70762688)
\curveto(136.64312872,219.63829354)(136.00312872,218.12362688)(135.17112872,217.16362688)
\curveto(134.36046205,216.22496021)(133.28312872,215.75562688)(131.93912872,215.75562688)
\curveto(130.82979538,215.75562688)(129.92312872,215.92629354)(129.21912872,216.26762688)
\lineto(129.21912872,220.07562688)
\curveto(129.70979538,219.86229354)(130.22179538,219.75562688)(130.75512872,219.75562688)
\curveto(131.13912872,219.75562688)(131.49112872,219.94762688)(131.81112872,220.33162688)
\curveto(132.13112872,220.71562688)(132.42979538,221.40896021)(132.70712872,222.41162688)
\curveto(133.00579538,223.41429354)(133.27246205,224.81162688)(133.50712872,226.60362688)
\curveto(133.74179538,228.41696021)(133.95512872,230.73162688)(134.14712872,233.54762688)
\lineto(147.36312872,233.54762688)
\closepath
}
}
{
\newrgbcolor{curcolor}{0 0 0}
\pscustom[linestyle=none,fillstyle=solid,fillcolor=curcolor]
{
\newpath
\moveto(156.96314336,233.54762688)
\lineto(156.96314336,226.63562688)
\curveto(156.96314336,226.27296021)(156.94181003,225.82496021)(156.89914336,225.29162688)
\curveto(156.87781003,224.75829354)(156.84581003,224.21429354)(156.80314336,223.65962688)
\curveto(156.78181003,223.10496021)(156.74981003,222.60362688)(156.70714336,222.15562688)
\curveto(156.6644767,221.72896021)(156.6324767,221.44096021)(156.61114336,221.29162688)
\lineto(164.67514336,233.54762688)
\lineto(170.40314336,233.54762688)
\lineto(170.40314336,216.07562688)
\lineto(165.79514336,216.07562688)
\lineto(165.79514336,223.05162688)
\curveto(165.79514336,223.60629354)(165.8164767,224.23562688)(165.85914336,224.93962688)
\curveto(165.90181003,225.64362688)(165.9444767,226.29429354)(165.98714336,226.89162688)
\curveto(166.05114336,227.51029354)(166.09381003,227.97962688)(166.11514336,228.29962688)
\lineto(158.08314336,216.07562688)
\lineto(152.35514336,216.07562688)
\lineto(152.35514336,233.54762688)
\closepath
}
}
{
\newrgbcolor{curcolor}{0 0 0}
\pscustom[linestyle=none,fillstyle=solid,fillcolor=curcolor]
{
\newpath
\moveto(182.59512139,233.86762688)
\curveto(185.00578806,233.86762688)(186.91512139,233.17429354)(188.32312139,231.78762688)
\curveto(189.73112139,230.42229354)(190.43512139,228.47029354)(190.43512139,225.93162688)
\lineto(190.43512139,223.62762688)
\lineto(179.17112139,223.62762688)
\curveto(179.21378806,222.28362688)(179.60845473,221.22762688)(180.35512139,220.45962688)
\curveto(181.12312139,219.69162688)(182.17912139,219.30762688)(183.52312139,219.30762688)
\curveto(184.63245473,219.30762688)(185.64578806,219.41429354)(186.56312139,219.62762688)
\curveto(187.50178806,219.86229354)(188.46178806,220.21429354)(189.44312139,220.68362688)
\lineto(189.44312139,217.00362688)
\curveto(188.56845473,216.57696021)(187.66178806,216.26762688)(186.72312139,216.07562688)
\curveto(185.78445473,215.86229354)(184.64312139,215.75562688)(183.29912139,215.75562688)
\curveto(181.54978806,215.75562688)(180.00312139,216.07562688)(178.65912139,216.71562688)
\curveto(177.31512139,217.37696021)(176.25912139,218.35829354)(175.49112139,219.65962688)
\curveto(174.72312139,220.98229354)(174.33912139,222.65696021)(174.33912139,224.68362688)
\curveto(174.33912139,226.71029354)(174.68045473,228.40629354)(175.36312139,229.77162688)
\curveto(176.06712139,231.13696021)(177.03778806,232.16096021)(178.27512139,232.84362688)
\curveto(179.51245473,233.52629354)(180.95245473,233.86762688)(182.59512139,233.86762688)
\closepath
\moveto(182.62712139,230.47562688)
\curveto(181.68845473,230.47562688)(180.92045473,230.17696021)(180.32312139,229.57962688)
\curveto(179.72578806,228.98229354)(179.37378806,228.05429354)(179.26712139,226.79562688)
\lineto(185.95512139,226.79562688)
\curveto(185.93378806,227.84096021)(185.64578806,228.71562688)(185.09112139,229.41962688)
\curveto(184.55778806,230.12362688)(183.73645473,230.47562688)(182.62712139,230.47562688)
\closepath
}
}
{
\newrgbcolor{curcolor}{0 0 0}
\pscustom[linestyle=none,fillstyle=solid,fillcolor=curcolor]
{
\newpath
\moveto(199.0751087,233.54762688)
\lineto(199.0751087,226.82762688)
\lineto(205.7311087,226.82762688)
\lineto(205.7311087,233.54762688)
\lineto(210.4991087,233.54762688)
\lineto(210.4991087,216.07562688)
\lineto(205.7311087,216.07562688)
\lineto(205.7311087,223.27562688)
\lineto(199.0751087,223.27562688)
\lineto(199.0751087,216.07562688)
\lineto(194.3071087,216.07562688)
\lineto(194.3071087,233.54762688)
\closepath
}
}
{
\newrgbcolor{curcolor}{0 0 0}
\pscustom[linestyle=none,fillstyle=solid,fillcolor=curcolor]
{
\newpath
\moveto(229.95512969,229.96362688)
\lineto(224.22712969,229.96362688)
\lineto(224.22712969,216.07562688)
\lineto(219.45912969,216.07562688)
\lineto(219.45912969,229.96362688)
\lineto(213.73112969,229.96362688)
\lineto(213.73112969,233.54762688)
\lineto(229.95512969,233.54762688)
\closepath
}
}
{
\newrgbcolor{curcolor}{0 0 0}
\pscustom[linestyle=none,fillstyle=solid,fillcolor=curcolor]
{
\newpath
\moveto(92.06709136,184.84362688)
\curveto(92.06709136,187.44629354)(92.4404247,189.95296021)(93.18709136,192.36362688)
\curveto(93.95509136,194.79562688)(95.14975803,196.98229354)(96.77109136,198.92362688)
\lineto(100.67509136,198.92362688)
\curveto(99.2244247,196.91829354)(98.11509136,194.71029354)(97.34709136,192.29962688)
\curveto(96.6004247,189.88896021)(96.22709136,187.41429354)(96.22709136,184.87562688)
\curveto(96.22709136,182.40096021)(96.6004247,179.95829354)(97.34709136,177.54762688)
\curveto(98.09375803,175.15829354)(99.1924247,172.98229354)(100.64309136,171.01962688)
\lineto(96.77109136,171.01962688)
\curveto(95.14975803,172.89696021)(93.95509136,175.01962688)(93.18709136,177.38762688)
\curveto(92.4404247,179.77696021)(92.06709136,182.26229354)(92.06709136,184.84362688)
\closepath
}
}
{
\newrgbcolor{curcolor}{0 0 0}
\pscustom[linestyle=none,fillstyle=solid,fillcolor=curcolor]
{
\newpath
\moveto(104.51511382,176.07562688)
\lineto(104.51511382,198.92362688)
\lineto(118.94711382,198.92362688)
\lineto(118.94711382,194.92362688)
\lineto(109.34711382,194.92362688)
\lineto(109.34711382,190.15562688)
\lineto(111.26711382,190.15562688)
\curveto(113.42178049,190.15562688)(115.18178049,189.85696021)(116.54711382,189.25962688)
\curveto(117.93378049,188.66229354)(118.95778049,187.84096021)(119.61911382,186.79562688)
\curveto(120.28044716,185.75029354)(120.61111382,184.55562688)(120.61111382,183.21162688)
\curveto(120.61111382,180.95029354)(119.85378049,179.19029354)(118.33911382,177.93162688)
\curveto(116.84578049,176.69429354)(114.45644716,176.07562688)(111.17111382,176.07562688)
\closepath
\moveto(109.34711382,180.04362688)
\lineto(110.97911382,180.04362688)
\curveto(112.45111382,180.04362688)(113.60311382,180.27829354)(114.43511382,180.74762688)
\curveto(115.28844716,181.21696021)(115.71511382,182.03829354)(115.71511382,183.21162688)
\curveto(115.71511382,184.42762688)(115.25644716,185.22762688)(114.33911382,185.61162688)
\curveto(113.42178049,185.99562688)(112.17378049,186.18762688)(110.59511382,186.18762688)
\lineto(109.34711382,186.18762688)
\closepath
}
}
{
\newrgbcolor{curcolor}{0 0 0}
\pscustom[linestyle=none,fillstyle=solid,fillcolor=curcolor]
{
\newpath
\moveto(134.27512945,193.86762688)
\curveto(136.23779612,193.86762688)(137.82712945,193.09962688)(139.04312945,191.56362688)
\curveto(140.25912945,190.04896021)(140.86712945,187.80896021)(140.86712945,184.84362688)
\curveto(140.86712945,181.85696021)(140.23779612,179.59562688)(138.97912945,178.05962688)
\curveto(137.72046278,176.52362688)(136.10979612,175.75562688)(134.14712945,175.75562688)
\curveto(132.88846278,175.75562688)(131.88579612,175.97962688)(131.13912945,176.42762688)
\curveto(130.39246278,176.89696021)(129.78446278,177.41962688)(129.31512945,177.99562688)
\lineto(129.05912945,177.99562688)
\curveto(129.22979612,177.09962688)(129.31512945,176.24629354)(129.31512945,175.43562688)
\lineto(129.31512945,168.39562688)
\lineto(124.54712945,168.39562688)
\lineto(124.54712945,193.54762688)
\lineto(128.41912945,193.54762688)
\lineto(129.09112945,191.27562688)
\lineto(129.31512945,191.27562688)
\curveto(129.78446278,191.97962688)(130.41379612,192.58762688)(131.20312945,193.09962688)
\curveto(131.99246278,193.61162688)(133.01646278,193.86762688)(134.27512945,193.86762688)
\closepath
\moveto(132.73912945,190.05962688)
\curveto(131.50179612,190.05962688)(130.62712945,189.66496021)(130.11512945,188.87562688)
\curveto(129.60312945,188.10762688)(129.33646278,186.94496021)(129.31512945,185.38762688)
\lineto(129.31512945,184.87562688)
\curveto(129.31512945,183.19029354)(129.56046278,181.88896021)(130.05112945,180.97162688)
\curveto(130.56312945,180.07562688)(131.48046278,179.62762688)(132.80312945,179.62762688)
\curveto(133.89112945,179.62762688)(134.69112945,180.07562688)(135.20312945,180.97162688)
\curveto(135.73646278,181.88896021)(136.00312945,183.20096021)(136.00312945,184.90762688)
\curveto(136.00312945,188.34229354)(134.91512945,190.05962688)(132.73912945,190.05962688)
\closepath
}
}
{
\newrgbcolor{curcolor}{0 0 0}
\pscustom[linestyle=none,fillstyle=solid,fillcolor=curcolor]
{
\newpath
\moveto(151.97111089,193.89962688)
\curveto(154.31777756,193.89962688)(156.10977756,193.38762688)(157.34711089,192.36362688)
\curveto(158.60577756,191.36096021)(159.23511089,189.81429354)(159.23511089,187.72362688)
\lineto(159.23511089,176.07562688)
\lineto(155.90711089,176.07562688)
\lineto(154.97911089,178.44362688)
\lineto(154.85111089,178.44362688)
\curveto(154.10444423,177.50496021)(153.31511089,176.82229354)(152.48311089,176.39562688)
\curveto(151.65111089,175.96896021)(150.50977756,175.75562688)(149.05911089,175.75562688)
\curveto(147.50177756,175.75562688)(146.21111089,176.20362688)(145.18711089,177.09962688)
\curveto(144.16311089,177.99562688)(143.65111089,179.39296021)(143.65111089,181.29162688)
\curveto(143.65111089,183.14762688)(144.30177756,184.51296021)(145.60311089,185.38762688)
\curveto(146.90444423,186.26229354)(148.85644423,186.75296021)(151.45911089,186.85962688)
\lineto(154.49911089,186.95562688)
\lineto(154.49911089,187.72362688)
\curveto(154.49911089,188.64096021)(154.25377756,189.31296021)(153.76311089,189.73962688)
\curveto(153.29377756,190.16629354)(152.63244423,190.37962688)(151.77911089,190.37962688)
\curveto(150.92577756,190.37962688)(150.09377756,190.25162688)(149.28311089,189.99562688)
\curveto(148.47244423,189.76096021)(147.66177756,189.46229354)(146.85111089,189.09962688)
\lineto(145.28311089,192.33162688)
\curveto(146.20044423,192.80096021)(147.23511089,193.17429354)(148.38711089,193.45162688)
\curveto(149.53911089,193.75029354)(150.73377756,193.89962688)(151.97111089,193.89962688)
\closepath
\moveto(154.49911089,184.17162688)
\lineto(152.64311089,184.10762688)
\curveto(151.10711089,184.06496021)(150.04044423,183.78762688)(149.44311089,183.27562688)
\curveto(148.84577756,182.76362688)(148.54711089,182.09162688)(148.54711089,181.25962688)
\curveto(148.54711089,180.53429354)(148.76044423,180.01162688)(149.18711089,179.69162688)
\curveto(149.61377756,179.39296021)(150.16844423,179.24362688)(150.85111089,179.24362688)
\curveto(151.87511089,179.24362688)(152.73911089,179.54229354)(153.44311089,180.13962688)
\curveto(154.14711089,180.75829354)(154.49911089,181.62229354)(154.49911089,182.73162688)
\closepath
}
}
{
\newrgbcolor{curcolor}{0 0 0}
\pscustom[linestyle=none,fillstyle=solid,fillcolor=curcolor]
{
\newpath
\moveto(161.63511382,193.54762688)
\lineto(166.85111382,193.54762688)
\lineto(170.14711382,183.72362688)
\curveto(170.31778049,183.23296021)(170.44578049,182.74229354)(170.53111382,182.25162688)
\curveto(170.61644716,181.76096021)(170.68044716,181.23829354)(170.72311382,180.68362688)
\lineto(170.81911382,180.68362688)
\curveto(170.88311382,181.23829354)(170.96844716,181.76096021)(171.07511382,182.25162688)
\curveto(171.18178049,182.74229354)(171.32044716,183.23296021)(171.49111382,183.72362688)
\lineto(174.72311382,193.54762688)
\lineto(179.84311382,193.54762688)
\lineto(172.45111382,173.83562688)
\curveto(171.76844716,172.02229354)(170.79778049,170.66762688)(169.53911382,169.77162688)
\curveto(168.28044716,168.85429354)(166.81911382,168.39562688)(165.15511382,168.39562688)
\curveto(164.60044716,168.39562688)(164.13111382,168.42762688)(163.74711382,168.49162688)
\curveto(163.36311382,168.53429354)(163.02178049,168.58762688)(162.72311382,168.65162688)
\lineto(162.72311382,172.42762688)
\curveto(162.93644716,172.38496021)(163.21378049,172.34229354)(163.55511382,172.29962688)
\curveto(163.89644716,172.25696021)(164.24844716,172.23562688)(164.61111382,172.23562688)
\curveto(165.61378049,172.23562688)(166.40311382,172.54496021)(166.97911382,173.16362688)
\curveto(167.55511382,173.76096021)(167.99244716,174.48629354)(168.29111382,175.33962688)
\lineto(168.57911382,176.20362688)
\closepath
}
}
{
\newrgbcolor{curcolor}{0 0 0}
\pscustom[linestyle=none,fillstyle=solid,fillcolor=curcolor]
{
\newpath
\moveto(188.70712164,193.86762688)
\curveto(189.9657883,193.86762688)(191.13912164,193.69696021)(192.22712164,193.35562688)
\curveto(193.33645497,193.03562688)(194.2217883,192.53429354)(194.88312164,191.85162688)
\curveto(195.5657883,191.16896021)(195.90712164,190.29429354)(195.90712164,189.22762688)
\curveto(195.90712164,188.18229354)(195.58712164,187.35029354)(194.94712164,186.73162688)
\curveto(194.30712164,186.11296021)(193.46445497,185.66496021)(192.41912164,185.38762688)
\lineto(192.41912164,185.22762688)
\curveto(193.1657883,185.05696021)(193.8377883,184.81162688)(194.43512164,184.49162688)
\curveto(195.03245497,184.19296021)(195.5017883,183.77696021)(195.84312164,183.24362688)
\curveto(196.2057883,182.73162688)(196.38712164,182.03829354)(196.38712164,181.16362688)
\curveto(196.38712164,180.20362688)(196.0777883,179.30762688)(195.45912164,178.47562688)
\curveto(194.8617883,177.66496021)(193.92312164,177.00362688)(192.64312164,176.49162688)
\curveto(191.36312164,176.00096021)(189.7417883,175.75562688)(187.77912164,175.75562688)
\curveto(184.8777883,175.75562688)(182.6377883,176.11829354)(181.05912164,176.84362688)
\lineto(181.05912164,180.77962688)
\curveto(181.78445497,180.43829354)(182.6697883,180.12896021)(183.71512164,179.85162688)
\curveto(184.7817883,179.57429354)(185.91245497,179.43562688)(187.10712164,179.43562688)
\curveto(188.40845497,179.43562688)(189.50712164,179.58496021)(190.40312164,179.88362688)
\curveto(191.29912164,180.18229354)(191.74712164,180.70496021)(191.74712164,181.45162688)
\curveto(191.74712164,182.83829354)(190.0937883,183.53162688)(186.78712164,183.53162688)
\lineto(184.93112164,183.53162688)
\lineto(184.93112164,186.82762688)
\lineto(186.69112164,186.82762688)
\curveto(188.2697883,186.82762688)(189.4857883,186.95562688)(190.33912164,187.21162688)
\curveto(191.2137883,187.46762688)(191.65112164,187.94762688)(191.65112164,188.65162688)
\curveto(191.65112164,189.20629354)(191.3737883,189.62229354)(190.81912164,189.89962688)
\curveto(190.26445497,190.19829354)(189.3577883,190.34762688)(188.09912164,190.34762688)
\curveto(187.26712164,190.34762688)(186.3497883,190.25162688)(185.34712164,190.05962688)
\curveto(184.3657883,189.86762688)(183.44845497,189.59029354)(182.59512164,189.22762688)
\lineto(181.18712164,192.55562688)
\curveto(182.1897883,192.93962688)(183.2777883,193.24896021)(184.45112164,193.48362688)
\curveto(185.6457883,193.73962688)(187.06445497,193.86762688)(188.70712164,193.86762688)
\closepath
}
}
{
\newrgbcolor{curcolor}{0 0 0}
\pscustom[linestyle=none,fillstyle=solid,fillcolor=curcolor]
{
\newpath
\moveto(207.2991314,193.86762688)
\curveto(209.70979807,193.86762688)(211.6191314,193.17429354)(213.0271314,191.78762688)
\curveto(214.4351314,190.42229354)(215.1391314,188.47029354)(215.1391314,185.93162688)
\lineto(215.1391314,183.62762688)
\lineto(203.8751314,183.62762688)
\curveto(203.91779807,182.28362688)(204.31246473,181.22762688)(205.0591314,180.45962688)
\curveto(205.8271314,179.69162688)(206.8831314,179.30762688)(208.2271314,179.30762688)
\curveto(209.33646473,179.30762688)(210.34979807,179.41429354)(211.2671314,179.62762688)
\curveto(212.20579807,179.86229354)(213.16579807,180.21429354)(214.1471314,180.68362688)
\lineto(214.1471314,177.00362688)
\curveto(213.27246473,176.57696021)(212.36579807,176.26762688)(211.4271314,176.07562688)
\curveto(210.48846473,175.86229354)(209.3471314,175.75562688)(208.0031314,175.75562688)
\curveto(206.25379807,175.75562688)(204.7071314,176.07562688)(203.3631314,176.71562688)
\curveto(202.0191314,177.37696021)(200.9631314,178.35829354)(200.1951314,179.65962688)
\curveto(199.4271314,180.98229354)(199.0431314,182.65696021)(199.0431314,184.68362688)
\curveto(199.0431314,186.71029354)(199.38446473,188.40629354)(200.0671314,189.77162688)
\curveto(200.7711314,191.13696021)(201.74179807,192.16096021)(202.9791314,192.84362688)
\curveto(204.21646473,193.52629354)(205.65646473,193.86762688)(207.2991314,193.86762688)
\closepath
\moveto(207.3311314,190.47562688)
\curveto(206.39246473,190.47562688)(205.62446473,190.17696021)(205.0271314,189.57962688)
\curveto(204.42979807,188.98229354)(204.07779807,188.05429354)(203.9711314,186.79562688)
\lineto(210.6591314,186.79562688)
\curveto(210.63779807,187.84096021)(210.34979807,188.71562688)(209.7951314,189.41962688)
\curveto(209.26179807,190.12362688)(208.44046473,190.47562688)(207.3311314,190.47562688)
\closepath
}
}
{
\newrgbcolor{curcolor}{0 0 0}
\pscustom[linestyle=none,fillstyle=solid,fillcolor=curcolor]
{
\newpath
\moveto(228.73911871,193.86762688)
\curveto(230.70178537,193.86762688)(232.29111871,193.09962688)(233.50711871,191.56362688)
\curveto(234.72311871,190.04896021)(235.33111871,187.80896021)(235.33111871,184.84362688)
\curveto(235.33111871,181.85696021)(234.70178537,179.59562688)(233.44311871,178.05962688)
\curveto(232.18445204,176.52362688)(230.57378537,175.75562688)(228.61111871,175.75562688)
\curveto(227.35245204,175.75562688)(226.34978537,175.97962688)(225.60311871,176.42762688)
\curveto(224.85645204,176.89696021)(224.24845204,177.41962688)(223.77911871,177.99562688)
\lineto(223.52311871,177.99562688)
\curveto(223.69378537,177.09962688)(223.77911871,176.24629354)(223.77911871,175.43562688)
\lineto(223.77911871,168.39562688)
\lineto(219.01111871,168.39562688)
\lineto(219.01111871,193.54762688)
\lineto(222.88311871,193.54762688)
\lineto(223.55511871,191.27562688)
\lineto(223.77911871,191.27562688)
\curveto(224.24845204,191.97962688)(224.87778537,192.58762688)(225.66711871,193.09962688)
\curveto(226.45645204,193.61162688)(227.48045204,193.86762688)(228.73911871,193.86762688)
\closepath
\moveto(227.20311871,190.05962688)
\curveto(225.96578537,190.05962688)(225.09111871,189.66496021)(224.57911871,188.87562688)
\curveto(224.06711871,188.10762688)(223.80045204,186.94496021)(223.77911871,185.38762688)
\lineto(223.77911871,184.87562688)
\curveto(223.77911871,183.19029354)(224.02445204,181.88896021)(224.51511871,180.97162688)
\curveto(225.02711871,180.07562688)(225.94445204,179.62762688)(227.26711871,179.62762688)
\curveto(228.35511871,179.62762688)(229.15511871,180.07562688)(229.66711871,180.97162688)
\curveto(230.20045204,181.88896021)(230.46711871,183.20096021)(230.46711871,184.90762688)
\curveto(230.46711871,188.34229354)(229.37911871,190.05962688)(227.20311871,190.05962688)
\closepath
}
}
{
\newrgbcolor{curcolor}{0 0 0}
\pscustom[linestyle=none,fillstyle=solid,fillcolor=curcolor]
{
\newpath
\moveto(246.33910015,184.84362688)
\curveto(246.33910015,182.26229354)(245.95510015,179.77696021)(245.18710015,177.38762688)
\curveto(244.44043348,175.01962688)(243.25643348,172.89696021)(241.63510015,171.01962688)
\lineto(237.76310015,171.01962688)
\curveto(239.19243348,172.98229354)(240.28043348,175.15829354)(241.02710015,177.54762688)
\curveto(241.79510015,179.95829354)(242.17910015,182.40096021)(242.17910015,184.87562688)
\curveto(242.17910015,187.41429354)(241.79510015,189.88896021)(241.02710015,192.29962688)
\curveto(240.28043348,194.71029354)(239.18176682,196.91829354)(237.73110015,198.92362688)
\lineto(241.63510015,198.92362688)
\curveto(243.25643348,196.98229354)(244.44043348,194.79562688)(245.18710015,192.36362688)
\curveto(245.95510015,189.95296021)(246.33910015,187.44629354)(246.33910015,184.84362688)
\closepath
}
}
{
\newrgbcolor{curcolor}{0 0 0}
\pscustom[linestyle=none,fillstyle=solid,fillcolor=curcolor]
{
\newpath
\moveto(483.55828179,233.18623725)
\curveto(481.70228179,233.18623725)(480.28361513,232.49290392)(479.30228179,231.10623725)
\curveto(478.32094846,229.71957058)(477.83028179,227.82090392)(477.83028179,225.41023725)
\curveto(477.83028179,222.97823725)(478.27828179,221.09023725)(479.17428179,219.74623725)
\curveto(480.09161513,218.42357058)(481.55294846,217.76223725)(483.55828179,217.76223725)
\curveto(484.47561513,217.76223725)(485.40361513,217.86890392)(486.34228179,218.08223725)
\curveto(487.28094846,218.29557058)(488.29428179,218.59423725)(489.38228179,218.97823725)
\lineto(489.38228179,214.91423725)
\curveto(488.37961513,214.50890392)(487.38761513,214.21023725)(486.40628179,214.01823725)
\curveto(485.42494846,213.82623725)(484.32628179,213.73023725)(483.11028179,213.73023725)
\curveto(480.74228179,213.73023725)(478.80094846,214.21023725)(477.28628179,215.17023725)
\curveto(475.77161513,216.15157058)(474.65161513,217.51690392)(473.92628179,219.26623725)
\curveto(473.20094846,221.03690392)(472.83828179,223.09557058)(472.83828179,225.44223725)
\curveto(472.83828179,227.74623725)(473.25428179,229.78357058)(474.08628179,231.55423725)
\curveto(474.91828179,233.32490392)(476.12361513,234.71157058)(477.70228179,235.71423725)
\curveto(479.30228179,236.71690392)(481.25428179,237.21823725)(483.55828179,237.21823725)
\curveto(484.68894846,237.21823725)(485.81961513,237.06890392)(486.95028179,236.77023725)
\curveto(488.10228179,236.49290392)(489.20094846,236.10890392)(490.24628179,235.61823725)
\lineto(488.67828179,231.68223725)
\curveto(487.82494846,232.08757058)(486.96094846,232.43957058)(486.08628179,232.73823725)
\curveto(485.23294846,233.03690392)(484.39028179,233.18623725)(483.55828179,233.18623725)
\closepath
}
}
{
\newrgbcolor{curcolor}{0 0 0}
\pscustom[linestyle=none,fillstyle=solid,fillcolor=curcolor]
{
\newpath
\moveto(501.06225396,231.84223725)
\curveto(503.47292063,231.84223725)(505.38225396,231.14890392)(506.79025396,229.76223725)
\curveto(508.19825396,228.39690392)(508.90225396,226.44490392)(508.90225396,223.90623725)
\lineto(508.90225396,221.60223725)
\lineto(497.63825396,221.60223725)
\curveto(497.68092063,220.25823725)(498.07558729,219.20223725)(498.82225396,218.43423725)
\curveto(499.59025396,217.66623725)(500.64625396,217.28223725)(501.99025396,217.28223725)
\curveto(503.09958729,217.28223725)(504.11292063,217.38890392)(505.03025396,217.60223725)
\curveto(505.96892063,217.83690392)(506.92892063,218.18890392)(507.91025396,218.65823725)
\lineto(507.91025396,214.97823725)
\curveto(507.03558729,214.55157058)(506.12892063,214.24223725)(505.19025396,214.05023725)
\curveto(504.25158729,213.83690392)(503.11025396,213.73023725)(501.76625396,213.73023725)
\curveto(500.01692063,213.73023725)(498.47025396,214.05023725)(497.12625396,214.69023725)
\curveto(495.78225396,215.35157058)(494.72625396,216.33290392)(493.95825396,217.63423725)
\curveto(493.19025396,218.95690392)(492.80625396,220.63157058)(492.80625396,222.65823725)
\curveto(492.80625396,224.68490392)(493.14758729,226.38090392)(493.83025396,227.74623725)
\curveto(494.53425396,229.11157058)(495.50492063,230.13557058)(496.74225396,230.81823725)
\curveto(497.97958729,231.50090392)(499.41958729,231.84223725)(501.06225396,231.84223725)
\closepath
\moveto(501.09425396,228.45023725)
\curveto(500.15558729,228.45023725)(499.38758729,228.15157058)(498.79025396,227.55423725)
\curveto(498.19292063,226.95690392)(497.84092063,226.02890392)(497.73425396,224.77023725)
\lineto(504.42225396,224.77023725)
\curveto(504.40092063,225.81557058)(504.11292063,226.69023725)(503.55825396,227.39423725)
\curveto(503.02492063,228.09823725)(502.20358729,228.45023725)(501.09425396,228.45023725)
\closepath
}
}
{
\newrgbcolor{curcolor}{0 0 0}
\pscustom[linestyle=none,fillstyle=solid,fillcolor=curcolor]
{
\newpath
\moveto(522.50224126,231.84223725)
\curveto(524.46490793,231.84223725)(526.05424126,231.07423725)(527.27024126,229.53823725)
\curveto(528.48624126,228.02357058)(529.09424126,225.78357058)(529.09424126,222.81823725)
\curveto(529.09424126,219.83157058)(528.46490793,217.57023725)(527.20624126,216.03423725)
\curveto(525.9475746,214.49823725)(524.33690793,213.73023725)(522.37424126,213.73023725)
\curveto(521.1155746,213.73023725)(520.11290793,213.95423725)(519.36624126,214.40223725)
\curveto(518.6195746,214.87157058)(518.0115746,215.39423725)(517.54224126,215.97023725)
\lineto(517.28624126,215.97023725)
\curveto(517.45690793,215.07423725)(517.54224126,214.22090392)(517.54224126,213.41023725)
\lineto(517.54224126,206.37023725)
\lineto(512.77424126,206.37023725)
\lineto(512.77424126,231.52223725)
\lineto(516.64624126,231.52223725)
\lineto(517.31824126,229.25023725)
\lineto(517.54224126,229.25023725)
\curveto(518.0115746,229.95423725)(518.64090793,230.56223725)(519.43024126,231.07423725)
\curveto(520.2195746,231.58623725)(521.2435746,231.84223725)(522.50224126,231.84223725)
\closepath
\moveto(520.96624126,228.03423725)
\curveto(519.72890793,228.03423725)(518.85424126,227.63957058)(518.34224126,226.85023725)
\curveto(517.83024126,226.08223725)(517.5635746,224.91957058)(517.54224126,223.36223725)
\lineto(517.54224126,222.85023725)
\curveto(517.54224126,221.16490392)(517.7875746,219.86357058)(518.27824126,218.94623725)
\curveto(518.79024126,218.05023725)(519.7075746,217.60223725)(521.03024126,217.60223725)
\curveto(522.11824126,217.60223725)(522.91824126,218.05023725)(523.43024126,218.94623725)
\curveto(523.9635746,219.86357058)(524.23024126,221.17557058)(524.23024126,222.88223725)
\curveto(524.23024126,226.31690392)(523.14224126,228.03423725)(520.96624126,228.03423725)
\closepath
}
}
{
\newrgbcolor{curcolor}{0 0 0}
\pscustom[linestyle=none,fillstyle=solid,fillcolor=curcolor]
{
\newpath
\moveto(548.51822271,226.94623725)
\curveto(548.51822271,226.00757058)(548.21955604,225.20757058)(547.62222271,224.54623725)
\curveto(547.04622271,223.88490392)(546.18222271,223.45823725)(545.03022271,223.26623725)
\lineto(545.03022271,223.13823725)
\curveto(546.24622271,222.98890392)(547.21688938,222.56223725)(547.94222271,221.85823725)
\curveto(548.68888938,221.17557058)(549.06222271,220.31157058)(549.06222271,219.26623725)
\curveto(549.06222271,218.26357058)(548.79555604,217.36757058)(548.26222271,216.57823725)
\curveto(547.75022271,215.78890392)(546.92888938,215.17023725)(545.79822271,214.72223725)
\curveto(544.66755604,214.27423725)(543.18488938,214.05023725)(541.35022271,214.05023725)
\lineto(533.03022271,214.05023725)
\lineto(533.03022271,231.52223725)
\lineto(541.35022271,231.52223725)
\curveto(542.71555604,231.52223725)(543.93155604,231.37290392)(544.99822271,231.07423725)
\curveto(546.08622271,230.79690392)(546.93955604,230.31690392)(547.55822271,229.63423725)
\curveto(548.19822271,228.97290392)(548.51822271,228.07690392)(548.51822271,226.94623725)
\closepath
\moveto(543.68622271,226.56223725)
\curveto(543.68622271,227.62890392)(542.84355604,228.16223725)(541.15822271,228.16223725)
\lineto(537.79822271,228.16223725)
\lineto(537.79822271,224.70623725)
\lineto(540.61422271,224.70623725)
\curveto(541.61688938,224.70623725)(542.37422271,224.84490392)(542.88622271,225.12223725)
\curveto(543.41955604,225.42090392)(543.68622271,225.90090392)(543.68622271,226.56223725)
\closepath
\moveto(544.13422271,219.52223725)
\curveto(544.13422271,220.20490392)(543.85688938,220.69557058)(543.30222271,220.99423725)
\curveto(542.76888938,221.31423725)(541.97955604,221.47423725)(540.93422271,221.47423725)
\lineto(537.79822271,221.47423725)
\lineto(537.79822271,217.34623725)
\lineto(541.03022271,217.34623725)
\curveto(541.92622271,217.34623725)(542.66222271,217.50623725)(543.23822271,217.82623725)
\curveto(543.83555604,218.16757058)(544.13422271,218.73290392)(544.13422271,219.52223725)
\closepath
}
}
{
\newrgbcolor{curcolor}{0 0 0}
\pscustom[linestyle=none,fillstyle=solid,fillcolor=curcolor]
{
\newpath
\moveto(560.35821343,231.84223725)
\curveto(562.7688801,231.84223725)(564.67821343,231.14890392)(566.08621343,229.76223725)
\curveto(567.49421343,228.39690392)(568.19821343,226.44490392)(568.19821343,223.90623725)
\lineto(568.19821343,221.60223725)
\lineto(556.93421343,221.60223725)
\curveto(556.9768801,220.25823725)(557.37154677,219.20223725)(558.11821343,218.43423725)
\curveto(558.88621343,217.66623725)(559.94221343,217.28223725)(561.28621343,217.28223725)
\curveto(562.39554677,217.28223725)(563.4088801,217.38890392)(564.32621343,217.60223725)
\curveto(565.2648801,217.83690392)(566.2248801,218.18890392)(567.20621343,218.65823725)
\lineto(567.20621343,214.97823725)
\curveto(566.33154677,214.55157058)(565.4248801,214.24223725)(564.48621343,214.05023725)
\curveto(563.54754677,213.83690392)(562.40621343,213.73023725)(561.06221343,213.73023725)
\curveto(559.3128801,213.73023725)(557.76621343,214.05023725)(556.42221343,214.69023725)
\curveto(555.07821343,215.35157058)(554.02221343,216.33290392)(553.25421343,217.63423725)
\curveto(552.48621343,218.95690392)(552.10221343,220.63157058)(552.10221343,222.65823725)
\curveto(552.10221343,224.68490392)(552.44354677,226.38090392)(553.12621343,227.74623725)
\curveto(553.83021343,229.11157058)(554.8008801,230.13557058)(556.03821343,230.81823725)
\curveto(557.27554677,231.50090392)(558.71554677,231.84223725)(560.35821343,231.84223725)
\closepath
\moveto(560.39021343,228.45023725)
\curveto(559.45154677,228.45023725)(558.68354677,228.15157058)(558.08621343,227.55423725)
\curveto(557.4888801,226.95690392)(557.1368801,226.02890392)(557.03021343,224.77023725)
\lineto(563.71821343,224.77023725)
\curveto(563.6968801,225.81557058)(563.4088801,226.69023725)(562.85421343,227.39423725)
\curveto(562.3208801,228.09823725)(561.49954677,228.45023725)(560.39021343,228.45023725)
\closepath
}
}
{
\newrgbcolor{curcolor}{0 0 0}
\pscustom[linestyle=none,fillstyle=solid,fillcolor=curcolor]
{
\newpath
\moveto(581.79820074,231.84223725)
\curveto(583.7608674,231.84223725)(585.35020074,231.07423725)(586.56620074,229.53823725)
\curveto(587.78220074,228.02357058)(588.39020074,225.78357058)(588.39020074,222.81823725)
\curveto(588.39020074,219.83157058)(587.7608674,217.57023725)(586.50220074,216.03423725)
\curveto(585.24353407,214.49823725)(583.6328674,213.73023725)(581.67020074,213.73023725)
\curveto(580.41153407,213.73023725)(579.4088674,213.95423725)(578.66220074,214.40223725)
\curveto(577.91553407,214.87157058)(577.30753407,215.39423725)(576.83820074,215.97023725)
\lineto(576.58220074,215.97023725)
\curveto(576.7528674,215.07423725)(576.83820074,214.22090392)(576.83820074,213.41023725)
\lineto(576.83820074,206.37023725)
\lineto(572.07020074,206.37023725)
\lineto(572.07020074,231.52223725)
\lineto(575.94220074,231.52223725)
\lineto(576.61420074,229.25023725)
\lineto(576.83820074,229.25023725)
\curveto(577.30753407,229.95423725)(577.9368674,230.56223725)(578.72620074,231.07423725)
\curveto(579.51553407,231.58623725)(580.53953407,231.84223725)(581.79820074,231.84223725)
\closepath
\moveto(580.26220074,228.03423725)
\curveto(579.0248674,228.03423725)(578.15020074,227.63957058)(577.63820074,226.85023725)
\curveto(577.12620074,226.08223725)(576.85953407,224.91957058)(576.83820074,223.36223725)
\lineto(576.83820074,222.85023725)
\curveto(576.83820074,221.16490392)(577.08353407,219.86357058)(577.57420074,218.94623725)
\curveto(578.08620074,218.05023725)(579.00353407,217.60223725)(580.32620074,217.60223725)
\curveto(581.41420074,217.60223725)(582.21420074,218.05023725)(582.72620074,218.94623725)
\curveto(583.25953407,219.86357058)(583.52620074,221.17557058)(583.52620074,222.88223725)
\curveto(583.52620074,226.31690392)(582.43820074,228.03423725)(580.26220074,228.03423725)
\closepath
}
}
{
\newrgbcolor{curcolor}{0 0 0}
\pscustom[linestyle=none,fillstyle=solid,fillcolor=curcolor]
{
\newpath
\moveto(460.95023956,182.81823725)
\curveto(460.95023956,185.42090392)(461.32357289,187.92757058)(462.07023956,190.33823725)
\curveto(462.83823956,192.77023725)(464.03290622,194.95690392)(465.65423956,196.89823725)
\lineto(469.55823956,196.89823725)
\curveto(468.10757289,194.89290392)(466.99823956,192.68490392)(466.23023956,190.27423725)
\curveto(465.48357289,187.86357058)(465.11023956,185.38890392)(465.11023956,182.85023725)
\curveto(465.11023956,180.37557058)(465.48357289,177.93290392)(466.23023956,175.52223725)
\curveto(466.97690622,173.13290392)(468.07557289,170.95690392)(469.52623956,168.99423725)
\lineto(465.65423956,168.99423725)
\curveto(464.03290622,170.87157058)(462.83823956,172.99423725)(462.07023956,175.36223725)
\curveto(461.32357289,177.75157058)(460.95023956,180.23690392)(460.95023956,182.81823725)
\closepath
}
}
{
\newrgbcolor{curcolor}{0 0 0}
\pscustom[linestyle=none,fillstyle=solid,fillcolor=curcolor]
{
\newpath
\moveto(493.65426202,174.05023725)
\lineto(487.51026202,174.05023725)
\lineto(477.55826202,191.33023725)
\lineto(477.43026202,191.33023725)
\lineto(477.55826202,188.06623725)
\curveto(477.62226202,186.97823725)(477.67559535,185.89023725)(477.71826202,184.80223725)
\lineto(477.71826202,174.05023725)
\lineto(473.39826202,174.05023725)
\lineto(473.39826202,196.89823725)
\lineto(479.51026202,196.89823725)
\lineto(489.43026202,179.77823725)
\lineto(489.52626202,179.77823725)
\lineto(489.39826202,182.91423725)
\curveto(489.35559535,183.95957058)(489.32359535,185.01557058)(489.30226202,186.08223725)
\lineto(489.30226202,196.89823725)
\lineto(493.65426202,196.89823725)
\closepath
}
}
{
\newrgbcolor{curcolor}{0 0 0}
\pscustom[linestyle=none,fillstyle=solid,fillcolor=curcolor]
{
\newpath
\moveto(514.90225323,182.81823725)
\curveto(514.90225323,179.91690392)(514.13425323,177.67690392)(512.59825323,176.09823725)
\curveto(511.08358656,174.51957058)(509.01425323,173.73023725)(506.39025323,173.73023725)
\curveto(504.76891989,173.73023725)(503.31825323,174.08223725)(502.03825323,174.78623725)
\curveto(500.77958656,175.49023725)(499.78758656,176.51423725)(499.06225323,177.85823725)
\curveto(498.33691989,179.22357058)(497.97425323,180.87690392)(497.97425323,182.81823725)
\curveto(497.97425323,185.71957058)(498.73158656,187.94890392)(500.24625323,189.50623725)
\curveto(501.76091989,191.06357058)(503.84091989,191.84223725)(506.48625323,191.84223725)
\curveto(508.12891989,191.84223725)(509.57958656,191.49023725)(510.83825323,190.78623725)
\curveto(512.09691989,190.08223725)(513.08891989,189.05823725)(513.81425323,187.71423725)
\curveto(514.53958656,186.37023725)(514.90225323,184.73823725)(514.90225323,182.81823725)
\closepath
\moveto(502.83825323,182.81823725)
\curveto(502.83825323,181.09023725)(503.11558656,179.77823725)(503.67025323,178.88223725)
\curveto(504.24625323,178.00757058)(505.17425323,177.57023725)(506.45425323,177.57023725)
\curveto(507.71291989,177.57023725)(508.61958656,178.00757058)(509.17425323,178.88223725)
\curveto(509.75025323,179.77823725)(510.03825323,181.09023725)(510.03825323,182.81823725)
\curveto(510.03825323,184.54623725)(509.75025323,185.83690392)(509.17425323,186.69023725)
\curveto(508.61958656,187.56490392)(507.70225323,188.00223725)(506.42225323,188.00223725)
\curveto(505.16358656,188.00223725)(504.24625323,187.56490392)(503.67025323,186.69023725)
\curveto(503.11558656,185.83690392)(502.83825323,184.54623725)(502.83825323,182.81823725)
\closepath
}
}
{
\newrgbcolor{curcolor}{0 0 0}
\pscustom[linestyle=none,fillstyle=solid,fillcolor=curcolor]
{
\newpath
\moveto(524.37423663,173.73023725)
\curveto(522.43290329,173.73023725)(520.84356996,174.48757058)(519.60623663,176.00223725)
\curveto(518.39023663,177.53823725)(517.78223663,179.78890392)(517.78223663,182.75423725)
\curveto(517.78223663,185.74090392)(518.40090329,188.00223725)(519.63823663,189.53823725)
\curveto(520.87556996,191.07423725)(522.49690329,191.84223725)(524.50223663,191.84223725)
\curveto(525.76090329,191.84223725)(526.79556996,191.59690392)(527.60623663,191.10623725)
\curveto(528.41690329,190.61557058)(529.05690329,190.00757058)(529.52623663,189.28223725)
\lineto(529.68623663,189.28223725)
\curveto(529.62223663,189.62357058)(529.54756996,190.11423725)(529.46223663,190.75423725)
\curveto(529.37690329,191.41557058)(529.33423663,192.08757058)(529.33423663,192.77023725)
\lineto(529.33423663,198.37023725)
\lineto(534.10223663,198.37023725)
\lineto(534.10223663,174.05023725)
\lineto(530.45423663,174.05023725)
\lineto(529.52623663,176.32223725)
\lineto(529.33423663,176.32223725)
\curveto(528.86490329,175.59690392)(528.23556996,174.97823725)(527.44623663,174.46623725)
\curveto(526.65690329,173.97557058)(525.63290329,173.73023725)(524.37423663,173.73023725)
\closepath
\moveto(526.03823663,177.53823725)
\curveto(527.33956996,177.53823725)(528.25690329,177.92223725)(528.79023663,178.69023725)
\curveto(529.32356996,179.47957058)(529.61156996,180.65290392)(529.65423663,182.21023725)
\lineto(529.65423663,182.72223725)
\curveto(529.65423663,184.40757058)(529.38756996,185.69823725)(528.85423663,186.59423725)
\curveto(528.34223663,187.51157058)(527.38223663,187.97023725)(525.97423663,187.97023725)
\curveto(524.92890329,187.97023725)(524.10756996,187.51157058)(523.51023663,186.59423725)
\curveto(522.91290329,185.69823725)(522.61423663,184.39690392)(522.61423663,182.69023725)
\curveto(522.61423663,180.98357058)(522.91290329,179.69290392)(523.51023663,178.81823725)
\curveto(524.10756996,177.96490392)(524.95023663,177.53823725)(526.03823663,177.53823725)
\closepath
}
}
{
\newrgbcolor{curcolor}{0 0 0}
\pscustom[linestyle=none,fillstyle=solid,fillcolor=curcolor]
{
\newpath
\moveto(546.29421807,191.84223725)
\curveto(548.70488474,191.84223725)(550.61421807,191.14890392)(552.02221807,189.76223725)
\curveto(553.43021807,188.39690392)(554.13421807,186.44490392)(554.13421807,183.90623725)
\lineto(554.13421807,181.60223725)
\lineto(542.87021807,181.60223725)
\curveto(542.91288474,180.25823725)(543.3075514,179.20223725)(544.05421807,178.43423725)
\curveto(544.82221807,177.66623725)(545.87821807,177.28223725)(547.22221807,177.28223725)
\curveto(548.3315514,177.28223725)(549.34488474,177.38890392)(550.26221807,177.60223725)
\curveto(551.20088474,177.83690392)(552.16088474,178.18890392)(553.14221807,178.65823725)
\lineto(553.14221807,174.97823725)
\curveto(552.2675514,174.55157058)(551.36088474,174.24223725)(550.42221807,174.05023725)
\curveto(549.4835514,173.83690392)(548.34221807,173.73023725)(546.99821807,173.73023725)
\curveto(545.24888474,173.73023725)(543.70221807,174.05023725)(542.35821807,174.69023725)
\curveto(541.01421807,175.35157058)(539.95821807,176.33290392)(539.19021807,177.63423725)
\curveto(538.42221807,178.95690392)(538.03821807,180.63157058)(538.03821807,182.65823725)
\curveto(538.03821807,184.68490392)(538.3795514,186.38090392)(539.06221807,187.74623725)
\curveto(539.76621807,189.11157058)(540.73688474,190.13557058)(541.97421807,190.81823725)
\curveto(543.2115514,191.50090392)(544.6515514,191.84223725)(546.29421807,191.84223725)
\closepath
\moveto(546.32621807,188.45023725)
\curveto(545.3875514,188.45023725)(544.6195514,188.15157058)(544.02221807,187.55423725)
\curveto(543.42488474,186.95690392)(543.07288474,186.02890392)(542.96621807,184.77023725)
\lineto(549.65421807,184.77023725)
\curveto(549.63288474,185.81557058)(549.34488474,186.69023725)(548.79021807,187.39423725)
\curveto(548.25688474,188.09823725)(547.4355514,188.45023725)(546.32621807,188.45023725)
\closepath
}
}
{
\newrgbcolor{curcolor}{0 0 0}
\pscustom[linestyle=none,fillstyle=solid,fillcolor=curcolor]
{
\newpath
\moveto(557.33420538,176.29023725)
\curveto(557.33420538,177.27157058)(557.60087204,177.95423725)(558.13420538,178.33823725)
\curveto(558.66753871,178.74357058)(559.31820538,178.94623725)(560.08620538,178.94623725)
\curveto(560.83287204,178.94623725)(561.47287204,178.74357058)(562.00620538,178.33823725)
\curveto(562.53953871,177.95423725)(562.80620538,177.27157058)(562.80620538,176.29023725)
\curveto(562.80620538,175.35157058)(562.53953871,174.66890392)(562.00620538,174.24223725)
\curveto(561.47287204,173.83690392)(560.83287204,173.63423725)(560.08620538,173.63423725)
\curveto(559.31820538,173.63423725)(558.66753871,173.83690392)(558.13420538,174.24223725)
\curveto(557.60087204,174.66890392)(557.33420538,175.35157058)(557.33420538,176.29023725)
\closepath
}
}
{
\newrgbcolor{curcolor}{0 0 0}
\pscustom[linestyle=none,fillstyle=solid,fillcolor=curcolor]
{
\newpath
\moveto(566.93420049,196.03423725)
\curveto(566.93420049,196.93023725)(567.17953383,197.53823725)(567.67020049,197.85823725)
\curveto(568.18220049,198.19957058)(568.80086716,198.37023725)(569.52620049,198.37023725)
\curveto(570.23020049,198.37023725)(570.83820049,198.19957058)(571.35020049,197.85823725)
\curveto(571.86220049,197.53823725)(572.11820049,196.93023725)(572.11820049,196.03423725)
\curveto(572.11820049,195.15957058)(571.86220049,194.55157058)(571.35020049,194.21023725)
\curveto(570.83820049,193.86890392)(570.23020049,193.69823725)(569.52620049,193.69823725)
\curveto(568.80086716,193.69823725)(568.18220049,193.86890392)(567.67020049,194.21023725)
\curveto(567.17953383,194.55157058)(566.93420049,195.15957058)(566.93420049,196.03423725)
\closepath
\moveto(565.71820049,166.37023725)
\curveto(565.16353383,166.37023725)(564.59820049,166.41290392)(564.02220049,166.49823725)
\curveto(563.44620049,166.56223725)(562.96620049,166.64757058)(562.58220049,166.75423725)
\lineto(562.58220049,170.49823725)
\curveto(562.96620049,170.39157058)(563.32886716,170.31690392)(563.67020049,170.27423725)
\curveto(564.01153383,170.23157058)(564.39553383,170.21023725)(564.82220049,170.21023725)
\curveto(565.46220049,170.21023725)(566.00620049,170.39157058)(566.45420049,170.75423725)
\curveto(566.90220049,171.11690392)(567.12620049,171.82090392)(567.12620049,172.86623725)
\lineto(567.12620049,191.52223725)
\lineto(571.89420049,191.52223725)
\lineto(571.89420049,172.16223725)
\curveto(571.89420049,171.09557058)(571.69153383,170.12490392)(571.28620049,169.25023725)
\curveto(570.88086716,168.37557058)(570.21953383,167.68223725)(569.30220049,167.17023725)
\curveto(568.40620049,166.63690392)(567.21153383,166.37023725)(565.71820049,166.37023725)
\closepath
}
}
{
\newrgbcolor{curcolor}{0 0 0}
\pscustom[linestyle=none,fillstyle=solid,fillcolor=curcolor]
{
\newpath
\moveto(589.07821026,179.23423725)
\curveto(589.07821026,177.46357058)(588.44887693,176.09823725)(587.19021026,175.13823725)
\curveto(585.95287693,174.19957058)(584.09687693,173.73023725)(581.62221026,173.73023725)
\curveto(580.40621026,173.73023725)(579.36087693,173.81557058)(578.48621026,173.98623725)
\curveto(577.61154359,174.13557058)(576.73687693,174.39157058)(575.86221026,174.75423725)
\lineto(575.86221026,178.69023725)
\curveto(576.80087693,178.26357058)(577.81421026,177.91157058)(578.90221026,177.63423725)
\curveto(579.99021026,177.35690392)(580.95021026,177.21823725)(581.78221026,177.21823725)
\curveto(582.69954359,177.21823725)(583.36087693,177.35690392)(583.76621026,177.63423725)
\curveto(584.17154359,177.91157058)(584.37421026,178.27423725)(584.37421026,178.72223725)
\curveto(584.37421026,179.02090392)(584.28887693,179.28757058)(584.11821026,179.52223725)
\curveto(583.96887693,179.75690392)(583.62754359,180.02357058)(583.09421026,180.32223725)
\curveto(582.56087693,180.62090392)(581.72887693,181.00490392)(580.59821026,181.47423725)
\curveto(579.48887693,181.94357058)(578.58221026,182.40223725)(577.87821026,182.85023725)
\curveto(577.19554359,183.31957058)(576.68354359,183.87423725)(576.34221026,184.51423725)
\curveto(576.00087693,185.17557058)(575.83021026,185.99690392)(575.83021026,186.97823725)
\curveto(575.83021026,188.59957058)(576.45954359,189.81557058)(577.71821026,190.62623725)
\curveto(578.97687693,191.43690392)(580.65154359,191.84223725)(582.74221026,191.84223725)
\curveto(583.83021026,191.84223725)(584.86487693,191.73557058)(585.84621026,191.52223725)
\curveto(586.82754359,191.30890392)(587.84087693,190.95690392)(588.88621026,190.46623725)
\lineto(587.44621026,187.04223725)
\curveto(586.59287693,187.40490392)(585.78221026,187.70357058)(585.01421026,187.93823725)
\curveto(584.24621026,188.19423725)(583.46754359,188.32223725)(582.67821026,188.32223725)
\curveto(581.27021026,188.32223725)(580.56621026,187.93823725)(580.56621026,187.17023725)
\curveto(580.56621026,186.89290392)(580.65154359,186.63690392)(580.82221026,186.40223725)
\curveto(581.01421026,186.18890392)(581.36621026,185.95423725)(581.87821026,185.69823725)
\curveto(582.41154359,185.44223725)(583.19021026,185.10090392)(584.21421026,184.67423725)
\curveto(585.21687693,184.26890392)(586.08087693,183.84223725)(586.80621026,183.39423725)
\curveto(587.53154359,182.96757058)(588.08621026,182.42357058)(588.47021026,181.76223725)
\curveto(588.87554359,181.10090392)(589.07821026,180.25823725)(589.07821026,179.23423725)
\closepath
}
}
{
\newrgbcolor{curcolor}{0 0 0}
\pscustom[linestyle=none,fillstyle=solid,fillcolor=curcolor]
{
\newpath
\moveto(599.86220196,182.81823725)
\curveto(599.86220196,180.23690392)(599.47820196,177.75157058)(598.71020196,175.36223725)
\curveto(597.96353529,172.99423725)(596.77953529,170.87157058)(595.15820196,168.99423725)
\lineto(591.28620196,168.99423725)
\curveto(592.71553529,170.95690392)(593.80353529,173.13290392)(594.55020196,175.52223725)
\curveto(595.31820196,177.93290392)(595.70220196,180.37557058)(595.70220196,182.85023725)
\curveto(595.70220196,185.38890392)(595.31820196,187.86357058)(594.55020196,190.27423725)
\curveto(593.80353529,192.68490392)(592.70486862,194.89290392)(591.25420196,196.89823725)
\lineto(595.15820196,196.89823725)
\curveto(596.77953529,194.95690392)(597.96353529,192.77023725)(598.71020196,190.33823725)
\curveto(599.47820196,187.92757058)(599.86220196,185.42090392)(599.86220196,182.81823725)
\closepath
}
}
{
\newrgbcolor{curcolor}{0 0 0}
\pscustom[linestyle=none,fillstyle=solid,fillcolor=curcolor]
{
\newpath
\moveto(611.80517239,574.747023)
\lineto(611.80517239,597.595023)
\lineto(626.23717239,597.595023)
\lineto(626.23717239,593.595023)
\lineto(616.63717239,593.595023)
\lineto(616.63717239,588.827023)
\lineto(618.55717239,588.827023)
\curveto(620.71183906,588.827023)(622.47183906,588.52835633)(623.83717239,587.931023)
\curveto(625.22383906,587.33368967)(626.24783906,586.51235633)(626.90917239,585.467023)
\curveto(627.57050572,584.42168967)(627.90117239,583.227023)(627.90117239,581.883023)
\curveto(627.90117239,579.62168967)(627.14383906,577.86168967)(625.62917239,576.603023)
\curveto(624.13583906,575.36568967)(621.74650572,574.747023)(618.46117239,574.747023)
\closepath
\moveto(616.63717239,578.715023)
\lineto(618.26917239,578.715023)
\curveto(619.74117239,578.715023)(620.89317239,578.94968967)(621.72517239,579.419023)
\curveto(622.57850572,579.88835633)(623.00517239,580.70968967)(623.00517239,581.883023)
\curveto(623.00517239,583.099023)(622.54650572,583.899023)(621.62917239,584.283023)
\curveto(620.71183906,584.667023)(619.46383906,584.859023)(617.88517239,584.859023)
\lineto(616.63717239,584.859023)
\closepath
}
}
{
\newrgbcolor{curcolor}{0 0 0}
\pscustom[linestyle=none,fillstyle=solid,fillcolor=curcolor]
{
\newpath
\moveto(639.00518801,592.571023)
\curveto(641.35185468,592.571023)(643.14385468,592.059023)(644.38118801,591.035023)
\curveto(645.63985468,590.03235633)(646.26918801,588.48568967)(646.26918801,586.395023)
\lineto(646.26918801,574.747023)
\lineto(642.94118801,574.747023)
\lineto(642.01318801,577.115023)
\lineto(641.88518801,577.115023)
\curveto(641.13852135,576.17635633)(640.34918801,575.49368967)(639.51718801,575.067023)
\curveto(638.68518801,574.64035633)(637.54385468,574.427023)(636.09318801,574.427023)
\curveto(634.53585468,574.427023)(633.24518801,574.875023)(632.22118801,575.771023)
\curveto(631.19718801,576.667023)(630.68518801,578.06435633)(630.68518801,579.963023)
\curveto(630.68518801,581.819023)(631.33585468,583.18435633)(632.63718801,584.059023)
\curveto(633.93852135,584.93368967)(635.89052135,585.42435633)(638.49318801,585.531023)
\lineto(641.53318801,585.627023)
\lineto(641.53318801,586.395023)
\curveto(641.53318801,587.31235633)(641.28785468,587.98435633)(640.79718801,588.411023)
\curveto(640.32785468,588.83768967)(639.66652135,589.051023)(638.81318801,589.051023)
\curveto(637.95985468,589.051023)(637.12785468,588.923023)(636.31718801,588.667023)
\curveto(635.50652135,588.43235633)(634.69585468,588.13368967)(633.88518801,587.771023)
\lineto(632.31718801,591.003023)
\curveto(633.23452135,591.47235633)(634.26918801,591.84568967)(635.42118801,592.123023)
\curveto(636.57318801,592.42168967)(637.76785468,592.571023)(639.00518801,592.571023)
\closepath
\moveto(641.53318801,582.843023)
\lineto(639.67718801,582.779023)
\curveto(638.14118801,582.73635633)(637.07452135,582.459023)(636.47718801,581.947023)
\curveto(635.87985468,581.435023)(635.58118801,580.763023)(635.58118801,579.931023)
\curveto(635.58118801,579.20568967)(635.79452135,578.683023)(636.22118801,578.363023)
\curveto(636.64785468,578.06435633)(637.20252135,577.915023)(637.88518801,577.915023)
\curveto(638.90918801,577.915023)(639.77318801,578.21368967)(640.47718801,578.811023)
\curveto(641.18118801,579.42968967)(641.53318801,580.29368967)(641.53318801,581.403023)
\closepath
}
}
{
\newrgbcolor{curcolor}{0 0 0}
\pscustom[linestyle=none,fillstyle=solid,fillcolor=curcolor]
{
\newpath
\moveto(657.53319094,592.539023)
\curveto(658.79185761,592.539023)(659.96519094,592.36835633)(661.05319094,592.027023)
\curveto(662.16252428,591.707023)(663.04785761,591.20568967)(663.70919094,590.523023)
\curveto(664.39185761,589.84035633)(664.73319094,588.96568967)(664.73319094,587.899023)
\curveto(664.73319094,586.85368967)(664.41319094,586.02168967)(663.77319094,585.403023)
\curveto(663.13319094,584.78435633)(662.29052428,584.33635633)(661.24519094,584.059023)
\lineto(661.24519094,583.899023)
\curveto(661.99185761,583.72835633)(662.66385761,583.483023)(663.26119094,583.163023)
\curveto(663.85852428,582.86435633)(664.32785761,582.44835633)(664.66919094,581.915023)
\curveto(665.03185761,581.403023)(665.21319094,580.70968967)(665.21319094,579.835023)
\curveto(665.21319094,578.875023)(664.90385761,577.979023)(664.28519094,577.147023)
\curveto(663.68785761,576.33635633)(662.74919094,575.675023)(661.46919094,575.163023)
\curveto(660.18919094,574.67235633)(658.56785761,574.427023)(656.60519094,574.427023)
\curveto(653.70385761,574.427023)(651.46385761,574.78968967)(649.88519094,575.515023)
\lineto(649.88519094,579.451023)
\curveto(650.61052428,579.10968967)(651.49585761,578.80035633)(652.54119094,578.523023)
\curveto(653.60785761,578.24568967)(654.73852428,578.107023)(655.93319094,578.107023)
\curveto(657.23452428,578.107023)(658.33319094,578.25635633)(659.22919094,578.555023)
\curveto(660.12519094,578.85368967)(660.57319094,579.37635633)(660.57319094,580.123023)
\curveto(660.57319094,581.50968967)(658.91985761,582.203023)(655.61319094,582.203023)
\lineto(653.75719094,582.203023)
\lineto(653.75719094,585.499023)
\lineto(655.51719094,585.499023)
\curveto(657.09585761,585.499023)(658.31185761,585.627023)(659.16519094,585.883023)
\curveto(660.03985761,586.139023)(660.47719094,586.619023)(660.47719094,587.323023)
\curveto(660.47719094,587.87768967)(660.19985761,588.29368967)(659.64519094,588.571023)
\curveto(659.09052428,588.86968967)(658.18385761,589.019023)(656.92519094,589.019023)
\curveto(656.09319094,589.019023)(655.17585761,588.923023)(654.17319094,588.731023)
\curveto(653.19185761,588.539023)(652.27452428,588.26168967)(651.42119094,587.899023)
\lineto(650.01319094,591.227023)
\curveto(651.01585761,591.611023)(652.10385761,591.92035633)(653.27719094,592.155023)
\curveto(654.47185761,592.411023)(655.89052428,592.539023)(657.53319094,592.539023)
\closepath
}
}
{
\newrgbcolor{curcolor}{0 0 0}
\pscustom[linestyle=none,fillstyle=solid,fillcolor=curcolor]
{
\newpath
\moveto(676.09320071,592.571023)
\curveto(678.43986738,592.571023)(680.23186738,592.059023)(681.46920071,591.035023)
\curveto(682.72786738,590.03235633)(683.35720071,588.48568967)(683.35720071,586.395023)
\lineto(683.35720071,574.747023)
\lineto(680.02920071,574.747023)
\lineto(679.10120071,577.115023)
\lineto(678.97320071,577.115023)
\curveto(678.22653404,576.17635633)(677.43720071,575.49368967)(676.60520071,575.067023)
\curveto(675.77320071,574.64035633)(674.63186738,574.427023)(673.18120071,574.427023)
\curveto(671.62386738,574.427023)(670.33320071,574.875023)(669.30920071,575.771023)
\curveto(668.28520071,576.667023)(667.77320071,578.06435633)(667.77320071,579.963023)
\curveto(667.77320071,581.819023)(668.42386738,583.18435633)(669.72520071,584.059023)
\curveto(671.02653404,584.93368967)(672.97853404,585.42435633)(675.58120071,585.531023)
\lineto(678.62120071,585.627023)
\lineto(678.62120071,586.395023)
\curveto(678.62120071,587.31235633)(678.37586738,587.98435633)(677.88520071,588.411023)
\curveto(677.41586738,588.83768967)(676.75453404,589.051023)(675.90120071,589.051023)
\curveto(675.04786738,589.051023)(674.21586738,588.923023)(673.40520071,588.667023)
\curveto(672.59453404,588.43235633)(671.78386738,588.13368967)(670.97320071,587.771023)
\lineto(669.40520071,591.003023)
\curveto(670.32253404,591.47235633)(671.35720071,591.84568967)(672.50920071,592.123023)
\curveto(673.66120071,592.42168967)(674.85586738,592.571023)(676.09320071,592.571023)
\closepath
\moveto(678.62120071,582.843023)
\lineto(676.76520071,582.779023)
\curveto(675.22920071,582.73635633)(674.16253404,582.459023)(673.56520071,581.947023)
\curveto(672.96786738,581.435023)(672.66920071,580.763023)(672.66920071,579.931023)
\curveto(672.66920071,579.20568967)(672.88253404,578.683023)(673.30920071,578.363023)
\curveto(673.73586738,578.06435633)(674.29053404,577.915023)(674.97320071,577.915023)
\curveto(675.99720071,577.915023)(676.86120071,578.21368967)(677.56520071,578.811023)
\curveto(678.26920071,579.42968967)(678.62120071,580.29368967)(678.62120071,581.403023)
\closepath
}
}
{
\newrgbcolor{curcolor}{0 0 0}
\pscustom[linestyle=none,fillstyle=solid,fillcolor=curcolor]
{
\newpath
\moveto(712.28521096,592.219023)
\lineto(712.28521096,578.235023)
\lineto(714.84521096,578.235023)
\lineto(714.84521096,568.475023)
\lineto(710.55721096,568.475023)
\lineto(710.55721096,574.747023)
\lineto(698.81321096,574.747023)
\lineto(698.81321096,568.475023)
\lineto(694.52521096,568.475023)
\lineto(694.52521096,578.235023)
\lineto(695.99721096,578.235023)
\curveto(696.76521096,579.40835633)(697.41587763,580.74168967)(697.94921096,582.235023)
\curveto(698.4825443,583.74968967)(698.90921096,585.34968967)(699.22921096,587.035023)
\curveto(699.54921096,588.74168967)(699.78387763,590.46968967)(699.93321096,592.219023)
\closepath
\moveto(707.51721096,588.635023)
\lineto(703.93321096,588.635023)
\curveto(703.67721096,586.69368967)(703.32521096,584.84835633)(702.87721096,583.099023)
\curveto(702.42921096,581.371023)(701.79987763,579.74968967)(700.98921096,578.235023)
\lineto(707.51721096,578.235023)
\closepath
}
}
{
\newrgbcolor{curcolor}{0 0 0}
\pscustom[linestyle=none,fillstyle=solid,fillcolor=curcolor]
{
\newpath
\moveto(724.9891968,592.571023)
\curveto(727.33586347,592.571023)(729.12786347,592.059023)(730.3651968,591.035023)
\curveto(731.62386347,590.03235633)(732.2531968,588.48568967)(732.2531968,586.395023)
\lineto(732.2531968,574.747023)
\lineto(728.9251968,574.747023)
\lineto(727.9971968,577.115023)
\lineto(727.8691968,577.115023)
\curveto(727.12253014,576.17635633)(726.3331968,575.49368967)(725.5011968,575.067023)
\curveto(724.6691968,574.64035633)(723.52786347,574.427023)(722.0771968,574.427023)
\curveto(720.51986347,574.427023)(719.2291968,574.875023)(718.2051968,575.771023)
\curveto(717.1811968,576.667023)(716.6691968,578.06435633)(716.6691968,579.963023)
\curveto(716.6691968,581.819023)(717.31986347,583.18435633)(718.6211968,584.059023)
\curveto(719.92253014,584.93368967)(721.87453014,585.42435633)(724.4771968,585.531023)
\lineto(727.5171968,585.627023)
\lineto(727.5171968,586.395023)
\curveto(727.5171968,587.31235633)(727.27186347,587.98435633)(726.7811968,588.411023)
\curveto(726.31186347,588.83768967)(725.65053014,589.051023)(724.7971968,589.051023)
\curveto(723.94386347,589.051023)(723.11186347,588.923023)(722.3011968,588.667023)
\curveto(721.49053014,588.43235633)(720.67986347,588.13368967)(719.8691968,587.771023)
\lineto(718.3011968,591.003023)
\curveto(719.21853014,591.47235633)(720.2531968,591.84568967)(721.4051968,592.123023)
\curveto(722.5571968,592.42168967)(723.75186347,592.571023)(724.9891968,592.571023)
\closepath
\moveto(727.5171968,582.843023)
\lineto(725.6611968,582.779023)
\curveto(724.1251968,582.73635633)(723.05853014,582.459023)(722.4611968,581.947023)
\curveto(721.86386347,581.435023)(721.5651968,580.763023)(721.5651968,579.931023)
\curveto(721.5651968,579.20568967)(721.77853014,578.683023)(722.2051968,578.363023)
\curveto(722.63186347,578.06435633)(723.18653014,577.915023)(723.8691968,577.915023)
\curveto(724.8931968,577.915023)(725.7571968,578.21368967)(726.4611968,578.811023)
\curveto(727.1651968,579.42968967)(727.5171968,580.29368967)(727.5171968,581.403023)
\closepath
}
}
{
\newrgbcolor{curcolor}{0 0 0}
\pscustom[linestyle=none,fillstyle=solid,fillcolor=curcolor]
{
\newpath
\moveto(741.91719973,592.219023)
\lineto(741.91719973,585.499023)
\lineto(748.57319973,585.499023)
\lineto(748.57319973,592.219023)
\lineto(753.34119973,592.219023)
\lineto(753.34119973,574.747023)
\lineto(748.57319973,574.747023)
\lineto(748.57319973,581.947023)
\lineto(741.91719973,581.947023)
\lineto(741.91719973,574.747023)
\lineto(737.14919973,574.747023)
\lineto(737.14919973,592.219023)
\closepath
}
}
{
\newrgbcolor{curcolor}{0 0 0}
\pscustom[linestyle=none,fillstyle=solid,fillcolor=curcolor]
{
\newpath
\moveto(763.10122073,592.219023)
\lineto(763.10122073,585.499023)
\lineto(769.75722073,585.499023)
\lineto(769.75722073,592.219023)
\lineto(774.52522073,592.219023)
\lineto(774.52522073,574.747023)
\lineto(769.75722073,574.747023)
\lineto(769.75722073,581.947023)
\lineto(763.10122073,581.947023)
\lineto(763.10122073,574.747023)
\lineto(758.33322073,574.747023)
\lineto(758.33322073,592.219023)
\closepath
}
}
{
\newrgbcolor{curcolor}{0 0 0}
\pscustom[linestyle=none,fillstyle=solid,fillcolor=curcolor]
{
\newpath
\moveto(779.51724172,574.747023)
\lineto(779.51724172,592.219023)
\lineto(784.28524172,592.219023)
\lineto(784.28524172,585.467023)
\lineto(786.58924172,585.467023)
\curveto(789.25590839,585.467023)(791.22924172,585.04035633)(792.50924172,584.187023)
\curveto(793.78924172,583.33368967)(794.42924172,582.043023)(794.42924172,580.315023)
\curveto(794.42924172,578.60835633)(793.83190839,577.25368967)(792.63724172,576.251023)
\curveto(791.44257506,575.24835633)(789.47990839,574.747023)(786.74924172,574.747023)
\closepath
\moveto(796.95724172,574.747023)
\lineto(796.95724172,592.219023)
\lineto(801.72524172,592.219023)
\lineto(801.72524172,574.747023)
\closepath
\moveto(784.28524172,578.043023)
\lineto(786.49324172,578.043023)
\curveto(787.43190839,578.043023)(788.18924172,578.203023)(788.76524172,578.523023)
\curveto(789.36257506,578.86435633)(789.66124172,579.44035633)(789.66124172,580.251023)
\curveto(789.66124172,581.531023)(788.58390839,582.171023)(786.42924172,582.171023)
\lineto(784.28524172,582.171023)
\closepath
}
}
{
\newrgbcolor{curcolor}{0 0 0}
\pscustom[linestyle=none,fillstyle=solid,fillcolor=curcolor]
{
\newpath
\moveto(810.30125393,583.675023)
\lineto(804.66925393,592.219023)
\lineto(810.07725393,592.219023)
\lineto(813.46925393,586.651023)
\lineto(816.89325393,592.219023)
\lineto(822.30125393,592.219023)
\lineto(816.60525393,583.675023)
\lineto(822.55725393,574.747023)
\lineto(817.14925393,574.747023)
\lineto(813.46925393,580.731023)
\lineto(809.78925393,574.747023)
\lineto(804.38125393,574.747023)
\closepath
}
}
{
\newrgbcolor{curcolor}{0 0 0}
\pscustom[linestyle=none,fillstyle=solid,fillcolor=curcolor]
{
\newpath
\moveto(672.70117459,543.515023)
\curveto(672.70117459,546.11768967)(673.07450792,548.62435633)(673.82117459,551.035023)
\curveto(674.58917459,553.467023)(675.78384125,555.65368967)(677.40517459,557.595023)
\lineto(681.30917459,557.595023)
\curveto(679.85850792,555.58968967)(678.74917459,553.38168967)(677.98117459,550.971023)
\curveto(677.23450792,548.56035633)(676.86117459,546.08568967)(676.86117459,543.547023)
\curveto(676.86117459,541.07235633)(677.23450792,538.62968967)(677.98117459,536.219023)
\curveto(678.72784125,533.82968967)(679.82650792,531.65368967)(681.27717459,529.691023)
\lineto(677.40517459,529.691023)
\curveto(675.78384125,531.56835633)(674.58917459,533.691023)(673.82117459,536.059023)
\curveto(673.07450792,538.44835633)(672.70117459,540.93368967)(672.70117459,543.515023)
\closepath
}
}
{
\newrgbcolor{curcolor}{0 0 0}
\pscustom[linestyle=none,fillstyle=solid,fillcolor=curcolor]
{
\newpath
\moveto(705.88519705,546.203023)
\curveto(705.88519705,543.835023)(705.49053038,541.76568967)(704.70119705,539.995023)
\curveto(703.93319705,538.24568967)(702.73853038,536.88035633)(701.11719705,535.899023)
\curveto(699.51719705,534.91768967)(697.47986371,534.427023)(695.00519705,534.427023)
\curveto(692.53053038,534.427023)(690.48253038,534.91768967)(688.86119705,535.899023)
\curveto(687.26119705,536.88035633)(686.06653038,538.25635633)(685.27719705,540.027023)
\curveto(684.50919705,541.79768967)(684.12519705,543.867023)(684.12519705,546.235023)
\curveto(684.12519705,548.603023)(684.50919705,550.66168967)(685.27719705,552.411023)
\curveto(686.06653038,554.16035633)(687.26119705,555.515023)(688.86119705,556.475023)
\curveto(690.48253038,557.45635633)(692.54119705,557.947023)(695.03719705,557.947023)
\curveto(697.51186371,557.947023)(699.54919705,557.45635633)(701.14919705,556.475023)
\curveto(702.74919705,555.515023)(703.93319705,554.14968967)(704.70119705,552.379023)
\curveto(705.49053038,550.62968967)(705.88519705,548.571023)(705.88519705,546.203023)
\closepath
\moveto(689.21319705,546.203023)
\curveto(689.21319705,543.81368967)(689.67186371,541.92568967)(690.58919705,540.539023)
\curveto(691.50653038,539.17368967)(692.97853038,538.491023)(695.00519705,538.491023)
\curveto(697.07453038,538.491023)(698.55719705,539.17368967)(699.45319705,540.539023)
\curveto(700.34919705,541.92568967)(700.79719705,543.81368967)(700.79719705,546.203023)
\curveto(700.79719705,548.59235633)(700.34919705,550.46968967)(699.45319705,551.835023)
\curveto(698.55719705,553.22168967)(697.08519705,553.915023)(695.03719705,553.915023)
\curveto(692.98919705,553.915023)(691.50653038,553.22168967)(690.58919705,551.835023)
\curveto(689.67186371,550.46968967)(689.21319705,548.59235633)(689.21319705,546.203023)
\closepath
}
}
{
\newrgbcolor{curcolor}{0 0 0}
\pscustom[linestyle=none,fillstyle=solid,fillcolor=curcolor]
{
\newpath
\moveto(726.07717459,552.251023)
\curveto(726.07717459,550.715023)(725.53317459,549.47768967)(724.44517459,548.539023)
\curveto(723.35717459,547.60035633)(721.97050792,547.01368967)(720.28517459,546.779023)
\lineto(720.28517459,546.683023)
\curveto(722.37584125,546.46968967)(723.97584125,545.883023)(725.08517459,544.923023)
\curveto(726.21584125,543.98435633)(726.78117459,542.75768967)(726.78117459,541.243023)
\curveto(726.78117459,539.23768967)(725.95984125,537.595023)(724.31717459,536.315023)
\curveto(722.69584125,535.05635633)(720.30650792,534.427023)(717.14917459,534.427023)
\curveto(715.42117459,534.427023)(713.88517459,534.53368967)(712.54117459,534.747023)
\curveto(711.21850792,534.96035633)(710.06650792,535.26968967)(709.08517459,535.675023)
\lineto(709.08517459,539.739023)
\curveto(709.74650792,539.419023)(710.49317459,539.14168967)(711.32517459,538.907023)
\curveto(712.15717459,538.69368967)(712.99984125,538.523023)(713.85317459,538.395023)
\curveto(714.70650792,538.28835633)(715.49584125,538.235023)(716.22117459,538.235023)
\curveto(718.24784125,538.235023)(719.70917459,538.523023)(720.60517459,539.099023)
\curveto(721.52250792,539.69635633)(721.98117459,540.52835633)(721.98117459,541.595023)
\curveto(721.98117459,542.683023)(721.31984125,543.47235633)(719.99717459,543.963023)
\curveto(718.67450792,544.475023)(716.89317459,544.731023)(714.65317459,544.731023)
\lineto(712.50917459,544.731023)
\lineto(712.50917459,548.507023)
\lineto(714.42917459,548.507023)
\curveto(716.24250792,548.507023)(717.65050792,548.62435633)(718.65317459,548.859023)
\curveto(719.65584125,549.09368967)(720.35984125,549.435023)(720.76517459,549.883023)
\curveto(721.17050792,550.331023)(721.37317459,550.86435633)(721.37317459,551.483023)
\curveto(721.37317459,552.27235633)(721.02117459,552.891023)(720.31717459,553.339023)
\curveto(719.61317459,553.80835633)(718.57850792,554.043023)(717.21317459,554.043023)
\curveto(716.01850792,554.043023)(714.90917459,553.87235633)(713.88517459,553.531023)
\curveto(712.86117459,553.18968967)(711.90117459,552.74168967)(711.00517459,552.187023)
\lineto(708.89317459,555.419023)
\curveto(710.06650792,556.187023)(711.36784125,556.795023)(712.79717459,557.243023)
\curveto(714.22650792,557.691023)(715.92250792,557.915023)(717.88517459,557.915023)
\curveto(720.46650792,557.915023)(722.47184125,557.38168967)(723.90117459,556.315023)
\curveto(725.35184125,555.24835633)(726.07717459,553.89368967)(726.07717459,552.251023)
\closepath
}
}
{
\newrgbcolor{curcolor}{0 0 0}
\pscustom[linestyle=none,fillstyle=solid,fillcolor=curcolor]
{
\newpath
\moveto(749.37319754,557.595023)
\lineto(742.30119754,541.435023)
\curveto(741.66119754,539.963023)(740.97853087,538.70435633)(740.25319754,537.659023)
\curveto(739.5278642,536.61368967)(738.61053087,535.81368967)(737.50119754,535.259023)
\curveto(736.41319754,534.70435633)(734.97319754,534.427023)(733.18119754,534.427023)
\curveto(732.62653087,534.427023)(732.01853087,534.46968967)(731.35719754,534.555023)
\curveto(730.6958642,534.64035633)(730.0878642,534.75768967)(729.53319754,534.907023)
\lineto(729.53319754,539.067023)
\curveto(730.04519754,538.85368967)(730.5998642,538.70435633)(731.19719754,538.619023)
\curveto(731.8158642,538.53368967)(732.40253087,538.491023)(732.95719754,538.491023)
\curveto(734.0238642,538.491023)(734.7918642,538.747023)(735.26119754,539.259023)
\curveto(735.73053087,539.79235633)(736.1038642,540.43235633)(736.38119754,541.179023)
\lineto(728.47719754,557.595023)
\lineto(733.59719754,557.595023)
\lineto(737.85319754,547.707023)
\curveto(738.00253087,547.387023)(738.20519754,546.92835633)(738.46119754,546.331023)
\curveto(738.71719754,545.755023)(738.90919754,545.26435633)(739.03719754,544.859023)
\lineto(739.19719754,544.859023)
\curveto(739.32519754,545.243023)(739.50653087,545.74435633)(739.74119754,546.363023)
\curveto(739.99719754,546.98168967)(740.22119754,547.52568967)(740.41319754,547.995023)
\lineto(744.38119754,557.595023)
\closepath
}
}
{
\newrgbcolor{curcolor}{0 0 0}
\pscustom[linestyle=none,fillstyle=solid,fillcolor=curcolor]
{
\newpath
\moveto(758.94120095,543.515023)
\curveto(758.94120095,540.93368967)(758.55720095,538.44835633)(757.78920095,536.059023)
\curveto(757.04253429,533.691023)(755.85853429,531.56835633)(754.23720095,529.691023)
\lineto(750.36520095,529.691023)
\curveto(751.79453429,531.65368967)(752.88253429,533.82968967)(753.62920095,536.219023)
\curveto(754.39720095,538.62968967)(754.78120095,541.07235633)(754.78120095,543.547023)
\curveto(754.78120095,546.08568967)(754.39720095,548.56035633)(753.62920095,550.971023)
\curveto(752.88253429,553.38168967)(751.78386762,555.58968967)(750.33320095,557.595023)
\lineto(754.23720095,557.595023)
\curveto(755.85853429,555.65368967)(757.04253429,553.467023)(757.78920095,551.035023)
\curveto(758.55720095,548.62435633)(758.94120095,546.11768967)(758.94120095,543.515023)
\closepath
}
}
{
\newrgbcolor{curcolor}{0 0 0}
\pscustom[linestyle=none,fillstyle=solid,fillcolor=curcolor]
{
\newpath
\moveto(764.29063203,314.3569135)
\lineto(764.29063203,337.2049135)
\lineto(778.72263203,337.2049135)
\lineto(778.72263203,333.2049135)
\lineto(769.12263203,333.2049135)
\lineto(769.12263203,328.4369135)
\lineto(771.04263203,328.4369135)
\curveto(773.19729869,328.4369135)(774.95729869,328.13824683)(776.32263203,327.5409135)
\curveto(777.70929869,326.94358017)(778.73329869,326.12224683)(779.39463203,325.0769135)
\curveto(780.05596536,324.03158017)(780.38663203,322.8369135)(780.38663203,321.4929135)
\curveto(780.38663203,319.23158017)(779.62929869,317.47158017)(778.11463203,316.2129135)
\curveto(776.62129869,314.97558017)(774.23196536,314.3569135)(770.94663203,314.3569135)
\closepath
\moveto(769.12263203,318.3249135)
\lineto(770.75463203,318.3249135)
\curveto(772.22663203,318.3249135)(773.37863203,318.55958017)(774.21063203,319.0289135)
\curveto(775.06396536,319.49824683)(775.49063203,320.31958017)(775.49063203,321.4929135)
\curveto(775.49063203,322.7089135)(775.03196536,323.5089135)(774.11463203,323.8929135)
\curveto(773.19729869,324.2769135)(771.94929869,324.4689135)(770.37063203,324.4689135)
\lineto(769.12263203,324.4689135)
\closepath
}
}
{
\newrgbcolor{curcolor}{0 0 0}
\pscustom[linestyle=none,fillstyle=solid,fillcolor=curcolor]
{
\newpath
\moveto(791.49064765,332.1809135)
\curveto(793.83731432,332.1809135)(795.62931432,331.6689135)(796.86664765,330.6449135)
\curveto(798.12531432,329.64224683)(798.75464765,328.09558017)(798.75464765,326.0049135)
\lineto(798.75464765,314.3569135)
\lineto(795.42664765,314.3569135)
\lineto(794.49864765,316.7249135)
\lineto(794.37064765,316.7249135)
\curveto(793.62398098,315.78624683)(792.83464765,315.10358017)(792.00264765,314.6769135)
\curveto(791.17064765,314.25024683)(790.02931432,314.0369135)(788.57864765,314.0369135)
\curveto(787.02131432,314.0369135)(785.73064765,314.4849135)(784.70664765,315.3809135)
\curveto(783.68264765,316.2769135)(783.17064765,317.67424683)(783.17064765,319.5729135)
\curveto(783.17064765,321.4289135)(783.82131432,322.79424683)(785.12264765,323.6689135)
\curveto(786.42398098,324.54358017)(788.37598098,325.03424683)(790.97864765,325.1409135)
\lineto(794.01864765,325.2369135)
\lineto(794.01864765,326.0049135)
\curveto(794.01864765,326.92224683)(793.77331432,327.59424683)(793.28264765,328.0209135)
\curveto(792.81331432,328.44758017)(792.15198098,328.6609135)(791.29864765,328.6609135)
\curveto(790.44531432,328.6609135)(789.61331432,328.5329135)(788.80264765,328.2769135)
\curveto(787.99198098,328.04224683)(787.18131432,327.74358017)(786.37064765,327.3809135)
\lineto(784.80264765,330.6129135)
\curveto(785.71998098,331.08224683)(786.75464765,331.45558017)(787.90664765,331.7329135)
\curveto(789.05864765,332.03158017)(790.25331432,332.1809135)(791.49064765,332.1809135)
\closepath
\moveto(794.01864765,322.4529135)
\lineto(792.16264765,322.3889135)
\curveto(790.62664765,322.34624683)(789.55998098,322.0689135)(788.96264765,321.5569135)
\curveto(788.36531432,321.0449135)(788.06664765,320.3729135)(788.06664765,319.5409135)
\curveto(788.06664765,318.81558017)(788.27998098,318.2929135)(788.70664765,317.9729135)
\curveto(789.13331432,317.67424683)(789.68798098,317.5249135)(790.37064765,317.5249135)
\curveto(791.39464765,317.5249135)(792.25864765,317.82358017)(792.96264765,318.4209135)
\curveto(793.66664765,319.03958017)(794.01864765,319.90358017)(794.01864765,321.0129135)
\closepath
}
}
{
\newrgbcolor{curcolor}{0 0 0}
\pscustom[linestyle=none,fillstyle=solid,fillcolor=curcolor]
{
\newpath
\moveto(810.01865058,332.1489135)
\curveto(811.27731725,332.1489135)(812.45065058,331.97824683)(813.53865058,331.6369135)
\curveto(814.64798391,331.3169135)(815.53331725,330.81558017)(816.19465058,330.1329135)
\curveto(816.87731725,329.45024683)(817.21865058,328.57558017)(817.21865058,327.5089135)
\curveto(817.21865058,326.46358017)(816.89865058,325.63158017)(816.25865058,325.0129135)
\curveto(815.61865058,324.39424683)(814.77598391,323.94624683)(813.73065058,323.6689135)
\lineto(813.73065058,323.5089135)
\curveto(814.47731725,323.33824683)(815.14931725,323.0929135)(815.74665058,322.7729135)
\curveto(816.34398391,322.47424683)(816.81331725,322.05824683)(817.15465058,321.5249135)
\curveto(817.51731725,321.0129135)(817.69865058,320.31958017)(817.69865058,319.4449135)
\curveto(817.69865058,318.4849135)(817.38931725,317.5889135)(816.77065058,316.7569135)
\curveto(816.17331725,315.94624683)(815.23465058,315.2849135)(813.95465058,314.7729135)
\curveto(812.67465058,314.28224683)(811.05331725,314.0369135)(809.09065058,314.0369135)
\curveto(806.18931725,314.0369135)(803.94931725,314.39958017)(802.37065058,315.1249135)
\lineto(802.37065058,319.0609135)
\curveto(803.09598391,318.71958017)(803.98131725,318.41024683)(805.02665058,318.1329135)
\curveto(806.09331725,317.85558017)(807.22398391,317.7169135)(808.41865058,317.7169135)
\curveto(809.71998391,317.7169135)(810.81865058,317.86624683)(811.71465058,318.1649135)
\curveto(812.61065058,318.46358017)(813.05865058,318.98624683)(813.05865058,319.7329135)
\curveto(813.05865058,321.11958017)(811.40531725,321.8129135)(808.09865058,321.8129135)
\lineto(806.24265058,321.8129135)
\lineto(806.24265058,325.1089135)
\lineto(808.00265058,325.1089135)
\curveto(809.58131725,325.1089135)(810.79731725,325.2369135)(811.65065058,325.4929135)
\curveto(812.52531725,325.7489135)(812.96265058,326.2289135)(812.96265058,326.9329135)
\curveto(812.96265058,327.48758017)(812.68531725,327.90358017)(812.13065058,328.1809135)
\curveto(811.57598391,328.47958017)(810.66931725,328.6289135)(809.41065058,328.6289135)
\curveto(808.57865058,328.6289135)(807.66131725,328.5329135)(806.65865058,328.3409135)
\curveto(805.67731725,328.1489135)(804.75998391,327.87158017)(803.90665058,327.5089135)
\lineto(802.49865058,330.8369135)
\curveto(803.50131725,331.2209135)(804.58931725,331.53024683)(805.76265058,331.7649135)
\curveto(806.95731725,332.0209135)(808.37598391,332.1489135)(810.01865058,332.1489135)
\closepath
}
}
{
\newrgbcolor{curcolor}{0 0 0}
\pscustom[linestyle=none,fillstyle=solid,fillcolor=curcolor]
{
\newpath
\moveto(828.57866035,332.1809135)
\curveto(830.92532701,332.1809135)(832.71732701,331.6689135)(833.95466035,330.6449135)
\curveto(835.21332701,329.64224683)(835.84266035,328.09558017)(835.84266035,326.0049135)
\lineto(835.84266035,314.3569135)
\lineto(832.51466035,314.3569135)
\lineto(831.58666035,316.7249135)
\lineto(831.45866035,316.7249135)
\curveto(830.71199368,315.78624683)(829.92266035,315.10358017)(829.09066035,314.6769135)
\curveto(828.25866035,314.25024683)(827.11732701,314.0369135)(825.66666035,314.0369135)
\curveto(824.10932701,314.0369135)(822.81866035,314.4849135)(821.79466035,315.3809135)
\curveto(820.77066035,316.2769135)(820.25866035,317.67424683)(820.25866035,319.5729135)
\curveto(820.25866035,321.4289135)(820.90932701,322.79424683)(822.21066035,323.6689135)
\curveto(823.51199368,324.54358017)(825.46399368,325.03424683)(828.06666035,325.1409135)
\lineto(831.10666035,325.2369135)
\lineto(831.10666035,326.0049135)
\curveto(831.10666035,326.92224683)(830.86132701,327.59424683)(830.37066035,328.0209135)
\curveto(829.90132701,328.44758017)(829.23999368,328.6609135)(828.38666035,328.6609135)
\curveto(827.53332701,328.6609135)(826.70132701,328.5329135)(825.89066035,328.2769135)
\curveto(825.07999368,328.04224683)(824.26932701,327.74358017)(823.45866035,327.3809135)
\lineto(821.89066035,330.6129135)
\curveto(822.80799368,331.08224683)(823.84266035,331.45558017)(824.99466035,331.7329135)
\curveto(826.14666035,332.03158017)(827.34132701,332.1809135)(828.57866035,332.1809135)
\closepath
\moveto(831.10666035,322.4529135)
\lineto(829.25066035,322.3889135)
\curveto(827.71466035,322.34624683)(826.64799368,322.0689135)(826.05066035,321.5569135)
\curveto(825.45332701,321.0449135)(825.15466035,320.3729135)(825.15466035,319.5409135)
\curveto(825.15466035,318.81558017)(825.36799368,318.2929135)(825.79466035,317.9729135)
\curveto(826.22132701,317.67424683)(826.77599368,317.5249135)(827.45866035,317.5249135)
\curveto(828.48266035,317.5249135)(829.34666035,317.82358017)(830.05066035,318.4209135)
\curveto(830.75466035,319.03958017)(831.10666035,319.90358017)(831.10666035,321.0129135)
\closepath
}
}
{
\newrgbcolor{curcolor}{0 0 0}
\pscustom[linestyle=none,fillstyle=solid,fillcolor=curcolor]
{
\newpath
\moveto(864.7706706,331.8289135)
\lineto(864.7706706,317.8449135)
\lineto(867.3306706,317.8449135)
\lineto(867.3306706,308.0849135)
\lineto(863.0426706,308.0849135)
\lineto(863.0426706,314.3569135)
\lineto(851.2986706,314.3569135)
\lineto(851.2986706,308.0849135)
\lineto(847.0106706,308.0849135)
\lineto(847.0106706,317.8449135)
\lineto(848.4826706,317.8449135)
\curveto(849.2506706,319.01824683)(849.90133727,320.35158017)(850.4346706,321.8449135)
\curveto(850.96800393,323.35958017)(851.3946706,324.95958017)(851.7146706,326.6449135)
\curveto(852.0346706,328.35158017)(852.26933727,330.07958017)(852.4186706,331.8289135)
\closepath
\moveto(860.0026706,328.2449135)
\lineto(856.4186706,328.2449135)
\curveto(856.1626706,326.30358017)(855.8106706,324.45824683)(855.3626706,322.7089135)
\curveto(854.9146706,320.9809135)(854.28533727,319.35958017)(853.4746706,317.8449135)
\lineto(860.0026706,317.8449135)
\closepath
}
}
{
\newrgbcolor{curcolor}{0 0 0}
\pscustom[linestyle=none,fillstyle=solid,fillcolor=curcolor]
{
\newpath
\moveto(877.47465644,332.1809135)
\curveto(879.82132311,332.1809135)(881.61332311,331.6689135)(882.85065644,330.6449135)
\curveto(884.10932311,329.64224683)(884.73865644,328.09558017)(884.73865644,326.0049135)
\lineto(884.73865644,314.3569135)
\lineto(881.41065644,314.3569135)
\lineto(880.48265644,316.7249135)
\lineto(880.35465644,316.7249135)
\curveto(879.60798977,315.78624683)(878.81865644,315.10358017)(877.98665644,314.6769135)
\curveto(877.15465644,314.25024683)(876.01332311,314.0369135)(874.56265644,314.0369135)
\curveto(873.00532311,314.0369135)(871.71465644,314.4849135)(870.69065644,315.3809135)
\curveto(869.66665644,316.2769135)(869.15465644,317.67424683)(869.15465644,319.5729135)
\curveto(869.15465644,321.4289135)(869.80532311,322.79424683)(871.10665644,323.6689135)
\curveto(872.40798977,324.54358017)(874.35998977,325.03424683)(876.96265644,325.1409135)
\lineto(880.00265644,325.2369135)
\lineto(880.00265644,326.0049135)
\curveto(880.00265644,326.92224683)(879.75732311,327.59424683)(879.26665644,328.0209135)
\curveto(878.79732311,328.44758017)(878.13598977,328.6609135)(877.28265644,328.6609135)
\curveto(876.42932311,328.6609135)(875.59732311,328.5329135)(874.78665644,328.2769135)
\curveto(873.97598977,328.04224683)(873.16532311,327.74358017)(872.35465644,327.3809135)
\lineto(870.78665644,330.6129135)
\curveto(871.70398977,331.08224683)(872.73865644,331.45558017)(873.89065644,331.7329135)
\curveto(875.04265644,332.03158017)(876.23732311,332.1809135)(877.47465644,332.1809135)
\closepath
\moveto(880.00265644,322.4529135)
\lineto(878.14665644,322.3889135)
\curveto(876.61065644,322.34624683)(875.54398977,322.0689135)(874.94665644,321.5569135)
\curveto(874.34932311,321.0449135)(874.05065644,320.3729135)(874.05065644,319.5409135)
\curveto(874.05065644,318.81558017)(874.26398977,318.2929135)(874.69065644,317.9729135)
\curveto(875.11732311,317.67424683)(875.67198977,317.5249135)(876.35465644,317.5249135)
\curveto(877.37865644,317.5249135)(878.24265644,317.82358017)(878.94665644,318.4209135)
\curveto(879.65065644,319.03958017)(880.00265644,319.90358017)(880.00265644,321.0129135)
\closepath
}
}
{
\newrgbcolor{curcolor}{0 0 0}
\pscustom[linestyle=none,fillstyle=solid,fillcolor=curcolor]
{
\newpath
\moveto(894.40265937,331.8289135)
\lineto(894.40265937,325.1089135)
\lineto(901.05865937,325.1089135)
\lineto(901.05865937,331.8289135)
\lineto(905.82665937,331.8289135)
\lineto(905.82665937,314.3569135)
\lineto(901.05865937,314.3569135)
\lineto(901.05865937,321.5569135)
\lineto(894.40265937,321.5569135)
\lineto(894.40265937,314.3569135)
\lineto(889.63465937,314.3569135)
\lineto(889.63465937,331.8289135)
\closepath
}
}
{
\newrgbcolor{curcolor}{0 0 0}
\pscustom[linestyle=none,fillstyle=solid,fillcolor=curcolor]
{
\newpath
\moveto(915.58668037,331.8289135)
\lineto(915.58668037,325.1089135)
\lineto(922.24268037,325.1089135)
\lineto(922.24268037,331.8289135)
\lineto(927.01068037,331.8289135)
\lineto(927.01068037,314.3569135)
\lineto(922.24268037,314.3569135)
\lineto(922.24268037,321.5569135)
\lineto(915.58668037,321.5569135)
\lineto(915.58668037,314.3569135)
\lineto(910.81868037,314.3569135)
\lineto(910.81868037,331.8289135)
\closepath
}
}
{
\newrgbcolor{curcolor}{0 0 0}
\pscustom[linestyle=none,fillstyle=solid,fillcolor=curcolor]
{
\newpath
\moveto(932.00270136,314.3569135)
\lineto(932.00270136,331.8289135)
\lineto(936.77070136,331.8289135)
\lineto(936.77070136,325.0769135)
\lineto(939.07470136,325.0769135)
\curveto(941.74136803,325.0769135)(943.71470136,324.65024683)(944.99470136,323.7969135)
\curveto(946.27470136,322.94358017)(946.91470136,321.6529135)(946.91470136,319.9249135)
\curveto(946.91470136,318.21824683)(946.31736803,316.86358017)(945.12270136,315.8609135)
\curveto(943.9280347,314.85824683)(941.96536803,314.3569135)(939.23470136,314.3569135)
\closepath
\moveto(949.44270136,314.3569135)
\lineto(949.44270136,331.8289135)
\lineto(954.21070136,331.8289135)
\lineto(954.21070136,314.3569135)
\closepath
\moveto(936.77070136,317.6529135)
\lineto(938.97870136,317.6529135)
\curveto(939.91736803,317.6529135)(940.67470136,317.8129135)(941.25070136,318.1329135)
\curveto(941.8480347,318.47424683)(942.14670136,319.05024683)(942.14670136,319.8609135)
\curveto(942.14670136,321.1409135)(941.06936803,321.7809135)(938.91470136,321.7809135)
\lineto(936.77070136,321.7809135)
\closepath
}
}
{
\newrgbcolor{curcolor}{0 0 0}
\pscustom[linestyle=none,fillstyle=solid,fillcolor=curcolor]
{
\newpath
\moveto(962.78671357,323.2849135)
\lineto(957.15471357,331.8289135)
\lineto(962.56271357,331.8289135)
\lineto(965.95471357,326.2609135)
\lineto(969.37871357,331.8289135)
\lineto(974.78671357,331.8289135)
\lineto(969.09071357,323.2849135)
\lineto(975.04271357,314.3569135)
\lineto(969.63471357,314.3569135)
\lineto(965.95471357,320.3409135)
\lineto(962.27471357,314.3569135)
\lineto(956.86671357,314.3569135)
\closepath
}
}
{
\newrgbcolor{curcolor}{0 0 0}
\pscustom[linestyle=none,fillstyle=solid,fillcolor=curcolor]
{
\newpath
\moveto(757.10667817,283.1249135)
\curveto(757.10667817,285.72758017)(757.4800115,288.23424683)(758.22667817,290.6449135)
\curveto(758.99467817,293.0769135)(760.18934484,295.26358017)(761.81067817,297.2049135)
\lineto(765.71467817,297.2049135)
\curveto(764.2640115,295.19958017)(763.15467817,292.99158017)(762.38667817,290.5809135)
\curveto(761.6400115,288.17024683)(761.26667817,285.69558017)(761.26667817,283.1569135)
\curveto(761.26667817,280.68224683)(761.6400115,278.23958017)(762.38667817,275.8289135)
\curveto(763.13334484,273.43958017)(764.2320115,271.26358017)(765.68267817,269.3009135)
\lineto(761.81067817,269.3009135)
\curveto(760.18934484,271.17824683)(758.99467817,273.3009135)(758.22667817,275.6689135)
\curveto(757.4800115,278.05824683)(757.10667817,280.54358017)(757.10667817,283.1249135)
\closepath
}
}
{
\newrgbcolor{curcolor}{0 0 0}
\pscustom[linestyle=none,fillstyle=solid,fillcolor=curcolor]
{
\newpath
\moveto(788.27470063,274.3569135)
\lineto(783.44270063,274.3569135)
\lineto(783.44270063,284.2129135)
\lineto(774.38670063,284.2129135)
\lineto(774.38670063,274.3569135)
\lineto(769.55470063,274.3569135)
\lineto(769.55470063,297.2049135)
\lineto(774.38670063,297.2049135)
\lineto(774.38670063,288.2449135)
\lineto(783.44270063,288.2449135)
\lineto(783.44270063,297.2049135)
\lineto(788.27470063,297.2049135)
\closepath
}
}
{
\newrgbcolor{curcolor}{0 0 0}
\pscustom[linestyle=none,fillstyle=solid,fillcolor=curcolor]
{
\newpath
\moveto(800.8186811,292.1809135)
\curveto(803.16534777,292.1809135)(804.95734777,291.6689135)(806.1946811,290.6449135)
\curveto(807.45334777,289.64224683)(808.0826811,288.09558017)(808.0826811,286.0049135)
\lineto(808.0826811,274.3569135)
\lineto(804.7546811,274.3569135)
\lineto(803.8266811,276.7249135)
\lineto(803.6986811,276.7249135)
\curveto(802.95201443,275.78624683)(802.1626811,275.10358017)(801.3306811,274.6769135)
\curveto(800.4986811,274.25024683)(799.35734777,274.0369135)(797.9066811,274.0369135)
\curveto(796.34934777,274.0369135)(795.0586811,274.4849135)(794.0346811,275.3809135)
\curveto(793.0106811,276.2769135)(792.4986811,277.67424683)(792.4986811,279.5729135)
\curveto(792.4986811,281.4289135)(793.14934777,282.79424683)(794.4506811,283.6689135)
\curveto(795.75201443,284.54358017)(797.70401443,285.03424683)(800.3066811,285.1409135)
\lineto(803.3466811,285.2369135)
\lineto(803.3466811,286.0049135)
\curveto(803.3466811,286.92224683)(803.10134777,287.59424683)(802.6106811,288.0209135)
\curveto(802.14134777,288.44758017)(801.48001443,288.6609135)(800.6266811,288.6609135)
\curveto(799.77334777,288.6609135)(798.94134777,288.5329135)(798.1306811,288.2769135)
\curveto(797.32001443,288.04224683)(796.50934777,287.74358017)(795.6986811,287.3809135)
\lineto(794.1306811,290.6129135)
\curveto(795.04801443,291.08224683)(796.0826811,291.45558017)(797.2346811,291.7329135)
\curveto(798.3866811,292.03158017)(799.58134777,292.1809135)(800.8186811,292.1809135)
\closepath
\moveto(803.3466811,282.4529135)
\lineto(801.4906811,282.3889135)
\curveto(799.9546811,282.34624683)(798.88801443,282.0689135)(798.2906811,281.5569135)
\curveto(797.69334777,281.0449135)(797.3946811,280.3729135)(797.3946811,279.5409135)
\curveto(797.3946811,278.81558017)(797.60801443,278.2929135)(798.0346811,277.9729135)
\curveto(798.46134777,277.67424683)(799.01601443,277.5249135)(799.6986811,277.5249135)
\curveto(800.7226811,277.5249135)(801.5866811,277.82358017)(802.2906811,278.4209135)
\curveto(802.9946811,279.03958017)(803.3466811,279.90358017)(803.3466811,281.0129135)
\closepath
}
}
{
\newrgbcolor{curcolor}{0 0 0}
\pscustom[linestyle=none,fillstyle=solid,fillcolor=curcolor]
{
\newpath
\moveto(824.43468403,291.8289135)
\lineto(829.68268403,291.8289135)
\lineto(822.77068403,283.4449135)
\lineto(830.29068403,274.3569135)
\lineto(824.88268403,274.3569135)
\lineto(817.74668403,283.2209135)
\lineto(817.74668403,274.3569135)
\lineto(812.97868403,274.3569135)
\lineto(812.97868403,291.8289135)
\lineto(817.74668403,291.8289135)
\lineto(817.74668403,283.3489135)
\closepath
}
}
{
\newrgbcolor{curcolor}{0 0 0}
\pscustom[linestyle=none,fillstyle=solid,fillcolor=curcolor]
{
\newpath
\moveto(848.01865278,283.1249135)
\curveto(848.01865278,280.22358017)(847.25065278,277.98358017)(845.71465278,276.4049135)
\curveto(844.19998611,274.82624683)(842.13065278,274.0369135)(839.50665278,274.0369135)
\curveto(837.88531944,274.0369135)(836.43465278,274.3889135)(835.15465278,275.0929135)
\curveto(833.89598611,275.7969135)(832.90398611,276.8209135)(832.17865278,278.1649135)
\curveto(831.45331944,279.53024683)(831.09065278,281.18358017)(831.09065278,283.1249135)
\curveto(831.09065278,286.02624683)(831.84798611,288.25558017)(833.36265278,289.8129135)
\curveto(834.87731944,291.37024683)(836.95731944,292.1489135)(839.60265278,292.1489135)
\curveto(841.24531944,292.1489135)(842.69598611,291.7969135)(843.95465278,291.0929135)
\curveto(845.21331944,290.3889135)(846.20531944,289.3649135)(846.93065278,288.0209135)
\curveto(847.65598611,286.6769135)(848.01865278,285.0449135)(848.01865278,283.1249135)
\closepath
\moveto(835.95465278,283.1249135)
\curveto(835.95465278,281.3969135)(836.23198611,280.0849135)(836.78665278,279.1889135)
\curveto(837.36265278,278.31424683)(838.29065278,277.8769135)(839.57065278,277.8769135)
\curveto(840.82931944,277.8769135)(841.73598611,278.31424683)(842.29065278,279.1889135)
\curveto(842.86665278,280.0849135)(843.15465278,281.3969135)(843.15465278,283.1249135)
\curveto(843.15465278,284.8529135)(842.86665278,286.14358017)(842.29065278,286.9969135)
\curveto(841.73598611,287.87158017)(840.81865278,288.3089135)(839.53865278,288.3089135)
\curveto(838.27998611,288.3089135)(837.36265278,287.87158017)(836.78665278,286.9969135)
\curveto(836.23198611,286.14358017)(835.95465278,284.8529135)(835.95465278,283.1249135)
\closepath
}
}
{
\newrgbcolor{curcolor}{0 0 0}
\pscustom[linestyle=none,fillstyle=solid,fillcolor=curcolor]
{
\newpath
\moveto(867.82663618,291.8289135)
\lineto(867.82663618,274.3569135)
\lineto(863.05863618,274.3569135)
\lineto(863.05863618,288.2449135)
\lineto(856.72263618,288.2449135)
\lineto(856.72263618,274.3569135)
\lineto(851.95463618,274.3569135)
\lineto(851.95463618,291.8289135)
\closepath
}
}
{
\newrgbcolor{curcolor}{0 0 0}
\pscustom[linestyle=none,fillstyle=solid,fillcolor=curcolor]
{
\newpath
\moveto(877.42664985,291.8289135)
\lineto(877.42664985,284.9169135)
\curveto(877.42664985,284.55424683)(877.40531652,284.10624683)(877.36264985,283.5729135)
\curveto(877.34131652,283.03958017)(877.30931652,282.49558017)(877.26664985,281.9409135)
\curveto(877.24531652,281.38624683)(877.21331652,280.8849135)(877.17064985,280.4369135)
\curveto(877.12798318,280.01024683)(877.09598318,279.72224683)(877.07464985,279.5729135)
\lineto(885.13864985,291.8289135)
\lineto(890.86664985,291.8289135)
\lineto(890.86664985,274.3569135)
\lineto(886.25864985,274.3569135)
\lineto(886.25864985,281.3329135)
\curveto(886.25864985,281.88758017)(886.27998318,282.5169135)(886.32264985,283.2209135)
\curveto(886.36531652,283.9249135)(886.40798318,284.57558017)(886.45064985,285.1729135)
\curveto(886.51464985,285.79158017)(886.55731652,286.2609135)(886.57864985,286.5809135)
\lineto(878.54664985,274.3569135)
\lineto(872.81864985,274.3569135)
\lineto(872.81864985,291.8289135)
\closepath
}
}
{
\newrgbcolor{curcolor}{0 0 0}
\pscustom[linestyle=none,fillstyle=solid,fillcolor=curcolor]
{
\newpath
\moveto(910.32262788,288.2449135)
\lineto(904.59462788,288.2449135)
\lineto(904.59462788,274.3569135)
\lineto(899.82662788,274.3569135)
\lineto(899.82662788,288.2449135)
\lineto(894.09862788,288.2449135)
\lineto(894.09862788,291.8289135)
\lineto(910.32262788,291.8289135)
\closepath
}
}
{
\newrgbcolor{curcolor}{0 0 0}
\pscustom[linestyle=none,fillstyle=solid,fillcolor=curcolor]
{
\newpath
\moveto(920.43460444,292.1489135)
\curveto(922.8452711,292.1489135)(924.75460444,291.45558017)(926.16260444,290.0689135)
\curveto(927.57060444,288.70358017)(928.27460444,286.75158017)(928.27460444,284.2129135)
\lineto(928.27460444,281.9089135)
\lineto(917.01060444,281.9089135)
\curveto(917.0532711,280.5649135)(917.44793777,279.5089135)(918.19460444,278.7409135)
\curveto(918.96260444,277.9729135)(920.01860444,277.5889135)(921.36260444,277.5889135)
\curveto(922.47193777,277.5889135)(923.4852711,277.69558017)(924.40260444,277.9089135)
\curveto(925.3412711,278.14358017)(926.3012711,278.49558017)(927.28260444,278.9649135)
\lineto(927.28260444,275.2849135)
\curveto(926.40793777,274.85824683)(925.5012711,274.5489135)(924.56260444,274.3569135)
\curveto(923.62393777,274.14358017)(922.48260444,274.0369135)(921.13860444,274.0369135)
\curveto(919.3892711,274.0369135)(917.84260444,274.3569135)(916.49860444,274.9969135)
\curveto(915.15460444,275.65824683)(914.09860444,276.63958017)(913.33060444,277.9409135)
\curveto(912.56260444,279.26358017)(912.17860444,280.93824683)(912.17860444,282.9649135)
\curveto(912.17860444,284.99158017)(912.51993777,286.68758017)(913.20260444,288.0529135)
\curveto(913.90660444,289.41824683)(914.8772711,290.44224683)(916.11460444,291.1249135)
\curveto(917.35193777,291.80758017)(918.79193777,292.1489135)(920.43460444,292.1489135)
\closepath
\moveto(920.46660444,288.7569135)
\curveto(919.52793777,288.7569135)(918.75993777,288.45824683)(918.16260444,287.8609135)
\curveto(917.5652711,287.26358017)(917.2132711,286.33558017)(917.10660444,285.0769135)
\lineto(923.79460444,285.0769135)
\curveto(923.7732711,286.12224683)(923.4852711,286.9969135)(922.93060444,287.7009135)
\curveto(922.3972711,288.4049135)(921.57593777,288.7569135)(920.46660444,288.7569135)
\closepath
}
}
{
\newrgbcolor{curcolor}{0 0 0}
\pscustom[linestyle=none,fillstyle=solid,fillcolor=curcolor]
{
\newpath
\moveto(947.79459174,274.3569135)
\lineto(943.02659174,274.3569135)
\lineto(943.02659174,288.2449135)
\lineto(938.64259174,288.2449135)
\curveto(938.36525841,284.83158017)(937.99192508,282.07958017)(937.52259174,279.9889135)
\curveto(937.07459174,277.91958017)(936.43459174,276.4049135)(935.60259174,275.4449135)
\curveto(934.79192508,274.50624683)(933.71459174,274.0369135)(932.37059174,274.0369135)
\curveto(931.26125841,274.0369135)(930.35459174,274.20758017)(929.65059174,274.5489135)
\lineto(929.65059174,278.3569135)
\curveto(930.14125841,278.14358017)(930.65325841,278.0369135)(931.18659174,278.0369135)
\curveto(931.57059174,278.0369135)(931.92259174,278.2289135)(932.24259174,278.6129135)
\curveto(932.56259174,278.9969135)(932.86125841,279.69024683)(933.13859174,280.6929135)
\curveto(933.43725841,281.69558017)(933.70392508,283.0929135)(933.93859174,284.8849135)
\curveto(934.17325841,286.69824683)(934.38659174,289.0129135)(934.57859174,291.8289135)
\lineto(947.79459174,291.8289135)
\closepath
}
}
{
\newrgbcolor{curcolor}{0 0 0}
\pscustom[linestyle=none,fillstyle=solid,fillcolor=curcolor]
{
\newpath
\moveto(957.55460639,285.0769135)
\lineto(960.91460639,285.0769135)
\curveto(963.60260639,285.0769135)(965.58660639,284.65024683)(966.86660639,283.7969135)
\curveto(968.16793972,282.94358017)(968.81860639,281.6529135)(968.81860639,279.9249135)
\curveto(968.81860639,278.21824683)(968.22127306,276.86358017)(967.02660639,275.8609135)
\curveto(965.83193972,274.85824683)(963.85860639,274.3569135)(961.10660639,274.3569135)
\lineto(952.78660639,274.3569135)
\lineto(952.78660639,291.8289135)
\lineto(957.55460639,291.8289135)
\closepath
\moveto(964.05060639,279.8609135)
\curveto(964.05060639,281.1409135)(962.97327306,281.7809135)(960.81860639,281.7809135)
\lineto(957.55460639,281.7809135)
\lineto(957.55460639,277.6529135)
\lineto(960.88260639,277.6529135)
\curveto(961.79993972,277.6529135)(962.55727306,277.8129135)(963.15460639,278.1329135)
\curveto(963.75193972,278.47424683)(964.05060639,279.05024683)(964.05060639,279.8609135)
\closepath
}
}
{
\newrgbcolor{curcolor}{0 0 0}
\pscustom[linestyle=none,fillstyle=solid,fillcolor=curcolor]
{
\newpath
\moveto(979.50661665,283.1249135)
\curveto(979.50661665,280.54358017)(979.12261665,278.05824683)(978.35461665,275.6689135)
\curveto(977.60794998,273.3009135)(976.42394998,271.17824683)(974.80261665,269.3009135)
\lineto(970.93061665,269.3009135)
\curveto(972.35994998,271.26358017)(973.44794998,273.43958017)(974.19461665,275.8289135)
\curveto(974.96261665,278.23958017)(975.34661665,280.68224683)(975.34661665,283.1569135)
\curveto(975.34661665,285.69558017)(974.96261665,288.17024683)(974.19461665,290.5809135)
\curveto(973.44794998,292.99158017)(972.34928331,295.19958017)(970.89861665,297.2049135)
\lineto(974.80261665,297.2049135)
\curveto(976.42394998,295.26358017)(977.60794998,293.0769135)(978.35461665,290.6449135)
\curveto(979.12261665,288.23424683)(979.50661665,285.72758017)(979.50661665,283.1249135)
\closepath
}
}
\end{pspicture}
}
		\caption{Схема клиент-серверного приложения}
		\label{g6_ink1}
	\end{center}
\end{figure}

Первоочередная задача — создание инструкций передачи данных
апплета \emph{GeoGebra} как на сервер, так и из него. В базовом API апплета
предусмотрено несколько вариантов сохранения/загрузки данных,
в том числе с помощью \emph{json}-файлов и \emph{XML}-файлов.
Опытным путем был выбран вариант работы с \emph{XML}-файлами
как основной и с \emph{json}-файлами как вспомогательный. Соответственно была написана
функция передачи/приема этих файлов с помощью сокетов [Исходный код \ref{sc12_s_skt_xml}] [Исходный код \ref{sc32_k_js_xml}].

Рассмотрим работу инструкций по этапам на примере:

1. Пользователь, владеющий правами на редактирование доски,
добавляет точку в координатную плоскость апплета.

2. В API апплета есть обработчики событий, которые реагируют
на изменение состояния апплета, в нашем случае была добавлена точка,
и вызывают функцию сохранения/передачи.

3. Функция сохранения/передачи копирует \emph{XML}-код
апплета и, если есть возможность, редактирует его.
После этого через сокет отправляет этот код на сервер вместе с
номером комнаты.

4. Сервер принимает данные из сокета, т.е. \emph{XML}-код и номер
комнаты. Далее по номеру комнаты ищет всех ее участников и отправляет
им через сокет полученный \emph{XML}-код.

5. Участники через сокет получают \emph{XML}-код от сервера.
Вызывается функция загрузки \emph{XML}-кода в апплет.

Причем перед отправкой любых данных через веб-сокеты, сервер проверяет
id пользователя, его принадлежность к комнате, и наличие у него прав
владельца комнаты (для комнаты с ограниченными правами). Тем самым
отсеивая любую возможность взаимодействия участников одной комнаты с другой.

Таким образом происходит синхронизация \emph{XML}-кода апплета
пользователя с правами на редактирование доски с другими участниками
комнаты.

Стоит отметить, что был проведен анализ, какие элементы в
\emph{XML}-коде постоянны, а какие изменяются. Исходя
из этого, для уменьшения нагрузки на сеть, \emph{XML}-код
перед отправкой обрезается и в нем остаются только изменяющиеся
элементы.

Координаты плоскости апплета так же входят в \emph{XML}-код,
соответственно функция трансляции координат отключается путем той самой обрезки
части \emph{XML}-кода, если данная функция включена, то эта часть с 
координатами остается и передается вместе с остальным \emph{XML}-кодом.

Следующая задача - инструкции для передачи команд управления правами
и работы чата [Исходный код \ref{sc31_k_js_permission}]. Это достаточно тривиальная задача, т.к. в ней требуется
передавать через веб-сокеты малый объем информации.

Инструкции для передачи команд управления правами работают просто.
Рассмотрим пример. Пользователь запрашивает права у владельца, нажимая
соответствующую кнопку. Обработчик клика этой кнопки вызывает
функцию, которая через сокет передает номер комнаты, специальный
id участника и команду для запроса прав. Сервер принимает данные
сокета, по номеру комнаты ищет владельца и отправляет через сокет
id участника, запросившего права. Владелец получает уведомление
и принимает решение: передать права или нет. Это решение (true или false)
отправляется на сервер через сокет вместе с номером комнаты и id участника.
Сервер рассматривает решение владельца, и в соответствии с ним
либо отдает права участнику, либо нет, отправляя ему соответствующее
уведомление через сокет. В свободной системе прав все происходит
намного проще, но общая схема остается такой же.

В комнате с ограниченной системой прав, если участник отказывается
от права редактирования доски (повторно нажимая кнопку, которой он запрашивал
права), то права на редактирование немедленно возвращаются владельцу
комнаты.

В комнате со свободной системой прав, если участник отказывается от прав
на редактирование доски, то права никому не передаются, пока их не
запросит другой пользователь.

Инструкции работы чата оказались самой простой задачей [Исходный код \ref{sc34_k_js_messages}], т.к. в документации
\emph{socket.io} много информации об этом.
Снова рассмотрим пример. В случае чата нет никакой разницы,
какая система прав используется. Участник пишет сообщение, нажимает
кнопку, на которой размещен обработчик клика. Обработчик вызывает
функцию, которая отправляет через сокет сообщение пользователя, 
номер комнаты и имя пользователя на сервер. Сервер принимает все данные,
по номеру комнаты ищет всех участников и через сокет отправляет
это сообщение, соединенное с именем пользователя, всем участникам комнаты.
Участники принимают это сообщение, оно оборачивается в элемент списка
и помещается в контейнер чата.

Аналогично чату работает и голосовой канал [Исходный код \ref{sc37_k_js_voice}],
различие заключается только в том, что передача осуществляется чанками и непрерывно.

Так же, через систему сокетов, связана и трансляция курсора мыши, только передаются уже
координаты. Однако, в чистом виде, как они есть, передавать координаты
нельзя, т.к. в этом случае существует привязка к разрешению экрана пользователя.
Поэтому используются методы преобразования координат. Когда другой пользователь получает такие данные,
то происходит обратное преобразование, но с учетом его разрешения.

Помимо этого, для оптимизации, каждый сокет был связан с таймером,
который ограничивает количество тиков передачи данных, тем
самым уменьшая нагрузку как на сеть, так и на ПК пользователя.
