Следующая задача: создать схему интерфейса [Рис.~\ref{g5_ink1}] [Исходный код \ref{sc18_k_hb_room}].

\begin{figure}[H]
	\begin{center}
		\scalebox{0.6}{%LaTeX with PSTricks extensions
%%Creator: Inkscape 1.2 (dc2aedaf03, 2022-05-15)
%%Please note this file requires PSTricks extensions
\psset{xunit=.5pt,yunit=.5pt,runit=.5pt}
\begin{pspicture}(1024,768)
{
\newrgbcolor{curcolor}{0.50196081 0.50196081 0.50196081}
\pscustom[linestyle=none,fillstyle=solid,fillcolor=curcolor]
{
\newpath
\moveto(600.12784576,374.23065186)
\curveto(600.12784576,326.38734633)(561.34318346,287.60268402)(513.49987793,287.60268402)
\curveto(465.6565724,287.60268402)(426.8719101,326.38734633)(426.8719101,374.23065186)
\curveto(426.8719101,422.07395738)(465.6565724,460.85861969)(513.49987793,460.85861969)
\curveto(561.34318346,460.85861969)(600.12784576,422.07395738)(600.12784576,374.23065186)
\closepath
}
}
{
\newrgbcolor{curcolor}{0.50196081 0.50196081 0.50196081}
\pscustom[linestyle=none,fillstyle=solid,fillcolor=curcolor]
{
\newpath
\moveto(67.8839035,734.56464386)
\lineto(320.16886139,734.56464386)
\lineto(320.16886139,570.42743683)
\lineto(67.8839035,570.42743683)
\closepath
}
}
{
\newrgbcolor{curcolor}{0.50196081 0.50196081 0.50196081}
\pscustom[linestyle=none,fillstyle=solid,fillcolor=curcolor]
{
\newpath
\moveto(39.00791931,469.10818481)
\lineto(291.2928772,469.10818481)
\lineto(291.2928772,304.97097778)
\lineto(39.00791931,304.97097778)
\closepath
}
}
{
\newrgbcolor{curcolor}{0.50196081 0.50196081 0.50196081}
\pscustom[linestyle=none,fillstyle=solid,fillcolor=curcolor]
{
\newpath
\moveto(95.7467041,209.73083496)
\lineto(348.03166199,209.73083496)
\lineto(348.03166199,45.59362793)
\lineto(95.7467041,45.59362793)
\closepath
}
}
{
\newrgbcolor{curcolor}{0.50196081 0.50196081 0.50196081}
\pscustom[linestyle=none,fillstyle=solid,fillcolor=curcolor]
{
\newpath
\moveto(444.28497314,186.42749023)
\lineto(696.56993103,186.42749023)
\lineto(696.56993103,22.2902832)
\lineto(444.28497314,22.2902832)
\closepath
}
}
{
\newrgbcolor{curcolor}{0.50196081 0.50196081 0.50196081}
\pscustom[linestyle=none,fillstyle=solid,fillcolor=curcolor]
{
\newpath
\moveto(429.08706665,746.7229557)
\lineto(681.37202454,746.7229557)
\lineto(681.37202454,582.58574867)
\lineto(429.08706665,582.58574867)
\closepath
}
}
{
\newrgbcolor{curcolor}{0.50196081 0.50196081 0.50196081}
\pscustom[linestyle=none,fillstyle=solid,fillcolor=curcolor]
{
\newpath
\moveto(735.07122803,597.78366089)
\lineto(987.35618591,597.78366089)
\lineto(987.35618591,433.64645386)
\lineto(735.07122803,433.64645386)
\closepath
}
}
{
\newrgbcolor{curcolor}{0.50196081 0.50196081 0.50196081}
\pscustom[linestyle=none,fillstyle=solid,fillcolor=curcolor]
{
\newpath
\moveto(754.32189941,322.19525146)
\lineto(1006.6068573,322.19525146)
\lineto(1006.6068573,158.05804443)
\lineto(754.32189941,158.05804443)
\closepath
}
}
{
\newrgbcolor{curcolor}{0.50196081 0.50196081 0.50196081}
\pscustom[linestyle=none,fillstyle=solid,fillcolor=curcolor]
{
\newpath
\moveto(313.79867554,594.62263489)
\lineto(394.03943634,594.62263489)
\lineto(394.03943634,583.87610435)
\lineto(313.79867554,583.87610435)
\closepath
}
}
{
\newrgbcolor{curcolor}{0.50196081 0.50196081 0.50196081}
\pscustom[linestyle=none,fillstyle=solid,fillcolor=curcolor]
{
\newpath
\moveto(382.67831421,444.97894287)
\lineto(470.97897339,444.97894287)
\lineto(470.97897339,434.23241234)
\lineto(382.67831421,434.23241234)
\closepath
}
}
{
\newrgbcolor{curcolor}{0.50196081 0.50196081 0.50196081}
\pscustom[linestyle=none,fillstyle=solid,fillcolor=curcolor]
{
\newpath
\moveto(285.64385986,387.85348511)
\lineto(446.89439392,387.85348511)
\lineto(446.89439392,377.10695457)
\lineto(285.64385986,377.10695457)
\closepath
}
}
{
\newrgbcolor{curcolor}{0.50196081 0.50196081 0.50196081}
\pscustom[linestyle=none,fillstyle=solid,fillcolor=curcolor]
{
\newpath
\moveto(330.60947553,205.08183262)
\lineto(330.39066842,326.80749602)
\lineto(341.13716587,326.84420616)
\lineto(341.35597297,205.11854276)
\closepath
}
}
{
\newrgbcolor{curcolor}{0.50196081 0.50196081 0.50196081}
\pscustom[linestyle=none,fillstyle=solid,fillcolor=curcolor]
{
\newpath
\moveto(330.92089844,326.840271)
\lineto(461.90229797,326.840271)
\lineto(461.90229797,316.09374046)
\lineto(330.92089844,316.09374046)
\closepath
}
}
{
\newrgbcolor{curcolor}{0.50196081 0.50196081 0.50196081}
\pscustom[linestyle=none,fillstyle=solid,fillcolor=curcolor]
{
\newpath
\moveto(382.57647705,594.00695801)
\lineto(394.03944302,594.00695801)
\lineto(394.03944302,434.24186707)
\lineto(382.57647705,434.24186707)
\closepath
}
}
{
\newrgbcolor{curcolor}{0.50196081 0.50196081 0.50196081}
\pscustom[linestyle=none,fillstyle=solid,fillcolor=curcolor]
{
\newpath
\moveto(514.24965441,177.18478123)
\lineto(513.8351335,296.27359812)
\lineto(524.58154595,296.34163792)
\lineto(524.99606686,177.25282103)
\closepath
}
}
{
\newrgbcolor{curcolor}{0.50196081 0.50196081 0.50196081}
\pscustom[linestyle=none,fillstyle=solid,fillcolor=curcolor]
{
\newpath
\moveto(673.71789551,264.7041626)
\lineto(769.1829071,264.7041626)
\lineto(769.1829071,253.95763206)
\lineto(673.71789551,253.95763206)
\closepath
}
}
{
\newrgbcolor{curcolor}{0.50196081 0.50196081 0.50196081}
\pscustom[linestyle=none,fillstyle=solid,fillcolor=curcolor]
{
\newpath
\moveto(683.04614003,342.22914301)
\lineto(682.43288177,253.93061343)
\lineto(671.68661041,254.00524932)
\lineto(672.29986868,342.3037789)
\closepath
}
}
{
\newrgbcolor{curcolor}{0.50196081 0.50196081 0.50196081}
\pscustom[linestyle=none,fillstyle=solid,fillcolor=curcolor]
{
\newpath
\moveto(559.59301758,342.27557373)
\lineto(682.81990051,342.27557373)
\lineto(682.81990051,331.5290432)
\lineto(559.59301758,331.5290432)
\closepath
}
}
{
\newrgbcolor{curcolor}{0.50196081 0.50196081 0.50196081}
\pscustom[linestyle=none,fillstyle=solid,fillcolor=curcolor]
{
\newpath
\moveto(527.68161008,589.24600695)
\lineto(527.66652697,451.80652469)
\lineto(516.91999687,451.8093819)
\lineto(516.93507998,589.24886417)
\closepath
}
}
{
\newrgbcolor{curcolor}{0.50196081 0.50196081 0.50196081}
\pscustom[linestyle=none,fillstyle=solid,fillcolor=curcolor]
{
\newpath
\moveto(659.51196289,500.82443237)
\lineto(747.81262207,500.82443237)
\lineto(747.81262207,490.07790184)
\lineto(659.51196289,490.07790184)
\closepath
}
}
{
\newrgbcolor{curcolor}{0.50196081 0.50196081 0.50196081}
\pscustom[linestyle=none,fillstyle=solid,fillcolor=curcolor]
{
\newpath
\moveto(659.94961687,420.78382899)
\lineto(659.62589085,500.13245011)
\lineto(670.37234872,500.16785472)
\lineto(670.69607474,420.8192336)
\closepath
}
}
{
\newrgbcolor{curcolor}{0.50196081 0.50196081 0.50196081}
\pscustom[linestyle=none,fillstyle=solid,fillcolor=curcolor]
{
\newpath
\moveto(576.84246826,431.52166748)
\lineto(665.14312744,431.52166748)
\lineto(665.14312744,420.77513695)
\lineto(576.84246826,420.77513695)
\closepath
}
}
{
\newrgbcolor{curcolor}{0 0 0}
\pscustom[linestyle=none,fillstyle=solid,fillcolor=curcolor]
{
\newpath
\moveto(131.86485746,685.78957373)
\lineto(131.86485746,666.97357373)
\lineto(134.90485746,666.97357373)
\lineto(134.90485746,656.28557373)
\lineto(130.23285746,656.28557373)
\lineto(130.23285746,662.94157373)
\lineto(115.92885746,662.94157373)
\lineto(115.92885746,656.28557373)
\lineto(111.25685746,656.28557373)
\lineto(111.25685746,666.97357373)
\lineto(113.01685746,666.97357373)
\curveto(113.82752413,668.55224039)(114.58485746,670.25890706)(115.28885746,672.09357373)
\curveto(115.99285746,673.92824039)(116.62219079,675.96557373)(117.17685746,678.20557373)
\curveto(117.73152413,680.44557373)(118.19019079,682.97357373)(118.55285746,685.78957373)
\closepath
\moveto(127.03285746,681.75757373)
\lineto(122.39285746,681.75757373)
\curveto(122.20085746,680.26424039)(121.89152413,678.66424039)(121.46485746,676.95757373)
\curveto(121.05952413,675.25090706)(120.56885746,673.53357373)(119.99285746,671.80557373)
\curveto(119.41685746,670.09890706)(118.78752413,668.48824039)(118.10485746,666.97357373)
\lineto(127.03285746,666.97357373)
\closepath
}
}
{
\newrgbcolor{curcolor}{0 0 0}
\pscustom[linestyle=none,fillstyle=solid,fillcolor=curcolor]
{
\newpath
\moveto(145.20885551,680.76557373)
\curveto(147.55552217,680.76557373)(149.34752217,680.25357373)(150.58485551,679.22957373)
\curveto(151.84352217,678.22690706)(152.47285551,676.68024039)(152.47285551,674.58957373)
\lineto(152.47285551,662.94157373)
\lineto(149.14485551,662.94157373)
\lineto(148.21685551,665.30957373)
\lineto(148.08885551,665.30957373)
\curveto(147.34218884,664.37090706)(146.55285551,663.68824039)(145.72085551,663.26157373)
\curveto(144.88885551,662.83490706)(143.74752217,662.62157373)(142.29685551,662.62157373)
\curveto(140.73952217,662.62157373)(139.44885551,663.06957373)(138.42485551,663.96557373)
\curveto(137.40085551,664.86157373)(136.88885551,666.25890706)(136.88885551,668.15757373)
\curveto(136.88885551,670.01357373)(137.53952217,671.37890706)(138.84085551,672.25357373)
\curveto(140.14218884,673.12824039)(142.09418884,673.61890706)(144.69685551,673.72557373)
\lineto(147.73685551,673.82157373)
\lineto(147.73685551,674.58957373)
\curveto(147.73685551,675.50690706)(147.49152217,676.17890706)(147.00085551,676.60557373)
\curveto(146.53152217,677.03224039)(145.87018884,677.24557373)(145.01685551,677.24557373)
\curveto(144.16352217,677.24557373)(143.33152217,677.11757373)(142.52085551,676.86157373)
\curveto(141.71018884,676.62690706)(140.89952217,676.32824039)(140.08885551,675.96557373)
\lineto(138.52085551,679.19757373)
\curveto(139.43818884,679.66690706)(140.47285551,680.04024039)(141.62485551,680.31757373)
\curveto(142.77685551,680.61624039)(143.97152217,680.76557373)(145.20885551,680.76557373)
\closepath
\moveto(147.73685551,671.03757373)
\lineto(145.88085551,670.97357373)
\curveto(144.34485551,670.93090706)(143.27818884,670.65357373)(142.68085551,670.14157373)
\curveto(142.08352217,669.62957373)(141.78485551,668.95757373)(141.78485551,668.12557373)
\curveto(141.78485551,667.40024039)(141.99818884,666.87757373)(142.42485551,666.55757373)
\curveto(142.85152217,666.25890706)(143.40618884,666.10957373)(144.08885551,666.10957373)
\curveto(145.11285551,666.10957373)(145.97685551,666.40824039)(146.68085551,667.00557373)
\curveto(147.38485551,667.62424039)(147.73685551,668.48824039)(147.73685551,669.59757373)
\closepath
}
}
{
\newrgbcolor{curcolor}{0 0 0}
\pscustom[linestyle=none,fillstyle=solid,fillcolor=curcolor]
{
\newpath
\moveto(162.13685844,680.41357373)
\lineto(162.13685844,673.69357373)
\lineto(168.79285844,673.69357373)
\lineto(168.79285844,680.41357373)
\lineto(173.56085844,680.41357373)
\lineto(173.56085844,662.94157373)
\lineto(168.79285844,662.94157373)
\lineto(168.79285844,670.14157373)
\lineto(162.13685844,670.14157373)
\lineto(162.13685844,662.94157373)
\lineto(157.36885844,662.94157373)
\lineto(157.36885844,680.41357373)
\closepath
}
}
{
\newrgbcolor{curcolor}{0 0 0}
\pscustom[linestyle=none,fillstyle=solid,fillcolor=curcolor]
{
\newpath
\moveto(183.32087943,680.41357373)
\lineto(183.32087943,673.69357373)
\lineto(189.97687943,673.69357373)
\lineto(189.97687943,680.41357373)
\lineto(194.74487943,680.41357373)
\lineto(194.74487943,662.94157373)
\lineto(189.97687943,662.94157373)
\lineto(189.97687943,670.14157373)
\lineto(183.32087943,670.14157373)
\lineto(183.32087943,662.94157373)
\lineto(178.55287943,662.94157373)
\lineto(178.55287943,680.41357373)
\closepath
}
}
{
\newrgbcolor{curcolor}{0 0 0}
\pscustom[linestyle=none,fillstyle=solid,fillcolor=curcolor]
{
\newpath
\moveto(199.73690043,662.94157373)
\lineto(199.73690043,680.41357373)
\lineto(204.50490043,680.41357373)
\lineto(204.50490043,673.66157373)
\lineto(206.80890043,673.66157373)
\curveto(209.4755671,673.66157373)(211.44890043,673.23490706)(212.72890043,672.38157373)
\curveto(214.00890043,671.52824039)(214.64890043,670.23757373)(214.64890043,668.50957373)
\curveto(214.64890043,666.80290706)(214.0515671,665.44824039)(212.85690043,664.44557373)
\curveto(211.66223376,663.44290706)(209.6995671,662.94157373)(206.96890043,662.94157373)
\closepath
\moveto(217.17690043,662.94157373)
\lineto(217.17690043,680.41357373)
\lineto(221.94490043,680.41357373)
\lineto(221.94490043,662.94157373)
\closepath
\moveto(204.50490043,666.23757373)
\lineto(206.71290043,666.23757373)
\curveto(207.6515671,666.23757373)(208.40890043,666.39757373)(208.98490043,666.71757373)
\curveto(209.58223376,667.05890706)(209.88090043,667.63490706)(209.88090043,668.44557373)
\curveto(209.88090043,669.72557373)(208.8035671,670.36557373)(206.64890043,670.36557373)
\lineto(204.50490043,670.36557373)
\closepath
}
}
{
\newrgbcolor{curcolor}{0 0 0}
\pscustom[linestyle=none,fillstyle=solid,fillcolor=curcolor]
{
\newpath
\moveto(234.13691264,680.73357373)
\curveto(236.5475793,680.73357373)(238.45691264,680.04024039)(239.86491264,678.65357373)
\curveto(241.27291264,677.28824039)(241.97691264,675.33624039)(241.97691264,672.79757373)
\lineto(241.97691264,670.49357373)
\lineto(230.71291264,670.49357373)
\curveto(230.7555793,669.14957373)(231.15024597,668.09357373)(231.89691264,667.32557373)
\curveto(232.66491264,666.55757373)(233.72091264,666.17357373)(235.06491264,666.17357373)
\curveto(236.17424597,666.17357373)(237.1875793,666.28024039)(238.10491264,666.49357373)
\curveto(239.0435793,666.72824039)(240.0035793,667.08024039)(240.98491264,667.54957373)
\lineto(240.98491264,663.86957373)
\curveto(240.11024597,663.44290706)(239.2035793,663.13357373)(238.26491264,662.94157373)
\curveto(237.32624597,662.72824039)(236.18491264,662.62157373)(234.84091264,662.62157373)
\curveto(233.0915793,662.62157373)(231.54491264,662.94157373)(230.20091264,663.58157373)
\curveto(228.85691264,664.24290706)(227.80091264,665.22424039)(227.03291264,666.52557373)
\curveto(226.26491264,667.84824039)(225.88091264,669.52290706)(225.88091264,671.54957373)
\curveto(225.88091264,673.57624039)(226.22224597,675.27224039)(226.90491264,676.63757373)
\curveto(227.60891264,678.00290706)(228.5795793,679.02690706)(229.81691264,679.70957373)
\curveto(231.05424597,680.39224039)(232.49424597,680.73357373)(234.13691264,680.73357373)
\closepath
\moveto(234.16891264,677.34157373)
\curveto(233.23024597,677.34157373)(232.46224597,677.04290706)(231.86491264,676.44557373)
\curveto(231.2675793,675.84824039)(230.9155793,674.92024039)(230.80891264,673.66157373)
\lineto(237.49691264,673.66157373)
\curveto(237.4755793,674.70690706)(237.1875793,675.58157373)(236.63291264,676.28557373)
\curveto(236.0995793,676.98957373)(235.27824597,677.34157373)(234.16891264,677.34157373)
\closepath
}
}
{
\newrgbcolor{curcolor}{0 0 0}
\pscustom[linestyle=none,fillstyle=solid,fillcolor=curcolor]
{
\newpath
\moveto(270.04090727,671.70957373)
\curveto(270.04090727,668.80824039)(269.27290727,666.56824039)(267.73690727,664.98957373)
\curveto(266.2222406,663.41090706)(264.15290727,662.62157373)(261.52890727,662.62157373)
\curveto(259.90757393,662.62157373)(258.45690727,662.97357373)(257.17690727,663.67757373)
\curveto(255.9182406,664.38157373)(254.9262406,665.40557373)(254.20090727,666.74957373)
\curveto(253.47557393,668.11490706)(253.11290727,669.76824039)(253.11290727,671.70957373)
\curveto(253.11290727,674.61090706)(253.8702406,676.84024039)(255.38490727,678.39757373)
\curveto(256.89957393,679.95490706)(258.97957393,680.73357373)(261.62490727,680.73357373)
\curveto(263.26757393,680.73357373)(264.7182406,680.38157373)(265.97690727,679.67757373)
\curveto(267.23557393,678.97357373)(268.22757393,677.94957373)(268.95290727,676.60557373)
\curveto(269.6782406,675.26157373)(270.04090727,673.62957373)(270.04090727,671.70957373)
\closepath
\moveto(257.97690727,671.70957373)
\curveto(257.97690727,669.98157373)(258.2542406,668.66957373)(258.80890727,667.77357373)
\curveto(259.38490727,666.89890706)(260.31290727,666.46157373)(261.59290727,666.46157373)
\curveto(262.85157393,666.46157373)(263.7582406,666.89890706)(264.31290727,667.77357373)
\curveto(264.88890727,668.66957373)(265.17690727,669.98157373)(265.17690727,671.70957373)
\curveto(265.17690727,673.43757373)(264.88890727,674.72824039)(264.31290727,675.58157373)
\curveto(263.7582406,676.45624039)(262.84090727,676.89357373)(261.56090727,676.89357373)
\curveto(260.3022406,676.89357373)(259.38490727,676.45624039)(258.80890727,675.58157373)
\curveto(258.2542406,674.72824039)(257.97690727,673.43757373)(257.97690727,671.70957373)
\closepath
}
}
{
\newrgbcolor{curcolor}{0 0 0}
\pscustom[linestyle=none,fillstyle=solid,fillcolor=curcolor]
{
\newpath
\moveto(133.84890629,640.41357373)
\lineto(139.09690629,640.41357373)
\lineto(132.18490629,632.02957373)
\lineto(139.70490629,622.94157373)
\lineto(134.29690629,622.94157373)
\lineto(127.16090629,631.80557373)
\lineto(127.16090629,622.94157373)
\lineto(122.39290629,622.94157373)
\lineto(122.39290629,640.41357373)
\lineto(127.16090629,640.41357373)
\lineto(127.16090629,631.93357373)
\closepath
}
}
{
\newrgbcolor{curcolor}{0 0 0}
\pscustom[linestyle=none,fillstyle=solid,fillcolor=curcolor]
{
\newpath
\moveto(157.43287504,631.70957373)
\curveto(157.43287504,628.80824039)(156.66487504,626.56824039)(155.12887504,624.98957373)
\curveto(153.61420837,623.41090706)(151.54487504,622.62157373)(148.92087504,622.62157373)
\curveto(147.29954171,622.62157373)(145.84887504,622.97357373)(144.56887504,623.67757373)
\curveto(143.31020837,624.38157373)(142.31820837,625.40557373)(141.59287504,626.74957373)
\curveto(140.86754171,628.11490706)(140.50487504,629.76824039)(140.50487504,631.70957373)
\curveto(140.50487504,634.61090706)(141.26220837,636.84024039)(142.77687504,638.39757373)
\curveto(144.29154171,639.95490706)(146.37154171,640.73357373)(149.01687504,640.73357373)
\curveto(150.65954171,640.73357373)(152.11020837,640.38157373)(153.36887504,639.67757373)
\curveto(154.62754171,638.97357373)(155.61954171,637.94957373)(156.34487504,636.60557373)
\curveto(157.07020837,635.26157373)(157.43287504,633.62957373)(157.43287504,631.70957373)
\closepath
\moveto(145.36887504,631.70957373)
\curveto(145.36887504,629.98157373)(145.64620837,628.66957373)(146.20087504,627.77357373)
\curveto(146.77687504,626.89890706)(147.70487504,626.46157373)(148.98487504,626.46157373)
\curveto(150.24354171,626.46157373)(151.15020837,626.89890706)(151.70487504,627.77357373)
\curveto(152.28087504,628.66957373)(152.56887504,629.98157373)(152.56887504,631.70957373)
\curveto(152.56887504,633.43757373)(152.28087504,634.72824039)(151.70487504,635.58157373)
\curveto(151.15020837,636.45624039)(150.23287504,636.89357373)(148.95287504,636.89357373)
\curveto(147.69420837,636.89357373)(146.77687504,636.45624039)(146.20087504,635.58157373)
\curveto(145.64620837,634.72824039)(145.36887504,633.43757373)(145.36887504,631.70957373)
\closepath
}
}
{
\newrgbcolor{curcolor}{0 0 0}
\pscustom[linestyle=none,fillstyle=solid,fillcolor=curcolor]
{
\newpath
\moveto(183.38485844,640.41357373)
\lineto(183.38485844,622.94157373)
\lineto(178.93685844,622.94157373)
\lineto(178.93685844,631.51757373)
\curveto(178.93685844,632.37090706)(178.9475251,633.20290706)(178.96885844,634.01357373)
\curveto(179.0115251,634.82424039)(179.06485844,635.57090706)(179.12885844,636.25357373)
\lineto(179.03285844,636.25357373)
\lineto(174.20085844,622.94157373)
\lineto(170.61685844,622.94157373)
\lineto(165.72085844,636.28557373)
\lineto(165.59285844,636.28557373)
\curveto(165.67819177,635.58157373)(165.7315251,634.82424039)(165.75285844,634.01357373)
\curveto(165.7955251,633.22424039)(165.81685844,632.34957373)(165.81685844,631.38957373)
\lineto(165.81685844,622.94157373)
\lineto(161.36885844,622.94157373)
\lineto(161.36885844,640.41357373)
\lineto(168.12085844,640.41357373)
\lineto(172.47285844,628.57357373)
\lineto(176.88885844,640.41357373)
\closepath
}
}
{
\newrgbcolor{curcolor}{0 0 0}
\pscustom[linestyle=none,fillstyle=solid,fillcolor=curcolor]
{
\newpath
\moveto(193.14485404,640.41357373)
\lineto(193.14485404,633.69357373)
\lineto(199.80085404,633.69357373)
\lineto(199.80085404,640.41357373)
\lineto(204.56885404,640.41357373)
\lineto(204.56885404,622.94157373)
\lineto(199.80085404,622.94157373)
\lineto(199.80085404,630.14157373)
\lineto(193.14485404,630.14157373)
\lineto(193.14485404,622.94157373)
\lineto(188.37685404,622.94157373)
\lineto(188.37685404,640.41357373)
\closepath
}
}
{
\newrgbcolor{curcolor}{0 0 0}
\pscustom[linestyle=none,fillstyle=solid,fillcolor=curcolor]
{
\newpath
\moveto(216.72887504,640.76557373)
\curveto(219.07554171,640.76557373)(220.86754171,640.25357373)(222.10487504,639.22957373)
\curveto(223.36354171,638.22690706)(223.99287504,636.68024039)(223.99287504,634.58957373)
\lineto(223.99287504,622.94157373)
\lineto(220.66487504,622.94157373)
\lineto(219.73687504,625.30957373)
\lineto(219.60887504,625.30957373)
\curveto(218.86220837,624.37090706)(218.07287504,623.68824039)(217.24087504,623.26157373)
\curveto(216.40887504,622.83490706)(215.26754171,622.62157373)(213.81687504,622.62157373)
\curveto(212.25954171,622.62157373)(210.96887504,623.06957373)(209.94487504,623.96557373)
\curveto(208.92087504,624.86157373)(208.40887504,626.25890706)(208.40887504,628.15757373)
\curveto(208.40887504,630.01357373)(209.05954171,631.37890706)(210.36087504,632.25357373)
\curveto(211.66220837,633.12824039)(213.61420837,633.61890706)(216.21687504,633.72557373)
\lineto(219.25687504,633.82157373)
\lineto(219.25687504,634.58957373)
\curveto(219.25687504,635.50690706)(219.01154171,636.17890706)(218.52087504,636.60557373)
\curveto(218.05154171,637.03224039)(217.39020837,637.24557373)(216.53687504,637.24557373)
\curveto(215.68354171,637.24557373)(214.85154171,637.11757373)(214.04087504,636.86157373)
\curveto(213.23020837,636.62690706)(212.41954171,636.32824039)(211.60887504,635.96557373)
\lineto(210.04087504,639.19757373)
\curveto(210.95820837,639.66690706)(211.99287504,640.04024039)(213.14487504,640.31757373)
\curveto(214.29687504,640.61624039)(215.49154171,640.76557373)(216.72887504,640.76557373)
\closepath
\moveto(219.25687504,631.03757373)
\lineto(217.40087504,630.97357373)
\curveto(215.86487504,630.93090706)(214.79820837,630.65357373)(214.20087504,630.14157373)
\curveto(213.60354171,629.62957373)(213.30487504,628.95757373)(213.30487504,628.12557373)
\curveto(213.30487504,627.40024039)(213.51820837,626.87757373)(213.94487504,626.55757373)
\curveto(214.37154171,626.25890706)(214.92620837,626.10957373)(215.60887504,626.10957373)
\curveto(216.63287504,626.10957373)(217.49687504,626.40824039)(218.20087504,627.00557373)
\curveto(218.90487504,627.62424039)(219.25687504,628.48824039)(219.25687504,629.59757373)
\closepath
}
}
{
\newrgbcolor{curcolor}{0 0 0}
\pscustom[linestyle=none,fillstyle=solid,fillcolor=curcolor]
{
\newpath
\moveto(243.35287797,636.82957373)
\lineto(237.62487797,636.82957373)
\lineto(237.62487797,622.94157373)
\lineto(232.85687797,622.94157373)
\lineto(232.85687797,636.82957373)
\lineto(227.12887797,636.82957373)
\lineto(227.12887797,640.41357373)
\lineto(243.35287797,640.41357373)
\closepath
}
}
{
\newrgbcolor{curcolor}{0 0 0}
\pscustom[linestyle=none,fillstyle=solid,fillcolor=curcolor]
{
\newpath
\moveto(253.46485453,640.73357373)
\curveto(255.8755212,640.73357373)(257.78485453,640.04024039)(259.19285453,638.65357373)
\curveto(260.60085453,637.28824039)(261.30485453,635.33624039)(261.30485453,632.79757373)
\lineto(261.30485453,630.49357373)
\lineto(250.04085453,630.49357373)
\curveto(250.0835212,629.14957373)(250.47818786,628.09357373)(251.22485453,627.32557373)
\curveto(251.99285453,626.55757373)(253.04885453,626.17357373)(254.39285453,626.17357373)
\curveto(255.50218786,626.17357373)(256.5155212,626.28024039)(257.43285453,626.49357373)
\curveto(258.3715212,626.72824039)(259.3315212,627.08024039)(260.31285453,627.54957373)
\lineto(260.31285453,623.86957373)
\curveto(259.43818786,623.44290706)(258.5315212,623.13357373)(257.59285453,622.94157373)
\curveto(256.65418786,622.72824039)(255.51285453,622.62157373)(254.16885453,622.62157373)
\curveto(252.4195212,622.62157373)(250.87285453,622.94157373)(249.52885453,623.58157373)
\curveto(248.18485453,624.24290706)(247.12885453,625.22424039)(246.36085453,626.52557373)
\curveto(245.59285453,627.84824039)(245.20885453,629.52290706)(245.20885453,631.54957373)
\curveto(245.20885453,633.57624039)(245.55018786,635.27224039)(246.23285453,636.63757373)
\curveto(246.93685453,638.00290706)(247.9075212,639.02690706)(249.14485453,639.70957373)
\curveto(250.38218786,640.39224039)(251.82218786,640.73357373)(253.46485453,640.73357373)
\closepath
\moveto(253.49685453,637.34157373)
\curveto(252.55818786,637.34157373)(251.79018786,637.04290706)(251.19285453,636.44557373)
\curveto(250.5955212,635.84824039)(250.2435212,634.92024039)(250.13685453,633.66157373)
\lineto(256.82485453,633.66157373)
\curveto(256.8035212,634.70690706)(256.5155212,635.58157373)(255.96085453,636.28557373)
\curveto(255.4275212,636.98957373)(254.60618786,637.34157373)(253.49685453,637.34157373)
\closepath
}
}
{
\newrgbcolor{curcolor}{0 0 0}
\pscustom[linestyle=none,fillstyle=solid,fillcolor=curcolor]
{
\newpath
\moveto(470.79520903,693.03228088)
\lineto(465.29120903,710.95228088)
\lineto(465.16320903,710.95228088)
\lineto(465.25920903,709.03228088)
\curveto(465.3018757,708.17894754)(465.34454236,707.26161421)(465.38720903,706.28028088)
\curveto(465.4298757,705.29894754)(465.45120903,704.42428088)(465.45120903,703.65628088)
\lineto(465.45120903,693.03228088)
\lineto(461.13120903,693.03228088)
\lineto(461.13120903,715.88028088)
\lineto(467.72320903,715.88028088)
\lineto(473.13120903,698.40828088)
\lineto(473.22720903,698.40828088)
\lineto(478.95520903,715.88028088)
\lineto(485.54720903,715.88028088)
\lineto(485.54720903,693.03228088)
\lineto(481.03520903,693.03228088)
\lineto(481.03520903,703.84828088)
\curveto(481.03520903,704.57361421)(481.0458757,705.40561421)(481.06720903,706.34428088)
\curveto(481.1098757,707.28294754)(481.1418757,708.16828088)(481.16320903,709.00028088)
\lineto(481.25920903,710.92028088)
\lineto(481.13120903,710.92028088)
\lineto(475.24320903,693.03228088)
\closepath
}
}
{
\newrgbcolor{curcolor}{0 0 0}
\pscustom[linestyle=none,fillstyle=solid,fillcolor=curcolor]
{
\newpath
\moveto(506.79523442,701.80028088)
\curveto(506.79523442,698.89894754)(506.02723442,696.65894754)(504.49123442,695.08028088)
\curveto(502.97656775,693.50161421)(500.90723442,692.71228088)(498.28323442,692.71228088)
\curveto(496.66190109,692.71228088)(495.21123442,693.06428088)(493.93123442,693.76828088)
\curveto(492.67256775,694.47228088)(491.68056775,695.49628088)(490.95523442,696.84028088)
\curveto(490.22990109,698.20561421)(489.86723442,699.85894754)(489.86723442,701.80028088)
\curveto(489.86723442,704.70161421)(490.62456775,706.93094754)(492.13923442,708.48828088)
\curveto(493.65390109,710.04561421)(495.73390109,710.82428088)(498.37923442,710.82428088)
\curveto(500.02190109,710.82428088)(501.47256775,710.47228088)(502.73123442,709.76828088)
\curveto(503.98990109,709.06428088)(504.98190109,708.04028088)(505.70723442,706.69628088)
\curveto(506.43256775,705.35228088)(506.79523442,703.72028088)(506.79523442,701.80028088)
\closepath
\moveto(494.73123442,701.80028088)
\curveto(494.73123442,700.07228088)(495.00856775,698.76028088)(495.56323442,697.86428088)
\curveto(496.13923442,696.98961421)(497.06723442,696.55228088)(498.34723442,696.55228088)
\curveto(499.60590109,696.55228088)(500.51256775,696.98961421)(501.06723442,697.86428088)
\curveto(501.64323442,698.76028088)(501.93123442,700.07228088)(501.93123442,701.80028088)
\curveto(501.93123442,703.52828088)(501.64323442,704.81894754)(501.06723442,705.67228088)
\curveto(500.51256775,706.54694754)(499.59523442,706.98428088)(498.31523442,706.98428088)
\curveto(497.05656775,706.98428088)(496.13923442,706.54694754)(495.56323442,705.67228088)
\curveto(495.00856775,704.81894754)(494.73123442,703.52828088)(494.73123442,701.80028088)
\closepath
}
}
{
\newrgbcolor{curcolor}{0 0 0}
\pscustom[linestyle=none,fillstyle=solid,fillcolor=curcolor]
{
\newpath
\moveto(526.44321782,710.50428088)
\lineto(526.44321782,696.52028088)
\lineto(529.00321782,696.52028088)
\lineto(529.00321782,686.76028088)
\lineto(524.71521782,686.76028088)
\lineto(524.71521782,693.03228088)
\lineto(512.97121782,693.03228088)
\lineto(512.97121782,686.76028088)
\lineto(508.68321782,686.76028088)
\lineto(508.68321782,696.52028088)
\lineto(510.15521782,696.52028088)
\curveto(510.92321782,697.69361421)(511.57388449,699.02694754)(512.10721782,700.52028088)
\curveto(512.64055115,702.03494754)(513.06721782,703.63494754)(513.38721782,705.32028088)
\curveto(513.70721782,707.02694754)(513.94188449,708.75494754)(514.09121782,710.50428088)
\closepath
\moveto(521.67521782,706.92028088)
\lineto(518.09121782,706.92028088)
\curveto(517.83521782,704.97894754)(517.48321782,703.13361421)(517.03521782,701.38428088)
\curveto(516.58721782,699.65628088)(515.95788449,698.03494754)(515.14721782,696.52028088)
\lineto(521.67521782,696.52028088)
\closepath
}
}
{
\newrgbcolor{curcolor}{0 0 0}
\pscustom[linestyle=none,fillstyle=solid,fillcolor=curcolor]
{
\newpath
\moveto(529.48320366,710.50428088)
\lineto(534.69920366,710.50428088)
\lineto(537.99520366,700.68028088)
\curveto(538.16587032,700.18961421)(538.29387032,699.69894754)(538.37920366,699.20828088)
\curveto(538.46453699,698.71761421)(538.52853699,698.19494754)(538.57120366,697.64028088)
\lineto(538.66720366,697.64028088)
\curveto(538.73120366,698.19494754)(538.81653699,698.71761421)(538.92320366,699.20828088)
\curveto(539.02987032,699.69894754)(539.16853699,700.18961421)(539.33920366,700.68028088)
\lineto(542.57120366,710.50428088)
\lineto(547.69120366,710.50428088)
\lineto(540.29920366,690.79228088)
\curveto(539.61653699,688.97894754)(538.64587032,687.62428088)(537.38720366,686.72828088)
\curveto(536.12853699,685.81094754)(534.66720366,685.35228088)(533.00320366,685.35228088)
\curveto(532.44853699,685.35228088)(531.97920366,685.38428088)(531.59520366,685.44828088)
\curveto(531.21120366,685.49094754)(530.86987032,685.54428088)(530.57120366,685.60828088)
\lineto(530.57120366,689.38428088)
\curveto(530.78453699,689.34161421)(531.06187032,689.29894754)(531.40320366,689.25628088)
\curveto(531.74453699,689.21361421)(532.09653699,689.19228088)(532.45920366,689.19228088)
\curveto(533.46187032,689.19228088)(534.25120366,689.50161421)(534.82720366,690.12028088)
\curveto(535.40320366,690.71761421)(535.84053699,691.44294754)(536.13920366,692.29628088)
\lineto(536.42720366,693.16028088)
\closepath
}
}
{
\newrgbcolor{curcolor}{0 0 0}
\pscustom[linestyle=none,fillstyle=solid,fillcolor=curcolor]
{
\newpath
\moveto(565.19519682,693.03228088)
\lineto(560.42719682,693.03228088)
\lineto(560.42719682,706.92028088)
\lineto(556.04319682,706.92028088)
\curveto(555.76586349,703.50694754)(555.39253016,700.75494754)(554.92319682,698.66428088)
\curveto(554.47519682,696.59494754)(553.83519682,695.08028088)(553.00319682,694.12028088)
\curveto(552.19253016,693.18161421)(551.11519682,692.71228088)(549.77119682,692.71228088)
\curveto(548.66186349,692.71228088)(547.75519682,692.88294754)(547.05119682,693.22428088)
\lineto(547.05119682,697.03228088)
\curveto(547.54186349,696.81894754)(548.05386349,696.71228088)(548.58719682,696.71228088)
\curveto(548.97119682,696.71228088)(549.32319682,696.90428088)(549.64319682,697.28828088)
\curveto(549.96319682,697.67228088)(550.26186349,698.36561421)(550.53919682,699.36828088)
\curveto(550.83786349,700.37094754)(551.10453016,701.76828088)(551.33919682,703.56028088)
\curveto(551.57386349,705.37361421)(551.78719682,707.68828088)(551.97919682,710.50428088)
\lineto(565.19519682,710.50428088)
\closepath
}
}
{
\newrgbcolor{curcolor}{0 0 0}
\pscustom[linestyle=none,fillstyle=solid,fillcolor=curcolor]
{
\newpath
\moveto(574.95521147,703.75228088)
\lineto(578.31521147,703.75228088)
\curveto(581.00321147,703.75228088)(582.98721147,703.32561421)(584.26721147,702.47228088)
\curveto(585.5685448,701.61894754)(586.21921147,700.32828088)(586.21921147,698.60028088)
\curveto(586.21921147,696.89361421)(585.62187814,695.53894754)(584.42721147,694.53628088)
\curveto(583.2325448,693.53361421)(581.25921147,693.03228088)(578.50721147,693.03228088)
\lineto(570.18721147,693.03228088)
\lineto(570.18721147,710.50428088)
\lineto(574.95521147,710.50428088)
\closepath
\moveto(581.45121147,698.53628088)
\curveto(581.45121147,699.81628088)(580.37387814,700.45628088)(578.21921147,700.45628088)
\lineto(574.95521147,700.45628088)
\lineto(574.95521147,696.32828088)
\lineto(578.28321147,696.32828088)
\curveto(579.2005448,696.32828088)(579.95787814,696.48828088)(580.55521147,696.80828088)
\curveto(581.1525448,697.14961421)(581.45121147,697.72561421)(581.45121147,698.53628088)
\closepath
}
}
{
\newrgbcolor{curcolor}{0 0 0}
\pscustom[linestyle=none,fillstyle=solid,fillcolor=curcolor]
{
\newpath
\moveto(613.86722905,710.50428088)
\lineto(613.86722905,696.52028088)
\lineto(616.42722905,696.52028088)
\lineto(616.42722905,686.76028088)
\lineto(612.13922905,686.76028088)
\lineto(612.13922905,693.03228088)
\lineto(600.39522905,693.03228088)
\lineto(600.39522905,686.76028088)
\lineto(596.10722905,686.76028088)
\lineto(596.10722905,696.52028088)
\lineto(597.57922905,696.52028088)
\curveto(598.34722905,697.69361421)(598.99789572,699.02694754)(599.53122905,700.52028088)
\curveto(600.06456238,702.03494754)(600.49122905,703.63494754)(600.81122905,705.32028088)
\curveto(601.13122905,707.02694754)(601.36589572,708.75494754)(601.51522905,710.50428088)
\closepath
\moveto(609.09922905,706.92028088)
\lineto(605.51522905,706.92028088)
\curveto(605.25922905,704.97894754)(604.90722905,703.13361421)(604.45922905,701.38428088)
\curveto(604.01122905,699.65628088)(603.38189572,698.03494754)(602.57122905,696.52028088)
\lineto(609.09922905,696.52028088)
\closepath
}
}
{
\newrgbcolor{curcolor}{0 0 0}
\pscustom[linestyle=none,fillstyle=solid,fillcolor=curcolor]
{
\newpath
\moveto(635.05121489,693.03228088)
\lineto(630.28321489,693.03228088)
\lineto(630.28321489,706.92028088)
\lineto(625.89921489,706.92028088)
\curveto(625.62188156,703.50694754)(625.24854822,700.75494754)(624.77921489,698.66428088)
\curveto(624.33121489,696.59494754)(623.69121489,695.08028088)(622.85921489,694.12028088)
\curveto(622.04854822,693.18161421)(620.97121489,692.71228088)(619.62721489,692.71228088)
\curveto(618.51788156,692.71228088)(617.61121489,692.88294754)(616.90721489,693.22428088)
\lineto(616.90721489,697.03228088)
\curveto(617.39788156,696.81894754)(617.90988156,696.71228088)(618.44321489,696.71228088)
\curveto(618.82721489,696.71228088)(619.17921489,696.90428088)(619.49921489,697.28828088)
\curveto(619.81921489,697.67228088)(620.11788156,698.36561421)(620.39521489,699.36828088)
\curveto(620.69388156,700.37094754)(620.96054822,701.76828088)(621.19521489,703.56028088)
\curveto(621.42988156,705.37361421)(621.64321489,707.68828088)(621.83521489,710.50428088)
\lineto(635.05121489,710.50428088)
\closepath
}
}
{
\newrgbcolor{curcolor}{0 0 0}
\pscustom[linestyle=none,fillstyle=solid,fillcolor=curcolor]
{
\newpath
\moveto(642.69922954,693.03228088)
\lineto(637.54722954,693.03228088)
\lineto(642.25122954,699.94428088)
\curveto(641.35522954,700.30694754)(640.55522954,700.89361421)(639.85122954,701.70428088)
\curveto(639.16856287,702.53628088)(638.82722954,703.66694754)(638.82722954,705.09628088)
\curveto(638.82722954,706.84561421)(639.48856287,708.17894754)(640.81122954,709.09628088)
\curveto(642.1338962,710.03494754)(643.8298962,710.50428088)(645.89922954,710.50428088)
\lineto(654.02722954,710.50428088)
\lineto(654.02722954,693.03228088)
\lineto(649.25922954,693.03228088)
\lineto(649.25922954,699.52828088)
\lineto(646.63522954,699.52828088)
\closepath
\moveto(643.49922954,705.06428088)
\curveto(643.49922954,704.33894754)(643.78722954,703.76294754)(644.36322954,703.33628088)
\curveto(644.93922954,702.93094754)(645.6858962,702.72828088)(646.60322954,702.72828088)
\lineto(649.25922954,702.72828088)
\lineto(649.25922954,707.14428088)
\lineto(645.99522954,707.14428088)
\curveto(645.1418962,707.14428088)(644.51256287,706.93094754)(644.10722954,706.50428088)
\curveto(643.7018962,706.09894754)(643.49922954,705.61894754)(643.49922954,705.06428088)
\closepath
}
}
{
\newrgbcolor{curcolor}{0 0 0}
\pscustom[linestyle=none,fillstyle=solid,fillcolor=curcolor]
{
\newpath
\moveto(495.29124907,670.82428088)
\curveto(497.25391574,670.82428088)(498.84324907,670.05628088)(500.05924907,668.52028088)
\curveto(501.27524907,667.00561421)(501.88324907,664.76561421)(501.88324907,661.80028088)
\curveto(501.88324907,658.81361421)(501.25391574,656.55228088)(499.99524907,655.01628088)
\curveto(498.7365824,653.48028088)(497.12591574,652.71228088)(495.16324907,652.71228088)
\curveto(493.9045824,652.71228088)(492.90191574,652.93628088)(492.15524907,653.38428088)
\curveto(491.4085824,653.85361421)(490.8005824,654.37628088)(490.33124907,654.95228088)
\lineto(490.07524907,654.95228088)
\curveto(490.24591574,654.05628088)(490.33124907,653.20294754)(490.33124907,652.39228088)
\lineto(490.33124907,645.35228088)
\lineto(485.56324907,645.35228088)
\lineto(485.56324907,670.50428088)
\lineto(489.43524907,670.50428088)
\lineto(490.10724907,668.23228088)
\lineto(490.33124907,668.23228088)
\curveto(490.8005824,668.93628088)(491.42991574,669.54428088)(492.21924907,670.05628088)
\curveto(493.0085824,670.56828088)(494.0325824,670.82428088)(495.29124907,670.82428088)
\closepath
\moveto(493.75524907,667.01628088)
\curveto(492.51791574,667.01628088)(491.64324907,666.62161421)(491.13124907,665.83228088)
\curveto(490.61924907,665.06428088)(490.3525824,663.90161421)(490.33124907,662.34428088)
\lineto(490.33124907,661.83228088)
\curveto(490.33124907,660.14694754)(490.5765824,658.84561421)(491.06724907,657.92828088)
\curveto(491.57924907,657.03228088)(492.4965824,656.58428088)(493.81924907,656.58428088)
\curveto(494.90724907,656.58428088)(495.70724907,657.03228088)(496.21924907,657.92828088)
\curveto(496.7525824,658.84561421)(497.01924907,660.15761421)(497.01924907,661.86428088)
\curveto(497.01924907,665.29894754)(495.93124907,667.01628088)(493.75524907,667.01628088)
\closepath
}
}
{
\newrgbcolor{curcolor}{0 0 0}
\pscustom[linestyle=none,fillstyle=solid,fillcolor=curcolor]
{
\newpath
\moveto(512.98723051,670.85628088)
\curveto(515.33389718,670.85628088)(517.12589718,670.34428088)(518.36323051,669.32028088)
\curveto(519.62189718,668.31761421)(520.25123051,666.77094754)(520.25123051,664.68028088)
\lineto(520.25123051,653.03228088)
\lineto(516.92323051,653.03228088)
\lineto(515.99523051,655.40028088)
\lineto(515.86723051,655.40028088)
\curveto(515.12056385,654.46161421)(514.33123051,653.77894754)(513.49923051,653.35228088)
\curveto(512.66723051,652.92561421)(511.52589718,652.71228088)(510.07523051,652.71228088)
\curveto(508.51789718,652.71228088)(507.22723051,653.16028088)(506.20323051,654.05628088)
\curveto(505.17923051,654.95228088)(504.66723051,656.34961421)(504.66723051,658.24828088)
\curveto(504.66723051,660.10428088)(505.31789718,661.46961421)(506.61923051,662.34428088)
\curveto(507.92056385,663.21894754)(509.87256385,663.70961421)(512.47523051,663.81628088)
\lineto(515.51523051,663.91228088)
\lineto(515.51523051,664.68028088)
\curveto(515.51523051,665.59761421)(515.26989718,666.26961421)(514.77923051,666.69628088)
\curveto(514.30989718,667.12294754)(513.64856385,667.33628088)(512.79523051,667.33628088)
\curveto(511.94189718,667.33628088)(511.10989718,667.20828088)(510.29923051,666.95228088)
\curveto(509.48856385,666.71761421)(508.67789718,666.41894754)(507.86723051,666.05628088)
\lineto(506.29923051,669.28828088)
\curveto(507.21656385,669.75761421)(508.25123051,670.13094754)(509.40323051,670.40828088)
\curveto(510.55523051,670.70694754)(511.74989718,670.85628088)(512.98723051,670.85628088)
\closepath
\moveto(515.51523051,661.12828088)
\lineto(513.65923051,661.06428088)
\curveto(512.12323051,661.02161421)(511.05656385,660.74428088)(510.45923051,660.23228088)
\curveto(509.86189718,659.72028088)(509.56323051,659.04828088)(509.56323051,658.21628088)
\curveto(509.56323051,657.49094754)(509.77656385,656.96828088)(510.20323051,656.64828088)
\curveto(510.62989718,656.34961421)(511.18456385,656.20028088)(511.86723051,656.20028088)
\curveto(512.89123051,656.20028088)(513.75523051,656.49894754)(514.45923051,657.09628088)
\curveto(515.16323051,657.71494754)(515.51523051,658.57894754)(515.51523051,659.68828088)
\closepath
}
}
{
\newrgbcolor{curcolor}{0 0 0}
\pscustom[linestyle=none,fillstyle=solid,fillcolor=curcolor]
{
\newpath
\moveto(524.09123344,663.49628088)
\curveto(524.09123344,667.35761421)(524.81656678,670.35494754)(526.26723344,672.48828088)
\curveto(527.73923344,674.62161421)(530.16056678,675.98694754)(533.53123344,676.58428088)
\curveto(534.64056678,676.77628088)(535.78190011,676.93628088)(536.95523344,677.06428088)
\curveto(538.12856678,677.21361421)(539.33390011,677.36294754)(540.57123344,677.51228088)
\lineto(541.11523344,673.35228088)
\curveto(540.38990011,673.26694754)(539.58990011,673.17094754)(538.71523344,673.06428088)
\lineto(536.15523344,672.74428088)
\curveto(535.30190011,672.65894754)(534.55523344,672.56294754)(533.91523344,672.45628088)
\curveto(532.84856678,672.28561421)(531.96323344,672.01894754)(531.25923344,671.65628088)
\curveto(530.55523344,671.31494754)(530.01123344,670.76028088)(529.62723344,669.99228088)
\curveto(529.24323344,669.22428088)(529.01923344,668.11494754)(528.95523344,666.66428088)
\lineto(529.17923344,666.66428088)
\curveto(529.43523344,667.04828088)(529.78723344,667.44294754)(530.23523344,667.84828088)
\curveto(530.70456678,668.27494754)(531.26990011,668.62694754)(531.93123344,668.90428088)
\curveto(532.61390011,669.18161421)(533.40323344,669.32028088)(534.29923344,669.32028088)
\curveto(536.38990011,669.32028088)(538.04323344,668.66961421)(539.25923344,667.36828088)
\curveto(540.49656678,666.08828088)(541.11523344,664.18961421)(541.11523344,661.67228088)
\curveto(541.11523344,659.68828088)(540.75256678,658.02428088)(540.02723344,656.68028088)
\curveto(539.30190011,655.35761421)(538.29923344,654.36561421)(537.01923344,653.70428088)
\curveto(535.73923344,653.04294754)(534.25656678,652.71228088)(532.57123344,652.71228088)
\curveto(529.98990011,652.71228088)(527.93123344,653.64028088)(526.39523344,655.49628088)
\curveto(524.85923344,657.35228088)(524.09123344,660.01894754)(524.09123344,663.49628088)
\closepath
\moveto(532.85923344,656.58428088)
\curveto(533.86190011,656.58428088)(534.67256678,656.92561421)(535.29123344,657.60828088)
\curveto(535.93123344,658.29094754)(536.25123344,659.50694754)(536.25123344,661.25628088)
\curveto(536.25123344,662.64294754)(536.01656678,663.74161421)(535.54723344,664.55228088)
\curveto(535.09923344,665.38428088)(534.30990011,665.80028088)(533.17923344,665.80028088)
\curveto(532.49656678,665.80028088)(531.85656678,665.62961421)(531.25923344,665.28828088)
\curveto(530.68323344,664.96828088)(530.19256678,664.59494754)(529.78723344,664.16828088)
\curveto(529.38190011,663.74161421)(529.10456678,663.38961421)(528.95523344,663.11228088)
\curveto(528.95523344,662.02428088)(529.07256678,660.97894754)(529.30723344,659.97628088)
\curveto(529.54190011,658.97361421)(529.93656678,658.15228088)(530.49123344,657.51228088)
\curveto(531.06723344,656.89361421)(531.85656678,656.58428088)(532.85923344,656.58428088)
\closepath
}
}
{
\newrgbcolor{curcolor}{0 0 0}
\pscustom[linestyle=none,fillstyle=solid,fillcolor=curcolor]
{
\newpath
\moveto(560.92322514,661.80028088)
\curveto(560.92322514,658.89894754)(560.15522514,656.65894754)(558.61922514,655.08028088)
\curveto(557.10455848,653.50161421)(555.03522514,652.71228088)(552.41122514,652.71228088)
\curveto(550.78989181,652.71228088)(549.33922514,653.06428088)(548.05922514,653.76828088)
\curveto(546.80055848,654.47228088)(545.80855848,655.49628088)(545.08322514,656.84028088)
\curveto(544.35789181,658.20561421)(543.99522514,659.85894754)(543.99522514,661.80028088)
\curveto(543.99522514,664.70161421)(544.75255848,666.93094754)(546.26722514,668.48828088)
\curveto(547.78189181,670.04561421)(549.86189181,670.82428088)(552.50722514,670.82428088)
\curveto(554.14989181,670.82428088)(555.60055848,670.47228088)(556.85922514,669.76828088)
\curveto(558.11789181,669.06428088)(559.10989181,668.04028088)(559.83522514,666.69628088)
\curveto(560.56055848,665.35228088)(560.92322514,663.72028088)(560.92322514,661.80028088)
\closepath
\moveto(548.85922514,661.80028088)
\curveto(548.85922514,660.07228088)(549.13655848,658.76028088)(549.69122514,657.86428088)
\curveto(550.26722514,656.98961421)(551.19522514,656.55228088)(552.47522514,656.55228088)
\curveto(553.73389181,656.55228088)(554.64055848,656.98961421)(555.19522514,657.86428088)
\curveto(555.77122514,658.76028088)(556.05922514,660.07228088)(556.05922514,661.80028088)
\curveto(556.05922514,663.52828088)(555.77122514,664.81894754)(555.19522514,665.67228088)
\curveto(554.64055848,666.54694754)(553.72322514,666.98428088)(552.44322514,666.98428088)
\curveto(551.18455848,666.98428088)(550.26722514,666.54694754)(549.69122514,665.67228088)
\curveto(549.13655848,664.81894754)(548.85922514,663.52828088)(548.85922514,661.80028088)
\closepath
}
}
{
\newrgbcolor{curcolor}{0 0 0}
\pscustom[linestyle=none,fillstyle=solid,fillcolor=curcolor]
{
\newpath
\moveto(579.00320122,666.92028088)
\lineto(573.27520122,666.92028088)
\lineto(573.27520122,653.03228088)
\lineto(568.50720122,653.03228088)
\lineto(568.50720122,666.92028088)
\lineto(562.77920122,666.92028088)
\lineto(562.77920122,670.50428088)
\lineto(579.00320122,670.50428088)
\closepath
}
}
{
\newrgbcolor{curcolor}{0 0 0}
\pscustom[linestyle=none,fillstyle=solid,fillcolor=curcolor]
{
\newpath
\moveto(582.2351851,653.03228088)
\lineto(582.2351851,670.50428088)
\lineto(587.0031851,670.50428088)
\lineto(587.0031851,663.75228088)
\lineto(589.3071851,663.75228088)
\curveto(591.97385177,663.75228088)(593.9471851,663.32561421)(595.2271851,662.47228088)
\curveto(596.5071851,661.61894754)(597.1471851,660.32828088)(597.1471851,658.60028088)
\curveto(597.1471851,656.89361421)(596.54985177,655.53894754)(595.3551851,654.53628088)
\curveto(594.16051844,653.53361421)(592.19785177,653.03228088)(589.4671851,653.03228088)
\closepath
\moveto(599.6751851,653.03228088)
\lineto(599.6751851,670.50428088)
\lineto(604.4431851,670.50428088)
\lineto(604.4431851,653.03228088)
\closepath
\moveto(587.0031851,656.32828088)
\lineto(589.2111851,656.32828088)
\curveto(590.14985177,656.32828088)(590.9071851,656.48828088)(591.4831851,656.80828088)
\curveto(592.08051844,657.14961421)(592.3791851,657.72561421)(592.3791851,658.53628088)
\curveto(592.3791851,659.81628088)(591.30185177,660.45628088)(589.1471851,660.45628088)
\lineto(587.0031851,660.45628088)
\closepath
}
}
{
\newrgbcolor{curcolor}{0 0 0}
\pscustom[linestyle=none,fillstyle=solid,fillcolor=curcolor]
{
\newpath
\moveto(624.85920463,652.71228088)
\curveto(622.25653797,652.71228088)(620.24053797,653.42694754)(618.81120463,654.85628088)
\curveto(617.40320463,656.28561421)(616.69920463,658.55761421)(616.69920463,661.67228088)
\curveto(616.69920463,663.80561421)(617.0618713,665.54428088)(617.78720463,666.88828088)
\curveto(618.51253797,668.23228088)(619.51520463,669.22428088)(620.79520463,669.86428088)
\curveto(622.09653797,670.50428088)(623.5898713,670.82428088)(625.27520463,670.82428088)
\curveto(626.4698713,670.82428088)(627.50453797,670.70694754)(628.37920463,670.47228088)
\curveto(629.27520463,670.23761421)(630.0538713,669.96028088)(630.71520463,669.64028088)
\lineto(629.30720463,665.96028088)
\curveto(628.56053797,666.25894754)(627.85653797,666.50428088)(627.19520463,666.69628088)
\curveto(626.55520463,666.88828088)(625.91520463,666.98428088)(625.27520463,666.98428088)
\curveto(622.80053797,666.98428088)(621.56320463,665.22428088)(621.56320463,661.70428088)
\curveto(621.56320463,659.95494754)(621.88320463,658.66428088)(622.52320463,657.83228088)
\curveto(623.18453797,657.00028088)(624.1018713,656.58428088)(625.27520463,656.58428088)
\curveto(626.2778713,656.58428088)(627.16320463,656.71228088)(627.93120463,656.96828088)
\curveto(628.69920463,657.24561421)(629.4458713,657.61894754)(630.17120463,658.08828088)
\lineto(630.17120463,654.02428088)
\curveto(629.4458713,653.55494754)(628.6778713,653.22428088)(627.86720463,653.03228088)
\curveto(627.0778713,652.81894754)(626.07520463,652.71228088)(624.85920463,652.71228088)
\closepath
}
}
{
\newrgbcolor{curcolor}{0 0 0}
\pscustom[linestyle=none,fillstyle=solid,fillcolor=curcolor]
{
\newpath
\moveto(486.2832769,612.71228088)
\curveto(483.68061023,612.71228088)(481.66461023,613.42694754)(480.2352769,614.85628088)
\curveto(478.8272769,616.28561421)(478.1232769,618.55761421)(478.1232769,621.67228088)
\curveto(478.1232769,623.80561421)(478.48594357,625.54428088)(479.2112769,626.88828088)
\curveto(479.93661023,628.23228088)(480.9392769,629.22428088)(482.2192769,629.86428088)
\curveto(483.52061023,630.50428088)(485.01394357,630.82428088)(486.6992769,630.82428088)
\curveto(487.89394357,630.82428088)(488.92861023,630.70694754)(489.8032769,630.47228088)
\curveto(490.6992769,630.23761421)(491.47794357,629.96028088)(492.1392769,629.64028088)
\lineto(490.7312769,625.96028088)
\curveto(489.98461023,626.25894754)(489.28061023,626.50428088)(488.6192769,626.69628088)
\curveto(487.9792769,626.88828088)(487.3392769,626.98428088)(486.6992769,626.98428088)
\curveto(484.22461023,626.98428088)(482.9872769,625.22428088)(482.9872769,621.70428088)
\curveto(482.9872769,619.95494754)(483.3072769,618.66428088)(483.9472769,617.83228088)
\curveto(484.60861023,617.00028088)(485.52594357,616.58428088)(486.6992769,616.58428088)
\curveto(487.70194357,616.58428088)(488.5872769,616.71228088)(489.3552769,616.96828088)
\curveto(490.1232769,617.24561421)(490.86994357,617.61894754)(491.5952769,618.08828088)
\lineto(491.5952769,614.02428088)
\curveto(490.86994357,613.55494754)(490.10194357,613.22428088)(489.2912769,613.03228088)
\curveto(488.50194357,612.81894754)(487.4992769,612.71228088)(486.2832769,612.71228088)
\closepath
}
}
{
\newrgbcolor{curcolor}{0 0 0}
\pscustom[linestyle=none,fillstyle=solid,fillcolor=curcolor]
{
\newpath
\moveto(511.49927495,621.80028088)
\curveto(511.49927495,618.89894754)(510.73127495,616.65894754)(509.19527495,615.08028088)
\curveto(507.68060828,613.50161421)(505.61127495,612.71228088)(502.98727495,612.71228088)
\curveto(501.36594161,612.71228088)(499.91527495,613.06428088)(498.63527495,613.76828088)
\curveto(497.37660828,614.47228088)(496.38460828,615.49628088)(495.65927495,616.84028088)
\curveto(494.93394161,618.20561421)(494.57127495,619.85894754)(494.57127495,621.80028088)
\curveto(494.57127495,624.70161421)(495.32860828,626.93094754)(496.84327495,628.48828088)
\curveto(498.35794161,630.04561421)(500.43794161,630.82428088)(503.08327495,630.82428088)
\curveto(504.72594161,630.82428088)(506.17660828,630.47228088)(507.43527495,629.76828088)
\curveto(508.69394161,629.06428088)(509.68594161,628.04028088)(510.41127495,626.69628088)
\curveto(511.13660828,625.35228088)(511.49927495,623.72028088)(511.49927495,621.80028088)
\closepath
\moveto(499.43527495,621.80028088)
\curveto(499.43527495,620.07228088)(499.71260828,618.76028088)(500.26727495,617.86428088)
\curveto(500.84327495,616.98961421)(501.77127495,616.55228088)(503.05127495,616.55228088)
\curveto(504.30994161,616.55228088)(505.21660828,616.98961421)(505.77127495,617.86428088)
\curveto(506.34727495,618.76028088)(506.63527495,620.07228088)(506.63527495,621.80028088)
\curveto(506.63527495,623.52828088)(506.34727495,624.81894754)(505.77127495,625.67228088)
\curveto(505.21660828,626.54694754)(504.29927495,626.98428088)(503.01927495,626.98428088)
\curveto(501.76060828,626.98428088)(500.84327495,626.54694754)(500.26727495,625.67228088)
\curveto(499.71260828,624.81894754)(499.43527495,623.52828088)(499.43527495,621.80028088)
\closepath
}
}
{
\newrgbcolor{curcolor}{0 0 0}
\pscustom[linestyle=none,fillstyle=solid,fillcolor=curcolor]
{
\newpath
\moveto(526.89125835,630.50428088)
\lineto(532.13925835,630.50428088)
\lineto(525.22725835,622.12028088)
\lineto(532.74725835,613.03228088)
\lineto(527.33925835,613.03228088)
\lineto(520.20325835,621.89628088)
\lineto(520.20325835,613.03228088)
\lineto(515.43525835,613.03228088)
\lineto(515.43525835,630.50428088)
\lineto(520.20325835,630.50428088)
\lineto(520.20325835,622.02428088)
\closepath
}
}
{
\newrgbcolor{curcolor}{0 0 0}
\pscustom[linestyle=none,fillstyle=solid,fillcolor=curcolor]
{
\newpath
\moveto(541.8032271,630.82428088)
\curveto(544.21389376,630.82428088)(546.1232271,630.13094754)(547.5312271,628.74428088)
\curveto(548.9392271,627.37894754)(549.6432271,625.42694754)(549.6432271,622.88828088)
\lineto(549.6432271,620.58428088)
\lineto(538.3792271,620.58428088)
\curveto(538.42189376,619.24028088)(538.81656043,618.18428088)(539.5632271,617.41628088)
\curveto(540.3312271,616.64828088)(541.3872271,616.26428088)(542.7312271,616.26428088)
\curveto(543.84056043,616.26428088)(544.85389376,616.37094754)(545.7712271,616.58428088)
\curveto(546.70989376,616.81894754)(547.66989376,617.17094754)(548.6512271,617.64028088)
\lineto(548.6512271,613.96028088)
\curveto(547.77656043,613.53361421)(546.86989376,613.22428088)(545.9312271,613.03228088)
\curveto(544.99256043,612.81894754)(543.8512271,612.71228088)(542.5072271,612.71228088)
\curveto(540.75789376,612.71228088)(539.2112271,613.03228088)(537.8672271,613.67228088)
\curveto(536.5232271,614.33361421)(535.4672271,615.31494754)(534.6992271,616.61628088)
\curveto(533.9312271,617.93894754)(533.5472271,619.61361421)(533.5472271,621.64028088)
\curveto(533.5472271,623.66694754)(533.88856043,625.36294754)(534.5712271,626.72828088)
\curveto(535.2752271,628.09361421)(536.24589376,629.11761421)(537.4832271,629.80028088)
\curveto(538.72056043,630.48294754)(540.16056043,630.82428088)(541.8032271,630.82428088)
\closepath
\moveto(541.8352271,627.43228088)
\curveto(540.89656043,627.43228088)(540.12856043,627.13361421)(539.5312271,626.53628088)
\curveto(538.93389376,625.93894754)(538.58189376,625.01094754)(538.4752271,623.75228088)
\lineto(545.1632271,623.75228088)
\curveto(545.14189376,624.79761421)(544.85389376,625.67228088)(544.2992271,626.37628088)
\curveto(543.76589376,627.08028088)(542.94456043,627.43228088)(541.8352271,627.43228088)
\closepath
}
}
{
\newrgbcolor{curcolor}{0 0 0}
\pscustom[linestyle=none,fillstyle=solid,fillcolor=curcolor]
{
\newpath
\moveto(567.9792144,626.92028088)
\lineto(562.2512144,626.92028088)
\lineto(562.2512144,613.03228088)
\lineto(557.4832144,613.03228088)
\lineto(557.4832144,626.92028088)
\lineto(551.7552144,626.92028088)
\lineto(551.7552144,630.50428088)
\lineto(567.9792144,630.50428088)
\closepath
}
}
{
\newrgbcolor{curcolor}{0 0 0}
\pscustom[linestyle=none,fillstyle=solid,fillcolor=curcolor]
{
\newpath
\moveto(578.37919829,630.85628088)
\curveto(580.72586495,630.85628088)(582.51786495,630.34428088)(583.75519829,629.32028088)
\curveto(585.01386495,628.31761421)(585.64319829,626.77094754)(585.64319829,624.68028088)
\lineto(585.64319829,613.03228088)
\lineto(582.31519829,613.03228088)
\lineto(581.38719829,615.40028088)
\lineto(581.25919829,615.40028088)
\curveto(580.51253162,614.46161421)(579.72319829,613.77894754)(578.89119829,613.35228088)
\curveto(578.05919829,612.92561421)(576.91786495,612.71228088)(575.46719829,612.71228088)
\curveto(573.90986495,612.71228088)(572.61919829,613.16028088)(571.59519829,614.05628088)
\curveto(570.57119829,614.95228088)(570.05919829,616.34961421)(570.05919829,618.24828088)
\curveto(570.05919829,620.10428088)(570.70986495,621.46961421)(572.01119829,622.34428088)
\curveto(573.31253162,623.21894754)(575.26453162,623.70961421)(577.86719829,623.81628088)
\lineto(580.90719829,623.91228088)
\lineto(580.90719829,624.68028088)
\curveto(580.90719829,625.59761421)(580.66186495,626.26961421)(580.17119829,626.69628088)
\curveto(579.70186495,627.12294754)(579.04053162,627.33628088)(578.18719829,627.33628088)
\curveto(577.33386495,627.33628088)(576.50186495,627.20828088)(575.69119829,626.95228088)
\curveto(574.88053162,626.71761421)(574.06986495,626.41894754)(573.25919829,626.05628088)
\lineto(571.69119829,629.28828088)
\curveto(572.60853162,629.75761421)(573.64319829,630.13094754)(574.79519829,630.40828088)
\curveto(575.94719829,630.70694754)(577.14186495,630.85628088)(578.37919829,630.85628088)
\closepath
\moveto(580.90719829,621.12828088)
\lineto(579.05119829,621.06428088)
\curveto(577.51519829,621.02161421)(576.44853162,620.74428088)(575.85119829,620.23228088)
\curveto(575.25386495,619.72028088)(574.95519829,619.04828088)(574.95519829,618.21628088)
\curveto(574.95519829,617.49094754)(575.16853162,616.96828088)(575.59519829,616.64828088)
\curveto(576.02186495,616.34961421)(576.57653162,616.20028088)(577.25919829,616.20028088)
\curveto(578.28319829,616.20028088)(579.14719829,616.49894754)(579.85119829,617.09628088)
\curveto(580.55519829,617.71494754)(580.90719829,618.57894754)(580.90719829,619.68828088)
\closepath
}
}
{
\newrgbcolor{curcolor}{0 0 0}
\pscustom[linestyle=none,fillstyle=solid,fillcolor=curcolor]
{
\newpath
\moveto(612.55520122,630.50428088)
\lineto(612.55520122,613.03228088)
\lineto(608.10720122,613.03228088)
\lineto(608.10720122,621.60828088)
\curveto(608.10720122,622.46161421)(608.11786788,623.29361421)(608.13920122,624.10428088)
\curveto(608.18186788,624.91494754)(608.23520122,625.66161421)(608.29920122,626.34428088)
\lineto(608.20320122,626.34428088)
\lineto(603.37120122,613.03228088)
\lineto(599.78720122,613.03228088)
\lineto(594.89120122,626.37628088)
\lineto(594.76320122,626.37628088)
\curveto(594.84853455,625.67228088)(594.90186788,624.91494754)(594.92320122,624.10428088)
\curveto(594.96586788,623.31494754)(594.98720122,622.44028088)(594.98720122,621.48028088)
\lineto(594.98720122,613.03228088)
\lineto(590.53920122,613.03228088)
\lineto(590.53920122,630.50428088)
\lineto(597.29120122,630.50428088)
\lineto(601.64320122,618.66428088)
\lineto(606.05920122,630.50428088)
\closepath
}
}
{
\newrgbcolor{curcolor}{0 0 0}
\pscustom[linestyle=none,fillstyle=solid,fillcolor=curcolor]
{
\newpath
\moveto(622.15519682,630.50428088)
\lineto(622.15519682,623.59228088)
\curveto(622.15519682,623.22961421)(622.13386349,622.78161421)(622.09119682,622.24828088)
\curveto(622.06986349,621.71494754)(622.03786349,621.17094754)(621.99519682,620.61628088)
\curveto(621.97386349,620.06161421)(621.94186349,619.56028088)(621.89919682,619.11228088)
\curveto(621.85653016,618.68561421)(621.82453016,618.39761421)(621.80319682,618.24828088)
\lineto(629.86719682,630.50428088)
\lineto(635.59519682,630.50428088)
\lineto(635.59519682,613.03228088)
\lineto(630.98719682,613.03228088)
\lineto(630.98719682,620.00828088)
\curveto(630.98719682,620.56294754)(631.00853016,621.19228088)(631.05119682,621.89628088)
\curveto(631.09386349,622.60028088)(631.13653016,623.25094754)(631.17919682,623.84828088)
\curveto(631.24319682,624.46694754)(631.28586349,624.93628088)(631.30719682,625.25628088)
\lineto(623.27519682,613.03228088)
\lineto(617.54719682,613.03228088)
\lineto(617.54719682,630.50428088)
\closepath
}
}
{
\newrgbcolor{curcolor}{0 0 0}
\pscustom[linestyle=none,fillstyle=solid,fillcolor=curcolor]
{
\newpath
\moveto(795.16591005,546.80179)
\lineto(795.16591005,542.80179)
\lineto(785.56591005,542.80179)
\lineto(785.56591005,523.95379)
\lineto(780.73391005,523.95379)
\lineto(780.73391005,546.80179)
\closepath
}
}
{
\newrgbcolor{curcolor}{0 0 0}
\pscustom[linestyle=none,fillstyle=solid,fillcolor=curcolor]
{
\newpath
\moveto(811.93394618,532.72179)
\curveto(811.93394618,529.82045667)(811.16594618,527.58045667)(809.62994618,526.00179)
\curveto(808.11527952,524.42312333)(806.04594618,523.63379)(803.42194618,523.63379)
\curveto(801.80061285,523.63379)(800.34994618,523.98579)(799.06994618,524.68979)
\curveto(797.81127952,525.39379)(796.81927952,526.41779)(796.09394618,527.76179)
\curveto(795.36861285,529.12712333)(795.00594618,530.78045667)(795.00594618,532.72179)
\curveto(795.00594618,535.62312333)(795.76327952,537.85245667)(797.27794618,539.40979)
\curveto(798.79261285,540.96712333)(800.87261285,541.74579)(803.51794618,541.74579)
\curveto(805.16061285,541.74579)(806.61127952,541.39379)(807.86994618,540.68979)
\curveto(809.12861285,539.98579)(810.12061285,538.96179)(810.84594618,537.61779)
\curveto(811.57127952,536.27379)(811.93394618,534.64179)(811.93394618,532.72179)
\closepath
\moveto(799.86994618,532.72179)
\curveto(799.86994618,530.99379)(800.14727952,529.68179)(800.70194618,528.78579)
\curveto(801.27794618,527.91112333)(802.20594618,527.47379)(803.48594618,527.47379)
\curveto(804.74461285,527.47379)(805.65127952,527.91112333)(806.20594618,528.78579)
\curveto(806.78194618,529.68179)(807.06994618,530.99379)(807.06994618,532.72179)
\curveto(807.06994618,534.44979)(806.78194618,535.74045667)(806.20594618,536.59379)
\curveto(805.65127952,537.46845667)(804.73394618,537.90579)(803.45394618,537.90579)
\curveto(802.19527952,537.90579)(801.27794618,537.46845667)(800.70194618,536.59379)
\curveto(800.14727952,535.74045667)(799.86994618,534.44979)(799.86994618,532.72179)
\closepath
}
}
{
\newrgbcolor{curcolor}{0 0 0}
\pscustom[linestyle=none,fillstyle=solid,fillcolor=curcolor]
{
\newpath
\moveto(831.51792958,523.95379)
\lineto(826.74992958,523.95379)
\lineto(826.74992958,537.84179)
\lineto(822.36592958,537.84179)
\curveto(822.08859625,534.42845667)(821.71526292,531.67645667)(821.24592958,529.58579)
\curveto(820.79792958,527.51645667)(820.15792958,526.00179)(819.32592958,525.04179)
\curveto(818.51526292,524.10312333)(817.43792958,523.63379)(816.09392958,523.63379)
\curveto(814.98459625,523.63379)(814.07792958,523.80445667)(813.37392958,524.14579)
\lineto(813.37392958,527.95379)
\curveto(813.86459625,527.74045667)(814.37659625,527.63379)(814.90992958,527.63379)
\curveto(815.29392958,527.63379)(815.64592958,527.82579)(815.96592958,528.20979)
\curveto(816.28592958,528.59379)(816.58459625,529.28712333)(816.86192958,530.28979)
\curveto(817.16059625,531.29245667)(817.42726292,532.68979)(817.66192958,534.48179)
\curveto(817.89659625,536.29512333)(818.10992958,538.60979)(818.30192958,541.42579)
\lineto(831.51792958,541.42579)
\closepath
}
}
{
\newrgbcolor{curcolor}{0 0 0}
\pscustom[linestyle=none,fillstyle=solid,fillcolor=curcolor]
{
\newpath
\moveto(852.38194423,532.72179)
\curveto(852.38194423,529.82045667)(851.61394423,527.58045667)(850.07794423,526.00179)
\curveto(848.56327756,524.42312333)(846.49394423,523.63379)(843.86994423,523.63379)
\curveto(842.2486109,523.63379)(840.79794423,523.98579)(839.51794423,524.68979)
\curveto(838.25927756,525.39379)(837.26727756,526.41779)(836.54194423,527.76179)
\curveto(835.8166109,529.12712333)(835.45394423,530.78045667)(835.45394423,532.72179)
\curveto(835.45394423,535.62312333)(836.21127756,537.85245667)(837.72594423,539.40979)
\curveto(839.2406109,540.96712333)(841.3206109,541.74579)(843.96594423,541.74579)
\curveto(845.6086109,541.74579)(847.05927756,541.39379)(848.31794423,540.68979)
\curveto(849.5766109,539.98579)(850.5686109,538.96179)(851.29394423,537.61779)
\curveto(852.01927756,536.27379)(852.38194423,534.64179)(852.38194423,532.72179)
\closepath
\moveto(840.31794423,532.72179)
\curveto(840.31794423,530.99379)(840.59527756,529.68179)(841.14994423,528.78579)
\curveto(841.72594423,527.91112333)(842.65394423,527.47379)(843.93394423,527.47379)
\curveto(845.1926109,527.47379)(846.09927756,527.91112333)(846.65394423,528.78579)
\curveto(847.22994423,529.68179)(847.51794423,530.99379)(847.51794423,532.72179)
\curveto(847.51794423,534.44979)(847.22994423,535.74045667)(846.65394423,536.59379)
\curveto(846.09927756,537.46845667)(845.18194423,537.90579)(843.90194423,537.90579)
\curveto(842.64327756,537.90579)(841.72594423,537.46845667)(841.14994423,536.59379)
\curveto(840.59527756,535.74045667)(840.31794423,534.44979)(840.31794423,532.72179)
\closepath
}
}
{
\newrgbcolor{curcolor}{0 0 0}
\pscustom[linestyle=none,fillstyle=solid,fillcolor=curcolor]
{
\newpath
\moveto(863.42192763,523.63379)
\curveto(860.81926096,523.63379)(858.80326096,524.34845667)(857.37392763,525.77779)
\curveto(855.96592763,527.20712333)(855.26192763,529.47912333)(855.26192763,532.59379)
\curveto(855.26192763,534.72712333)(855.6245943,536.46579)(856.34992763,537.80979)
\curveto(857.07526096,539.15379)(858.07792763,540.14579)(859.35792763,540.78579)
\curveto(860.65926096,541.42579)(862.1525943,541.74579)(863.83792763,541.74579)
\curveto(865.0325943,541.74579)(866.06726096,541.62845667)(866.94192763,541.39379)
\curveto(867.83792763,541.15912333)(868.6165943,540.88179)(869.27792763,540.56179)
\lineto(867.86992763,536.88179)
\curveto(867.12326096,537.18045667)(866.41926096,537.42579)(865.75792763,537.61779)
\curveto(865.11792763,537.80979)(864.47792763,537.90579)(863.83792763,537.90579)
\curveto(861.36326096,537.90579)(860.12592763,536.14579)(860.12592763,532.62579)
\curveto(860.12592763,530.87645667)(860.44592763,529.58579)(861.08592763,528.75379)
\curveto(861.74726096,527.92179)(862.6645943,527.50579)(863.83792763,527.50579)
\curveto(864.8405943,527.50579)(865.72592763,527.63379)(866.49392763,527.88979)
\curveto(867.26192763,528.16712333)(868.0085943,528.54045667)(868.73392763,529.00979)
\lineto(868.73392763,524.94579)
\curveto(868.0085943,524.47645667)(867.2405943,524.14579)(866.42992763,523.95379)
\curveto(865.6405943,523.74045667)(864.63792763,523.63379)(863.42192763,523.63379)
\closepath
}
}
{
\newrgbcolor{curcolor}{0 0 0}
\pscustom[linestyle=none,fillstyle=solid,fillcolor=curcolor]
{
\newpath
\moveto(888.63792568,532.72179)
\curveto(888.63792568,529.82045667)(887.86992568,527.58045667)(886.33392568,526.00179)
\curveto(884.81925901,524.42312333)(882.74992568,523.63379)(880.12592568,523.63379)
\curveto(878.50459234,523.63379)(877.05392568,523.98579)(875.77392568,524.68979)
\curveto(874.51525901,525.39379)(873.52325901,526.41779)(872.79792568,527.76179)
\curveto(872.07259234,529.12712333)(871.70992568,530.78045667)(871.70992568,532.72179)
\curveto(871.70992568,535.62312333)(872.46725901,537.85245667)(873.98192568,539.40979)
\curveto(875.49659234,540.96712333)(877.57659234,541.74579)(880.22192568,541.74579)
\curveto(881.86459234,541.74579)(883.31525901,541.39379)(884.57392568,540.68979)
\curveto(885.83259234,539.98579)(886.82459234,538.96179)(887.54992568,537.61779)
\curveto(888.27525901,536.27379)(888.63792568,534.64179)(888.63792568,532.72179)
\closepath
\moveto(876.57392568,532.72179)
\curveto(876.57392568,530.99379)(876.85125901,529.68179)(877.40592568,528.78579)
\curveto(877.98192568,527.91112333)(878.90992568,527.47379)(880.18992568,527.47379)
\curveto(881.44859234,527.47379)(882.35525901,527.91112333)(882.90992568,528.78579)
\curveto(883.48592568,529.68179)(883.77392568,530.99379)(883.77392568,532.72179)
\curveto(883.77392568,534.44979)(883.48592568,535.74045667)(882.90992568,536.59379)
\curveto(882.35525901,537.46845667)(881.43792568,537.90579)(880.15792568,537.90579)
\curveto(878.89925901,537.90579)(877.98192568,537.46845667)(877.40592568,536.59379)
\curveto(876.85125901,535.74045667)(876.57392568,534.44979)(876.57392568,532.72179)
\closepath
}
}
{
\newrgbcolor{curcolor}{0 0 0}
\pscustom[linestyle=none,fillstyle=solid,fillcolor=curcolor]
{
\newpath
\moveto(908.06190908,536.84979)
\curveto(908.06190908,535.91112333)(907.76324241,535.11112333)(907.16590907,534.44979)
\curveto(906.58990907,533.78845667)(905.72590908,533.36179)(904.57390907,533.16979)
\lineto(904.57390907,533.04179)
\curveto(905.78990908,532.89245667)(906.76057574,532.46579)(907.48590907,531.76179)
\curveto(908.23257574,531.07912333)(908.60590907,530.21512333)(908.60590907,529.16979)
\curveto(908.60590907,528.16712333)(908.33924241,527.27112333)(907.80590907,526.48179)
\curveto(907.29390907,525.69245667)(906.47257574,525.07379)(905.34190907,524.62579)
\curveto(904.21124241,524.17779)(902.72857574,523.95379)(900.89390908,523.95379)
\lineto(892.57390907,523.95379)
\lineto(892.57390907,541.42579)
\lineto(900.89390908,541.42579)
\curveto(902.25924241,541.42579)(903.47524241,541.27645667)(904.54190908,540.97779)
\curveto(905.62990907,540.70045667)(906.48324241,540.22045667)(907.10190907,539.53779)
\curveto(907.74190907,538.87645667)(908.06190908,537.98045667)(908.06190908,536.84979)
\closepath
\moveto(903.22990908,536.46579)
\curveto(903.22990907,537.53245667)(902.38724241,538.06579)(900.70190908,538.06579)
\lineto(897.34190907,538.06579)
\lineto(897.34190907,534.60979)
\lineto(900.15790908,534.60979)
\curveto(901.16057574,534.60979)(901.91790907,534.74845667)(902.42990907,535.02579)
\curveto(902.96324241,535.32445667)(903.22990908,535.80445667)(903.22990908,536.46579)
\closepath
\moveto(903.67790908,529.42579)
\curveto(903.67790908,530.10845667)(903.40057574,530.59912333)(902.84590908,530.89779)
\curveto(902.31257574,531.21779)(901.52324241,531.37779)(900.47790907,531.37779)
\lineto(897.34190907,531.37779)
\lineto(897.34190907,527.24979)
\lineto(900.57390907,527.24979)
\curveto(901.46990908,527.24979)(902.20590907,527.40979)(902.78190907,527.72979)
\curveto(903.37924241,528.07112333)(903.67790908,528.63645667)(903.67790908,529.42579)
\closepath
}
}
{
\newrgbcolor{curcolor}{0 0 0}
\pscustom[linestyle=none,fillstyle=solid,fillcolor=curcolor]
{
\newpath
\moveto(928.5738998,532.72179)
\curveto(928.5738998,529.82045667)(927.8058998,527.58045667)(926.2698998,526.00179)
\curveto(924.75523313,524.42312333)(922.6858998,523.63379)(920.0618998,523.63379)
\curveto(918.44056646,523.63379)(916.9898998,523.98579)(915.7098998,524.68979)
\curveto(914.45123313,525.39379)(913.45923313,526.41779)(912.7338998,527.76179)
\curveto(912.00856646,529.12712333)(911.6458998,530.78045667)(911.6458998,532.72179)
\curveto(911.6458998,535.62312333)(912.40323313,537.85245667)(913.9178998,539.40979)
\curveto(915.43256646,540.96712333)(917.51256646,541.74579)(920.1578998,541.74579)
\curveto(921.80056646,541.74579)(923.25123313,541.39379)(924.5098998,540.68979)
\curveto(925.76856646,539.98579)(926.76056646,538.96179)(927.4858998,537.61779)
\curveto(928.21123313,536.27379)(928.5738998,534.64179)(928.5738998,532.72179)
\closepath
\moveto(916.5098998,532.72179)
\curveto(916.5098998,530.99379)(916.78723313,529.68179)(917.3418998,528.78579)
\curveto(917.9178998,527.91112333)(918.8458998,527.47379)(920.1258998,527.47379)
\curveto(921.38456646,527.47379)(922.29123313,527.91112333)(922.8458998,528.78579)
\curveto(923.4218998,529.68179)(923.7098998,530.99379)(923.7098998,532.72179)
\curveto(923.7098998,534.44979)(923.4218998,535.74045667)(922.8458998,536.59379)
\curveto(922.29123313,537.46845667)(921.3738998,537.90579)(920.0938998,537.90579)
\curveto(918.83523313,537.90579)(917.9178998,537.46845667)(917.3418998,536.59379)
\curveto(916.78723313,535.74045667)(916.5098998,534.44979)(916.5098998,532.72179)
\closepath
}
}
{
\newrgbcolor{curcolor}{0 0 0}
\pscustom[linestyle=none,fillstyle=solid,fillcolor=curcolor]
{
\newpath
\moveto(937.1178832,541.42579)
\lineto(937.1178832,534.51379)
\curveto(937.1178832,534.15112333)(937.09654986,533.70312333)(937.0538832,533.16979)
\curveto(937.03254986,532.63645667)(937.00054986,532.09245667)(936.9578832,531.53779)
\curveto(936.93654986,530.98312333)(936.90454986,530.48179)(936.8618832,530.03379)
\curveto(936.81921653,529.60712333)(936.78721653,529.31912333)(936.7658832,529.16979)
\lineto(944.8298832,541.42579)
\lineto(950.5578832,541.42579)
\lineto(950.5578832,523.95379)
\lineto(945.9498832,523.95379)
\lineto(945.9498832,530.92979)
\curveto(945.9498832,531.48445667)(945.97121653,532.11379)(946.0138832,532.81779)
\curveto(946.05654986,533.52179)(946.09921653,534.17245667)(946.1418832,534.76979)
\curveto(946.2058832,535.38845667)(946.24854986,535.85779)(946.2698832,536.17779)
\lineto(938.2378832,523.95379)
\lineto(932.5098832,523.95379)
\lineto(932.5098832,541.42579)
\closepath
\moveto(949.2458832,548.94579)
\curveto(949.13921653,547.83645667)(948.8298832,546.85512333)(948.3178832,546.00179)
\curveto(947.8058832,545.16979)(947.0058832,544.51912333)(945.9178832,544.04979)
\curveto(944.8298832,543.58045667)(943.36854986,543.34579)(941.5338832,543.34579)
\curveto(939.65654986,543.34579)(938.18454986,543.56979)(937.1178832,544.01779)
\curveto(936.07254986,544.46579)(935.3258832,545.10579)(934.8778832,545.93779)
\curveto(934.4298832,546.79112333)(934.16321653,547.79379)(934.0778832,548.94579)
\lineto(938.3338832,548.94579)
\curveto(938.41921653,547.77245667)(938.70721653,546.99379)(939.1978832,546.60979)
\curveto(939.7098832,546.22579)(940.52054986,546.03379)(941.6298832,546.03379)
\curveto(942.54721653,546.03379)(943.2938832,546.23645667)(943.8698832,546.64179)
\curveto(944.46721653,547.06845667)(944.81921653,547.83645667)(944.9258832,548.94579)
\closepath
}
}
{
\newrgbcolor{curcolor}{0 0 0}
\pscustom[linestyle=none,fillstyle=solid,fillcolor=curcolor]
{
\newpath
\moveto(829.26187319,501.42579)
\lineto(834.50987319,501.42579)
\lineto(827.59787319,493.04179)
\lineto(835.11787319,483.95379)
\lineto(829.70987319,483.95379)
\lineto(822.57387319,492.81779)
\lineto(822.57387319,483.95379)
\lineto(817.80587319,483.95379)
\lineto(817.80587319,501.42579)
\lineto(822.57387319,501.42579)
\lineto(822.57387319,492.94579)
\closepath
}
}
{
\newrgbcolor{curcolor}{0 0 0}
\pscustom[linestyle=none,fillstyle=solid,fillcolor=curcolor]
{
\newpath
\moveto(844.78185658,501.77779)
\curveto(847.12852325,501.77779)(848.92052325,501.26579)(850.15785658,500.24179)
\curveto(851.41652325,499.23912333)(852.04585658,497.69245667)(852.04585658,495.60179)
\lineto(852.04585658,483.95379)
\lineto(848.71785658,483.95379)
\lineto(847.78985658,486.32179)
\lineto(847.66185658,486.32179)
\curveto(846.91518992,485.38312333)(846.12585658,484.70045667)(845.29385658,484.27379)
\curveto(844.46185658,483.84712333)(843.32052325,483.63379)(841.86985658,483.63379)
\curveto(840.31252325,483.63379)(839.02185658,484.08179)(837.99785658,484.97779)
\curveto(836.97385658,485.87379)(836.46185658,487.27112333)(836.46185658,489.16979)
\curveto(836.46185658,491.02579)(837.11252325,492.39112333)(838.41385658,493.26579)
\curveto(839.71518992,494.14045667)(841.66718992,494.63112333)(844.26985658,494.73779)
\lineto(847.30985658,494.83379)
\lineto(847.30985658,495.60179)
\curveto(847.30985658,496.51912333)(847.06452325,497.19112333)(846.57385658,497.61779)
\curveto(846.10452325,498.04445667)(845.44318992,498.25779)(844.58985658,498.25779)
\curveto(843.73652325,498.25779)(842.90452325,498.12979)(842.09385658,497.87379)
\curveto(841.28318992,497.63912333)(840.47252325,497.34045667)(839.66185658,496.97779)
\lineto(838.09385658,500.20979)
\curveto(839.01118992,500.67912333)(840.04585658,501.05245667)(841.19785658,501.32979)
\curveto(842.34985658,501.62845667)(843.54452325,501.77779)(844.78185658,501.77779)
\closepath
\moveto(847.30985658,492.04979)
\lineto(845.45385658,491.98579)
\curveto(843.91785658,491.94312333)(842.85118992,491.66579)(842.25385658,491.15379)
\curveto(841.65652325,490.64179)(841.35785658,489.96979)(841.35785658,489.13779)
\curveto(841.35785658,488.41245667)(841.57118992,487.88979)(841.99785658,487.56979)
\curveto(842.42452325,487.27112333)(842.97918992,487.12179)(843.66185658,487.12179)
\curveto(844.68585658,487.12179)(845.54985658,487.42045667)(846.25385658,488.01779)
\curveto(846.95785658,488.63645667)(847.30985658,489.50045667)(847.30985658,490.60979)
\closepath
}
}
{
\newrgbcolor{curcolor}{0 0 0}
\pscustom[linestyle=none,fillstyle=solid,fillcolor=curcolor]
{
\newpath
\moveto(861.70985951,501.42579)
\lineto(861.70985951,494.70579)
\lineto(868.36585951,494.70579)
\lineto(868.36585951,501.42579)
\lineto(873.13385951,501.42579)
\lineto(873.13385951,483.95379)
\lineto(868.36585951,483.95379)
\lineto(868.36585951,491.15379)
\lineto(861.70985951,491.15379)
\lineto(861.70985951,483.95379)
\lineto(856.94185951,483.95379)
\lineto(856.94185951,501.42579)
\closepath
}
}
{
\newrgbcolor{curcolor}{0 0 0}
\pscustom[linestyle=none,fillstyle=solid,fillcolor=curcolor]
{
\newpath
\moveto(885.29388051,501.77779)
\curveto(887.64054718,501.77779)(889.43254718,501.26579)(890.66988051,500.24179)
\curveto(891.92854718,499.23912333)(892.55788051,497.69245667)(892.55788051,495.60179)
\lineto(892.55788051,483.95379)
\lineto(889.22988051,483.95379)
\lineto(888.30188051,486.32179)
\lineto(888.17388051,486.32179)
\curveto(887.42721384,485.38312333)(886.63788051,484.70045667)(885.80588051,484.27379)
\curveto(884.97388051,483.84712333)(883.83254718,483.63379)(882.38188051,483.63379)
\curveto(880.82454718,483.63379)(879.53388051,484.08179)(878.50988051,484.97779)
\curveto(877.48588051,485.87379)(876.97388051,487.27112333)(876.97388051,489.16979)
\curveto(876.97388051,491.02579)(877.62454718,492.39112333)(878.92588051,493.26579)
\curveto(880.22721384,494.14045667)(882.17921384,494.63112333)(884.78188051,494.73779)
\lineto(887.82188051,494.83379)
\lineto(887.82188051,495.60179)
\curveto(887.82188051,496.51912333)(887.57654718,497.19112333)(887.08588051,497.61779)
\curveto(886.61654718,498.04445667)(885.95521384,498.25779)(885.10188051,498.25779)
\curveto(884.24854718,498.25779)(883.41654718,498.12979)(882.60588051,497.87379)
\curveto(881.79521384,497.63912333)(880.98454718,497.34045667)(880.17388051,496.97779)
\lineto(878.60588051,500.20979)
\curveto(879.52321384,500.67912333)(880.55788051,501.05245667)(881.70988051,501.32979)
\curveto(882.86188051,501.62845667)(884.05654718,501.77779)(885.29388051,501.77779)
\closepath
\moveto(887.82188051,492.04979)
\lineto(885.96588051,491.98579)
\curveto(884.42988051,491.94312333)(883.36321384,491.66579)(882.76588051,491.15379)
\curveto(882.16854718,490.64179)(881.86988051,489.96979)(881.86988051,489.13779)
\curveto(881.86988051,488.41245667)(882.08321384,487.88979)(882.50988051,487.56979)
\curveto(882.93654718,487.27112333)(883.49121384,487.12179)(884.17388051,487.12179)
\curveto(885.19788051,487.12179)(886.06188051,487.42045667)(886.76588051,488.01779)
\curveto(887.46988051,488.63645667)(887.82188051,489.50045667)(887.82188051,490.60979)
\closepath
}
}
{
\newrgbcolor{curcolor}{0 0 0}
\pscustom[linestyle=none,fillstyle=solid,fillcolor=curcolor]
{
\newpath
\moveto(913.10188344,483.95379)
\lineto(908.33388344,483.95379)
\lineto(908.33388344,497.84179)
\lineto(903.94988344,497.84179)
\curveto(903.67255011,494.42845667)(903.29921677,491.67645667)(902.82988344,489.58579)
\curveto(902.38188344,487.51645667)(901.74188344,486.00179)(900.90988344,485.04179)
\curveto(900.09921677,484.10312333)(899.02188344,483.63379)(897.67788344,483.63379)
\curveto(896.56855011,483.63379)(895.66188344,483.80445667)(894.95788344,484.14579)
\lineto(894.95788344,487.95379)
\curveto(895.44855011,487.74045667)(895.96055011,487.63379)(896.49388344,487.63379)
\curveto(896.87788344,487.63379)(897.22988344,487.82579)(897.54988344,488.20979)
\curveto(897.86988344,488.59379)(898.16855011,489.28712333)(898.44588344,490.28979)
\curveto(898.74455011,491.29245667)(899.01121677,492.68979)(899.24588344,494.48179)
\curveto(899.48055011,496.29512333)(899.69388344,498.60979)(899.88588344,501.42579)
\lineto(913.10188344,501.42579)
\closepath
}
}
{
\newrgbcolor{curcolor}{0 0 0}
\pscustom[linestyle=none,fillstyle=solid,fillcolor=curcolor]
{
\newpath
\moveto(870.75501978,228.78227088)
\lineto(865.92301978,228.78227088)
\lineto(865.92301978,237.61427088)
\curveto(864.57901978,237.14493754)(863.34168644,236.79293754)(862.21101978,236.55827088)
\curveto(861.10168644,236.32360421)(859.98168644,236.20627088)(858.85101978,236.20627088)
\curveto(856.71768644,236.20627088)(855.04301978,236.71827088)(853.82701978,237.74227088)
\curveto(852.63235311,238.78760421)(852.03501978,240.27027088)(852.03501978,242.19027088)
\lineto(852.03501978,251.63027088)
\lineto(856.86701978,251.63027088)
\lineto(856.86701978,243.56627088)
\curveto(856.86701978,242.45693754)(857.12301978,241.62493754)(857.63501978,241.07027088)
\curveto(858.14701978,240.51560421)(859.01101978,240.23827088)(860.22701978,240.23827088)
\curveto(861.12301978,240.23827088)(862.01901978,240.33427088)(862.91501978,240.52627088)
\curveto(863.81101978,240.71827088)(864.81368644,241.00627088)(865.92301978,241.39027088)
\lineto(865.92301978,251.63027088)
\lineto(870.75501978,251.63027088)
\closepath
}
}
{
\newrgbcolor{curcolor}{0 0 0}
\pscustom[linestyle=none,fillstyle=solid,fillcolor=curcolor]
{
\newpath
\moveto(883.29904028,246.60627088)
\curveto(885.64570695,246.60627088)(887.43770695,246.09427088)(888.67504028,245.07027088)
\curveto(889.93370695,244.06760421)(890.56304028,242.52093754)(890.56304028,240.43027088)
\lineto(890.56304028,228.78227088)
\lineto(887.23504028,228.78227088)
\lineto(886.30704028,231.15027088)
\lineto(886.17904028,231.15027088)
\curveto(885.43237362,230.21160421)(884.64304028,229.52893754)(883.81104028,229.10227088)
\curveto(882.97904028,228.67560421)(881.83770695,228.46227088)(880.38704028,228.46227088)
\curveto(878.82970695,228.46227088)(877.53904028,228.91027088)(876.51504028,229.80627088)
\curveto(875.49104028,230.70227088)(874.97904028,232.09960421)(874.97904028,233.99827088)
\curveto(874.97904028,235.85427088)(875.62970695,237.21960421)(876.93104028,238.09427088)
\curveto(878.23237362,238.96893754)(880.18437362,239.45960421)(882.78704028,239.56627088)
\lineto(885.82704028,239.66227088)
\lineto(885.82704028,240.43027088)
\curveto(885.82704028,241.34760421)(885.58170695,242.01960421)(885.09104028,242.44627088)
\curveto(884.62170695,242.87293754)(883.96037362,243.08627088)(883.10704028,243.08627088)
\curveto(882.25370695,243.08627088)(881.42170695,242.95827088)(880.61104028,242.70227088)
\curveto(879.80037362,242.46760421)(878.98970695,242.16893754)(878.17904028,241.80627088)
\lineto(876.61104028,245.03827088)
\curveto(877.52837362,245.50760421)(878.56304028,245.88093754)(879.71504028,246.15827088)
\curveto(880.86704028,246.45693754)(882.06170695,246.60627088)(883.29904028,246.60627088)
\closepath
\moveto(885.82704028,236.87827088)
\lineto(883.97104028,236.81427088)
\curveto(882.43504028,236.77160421)(881.36837362,236.49427088)(880.77104028,235.98227088)
\curveto(880.17370695,235.47027088)(879.87504028,234.79827088)(879.87504028,233.96627088)
\curveto(879.87504028,233.24093754)(880.08837362,232.71827088)(880.51504028,232.39827088)
\curveto(880.94170695,232.09960421)(881.49637362,231.95027088)(882.17904028,231.95027088)
\curveto(883.20304028,231.95027088)(884.06704028,232.24893754)(884.77104028,232.84627088)
\curveto(885.47504028,233.46493754)(885.82704028,234.32893754)(885.82704028,235.43827088)
\closepath
}
}
{
\newrgbcolor{curcolor}{0 0 0}
\pscustom[linestyle=none,fillstyle=solid,fillcolor=curcolor]
{
\newpath
\moveto(909.92304321,242.67027088)
\lineto(904.19504321,242.67027088)
\lineto(904.19504321,228.78227088)
\lineto(899.42704321,228.78227088)
\lineto(899.42704321,242.67027088)
\lineto(893.69904321,242.67027088)
\lineto(893.69904321,246.25427088)
\lineto(909.92304321,246.25427088)
\closepath
}
}
{
\newrgbcolor{curcolor}{0 0 0}
\pscustom[linestyle=none,fillstyle=solid,fillcolor=curcolor]
{
\newpath
\moveto(470.85849448,157.06069213)
\lineto(475.21049448,157.06069213)
\lineto(475.21049448,146.02069213)
\curveto(475.21049448,145.48735879)(475.19982781,144.89002546)(475.17849448,144.22869213)
\curveto(475.15716115,143.56735879)(475.13582781,142.91669213)(475.11449448,142.27669213)
\curveto(475.09316115,141.65802546)(475.07182781,141.11402546)(475.05049448,140.64469213)
\curveto(475.02916115,140.19669213)(475.00782781,139.88735879)(474.98649448,139.71669213)
\lineto(475.08249448,139.71669213)
\lineto(485.64249448,157.06069213)
\lineto(491.43449448,157.06069213)
\lineto(491.43449448,134.21269213)
\lineto(487.11449448,134.21269213)
\lineto(487.11449448,145.18869213)
\curveto(487.11449448,145.76469213)(487.12516115,146.39402546)(487.14649448,147.07669213)
\curveto(487.16782781,147.78069213)(487.18916115,148.45269213)(487.21049448,149.09269213)
\curveto(487.25316115,149.73269213)(487.28516115,150.28735879)(487.30649448,150.75669213)
\curveto(487.34916115,151.24735879)(487.38116115,151.56735879)(487.40249448,151.71669213)
\lineto(487.27449448,151.71669213)
\lineto(476.68249448,134.21269213)
\lineto(470.85849448,134.21269213)
\closepath
}
}
{
\newrgbcolor{curcolor}{0 0 0}
\pscustom[linestyle=none,fillstyle=solid,fillcolor=curcolor]
{
\newpath
\moveto(501.57849301,151.68469213)
\lineto(501.57849301,144.96469213)
\lineto(508.23449301,144.96469213)
\lineto(508.23449301,151.68469213)
\lineto(513.00249301,151.68469213)
\lineto(513.00249301,134.21269213)
\lineto(508.23449301,134.21269213)
\lineto(508.23449301,141.41269213)
\lineto(501.57849301,141.41269213)
\lineto(501.57849301,134.21269213)
\lineto(496.81049301,134.21269213)
\lineto(496.81049301,151.68469213)
\closepath
}
}
{
\newrgbcolor{curcolor}{0 0 0}
\pscustom[linestyle=none,fillstyle=solid,fillcolor=curcolor]
{
\newpath
\moveto(525.09851401,133.89269213)
\curveto(522.49584734,133.89269213)(520.47984734,134.60735879)(519.05051401,136.03669213)
\curveto(517.64251401,137.46602546)(516.93851401,139.73802546)(516.93851401,142.85269213)
\curveto(516.93851401,144.98602546)(517.30118068,146.72469213)(518.02651401,148.06869213)
\curveto(518.75184734,149.41269213)(519.75451401,150.40469213)(521.03451401,151.04469213)
\curveto(522.33584734,151.68469213)(523.82918068,152.00469213)(525.51451401,152.00469213)
\curveto(526.70918068,152.00469213)(527.74384734,151.88735879)(528.61851401,151.65269213)
\curveto(529.51451401,151.41802546)(530.29318068,151.14069213)(530.95451401,150.82069213)
\lineto(529.54651401,147.14069213)
\curveto(528.79984734,147.43935879)(528.09584734,147.68469213)(527.43451401,147.87669213)
\curveto(526.79451401,148.06869213)(526.15451401,148.16469213)(525.51451401,148.16469213)
\curveto(523.03984734,148.16469213)(521.80251401,146.40469213)(521.80251401,142.88469213)
\curveto(521.80251401,141.13535879)(522.12251401,139.84469213)(522.76251401,139.01269213)
\curveto(523.42384734,138.18069213)(524.34118068,137.76469213)(525.51451401,137.76469213)
\curveto(526.51718068,137.76469213)(527.40251401,137.89269213)(528.17051401,138.14869213)
\curveto(528.93851401,138.42602546)(529.68518068,138.79935879)(530.41051401,139.26869213)
\lineto(530.41051401,135.20469213)
\curveto(529.68518068,134.73535879)(528.91718068,134.40469213)(528.10651401,134.21269213)
\curveto(527.31718068,133.99935879)(526.31451401,133.89269213)(525.09851401,133.89269213)
\closepath
}
}
{
\newrgbcolor{curcolor}{0 0 0}
\pscustom[linestyle=none,fillstyle=solid,fillcolor=curcolor]
{
\newpath
\moveto(548.90651206,148.10069213)
\lineto(543.17851206,148.10069213)
\lineto(543.17851206,134.21269213)
\lineto(538.41051206,134.21269213)
\lineto(538.41051206,148.10069213)
\lineto(532.68251206,148.10069213)
\lineto(532.68251206,151.68469213)
\lineto(548.90651206,151.68469213)
\closepath
}
}
{
\newrgbcolor{curcolor}{0 0 0}
\pscustom[linestyle=none,fillstyle=solid,fillcolor=curcolor]
{
\newpath
\moveto(561.86649594,152.00469213)
\curveto(563.82916261,152.00469213)(565.41849594,151.23669213)(566.63449594,149.70069213)
\curveto(567.85049594,148.18602546)(568.45849594,145.94602546)(568.45849594,142.98069213)
\curveto(568.45849594,139.99402546)(567.82916261,137.73269213)(566.57049594,136.19669213)
\curveto(565.31182928,134.66069213)(563.70116261,133.89269213)(561.73849594,133.89269213)
\curveto(560.47982928,133.89269213)(559.47716261,134.11669213)(558.73049594,134.56469213)
\curveto(557.98382928,135.03402546)(557.37582928,135.55669213)(556.90649594,136.13269213)
\lineto(556.65049594,136.13269213)
\curveto(556.82116261,135.23669213)(556.90649594,134.38335879)(556.90649594,133.57269213)
\lineto(556.90649594,126.53269213)
\lineto(552.13849594,126.53269213)
\lineto(552.13849594,151.68469213)
\lineto(556.01049594,151.68469213)
\lineto(556.68249594,149.41269213)
\lineto(556.90649594,149.41269213)
\curveto(557.37582928,150.11669213)(558.00516261,150.72469213)(558.79449594,151.23669213)
\curveto(559.58382928,151.74869213)(560.60782928,152.00469213)(561.86649594,152.00469213)
\closepath
\moveto(560.33049594,148.19669213)
\curveto(559.09316261,148.19669213)(558.21849594,147.80202546)(557.70649594,147.01269213)
\curveto(557.19449594,146.24469213)(556.92782928,145.08202546)(556.90649594,143.52469213)
\lineto(556.90649594,143.01269213)
\curveto(556.90649594,141.32735879)(557.15182928,140.02602546)(557.64249594,139.10869213)
\curveto(558.15449594,138.21269213)(559.07182928,137.76469213)(560.39449594,137.76469213)
\curveto(561.48249594,137.76469213)(562.28249594,138.21269213)(562.79449594,139.10869213)
\curveto(563.32782928,140.02602546)(563.59449594,141.33802546)(563.59449594,143.04469213)
\curveto(563.59449594,146.47935879)(562.50649594,148.19669213)(560.33049594,148.19669213)
\closepath
}
}
{
\newrgbcolor{curcolor}{0 0 0}
\pscustom[linestyle=none,fillstyle=solid,fillcolor=curcolor]
{
\newpath
\moveto(569.57847007,151.68469213)
\lineto(574.79447007,151.68469213)
\lineto(578.09047007,141.86069213)
\curveto(578.26113673,141.37002546)(578.38913673,140.87935879)(578.47447007,140.38869213)
\curveto(578.5598034,139.89802546)(578.6238034,139.37535879)(578.66647007,138.82069213)
\lineto(578.76247007,138.82069213)
\curveto(578.82647007,139.37535879)(578.9118034,139.89802546)(579.01847007,140.38869213)
\curveto(579.12513673,140.87935879)(579.2638034,141.37002546)(579.43447007,141.86069213)
\lineto(582.66647007,151.68469213)
\lineto(587.78647007,151.68469213)
\lineto(580.39447007,131.97269213)
\curveto(579.7118034,130.15935879)(578.74113673,128.80469213)(577.48247007,127.90869213)
\curveto(576.2238034,126.99135879)(574.76247007,126.53269213)(573.09847007,126.53269213)
\curveto(572.5438034,126.53269213)(572.07447007,126.56469213)(571.69047007,126.62869213)
\curveto(571.30647007,126.67135879)(570.96513673,126.72469213)(570.66647007,126.78869213)
\lineto(570.66647007,130.56469213)
\curveto(570.8798034,130.52202546)(571.15713673,130.47935879)(571.49847007,130.43669213)
\curveto(571.8398034,130.39402546)(572.1918034,130.37269213)(572.55447007,130.37269213)
\curveto(573.55713673,130.37269213)(574.34647007,130.68202546)(574.92247007,131.30069213)
\curveto(575.49847007,131.89802546)(575.9358034,132.62335879)(576.23447007,133.47669213)
\lineto(576.52247007,134.34069213)
\closepath
}
}
{
\newrgbcolor{curcolor}{0 0 0}
\pscustom[linestyle=none,fillstyle=solid,fillcolor=curcolor]
{
\newpath
\moveto(612.29847788,151.68469213)
\lineto(612.29847788,134.21269213)
\lineto(607.85047788,134.21269213)
\lineto(607.85047788,142.78869213)
\curveto(607.85047788,143.64202546)(607.86114454,144.47402546)(607.88247788,145.28469213)
\curveto(607.92514454,146.09535879)(607.97847788,146.84202546)(608.04247788,147.52469213)
\lineto(607.94647788,147.52469213)
\lineto(603.11447788,134.21269213)
\lineto(599.53047788,134.21269213)
\lineto(594.63447788,147.55669213)
\lineto(594.50647788,147.55669213)
\curveto(594.59181121,146.85269213)(594.64514454,146.09535879)(594.66647788,145.28469213)
\curveto(594.70914454,144.49535879)(594.73047788,143.62069213)(594.73047788,142.66069213)
\lineto(594.73047788,134.21269213)
\lineto(590.28247788,134.21269213)
\lineto(590.28247788,151.68469213)
\lineto(597.03447788,151.68469213)
\lineto(601.38647788,139.84469213)
\lineto(605.80247788,151.68469213)
\closepath
}
}
{
\newrgbcolor{curcolor}{0 0 0}
\pscustom[linestyle=none,fillstyle=solid,fillcolor=curcolor]
{
\newpath
\moveto(624.49047348,152.00469213)
\curveto(626.90114015,152.00469213)(628.81047348,151.31135879)(630.21847348,149.92469213)
\curveto(631.62647348,148.55935879)(632.33047348,146.60735879)(632.33047348,144.06869213)
\lineto(632.33047348,141.76469213)
\lineto(621.06647348,141.76469213)
\curveto(621.10914015,140.42069213)(621.50380682,139.36469213)(622.25047348,138.59669213)
\curveto(623.01847348,137.82869213)(624.07447348,137.44469213)(625.41847348,137.44469213)
\curveto(626.52780682,137.44469213)(627.54114015,137.55135879)(628.45847348,137.76469213)
\curveto(629.39714015,137.99935879)(630.35714015,138.35135879)(631.33847348,138.82069213)
\lineto(631.33847348,135.14069213)
\curveto(630.46380682,134.71402546)(629.55714015,134.40469213)(628.61847348,134.21269213)
\curveto(627.67980682,133.99935879)(626.53847348,133.89269213)(625.19447348,133.89269213)
\curveto(623.44514015,133.89269213)(621.89847348,134.21269213)(620.55447348,134.85269213)
\curveto(619.21047348,135.51402546)(618.15447348,136.49535879)(617.38647348,137.79669213)
\curveto(616.61847348,139.11935879)(616.23447348,140.79402546)(616.23447348,142.82069213)
\curveto(616.23447348,144.84735879)(616.57580682,146.54335879)(617.25847348,147.90869213)
\curveto(617.96247348,149.27402546)(618.93314015,150.29802546)(620.17047348,150.98069213)
\curveto(621.40780682,151.66335879)(622.84780682,152.00469213)(624.49047348,152.00469213)
\closepath
\moveto(624.52247348,148.61269213)
\curveto(623.58380682,148.61269213)(622.81580682,148.31402546)(622.21847348,147.71669213)
\curveto(621.62114015,147.11935879)(621.26914015,146.19135879)(621.16247348,144.93269213)
\lineto(627.85047348,144.93269213)
\curveto(627.82914015,145.97802546)(627.54114015,146.85269213)(626.98647348,147.55669213)
\curveto(626.45314015,148.26069213)(625.63180682,148.61269213)(624.52247348,148.61269213)
\closepath
}
}
{
\newrgbcolor{curcolor}{0 0 0}
\pscustom[linestyle=none,fillstyle=solid,fillcolor=curcolor]
{
\newpath
\moveto(640.97046079,151.68469213)
\lineto(640.97046079,144.96469213)
\lineto(647.62646079,144.96469213)
\lineto(647.62646079,151.68469213)
\lineto(652.39446079,151.68469213)
\lineto(652.39446079,134.21269213)
\lineto(647.62646079,134.21269213)
\lineto(647.62646079,141.41269213)
\lineto(640.97046079,141.41269213)
\lineto(640.97046079,134.21269213)
\lineto(636.20246079,134.21269213)
\lineto(636.20246079,151.68469213)
\closepath
}
}
{
\newrgbcolor{curcolor}{0 0 0}
\pscustom[linestyle=none,fillstyle=solid,fillcolor=curcolor]
{
\newpath
\moveto(671.85048178,148.10069213)
\lineto(666.12248178,148.10069213)
\lineto(666.12248178,134.21269213)
\lineto(661.35448178,134.21269213)
\lineto(661.35448178,148.10069213)
\lineto(655.62648178,148.10069213)
\lineto(655.62648178,151.68469213)
\lineto(671.85048178,151.68469213)
\closepath
}
}
{
\newrgbcolor{curcolor}{0 0 0}
\pscustom[linestyle=none,fillstyle=solid,fillcolor=curcolor]
{
\newpath
\moveto(479.57849277,111.68469213)
\lineto(479.57849277,97.70069213)
\lineto(482.13849277,97.70069213)
\lineto(482.13849277,87.94069213)
\lineto(477.85049277,87.94069213)
\lineto(477.85049277,94.21269213)
\lineto(466.10649277,94.21269213)
\lineto(466.10649277,87.94069213)
\lineto(461.81849277,87.94069213)
\lineto(461.81849277,97.70069213)
\lineto(463.29049277,97.70069213)
\curveto(464.05849277,98.87402546)(464.70915944,100.20735879)(465.24249277,101.70069213)
\curveto(465.7758261,103.21535879)(466.20249277,104.81535879)(466.52249277,106.50069213)
\curveto(466.84249277,108.20735879)(467.07715944,109.93535879)(467.22649277,111.68469213)
\closepath
\moveto(474.81049277,108.10069213)
\lineto(471.22649277,108.10069213)
\curveto(470.97049277,106.15935879)(470.61849277,104.31402546)(470.17049277,102.56469213)
\curveto(469.72249277,100.83669213)(469.09315944,99.21535879)(468.28249277,97.70069213)
\lineto(474.81049277,97.70069213)
\closepath
}
}
{
\newrgbcolor{curcolor}{0 0 0}
\pscustom[linestyle=none,fillstyle=solid,fillcolor=curcolor]
{
\newpath
\moveto(500.76247861,94.21269213)
\lineto(495.99447861,94.21269213)
\lineto(495.99447861,108.10069213)
\lineto(491.61047861,108.10069213)
\curveto(491.33314528,104.68735879)(490.95981194,101.93535879)(490.49047861,99.84469213)
\curveto(490.04247861,97.77535879)(489.40247861,96.26069213)(488.57047861,95.30069213)
\curveto(487.75981194,94.36202546)(486.68247861,93.89269213)(485.33847861,93.89269213)
\curveto(484.22914528,93.89269213)(483.32247861,94.06335879)(482.61847861,94.40469213)
\lineto(482.61847861,98.21269213)
\curveto(483.10914528,97.99935879)(483.62114528,97.89269213)(484.15447861,97.89269213)
\curveto(484.53847861,97.89269213)(484.89047861,98.08469213)(485.21047861,98.46869213)
\curveto(485.53047861,98.85269213)(485.82914528,99.54602546)(486.10647861,100.54869213)
\curveto(486.40514528,101.55135879)(486.67181194,102.94869213)(486.90647861,104.74069213)
\curveto(487.14114528,106.55402546)(487.35447861,108.86869213)(487.54647861,111.68469213)
\lineto(500.76247861,111.68469213)
\closepath
}
}
{
\newrgbcolor{curcolor}{0 0 0}
\pscustom[linestyle=none,fillstyle=solid,fillcolor=curcolor]
{
\newpath
\moveto(508.41049326,94.21269213)
\lineto(503.25849326,94.21269213)
\lineto(507.96249326,101.12469213)
\curveto(507.06649326,101.48735879)(506.26649326,102.07402546)(505.56249326,102.88469213)
\curveto(504.87982659,103.71669213)(504.53849326,104.84735879)(504.53849326,106.27669213)
\curveto(504.53849326,108.02602546)(505.19982659,109.35935879)(506.52249326,110.27669213)
\curveto(507.84515993,111.21535879)(509.54115993,111.68469213)(511.61049326,111.68469213)
\lineto(519.73849326,111.68469213)
\lineto(519.73849326,94.21269213)
\lineto(514.97049326,94.21269213)
\lineto(514.97049326,100.70869213)
\lineto(512.34649326,100.70869213)
\closepath
\moveto(509.21049326,106.24469213)
\curveto(509.21049326,105.51935879)(509.49849326,104.94335879)(510.07449326,104.51669213)
\curveto(510.65049326,104.11135879)(511.39715993,103.90869213)(512.31449326,103.90869213)
\lineto(514.97049326,103.90869213)
\lineto(514.97049326,108.32469213)
\lineto(511.70649326,108.32469213)
\curveto(510.85315993,108.32469213)(510.22382659,108.11135879)(509.81849326,107.68469213)
\curveto(509.41315993,107.27935879)(509.21049326,106.79935879)(509.21049326,106.24469213)
\closepath
}
}
{
\newrgbcolor{curcolor}{0 0 0}
\pscustom[linestyle=none,fillstyle=solid,fillcolor=curcolor]
{
\newpath
\moveto(542.77851377,112.00469213)
\curveto(544.74118043,112.00469213)(546.33051377,111.23669213)(547.54651377,109.70069213)
\curveto(548.76251377,108.18602546)(549.37051377,105.94602546)(549.37051377,102.98069213)
\curveto(549.37051377,99.99402546)(548.74118043,97.73269213)(547.48251377,96.19669213)
\curveto(546.2238471,94.66069213)(544.61318043,93.89269213)(542.65051377,93.89269213)
\curveto(541.3918471,93.89269213)(540.38918043,94.11669213)(539.64251377,94.56469213)
\curveto(538.8958471,95.03402546)(538.2878471,95.55669213)(537.81851377,96.13269213)
\lineto(537.56251377,96.13269213)
\curveto(537.73318043,95.23669213)(537.81851377,94.38335879)(537.81851377,93.57269213)
\lineto(537.81851377,86.53269213)
\lineto(533.05051377,86.53269213)
\lineto(533.05051377,111.68469213)
\lineto(536.92251377,111.68469213)
\lineto(537.59451377,109.41269213)
\lineto(537.81851377,109.41269213)
\curveto(538.2878471,110.11669213)(538.91718043,110.72469213)(539.70651377,111.23669213)
\curveto(540.4958471,111.74869213)(541.5198471,112.00469213)(542.77851377,112.00469213)
\closepath
\moveto(541.24251377,108.19669213)
\curveto(540.00518043,108.19669213)(539.13051377,107.80202546)(538.61851377,107.01269213)
\curveto(538.10651377,106.24469213)(537.8398471,105.08202546)(537.81851377,103.52469213)
\lineto(537.81851377,103.01269213)
\curveto(537.81851377,101.32735879)(538.0638471,100.02602546)(538.55451377,99.10869213)
\curveto(539.06651377,98.21269213)(539.9838471,97.76469213)(541.30651377,97.76469213)
\curveto(542.39451377,97.76469213)(543.19451377,98.21269213)(543.70651377,99.10869213)
\curveto(544.2398471,100.02602546)(544.50651377,101.33802546)(544.50651377,103.04469213)
\curveto(544.50651377,106.47935879)(543.41851377,108.19669213)(541.24251377,108.19669213)
\closepath
}
}
{
\newrgbcolor{curcolor}{0 0 0}
\pscustom[linestyle=none,fillstyle=solid,fillcolor=curcolor]
{
\newpath
\moveto(560.47449521,112.03669213)
\curveto(562.82116188,112.03669213)(564.61316188,111.52469213)(565.85049521,110.50069213)
\curveto(567.10916188,109.49802546)(567.73849521,107.95135879)(567.73849521,105.86069213)
\lineto(567.73849521,94.21269213)
\lineto(564.41049521,94.21269213)
\lineto(563.48249521,96.58069213)
\lineto(563.35449521,96.58069213)
\curveto(562.60782855,95.64202546)(561.81849521,94.95935879)(560.98649521,94.53269213)
\curveto(560.15449521,94.10602546)(559.01316188,93.89269213)(557.56249521,93.89269213)
\curveto(556.00516188,93.89269213)(554.71449521,94.34069213)(553.69049521,95.23669213)
\curveto(552.66649521,96.13269213)(552.15449521,97.53002546)(552.15449521,99.42869213)
\curveto(552.15449521,101.28469213)(552.80516188,102.65002546)(554.10649521,103.52469213)
\curveto(555.40782855,104.39935879)(557.35982855,104.89002546)(559.96249521,104.99669213)
\lineto(563.00249521,105.09269213)
\lineto(563.00249521,105.86069213)
\curveto(563.00249521,106.77802546)(562.75716188,107.45002546)(562.26649521,107.87669213)
\curveto(561.79716188,108.30335879)(561.13582855,108.51669213)(560.28249521,108.51669213)
\curveto(559.42916188,108.51669213)(558.59716188,108.38869213)(557.78649521,108.13269213)
\curveto(556.97582855,107.89802546)(556.16516188,107.59935879)(555.35449521,107.23669213)
\lineto(553.78649521,110.46869213)
\curveto(554.70382855,110.93802546)(555.73849521,111.31135879)(556.89049521,111.58869213)
\curveto(558.04249521,111.88735879)(559.23716188,112.03669213)(560.47449521,112.03669213)
\closepath
\moveto(563.00249521,102.30869213)
\lineto(561.14649521,102.24469213)
\curveto(559.61049521,102.20202546)(558.54382855,101.92469213)(557.94649521,101.41269213)
\curveto(557.34916188,100.90069213)(557.05049521,100.22869213)(557.05049521,99.39669213)
\curveto(557.05049521,98.67135879)(557.26382855,98.14869213)(557.69049521,97.82869213)
\curveto(558.11716188,97.53002546)(558.67182855,97.38069213)(559.35449521,97.38069213)
\curveto(560.37849521,97.38069213)(561.24249521,97.67935879)(561.94649521,98.27669213)
\curveto(562.65049521,98.89535879)(563.00249521,99.75935879)(563.00249521,100.86869213)
\closepath
}
}
{
\newrgbcolor{curcolor}{0 0 0}
\pscustom[linestyle=none,fillstyle=solid,fillcolor=curcolor]
{
\newpath
\moveto(571.57849814,104.67669213)
\curveto(571.57849814,108.53802546)(572.30383147,111.53535879)(573.75449814,113.66869213)
\curveto(575.22649814,115.80202546)(577.64783147,117.16735879)(581.01849814,117.76469213)
\curveto(582.12783147,117.95669213)(583.26916481,118.11669213)(584.44249814,118.24469213)
\curveto(585.61583147,118.39402546)(586.82116481,118.54335879)(588.05849814,118.69269213)
\lineto(588.60249814,114.53269213)
\curveto(587.87716481,114.44735879)(587.07716481,114.35135879)(586.20249814,114.24469213)
\lineto(583.64249814,113.92469213)
\curveto(582.78916481,113.83935879)(582.04249814,113.74335879)(581.40249814,113.63669213)
\curveto(580.33583147,113.46602546)(579.45049814,113.19935879)(578.74649814,112.83669213)
\curveto(578.04249814,112.49535879)(577.49849814,111.94069213)(577.11449814,111.17269213)
\curveto(576.73049814,110.40469213)(576.50649814,109.29535879)(576.44249814,107.84469213)
\lineto(576.66649814,107.84469213)
\curveto(576.92249814,108.22869213)(577.27449814,108.62335879)(577.72249814,109.02869213)
\curveto(578.19183147,109.45535879)(578.75716481,109.80735879)(579.41849814,110.08469213)
\curveto(580.10116481,110.36202546)(580.89049814,110.50069213)(581.78649814,110.50069213)
\curveto(583.87716481,110.50069213)(585.53049814,109.85002546)(586.74649814,108.54869213)
\curveto(587.98383147,107.26869213)(588.60249814,105.37002546)(588.60249814,102.85269213)
\curveto(588.60249814,100.86869213)(588.23983147,99.20469213)(587.51449814,97.86069213)
\curveto(586.78916481,96.53802546)(585.78649814,95.54602546)(584.50649814,94.88469213)
\curveto(583.22649814,94.22335879)(581.74383147,93.89269213)(580.05849814,93.89269213)
\curveto(577.47716481,93.89269213)(575.41849814,94.82069213)(573.88249814,96.67669213)
\curveto(572.34649814,98.53269213)(571.57849814,101.19935879)(571.57849814,104.67669213)
\closepath
\moveto(580.34649814,97.76469213)
\curveto(581.34916481,97.76469213)(582.15983147,98.10602546)(582.77849814,98.78869213)
\curveto(583.41849814,99.47135879)(583.73849814,100.68735879)(583.73849814,102.43669213)
\curveto(583.73849814,103.82335879)(583.50383147,104.92202546)(583.03449814,105.73269213)
\curveto(582.58649814,106.56469213)(581.79716481,106.98069213)(580.66649814,106.98069213)
\curveto(579.98383147,106.98069213)(579.34383147,106.81002546)(578.74649814,106.46869213)
\curveto(578.17049814,106.14869213)(577.67983147,105.77535879)(577.27449814,105.34869213)
\curveto(576.86916481,104.92202546)(576.59183147,104.57002546)(576.44249814,104.29269213)
\curveto(576.44249814,103.20469213)(576.55983147,102.15935879)(576.79449814,101.15669213)
\curveto(577.02916481,100.15402546)(577.42383147,99.33269213)(577.97849814,98.69269213)
\curveto(578.55449814,98.07402546)(579.34383147,97.76469213)(580.34649814,97.76469213)
\closepath
}
}
{
\newrgbcolor{curcolor}{0 0 0}
\pscustom[linestyle=none,fillstyle=solid,fillcolor=curcolor]
{
\newpath
\moveto(608.41048984,102.98069213)
\curveto(608.41048984,100.07935879)(607.64248984,97.83935879)(606.10648984,96.26069213)
\curveto(604.59182317,94.68202546)(602.52248984,93.89269213)(599.89848984,93.89269213)
\curveto(598.27715651,93.89269213)(596.82648984,94.24469213)(595.54648984,94.94869213)
\curveto(594.28782317,95.65269213)(593.29582317,96.67669213)(592.57048984,98.02069213)
\curveto(591.84515651,99.38602546)(591.48248984,101.03935879)(591.48248984,102.98069213)
\curveto(591.48248984,105.88202546)(592.23982317,108.11135879)(593.75448984,109.66869213)
\curveto(595.26915651,111.22602546)(597.34915651,112.00469213)(599.99448984,112.00469213)
\curveto(601.63715651,112.00469213)(603.08782317,111.65269213)(604.34648984,110.94869213)
\curveto(605.60515651,110.24469213)(606.59715651,109.22069213)(607.32248984,107.87669213)
\curveto(608.04782317,106.53269213)(608.41048984,104.90069213)(608.41048984,102.98069213)
\closepath
\moveto(596.34648984,102.98069213)
\curveto(596.34648984,101.25269213)(596.62382317,99.94069213)(597.17848984,99.04469213)
\curveto(597.75448984,98.17002546)(598.68248984,97.73269213)(599.96248984,97.73269213)
\curveto(601.22115651,97.73269213)(602.12782317,98.17002546)(602.68248984,99.04469213)
\curveto(603.25848984,99.94069213)(603.54648984,101.25269213)(603.54648984,102.98069213)
\curveto(603.54648984,104.70869213)(603.25848984,105.99935879)(602.68248984,106.85269213)
\curveto(602.12782317,107.72735879)(601.21048984,108.16469213)(599.93048984,108.16469213)
\curveto(598.67182317,108.16469213)(597.75448984,107.72735879)(597.17848984,106.85269213)
\curveto(596.62382317,105.99935879)(596.34648984,104.70869213)(596.34648984,102.98069213)
\closepath
}
}
{
\newrgbcolor{curcolor}{0 0 0}
\pscustom[linestyle=none,fillstyle=solid,fillcolor=curcolor]
{
\newpath
\moveto(626.49046591,108.10069213)
\lineto(620.76246591,108.10069213)
\lineto(620.76246591,94.21269213)
\lineto(615.99446591,94.21269213)
\lineto(615.99446591,108.10069213)
\lineto(610.26646591,108.10069213)
\lineto(610.26646591,111.68469213)
\lineto(626.49046591,111.68469213)
\closepath
}
}
{
\newrgbcolor{curcolor}{0 0 0}
\pscustom[linestyle=none,fillstyle=solid,fillcolor=curcolor]
{
\newpath
\moveto(629.7224498,94.21269213)
\lineto(629.7224498,111.68469213)
\lineto(634.4904498,111.68469213)
\lineto(634.4904498,104.93269213)
\lineto(636.7944498,104.93269213)
\curveto(639.46111647,104.93269213)(641.4344498,104.50602546)(642.7144498,103.65269213)
\curveto(643.9944498,102.79935879)(644.6344498,101.50869213)(644.6344498,99.78069213)
\curveto(644.6344498,98.07402546)(644.03711647,96.71935879)(642.8424498,95.71669213)
\curveto(641.64778313,94.71402546)(639.68511647,94.21269213)(636.9544498,94.21269213)
\closepath
\moveto(647.1624498,94.21269213)
\lineto(647.1624498,111.68469213)
\lineto(651.9304498,111.68469213)
\lineto(651.9304498,94.21269213)
\closepath
\moveto(634.4904498,97.50869213)
\lineto(636.6984498,97.50869213)
\curveto(637.63711647,97.50869213)(638.3944498,97.66869213)(638.9704498,97.98869213)
\curveto(639.56778313,98.33002546)(639.8664498,98.90602546)(639.8664498,99.71669213)
\curveto(639.8664498,100.99669213)(638.78911647,101.63669213)(636.6344498,101.63669213)
\lineto(634.4904498,101.63669213)
\closepath
}
}
{
\newrgbcolor{curcolor}{0 0 0}
\pscustom[linestyle=none,fillstyle=solid,fillcolor=curcolor]
{
\newpath
\moveto(672.34646933,93.89269213)
\curveto(669.74380267,93.89269213)(667.72780267,94.60735879)(666.29846933,96.03669213)
\curveto(664.89046933,97.46602546)(664.18646933,99.73802546)(664.18646933,102.85269213)
\curveto(664.18646933,104.98602546)(664.549136,106.72469213)(665.27446933,108.06869213)
\curveto(665.99980267,109.41269213)(667.00246933,110.40469213)(668.28246933,111.04469213)
\curveto(669.58380267,111.68469213)(671.077136,112.00469213)(672.76246933,112.00469213)
\curveto(673.957136,112.00469213)(674.99180267,111.88735879)(675.86646933,111.65269213)
\curveto(676.76246933,111.41802546)(677.541136,111.14069213)(678.20246933,110.82069213)
\lineto(676.79446933,107.14069213)
\curveto(676.04780267,107.43935879)(675.34380267,107.68469213)(674.68246933,107.87669213)
\curveto(674.04246933,108.06869213)(673.40246933,108.16469213)(672.76246933,108.16469213)
\curveto(670.28780267,108.16469213)(669.05046933,106.40469213)(669.05046933,102.88469213)
\curveto(669.05046933,101.13535879)(669.37046933,99.84469213)(670.01046933,99.01269213)
\curveto(670.67180267,98.18069213)(671.589136,97.76469213)(672.76246933,97.76469213)
\curveto(673.765136,97.76469213)(674.65046933,97.89269213)(675.41846933,98.14869213)
\curveto(676.18646933,98.42602546)(676.933136,98.79935879)(677.65846933,99.26869213)
\lineto(677.65846933,95.20469213)
\curveto(676.933136,94.73535879)(676.165136,94.40469213)(675.35446933,94.21269213)
\curveto(674.565136,93.99935879)(673.56246933,93.89269213)(672.34646933,93.89269213)
\closepath
}
}
{
\newrgbcolor{curcolor}{0 0 0}
\pscustom[linestyle=none,fillstyle=solid,fillcolor=curcolor]
{
\newpath
\moveto(483.56254233,71.68469213)
\lineto(483.56254233,68.10069213)
\lineto(476.26654233,68.10069213)
\lineto(476.26654233,54.21269213)
\lineto(471.49854233,54.21269213)
\lineto(471.49854233,71.68469213)
\closepath
}
}
{
\newrgbcolor{curcolor}{0 0 0}
\pscustom[linestyle=none,fillstyle=solid,fillcolor=curcolor]
{
\newpath
\moveto(493.67455551,72.00469213)
\curveto(496.08522218,72.00469213)(497.99455551,71.31135879)(499.40255551,69.92469213)
\curveto(500.81055551,68.55935879)(501.51455551,66.60735879)(501.51455551,64.06869213)
\lineto(501.51455551,61.76469213)
\lineto(490.25055551,61.76469213)
\curveto(490.29322218,60.42069213)(490.68788885,59.36469213)(491.43455551,58.59669213)
\curveto(492.20255551,57.82869213)(493.25855551,57.44469213)(494.60255551,57.44469213)
\curveto(495.71188885,57.44469213)(496.72522218,57.55135879)(497.64255551,57.76469213)
\curveto(498.58122218,57.99935879)(499.54122218,58.35135879)(500.52255551,58.82069213)
\lineto(500.52255551,55.14069213)
\curveto(499.64788885,54.71402546)(498.74122218,54.40469213)(497.80255551,54.21269213)
\curveto(496.86388885,53.99935879)(495.72255551,53.89269213)(494.37855551,53.89269213)
\curveto(492.62922218,53.89269213)(491.08255551,54.21269213)(489.73855551,54.85269213)
\curveto(488.39455551,55.51402546)(487.33855551,56.49535879)(486.57055551,57.79669213)
\curveto(485.80255551,59.11935879)(485.41855551,60.79402546)(485.41855551,62.82069213)
\curveto(485.41855551,64.84735879)(485.75988885,66.54335879)(486.44255551,67.90869213)
\curveto(487.14655551,69.27402546)(488.11722218,70.29802546)(489.35455551,70.98069213)
\curveto(490.59188885,71.66335879)(492.03188885,72.00469213)(493.67455551,72.00469213)
\closepath
\moveto(493.70655551,68.61269213)
\curveto(492.76788885,68.61269213)(491.99988885,68.31402546)(491.40255551,67.71669213)
\curveto(490.80522218,67.11935879)(490.45322218,66.19135879)(490.34655551,64.93269213)
\lineto(497.03455551,64.93269213)
\curveto(497.01322218,65.97802546)(496.72522218,66.85269213)(496.17055551,67.55669213)
\curveto(495.63722218,68.26069213)(494.81588885,68.61269213)(493.70655551,68.61269213)
\closepath
}
}
{
\newrgbcolor{curcolor}{0 0 0}
\pscustom[linestyle=none,fillstyle=solid,fillcolor=curcolor]
{
\newpath
\moveto(521.25854282,62.98069213)
\curveto(521.25854282,60.07935879)(520.49054282,57.83935879)(518.95454282,56.26069213)
\curveto(517.43987615,54.68202546)(515.37054282,53.89269213)(512.74654282,53.89269213)
\curveto(511.12520949,53.89269213)(509.67454282,54.24469213)(508.39454282,54.94869213)
\curveto(507.13587615,55.65269213)(506.14387615,56.67669213)(505.41854282,58.02069213)
\curveto(504.69320949,59.38602546)(504.33054282,61.03935879)(504.33054282,62.98069213)
\curveto(504.33054282,65.88202546)(505.08787615,68.11135879)(506.60254282,69.66869213)
\curveto(508.11720949,71.22602546)(510.19720949,72.00469213)(512.84254282,72.00469213)
\curveto(514.48520949,72.00469213)(515.93587615,71.65269213)(517.19454282,70.94869213)
\curveto(518.45320949,70.24469213)(519.44520949,69.22069213)(520.17054282,67.87669213)
\curveto(520.89587615,66.53269213)(521.25854282,64.90069213)(521.25854282,62.98069213)
\closepath
\moveto(509.19454282,62.98069213)
\curveto(509.19454282,61.25269213)(509.47187615,59.94069213)(510.02654282,59.04469213)
\curveto(510.60254282,58.17002546)(511.53054282,57.73269213)(512.81054282,57.73269213)
\curveto(514.06920949,57.73269213)(514.97587615,58.17002546)(515.53054282,59.04469213)
\curveto(516.10654282,59.94069213)(516.39454282,61.25269213)(516.39454282,62.98069213)
\curveto(516.39454282,64.70869213)(516.10654282,65.99935879)(515.53054282,66.85269213)
\curveto(514.97587615,67.72735879)(514.05854282,68.16469213)(512.77854282,68.16469213)
\curveto(511.51987615,68.16469213)(510.60254282,67.72735879)(510.02654282,66.85269213)
\curveto(509.47187615,65.99935879)(509.19454282,64.70869213)(509.19454282,62.98069213)
\closepath
}
}
{
\newrgbcolor{curcolor}{0 0 0}
\pscustom[linestyle=none,fillstyle=solid,fillcolor=curcolor]
{
\newpath
\moveto(547.21052622,71.68469213)
\lineto(547.21052622,54.21269213)
\lineto(542.76252622,54.21269213)
\lineto(542.76252622,62.78869213)
\curveto(542.76252622,63.64202546)(542.77319288,64.47402546)(542.79452622,65.28469213)
\curveto(542.83719288,66.09535879)(542.89052622,66.84202546)(542.95452622,67.52469213)
\lineto(542.85852622,67.52469213)
\lineto(538.02652622,54.21269213)
\lineto(534.44252622,54.21269213)
\lineto(529.54652622,67.55669213)
\lineto(529.41852622,67.55669213)
\curveto(529.50385955,66.85269213)(529.55719288,66.09535879)(529.57852622,65.28469213)
\curveto(529.62119288,64.49535879)(529.64252622,63.62069213)(529.64252622,62.66069213)
\lineto(529.64252622,54.21269213)
\lineto(525.19452622,54.21269213)
\lineto(525.19452622,71.68469213)
\lineto(531.94652622,71.68469213)
\lineto(536.29852622,59.84469213)
\lineto(540.71452622,71.68469213)
\closepath
}
}
{
\newrgbcolor{curcolor}{0 0 0}
\pscustom[linestyle=none,fillstyle=solid,fillcolor=curcolor]
{
\newpath
\moveto(559.40252182,72.00469213)
\curveto(561.81318849,72.00469213)(563.72252182,71.31135879)(565.13052182,69.92469213)
\curveto(566.53852182,68.55935879)(567.24252182,66.60735879)(567.24252182,64.06869213)
\lineto(567.24252182,61.76469213)
\lineto(555.97852182,61.76469213)
\curveto(556.02118849,60.42069213)(556.41585516,59.36469213)(557.16252182,58.59669213)
\curveto(557.93052182,57.82869213)(558.98652182,57.44469213)(560.33052182,57.44469213)
\curveto(561.43985516,57.44469213)(562.45318849,57.55135879)(563.37052182,57.76469213)
\curveto(564.30918849,57.99935879)(565.26918849,58.35135879)(566.25052182,58.82069213)
\lineto(566.25052182,55.14069213)
\curveto(565.37585516,54.71402546)(564.46918849,54.40469213)(563.53052182,54.21269213)
\curveto(562.59185516,53.99935879)(561.45052182,53.89269213)(560.10652182,53.89269213)
\curveto(558.35718849,53.89269213)(556.81052182,54.21269213)(555.46652182,54.85269213)
\curveto(554.12252182,55.51402546)(553.06652182,56.49535879)(552.29852182,57.79669213)
\curveto(551.53052182,59.11935879)(551.14652182,60.79402546)(551.14652182,62.82069213)
\curveto(551.14652182,64.84735879)(551.48785516,66.54335879)(552.17052182,67.90869213)
\curveto(552.87452182,69.27402546)(553.84518849,70.29802546)(555.08252182,70.98069213)
\curveto(556.31985516,71.66335879)(557.75985516,72.00469213)(559.40252182,72.00469213)
\closepath
\moveto(559.43452182,68.61269213)
\curveto(558.49585516,68.61269213)(557.72785516,68.31402546)(557.13052182,67.71669213)
\curveto(556.53318849,67.11935879)(556.18118849,66.19135879)(556.07452182,64.93269213)
\lineto(562.76252182,64.93269213)
\curveto(562.74118849,65.97802546)(562.45318849,66.85269213)(561.89852182,67.55669213)
\curveto(561.36518849,68.26069213)(560.54385516,68.61269213)(559.43452182,68.61269213)
\closepath
}
}
{
\newrgbcolor{curcolor}{0 0 0}
\pscustom[linestyle=none,fillstyle=solid,fillcolor=curcolor]
{
\newpath
\moveto(585.57850913,68.10069213)
\lineto(579.85050913,68.10069213)
\lineto(579.85050913,54.21269213)
\lineto(575.08250913,54.21269213)
\lineto(575.08250913,68.10069213)
\lineto(569.35450913,68.10069213)
\lineto(569.35450913,71.68469213)
\lineto(585.57850913,71.68469213)
\closepath
}
}
{
\newrgbcolor{curcolor}{0 0 0}
\pscustom[linestyle=none,fillstyle=solid,fillcolor=curcolor]
{
\newpath
\moveto(598.53849301,72.00469213)
\curveto(600.50115968,72.00469213)(602.09049301,71.23669213)(603.30649301,69.70069213)
\curveto(604.52249301,68.18602546)(605.13049301,65.94602546)(605.13049301,62.98069213)
\curveto(605.13049301,59.99402546)(604.50115968,57.73269213)(603.24249301,56.19669213)
\curveto(601.98382635,54.66069213)(600.37315968,53.89269213)(598.41049301,53.89269213)
\curveto(597.15182635,53.89269213)(596.14915968,54.11669213)(595.40249301,54.56469213)
\curveto(594.65582635,55.03402546)(594.04782635,55.55669213)(593.57849301,56.13269213)
\lineto(593.32249301,56.13269213)
\curveto(593.49315968,55.23669213)(593.57849301,54.38335879)(593.57849301,53.57269213)
\lineto(593.57849301,46.53269213)
\lineto(588.81049301,46.53269213)
\lineto(588.81049301,71.68469213)
\lineto(592.68249301,71.68469213)
\lineto(593.35449301,69.41269213)
\lineto(593.57849301,69.41269213)
\curveto(594.04782635,70.11669213)(594.67715968,70.72469213)(595.46649301,71.23669213)
\curveto(596.25582635,71.74869213)(597.27982635,72.00469213)(598.53849301,72.00469213)
\closepath
\moveto(597.00249301,68.19669213)
\curveto(595.76515968,68.19669213)(594.89049301,67.80202546)(594.37849301,67.01269213)
\curveto(593.86649301,66.24469213)(593.59982635,65.08202546)(593.57849301,63.52469213)
\lineto(593.57849301,63.01269213)
\curveto(593.57849301,61.32735879)(593.82382635,60.02602546)(594.31449301,59.10869213)
\curveto(594.82649301,58.21269213)(595.74382635,57.76469213)(597.06649301,57.76469213)
\curveto(598.15449301,57.76469213)(598.95449301,58.21269213)(599.46649301,59.10869213)
\curveto(599.99982635,60.02602546)(600.26649301,61.33802546)(600.26649301,63.04469213)
\curveto(600.26649301,66.47935879)(599.17849301,68.19669213)(597.00249301,68.19669213)
\closepath
}
}
{
\newrgbcolor{curcolor}{0 0 0}
\pscustom[linestyle=none,fillstyle=solid,fillcolor=curcolor]
{
\newpath
\moveto(613.67447446,71.68469213)
\lineto(613.67447446,64.77269213)
\curveto(613.67447446,64.41002546)(613.65314113,63.96202546)(613.61047446,63.42869213)
\curveto(613.58914113,62.89535879)(613.55714113,62.35135879)(613.51447446,61.79669213)
\curveto(613.49314113,61.24202546)(613.46114113,60.74069213)(613.41847446,60.29269213)
\curveto(613.37580779,59.86602546)(613.34380779,59.57802546)(613.32247446,59.42869213)
\lineto(621.38647446,71.68469213)
\lineto(627.11447446,71.68469213)
\lineto(627.11447446,54.21269213)
\lineto(622.50647446,54.21269213)
\lineto(622.50647446,61.18869213)
\curveto(622.50647446,61.74335879)(622.52780779,62.37269213)(622.57047446,63.07669213)
\curveto(622.61314113,63.78069213)(622.65580779,64.43135879)(622.69847446,65.02869213)
\curveto(622.76247446,65.64735879)(622.80514113,66.11669213)(622.82647446,66.43669213)
\lineto(614.79447446,54.21269213)
\lineto(609.06647446,54.21269213)
\lineto(609.06647446,71.68469213)
\closepath
}
}
{
\newrgbcolor{curcolor}{0 0 0}
\pscustom[linestyle=none,fillstyle=solid,fillcolor=curcolor]
{
\newpath
\moveto(639.30645249,72.00469213)
\curveto(641.71711915,72.00469213)(643.62645249,71.31135879)(645.03445249,69.92469213)
\curveto(646.44245249,68.55935879)(647.14645249,66.60735879)(647.14645249,64.06869213)
\lineto(647.14645249,61.76469213)
\lineto(635.88245249,61.76469213)
\curveto(635.92511915,60.42069213)(636.31978582,59.36469213)(637.06645249,58.59669213)
\curveto(637.83445249,57.82869213)(638.89045249,57.44469213)(640.23445249,57.44469213)
\curveto(641.34378582,57.44469213)(642.35711915,57.55135879)(643.27445249,57.76469213)
\curveto(644.21311915,57.99935879)(645.17311915,58.35135879)(646.15445249,58.82069213)
\lineto(646.15445249,55.14069213)
\curveto(645.27978582,54.71402546)(644.37311915,54.40469213)(643.43445249,54.21269213)
\curveto(642.49578582,53.99935879)(641.35445249,53.89269213)(640.01045249,53.89269213)
\curveto(638.26111915,53.89269213)(636.71445249,54.21269213)(635.37045249,54.85269213)
\curveto(634.02645249,55.51402546)(632.97045249,56.49535879)(632.20245249,57.79669213)
\curveto(631.43445249,59.11935879)(631.05045249,60.79402546)(631.05045249,62.82069213)
\curveto(631.05045249,64.84735879)(631.39178582,66.54335879)(632.07445249,67.90869213)
\curveto(632.77845249,69.27402546)(633.74911915,70.29802546)(634.98645249,70.98069213)
\curveto(636.22378582,71.66335879)(637.66378582,72.00469213)(639.30645249,72.00469213)
\closepath
\moveto(639.33845249,68.61269213)
\curveto(638.39978582,68.61269213)(637.63178582,68.31402546)(637.03445249,67.71669213)
\curveto(636.43711915,67.11935879)(636.08511915,66.19135879)(635.97845249,64.93269213)
\lineto(642.66645249,64.93269213)
\curveto(642.64511915,65.97802546)(642.35711915,66.85269213)(641.80245249,67.55669213)
\curveto(641.26911915,68.26069213)(640.44778582,68.61269213)(639.33845249,68.61269213)
\closepath
}
}
{
\newrgbcolor{curcolor}{0 0 0}
\pscustom[linestyle=none,fillstyle=solid,fillcolor=curcolor]
{
\newpath
\moveto(655.62643979,71.68469213)
\lineto(655.62643979,64.77269213)
\curveto(655.62643979,64.41002546)(655.60510646,63.96202546)(655.56243979,63.42869213)
\curveto(655.54110646,62.89535879)(655.50910646,62.35135879)(655.46643979,61.79669213)
\curveto(655.44510646,61.24202546)(655.41310646,60.74069213)(655.37043979,60.29269213)
\curveto(655.32777313,59.86602546)(655.29577313,59.57802546)(655.27443979,59.42869213)
\lineto(663.33843979,71.68469213)
\lineto(669.06643979,71.68469213)
\lineto(669.06643979,54.21269213)
\lineto(664.45843979,54.21269213)
\lineto(664.45843979,61.18869213)
\curveto(664.45843979,61.74335879)(664.47977313,62.37269213)(664.52243979,63.07669213)
\curveto(664.56510646,63.78069213)(664.60777313,64.43135879)(664.65043979,65.02869213)
\curveto(664.71443979,65.64735879)(664.75710646,66.11669213)(664.77843979,66.43669213)
\lineto(656.74643979,54.21269213)
\lineto(651.01843979,54.21269213)
\lineto(651.01843979,71.68469213)
\closepath
\moveto(667.75443979,79.20469213)
\curveto(667.64777313,78.09535879)(667.33843979,77.11402546)(666.82643979,76.26069213)
\curveto(666.31443979,75.42869213)(665.51443979,74.77802546)(664.42643979,74.30869213)
\curveto(663.33843979,73.83935879)(661.87710646,73.60469213)(660.04243979,73.60469213)
\curveto(658.16510646,73.60469213)(656.69310646,73.82869213)(655.62643979,74.27669213)
\curveto(654.58110646,74.72469213)(653.83443979,75.36469213)(653.38643979,76.19669213)
\curveto(652.93843979,77.05002546)(652.67177313,78.05269213)(652.58643979,79.20469213)
\lineto(656.84243979,79.20469213)
\curveto(656.92777313,78.03135879)(657.21577313,77.25269213)(657.70643979,76.86869213)
\curveto(658.21843979,76.48469213)(659.02910646,76.29269213)(660.13843979,76.29269213)
\curveto(661.05577313,76.29269213)(661.80243979,76.49535879)(662.37843979,76.90069213)
\curveto(662.97577313,77.32735879)(663.32777313,78.09535879)(663.43443979,79.20469213)
\closepath
}
}
{
\newrgbcolor{curcolor}{0 0 0}
\pscustom[linestyle=none,fillstyle=solid,fillcolor=curcolor]
{
\newpath
\moveto(173.77868838,174.84189288)
\curveto(171.92268838,174.84189288)(170.50402172,174.14855954)(169.52268838,172.76189288)
\curveto(168.54135505,171.37522621)(168.05068838,169.47655954)(168.05068838,167.06589288)
\curveto(168.05068838,164.63389288)(168.49868838,162.74589288)(169.39468838,161.40189288)
\curveto(170.31202172,160.07922621)(171.77335505,159.41789288)(173.77868838,159.41789288)
\curveto(174.69602172,159.41789288)(175.62402172,159.52455954)(176.56268838,159.73789288)
\curveto(177.50135505,159.95122621)(178.51468838,160.24989288)(179.60268838,160.63389288)
\lineto(179.60268838,156.56989288)
\curveto(178.60002172,156.16455954)(177.60802172,155.86589288)(176.62668838,155.67389288)
\curveto(175.64535505,155.48189288)(174.54668838,155.38589288)(173.33068838,155.38589288)
\curveto(170.96268838,155.38589288)(169.02135505,155.86589288)(167.50668838,156.82589288)
\curveto(165.99202172,157.80722621)(164.87202172,159.17255954)(164.14668838,160.92189288)
\curveto(163.42135505,162.69255954)(163.05868838,164.75122621)(163.05868838,167.09789288)
\curveto(163.05868838,169.40189288)(163.47468838,171.43922621)(164.30668838,173.20989288)
\curveto(165.13868838,174.98055954)(166.34402172,176.36722621)(167.92268838,177.36989288)
\curveto(169.52268838,178.37255954)(171.47468838,178.87389288)(173.77868838,178.87389288)
\curveto(174.90935505,178.87389288)(176.04002172,178.72455954)(177.17068838,178.42589288)
\curveto(178.32268838,178.14855954)(179.42135505,177.76455954)(180.46668838,177.27389288)
\lineto(178.89868838,173.33789288)
\curveto(178.04535505,173.74322621)(177.18135505,174.09522621)(176.30668838,174.39389288)
\curveto(175.45335505,174.69255954)(174.61068838,174.84189288)(173.77868838,174.84189288)
\closepath
}
}
{
\newrgbcolor{curcolor}{0 0 0}
\pscustom[linestyle=none,fillstyle=solid,fillcolor=curcolor]
{
\newpath
\moveto(199.95466055,173.17789288)
\lineto(199.95466055,155.70589288)
\lineto(195.18666055,155.70589288)
\lineto(195.18666055,169.59389288)
\lineto(188.85066055,169.59389288)
\lineto(188.85066055,155.70589288)
\lineto(184.08266055,155.70589288)
\lineto(184.08266055,173.17789288)
\closepath
}
}
{
\newrgbcolor{curcolor}{0 0 0}
\pscustom[linestyle=none,fillstyle=solid,fillcolor=curcolor]
{
\newpath
\moveto(209.55467422,173.17789288)
\lineto(209.55467422,166.26589288)
\curveto(209.55467422,165.90322621)(209.53334089,165.45522621)(209.49067422,164.92189288)
\curveto(209.46934089,164.38855954)(209.43734089,163.84455954)(209.39467422,163.28989288)
\curveto(209.37334089,162.73522621)(209.34134089,162.23389288)(209.29867422,161.78589288)
\curveto(209.25600756,161.35922621)(209.22400756,161.07122621)(209.20267422,160.92189288)
\lineto(217.26667422,173.17789288)
\lineto(222.99467422,173.17789288)
\lineto(222.99467422,155.70589288)
\lineto(218.38667422,155.70589288)
\lineto(218.38667422,162.68189288)
\curveto(218.38667422,163.23655954)(218.40800756,163.86589288)(218.45067422,164.56989288)
\curveto(218.49334089,165.27389288)(218.53600756,165.92455954)(218.57867422,166.52189288)
\curveto(218.64267422,167.14055954)(218.68534089,167.60989288)(218.70667422,167.92989288)
\lineto(210.67467422,155.70589288)
\lineto(204.94667422,155.70589288)
\lineto(204.94667422,173.17789288)
\closepath
}
}
{
\newrgbcolor{curcolor}{0 0 0}
\pscustom[linestyle=none,fillstyle=solid,fillcolor=curcolor]
{
\newpath
\moveto(235.09065225,155.38589288)
\curveto(232.48798558,155.38589288)(230.47198558,156.10055954)(229.04265225,157.52989288)
\curveto(227.63465225,158.95922621)(226.93065225,161.23122621)(226.93065225,164.34589288)
\curveto(226.93065225,166.47922621)(227.29331892,168.21789288)(228.01865225,169.56189288)
\curveto(228.74398558,170.90589288)(229.74665225,171.89789288)(231.02665225,172.53789288)
\curveto(232.32798558,173.17789288)(233.82131892,173.49789288)(235.50665225,173.49789288)
\curveto(236.70131892,173.49789288)(237.73598558,173.38055954)(238.61065225,173.14589288)
\curveto(239.50665225,172.91122621)(240.28531892,172.63389288)(240.94665225,172.31389288)
\lineto(239.53865225,168.63389288)
\curveto(238.79198558,168.93255954)(238.08798558,169.17789288)(237.42665225,169.36989288)
\curveto(236.78665225,169.56189288)(236.14665225,169.65789288)(235.50665225,169.65789288)
\curveto(233.03198558,169.65789288)(231.79465225,167.89789288)(231.79465225,164.37789288)
\curveto(231.79465225,162.62855954)(232.11465225,161.33789288)(232.75465225,160.50589288)
\curveto(233.41598558,159.67389288)(234.33331892,159.25789288)(235.50665225,159.25789288)
\curveto(236.50931892,159.25789288)(237.39465225,159.38589288)(238.16265225,159.64189288)
\curveto(238.93065225,159.91922621)(239.67731892,160.29255954)(240.40265225,160.76189288)
\lineto(240.40265225,156.69789288)
\curveto(239.67731892,156.22855954)(238.90931892,155.89789288)(238.09865225,155.70589288)
\curveto(237.30931892,155.49255954)(236.30665225,155.38589288)(235.09065225,155.38589288)
\closepath
}
}
{
\newrgbcolor{curcolor}{0 0 0}
\pscustom[linestyle=none,fillstyle=solid,fillcolor=curcolor]
{
\newpath
\moveto(260.3066503,164.47389288)
\curveto(260.3066503,161.57255954)(259.5386503,159.33255954)(258.0026503,157.75389288)
\curveto(256.48798363,156.17522621)(254.4186503,155.38589288)(251.7946503,155.38589288)
\curveto(250.17331696,155.38589288)(248.7226503,155.73789288)(247.4426503,156.44189288)
\curveto(246.18398363,157.14589288)(245.19198363,158.16989288)(244.4666503,159.51389288)
\curveto(243.74131696,160.87922621)(243.3786503,162.53255954)(243.3786503,164.47389288)
\curveto(243.3786503,167.37522621)(244.13598363,169.60455954)(245.6506503,171.16189288)
\curveto(247.16531696,172.71922621)(249.24531696,173.49789288)(251.8906503,173.49789288)
\curveto(253.53331696,173.49789288)(254.98398363,173.14589288)(256.2426503,172.44189288)
\curveto(257.50131696,171.73789288)(258.49331696,170.71389288)(259.2186503,169.36989288)
\curveto(259.94398363,168.02589288)(260.3066503,166.39389288)(260.3066503,164.47389288)
\closepath
\moveto(248.2426503,164.47389288)
\curveto(248.2426503,162.74589288)(248.51998363,161.43389288)(249.0746503,160.53789288)
\curveto(249.6506503,159.66322621)(250.5786503,159.22589288)(251.8586503,159.22589288)
\curveto(253.11731696,159.22589288)(254.02398363,159.66322621)(254.5786503,160.53789288)
\curveto(255.1546503,161.43389288)(255.4426503,162.74589288)(255.4426503,164.47389288)
\curveto(255.4426503,166.20189288)(255.1546503,167.49255954)(254.5786503,168.34589288)
\curveto(254.02398363,169.22055954)(253.1066503,169.65789288)(251.8266503,169.65789288)
\curveto(250.56798363,169.65789288)(249.6506503,169.22055954)(249.0746503,168.34589288)
\curveto(248.51998363,167.49255954)(248.2426503,166.20189288)(248.2426503,164.47389288)
\closepath
}
}
{
\newrgbcolor{curcolor}{0 0 0}
\pscustom[linestyle=none,fillstyle=solid,fillcolor=curcolor]
{
\newpath
\moveto(275.69863369,173.17789288)
\lineto(280.94663369,173.17789288)
\lineto(274.03463369,164.79389288)
\lineto(281.55463369,155.70589288)
\lineto(276.14663369,155.70589288)
\lineto(269.01063369,164.56989288)
\lineto(269.01063369,155.70589288)
\lineto(264.24263369,155.70589288)
\lineto(264.24263369,173.17789288)
\lineto(269.01063369,173.17789288)
\lineto(269.01063369,164.69789288)
\closepath
}
}
{
\newrgbcolor{curcolor}{0 0 0}
\pscustom[linestyle=none,fillstyle=solid,fillcolor=curcolor]
{
\newpath
\moveto(123.37868326,133.17789288)
\lineto(128.59468326,133.17789288)
\lineto(131.89068326,123.35389288)
\curveto(132.06134992,122.86322621)(132.18934992,122.37255954)(132.27468326,121.88189288)
\curveto(132.36001659,121.39122621)(132.42401659,120.86855954)(132.46668326,120.31389288)
\lineto(132.56268326,120.31389288)
\curveto(132.62668326,120.86855954)(132.71201659,121.39122621)(132.81868326,121.88189288)
\curveto(132.92534992,122.37255954)(133.06401659,122.86322621)(133.23468326,123.35389288)
\lineto(136.46668326,133.17789288)
\lineto(141.58668326,133.17789288)
\lineto(134.19468326,113.46589288)
\curveto(133.51201659,111.65255954)(132.54134992,110.29789288)(131.28268326,109.40189288)
\curveto(130.02401659,108.48455954)(128.56268326,108.02589288)(126.89868326,108.02589288)
\curveto(126.34401659,108.02589288)(125.87468326,108.05789288)(125.49068326,108.12189288)
\curveto(125.10668326,108.16455954)(124.76534992,108.21789288)(124.46668326,108.28189288)
\lineto(124.46668326,112.05789288)
\curveto(124.68001659,112.01522621)(124.95734992,111.97255954)(125.29868326,111.92989288)
\curveto(125.64001659,111.88722621)(125.99201659,111.86589288)(126.35468326,111.86589288)
\curveto(127.35734992,111.86589288)(128.14668326,112.17522621)(128.72268326,112.79389288)
\curveto(129.29868326,113.39122621)(129.73601659,114.11655954)(130.03468326,114.96989288)
\lineto(130.32268326,115.83389288)
\closepath
}
}
{
\newrgbcolor{curcolor}{0 0 0}
\pscustom[linestyle=none,fillstyle=solid,fillcolor=curcolor]
{
\newpath
\moveto(148.27469107,133.17789288)
\lineto(148.27469107,126.77789288)
\curveto(148.27469107,125.26322621)(148.97869107,124.50589288)(150.38669107,124.50589288)
\curveto(151.3040244,124.50589288)(152.15735773,124.60189288)(152.94669107,124.79389288)
\curveto(153.7360244,125.00722621)(154.52535773,125.28455954)(155.31469107,125.62589288)
\lineto(155.31469107,133.17789288)
\lineto(160.08269107,133.17789288)
\lineto(160.08269107,115.70589288)
\lineto(155.31469107,115.70589288)
\lineto(155.31469107,122.64989288)
\curveto(154.5680244,122.24455954)(153.71469107,121.87122621)(152.75469107,121.52989288)
\curveto(151.79469107,121.20989288)(150.70669107,121.04989288)(149.49069107,121.04989288)
\curveto(147.67735773,121.04989288)(146.22669107,121.50855954)(145.13869107,122.42589288)
\curveto(144.05069107,123.36455954)(143.50669107,124.78322621)(143.50669107,126.68189288)
\lineto(143.50669107,133.17789288)
\closepath
}
}
{
\newrgbcolor{curcolor}{0 0 0}
\pscustom[linestyle=none,fillstyle=solid,fillcolor=curcolor]
{
\newpath
\moveto(172.24269546,133.52989288)
\curveto(174.58936213,133.52989288)(176.38136213,133.01789288)(177.61869546,131.99389288)
\curveto(178.87736213,130.99122621)(179.50669546,129.44455954)(179.50669546,127.35389288)
\lineto(179.50669546,115.70589288)
\lineto(176.17869546,115.70589288)
\lineto(175.25069546,118.07389288)
\lineto(175.12269546,118.07389288)
\curveto(174.3760288,117.13522621)(173.58669546,116.45255954)(172.75469546,116.02589288)
\curveto(171.92269546,115.59922621)(170.78136213,115.38589288)(169.33069546,115.38589288)
\curveto(167.77336213,115.38589288)(166.48269546,115.83389288)(165.45869546,116.72989288)
\curveto(164.43469546,117.62589288)(163.92269546,119.02322621)(163.92269546,120.92189288)
\curveto(163.92269546,122.77789288)(164.57336213,124.14322621)(165.87469546,125.01789288)
\curveto(167.1760288,125.89255954)(169.1280288,126.38322621)(171.73069546,126.48989288)
\lineto(174.77069546,126.58589288)
\lineto(174.77069546,127.35389288)
\curveto(174.77069546,128.27122621)(174.52536213,128.94322621)(174.03469546,129.36989288)
\curveto(173.56536213,129.79655954)(172.9040288,130.00989288)(172.05069546,130.00989288)
\curveto(171.19736213,130.00989288)(170.36536213,129.88189288)(169.55469546,129.62589288)
\curveto(168.7440288,129.39122621)(167.93336213,129.09255954)(167.12269546,128.72989288)
\lineto(165.55469546,131.96189288)
\curveto(166.4720288,132.43122621)(167.50669546,132.80455954)(168.65869546,133.08189288)
\curveto(169.81069546,133.38055954)(171.00536213,133.52989288)(172.24269546,133.52989288)
\closepath
\moveto(174.77069546,123.80189288)
\lineto(172.91469546,123.73789288)
\curveto(171.37869546,123.69522621)(170.3120288,123.41789288)(169.71469546,122.90589288)
\curveto(169.11736213,122.39389288)(168.81869546,121.72189288)(168.81869546,120.88989288)
\curveto(168.81869546,120.16455954)(169.0320288,119.64189288)(169.45869546,119.32189288)
\curveto(169.88536213,119.02322621)(170.4400288,118.87389288)(171.12269546,118.87389288)
\curveto(172.14669546,118.87389288)(173.01069546,119.17255954)(173.71469546,119.76989288)
\curveto(174.41869546,120.38855954)(174.77069546,121.25255954)(174.77069546,122.36189288)
\closepath
}
}
{
\newrgbcolor{curcolor}{0 0 0}
\pscustom[linestyle=none,fillstyle=solid,fillcolor=curcolor]
{
\newpath
\moveto(191.50669839,115.38589288)
\curveto(188.90403173,115.38589288)(186.88803173,116.10055954)(185.45869839,117.52989288)
\curveto(184.05069839,118.95922621)(183.34669839,121.23122621)(183.34669839,124.34589288)
\curveto(183.34669839,126.47922621)(183.70936506,128.21789288)(184.43469839,129.56189288)
\curveto(185.16003173,130.90589288)(186.16269839,131.89789288)(187.44269839,132.53789288)
\curveto(188.74403173,133.17789288)(190.23736506,133.49789288)(191.92269839,133.49789288)
\curveto(193.11736506,133.49789288)(194.15203173,133.38055954)(195.02669839,133.14589288)
\curveto(195.92269839,132.91122621)(196.70136506,132.63389288)(197.36269839,132.31389288)
\lineto(195.95469839,128.63389288)
\curveto(195.20803173,128.93255954)(194.50403173,129.17789288)(193.84269839,129.36989288)
\curveto(193.20269839,129.56189288)(192.56269839,129.65789288)(191.92269839,129.65789288)
\curveto(189.44803173,129.65789288)(188.21069839,127.89789288)(188.21069839,124.37789288)
\curveto(188.21069839,122.62855954)(188.53069839,121.33789288)(189.17069839,120.50589288)
\curveto(189.83203173,119.67389288)(190.74936506,119.25789288)(191.92269839,119.25789288)
\curveto(192.92536506,119.25789288)(193.81069839,119.38589288)(194.57869839,119.64189288)
\curveto(195.34669839,119.91922621)(196.09336506,120.29255954)(196.81869839,120.76189288)
\lineto(196.81869839,116.69789288)
\curveto(196.09336506,116.22855954)(195.32536506,115.89789288)(194.51469839,115.70589288)
\curveto(193.72536506,115.49255954)(192.72269839,115.38589288)(191.50669839,115.38589288)
\closepath
}
}
{
\newrgbcolor{curcolor}{0 0 0}
\pscustom[linestyle=none,fillstyle=solid,fillcolor=curcolor]
{
\newpath
\moveto(215.31469644,129.59389288)
\lineto(209.58669644,129.59389288)
\lineto(209.58669644,115.70589288)
\lineto(204.81869644,115.70589288)
\lineto(204.81869644,129.59389288)
\lineto(199.09069644,129.59389288)
\lineto(199.09069644,133.17789288)
\lineto(215.31469644,133.17789288)
\closepath
}
}
{
\newrgbcolor{curcolor}{0 0 0}
\pscustom[linestyle=none,fillstyle=solid,fillcolor=curcolor]
{
\newpath
\moveto(223.31468033,133.17789288)
\lineto(223.31468033,126.45789288)
\lineto(229.97068033,126.45789288)
\lineto(229.97068033,133.17789288)
\lineto(234.73868033,133.17789288)
\lineto(234.73868033,115.70589288)
\lineto(229.97068033,115.70589288)
\lineto(229.97068033,122.90589288)
\lineto(223.31468033,122.90589288)
\lineto(223.31468033,115.70589288)
\lineto(218.54668033,115.70589288)
\lineto(218.54668033,133.17789288)
\closepath
}
}
{
\newrgbcolor{curcolor}{0 0 0}
\pscustom[linestyle=none,fillstyle=solid,fillcolor=curcolor]
{
\newpath
\moveto(244.33870132,133.17789288)
\lineto(244.33870132,126.26589288)
\curveto(244.33870132,125.90322621)(244.31736799,125.45522621)(244.27470132,124.92189288)
\curveto(244.25336799,124.38855954)(244.22136799,123.84455954)(244.17870132,123.28989288)
\curveto(244.15736799,122.73522621)(244.12536799,122.23389288)(244.08270132,121.78589288)
\curveto(244.04003465,121.35922621)(244.00803465,121.07122621)(243.98670132,120.92189288)
\lineto(252.05070132,133.17789288)
\lineto(257.77870132,133.17789288)
\lineto(257.77870132,115.70589288)
\lineto(253.17070132,115.70589288)
\lineto(253.17070132,122.68189288)
\curveto(253.17070132,123.23655954)(253.19203465,123.86589288)(253.23470132,124.56989288)
\curveto(253.27736799,125.27389288)(253.32003465,125.92455954)(253.36270132,126.52189288)
\curveto(253.42670132,127.14055954)(253.46936799,127.60989288)(253.49070132,127.92989288)
\lineto(245.45870132,115.70589288)
\lineto(239.73070132,115.70589288)
\lineto(239.73070132,133.17789288)
\closepath
}
}
{
\newrgbcolor{curcolor}{0 0 0}
\pscustom[linestyle=none,fillstyle=solid,fillcolor=curcolor]
{
\newpath
\moveto(274.22667935,133.17789288)
\lineto(279.47467935,133.17789288)
\lineto(272.56267935,124.79389288)
\lineto(280.08267935,115.70589288)
\lineto(274.67467935,115.70589288)
\lineto(267.53867935,124.56989288)
\lineto(267.53867935,115.70589288)
\lineto(262.77067935,115.70589288)
\lineto(262.77067935,133.17789288)
\lineto(267.53867935,133.17789288)
\lineto(267.53867935,124.69789288)
\closepath
}
}
{
\newrgbcolor{curcolor}{0 0 0}
\pscustom[linestyle=none,fillstyle=solid,fillcolor=curcolor]
{
\newpath
\moveto(297.8106481,124.47389288)
\curveto(297.8106481,121.57255954)(297.0426481,119.33255954)(295.5066481,117.75389288)
\curveto(293.99198143,116.17522621)(291.9226481,115.38589288)(289.2986481,115.38589288)
\curveto(287.67731477,115.38589288)(286.2266481,115.73789288)(284.9466481,116.44189288)
\curveto(283.68798143,117.14589288)(282.69598143,118.16989288)(281.9706481,119.51389288)
\curveto(281.24531477,120.87922621)(280.8826481,122.53255954)(280.8826481,124.47389288)
\curveto(280.8826481,127.37522621)(281.63998143,129.60455954)(283.1546481,131.16189288)
\curveto(284.66931477,132.71922621)(286.74931477,133.49789288)(289.3946481,133.49789288)
\curveto(291.03731477,133.49789288)(292.48798143,133.14589288)(293.7466481,132.44189288)
\curveto(295.00531477,131.73789288)(295.99731477,130.71389288)(296.7226481,129.36989288)
\curveto(297.44798143,128.02589288)(297.8106481,126.39389288)(297.8106481,124.47389288)
\closepath
\moveto(285.7466481,124.47389288)
\curveto(285.7466481,122.74589288)(286.02398143,121.43389288)(286.5786481,120.53789288)
\curveto(287.1546481,119.66322621)(288.0826481,119.22589288)(289.3626481,119.22589288)
\curveto(290.62131477,119.22589288)(291.52798143,119.66322621)(292.0826481,120.53789288)
\curveto(292.6586481,121.43389288)(292.9466481,122.74589288)(292.9466481,124.47389288)
\curveto(292.9466481,126.20189288)(292.6586481,127.49255954)(292.0826481,128.34589288)
\curveto(291.52798143,129.22055954)(290.6106481,129.65789288)(289.3306481,129.65789288)
\curveto(288.07198143,129.65789288)(287.1546481,129.22055954)(286.5786481,128.34589288)
\curveto(286.02398143,127.49255954)(285.7466481,126.20189288)(285.7466481,124.47389288)
\closepath
}
}
{
\newrgbcolor{curcolor}{0 0 0}
\pscustom[linestyle=none,fillstyle=solid,fillcolor=curcolor]
{
\newpath
\moveto(317.2346315,128.60189288)
\curveto(317.2346315,127.66322621)(316.93596483,126.86322621)(316.3386315,126.20189288)
\curveto(315.7626315,125.54055954)(314.8986315,125.11389288)(313.7466315,124.92189288)
\lineto(313.7466315,124.79389288)
\curveto(314.9626315,124.64455954)(315.93329816,124.21789288)(316.6586315,123.51389288)
\curveto(317.40529816,122.83122621)(317.7786315,121.96722621)(317.7786315,120.92189288)
\curveto(317.7786315,119.91922621)(317.51196483,119.02322621)(316.9786315,118.23389288)
\curveto(316.4666315,117.44455954)(315.64529816,116.82589288)(314.5146315,116.37789288)
\curveto(313.38396483,115.92989288)(311.90129816,115.70589288)(310.0666315,115.70589288)
\lineto(301.7466315,115.70589288)
\lineto(301.7466315,133.17789288)
\lineto(310.0666315,133.17789288)
\curveto(311.43196483,133.17789288)(312.64796483,133.02855954)(313.7146315,132.72989288)
\curveto(314.8026315,132.45255954)(315.65596483,131.97255954)(316.2746315,131.28989288)
\curveto(316.9146315,130.62855954)(317.2346315,129.73255954)(317.2346315,128.60189288)
\closepath
\moveto(312.4026315,128.21789288)
\curveto(312.4026315,129.28455954)(311.55996483,129.81789288)(309.8746315,129.81789288)
\lineto(306.5146315,129.81789288)
\lineto(306.5146315,126.36189288)
\lineto(309.3306315,126.36189288)
\curveto(310.33329816,126.36189288)(311.0906315,126.50055954)(311.6026315,126.77789288)
\curveto(312.13596483,127.07655954)(312.4026315,127.55655954)(312.4026315,128.21789288)
\closepath
\moveto(312.8506315,121.17789288)
\curveto(312.8506315,121.86055954)(312.57329816,122.35122621)(312.0186315,122.64989288)
\curveto(311.48529816,122.96989288)(310.69596483,123.12989288)(309.6506315,123.12989288)
\lineto(306.5146315,123.12989288)
\lineto(306.5146315,119.00189288)
\lineto(309.7466315,119.00189288)
\curveto(310.6426315,119.00189288)(311.3786315,119.16189288)(311.9546315,119.48189288)
\curveto(312.55196483,119.82322621)(312.8506315,120.38855954)(312.8506315,121.17789288)
\closepath
}
}
{
\newrgbcolor{curcolor}{0 0 0}
\pscustom[linestyle=none,fillstyle=solid,fillcolor=curcolor]
{
\newpath
\moveto(159.63466885,93.17789288)
\lineto(164.88266885,93.17789288)
\lineto(157.97066885,84.79389288)
\lineto(165.49066885,75.70589288)
\lineto(160.08266885,75.70589288)
\lineto(152.94666885,84.56989288)
\lineto(152.94666885,75.70589288)
\lineto(148.17866885,75.70589288)
\lineto(148.17866885,93.17789288)
\lineto(152.94666885,93.17789288)
\lineto(152.94666885,84.69789288)
\closepath
}
}
{
\newrgbcolor{curcolor}{0 0 0}
\pscustom[linestyle=none,fillstyle=solid,fillcolor=curcolor]
{
\newpath
\moveto(183.2186376,84.47389288)
\curveto(183.2186376,81.57255954)(182.4506376,79.33255954)(180.9146376,77.75389288)
\curveto(179.39997093,76.17522621)(177.3306376,75.38589288)(174.7066376,75.38589288)
\curveto(173.08530427,75.38589288)(171.6346376,75.73789288)(170.3546376,76.44189288)
\curveto(169.09597093,77.14589288)(168.10397093,78.16989288)(167.3786376,79.51389288)
\curveto(166.65330427,80.87922621)(166.2906376,82.53255954)(166.2906376,84.47389288)
\curveto(166.2906376,87.37522621)(167.04797093,89.60455954)(168.5626376,91.16189288)
\curveto(170.07730427,92.71922621)(172.15730427,93.49789288)(174.8026376,93.49789288)
\curveto(176.44530427,93.49789288)(177.89597093,93.14589288)(179.1546376,92.44189288)
\curveto(180.41330427,91.73789288)(181.40530427,90.71389288)(182.1306376,89.36989288)
\curveto(182.85597093,88.02589288)(183.2186376,86.39389288)(183.2186376,84.47389288)
\closepath
\moveto(171.1546376,84.47389288)
\curveto(171.1546376,82.74589288)(171.43197093,81.43389288)(171.9866376,80.53789288)
\curveto(172.5626376,79.66322621)(173.4906376,79.22589288)(174.7706376,79.22589288)
\curveto(176.02930427,79.22589288)(176.93597093,79.66322621)(177.4906376,80.53789288)
\curveto(178.0666376,81.43389288)(178.3546376,82.74589288)(178.3546376,84.47389288)
\curveto(178.3546376,86.20189288)(178.0666376,87.49255954)(177.4906376,88.34589288)
\curveto(176.93597093,89.22055954)(176.0186376,89.65789288)(174.7386376,89.65789288)
\curveto(173.47997093,89.65789288)(172.5626376,89.22055954)(171.9866376,88.34589288)
\curveto(171.43197093,87.49255954)(171.1546376,86.20189288)(171.1546376,84.47389288)
\closepath
}
}
{
\newrgbcolor{curcolor}{0 0 0}
\pscustom[linestyle=none,fillstyle=solid,fillcolor=curcolor]
{
\newpath
\moveto(209.170621,93.17789288)
\lineto(209.170621,75.70589288)
\lineto(204.722621,75.70589288)
\lineto(204.722621,84.28189288)
\curveto(204.722621,85.13522621)(204.73328767,85.96722621)(204.754621,86.77789288)
\curveto(204.79728767,87.58855954)(204.850621,88.33522621)(204.914621,89.01789288)
\lineto(204.818621,89.01789288)
\lineto(199.986621,75.70589288)
\lineto(196.402621,75.70589288)
\lineto(191.506621,89.04989288)
\lineto(191.378621,89.04989288)
\curveto(191.46395433,88.34589288)(191.51728767,87.58855954)(191.538621,86.77789288)
\curveto(191.58128767,85.98855954)(191.602621,85.11389288)(191.602621,84.15389288)
\lineto(191.602621,75.70589288)
\lineto(187.154621,75.70589288)
\lineto(187.154621,93.17789288)
\lineto(193.906621,93.17789288)
\lineto(198.258621,81.33789288)
\lineto(202.674621,93.17789288)
\closepath
}
}
{
\newrgbcolor{curcolor}{0 0 0}
\pscustom[linestyle=none,fillstyle=solid,fillcolor=curcolor]
{
\newpath
\moveto(218.9306166,93.17789288)
\lineto(218.9306166,86.45789288)
\lineto(225.5866166,86.45789288)
\lineto(225.5866166,93.17789288)
\lineto(230.3546166,93.17789288)
\lineto(230.3546166,75.70589288)
\lineto(225.5866166,75.70589288)
\lineto(225.5866166,82.90589288)
\lineto(218.9306166,82.90589288)
\lineto(218.9306166,75.70589288)
\lineto(214.1626166,75.70589288)
\lineto(214.1626166,93.17789288)
\closepath
}
}
{
\newrgbcolor{curcolor}{0 0 0}
\pscustom[linestyle=none,fillstyle=solid,fillcolor=curcolor]
{
\newpath
\moveto(242.5146376,93.52989288)
\curveto(244.86130427,93.52989288)(246.65330427,93.01789288)(247.8906376,91.99389288)
\curveto(249.14930427,90.99122621)(249.7786376,89.44455954)(249.7786376,87.35389288)
\lineto(249.7786376,75.70589288)
\lineto(246.4506376,75.70589288)
\lineto(245.5226376,78.07389288)
\lineto(245.3946376,78.07389288)
\curveto(244.64797093,77.13522621)(243.8586376,76.45255954)(243.0266376,76.02589288)
\curveto(242.1946376,75.59922621)(241.05330427,75.38589288)(239.6026376,75.38589288)
\curveto(238.04530427,75.38589288)(236.7546376,75.83389288)(235.7306376,76.72989288)
\curveto(234.7066376,77.62589288)(234.1946376,79.02322621)(234.1946376,80.92189288)
\curveto(234.1946376,82.77789288)(234.84530427,84.14322621)(236.1466376,85.01789288)
\curveto(237.44797093,85.89255954)(239.39997093,86.38322621)(242.0026376,86.48989288)
\lineto(245.0426376,86.58589288)
\lineto(245.0426376,87.35389288)
\curveto(245.0426376,88.27122621)(244.79730427,88.94322621)(244.3066376,89.36989288)
\curveto(243.83730427,89.79655954)(243.17597093,90.00989288)(242.3226376,90.00989288)
\curveto(241.46930427,90.00989288)(240.63730427,89.88189288)(239.8266376,89.62589288)
\curveto(239.01597093,89.39122621)(238.20530427,89.09255954)(237.3946376,88.72989288)
\lineto(235.8266376,91.96189288)
\curveto(236.74397093,92.43122621)(237.7786376,92.80455954)(238.9306376,93.08189288)
\curveto(240.0826376,93.38055954)(241.27730427,93.52989288)(242.5146376,93.52989288)
\closepath
\moveto(245.0426376,83.80189288)
\lineto(243.1866376,83.73789288)
\curveto(241.6506376,83.69522621)(240.58397093,83.41789288)(239.9866376,82.90589288)
\curveto(239.38930427,82.39389288)(239.0906376,81.72189288)(239.0906376,80.88989288)
\curveto(239.0906376,80.16455954)(239.30397093,79.64189288)(239.7306376,79.32189288)
\curveto(240.15730427,79.02322621)(240.71197093,78.87389288)(241.3946376,78.87389288)
\curveto(242.4186376,78.87389288)(243.2826376,79.17255954)(243.9866376,79.76989288)
\curveto(244.6906376,80.38855954)(245.0426376,81.25255954)(245.0426376,82.36189288)
\closepath
}
}
{
\newrgbcolor{curcolor}{0 0 0}
\pscustom[linestyle=none,fillstyle=solid,fillcolor=curcolor]
{
\newpath
\moveto(269.13864053,89.59389288)
\lineto(263.41064053,89.59389288)
\lineto(263.41064053,75.70589288)
\lineto(258.64264053,75.70589288)
\lineto(258.64264053,89.59389288)
\lineto(252.91464053,89.59389288)
\lineto(252.91464053,93.17789288)
\lineto(269.13864053,93.17789288)
\closepath
}
}
{
\newrgbcolor{curcolor}{0 0 0}
\pscustom[linestyle=none,fillstyle=solid,fillcolor=curcolor]
{
\newpath
\moveto(272.37062442,75.70589288)
\lineto(272.37062442,93.17789288)
\lineto(277.13862442,93.17789288)
\lineto(277.13862442,86.42589288)
\lineto(279.44262442,86.42589288)
\curveto(282.10929108,86.42589288)(284.08262442,85.99922621)(285.36262442,85.14589288)
\curveto(286.64262442,84.29255954)(287.28262442,83.00189288)(287.28262442,81.27389288)
\curveto(287.28262442,79.56722621)(286.68529108,78.21255954)(285.49062442,77.20989288)
\curveto(284.29595775,76.20722621)(282.33329108,75.70589288)(279.60262442,75.70589288)
\closepath
\moveto(289.81062442,75.70589288)
\lineto(289.81062442,93.17789288)
\lineto(294.57862442,93.17789288)
\lineto(294.57862442,75.70589288)
\closepath
\moveto(277.13862442,79.00189288)
\lineto(279.34662442,79.00189288)
\curveto(280.28529108,79.00189288)(281.04262442,79.16189288)(281.61862442,79.48189288)
\curveto(282.21595775,79.82322621)(282.51462442,80.39922621)(282.51462442,81.20989288)
\curveto(282.51462442,82.48989288)(281.43729108,83.12989288)(279.28262442,83.12989288)
\lineto(277.13862442,83.12989288)
\closepath
}
}
{
\newrgbcolor{curcolor}{0 0 0}
\pscustom[linestyle=none,fillstyle=solid,fillcolor=curcolor]
{
\newpath
\moveto(105.16816812,417.92023)
\lineto(105.16816812,440.76823)
\lineto(123.24816812,440.76823)
\lineto(123.24816812,417.92023)
\lineto(118.41616812,417.92023)
\lineto(118.41616812,436.73623)
\lineto(110.00016812,436.73623)
\lineto(110.00016812,417.92023)
\closepath
}
}
{
\newrgbcolor{curcolor}{0 0 0}
\pscustom[linestyle=none,fillstyle=solid,fillcolor=curcolor]
{
\newpath
\moveto(135.79219497,435.74423)
\curveto(138.13886164,435.74423)(139.93086164,435.23223)(141.16819497,434.20823)
\curveto(142.42686164,433.20556333)(143.05619497,431.65889667)(143.05619497,429.56823)
\lineto(143.05619497,417.92023)
\lineto(139.72819497,417.92023)
\lineto(138.80019497,420.28823)
\lineto(138.67219497,420.28823)
\curveto(137.92552831,419.34956333)(137.13619497,418.66689667)(136.30419497,418.24023)
\curveto(135.47219497,417.81356333)(134.33086164,417.60023)(132.88019497,417.60023)
\curveto(131.32286164,417.60023)(130.03219497,418.04823)(129.00819497,418.94423)
\curveto(127.98419497,419.84023)(127.47219497,421.23756333)(127.47219497,423.13623)
\curveto(127.47219497,424.99223)(128.12286164,426.35756333)(129.42419497,427.23223)
\curveto(130.72552831,428.10689667)(132.67752831,428.59756333)(135.28019497,428.70423)
\lineto(138.32019497,428.80023)
\lineto(138.32019497,429.56823)
\curveto(138.32019497,430.48556333)(138.07486164,431.15756333)(137.58419497,431.58423)
\curveto(137.11486164,432.01089667)(136.45352831,432.22423)(135.60019497,432.22423)
\curveto(134.74686164,432.22423)(133.91486164,432.09623)(133.10419497,431.84023)
\curveto(132.29352831,431.60556333)(131.48286164,431.30689667)(130.67219497,430.94423)
\lineto(129.10419497,434.17623)
\curveto(130.02152831,434.64556333)(131.05619497,435.01889667)(132.20819497,435.29623)
\curveto(133.36019497,435.59489667)(134.55486164,435.74423)(135.79219497,435.74423)
\closepath
\moveto(138.32019497,426.01623)
\lineto(136.46419497,425.95223)
\curveto(134.92819497,425.90956333)(133.86152831,425.63223)(133.26419497,425.12023)
\curveto(132.66686164,424.60823)(132.36819497,423.93623)(132.36819497,423.10423)
\curveto(132.36819497,422.37889667)(132.58152831,421.85623)(133.00819497,421.53623)
\curveto(133.43486164,421.23756333)(133.98952831,421.08823)(134.67219497,421.08823)
\curveto(135.69619497,421.08823)(136.56019497,421.38689667)(137.26419497,421.98423)
\curveto(137.96819497,422.60289667)(138.32019497,423.46689667)(138.32019497,424.57623)
\closepath
}
}
{
\newrgbcolor{curcolor}{0 0 0}
\pscustom[linestyle=none,fillstyle=solid,fillcolor=curcolor]
{
\newpath
\moveto(152.7201979,435.39223)
\lineto(152.7201979,428.67223)
\lineto(159.3761979,428.67223)
\lineto(159.3761979,435.39223)
\lineto(164.1441979,435.39223)
\lineto(164.1441979,417.92023)
\lineto(159.3761979,417.92023)
\lineto(159.3761979,425.12023)
\lineto(152.7201979,425.12023)
\lineto(152.7201979,417.92023)
\lineto(147.9521979,417.92023)
\lineto(147.9521979,435.39223)
\closepath
}
}
{
\newrgbcolor{curcolor}{0 0 0}
\pscustom[linestyle=none,fillstyle=solid,fillcolor=curcolor]
{
\newpath
\moveto(176.3362189,435.71223)
\curveto(178.74688557,435.71223)(180.6562189,435.01889667)(182.0642189,433.63223)
\curveto(183.4722189,432.26689667)(184.1762189,430.31489667)(184.1762189,427.77623)
\lineto(184.1762189,425.47223)
\lineto(172.9122189,425.47223)
\curveto(172.95488557,424.12823)(173.34955223,423.07223)(174.0962189,422.30423)
\curveto(174.8642189,421.53623)(175.9202189,421.15223)(177.2642189,421.15223)
\curveto(178.37355223,421.15223)(179.38688557,421.25889667)(180.3042189,421.47223)
\curveto(181.24288557,421.70689667)(182.20288557,422.05889667)(183.1842189,422.52823)
\lineto(183.1842189,418.84823)
\curveto(182.30955223,418.42156333)(181.40288557,418.11223)(180.4642189,417.92023)
\curveto(179.52555223,417.70689667)(178.3842189,417.60023)(177.0402189,417.60023)
\curveto(175.29088557,417.60023)(173.7442189,417.92023)(172.4002189,418.56023)
\curveto(171.0562189,419.22156333)(170.0002189,420.20289667)(169.2322189,421.50423)
\curveto(168.4642189,422.82689667)(168.0802189,424.50156333)(168.0802189,426.52823)
\curveto(168.0802189,428.55489667)(168.42155223,430.25089667)(169.1042189,431.61623)
\curveto(169.8082189,432.98156333)(170.77888557,434.00556333)(172.0162189,434.68823)
\curveto(173.25355223,435.37089667)(174.69355223,435.71223)(176.3362189,435.71223)
\closepath
\moveto(176.3682189,432.32023)
\curveto(175.42955223,432.32023)(174.66155223,432.02156333)(174.0642189,431.42423)
\curveto(173.46688557,430.82689667)(173.11488557,429.89889667)(173.0082189,428.64023)
\lineto(179.6962189,428.64023)
\curveto(179.67488557,429.68556333)(179.38688557,430.56023)(178.8322189,431.26423)
\curveto(178.29888557,431.96823)(177.47755223,432.32023)(176.3682189,432.32023)
\closepath
}
}
{
\newrgbcolor{curcolor}{0 0 0}
\pscustom[linestyle=none,fillstyle=solid,fillcolor=curcolor]
{
\newpath
\moveto(203.6962062,417.92023)
\lineto(198.9282062,417.92023)
\lineto(198.9282062,431.80823)
\lineto(194.5442062,431.80823)
\curveto(194.26687287,428.39489667)(193.89353954,425.64289667)(193.4242062,423.55223)
\curveto(192.9762062,421.48289667)(192.3362062,419.96823)(191.5042062,419.00823)
\curveto(190.69353954,418.06956333)(189.6162062,417.60023)(188.2722062,417.60023)
\curveto(187.16287287,417.60023)(186.2562062,417.77089667)(185.5522062,418.11223)
\lineto(185.5522062,421.92023)
\curveto(186.04287287,421.70689667)(186.55487287,421.60023)(187.0882062,421.60023)
\curveto(187.4722062,421.60023)(187.8242062,421.79223)(188.1442062,422.17623)
\curveto(188.4642062,422.56023)(188.76287287,423.25356333)(189.0402062,424.25623)
\curveto(189.33887287,425.25889667)(189.60553954,426.65623)(189.8402062,428.44823)
\curveto(190.07487287,430.26156333)(190.2882062,432.57623)(190.4802062,435.39223)
\lineto(203.6962062,435.39223)
\closepath
}
}
{
\newrgbcolor{curcolor}{0 0 0}
\pscustom[linestyle=none,fillstyle=solid,fillcolor=curcolor]
{
\newpath
\moveto(213.45622085,428.64023)
\lineto(216.81622085,428.64023)
\curveto(219.50422085,428.64023)(221.48822085,428.21356333)(222.76822085,427.36023)
\curveto(224.06955419,426.50689667)(224.72022085,425.21623)(224.72022085,423.48823)
\curveto(224.72022085,421.78156333)(224.12288752,420.42689667)(222.92822085,419.42423)
\curveto(221.73355419,418.42156333)(219.76022085,417.92023)(217.00822085,417.92023)
\lineto(208.68822085,417.92023)
\lineto(208.68822085,435.39223)
\lineto(213.45622085,435.39223)
\closepath
\moveto(219.95222085,423.42423)
\curveto(219.95222085,424.70423)(218.87488752,425.34423)(216.72022085,425.34423)
\lineto(213.45622085,425.34423)
\lineto(213.45622085,421.21623)
\lineto(216.78422085,421.21623)
\curveto(217.70155419,421.21623)(218.45888752,421.37623)(219.05622085,421.69623)
\curveto(219.65355419,422.03756333)(219.95222085,422.61356333)(219.95222085,423.42423)
\closepath
}
}
{
\newrgbcolor{curcolor}{0 0 0}
\pscustom[linestyle=none,fillstyle=solid,fillcolor=curcolor]
{
\newpath
\moveto(63.29619424,395.39223)
\lineto(68.51219424,395.39223)
\lineto(71.80819424,385.56823)
\curveto(71.97886091,385.07756333)(72.10686091,384.58689667)(72.19219424,384.09623)
\curveto(72.27752757,383.60556333)(72.34152757,383.08289667)(72.38419424,382.52823)
\lineto(72.48019424,382.52823)
\curveto(72.54419424,383.08289667)(72.62952757,383.60556333)(72.73619424,384.09623)
\curveto(72.84286091,384.58689667)(72.98152757,385.07756333)(73.15219424,385.56823)
\lineto(76.38419424,395.39223)
\lineto(81.50419424,395.39223)
\lineto(74.11219424,375.68023)
\curveto(73.42952757,373.86689667)(72.45886091,372.51223)(71.20019424,371.61623)
\curveto(69.94152757,370.69889667)(68.48019424,370.24023)(66.81619424,370.24023)
\curveto(66.26152757,370.24023)(65.79219424,370.27223)(65.40819424,370.33623)
\curveto(65.02419424,370.37889667)(64.68286091,370.43223)(64.38419424,370.49623)
\lineto(64.38419424,374.27223)
\curveto(64.59752757,374.22956333)(64.87486091,374.18689667)(65.21619424,374.14423)
\curveto(65.55752757,374.10156333)(65.90952757,374.08023)(66.27219424,374.08023)
\curveto(67.27486091,374.08023)(68.06419424,374.38956333)(68.64019424,375.00823)
\curveto(69.21619424,375.60556333)(69.65352757,376.33089667)(69.95219424,377.18423)
\lineto(70.24019424,378.04823)
\closepath
}
}
{
\newrgbcolor{curcolor}{0 0 0}
\pscustom[linestyle=none,fillstyle=solid,fillcolor=curcolor]
{
\newpath
\moveto(99.87220205,395.39223)
\lineto(99.87220205,377.92023)
\lineto(95.10420205,377.92023)
\lineto(95.10420205,391.80823)
\lineto(88.76820205,391.80823)
\lineto(88.76820205,377.92023)
\lineto(84.00020205,377.92023)
\lineto(84.00020205,395.39223)
\closepath
}
}
{
\newrgbcolor{curcolor}{0 0 0}
\pscustom[linestyle=none,fillstyle=solid,fillcolor=curcolor]
{
\newpath
\moveto(114.59221573,395.71223)
\curveto(116.55488239,395.71223)(118.14421573,394.94423)(119.36021573,393.40823)
\curveto(120.57621573,391.89356333)(121.18421573,389.65356333)(121.18421573,386.68823)
\curveto(121.18421573,383.70156333)(120.55488239,381.44023)(119.29621573,379.90423)
\curveto(118.03754906,378.36823)(116.42688239,377.60023)(114.46421573,377.60023)
\curveto(113.20554906,377.60023)(112.20288239,377.82423)(111.45621573,378.27223)
\curveto(110.70954906,378.74156333)(110.10154906,379.26423)(109.63221573,379.84023)
\lineto(109.37621573,379.84023)
\curveto(109.54688239,378.94423)(109.63221573,378.09089667)(109.63221573,377.28023)
\lineto(109.63221573,370.24023)
\lineto(104.86421573,370.24023)
\lineto(104.86421573,395.39223)
\lineto(108.73621573,395.39223)
\lineto(109.40821573,393.12023)
\lineto(109.63221573,393.12023)
\curveto(110.10154906,393.82423)(110.73088239,394.43223)(111.52021573,394.94423)
\curveto(112.30954906,395.45623)(113.33354906,395.71223)(114.59221573,395.71223)
\closepath
\moveto(113.05621573,391.90423)
\curveto(111.81888239,391.90423)(110.94421573,391.50956333)(110.43221573,390.72023)
\curveto(109.92021573,389.95223)(109.65354906,388.78956333)(109.63221573,387.23223)
\lineto(109.63221573,386.72023)
\curveto(109.63221573,385.03489667)(109.87754906,383.73356333)(110.36821573,382.81623)
\curveto(110.88021573,381.92023)(111.79754906,381.47223)(113.12021573,381.47223)
\curveto(114.20821573,381.47223)(115.00821573,381.92023)(115.52021573,382.81623)
\curveto(116.05354906,383.73356333)(116.32021573,385.04556333)(116.32021573,386.75223)
\curveto(116.32021573,390.18689667)(115.23221573,391.90423)(113.05621573,391.90423)
\closepath
}
}
{
\newrgbcolor{curcolor}{0 0 0}
\pscustom[linestyle=none,fillstyle=solid,fillcolor=curcolor]
{
\newpath
\moveto(132.28819717,395.74423)
\curveto(134.63486384,395.74423)(136.42686384,395.23223)(137.66419717,394.20823)
\curveto(138.92286384,393.20556333)(139.55219717,391.65889667)(139.55219717,389.56823)
\lineto(139.55219717,377.92023)
\lineto(136.22419717,377.92023)
\lineto(135.29619717,380.28823)
\lineto(135.16819717,380.28823)
\curveto(134.4215305,379.34956333)(133.63219717,378.66689667)(132.80019717,378.24023)
\curveto(131.96819717,377.81356333)(130.82686384,377.60023)(129.37619717,377.60023)
\curveto(127.81886384,377.60023)(126.52819717,378.04823)(125.50419717,378.94423)
\curveto(124.48019717,379.84023)(123.96819717,381.23756333)(123.96819717,383.13623)
\curveto(123.96819717,384.99223)(124.61886384,386.35756333)(125.92019717,387.23223)
\curveto(127.2215305,388.10689667)(129.1735305,388.59756333)(131.77619717,388.70423)
\lineto(134.81619717,388.80023)
\lineto(134.81619717,389.56823)
\curveto(134.81619717,390.48556333)(134.57086384,391.15756333)(134.08019717,391.58423)
\curveto(133.61086384,392.01089667)(132.9495305,392.22423)(132.09619717,392.22423)
\curveto(131.24286384,392.22423)(130.41086384,392.09623)(129.60019717,391.84023)
\curveto(128.7895305,391.60556333)(127.97886384,391.30689667)(127.16819717,390.94423)
\lineto(125.60019717,394.17623)
\curveto(126.5175305,394.64556333)(127.55219717,395.01889667)(128.70419717,395.29623)
\curveto(129.85619717,395.59489667)(131.05086384,395.74423)(132.28819717,395.74423)
\closepath
\moveto(134.81619717,386.01623)
\lineto(132.96019717,385.95223)
\curveto(131.42419717,385.90956333)(130.3575305,385.63223)(129.76019717,385.12023)
\curveto(129.16286384,384.60823)(128.86419717,383.93623)(128.86419717,383.10423)
\curveto(128.86419717,382.37889667)(129.0775305,381.85623)(129.50419717,381.53623)
\curveto(129.93086384,381.23756333)(130.4855305,381.08823)(131.16819717,381.08823)
\curveto(132.19219717,381.08823)(133.05619717,381.38689667)(133.76019717,381.98423)
\curveto(134.46419717,382.60289667)(134.81619717,383.46689667)(134.81619717,384.57623)
\closepath
}
}
{
\newrgbcolor{curcolor}{0 0 0}
\pscustom[linestyle=none,fillstyle=solid,fillcolor=curcolor]
{
\newpath
\moveto(159.9362001,390.81623)
\curveto(159.9362001,389.87756333)(159.63753343,389.07756333)(159.0402001,388.41623)
\curveto(158.4642001,387.75489667)(157.6002001,387.32823)(156.4482001,387.13623)
\lineto(156.4482001,387.00823)
\curveto(157.6642001,386.85889667)(158.63486677,386.43223)(159.3602001,385.72823)
\curveto(160.10686677,385.04556333)(160.4802001,384.18156333)(160.4802001,383.13623)
\curveto(160.4802001,382.13356333)(160.21353343,381.23756333)(159.6802001,380.44823)
\curveto(159.1682001,379.65889667)(158.34686677,379.04023)(157.2162001,378.59223)
\curveto(156.08553343,378.14423)(154.60286677,377.92023)(152.7682001,377.92023)
\lineto(144.4482001,377.92023)
\lineto(144.4482001,395.39223)
\lineto(152.7682001,395.39223)
\curveto(154.13353343,395.39223)(155.34953343,395.24289667)(156.4162001,394.94423)
\curveto(157.5042001,394.66689667)(158.35753343,394.18689667)(158.9762001,393.50423)
\curveto(159.6162001,392.84289667)(159.9362001,391.94689667)(159.9362001,390.81623)
\closepath
\moveto(155.1042001,390.43223)
\curveto(155.1042001,391.49889667)(154.26153343,392.03223)(152.5762001,392.03223)
\lineto(149.2162001,392.03223)
\lineto(149.2162001,388.57623)
\lineto(152.0322001,388.57623)
\curveto(153.03486677,388.57623)(153.7922001,388.71489667)(154.3042001,388.99223)
\curveto(154.83753343,389.29089667)(155.1042001,389.77089667)(155.1042001,390.43223)
\closepath
\moveto(155.5522001,383.39223)
\curveto(155.5522001,384.07489667)(155.27486677,384.56556333)(154.7202001,384.86423)
\curveto(154.18686677,385.18423)(153.39753343,385.34423)(152.3522001,385.34423)
\lineto(149.2162001,385.34423)
\lineto(149.2162001,381.21623)
\lineto(152.4482001,381.21623)
\curveto(153.3442001,381.21623)(154.0802001,381.37623)(154.6562001,381.69623)
\curveto(155.25353343,382.03756333)(155.5522001,382.60289667)(155.5522001,383.39223)
\closepath
}
}
{
\newrgbcolor{curcolor}{0 0 0}
\pscustom[linestyle=none,fillstyle=solid,fillcolor=curcolor]
{
\newpath
\moveto(180.22419082,377.92023)
\lineto(175.45619082,377.92023)
\lineto(175.45619082,391.80823)
\lineto(171.07219082,391.80823)
\curveto(170.79485749,388.39489667)(170.42152416,385.64289667)(169.95219082,383.55223)
\curveto(169.50419082,381.48289667)(168.86419082,379.96823)(168.03219082,379.00823)
\curveto(167.22152416,378.06956333)(166.14419082,377.60023)(164.80019082,377.60023)
\curveto(163.69085749,377.60023)(162.78419082,377.77089667)(162.08019082,378.11223)
\lineto(162.08019082,381.92023)
\curveto(162.57085749,381.70689667)(163.08285749,381.60023)(163.61619082,381.60023)
\curveto(164.00019082,381.60023)(164.35219082,381.79223)(164.67219082,382.17623)
\curveto(164.99219082,382.56023)(165.29085749,383.25356333)(165.56819082,384.25623)
\curveto(165.86685749,385.25889667)(166.13352416,386.65623)(166.36819082,388.44823)
\curveto(166.60285749,390.26156333)(166.81619082,392.57623)(167.00819082,395.39223)
\lineto(180.22419082,395.39223)
\closepath
}
}
{
\newrgbcolor{curcolor}{0 0 0}
\pscustom[linestyle=none,fillstyle=solid,fillcolor=curcolor]
{
\newpath
\moveto(192.41620547,395.71223)
\curveto(194.82687214,395.71223)(196.73620547,395.01889667)(198.14420547,393.63223)
\curveto(199.55220547,392.26689667)(200.25620547,390.31489667)(200.25620547,387.77623)
\lineto(200.25620547,385.47223)
\lineto(188.99220547,385.47223)
\curveto(189.03487214,384.12823)(189.42953881,383.07223)(190.17620547,382.30423)
\curveto(190.94420547,381.53623)(192.00020547,381.15223)(193.34420547,381.15223)
\curveto(194.45353881,381.15223)(195.46687214,381.25889667)(196.38420547,381.47223)
\curveto(197.32287214,381.70689667)(198.28287214,382.05889667)(199.26420547,382.52823)
\lineto(199.26420547,378.84823)
\curveto(198.38953881,378.42156333)(197.48287214,378.11223)(196.54420547,377.92023)
\curveto(195.60553881,377.70689667)(194.46420547,377.60023)(193.12020547,377.60023)
\curveto(191.37087214,377.60023)(189.82420547,377.92023)(188.48020547,378.56023)
\curveto(187.13620547,379.22156333)(186.08020547,380.20289667)(185.31220547,381.50423)
\curveto(184.54420547,382.82689667)(184.16020547,384.50156333)(184.16020547,386.52823)
\curveto(184.16020547,388.55489667)(184.50153881,390.25089667)(185.18420547,391.61623)
\curveto(185.88820547,392.98156333)(186.85887214,394.00556333)(188.09620547,394.68823)
\curveto(189.33353881,395.37089667)(190.77353881,395.71223)(192.41620547,395.71223)
\closepath
\moveto(192.44820547,392.32023)
\curveto(191.50953881,392.32023)(190.74153881,392.02156333)(190.14420547,391.42423)
\curveto(189.54687214,390.82689667)(189.19487214,389.89889667)(189.08820547,388.64023)
\lineto(195.77620547,388.64023)
\curveto(195.75487214,389.68556333)(195.46687214,390.56023)(194.91220547,391.26423)
\curveto(194.37887214,391.96823)(193.55753881,392.32023)(192.44820547,392.32023)
\closepath
}
}
{
\newrgbcolor{curcolor}{0 0 0}
\pscustom[linestyle=none,fillstyle=solid,fillcolor=curcolor]
{
\newpath
\moveto(208.89619278,395.39223)
\lineto(208.89619278,388.67223)
\lineto(215.55219278,388.67223)
\lineto(215.55219278,395.39223)
\lineto(220.32019278,395.39223)
\lineto(220.32019278,377.92023)
\lineto(215.55219278,377.92023)
\lineto(215.55219278,385.12023)
\lineto(208.89619278,385.12023)
\lineto(208.89619278,377.92023)
\lineto(204.12819278,377.92023)
\lineto(204.12819278,395.39223)
\closepath
}
}
{
\newrgbcolor{curcolor}{0 0 0}
\pscustom[linestyle=none,fillstyle=solid,fillcolor=curcolor]
{
\newpath
\moveto(229.92021377,395.39223)
\lineto(229.92021377,388.48023)
\curveto(229.92021377,388.11756333)(229.89888044,387.66956333)(229.85621377,387.13623)
\curveto(229.83488044,386.60289667)(229.80288044,386.05889667)(229.76021377,385.50423)
\curveto(229.73888044,384.94956333)(229.70688044,384.44823)(229.66421377,384.00023)
\curveto(229.62154711,383.57356333)(229.58954711,383.28556333)(229.56821377,383.13623)
\lineto(237.63221377,395.39223)
\lineto(243.36021377,395.39223)
\lineto(243.36021377,377.92023)
\lineto(238.75221377,377.92023)
\lineto(238.75221377,384.89623)
\curveto(238.75221377,385.45089667)(238.77354711,386.08023)(238.81621377,386.78423)
\curveto(238.85888044,387.48823)(238.90154711,388.13889667)(238.94421377,388.73623)
\curveto(239.00821377,389.35489667)(239.05088044,389.82423)(239.07221377,390.14423)
\lineto(231.04021377,377.92023)
\lineto(225.31221377,377.92023)
\lineto(225.31221377,395.39223)
\closepath
}
}
{
\newrgbcolor{curcolor}{0 0 0}
\pscustom[linestyle=none,fillstyle=solid,fillcolor=curcolor]
{
\newpath
\moveto(251.0081918,377.92023)
\lineto(245.8561918,377.92023)
\lineto(250.5601918,384.83223)
\curveto(249.6641918,385.19489667)(248.8641918,385.78156333)(248.1601918,386.59223)
\curveto(247.47752513,387.42423)(247.1361918,388.55489667)(247.1361918,389.98423)
\curveto(247.1361918,391.73356333)(247.79752513,393.06689667)(249.1201918,393.98423)
\curveto(250.44285847,394.92289667)(252.13885847,395.39223)(254.2081918,395.39223)
\lineto(262.3361918,395.39223)
\lineto(262.3361918,377.92023)
\lineto(257.5681918,377.92023)
\lineto(257.5681918,384.41623)
\lineto(254.9441918,384.41623)
\closepath
\moveto(251.8081918,389.95223)
\curveto(251.8081918,389.22689667)(252.0961918,388.65089667)(252.6721918,388.22423)
\curveto(253.2481918,387.81889667)(253.99485847,387.61623)(254.9121918,387.61623)
\lineto(257.5681918,387.61623)
\lineto(257.5681918,392.03223)
\lineto(254.3041918,392.03223)
\curveto(253.45085847,392.03223)(252.82152513,391.81889667)(252.4161918,391.39223)
\curveto(252.01085847,390.98689667)(251.8081918,390.50689667)(251.8081918,389.95223)
\closepath
}
}
{
\newrgbcolor{curcolor}{0 0 0}
\pscustom[linestyle=none,fillstyle=solid,fillcolor=curcolor]
{
\newpath
\moveto(107.45621695,355.39223)
\lineto(107.45621695,337.92023)
\lineto(102.68821695,337.92023)
\lineto(102.68821695,351.80823)
\lineto(96.35221695,351.80823)
\lineto(96.35221695,337.92023)
\lineto(91.58421695,337.92023)
\lineto(91.58421695,355.39223)
\closepath
}
}
{
\newrgbcolor{curcolor}{0 0 0}
\pscustom[linestyle=none,fillstyle=solid,fillcolor=curcolor]
{
\newpath
\moveto(122.17623062,355.71223)
\curveto(124.13889729,355.71223)(125.72823062,354.94423)(126.94423062,353.40823)
\curveto(128.16023062,351.89356333)(128.76823062,349.65356333)(128.76823062,346.68823)
\curveto(128.76823062,343.70156333)(128.13889729,341.44023)(126.88023062,339.90423)
\curveto(125.62156395,338.36823)(124.01089729,337.60023)(122.04823062,337.60023)
\curveto(120.78956395,337.60023)(119.78689729,337.82423)(119.04023062,338.27223)
\curveto(118.29356395,338.74156333)(117.68556395,339.26423)(117.21623062,339.84023)
\lineto(116.96023062,339.84023)
\curveto(117.13089729,338.94423)(117.21623062,338.09089667)(117.21623062,337.28023)
\lineto(117.21623062,330.24023)
\lineto(112.44823062,330.24023)
\lineto(112.44823062,355.39223)
\lineto(116.32023062,355.39223)
\lineto(116.99223062,353.12023)
\lineto(117.21623062,353.12023)
\curveto(117.68556395,353.82423)(118.31489729,354.43223)(119.10423062,354.94423)
\curveto(119.89356395,355.45623)(120.91756395,355.71223)(122.17623062,355.71223)
\closepath
\moveto(120.64023062,351.90423)
\curveto(119.40289729,351.90423)(118.52823062,351.50956333)(118.01623062,350.72023)
\curveto(117.50423062,349.95223)(117.23756395,348.78956333)(117.21623062,347.23223)
\lineto(117.21623062,346.72023)
\curveto(117.21623062,345.03489667)(117.46156395,343.73356333)(117.95223062,342.81623)
\curveto(118.46423062,341.92023)(119.38156395,341.47223)(120.70423062,341.47223)
\curveto(121.79223062,341.47223)(122.59223062,341.92023)(123.10423062,342.81623)
\curveto(123.63756395,343.73356333)(123.90423062,345.04556333)(123.90423062,346.75223)
\curveto(123.90423062,350.18689667)(122.81623062,351.90423)(120.64023062,351.90423)
\closepath
}
}
{
\newrgbcolor{curcolor}{0 0 0}
\pscustom[linestyle=none,fillstyle=solid,fillcolor=curcolor]
{
\newpath
\moveto(139.87221206,355.74423)
\curveto(142.21887873,355.74423)(144.01087873,355.23223)(145.24821206,354.20823)
\curveto(146.50687873,353.20556333)(147.13621206,351.65889667)(147.13621206,349.56823)
\lineto(147.13621206,337.92023)
\lineto(143.80821206,337.92023)
\lineto(142.88021206,340.28823)
\lineto(142.75221206,340.28823)
\curveto(142.0055454,339.34956333)(141.21621206,338.66689667)(140.38421206,338.24023)
\curveto(139.55221206,337.81356333)(138.41087873,337.60023)(136.96021206,337.60023)
\curveto(135.40287873,337.60023)(134.11221206,338.04823)(133.08821206,338.94423)
\curveto(132.06421206,339.84023)(131.55221206,341.23756333)(131.55221206,343.13623)
\curveto(131.55221206,344.99223)(132.20287873,346.35756333)(133.50421206,347.23223)
\curveto(134.8055454,348.10689667)(136.7575454,348.59756333)(139.36021206,348.70423)
\lineto(142.40021206,348.80023)
\lineto(142.40021206,349.56823)
\curveto(142.40021206,350.48556333)(142.15487873,351.15756333)(141.66421206,351.58423)
\curveto(141.19487873,352.01089667)(140.5335454,352.22423)(139.68021206,352.22423)
\curveto(138.82687873,352.22423)(137.99487873,352.09623)(137.18421206,351.84023)
\curveto(136.3735454,351.60556333)(135.56287873,351.30689667)(134.75221206,350.94423)
\lineto(133.18421206,354.17623)
\curveto(134.1015454,354.64556333)(135.13621206,355.01889667)(136.28821206,355.29623)
\curveto(137.44021206,355.59489667)(138.63487873,355.74423)(139.87221206,355.74423)
\closepath
\moveto(142.40021206,346.01623)
\lineto(140.54421206,345.95223)
\curveto(139.00821206,345.90956333)(137.9415454,345.63223)(137.34421206,345.12023)
\curveto(136.74687873,344.60823)(136.44821206,343.93623)(136.44821206,343.10423)
\curveto(136.44821206,342.37889667)(136.6615454,341.85623)(137.08821206,341.53623)
\curveto(137.51487873,341.23756333)(138.0695454,341.08823)(138.75221206,341.08823)
\curveto(139.77621206,341.08823)(140.64021206,341.38689667)(141.34421206,341.98423)
\curveto(142.04821206,342.60289667)(142.40021206,343.46689667)(142.40021206,344.57623)
\closepath
}
}
{
\newrgbcolor{curcolor}{0 0 0}
\pscustom[linestyle=none,fillstyle=solid,fillcolor=curcolor]
{
\newpath
\moveto(167.52021499,350.81623)
\curveto(167.52021499,349.87756333)(167.22154833,349.07756333)(166.62421499,348.41623)
\curveto(166.04821499,347.75489667)(165.18421499,347.32823)(164.03221499,347.13623)
\lineto(164.03221499,347.00823)
\curveto(165.24821499,346.85889667)(166.21888166,346.43223)(166.94421499,345.72823)
\curveto(167.69088166,345.04556333)(168.06421499,344.18156333)(168.06421499,343.13623)
\curveto(168.06421499,342.13356333)(167.79754833,341.23756333)(167.26421499,340.44823)
\curveto(166.75221499,339.65889667)(165.93088166,339.04023)(164.80021499,338.59223)
\curveto(163.66954833,338.14423)(162.18688166,337.92023)(160.35221499,337.92023)
\lineto(152.03221499,337.92023)
\lineto(152.03221499,355.39223)
\lineto(160.35221499,355.39223)
\curveto(161.71754833,355.39223)(162.93354833,355.24289667)(164.00021499,354.94423)
\curveto(165.08821499,354.66689667)(165.94154833,354.18689667)(166.56021499,353.50423)
\curveto(167.20021499,352.84289667)(167.52021499,351.94689667)(167.52021499,350.81623)
\closepath
\moveto(162.68821499,350.43223)
\curveto(162.68821499,351.49889667)(161.84554833,352.03223)(160.16021499,352.03223)
\lineto(156.80021499,352.03223)
\lineto(156.80021499,348.57623)
\lineto(159.61621499,348.57623)
\curveto(160.61888166,348.57623)(161.37621499,348.71489667)(161.88821499,348.99223)
\curveto(162.42154833,349.29089667)(162.68821499,349.77089667)(162.68821499,350.43223)
\closepath
\moveto(163.13621499,343.39223)
\curveto(163.13621499,344.07489667)(162.85888166,344.56556333)(162.30421499,344.86423)
\curveto(161.77088166,345.18423)(160.98154833,345.34423)(159.93621499,345.34423)
\lineto(156.80021499,345.34423)
\lineto(156.80021499,341.21623)
\lineto(160.03221499,341.21623)
\curveto(160.92821499,341.21623)(161.66421499,341.37623)(162.24021499,341.69623)
\curveto(162.83754833,342.03756333)(163.13621499,342.60289667)(163.13621499,343.39223)
\closepath
}
}
{
\newrgbcolor{curcolor}{0 0 0}
\pscustom[linestyle=none,fillstyle=solid,fillcolor=curcolor]
{
\newpath
\moveto(179.32820572,355.74423)
\curveto(181.67487238,355.74423)(183.46687238,355.23223)(184.70420572,354.20823)
\curveto(185.96287238,353.20556333)(186.59220572,351.65889667)(186.59220572,349.56823)
\lineto(186.59220572,337.92023)
\lineto(183.26420572,337.92023)
\lineto(182.33620572,340.28823)
\lineto(182.20820572,340.28823)
\curveto(181.46153905,339.34956333)(180.67220572,338.66689667)(179.84020572,338.24023)
\curveto(179.00820572,337.81356333)(177.86687238,337.60023)(176.41620572,337.60023)
\curveto(174.85887238,337.60023)(173.56820572,338.04823)(172.54420572,338.94423)
\curveto(171.52020572,339.84023)(171.00820572,341.23756333)(171.00820572,343.13623)
\curveto(171.00820572,344.99223)(171.65887238,346.35756333)(172.96020572,347.23223)
\curveto(174.26153905,348.10689667)(176.21353905,348.59756333)(178.81620572,348.70423)
\lineto(181.85620572,348.80023)
\lineto(181.85620572,349.56823)
\curveto(181.85620572,350.48556333)(181.61087238,351.15756333)(181.12020572,351.58423)
\curveto(180.65087238,352.01089667)(179.98953905,352.22423)(179.13620572,352.22423)
\curveto(178.28287238,352.22423)(177.45087238,352.09623)(176.64020572,351.84023)
\curveto(175.82953905,351.60556333)(175.01887238,351.30689667)(174.20820572,350.94423)
\lineto(172.64020572,354.17623)
\curveto(173.55753905,354.64556333)(174.59220572,355.01889667)(175.74420572,355.29623)
\curveto(176.89620572,355.59489667)(178.09087238,355.74423)(179.32820572,355.74423)
\closepath
\moveto(181.85620572,346.01623)
\lineto(180.00020572,345.95223)
\curveto(178.46420572,345.90956333)(177.39753905,345.63223)(176.80020572,345.12023)
\curveto(176.20287238,344.60823)(175.90420572,343.93623)(175.90420572,343.10423)
\curveto(175.90420572,342.37889667)(176.11753905,341.85623)(176.54420572,341.53623)
\curveto(176.97087238,341.23756333)(177.52553905,341.08823)(178.20820572,341.08823)
\curveto(179.23220572,341.08823)(180.09620572,341.38689667)(180.80020572,341.98423)
\curveto(181.50420572,342.60289667)(181.85620572,343.46689667)(181.85620572,344.57623)
\closepath
}
}
{
\newrgbcolor{curcolor}{0 0 0}
\pscustom[linestyle=none,fillstyle=solid,fillcolor=curcolor]
{
\newpath
\moveto(213.50420865,355.39223)
\lineto(213.50420865,337.92023)
\lineto(209.05620865,337.92023)
\lineto(209.05620865,346.49623)
\curveto(209.05620865,347.34956333)(209.06687531,348.18156333)(209.08820865,348.99223)
\curveto(209.13087531,349.80289667)(209.18420865,350.54956333)(209.24820865,351.23223)
\lineto(209.15220865,351.23223)
\lineto(204.32020865,337.92023)
\lineto(200.73620865,337.92023)
\lineto(195.84020865,351.26423)
\lineto(195.71220865,351.26423)
\curveto(195.79754198,350.56023)(195.85087531,349.80289667)(195.87220865,348.99223)
\curveto(195.91487531,348.20289667)(195.93620865,347.32823)(195.93620865,346.36823)
\lineto(195.93620865,337.92023)
\lineto(191.48820865,337.92023)
\lineto(191.48820865,355.39223)
\lineto(198.24020865,355.39223)
\lineto(202.59220865,343.55223)
\lineto(207.00820865,355.39223)
\closepath
}
}
{
\newrgbcolor{curcolor}{0 0 0}
\pscustom[linestyle=none,fillstyle=solid,fillcolor=curcolor]
{
\newpath
\moveto(223.10420425,355.39223)
\lineto(223.10420425,348.48023)
\curveto(223.10420425,348.11756333)(223.08287092,347.66956333)(223.04020425,347.13623)
\curveto(223.01887092,346.60289667)(222.98687092,346.05889667)(222.94420425,345.50423)
\curveto(222.92287092,344.94956333)(222.89087092,344.44823)(222.84820425,344.00023)
\curveto(222.80553758,343.57356333)(222.77353758,343.28556333)(222.75220425,343.13623)
\lineto(230.81620425,355.39223)
\lineto(236.54420425,355.39223)
\lineto(236.54420425,337.92023)
\lineto(231.93620425,337.92023)
\lineto(231.93620425,344.89623)
\curveto(231.93620425,345.45089667)(231.95753758,346.08023)(232.00020425,346.78423)
\curveto(232.04287092,347.48823)(232.08553758,348.13889667)(232.12820425,348.73623)
\curveto(232.19220425,349.35489667)(232.23487092,349.82423)(232.25620425,350.14423)
\lineto(224.22420425,337.92023)
\lineto(218.49620425,337.92023)
\lineto(218.49620425,355.39223)
\closepath
}
}
{
\newrgbcolor{curcolor}{0 0 0}
\pscustom[linestyle=none,fillstyle=solid,fillcolor=curcolor]
{
\newpath
\moveto(472.98351323,362.038863)
\lineto(467.47951323,362.038863)
\lineto(459.19151323,373.622863)
\lineto(459.19151323,362.038863)
\lineto(454.35951323,362.038863)
\lineto(454.35951323,384.886863)
\lineto(459.19151323,384.886863)
\lineto(459.19151323,373.814863)
\lineto(467.38351323,384.886863)
\lineto(472.53551323,384.886863)
\lineto(464.21551323,373.910863)
\closepath
}
}
{
\newrgbcolor{curcolor}{0 0 0}
\pscustom[linestyle=none,fillstyle=solid,fillcolor=curcolor]
{
\newpath
\moveto(491.12754155,362.038863)
\lineto(486.35954155,362.038863)
\lineto(486.35954155,375.926863)
\lineto(481.97554155,375.926863)
\curveto(481.69820822,372.51352967)(481.32487489,369.76152967)(480.85554155,367.670863)
\curveto(480.40754155,365.60152967)(479.76754155,364.086863)(478.93554155,363.126863)
\curveto(478.12487489,362.18819633)(477.04754155,361.718863)(475.70354155,361.718863)
\curveto(474.59420822,361.718863)(473.68754155,361.88952967)(472.98354155,362.230863)
\lineto(472.98354155,366.038863)
\curveto(473.47420822,365.82552967)(473.98620822,365.718863)(474.51954155,365.718863)
\curveto(474.90354155,365.718863)(475.25554155,365.910863)(475.57554155,366.294863)
\curveto(475.89554155,366.678863)(476.19420822,367.37219633)(476.47154155,368.374863)
\curveto(476.77020822,369.37752967)(477.03687489,370.774863)(477.27154155,372.566863)
\curveto(477.50620822,374.38019633)(477.71954155,376.694863)(477.91154155,379.510863)
\lineto(491.12754155,379.510863)
\closepath
}
}
{
\newrgbcolor{curcolor}{0 0 0}
\pscustom[linestyle=none,fillstyle=solid,fillcolor=curcolor]
{
\newpath
\moveto(500.7275562,379.510863)
\lineto(500.7275562,372.598863)
\curveto(500.7275562,372.23619633)(500.70622287,371.78819633)(500.6635562,371.254863)
\curveto(500.64222287,370.72152967)(500.61022287,370.17752967)(500.5675562,369.622863)
\curveto(500.54622287,369.06819633)(500.51422287,368.566863)(500.4715562,368.118863)
\curveto(500.42888954,367.69219633)(500.39688954,367.40419633)(500.3755562,367.254863)
\lineto(508.4395562,379.510863)
\lineto(514.1675562,379.510863)
\lineto(514.1675562,362.038863)
\lineto(509.5595562,362.038863)
\lineto(509.5595562,369.014863)
\curveto(509.5595562,369.56952967)(509.58088954,370.198863)(509.6235562,370.902863)
\curveto(509.66622287,371.606863)(509.70888954,372.25752967)(509.7515562,372.854863)
\curveto(509.8155562,373.47352967)(509.85822287,373.942863)(509.8795562,374.262863)
\lineto(501.8475562,362.038863)
\lineto(496.1195562,362.038863)
\lineto(496.1195562,379.510863)
\closepath
}
}
{
\newrgbcolor{curcolor}{0 0 0}
\pscustom[linestyle=none,fillstyle=solid,fillcolor=curcolor]
{
\newpath
\moveto(526.35953423,379.830863)
\curveto(528.7702009,379.830863)(530.67953423,379.13752967)(532.08753423,377.750863)
\curveto(533.49553423,376.38552967)(534.19953423,374.43352967)(534.19953423,371.894863)
\lineto(534.19953423,369.590863)
\lineto(522.93553423,369.590863)
\curveto(522.9782009,368.246863)(523.37286756,367.190863)(524.11953423,366.422863)
\curveto(524.88753423,365.654863)(525.94353423,365.270863)(527.28753423,365.270863)
\curveto(528.39686756,365.270863)(529.4102009,365.37752967)(530.32753423,365.590863)
\curveto(531.2662009,365.82552967)(532.2262009,366.17752967)(533.20753423,366.646863)
\lineto(533.20753423,362.966863)
\curveto(532.33286756,362.54019633)(531.4262009,362.230863)(530.48753423,362.038863)
\curveto(529.54886756,361.82552967)(528.40753423,361.718863)(527.06353423,361.718863)
\curveto(525.3142009,361.718863)(523.76753423,362.038863)(522.42353423,362.678863)
\curveto(521.07953423,363.34019633)(520.02353423,364.32152967)(519.25553423,365.622863)
\curveto(518.48753423,366.94552967)(518.10353423,368.62019633)(518.10353423,370.646863)
\curveto(518.10353423,372.67352967)(518.44486756,374.36952967)(519.12753423,375.734863)
\curveto(519.83153423,377.10019633)(520.8022009,378.12419633)(522.03953423,378.806863)
\curveto(523.27686756,379.48952967)(524.71686756,379.830863)(526.35953423,379.830863)
\closepath
\moveto(526.39153423,376.438863)
\curveto(525.45286756,376.438863)(524.68486756,376.14019633)(524.08753423,375.542863)
\curveto(523.4902009,374.94552967)(523.1382009,374.01752967)(523.03153423,372.758863)
\lineto(529.71953423,372.758863)
\curveto(529.6982009,373.80419633)(529.4102009,374.678863)(528.85553423,375.382863)
\curveto(528.3222009,376.086863)(527.50086756,376.438863)(526.39153423,376.438863)
\closepath
}
}
{
\newrgbcolor{curcolor}{0 0 0}
\pscustom[linestyle=none,fillstyle=solid,fillcolor=curcolor]
{
\newpath
\moveto(542.83952153,379.510863)
\lineto(542.83952153,372.790863)
\lineto(549.49552153,372.790863)
\lineto(549.49552153,379.510863)
\lineto(554.26352153,379.510863)
\lineto(554.26352153,362.038863)
\lineto(549.49552153,362.038863)
\lineto(549.49552153,369.238863)
\lineto(542.83952153,369.238863)
\lineto(542.83952153,362.038863)
\lineto(538.07152153,362.038863)
\lineto(538.07152153,379.510863)
\closepath
}
}
{
\newrgbcolor{curcolor}{0 0 0}
\pscustom[linestyle=none,fillstyle=solid,fillcolor=curcolor]
{
\newpath
\moveto(573.71954253,375.926863)
\lineto(567.99154253,375.926863)
\lineto(567.99154253,362.038863)
\lineto(563.22354253,362.038863)
\lineto(563.22354253,375.926863)
\lineto(557.49554253,375.926863)
\lineto(557.49554253,379.510863)
\lineto(573.71954253,379.510863)
\closepath
}
}
\end{pspicture}
}
		\caption{Общая схема клиентской части}
		\label{g5_ink1}
	\end{center}
\end{figure}

Для удобства было решено разделить страницу на три части [Рис.~\ref{g5_img1}]. В первой части (левой)
располагается основная информация комнаты, панель управления правами и список участников.
Вторая часть (центральная) содержит апплет \emph{GeoGebra}.
В третьей части (правой) был размещен чат [Исходный код \ref{sc34_k_js_messages}]. При этом, для экономии места,
левую и правую части можно свернуть. Когда они свернуты пользователь
может отследить активность чата и, если используется комната с
ограниченными правами, запрос на предоставление прав. Отслеживание
выполняется путем мигания боковых вертикальных кнопок (с помощью которых
и сворачиваются элементы).

\begin{figure}[H]
	\begin{center}
		\includegraphics[width=\linewidth]{g5/img1.png}
		\caption{Веб-интерфейс}
		\label{g5_img1}
	\end{center}
\end{figure}

В комнате с ограниченными правами, если пользователь запрашивает доступ
с помощью специальной кнопки, то приходит уведомление владельцу и в
списке участников комнаты, напротив имени пользователя, запросившего
права, появляются две кнопки [Рис.~\ref{g5_img2}]: $+$ и $-$, нажимая первую, владелец передает
право редактировать доску, соответственно, нажимая вторую, он отказывает
в доступе.

\begin{figure}[H]
	\begin{center}
		\includegraphics[width=\linewidth]{g5/img2.png}
		\caption{Интерфейс передачи прав}
		\label{g5_img2}
	\end{center}
\end{figure}

Нeмaлoвaжной функцией является трансляция курсора мыши пользователя [Исходный код \ref{sc36_k_js_cursor}],
имеющего в данный момент право на редактирования доски. Она
включается соответствующей кнопкой в левой панели.
Данная функция реализована с помощью формул преобразования координат [Рис.~\ref{g5_img3}]:
\begin{equation*}
	\left\{
		\begin{aligned}
			& x=x_s+x', \\
			& y=y_s+y'.
		\end{aligned}
	\right.
\end{equation*}
\begin{figure}[H]
	\begin{center}
		\includegraphics[width=\linewidth]{g5/img3.png}
		\caption{Преобразование координат}
		\label{g5_img3}
	\end{center}
\end{figure}

Еще одной функцией является трансляция координат. Она
полностью синхронизирует перемещение координат апплета \emph{GeoGebra},
включается так же соответствующей кнопкой в левой панели.
Эта функция не включена по умолчанию, т.е. активируется
только кнопкой, т.к. в некоторых случаях она мешает просмотру.

Все элементы интерфейса были написаны с использованием
шаблонизатора \emph{handlebars}. Он обеспечивает более гибкую и свободную
компоновку элементов на момент создания веб-страницы.

Стоит отметить важность набора формул в чате, поэтому, используя библиотеку
\emph{KaTeX}, была добавлена поддержка синтаксиса \emph{LaTeX} [Исходный код \ref{sc35_k_js_latex}].
Т.е. каждое сообщение, которое приходит пользователю проверяется
библиотекой на наличие специальных символов.

Кроме чата в правой панели находится кнопка управления голосовой связью [Исходный код \ref{sc37_k_js_voice}],
по умолчанию голосовой канал пользователя отключен.

Также для эффективного использования экранного места был
добавлен полноэкранный режим [Исходный код \ref{sc30_k_js_fullScreen}] с помощью стандартных средств
языка \emph{JavaScript}.

Кроме основной страницы комнаты есть и вспомогательные: главная страница проекта [Исходный код \ref{sc14_k_hb_index}],
страница выбора типа прав [Исходный код \ref{sc15_k_hb_choiceCreateRoom}],
страница создания комнаты [Исходный код \ref{sc16_k_hb_createRoom}],
страница входа в комнату [Исходный код \ref{sc17_k_hb_connectRoom}] и
панель администратора [Исходный код \ref{sc20_k_hb_adminPanel}].

Вход в определенную комнату осуществляется с помощью сгенерированной
пары ключей [Рис.~\ref{g5_img4}]: ключ комнаты (номер комнаты) и ключ владельца,
последний нужен для работы
ограниченной системы прав. Эти ключи передаются на сервер, и он уже
отправляет пользователя в искомую комнату.
Данные ключи можно скопировать, нажав на соответствующую кнопку
при создании комнаты или, если комната уже создана, скопировать 
ключ комнаты через кнопку в левой панели.

\begin{figure}[h]
	\begin{center}
		\includegraphics[width=\linewidth]{g5/img4.png}
		\caption{Интерфейс подключения к комнате}
		\label{g5_img4}
	\end{center}
\end{figure}

Представленная версия находится на стадии доработки, так как
исследование удобства данного веб-интерфейса еще не окончено.
