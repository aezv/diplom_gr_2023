Проект «Geometry Room» должен открывать возможность
многопользовательской работы с инструментом «GeoGebra».
Проведя анализ работы современных мессенджеров и программ для организации видеоконференций,
можно выделить одну важную деталь — все они используют систему комнат.
То есть организатор видеоконференции создает комнату и приглашает в нее всех остальных участников.

Под комнатой тут подразумевается отдельное интерактивное пространство для проведения конференций.

Рассматривая наш случай, у нас есть виртуальная комната с, условно, меловой доской. Возникает вопрос: «А как организовать
взаимодействие с этой доской?».

\noindent
Это вопрос об одном меле и двух учениках: одновременно пользоваться
мелом не получится, только поочередно передавая его друг другу.
Остается решить, как этот мел будет передаваться между учениками у доски.
Появляются два варианта: ученики будут отбирать мел друг у друга,
или же спрашивать учителя: кто в данный момент должен писать.
Возникает очевидный вопрос: «почему мел один?». Предположим,
что два ученика попытаются на одной доске переместить одну и ту же
точку в разные места. В таком случае эта точка раздвоится, чего не должно быть.

Исходя из вышесказанного, объединим пример и нашу задачу: нам нужна комната, в которой поочередно
будет передаваться право (аналог мела как в примере) на редактирование элементов в инструменте «GeoGebra».

Организуем два типа таких прав.

\noindent
В первом случае, редактировать элементы может любой желающий, просто забрав на это право,
назовем такую систему свободной (free). Такой вариант хорошо подойдет для совместного обсуждения какой-либо задачи,
например, два студента пытаются построить кривую второго порядка, поочередно дополняя друг друга. 

\noindent
Во втором случае, правом на редактирование элементов в инструменте «GeoGebra» будет управлять только
владелец комнаты. Назовем такую систему ограниченной (restricted). Приведем пример: онлайн-урок
по геометрии, учитель просит желающих «выйти к доске» и провести высоту треугольника. Один и учеников
«тянет руку», учитель передает ему право на редактирование элементов на доске. После того, как ученик выполнил задачу —
учитель забирает право на редактирование.

Подводя итоги, получаем систему комнат с двумя типами прав:
свободные и ограниченные (free и restricted).

Свободная система прав (Рис.~\ref{fig:img1}) не предполагает наличие владельца комнаты,
каждый участник может забрать права на редактирование доски без
согласования с другими участниками.

\newpage
\begin{figure}[h]
	\begin{center}
		\scalebox{0.6}{%LaTeX with PSTricks extensions
%%Creator: Inkscape 1.2 (dc2aedaf03, 2022-05-15)
%%Please note this file requires PSTricks extensions
\psset{xunit=.5pt,yunit=.5pt,runit=.5pt}
\begin{pspicture}(1024,768)
{
\newrgbcolor{curcolor}{0.50196081 0.50196081 0.50196081}
\pscustom[linestyle=none,fillstyle=solid,fillcolor=curcolor]
{
\newpath
\moveto(619.24689484,370.52987671)
\curveto(619.24689484,313.45365043)(572.9774714,267.18422699)(515.90124512,267.18422699)
\curveto(458.82501884,267.18422699)(412.5555954,313.45365043)(412.5555954,370.52987671)
\curveto(412.5555954,427.60610299)(458.82501884,473.87552643)(515.90124512,473.87552643)
\curveto(572.9774714,473.87552643)(619.24689484,427.60610299)(619.24689484,370.52987671)
\closepath
}
}
{
\newrgbcolor{curcolor}{0.50196081 0.50196081 0.50196081}
\pscustom[linestyle=none,fillstyle=solid,fillcolor=curcolor]
{
\newpath
\moveto(59.77836227,746.7229557)
\lineto(586.63853073,746.7229557)
\lineto(586.63853073,654.5224247)
\lineto(59.77836227,654.5224247)
\closepath
}
}
{
\newrgbcolor{curcolor}{0.50196081 0.50196081 0.50196081}
\pscustom[linestyle=none,fillstyle=solid,fillcolor=curcolor]
{
\newpath
\moveto(422.50131226,622.6068573)
\lineto(949.36148071,622.6068573)
\lineto(949.36148071,530.40632629)
\lineto(422.50131226,530.40632629)
\closepath
}
}
{
\newrgbcolor{curcolor}{0.50196081 0.50196081 0.50196081}
\pscustom[linestyle=none,fillstyle=solid,fillcolor=curcolor]
{
\newpath
\moveto(434.65960693,109.93139648)
\lineto(961.51977539,109.93139648)
\lineto(961.51977539,17.73086548)
\lineto(434.65960693,17.73086548)
\closepath
}
}
{
\newrgbcolor{curcolor}{0.50196081 0.50196081 0.50196081}
\pscustom[linestyle=none,fillstyle=solid,fillcolor=curcolor]
{
\newpath
\moveto(48.63323975,231.51452637)
\lineto(575.4934082,231.51452637)
\lineto(575.4934082,139.31399536)
\lineto(48.63323975,139.31399536)
\closepath
}
}
{
\newrgbcolor{curcolor}{0.50196081 0.50196081 0.50196081}
\pscustom[linestyle=none,fillstyle=solid,fillcolor=curcolor]
{
\newpath
\moveto(736.59100342,523.3139801)
\lineto(746.72292995,523.3139801)
\lineto(746.72292995,407.81002045)
\lineto(736.59100342,407.81002045)
\closepath
}
}
{
\newrgbcolor{curcolor}{0.50196081 0.50196081 0.50196081}
\pscustom[linestyle=none,fillstyle=solid,fillcolor=curcolor]
{
\newpath
\moveto(626.38939551,407.83844403)
\lineto(626.39160082,417.97037032)
\lineto(741.89555774,417.94522973)
\lineto(741.89335243,407.81330343)
\closepath
}
}
{
\newrgbcolor{curcolor}{0.50196081 0.50196081 0.50196081}
\pscustom[linestyle=none,fillstyle=solid,fillcolor=curcolor]
{
\newpath
\moveto(295.50967302,239.25205031)
\lineto(285.37774798,239.24654896)
\lineto(285.31503257,354.75049159)
\lineto(295.44695762,354.75599294)
\closepath
}
}
{
\newrgbcolor{curcolor}{0.50196081 0.50196081 0.50196081}
\pscustom[linestyle=none,fillstyle=solid,fillcolor=curcolor]
{
\newpath
\moveto(405.64860293,354.78742531)
\lineto(405.65189968,344.65549931)
\lineto(290.14794614,344.61791643)
\lineto(290.14464939,354.74984243)
\closepath
}
}
{
\newrgbcolor{curcolor}{0.50196081 0.50196081 0.50196081}
\pscustom[linestyle=none,fillstyle=solid,fillcolor=curcolor]
{
\newpath
\moveto(214.34463501,647.33656311)
\lineto(224.37473011,647.33656311)
\lineto(224.37473011,430.96183777)
\lineto(214.34463501,430.96183777)
\closepath
}
}
{
\newrgbcolor{curcolor}{0.50196081 0.50196081 0.50196081}
\pscustom[linestyle=none,fillstyle=solid,fillcolor=curcolor]
{
\newpath
\moveto(214.34977419,423.04143657)
\lineto(214.34515458,433.0715306)
\lineto(414.9470315,433.16392281)
\lineto(414.95165111,423.13382877)
\closepath
}
}
{
\newrgbcolor{curcolor}{0.50196081 0.50196081 0.50196081}
\pscustom[linestyle=none,fillstyle=solid,fillcolor=curcolor]
{
\newpath
\moveto(829.30119952,117.05295297)
\lineto(819.27117698,117.01480088)
\lineto(818.44813915,333.3879609)
\lineto(828.47816169,333.42611299)
\closepath
}
}
{
\newrgbcolor{curcolor}{0.50196081 0.50196081 0.50196081}
\pscustom[linestyle=none,fillstyle=solid,fillcolor=curcolor]
{
\newpath
\moveto(828.44294551,341.34642847)
\lineto(828.485717,331.31642457)
\lineto(627.88564272,330.46099486)
\lineto(627.84287123,340.49099876)
\closepath
}
}
{
\newrgbcolor{curcolor}{0 0 0}
\pscustom[linestyle=none,fillstyle=solid,fillcolor=curcolor]
{
\newpath
\moveto(449.84000286,356.17862991)
\lineto(449.84000286,386.64265261)
\lineto(473.94668749,386.64265261)
\lineto(473.94668749,356.17862991)
\lineto(467.50401602,356.17862991)
\lineto(467.50401602,381.2666486)
\lineto(456.28267433,381.2666486)
\lineto(456.28267433,356.17862991)
\closepath
}
}
{
\newrgbcolor{curcolor}{0 0 0}
\pscustom[linestyle=none,fillstyle=solid,fillcolor=curcolor]
{
\newpath
\moveto(494.08540481,379.90131425)
\curveto(496.70229565,379.90131425)(498.82140834,378.87731349)(500.44274288,376.82931196)
\curveto(502.06407742,374.8097549)(502.87474469,371.82308601)(502.87474469,367.86930529)
\curveto(502.87474469,363.8870801)(502.03563295,360.87196674)(500.35740948,358.82396522)
\curveto(498.67918601,356.77596369)(496.53162885,355.75196293)(493.91473801,355.75196293)
\curveto(492.23651454,355.75196293)(490.89962466,356.05062982)(489.90406836,356.64796359)
\curveto(488.90851206,357.27374184)(488.09784479,357.97063125)(487.47206655,358.73863182)
\lineto(487.13073296,358.73863182)
\curveto(487.35828868,357.54396426)(487.47206655,356.40618564)(487.47206655,355.32529594)
\lineto(487.47206655,345.93862228)
\lineto(481.11472848,345.93862228)
\lineto(481.11472848,379.47464727)
\lineto(486.27739899,379.47464727)
\lineto(487.17339966,376.44531168)
\lineto(487.47206655,376.44531168)
\curveto(488.09784479,377.38397904)(488.93695653,378.19464631)(489.98940176,378.87731349)
\curveto(491.04184698,379.55998066)(492.40718134,379.90131425)(494.08540481,379.90131425)
\closepath
\moveto(492.03740328,374.82397714)
\curveto(490.38762428,374.82397714)(489.22140118,374.29775452)(488.53873401,373.24530929)
\curveto(487.85606683,372.22130853)(487.50051101,370.67108515)(487.47206655,368.59463916)
\lineto(487.47206655,367.91197199)
\curveto(487.47206655,365.6648592)(487.7991779,363.9297468)(488.45340061,362.70663477)
\curveto(489.13606779,361.51196722)(490.35917981,360.91463344)(492.12273668,360.91463344)
\curveto(493.57340443,360.91463344)(494.64007189,361.51196722)(495.32273906,362.70663477)
\curveto(496.0338507,363.9297468)(496.38940652,365.67908143)(496.38940652,367.95463868)
\curveto(496.38940652,372.53419765)(494.93873878,374.82397714)(492.03740328,374.82397714)
\closepath
}
}
{
\newrgbcolor{curcolor}{0 0 0}
\pscustom[linestyle=none,fillstyle=solid,fillcolor=curcolor]
{
\newpath
\moveto(517.68006644,379.94398095)
\curveto(520.80895766,379.94398095)(523.19829278,379.26131378)(524.84807178,377.89597942)
\curveto(526.52629526,376.55908954)(527.36540699,374.49686578)(527.36540699,371.70930815)
\lineto(527.36540699,356.17862991)
\lineto(522.92807035,356.17862991)
\lineto(521.6907361,359.3359656)
\lineto(521.52006931,359.3359656)
\curveto(520.52451301,358.08440911)(519.47206778,357.17418621)(518.36273362,356.6052969)
\curveto(517.25339946,356.03640758)(515.73162055,355.75196293)(513.79739688,355.75196293)
\curveto(511.72095089,355.75196293)(510.00006072,356.3492967)(508.63472637,357.54396426)
\curveto(507.26939202,358.73863182)(506.58672485,360.60174432)(506.58672485,363.13330176)
\curveto(506.58672485,365.60797027)(507.45428105,367.42841607)(509.18939345,368.59463916)
\curveto(510.92450586,369.76086225)(513.52717446,370.41508496)(516.99739927,370.55730729)
\lineto(521.05073562,370.68530739)
\lineto(521.05073562,371.70930815)
\curveto(521.05073562,372.93242017)(520.72362427,373.82842084)(520.06940156,374.39731015)
\curveto(519.44362331,374.96619946)(518.56184488,375.25064412)(517.42406625,375.25064412)
\curveto(516.28628763,375.25064412)(515.17695347,375.07997733)(514.09606377,374.73864374)
\curveto(513.01517408,374.42575462)(511.93428439,374.0275321)(510.85339469,373.54397618)
\lineto(508.76272647,377.85331273)
\curveto(509.98583849,378.47909097)(511.36539507,378.97686912)(512.90139622,379.34664717)
\curveto(514.43739736,379.74486969)(516.03028744,379.94398095)(517.68006644,379.94398095)
\closepath
\moveto(521.05073562,366.97330462)
\lineto(518.57606711,366.88797122)
\curveto(516.52806559,366.83108229)(515.1058423,366.46130424)(514.30939727,365.77863706)
\curveto(513.51295223,365.09596989)(513.11472971,364.19996922)(513.11472971,363.09063506)
\curveto(513.11472971,362.12352323)(513.39917437,361.42663382)(513.96806368,360.99996684)
\curveto(514.53695299,360.60174432)(515.2765091,360.40263306)(516.186732,360.40263306)
\curveto(517.55206635,360.40263306)(518.70406721,360.80085558)(519.64273457,361.59730061)
\curveto(520.58140194,362.42219012)(521.05073562,363.57419098)(521.05073562,365.05330319)
\closepath
}
}
{
\newrgbcolor{curcolor}{0 0 0}
\pscustom[linestyle=none,fillstyle=solid,fillcolor=curcolor]
{
\newpath
\moveto(554.54409896,373.37330939)
\curveto(554.54409896,372.1217529)(554.14587644,371.05508544)(553.3494314,370.173307)
\curveto(552.58143083,369.29152857)(551.42942997,368.72263926)(549.89342883,368.46663907)
\lineto(549.89342883,368.29597227)
\curveto(551.51476337,368.09686101)(552.80898656,367.5279717)(553.77609839,366.58930433)
\curveto(554.77165469,365.67908143)(555.26943283,364.52708058)(555.26943283,363.13330176)
\curveto(555.26943283,361.79641187)(554.91387701,360.60174432)(554.20276537,359.54929909)
\curveto(553.5200982,358.49685386)(552.42498627,357.67196436)(550.91742959,357.07463058)
\curveto(549.40987291,356.4772968)(547.43298255,356.17862991)(544.98675851,356.17862991)
\lineto(533.89341691,356.17862991)
\lineto(533.89341691,379.47464727)
\lineto(544.98675851,379.47464727)
\curveto(546.80720431,379.47464727)(548.42853885,379.27553601)(549.85076213,378.87731349)
\curveto(551.30142988,378.50753544)(552.4392085,377.86753496)(553.26409801,376.95731206)
\curveto(554.11743198,376.07553362)(554.54409896,374.88086607)(554.54409896,373.37330939)
\closepath
\moveto(548.10142749,372.86130901)
\curveto(548.10142749,374.28353229)(546.9778711,374.99464393)(544.73075832,374.99464393)
\lineto(540.25075498,374.99464393)
\lineto(540.25075498,370.3866405)
\lineto(544.00542444,370.3866405)
\curveto(545.34231433,370.3866405)(546.35209286,370.57152952)(547.03476003,370.94130758)
\curveto(547.74587167,371.3395301)(548.10142749,371.97953057)(548.10142749,372.86130901)
\closepath
\moveto(548.69876127,363.47463535)
\curveto(548.69876127,364.38485825)(548.32898322,365.03908096)(547.58942711,365.43730348)
\curveto(546.87831547,365.86397046)(545.82587024,366.07730395)(544.43209143,366.07730395)
\lineto(540.25075498,366.07730395)
\lineto(540.25075498,360.57329985)
\lineto(544.56009152,360.57329985)
\curveto(545.75475908,360.57329985)(546.73609314,360.78663334)(547.50409372,361.21330033)
\curveto(548.30053875,361.66841178)(548.69876127,362.42219012)(548.69876127,363.47463535)
\closepath
}
}
{
\newrgbcolor{curcolor}{0 0 0}
\pscustom[linestyle=none,fillstyle=solid,fillcolor=curcolor]
{
\newpath
\moveto(570.28809867,379.94398095)
\curveto(573.41698989,379.94398095)(575.806325,379.26131378)(577.45610401,377.89597942)
\curveto(579.13432748,376.55908954)(579.97343922,374.49686578)(579.97343922,371.70930815)
\lineto(579.97343922,356.17862991)
\lineto(575.53610258,356.17862991)
\lineto(574.29876833,359.3359656)
\lineto(574.12810153,359.3359656)
\curveto(573.13254523,358.08440911)(572.08010001,357.17418621)(570.97076585,356.6052969)
\curveto(569.86143169,356.03640758)(568.33965277,355.75196293)(566.40542911,355.75196293)
\curveto(564.32898312,355.75196293)(562.60809295,356.3492967)(561.2427586,357.54396426)
\curveto(559.87742425,358.73863182)(559.19475707,360.60174432)(559.19475707,363.13330176)
\curveto(559.19475707,365.60797027)(560.06231327,367.42841607)(561.79742568,368.59463916)
\curveto(563.53253808,369.76086225)(566.13520669,370.41508496)(569.6054315,370.55730729)
\lineto(573.65876785,370.68530739)
\lineto(573.65876785,371.70930815)
\curveto(573.65876785,372.93242017)(573.33165649,373.82842084)(572.67743378,374.39731015)
\curveto(572.05165554,374.96619946)(571.16987711,375.25064412)(570.03209848,375.25064412)
\curveto(568.89431985,375.25064412)(567.78498569,375.07997733)(566.704096,374.73864374)
\curveto(565.62320631,374.42575462)(564.54231661,374.0275321)(563.46142692,373.54397618)
\lineto(561.37075869,377.85331273)
\curveto(562.59387072,378.47909097)(563.9734273,378.97686912)(565.50942844,379.34664717)
\curveto(567.04542959,379.74486969)(568.63831966,379.94398095)(570.28809867,379.94398095)
\closepath
\moveto(573.65876785,366.97330462)
\lineto(571.18409934,366.88797122)
\curveto(569.13609781,366.83108229)(567.71387453,366.46130424)(566.91742949,365.77863706)
\curveto(566.12098446,365.09596989)(565.72276194,364.19996922)(565.72276194,363.09063506)
\curveto(565.72276194,362.12352323)(566.00720659,361.42663382)(566.57609591,360.99996684)
\curveto(567.14498522,360.60174432)(567.88454132,360.40263306)(568.79476423,360.40263306)
\curveto(570.16009858,360.40263306)(571.31209943,360.80085558)(572.2507668,361.59730061)
\curveto(573.18943417,362.42219012)(573.65876785,363.57419098)(573.65876785,365.05330319)
\closepath
}
}
{
\newrgbcolor{curcolor}{0 0 0}
\pscustom[linestyle=none,fillstyle=solid,fillcolor=curcolor]
{
\newpath
\moveto(139.28802108,716.64265261)
\lineto(129.85868072,695.09596989)
\curveto(129.00534675,693.13330176)(128.09512385,691.45507829)(127.12801202,690.06129947)
\curveto(126.16090018,688.66752065)(124.93778816,687.60085319)(123.45867595,686.86129709)
\curveto(122.0080082,686.12174098)(120.08800677,685.75196293)(117.69867166,685.75196293)
\curveto(116.95911555,685.75196293)(116.14844828,685.80885186)(115.26666985,685.92262972)
\curveto(114.38489141,686.03640758)(113.57422414,686.19285214)(112.83466803,686.3919634)
\lineto(112.83466803,691.9386342)
\curveto(113.51733521,691.65418955)(114.25689132,691.45507829)(115.05333635,691.34130042)
\curveto(115.87822586,691.22752256)(116.66044866,691.17063363)(117.40000477,691.17063363)
\curveto(118.82222805,691.17063363)(119.84622881,691.51196722)(120.47200706,692.19463439)
\curveto(121.0977853,692.90574603)(121.59556345,693.75908)(121.9653415,694.7546363)
\lineto(111.42666698,716.64265261)
\lineto(118.25333874,716.64265261)
\lineto(123.92800963,703.45864279)
\curveto(124.12712089,703.0319758)(124.39734332,702.42041979)(124.7386769,701.62397475)
\curveto(125.08001049,700.85597418)(125.33601068,700.20175147)(125.50667747,699.66130662)
\lineto(125.72001097,699.66130662)
\curveto(125.89067776,700.173307)(126.13245572,700.84175195)(126.44534484,701.66664145)
\curveto(126.78667843,702.49153095)(127.08534532,703.21686483)(127.34134551,703.84264307)
\lineto(132.63201612,716.64265261)
\closepath
}
}
{
\newrgbcolor{curcolor}{0 0 0}
\pscustom[linestyle=none,fillstyle=solid,fillcolor=curcolor]
{
\newpath
\moveto(147.35201138,709.47464727)
\lineto(147.35201138,700.94130758)
\curveto(147.35201138,698.92175052)(148.29067875,697.91197199)(150.16801348,697.91197199)
\curveto(151.3911255,697.91197199)(152.52890413,698.03997208)(153.58134935,698.29597227)
\curveto(154.63379458,698.58041693)(155.68623981,698.95019498)(156.73868504,699.40530643)
\lineto(156.73868504,709.47464727)
\lineto(163.09602311,709.47464727)
\lineto(163.09602311,686.17862991)
\lineto(156.73868504,686.17862991)
\lineto(156.73868504,695.43730348)
\curveto(155.74312874,694.89685863)(154.60535012,694.39908048)(153.32534916,693.94396903)
\curveto(152.04534821,693.51730205)(150.59468046,693.30396855)(148.97334592,693.30396855)
\curveto(146.55556634,693.30396855)(144.62134268,693.91552456)(143.17067493,695.13863659)
\curveto(141.72000718,696.39019307)(140.99467331,698.28175004)(140.99467331,700.81330748)
\lineto(140.99467331,709.47464727)
\closepath
}
}
{
\newrgbcolor{curcolor}{0 0 0}
\pscustom[linestyle=none,fillstyle=solid,fillcolor=curcolor]
{
\newpath
\moveto(179.30937705,709.94398095)
\curveto(182.43826827,709.94398095)(184.82760339,709.26131378)(186.47738239,707.89597942)
\curveto(188.15560587,706.55908954)(188.9947176,704.49686578)(188.9947176,701.70930815)
\lineto(188.9947176,686.17862991)
\lineto(184.55738096,686.17862991)
\lineto(183.32004671,689.3359656)
\lineto(183.14937991,689.3359656)
\curveto(182.15382362,688.08440911)(181.10137839,687.17418621)(179.99204423,686.6052969)
\curveto(178.88271007,686.03640758)(177.36093116,685.75196293)(175.42670749,685.75196293)
\curveto(173.3502615,685.75196293)(171.62937133,686.3492967)(170.26403698,687.54396426)
\curveto(168.89870263,688.73863182)(168.21603545,690.60174432)(168.21603545,693.13330176)
\curveto(168.21603545,695.60797027)(169.08359166,697.42841607)(170.81870406,698.59463916)
\curveto(172.55381646,699.76086225)(175.15648507,700.41508496)(178.62670988,700.55730729)
\lineto(182.68004623,700.68530739)
\lineto(182.68004623,701.70930815)
\curveto(182.68004623,702.93242017)(182.35293488,703.82842084)(181.69871217,704.39731015)
\curveto(181.07293392,704.96619946)(180.19115549,705.25064412)(179.05337686,705.25064412)
\curveto(177.91559824,705.25064412)(176.80626408,705.07997733)(175.72537438,704.73864374)
\curveto(174.64448469,704.42575462)(173.56359499,704.0275321)(172.4827053,703.54397618)
\lineto(170.39203708,707.85331273)
\curveto(171.6151491,708.47909097)(172.99470568,708.97686912)(174.53070683,709.34664717)
\curveto(176.06670797,709.74486969)(177.65959805,709.94398095)(179.30937705,709.94398095)
\closepath
\moveto(182.68004623,696.97330462)
\lineto(180.20537772,696.88797122)
\curveto(178.15737619,696.83108229)(176.73515291,696.46130424)(175.93870787,695.77863706)
\curveto(175.14226284,695.09596989)(174.74404032,694.19996922)(174.74404032,693.09063506)
\curveto(174.74404032,692.12352323)(175.02848497,691.42663382)(175.59737429,690.99996684)
\curveto(176.1662636,690.60174432)(176.90581971,690.40263306)(177.81604261,690.40263306)
\curveto(179.18137696,690.40263306)(180.33337782,690.80085558)(181.27204518,691.59730061)
\curveto(182.21071255,692.42219012)(182.68004623,693.57419098)(182.68004623,695.05330319)
\closepath
}
}
{
\newrgbcolor{curcolor}{0 0 0}
\pscustom[linestyle=none,fillstyle=solid,fillcolor=curcolor]
{
\newpath
\moveto(204.99472694,685.75196293)
\curveto(201.52450214,685.75196293)(198.83650013,686.70485253)(196.93072094,688.61063172)
\curveto(195.0533862,690.51641092)(194.11471884,693.54574651)(194.11471884,697.69863849)
\curveto(194.11471884,700.54308506)(194.59827475,702.86130901)(195.56538659,704.65331034)
\curveto(196.53249842,706.44531168)(197.8693883,707.76797933)(199.57605624,708.6213133)
\curveto(201.31116864,709.47464727)(203.30228124,709.90131425)(205.54939402,709.90131425)
\curveto(207.1422841,709.90131425)(208.52184068,709.74486969)(209.68806377,709.43198057)
\curveto(210.88273133,709.11909145)(211.92095433,708.74931339)(212.80273276,708.32264641)
\lineto(210.92539803,703.41597609)
\curveto(209.92984173,703.81419861)(208.99117437,704.14130996)(208.10939593,704.39731015)
\curveto(207.25606196,704.65331034)(206.40272799,704.78131044)(205.54939402,704.78131044)
\curveto(202.24983601,704.78131044)(200.600057,702.43464202)(200.600057,697.74130519)
\curveto(200.600057,695.40885901)(201.02672399,693.68796884)(201.88005796,692.57863468)
\curveto(202.76183639,691.46930052)(203.98494841,690.91463344)(205.54939402,690.91463344)
\curveto(206.88628391,690.91463344)(208.06672923,691.08530023)(209.09073,691.42663382)
\curveto(210.11473076,691.79641187)(211.11028706,692.29419002)(212.07739889,692.91996827)
\lineto(212.07739889,687.50129756)
\curveto(211.11028706,686.87551932)(210.08628629,686.4346301)(209.0053966,686.17862991)
\curveto(207.95295137,685.89418525)(206.61606149,685.75196293)(204.99472694,685.75196293)
\closepath
}
}
{
\newrgbcolor{curcolor}{0 0 0}
\pscustom[linestyle=none,fillstyle=solid,fillcolor=curcolor]
{
\newpath
\moveto(236.73875454,704.69597704)
\lineto(229.10141552,704.69597704)
\lineto(229.10141552,686.17862991)
\lineto(222.74407745,686.17862991)
\lineto(222.74407745,704.69597704)
\lineto(215.10673842,704.69597704)
\lineto(215.10673842,709.47464727)
\lineto(236.73875454,709.47464727)
\closepath
}
}
{
\newrgbcolor{curcolor}{0 0 0}
\pscustom[linestyle=none,fillstyle=solid,fillcolor=curcolor]
{
\newpath
\moveto(247.40540281,709.47464727)
\lineto(247.40540281,700.51464059)
\lineto(256.28007608,700.51464059)
\lineto(256.28007608,709.47464727)
\lineto(262.63741415,709.47464727)
\lineto(262.63741415,686.17862991)
\lineto(256.28007608,686.17862991)
\lineto(256.28007608,695.77863706)
\lineto(247.40540281,695.77863706)
\lineto(247.40540281,686.17862991)
\lineto(241.04806474,686.17862991)
\lineto(241.04806474,709.47464727)
\closepath
}
}
{
\newrgbcolor{curcolor}{0 0 0}
\pscustom[linestyle=none,fillstyle=solid,fillcolor=curcolor]
{
\newpath
\moveto(275.43746116,709.47464727)
\lineto(275.43746116,700.2586404)
\curveto(275.43746116,699.77508449)(275.40901669,699.17775071)(275.35212776,698.46663907)
\curveto(275.3236833,697.75552743)(275.2810166,697.03019355)(275.22412767,696.29063744)
\curveto(275.1956832,695.55108134)(275.1530165,694.8826364)(275.09612757,694.28530262)
\curveto(275.03923864,693.7164133)(274.99657194,693.33241302)(274.96812748,693.13330176)
\lineto(285.72013549,709.47464727)
\lineto(293.35747451,709.47464727)
\lineto(293.35747451,686.17862991)
\lineto(287.21346993,686.17862991)
\lineto(287.21346993,695.47997017)
\curveto(287.21346993,696.21952628)(287.2419144,697.05863802)(287.29880333,697.99730538)
\curveto(287.35569226,698.93597275)(287.41258119,699.80352895)(287.46947012,700.59997399)
\curveto(287.55480352,701.42486349)(287.61169245,702.05064174)(287.64013692,702.47730872)
\lineto(276.9307956,686.17862991)
\lineto(269.29345658,686.17862991)
\lineto(269.29345658,709.47464727)
\closepath
}
}
{
\newrgbcolor{curcolor}{0 0 0}
\pscustom[linestyle=none,fillstyle=solid,fillcolor=curcolor]
{
\newpath
\moveto(315.28812059,709.47464727)
\lineto(322.28545914,709.47464727)
\lineto(313.06945227,698.29597227)
\lineto(323.09612641,686.17862991)
\lineto(315.88545437,686.17862991)
\lineto(306.37078061,697.99730538)
\lineto(306.37078061,686.17862991)
\lineto(300.01344254,686.17862991)
\lineto(300.01344254,709.47464727)
\lineto(306.37078061,709.47464727)
\lineto(306.37078061,698.16797218)
\closepath
}
}
{
\newrgbcolor{curcolor}{0 0 0}
\pscustom[linestyle=none,fillstyle=solid,fillcolor=curcolor]
{
\newpath
\moveto(352.79213365,709.47464727)
\lineto(359.7894722,709.47464727)
\lineto(350.57346533,698.29597227)
\lineto(360.60013947,686.17862991)
\lineto(353.38946743,686.17862991)
\lineto(343.87479367,697.99730538)
\lineto(343.87479367,686.17862991)
\lineto(337.5174556,686.17862991)
\lineto(337.5174556,709.47464727)
\lineto(343.87479367,709.47464727)
\lineto(343.87479367,698.16797218)
\closepath
}
}
{
\newrgbcolor{curcolor}{0 0 0}
\pscustom[linestyle=none,fillstyle=solid,fillcolor=curcolor]
{
\newpath
\moveto(384.23745005,697.86930529)
\curveto(384.23745005,694.00085796)(383.21344929,691.01418907)(381.16544776,688.90929861)
\curveto(379.1458907,686.80440815)(376.38677753,685.75196293)(372.88810826,685.75196293)
\curveto(370.72632887,685.75196293)(368.79210521,686.22129661)(367.08543727,687.15996398)
\curveto(365.4072138,688.09863134)(364.08454615,689.46396569)(363.11743431,691.25596703)
\curveto(362.15032248,693.07641283)(361.66676657,695.28085891)(361.66676657,697.86930529)
\curveto(361.66676657,701.73775261)(362.6765451,704.71019927)(364.69610216,706.78664527)
\curveto(366.71565922,708.86309126)(369.48899462,709.90131425)(373.01610836,709.90131425)
\curveto(375.20633221,709.90131425)(377.14055587,709.43198057)(378.81877935,708.4933132)
\curveto(380.49700282,707.55464584)(381.81967047,706.18931149)(382.7867823,704.39731015)
\curveto(383.75389413,702.60530882)(384.23745005,700.4293072)(384.23745005,697.86930529)
\closepath
\moveto(368.15210473,697.86930529)
\curveto(368.15210473,695.56530357)(368.52188279,693.81596893)(369.26143889,692.62130138)
\curveto(370.02943946,691.45507829)(371.26677372,690.87196674)(372.97344166,690.87196674)
\curveto(374.65166513,690.87196674)(375.86055492,691.45507829)(376.60011103,692.62130138)
\curveto(377.3681116,693.81596893)(377.75211189,695.56530357)(377.75211189,697.86930529)
\curveto(377.75211189,700.173307)(377.3681116,701.89419718)(376.60011103,703.0319758)
\curveto(375.86055492,704.19819889)(374.6374429,704.78131044)(372.93077496,704.78131044)
\curveto(371.25255149,704.78131044)(370.02943946,704.19819889)(369.26143889,703.0319758)
\curveto(368.52188279,701.89419718)(368.15210473,700.173307)(368.15210473,697.86930529)
\closepath
}
}
{
\newrgbcolor{curcolor}{0 0 0}
\pscustom[linestyle=none,fillstyle=solid,fillcolor=curcolor]
{
\newpath
\moveto(418.8401312,709.47464727)
\lineto(418.8401312,686.17862991)
\lineto(412.90946012,686.17862991)
\lineto(412.90946012,697.6133051)
\curveto(412.90946012,698.75108372)(412.92368235,699.86041788)(412.95212682,700.94130758)
\curveto(413.00901575,702.02219727)(413.08012691,703.01775357)(413.16546031,703.92797647)
\lineto(413.03746021,703.92797647)
\lineto(406.59478875,686.17862991)
\lineto(401.81611852,686.17862991)
\lineto(395.28811366,703.97064317)
\lineto(395.11744686,703.97064317)
\curveto(395.23122472,703.0319758)(395.30233589,702.02219727)(395.33078035,700.94130758)
\curveto(395.38766928,699.88886235)(395.41611375,698.72263926)(395.41611375,697.4426383)
\lineto(395.41611375,686.17862991)
\lineto(389.48544267,686.17862991)
\lineto(389.48544267,709.47464727)
\lineto(398.48811604,709.47464727)
\lineto(404.29078703,693.68796884)
\lineto(410.17879142,709.47464727)
\closepath
}
}
{
\newrgbcolor{curcolor}{0 0 0}
\pscustom[linestyle=none,fillstyle=solid,fillcolor=curcolor]
{
\newpath
\moveto(431.85346189,709.47464727)
\lineto(431.85346189,700.51464059)
\lineto(440.72813517,700.51464059)
\lineto(440.72813517,709.47464727)
\lineto(447.08547324,709.47464727)
\lineto(447.08547324,686.17862991)
\lineto(440.72813517,686.17862991)
\lineto(440.72813517,695.77863706)
\lineto(431.85346189,695.77863706)
\lineto(431.85346189,686.17862991)
\lineto(425.49612382,686.17862991)
\lineto(425.49612382,709.47464727)
\closepath
}
}
{
\newrgbcolor{curcolor}{0 0 0}
\pscustom[linestyle=none,fillstyle=solid,fillcolor=curcolor]
{
\newpath
\moveto(463.29885612,709.94398095)
\curveto(466.42774734,709.94398095)(468.81708245,709.26131378)(470.46686146,707.89597942)
\curveto(472.14508493,706.55908954)(472.98419667,704.49686578)(472.98419667,701.70930815)
\lineto(472.98419667,686.17862991)
\lineto(468.54686003,686.17862991)
\lineto(467.30952577,689.3359656)
\lineto(467.13885898,689.3359656)
\curveto(466.14330268,688.08440911)(465.09085745,687.17418621)(463.98152329,686.6052969)
\curveto(462.87218913,686.03640758)(461.35041022,685.75196293)(459.41618656,685.75196293)
\curveto(457.33974057,685.75196293)(455.6188504,686.3492967)(454.25351604,687.54396426)
\curveto(452.88818169,688.73863182)(452.20551452,690.60174432)(452.20551452,693.13330176)
\curveto(452.20551452,695.60797027)(453.07307072,697.42841607)(454.80818312,698.59463916)
\curveto(456.54329553,699.76086225)(459.14596413,700.41508496)(462.61618894,700.55730729)
\lineto(466.6695253,700.68530739)
\lineto(466.6695253,701.70930815)
\curveto(466.6695253,702.93242017)(466.34241394,703.82842084)(465.68819123,704.39731015)
\curveto(465.06241299,704.96619946)(464.18063455,705.25064412)(463.04285593,705.25064412)
\curveto(461.9050773,705.25064412)(460.79574314,705.07997733)(459.71485345,704.73864374)
\curveto(458.63396375,704.42575462)(457.55307406,704.0275321)(456.47218436,703.54397618)
\lineto(454.38151614,707.85331273)
\curveto(455.60462816,708.47909097)(456.98418475,708.97686912)(458.52018589,709.34664717)
\curveto(460.05618703,709.74486969)(461.64907711,709.94398095)(463.29885612,709.94398095)
\closepath
\moveto(466.6695253,696.97330462)
\lineto(464.19485678,696.88797122)
\curveto(462.14685526,696.83108229)(460.72463198,696.46130424)(459.92818694,695.77863706)
\curveto(459.1317419,695.09596989)(458.73351938,694.19996922)(458.73351938,693.09063506)
\curveto(458.73351938,692.12352323)(459.01796404,691.42663382)(459.58685335,690.99996684)
\curveto(460.15574266,690.60174432)(460.89529877,690.40263306)(461.80552167,690.40263306)
\curveto(463.17085602,690.40263306)(464.32285688,690.80085558)(465.26152425,691.59730061)
\curveto(466.20019161,692.42219012)(466.6695253,693.57419098)(466.6695253,695.05330319)
\closepath
}
}
{
\newrgbcolor{curcolor}{0 0 0}
\pscustom[linestyle=none,fillstyle=solid,fillcolor=curcolor]
{
\newpath
\moveto(498.79755428,704.69597704)
\lineto(491.16021526,704.69597704)
\lineto(491.16021526,686.17862991)
\lineto(484.80287719,686.17862991)
\lineto(484.80287719,704.69597704)
\lineto(477.16553817,704.69597704)
\lineto(477.16553817,709.47464727)
\lineto(498.79755428,709.47464727)
\closepath
}
}
{
\newrgbcolor{curcolor}{0 0 0}
\pscustom[linestyle=none,fillstyle=solid,fillcolor=curcolor]
{
\newpath
\moveto(503.10687211,686.17862991)
\lineto(503.10687211,709.47464727)
\lineto(509.46421018,709.47464727)
\lineto(509.46421018,700.47197389)
\lineto(512.53621247,700.47197389)
\curveto(516.09177067,700.47197389)(518.72288374,699.90308458)(520.42955168,698.76530596)
\curveto(522.13621962,697.62752733)(522.98955359,695.90663716)(522.98955359,693.60263544)
\curveto(522.98955359,691.32707819)(522.19310855,689.52085462)(520.60021848,688.18396474)
\curveto(519.0073284,686.84707485)(516.39043756,686.17862991)(512.74954596,686.17862991)
\closepath
\moveto(526.36022277,686.17862991)
\lineto(526.36022277,709.47464727)
\lineto(532.71756084,709.47464727)
\lineto(532.71756084,686.17862991)
\closepath
\moveto(509.46421018,690.57329985)
\lineto(512.40821237,690.57329985)
\curveto(513.65976886,690.57329985)(514.66954739,690.78663334)(515.43754796,691.21330033)
\curveto(516.233993,691.66841178)(516.63221552,692.43641235)(516.63221552,693.51730205)
\curveto(516.63221552,695.22396998)(515.19577,696.07730395)(512.32287897,696.07730395)
\lineto(509.46421018,696.07730395)
\closepath
}
}
{
\newrgbcolor{curcolor}{0 0 0}
\pscustom[linestyle=none,fillstyle=solid,fillcolor=curcolor]
{
\newpath
\moveto(512.21590108,590.44534261)
\lineto(502.78656072,568.89865989)
\curveto(501.93322675,566.93599176)(501.02300385,565.25776829)(500.05589202,563.86398947)
\curveto(499.08878018,562.47021065)(497.86566816,561.40354319)(496.38655595,560.66398709)
\curveto(494.9358882,559.92443098)(493.01588677,559.55465293)(490.62655166,559.55465293)
\curveto(489.88699555,559.55465293)(489.07632828,559.61154186)(488.19454985,559.72531972)
\curveto(487.31277141,559.83909758)(486.50210414,559.99554214)(485.76254803,560.1946534)
\lineto(485.76254803,565.7413242)
\curveto(486.44521521,565.45687955)(487.18477132,565.25776829)(487.98121635,565.14399042)
\curveto(488.80610586,565.03021256)(489.58832866,564.97332363)(490.32788477,564.97332363)
\curveto(491.75010805,564.97332363)(492.77410881,565.31465722)(493.39988706,565.99732439)
\curveto(494.0256653,566.70843603)(494.52344345,567.56177)(494.8932215,568.5573263)
\lineto(484.35454698,590.44534261)
\lineto(491.18121874,590.44534261)
\lineto(496.85588963,577.26133279)
\curveto(497.05500089,576.8346658)(497.32522332,576.22310979)(497.6665569,575.42666475)
\curveto(498.00789049,574.65866418)(498.26389068,574.00444147)(498.43455747,573.46399662)
\lineto(498.64789097,573.46399662)
\curveto(498.81855776,573.975997)(499.06033572,574.64444195)(499.37322484,575.46933145)
\curveto(499.71455843,576.29422095)(500.01322532,577.01955483)(500.26922551,577.64533307)
\lineto(505.55989612,590.44534261)
\closepath
}
}
{
\newrgbcolor{curcolor}{0 0 0}
\pscustom[linestyle=none,fillstyle=solid,fillcolor=curcolor]
{
\newpath
\moveto(520.27989138,583.27733727)
\lineto(520.27989138,574.74399758)
\curveto(520.27989138,572.72444052)(521.21855875,571.71466199)(523.09589348,571.71466199)
\curveto(524.3190055,571.71466199)(525.45678413,571.84266208)(526.50922935,572.09866227)
\curveto(527.56167458,572.38310693)(528.61411981,572.75288498)(529.66656504,573.20799643)
\lineto(529.66656504,583.27733727)
\lineto(536.02390311,583.27733727)
\lineto(536.02390311,559.98131991)
\lineto(529.66656504,559.98131991)
\lineto(529.66656504,569.23999348)
\curveto(528.67100874,568.69954863)(527.53323012,568.20177048)(526.25322916,567.74665903)
\curveto(524.97322821,567.31999205)(523.52256046,567.10665855)(521.90122592,567.10665855)
\curveto(519.48344634,567.10665855)(517.54922268,567.71821456)(516.09855493,568.94132659)
\curveto(514.64788718,570.19288307)(513.92255331,572.08444004)(513.92255331,574.61599748)
\lineto(513.92255331,583.27733727)
\closepath
}
}
{
\newrgbcolor{curcolor}{0 0 0}
\pscustom[linestyle=none,fillstyle=solid,fillcolor=curcolor]
{
\newpath
\moveto(552.23725705,583.74667095)
\curveto(555.36614827,583.74667095)(557.75548339,583.06400378)(559.40526239,581.69866942)
\curveto(561.08348587,580.36177954)(561.9225976,578.29955578)(561.9225976,575.51199815)
\lineto(561.9225976,559.98131991)
\lineto(557.48526096,559.98131991)
\lineto(556.24792671,563.1386556)
\lineto(556.07725991,563.1386556)
\curveto(555.08170362,561.88709911)(554.02925839,560.97687621)(552.91992423,560.4079869)
\curveto(551.81059007,559.83909758)(550.28881116,559.55465293)(548.35458749,559.55465293)
\curveto(546.2781415,559.55465293)(544.55725133,560.1519867)(543.19191698,561.34665426)
\curveto(541.82658263,562.54132182)(541.14391545,564.40443432)(541.14391545,566.93599176)
\curveto(541.14391545,569.41066027)(542.01147166,571.23110607)(543.74658406,572.39732916)
\curveto(545.48169646,573.56355225)(548.08436507,574.21777496)(551.55458988,574.35999729)
\lineto(555.60792623,574.48799739)
\lineto(555.60792623,575.51199815)
\curveto(555.60792623,576.73511017)(555.28081488,577.63111084)(554.62659217,578.20000015)
\curveto(554.00081392,578.76888946)(553.11903549,579.05333412)(551.98125686,579.05333412)
\curveto(550.84347824,579.05333412)(549.73414408,578.88266733)(548.65325438,578.54133374)
\curveto(547.57236469,578.22844462)(546.49147499,577.8302221)(545.4105853,577.34666618)
\lineto(543.31991708,581.65600273)
\curveto(544.5430291,582.28178097)(545.92258568,582.77955912)(547.45858683,583.14933717)
\curveto(548.99458797,583.54755969)(550.58747805,583.74667095)(552.23725705,583.74667095)
\closepath
\moveto(555.60792623,570.77599462)
\lineto(553.13325772,570.69066122)
\curveto(551.08525619,570.63377229)(549.66303291,570.26399424)(548.86658787,569.58132706)
\curveto(548.07014284,568.89865989)(547.67192032,568.00265922)(547.67192032,566.89332506)
\curveto(547.67192032,565.92621323)(547.95636497,565.22932382)(548.52525429,564.80265684)
\curveto(549.0941436,564.40443432)(549.83369971,564.20532306)(550.74392261,564.20532306)
\curveto(552.10925696,564.20532306)(553.26125782,564.60354558)(554.19992518,565.39999061)
\curveto(555.13859255,566.22488012)(555.60792623,567.37688098)(555.60792623,568.85599319)
\closepath
}
}
{
\newrgbcolor{curcolor}{0 0 0}
\pscustom[linestyle=none,fillstyle=solid,fillcolor=curcolor]
{
\newpath
\moveto(577.92260694,559.55465293)
\curveto(574.45238214,559.55465293)(571.76438013,560.50754253)(569.85860094,562.41332172)
\curveto(567.9812662,564.31910092)(567.04259884,567.34843651)(567.04259884,571.50132849)
\curveto(567.04259884,574.34577506)(567.52615475,576.66399901)(568.49326659,578.45600034)
\curveto(569.46037842,580.24800168)(570.7972683,581.57066933)(572.50393624,582.4240033)
\curveto(574.23904864,583.27733727)(576.23016124,583.70400425)(578.47727402,583.70400425)
\curveto(580.0701641,583.70400425)(581.44972068,583.54755969)(582.61594377,583.23467057)
\curveto(583.81061133,582.92178145)(584.84883433,582.55200339)(585.73061276,582.12533641)
\lineto(583.85327803,577.21866609)
\curveto(582.85772173,577.61688861)(581.91905437,577.94399996)(581.03727593,578.20000015)
\curveto(580.18394196,578.45600034)(579.33060799,578.58400044)(578.47727402,578.58400044)
\curveto(575.17771601,578.58400044)(573.527937,576.23733202)(573.527937,571.54399519)
\curveto(573.527937,569.21154901)(573.95460399,567.49065884)(574.80793796,566.38132468)
\curveto(575.68971639,565.27199052)(576.91282841,564.71732344)(578.47727402,564.71732344)
\curveto(579.81416391,564.71732344)(580.99460923,564.88799023)(582.01861,565.22932382)
\curveto(583.04261076,565.59910187)(584.03816706,566.09688002)(585.00527889,566.72265827)
\lineto(585.00527889,561.30398756)
\curveto(584.03816706,560.67820932)(583.01416629,560.2373201)(581.9332766,559.98131991)
\curveto(580.88083137,559.69687525)(579.54394149,559.55465293)(577.92260694,559.55465293)
\closepath
}
}
{
\newrgbcolor{curcolor}{0 0 0}
\pscustom[linestyle=none,fillstyle=solid,fillcolor=curcolor]
{
\newpath
\moveto(609.66663454,578.49866704)
\lineto(602.02929552,578.49866704)
\lineto(602.02929552,559.98131991)
\lineto(595.67195745,559.98131991)
\lineto(595.67195745,578.49866704)
\lineto(588.03461842,578.49866704)
\lineto(588.03461842,583.27733727)
\lineto(609.66663454,583.27733727)
\closepath
}
}
{
\newrgbcolor{curcolor}{0 0 0}
\pscustom[linestyle=none,fillstyle=solid,fillcolor=curcolor]
{
\newpath
\moveto(620.33328281,583.27733727)
\lineto(620.33328281,574.31733059)
\lineto(629.20795608,574.31733059)
\lineto(629.20795608,583.27733727)
\lineto(635.56529415,583.27733727)
\lineto(635.56529415,559.98131991)
\lineto(629.20795608,559.98131991)
\lineto(629.20795608,569.58132706)
\lineto(620.33328281,569.58132706)
\lineto(620.33328281,559.98131991)
\lineto(613.97594474,559.98131991)
\lineto(613.97594474,583.27733727)
\closepath
}
}
{
\newrgbcolor{curcolor}{0 0 0}
\pscustom[linestyle=none,fillstyle=solid,fillcolor=curcolor]
{
\newpath
\moveto(648.36534116,583.27733727)
\lineto(648.36534116,574.0613304)
\curveto(648.36534116,573.57777449)(648.33689669,572.98044071)(648.28000776,572.26932907)
\curveto(648.2515633,571.55821743)(648.2088966,570.83288355)(648.15200767,570.09332744)
\curveto(648.1235632,569.35377134)(648.0808965,568.6853264)(648.02400757,568.08799262)
\curveto(647.96711864,567.5191033)(647.92445194,567.13510302)(647.89600748,566.93599176)
\lineto(658.64801549,583.27733727)
\lineto(666.28535451,583.27733727)
\lineto(666.28535451,559.98131991)
\lineto(660.14134993,559.98131991)
\lineto(660.14134993,569.28266017)
\curveto(660.14134993,570.02221628)(660.1697944,570.86132802)(660.22668333,571.79999538)
\curveto(660.28357226,572.73866275)(660.34046119,573.60621895)(660.39735012,574.40266399)
\curveto(660.48268352,575.22755349)(660.53957245,575.85333174)(660.56801692,576.27999872)
\lineto(649.8586756,559.98131991)
\lineto(642.22133658,559.98131991)
\lineto(642.22133658,583.27733727)
\closepath
}
}
{
\newrgbcolor{curcolor}{0 0 0}
\pscustom[linestyle=none,fillstyle=solid,fillcolor=curcolor]
{
\newpath
\moveto(688.21600059,583.27733727)
\lineto(695.21333914,583.27733727)
\lineto(685.99733227,572.09866227)
\lineto(696.02400641,559.98131991)
\lineto(688.81333437,559.98131991)
\lineto(679.29866061,571.79999538)
\lineto(679.29866061,559.98131991)
\lineto(672.94132254,559.98131991)
\lineto(672.94132254,583.27733727)
\lineto(679.29866061,583.27733727)
\lineto(679.29866061,571.97066218)
\closepath
}
}
{
\newrgbcolor{curcolor}{0 0 0}
\pscustom[linestyle=none,fillstyle=solid,fillcolor=curcolor]
{
\newpath
\moveto(725.72001365,583.27733727)
\lineto(732.7173522,583.27733727)
\lineto(723.50134533,572.09866227)
\lineto(733.52801947,559.98131991)
\lineto(726.31734743,559.98131991)
\lineto(716.80267367,571.79999538)
\lineto(716.80267367,559.98131991)
\lineto(710.4453356,559.98131991)
\lineto(710.4453356,583.27733727)
\lineto(716.80267367,583.27733727)
\lineto(716.80267367,571.97066218)
\closepath
}
}
{
\newrgbcolor{curcolor}{0 0 0}
\pscustom[linestyle=none,fillstyle=solid,fillcolor=curcolor]
{
\newpath
\moveto(757.16533005,571.67199529)
\curveto(757.16533005,567.80354796)(756.14132929,564.81687907)(754.09332776,562.71198861)
\curveto(752.0737707,560.60709815)(749.31465753,559.55465293)(745.81598826,559.55465293)
\curveto(743.65420887,559.55465293)(741.71998521,560.02398661)(740.01331727,560.96265398)
\curveto(738.3350938,561.90132134)(737.01242615,563.26665569)(736.04531431,565.05865703)
\curveto(735.07820248,566.87910283)(734.59464657,569.08354891)(734.59464657,571.67199529)
\curveto(734.59464657,575.54044261)(735.6044251,578.51288927)(737.62398216,580.58933527)
\curveto(739.64353922,582.66578126)(742.41687462,583.70400425)(745.94398836,583.70400425)
\curveto(748.13421221,583.70400425)(750.06843587,583.23467057)(751.74665935,582.2960032)
\curveto(753.42488282,581.35733584)(754.74755047,579.99200149)(755.7146623,578.20000015)
\curveto(756.68177413,576.40799882)(757.16533005,574.2319972)(757.16533005,571.67199529)
\closepath
\moveto(741.07998473,571.67199529)
\curveto(741.07998473,569.36799357)(741.44976279,567.61865893)(742.18931889,566.42399138)
\curveto(742.95731946,565.25776829)(744.19465372,564.67465674)(745.90132166,564.67465674)
\curveto(747.57954513,564.67465674)(748.78843492,565.25776829)(749.52799103,566.42399138)
\curveto(750.2959916,567.61865893)(750.67999189,569.36799357)(750.67999189,571.67199529)
\curveto(750.67999189,573.975997)(750.2959916,575.69688718)(749.52799103,576.8346658)
\curveto(748.78843492,578.00088889)(747.5653229,578.58400044)(745.85865496,578.58400044)
\curveto(744.18043149,578.58400044)(742.95731946,578.00088889)(742.18931889,576.8346658)
\curveto(741.44976279,575.69688718)(741.07998473,573.975997)(741.07998473,571.67199529)
\closepath
}
}
{
\newrgbcolor{curcolor}{0 0 0}
\pscustom[linestyle=none,fillstyle=solid,fillcolor=curcolor]
{
\newpath
\moveto(791.7680112,583.27733727)
\lineto(791.7680112,559.98131991)
\lineto(785.83734012,559.98131991)
\lineto(785.83734012,571.4159951)
\curveto(785.83734012,572.55377372)(785.85156235,573.66310788)(785.88000682,574.74399758)
\curveto(785.93689575,575.82488727)(786.00800691,576.82044357)(786.09334031,577.73066647)
\lineto(785.96534021,577.73066647)
\lineto(779.52266875,559.98131991)
\lineto(774.74399852,559.98131991)
\lineto(768.21599366,577.77333317)
\lineto(768.04532686,577.77333317)
\curveto(768.15910472,576.8346658)(768.23021589,575.82488727)(768.25866035,574.74399758)
\curveto(768.31554928,573.69155235)(768.34399375,572.52532926)(768.34399375,571.2453283)
\lineto(768.34399375,559.98131991)
\lineto(762.41332267,559.98131991)
\lineto(762.41332267,583.27733727)
\lineto(771.41599604,583.27733727)
\lineto(777.21866703,567.49065884)
\lineto(783.10667142,583.27733727)
\closepath
}
}
{
\newrgbcolor{curcolor}{0 0 0}
\pscustom[linestyle=none,fillstyle=solid,fillcolor=curcolor]
{
\newpath
\moveto(804.78134189,583.27733727)
\lineto(804.78134189,574.31733059)
\lineto(813.65601517,574.31733059)
\lineto(813.65601517,583.27733727)
\lineto(820.01335324,583.27733727)
\lineto(820.01335324,559.98131991)
\lineto(813.65601517,559.98131991)
\lineto(813.65601517,569.58132706)
\lineto(804.78134189,569.58132706)
\lineto(804.78134189,559.98131991)
\lineto(798.42400382,559.98131991)
\lineto(798.42400382,583.27733727)
\closepath
}
}
{
\newrgbcolor{curcolor}{0 0 0}
\pscustom[linestyle=none,fillstyle=solid,fillcolor=curcolor]
{
\newpath
\moveto(836.22673612,583.74667095)
\curveto(839.35562734,583.74667095)(841.74496245,583.06400378)(843.39474146,581.69866942)
\curveto(845.07296493,580.36177954)(845.91207667,578.29955578)(845.91207667,575.51199815)
\lineto(845.91207667,559.98131991)
\lineto(841.47474003,559.98131991)
\lineto(840.23740577,563.1386556)
\lineto(840.06673898,563.1386556)
\curveto(839.07118268,561.88709911)(838.01873745,560.97687621)(836.90940329,560.4079869)
\curveto(835.80006913,559.83909758)(834.27829022,559.55465293)(832.34406656,559.55465293)
\curveto(830.26762057,559.55465293)(828.5467304,560.1519867)(827.18139604,561.34665426)
\curveto(825.81606169,562.54132182)(825.13339452,564.40443432)(825.13339452,566.93599176)
\curveto(825.13339452,569.41066027)(826.00095072,571.23110607)(827.73606312,572.39732916)
\curveto(829.47117553,573.56355225)(832.07384413,574.21777496)(835.54406894,574.35999729)
\lineto(839.5974053,574.48799739)
\lineto(839.5974053,575.51199815)
\curveto(839.5974053,576.73511017)(839.27029394,577.63111084)(838.61607123,578.20000015)
\curveto(837.99029299,578.76888946)(837.10851455,579.05333412)(835.97073593,579.05333412)
\curveto(834.8329573,579.05333412)(833.72362314,578.88266733)(832.64273345,578.54133374)
\curveto(831.56184375,578.22844462)(830.48095406,577.8302221)(829.40006436,577.34666618)
\lineto(827.30939614,581.65600273)
\curveto(828.53250816,582.28178097)(829.91206475,582.77955912)(831.44806589,583.14933717)
\curveto(832.98406703,583.54755969)(834.57695711,583.74667095)(836.22673612,583.74667095)
\closepath
\moveto(839.5974053,570.77599462)
\lineto(837.12273678,570.69066122)
\curveto(835.07473526,570.63377229)(833.65251198,570.26399424)(832.85606694,569.58132706)
\curveto(832.0596219,568.89865989)(831.66139938,568.00265922)(831.66139938,566.89332506)
\curveto(831.66139938,565.92621323)(831.94584404,565.22932382)(832.51473335,564.80265684)
\curveto(833.08362266,564.40443432)(833.82317877,564.20532306)(834.73340167,564.20532306)
\curveto(836.09873602,564.20532306)(837.25073688,564.60354558)(838.18940425,565.39999061)
\curveto(839.12807161,566.22488012)(839.5974053,567.37688098)(839.5974053,568.85599319)
\closepath
}
}
{
\newrgbcolor{curcolor}{0 0 0}
\pscustom[linestyle=none,fillstyle=solid,fillcolor=curcolor]
{
\newpath
\moveto(871.72543428,578.49866704)
\lineto(864.08809526,578.49866704)
\lineto(864.08809526,559.98131991)
\lineto(857.73075719,559.98131991)
\lineto(857.73075719,578.49866704)
\lineto(850.09341817,578.49866704)
\lineto(850.09341817,583.27733727)
\lineto(871.72543428,583.27733727)
\closepath
}
}
{
\newrgbcolor{curcolor}{0 0 0}
\pscustom[linestyle=none,fillstyle=solid,fillcolor=curcolor]
{
\newpath
\moveto(876.03475211,559.98131991)
\lineto(876.03475211,583.27733727)
\lineto(882.39209018,583.27733727)
\lineto(882.39209018,574.27466389)
\lineto(885.46409247,574.27466389)
\curveto(889.01965067,574.27466389)(891.65076374,573.70577458)(893.35743168,572.56799596)
\curveto(895.06409962,571.43021733)(895.91743359,569.70932716)(895.91743359,567.40532544)
\curveto(895.91743359,565.12976819)(895.12098855,563.32354462)(893.52809848,561.98665474)
\curveto(891.9352084,560.64976485)(889.31831756,559.98131991)(885.67742596,559.98131991)
\closepath
\moveto(899.28810277,559.98131991)
\lineto(899.28810277,583.27733727)
\lineto(905.64544084,583.27733727)
\lineto(905.64544084,559.98131991)
\closepath
\moveto(882.39209018,564.37598985)
\lineto(885.33609237,564.37598985)
\curveto(886.58764886,564.37598985)(887.59742739,564.58932334)(888.36542796,565.01599033)
\curveto(889.161873,565.47110178)(889.56009552,566.23910235)(889.56009552,567.31999205)
\curveto(889.56009552,569.02665998)(888.12365,569.87999395)(885.25075897,569.87999395)
\lineto(882.39209018,569.87999395)
\closepath
}
}
{
\newrgbcolor{curcolor}{0 0 0}
\pscustom[linestyle=none,fillstyle=solid,fillcolor=curcolor]
{
\newpath
\moveto(525.21590108,76.44535261)
\lineto(515.78656072,54.89866989)
\curveto(514.93322675,52.93600176)(514.02300385,51.25777829)(513.05589202,49.86399947)
\curveto(512.08878018,48.47022065)(510.86566816,47.40355319)(509.38655595,46.66399709)
\curveto(507.9358882,45.92444098)(506.01588677,45.55466293)(503.62655166,45.55466293)
\curveto(502.88699555,45.55466293)(502.07632828,45.61155186)(501.19454985,45.72532972)
\curveto(500.31277141,45.83910758)(499.50210414,45.99555214)(498.76254803,46.1946634)
\lineto(498.76254803,51.7413342)
\curveto(499.44521521,51.45688955)(500.18477132,51.25777829)(500.98121635,51.14400042)
\curveto(501.80610586,51.03022256)(502.58832866,50.97333363)(503.32788477,50.97333363)
\curveto(504.75010805,50.97333363)(505.77410881,51.31466722)(506.39988706,51.99733439)
\curveto(507.0256653,52.70844603)(507.52344345,53.56178)(507.8932215,54.5573363)
\lineto(497.35454698,76.44535261)
\lineto(504.18121874,76.44535261)
\lineto(509.85588963,63.26134279)
\curveto(510.05500089,62.8346758)(510.32522332,62.22311979)(510.6665569,61.42667475)
\curveto(511.00789049,60.65867418)(511.26389068,60.00445147)(511.43455747,59.46400662)
\lineto(511.64789097,59.46400662)
\curveto(511.81855776,59.976007)(512.06033572,60.64445195)(512.37322484,61.46934145)
\curveto(512.71455843,62.29423095)(513.01322532,63.01956483)(513.26922551,63.64534307)
\lineto(518.55989612,76.44535261)
\closepath
}
}
{
\newrgbcolor{curcolor}{0 0 0}
\pscustom[linestyle=none,fillstyle=solid,fillcolor=curcolor]
{
\newpath
\moveto(533.27989138,69.27734727)
\lineto(533.27989138,60.74400758)
\curveto(533.27989138,58.72445052)(534.21855875,57.71467199)(536.09589348,57.71467199)
\curveto(537.3190055,57.71467199)(538.45678413,57.84267208)(539.50922935,58.09867227)
\curveto(540.56167458,58.38311693)(541.61411981,58.75289498)(542.66656504,59.20800643)
\lineto(542.66656504,69.27734727)
\lineto(549.02390311,69.27734727)
\lineto(549.02390311,45.98132991)
\lineto(542.66656504,45.98132991)
\lineto(542.66656504,55.24000348)
\curveto(541.67100874,54.69955863)(540.53323012,54.20178048)(539.25322916,53.74666903)
\curveto(537.97322821,53.32000205)(536.52256046,53.10666855)(534.90122592,53.10666855)
\curveto(532.48344634,53.10666855)(530.54922268,53.71822456)(529.09855493,54.94133659)
\curveto(527.64788718,56.19289307)(526.92255331,58.08445004)(526.92255331,60.61600748)
\lineto(526.92255331,69.27734727)
\closepath
}
}
{
\newrgbcolor{curcolor}{0 0 0}
\pscustom[linestyle=none,fillstyle=solid,fillcolor=curcolor]
{
\newpath
\moveto(565.23725705,69.74668095)
\curveto(568.36614827,69.74668095)(570.75548339,69.06401378)(572.40526239,67.69867942)
\curveto(574.08348587,66.36178954)(574.9225976,64.29956578)(574.9225976,61.51200815)
\lineto(574.9225976,45.98132991)
\lineto(570.48526096,45.98132991)
\lineto(569.24792671,49.1386656)
\lineto(569.07725991,49.1386656)
\curveto(568.08170362,47.88710911)(567.02925839,46.97688621)(565.91992423,46.4079969)
\curveto(564.81059007,45.83910758)(563.28881116,45.55466293)(561.35458749,45.55466293)
\curveto(559.2781415,45.55466293)(557.55725133,46.1519967)(556.19191698,47.34666426)
\curveto(554.82658263,48.54133182)(554.14391545,50.40444432)(554.14391545,52.93600176)
\curveto(554.14391545,55.41067027)(555.01147166,57.23111607)(556.74658406,58.39733916)
\curveto(558.48169646,59.56356225)(561.08436507,60.21778496)(564.55458988,60.36000729)
\lineto(568.60792623,60.48800739)
\lineto(568.60792623,61.51200815)
\curveto(568.60792623,62.73512017)(568.28081488,63.63112084)(567.62659217,64.20001015)
\curveto(567.00081392,64.76889946)(566.11903549,65.05334412)(564.98125686,65.05334412)
\curveto(563.84347824,65.05334412)(562.73414408,64.88267733)(561.65325438,64.54134374)
\curveto(560.57236469,64.22845462)(559.49147499,63.8302321)(558.4105853,63.34667618)
\lineto(556.31991708,67.65601273)
\curveto(557.5430291,68.28179097)(558.92258568,68.77956912)(560.45858683,69.14934717)
\curveto(561.99458797,69.54756969)(563.58747805,69.74668095)(565.23725705,69.74668095)
\closepath
\moveto(568.60792623,56.77600462)
\lineto(566.13325772,56.69067122)
\curveto(564.08525619,56.63378229)(562.66303291,56.26400424)(561.86658787,55.58133706)
\curveto(561.07014284,54.89866989)(560.67192032,54.00266922)(560.67192032,52.89333506)
\curveto(560.67192032,51.92622323)(560.95636497,51.22933382)(561.52525429,50.80266684)
\curveto(562.0941436,50.40444432)(562.83369971,50.20533306)(563.74392261,50.20533306)
\curveto(565.10925696,50.20533306)(566.26125782,50.60355558)(567.19992518,51.40000061)
\curveto(568.13859255,52.22489012)(568.60792623,53.37689098)(568.60792623,54.85600319)
\closepath
}
}
{
\newrgbcolor{curcolor}{0 0 0}
\pscustom[linestyle=none,fillstyle=solid,fillcolor=curcolor]
{
\newpath
\moveto(590.92260694,45.55466293)
\curveto(587.45238214,45.55466293)(584.76438013,46.50755253)(582.85860094,48.41333172)
\curveto(580.9812662,50.31911092)(580.04259884,53.34844651)(580.04259884,57.50133849)
\curveto(580.04259884,60.34578506)(580.52615475,62.66400901)(581.49326659,64.45601034)
\curveto(582.46037842,66.24801168)(583.7972683,67.57067933)(585.50393624,68.4240133)
\curveto(587.23904864,69.27734727)(589.23016124,69.70401425)(591.47727402,69.70401425)
\curveto(593.0701641,69.70401425)(594.44972068,69.54756969)(595.61594377,69.23468057)
\curveto(596.81061133,68.92179145)(597.84883433,68.55201339)(598.73061276,68.12534641)
\lineto(596.85327803,63.21867609)
\curveto(595.85772173,63.61689861)(594.91905437,63.94400996)(594.03727593,64.20001015)
\curveto(593.18394196,64.45601034)(592.33060799,64.58401044)(591.47727402,64.58401044)
\curveto(588.17771601,64.58401044)(586.527937,62.23734202)(586.527937,57.54400519)
\curveto(586.527937,55.21155901)(586.95460399,53.49066884)(587.80793796,52.38133468)
\curveto(588.68971639,51.27200052)(589.91282841,50.71733344)(591.47727402,50.71733344)
\curveto(592.81416391,50.71733344)(593.99460923,50.88800023)(595.01861,51.22933382)
\curveto(596.04261076,51.59911187)(597.03816706,52.09689002)(598.00527889,52.72266827)
\lineto(598.00527889,47.30399756)
\curveto(597.03816706,46.67821932)(596.01416629,46.2373301)(594.9332766,45.98132991)
\curveto(593.88083137,45.69688525)(592.54394149,45.55466293)(590.92260694,45.55466293)
\closepath
}
}
{
\newrgbcolor{curcolor}{0 0 0}
\pscustom[linestyle=none,fillstyle=solid,fillcolor=curcolor]
{
\newpath
\moveto(622.66663454,64.49867704)
\lineto(615.02929552,64.49867704)
\lineto(615.02929552,45.98132991)
\lineto(608.67195745,45.98132991)
\lineto(608.67195745,64.49867704)
\lineto(601.03461842,64.49867704)
\lineto(601.03461842,69.27734727)
\lineto(622.66663454,69.27734727)
\closepath
}
}
{
\newrgbcolor{curcolor}{0 0 0}
\pscustom[linestyle=none,fillstyle=solid,fillcolor=curcolor]
{
\newpath
\moveto(633.33328281,69.27734727)
\lineto(633.33328281,60.31734059)
\lineto(642.20795608,60.31734059)
\lineto(642.20795608,69.27734727)
\lineto(648.56529415,69.27734727)
\lineto(648.56529415,45.98132991)
\lineto(642.20795608,45.98132991)
\lineto(642.20795608,55.58133706)
\lineto(633.33328281,55.58133706)
\lineto(633.33328281,45.98132991)
\lineto(626.97594474,45.98132991)
\lineto(626.97594474,69.27734727)
\closepath
}
}
{
\newrgbcolor{curcolor}{0 0 0}
\pscustom[linestyle=none,fillstyle=solid,fillcolor=curcolor]
{
\newpath
\moveto(661.36534116,69.27734727)
\lineto(661.36534116,60.0613404)
\curveto(661.36534116,59.57778449)(661.33689669,58.98045071)(661.28000776,58.26933907)
\curveto(661.2515633,57.55822743)(661.2088966,56.83289355)(661.15200767,56.09333744)
\curveto(661.1235632,55.35378134)(661.0808965,54.6853364)(661.02400757,54.08800262)
\curveto(660.96711864,53.5191133)(660.92445194,53.13511302)(660.89600748,52.93600176)
\lineto(671.64801549,69.27734727)
\lineto(679.28535451,69.27734727)
\lineto(679.28535451,45.98132991)
\lineto(673.14134993,45.98132991)
\lineto(673.14134993,55.28267017)
\curveto(673.14134993,56.02222628)(673.1697944,56.86133802)(673.22668333,57.80000538)
\curveto(673.28357226,58.73867275)(673.34046119,59.60622895)(673.39735012,60.40267399)
\curveto(673.48268352,61.22756349)(673.53957245,61.85334174)(673.56801692,62.28000872)
\lineto(662.8586756,45.98132991)
\lineto(655.22133658,45.98132991)
\lineto(655.22133658,69.27734727)
\closepath
}
}
{
\newrgbcolor{curcolor}{0 0 0}
\pscustom[linestyle=none,fillstyle=solid,fillcolor=curcolor]
{
\newpath
\moveto(701.21600059,69.27734727)
\lineto(708.21333914,69.27734727)
\lineto(698.99733227,58.09867227)
\lineto(709.02400641,45.98132991)
\lineto(701.81333437,45.98132991)
\lineto(692.29866061,57.80000538)
\lineto(692.29866061,45.98132991)
\lineto(685.94132254,45.98132991)
\lineto(685.94132254,69.27734727)
\lineto(692.29866061,69.27734727)
\lineto(692.29866061,57.97067218)
\closepath
}
}
{
\newrgbcolor{curcolor}{0 0 0}
\pscustom[linestyle=none,fillstyle=solid,fillcolor=curcolor]
{
\newpath
\moveto(738.72001365,69.27734727)
\lineto(745.7173522,69.27734727)
\lineto(736.50134533,58.09867227)
\lineto(746.52801947,45.98132991)
\lineto(739.31734743,45.98132991)
\lineto(729.80267367,57.80000538)
\lineto(729.80267367,45.98132991)
\lineto(723.4453356,45.98132991)
\lineto(723.4453356,69.27734727)
\lineto(729.80267367,69.27734727)
\lineto(729.80267367,57.97067218)
\closepath
}
}
{
\newrgbcolor{curcolor}{0 0 0}
\pscustom[linestyle=none,fillstyle=solid,fillcolor=curcolor]
{
\newpath
\moveto(770.16533005,57.67200529)
\curveto(770.16533005,53.80355796)(769.14132929,50.81688907)(767.09332776,48.71199861)
\curveto(765.0737707,46.60710815)(762.31465753,45.55466293)(758.81598826,45.55466293)
\curveto(756.65420887,45.55466293)(754.71998521,46.02399661)(753.01331727,46.96266398)
\curveto(751.3350938,47.90133134)(750.01242615,49.26666569)(749.04531431,51.05866703)
\curveto(748.07820248,52.87911283)(747.59464657,55.08355891)(747.59464657,57.67200529)
\curveto(747.59464657,61.54045261)(748.6044251,64.51289927)(750.62398216,66.58934527)
\curveto(752.64353922,68.66579126)(755.41687462,69.70401425)(758.94398836,69.70401425)
\curveto(761.13421221,69.70401425)(763.06843587,69.23468057)(764.74665935,68.2960132)
\curveto(766.42488282,67.35734584)(767.74755047,65.99201149)(768.7146623,64.20001015)
\curveto(769.68177413,62.40800882)(770.16533005,60.2320072)(770.16533005,57.67200529)
\closepath
\moveto(754.07998473,57.67200529)
\curveto(754.07998473,55.36800357)(754.44976279,53.61866893)(755.18931889,52.42400138)
\curveto(755.95731946,51.25777829)(757.19465372,50.67466674)(758.90132166,50.67466674)
\curveto(760.57954513,50.67466674)(761.78843492,51.25777829)(762.52799103,52.42400138)
\curveto(763.2959916,53.61866893)(763.67999189,55.36800357)(763.67999189,57.67200529)
\curveto(763.67999189,59.976007)(763.2959916,61.69689718)(762.52799103,62.8346758)
\curveto(761.78843492,64.00089889)(760.5653229,64.58401044)(758.85865496,64.58401044)
\curveto(757.18043149,64.58401044)(755.95731946,64.00089889)(755.18931889,62.8346758)
\curveto(754.44976279,61.69689718)(754.07998473,59.976007)(754.07998473,57.67200529)
\closepath
}
}
{
\newrgbcolor{curcolor}{0 0 0}
\pscustom[linestyle=none,fillstyle=solid,fillcolor=curcolor]
{
\newpath
\moveto(804.7680112,69.27734727)
\lineto(804.7680112,45.98132991)
\lineto(798.83734012,45.98132991)
\lineto(798.83734012,57.4160051)
\curveto(798.83734012,58.55378372)(798.85156235,59.66311788)(798.88000682,60.74400758)
\curveto(798.93689575,61.82489727)(799.00800691,62.82045357)(799.09334031,63.73067647)
\lineto(798.96534021,63.73067647)
\lineto(792.52266875,45.98132991)
\lineto(787.74399852,45.98132991)
\lineto(781.21599366,63.77334317)
\lineto(781.04532686,63.77334317)
\curveto(781.15910472,62.8346758)(781.23021589,61.82489727)(781.25866035,60.74400758)
\curveto(781.31554928,59.69156235)(781.34399375,58.52533926)(781.34399375,57.2453383)
\lineto(781.34399375,45.98132991)
\lineto(775.41332267,45.98132991)
\lineto(775.41332267,69.27734727)
\lineto(784.41599604,69.27734727)
\lineto(790.21866703,53.49066884)
\lineto(796.10667142,69.27734727)
\closepath
}
}
{
\newrgbcolor{curcolor}{0 0 0}
\pscustom[linestyle=none,fillstyle=solid,fillcolor=curcolor]
{
\newpath
\moveto(817.78134189,69.27734727)
\lineto(817.78134189,60.31734059)
\lineto(826.65601517,60.31734059)
\lineto(826.65601517,69.27734727)
\lineto(833.01335324,69.27734727)
\lineto(833.01335324,45.98132991)
\lineto(826.65601517,45.98132991)
\lineto(826.65601517,55.58133706)
\lineto(817.78134189,55.58133706)
\lineto(817.78134189,45.98132991)
\lineto(811.42400382,45.98132991)
\lineto(811.42400382,69.27734727)
\closepath
}
}
{
\newrgbcolor{curcolor}{0 0 0}
\pscustom[linestyle=none,fillstyle=solid,fillcolor=curcolor]
{
\newpath
\moveto(849.22673612,69.74668095)
\curveto(852.35562734,69.74668095)(854.74496245,69.06401378)(856.39474146,67.69867942)
\curveto(858.07296493,66.36178954)(858.91207667,64.29956578)(858.91207667,61.51200815)
\lineto(858.91207667,45.98132991)
\lineto(854.47474003,45.98132991)
\lineto(853.23740577,49.1386656)
\lineto(853.06673898,49.1386656)
\curveto(852.07118268,47.88710911)(851.01873745,46.97688621)(849.90940329,46.4079969)
\curveto(848.80006913,45.83910758)(847.27829022,45.55466293)(845.34406656,45.55466293)
\curveto(843.26762057,45.55466293)(841.5467304,46.1519967)(840.18139604,47.34666426)
\curveto(838.81606169,48.54133182)(838.13339452,50.40444432)(838.13339452,52.93600176)
\curveto(838.13339452,55.41067027)(839.00095072,57.23111607)(840.73606312,58.39733916)
\curveto(842.47117553,59.56356225)(845.07384413,60.21778496)(848.54406894,60.36000729)
\lineto(852.5974053,60.48800739)
\lineto(852.5974053,61.51200815)
\curveto(852.5974053,62.73512017)(852.27029394,63.63112084)(851.61607123,64.20001015)
\curveto(850.99029299,64.76889946)(850.10851455,65.05334412)(848.97073593,65.05334412)
\curveto(847.8329573,65.05334412)(846.72362314,64.88267733)(845.64273345,64.54134374)
\curveto(844.56184375,64.22845462)(843.48095406,63.8302321)(842.40006436,63.34667618)
\lineto(840.30939614,67.65601273)
\curveto(841.53250816,68.28179097)(842.91206475,68.77956912)(844.44806589,69.14934717)
\curveto(845.98406703,69.54756969)(847.57695711,69.74668095)(849.22673612,69.74668095)
\closepath
\moveto(852.5974053,56.77600462)
\lineto(850.12273678,56.69067122)
\curveto(848.07473526,56.63378229)(846.65251198,56.26400424)(845.85606694,55.58133706)
\curveto(845.0596219,54.89866989)(844.66139938,54.00266922)(844.66139938,52.89333506)
\curveto(844.66139938,51.92622323)(844.94584404,51.22933382)(845.51473335,50.80266684)
\curveto(846.08362266,50.40444432)(846.82317877,50.20533306)(847.73340167,50.20533306)
\curveto(849.09873602,50.20533306)(850.25073688,50.60355558)(851.18940425,51.40000061)
\curveto(852.12807161,52.22489012)(852.5974053,53.37689098)(852.5974053,54.85600319)
\closepath
}
}
{
\newrgbcolor{curcolor}{0 0 0}
\pscustom[linestyle=none,fillstyle=solid,fillcolor=curcolor]
{
\newpath
\moveto(884.72543428,64.49867704)
\lineto(877.08809526,64.49867704)
\lineto(877.08809526,45.98132991)
\lineto(870.73075719,45.98132991)
\lineto(870.73075719,64.49867704)
\lineto(863.09341817,64.49867704)
\lineto(863.09341817,69.27734727)
\lineto(884.72543428,69.27734727)
\closepath
}
}
{
\newrgbcolor{curcolor}{0 0 0}
\pscustom[linestyle=none,fillstyle=solid,fillcolor=curcolor]
{
\newpath
\moveto(889.03475211,45.98132991)
\lineto(889.03475211,69.27734727)
\lineto(895.39209018,69.27734727)
\lineto(895.39209018,60.27467389)
\lineto(898.46409247,60.27467389)
\curveto(902.01965067,60.27467389)(904.65076374,59.70578458)(906.35743168,58.56800596)
\curveto(908.06409962,57.43022733)(908.91743359,55.70933716)(908.91743359,53.40533544)
\curveto(908.91743359,51.12977819)(908.12098855,49.32355462)(906.52809848,47.98666474)
\curveto(904.9352084,46.64977485)(902.31831756,45.98132991)(898.67742596,45.98132991)
\closepath
\moveto(912.28810277,45.98132991)
\lineto(912.28810277,69.27734727)
\lineto(918.64544084,69.27734727)
\lineto(918.64544084,45.98132991)
\closepath
\moveto(895.39209018,50.37599985)
\lineto(898.33609237,50.37599985)
\curveto(899.58764886,50.37599985)(900.59742739,50.58933334)(901.36542796,51.01600033)
\curveto(902.161873,51.47111178)(902.56009552,52.23911235)(902.56009552,53.32000205)
\curveto(902.56009552,55.02666998)(901.12365,55.88000395)(898.25075897,55.88000395)
\lineto(895.39209018,55.88000395)
\closepath
}
}
{
\newrgbcolor{curcolor}{0 0 0}
\pscustom[linestyle=none,fillstyle=solid,fillcolor=curcolor]
{
\newpath
\moveto(129.21590608,197.44534261)
\lineto(119.78656572,175.89865989)
\curveto(118.93323175,173.93599176)(118.02300885,172.25776829)(117.05589702,170.86398947)
\curveto(116.08878518,169.47021065)(114.86567316,168.40354319)(113.38656095,167.66398709)
\curveto(111.9358932,166.92443098)(110.01589177,166.55465293)(107.62655666,166.55465293)
\curveto(106.88700055,166.55465293)(106.07633328,166.61154186)(105.19455485,166.72531972)
\curveto(104.31277641,166.83909758)(103.50210914,166.99554214)(102.76255303,167.1946534)
\lineto(102.76255303,172.7413242)
\curveto(103.44522021,172.45687955)(104.18477632,172.25776829)(104.98122135,172.14399042)
\curveto(105.80611086,172.03021256)(106.58833366,171.97332363)(107.32788977,171.97332363)
\curveto(108.75011305,171.97332363)(109.77411381,172.31465722)(110.39989206,172.99732439)
\curveto(111.0256703,173.70843603)(111.52344845,174.56177)(111.8932265,175.5573263)
\lineto(101.35455198,197.44534261)
\lineto(108.18122374,197.44534261)
\lineto(113.85589463,184.26133279)
\curveto(114.05500589,183.8346658)(114.32522832,183.22310979)(114.6665619,182.42666475)
\curveto(115.00789549,181.65866418)(115.26389568,181.00444147)(115.43456247,180.46399662)
\lineto(115.64789597,180.46399662)
\curveto(115.81856276,180.975997)(116.06034072,181.64444195)(116.37322984,182.46933145)
\curveto(116.71456343,183.29422095)(117.01323032,184.01955483)(117.26923051,184.64533307)
\lineto(122.55990112,197.44534261)
\closepath
}
}
{
\newrgbcolor{curcolor}{0 0 0}
\pscustom[linestyle=none,fillstyle=solid,fillcolor=curcolor]
{
\newpath
\moveto(137.27989638,190.27733727)
\lineto(137.27989638,181.74399758)
\curveto(137.27989638,179.72444052)(138.21856375,178.71466199)(140.09589848,178.71466199)
\curveto(141.3190105,178.71466199)(142.45678913,178.84266208)(143.50923435,179.09866227)
\curveto(144.56167958,179.38310693)(145.61412481,179.75288498)(146.66657004,180.20799643)
\lineto(146.66657004,190.27733727)
\lineto(153.02390811,190.27733727)
\lineto(153.02390811,166.98131991)
\lineto(146.66657004,166.98131991)
\lineto(146.66657004,176.23999348)
\curveto(145.67101374,175.69954863)(144.53323512,175.20177048)(143.25323416,174.74665903)
\curveto(141.97323321,174.31999205)(140.52256546,174.10665855)(138.90123092,174.10665855)
\curveto(136.48345134,174.10665855)(134.54922768,174.71821456)(133.09855993,175.94132659)
\curveto(131.64789218,177.19288307)(130.92255831,179.08444004)(130.92255831,181.61599748)
\lineto(130.92255831,190.27733727)
\closepath
}
}
{
\newrgbcolor{curcolor}{0 0 0}
\pscustom[linestyle=none,fillstyle=solid,fillcolor=curcolor]
{
\newpath
\moveto(169.23726205,190.74667095)
\curveto(172.36615327,190.74667095)(174.75548839,190.06400378)(176.40526739,188.69866942)
\curveto(178.08349087,187.36177954)(178.9226026,185.29955578)(178.9226026,182.51199815)
\lineto(178.9226026,166.98131991)
\lineto(174.48526596,166.98131991)
\lineto(173.24793171,170.1386556)
\lineto(173.07726491,170.1386556)
\curveto(172.08170862,168.88709911)(171.02926339,167.97687621)(169.91992923,167.4079869)
\curveto(168.81059507,166.83909758)(167.28881616,166.55465293)(165.35459249,166.55465293)
\curveto(163.2781465,166.55465293)(161.55725633,167.1519867)(160.19192198,168.34665426)
\curveto(158.82658763,169.54132182)(158.14392045,171.40443432)(158.14392045,173.93599176)
\curveto(158.14392045,176.41066027)(159.01147666,178.23110607)(160.74658906,179.39732916)
\curveto(162.48170146,180.56355225)(165.08437007,181.21777496)(168.55459488,181.35999729)
\lineto(172.60793123,181.48799739)
\lineto(172.60793123,182.51199815)
\curveto(172.60793123,183.73511017)(172.28081988,184.63111084)(171.62659717,185.20000015)
\curveto(171.00081892,185.76888946)(170.11904049,186.05333412)(168.98126186,186.05333412)
\curveto(167.84348324,186.05333412)(166.73414908,185.88266733)(165.65325938,185.54133374)
\curveto(164.57236969,185.22844462)(163.49147999,184.8302221)(162.4105903,184.34666618)
\lineto(160.31992208,188.65600273)
\curveto(161.5430341,189.28178097)(162.92259068,189.77955912)(164.45859183,190.14933717)
\curveto(165.99459297,190.54755969)(167.58748305,190.74667095)(169.23726205,190.74667095)
\closepath
\moveto(172.60793123,177.77599462)
\lineto(170.13326272,177.69066122)
\curveto(168.08526119,177.63377229)(166.66303791,177.26399424)(165.86659287,176.58132706)
\curveto(165.07014784,175.89865989)(164.67192532,175.00265922)(164.67192532,173.89332506)
\curveto(164.67192532,172.92621323)(164.95636997,172.22932382)(165.52525929,171.80265684)
\curveto(166.0941486,171.40443432)(166.83370471,171.20532306)(167.74392761,171.20532306)
\curveto(169.10926196,171.20532306)(170.26126282,171.60354558)(171.19993018,172.39999061)
\curveto(172.13859755,173.22488012)(172.60793123,174.37688098)(172.60793123,175.85599319)
\closepath
}
}
{
\newrgbcolor{curcolor}{0 0 0}
\pscustom[linestyle=none,fillstyle=solid,fillcolor=curcolor]
{
\newpath
\moveto(194.92261194,166.55465293)
\curveto(191.45238714,166.55465293)(188.76438513,167.50754253)(186.85860594,169.41332172)
\curveto(184.9812712,171.31910092)(184.04260384,174.34843651)(184.04260384,178.50132849)
\curveto(184.04260384,181.34577506)(184.52615975,183.66399901)(185.49327159,185.45600034)
\curveto(186.46038342,187.24800168)(187.7972733,188.57066933)(189.50394124,189.4240033)
\curveto(191.23905364,190.27733727)(193.23016624,190.70400425)(195.47727902,190.70400425)
\curveto(197.0701691,190.70400425)(198.44972568,190.54755969)(199.61594877,190.23467057)
\curveto(200.81061633,189.92178145)(201.84883933,189.55200339)(202.73061776,189.12533641)
\lineto(200.85328303,184.21866609)
\curveto(199.85772673,184.61688861)(198.91905937,184.94399996)(198.03728093,185.20000015)
\curveto(197.18394696,185.45600034)(196.33061299,185.58400044)(195.47727902,185.58400044)
\curveto(192.17772101,185.58400044)(190.527942,183.23733202)(190.527942,178.54399519)
\curveto(190.527942,176.21154901)(190.95460899,174.49065884)(191.80794296,173.38132468)
\curveto(192.68972139,172.27199052)(193.91283341,171.71732344)(195.47727902,171.71732344)
\curveto(196.81416891,171.71732344)(197.99461423,171.88799023)(199.018615,172.22932382)
\curveto(200.04261576,172.59910187)(201.03817206,173.09688002)(202.00528389,173.72265827)
\lineto(202.00528389,168.30398756)
\curveto(201.03817206,167.67820932)(200.01417129,167.2373201)(198.9332816,166.98131991)
\curveto(197.88083637,166.69687525)(196.54394649,166.55465293)(194.92261194,166.55465293)
\closepath
}
}
{
\newrgbcolor{curcolor}{0 0 0}
\pscustom[linestyle=none,fillstyle=solid,fillcolor=curcolor]
{
\newpath
\moveto(226.66663954,185.49866704)
\lineto(219.02930052,185.49866704)
\lineto(219.02930052,166.98131991)
\lineto(212.67196245,166.98131991)
\lineto(212.67196245,185.49866704)
\lineto(205.03462342,185.49866704)
\lineto(205.03462342,190.27733727)
\lineto(226.66663954,190.27733727)
\closepath
}
}
{
\newrgbcolor{curcolor}{0 0 0}
\pscustom[linestyle=none,fillstyle=solid,fillcolor=curcolor]
{
\newpath
\moveto(237.33328781,190.27733727)
\lineto(237.33328781,181.31733059)
\lineto(246.20796108,181.31733059)
\lineto(246.20796108,190.27733727)
\lineto(252.56529915,190.27733727)
\lineto(252.56529915,166.98131991)
\lineto(246.20796108,166.98131991)
\lineto(246.20796108,176.58132706)
\lineto(237.33328781,176.58132706)
\lineto(237.33328781,166.98131991)
\lineto(230.97594974,166.98131991)
\lineto(230.97594974,190.27733727)
\closepath
}
}
{
\newrgbcolor{curcolor}{0 0 0}
\pscustom[linestyle=none,fillstyle=solid,fillcolor=curcolor]
{
\newpath
\moveto(265.36534616,190.27733727)
\lineto(265.36534616,181.0613304)
\curveto(265.36534616,180.57777449)(265.33690169,179.98044071)(265.28001276,179.26932907)
\curveto(265.2515683,178.55821743)(265.2089016,177.83288355)(265.15201267,177.09332744)
\curveto(265.1235682,176.35377134)(265.0809015,175.6853264)(265.02401257,175.08799262)
\curveto(264.96712364,174.5191033)(264.92445694,174.13510302)(264.89601248,173.93599176)
\lineto(275.64802049,190.27733727)
\lineto(283.28535951,190.27733727)
\lineto(283.28535951,166.98131991)
\lineto(277.14135493,166.98131991)
\lineto(277.14135493,176.28266017)
\curveto(277.14135493,177.02221628)(277.1697994,177.86132802)(277.22668833,178.79999538)
\curveto(277.28357726,179.73866275)(277.34046619,180.60621895)(277.39735512,181.40266399)
\curveto(277.48268852,182.22755349)(277.53957745,182.85333174)(277.56802192,183.27999872)
\lineto(266.8586806,166.98131991)
\lineto(259.22134158,166.98131991)
\lineto(259.22134158,190.27733727)
\closepath
}
}
{
\newrgbcolor{curcolor}{0 0 0}
\pscustom[linestyle=none,fillstyle=solid,fillcolor=curcolor]
{
\newpath
\moveto(305.21600559,190.27733727)
\lineto(312.21334414,190.27733727)
\lineto(302.99733727,179.09866227)
\lineto(313.02401141,166.98131991)
\lineto(305.81333937,166.98131991)
\lineto(296.29866561,178.79999538)
\lineto(296.29866561,166.98131991)
\lineto(289.94132754,166.98131991)
\lineto(289.94132754,190.27733727)
\lineto(296.29866561,190.27733727)
\lineto(296.29866561,178.97066218)
\closepath
}
}
{
\newrgbcolor{curcolor}{0 0 0}
\pscustom[linestyle=none,fillstyle=solid,fillcolor=curcolor]
{
\newpath
\moveto(342.72001865,190.27733727)
\lineto(349.7173572,190.27733727)
\lineto(340.50135033,179.09866227)
\lineto(350.52802447,166.98131991)
\lineto(343.31735243,166.98131991)
\lineto(333.80267867,178.79999538)
\lineto(333.80267867,166.98131991)
\lineto(327.4453406,166.98131991)
\lineto(327.4453406,190.27733727)
\lineto(333.80267867,190.27733727)
\lineto(333.80267867,178.97066218)
\closepath
}
}
{
\newrgbcolor{curcolor}{0 0 0}
\pscustom[linestyle=none,fillstyle=solid,fillcolor=curcolor]
{
\newpath
\moveto(374.16533505,178.67199529)
\curveto(374.16533505,174.80354796)(373.14133429,171.81687907)(371.09333276,169.71198861)
\curveto(369.0737757,167.60709815)(366.31466253,166.55465293)(362.81599326,166.55465293)
\curveto(360.65421387,166.55465293)(358.71999021,167.02398661)(357.01332227,167.96265398)
\curveto(355.3350988,168.90132134)(354.01243115,170.26665569)(353.04531931,172.05865703)
\curveto(352.07820748,173.87910283)(351.59465157,176.08354891)(351.59465157,178.67199529)
\curveto(351.59465157,182.54044261)(352.6044301,185.51288927)(354.62398716,187.58933527)
\curveto(356.64354422,189.66578126)(359.41687962,190.70400425)(362.94399336,190.70400425)
\curveto(365.13421721,190.70400425)(367.06844087,190.23467057)(368.74666435,189.2960032)
\curveto(370.42488782,188.35733584)(371.74755547,186.99200149)(372.7146673,185.20000015)
\curveto(373.68177913,183.40799882)(374.16533505,181.2319972)(374.16533505,178.67199529)
\closepath
\moveto(358.07998973,178.67199529)
\curveto(358.07998973,176.36799357)(358.44976779,174.61865893)(359.18932389,173.42399138)
\curveto(359.95732446,172.25776829)(361.19465872,171.67465674)(362.90132666,171.67465674)
\curveto(364.57955013,171.67465674)(365.78843992,172.25776829)(366.52799603,173.42399138)
\curveto(367.2959966,174.61865893)(367.67999689,176.36799357)(367.67999689,178.67199529)
\curveto(367.67999689,180.975997)(367.2959966,182.69688718)(366.52799603,183.8346658)
\curveto(365.78843992,185.00088889)(364.5653279,185.58400044)(362.85865996,185.58400044)
\curveto(361.18043649,185.58400044)(359.95732446,185.00088889)(359.18932389,183.8346658)
\curveto(358.44976779,182.69688718)(358.07998973,180.975997)(358.07998973,178.67199529)
\closepath
}
}
{
\newrgbcolor{curcolor}{0 0 0}
\pscustom[linestyle=none,fillstyle=solid,fillcolor=curcolor]
{
\newpath
\moveto(408.7680162,190.27733727)
\lineto(408.7680162,166.98131991)
\lineto(402.83734512,166.98131991)
\lineto(402.83734512,178.4159951)
\curveto(402.83734512,179.55377372)(402.85156735,180.66310788)(402.88001182,181.74399758)
\curveto(402.93690075,182.82488727)(403.00801191,183.82044357)(403.09334531,184.73066647)
\lineto(402.96534521,184.73066647)
\lineto(396.52267375,166.98131991)
\lineto(391.74400352,166.98131991)
\lineto(385.21599866,184.77333317)
\lineto(385.04533186,184.77333317)
\curveto(385.15910972,183.8346658)(385.23022089,182.82488727)(385.25866535,181.74399758)
\curveto(385.31555428,180.69155235)(385.34399875,179.52532926)(385.34399875,178.2453283)
\lineto(385.34399875,166.98131991)
\lineto(379.41332767,166.98131991)
\lineto(379.41332767,190.27733727)
\lineto(388.41600104,190.27733727)
\lineto(394.21867203,174.49065884)
\lineto(400.10667642,190.27733727)
\closepath
}
}
{
\newrgbcolor{curcolor}{0 0 0}
\pscustom[linestyle=none,fillstyle=solid,fillcolor=curcolor]
{
\newpath
\moveto(421.78134689,190.27733727)
\lineto(421.78134689,181.31733059)
\lineto(430.65602017,181.31733059)
\lineto(430.65602017,190.27733727)
\lineto(437.01335824,190.27733727)
\lineto(437.01335824,166.98131991)
\lineto(430.65602017,166.98131991)
\lineto(430.65602017,176.58132706)
\lineto(421.78134689,176.58132706)
\lineto(421.78134689,166.98131991)
\lineto(415.42400882,166.98131991)
\lineto(415.42400882,190.27733727)
\closepath
}
}
{
\newrgbcolor{curcolor}{0 0 0}
\pscustom[linestyle=none,fillstyle=solid,fillcolor=curcolor]
{
\newpath
\moveto(453.22674112,190.74667095)
\curveto(456.35563234,190.74667095)(458.74496745,190.06400378)(460.39474646,188.69866942)
\curveto(462.07296993,187.36177954)(462.91208167,185.29955578)(462.91208167,182.51199815)
\lineto(462.91208167,166.98131991)
\lineto(458.47474503,166.98131991)
\lineto(457.23741077,170.1386556)
\lineto(457.06674398,170.1386556)
\curveto(456.07118768,168.88709911)(455.01874245,167.97687621)(453.90940829,167.4079869)
\curveto(452.80007413,166.83909758)(451.27829522,166.55465293)(449.34407156,166.55465293)
\curveto(447.26762557,166.55465293)(445.5467354,167.1519867)(444.18140104,168.34665426)
\curveto(442.81606669,169.54132182)(442.13339952,171.40443432)(442.13339952,173.93599176)
\curveto(442.13339952,176.41066027)(443.00095572,178.23110607)(444.73606812,179.39732916)
\curveto(446.47118053,180.56355225)(449.07384913,181.21777496)(452.54407394,181.35999729)
\lineto(456.5974103,181.48799739)
\lineto(456.5974103,182.51199815)
\curveto(456.5974103,183.73511017)(456.27029894,184.63111084)(455.61607623,185.20000015)
\curveto(454.99029799,185.76888946)(454.10851955,186.05333412)(452.97074093,186.05333412)
\curveto(451.8329623,186.05333412)(450.72362814,185.88266733)(449.64273845,185.54133374)
\curveto(448.56184875,185.22844462)(447.48095906,184.8302221)(446.40006936,184.34666618)
\lineto(444.30940114,188.65600273)
\curveto(445.53251316,189.28178097)(446.91206975,189.77955912)(448.44807089,190.14933717)
\curveto(449.98407203,190.54755969)(451.57696211,190.74667095)(453.22674112,190.74667095)
\closepath
\moveto(456.5974103,177.77599462)
\lineto(454.12274178,177.69066122)
\curveto(452.07474026,177.63377229)(450.65251698,177.26399424)(449.85607194,176.58132706)
\curveto(449.0596269,175.89865989)(448.66140438,175.00265922)(448.66140438,173.89332506)
\curveto(448.66140438,172.92621323)(448.94584904,172.22932382)(449.51473835,171.80265684)
\curveto(450.08362766,171.40443432)(450.82318377,171.20532306)(451.73340667,171.20532306)
\curveto(453.09874102,171.20532306)(454.25074188,171.60354558)(455.18940925,172.39999061)
\curveto(456.12807661,173.22488012)(456.5974103,174.37688098)(456.5974103,175.85599319)
\closepath
}
}
{
\newrgbcolor{curcolor}{0 0 0}
\pscustom[linestyle=none,fillstyle=solid,fillcolor=curcolor]
{
\newpath
\moveto(488.72543928,185.49866704)
\lineto(481.08810026,185.49866704)
\lineto(481.08810026,166.98131991)
\lineto(474.73076219,166.98131991)
\lineto(474.73076219,185.49866704)
\lineto(467.09342317,185.49866704)
\lineto(467.09342317,190.27733727)
\lineto(488.72543928,190.27733727)
\closepath
}
}
{
\newrgbcolor{curcolor}{0 0 0}
\pscustom[linestyle=none,fillstyle=solid,fillcolor=curcolor]
{
\newpath
\moveto(493.03475711,166.98131991)
\lineto(493.03475711,190.27733727)
\lineto(499.39209518,190.27733727)
\lineto(499.39209518,181.27466389)
\lineto(502.46409747,181.27466389)
\curveto(506.01965567,181.27466389)(508.65076874,180.70577458)(510.35743668,179.56799596)
\curveto(512.06410462,178.43021733)(512.91743859,176.70932716)(512.91743859,174.40532544)
\curveto(512.91743859,172.12976819)(512.12099355,170.32354462)(510.52810348,168.98665474)
\curveto(508.9352134,167.64976485)(506.31832256,166.98131991)(502.67743096,166.98131991)
\closepath
\moveto(516.28810777,166.98131991)
\lineto(516.28810777,190.27733727)
\lineto(522.64544584,190.27733727)
\lineto(522.64544584,166.98131991)
\closepath
\moveto(499.39209518,171.37598985)
\lineto(502.33609737,171.37598985)
\curveto(503.58765386,171.37598985)(504.59743239,171.58932334)(505.36543296,172.01599033)
\curveto(506.161878,172.47110178)(506.56010052,173.23910235)(506.56010052,174.31999205)
\curveto(506.56010052,176.02665998)(505.123655,176.87999395)(502.25076397,176.87999395)
\lineto(499.39209518,176.87999395)
\closepath
}
}
\end{pspicture}
}
		\caption{Свободная система прав}
		\label{fig:img1}
	\end{center}
\end{figure}

Ограниченная система прав (Рис.~\ref{fig:img2}) предполагает обязательное наличие владельца комнаты
и согласование с ним всех действий по передаче прав.

\begin{figure}[h]
	\begin{center}
		\scalebox{0.6}{%LaTeX with PSTricks extensions
%%Creator: Inkscape 1.2 (dc2aedaf03, 2022-05-15)
%%Please note this file requires PSTricks extensions
\psset{xunit=.5pt,yunit=.5pt,runit=.5pt}
\begin{pspicture}(1024,768)
{
\newrgbcolor{curcolor}{0.50196081 0.50196081 0.50196081}
\pscustom[linestyle=none,fillstyle=solid,fillcolor=curcolor]
{
\newpath
\moveto(9.11873341,757.86807346)
\lineto(1015.21901417,757.86807346)
\lineto(1015.21901417,615.00792027)
\lineto(9.11873341,615.00792027)
\closepath
}
}
{
\newrgbcolor{curcolor}{0.50196081 0.50196081 0.50196081}
\pscustom[linestyle=none,fillstyle=solid,fillcolor=curcolor]
{
\newpath
\moveto(8.61212349,149.44592285)
\lineto(1014.71240425,149.44592285)
\lineto(1014.71240425,6.58576965)
\lineto(8.61212349,6.58576965)
\closepath
}
}
{
\newrgbcolor{curcolor}{0.50196081 0.50196081 0.50196081}
\pscustom[linestyle=none,fillstyle=solid,fillcolor=curcolor]
{
\newpath
\moveto(8.6121273,445.24536133)
\lineto(1014.71240807,445.24536133)
\lineto(1014.71240807,302.38520813)
\lineto(8.6121273,302.38520813)
\closepath
}
}
{
\newrgbcolor{curcolor}{0.50196081 0.50196081 0.50196081}
\pscustom[linestyle=none,fillstyle=solid,fillcolor=curcolor]
{
\newpath
\moveto(30.39577866,609.94195557)
\lineto(45.59366798,609.94195557)
\lineto(45.59366798,465.05540466)
\lineto(30.39577866,465.05540466)
\closepath
}
}
{
\newrgbcolor{curcolor}{0.50196081 0.50196081 0.50196081}
\pscustom[linestyle=none,fillstyle=solid,fillcolor=curcolor]
{
\newpath
\moveto(23.284149,467.99269)
\lineto(53.463989,468.26135)
\lineto(38.329292,449.81314)
\lineto(38.329292,449.81314)
\closepath
}
}
{
\newrgbcolor{curcolor}{0.50196081 0.50196081 0.50196081}
\pscustom[linestyle=none,fillstyle=solid,fillcolor=curcolor]
{
\newpath
\moveto(31.42565536,297.40866089)
\lineto(46.62354469,297.40866089)
\lineto(46.62354469,167.28443909)
\lineto(31.42565536,167.28443909)
\closepath
}
}
{
\newrgbcolor{curcolor}{0.50196081 0.50196081 0.50196081}
\pscustom[linestyle=none,fillstyle=solid,fillcolor=curcolor]
{
\newpath
\moveto(24.314026,169.92244)
\lineto(54.493866,170.16372)
\lineto(39.359169,153.59518)
\lineto(39.359169,153.59518)
\closepath
}
}
{
\newrgbcolor{curcolor}{0 0 0}
\pscustom[linestyle=none,fillstyle=solid,fillcolor=curcolor]
{
\newpath
\moveto(240.0053225,710.88383798)
\lineto(225.8613225,678.56383798)
\curveto(224.5813225,675.61983798)(223.21598917,673.10250464)(221.7653225,671.01183798)
\curveto(220.31465583,668.92117131)(218.47998917,667.32117131)(216.2613225,666.21183798)
\curveto(214.0853225,665.10250464)(211.2053225,664.54783798)(207.6213225,664.54783798)
\curveto(206.51198917,664.54783798)(205.29598917,664.63317131)(203.9733225,664.80383798)
\curveto(202.65065583,664.97450464)(201.43465583,665.20917131)(200.3253225,665.50783798)
\lineto(200.3253225,673.82783798)
\curveto(201.3493225,673.40117131)(202.45865583,673.10250464)(203.6533225,672.93183798)
\curveto(204.89065583,672.76117131)(206.06398917,672.67583798)(207.1733225,672.67583798)
\curveto(209.30665583,672.67583798)(210.84265583,673.18783798)(211.7813225,674.21183798)
\curveto(212.71998917,675.27850464)(213.46665583,676.55850464)(214.0213225,678.05183798)
\lineto(198.2133225,710.88383798)
\lineto(208.4533225,710.88383798)
\lineto(216.9653225,691.10783798)
\curveto(217.26398917,690.46783798)(217.6693225,689.55050464)(218.1813225,688.35583798)
\curveto(218.6933225,687.20383798)(219.0773225,686.22250464)(219.3333225,685.41183798)
\lineto(219.6533225,685.41183798)
\curveto(219.9093225,686.17983798)(220.27198917,687.18250464)(220.7413225,688.41983798)
\curveto(221.2533225,689.65717131)(221.7013225,690.74517131)(222.0853225,691.68383798)
\lineto(230.0213225,710.88383798)
\closepath
}
}
{
\newrgbcolor{curcolor}{0 0 0}
\pscustom[linestyle=none,fillstyle=solid,fillcolor=curcolor]
{
\newpath
\moveto(252.10130004,700.13183798)
\lineto(252.10130004,687.33183798)
\curveto(252.10130004,684.30250464)(253.50930004,682.78783798)(256.32530004,682.78783798)
\curveto(258.15996671,682.78783798)(259.86663337,682.97983798)(261.44530004,683.36383798)
\curveto(263.02396671,683.79050464)(264.60263337,684.34517131)(266.18130004,685.02783798)
\lineto(266.18130004,700.13183798)
\lineto(275.71730004,700.13183798)
\lineto(275.71730004,665.18783798)
\lineto(266.18130004,665.18783798)
\lineto(266.18130004,679.07583798)
\curveto(264.68796671,678.26517131)(262.98130004,677.51850464)(261.06130004,676.83583798)
\curveto(259.14130004,676.19583798)(256.96530004,675.87583798)(254.53330004,675.87583798)
\curveto(250.90663337,675.87583798)(248.00530004,676.79317131)(245.82930004,678.62783798)
\curveto(243.65330004,680.50517131)(242.56530004,683.34250464)(242.56530004,687.13983798)
\lineto(242.56530004,700.13183798)
\closepath
}
}
{
\newrgbcolor{curcolor}{0 0 0}
\pscustom[linestyle=none,fillstyle=solid,fillcolor=curcolor]
{
\newpath
\moveto(300.03730883,700.83583798)
\curveto(304.73064216,700.83583798)(308.31464216,699.81183798)(310.78930883,697.76383798)
\curveto(313.30664216,695.75850464)(314.56530883,692.66517131)(314.56530883,688.48383798)
\lineto(314.56530883,665.18783798)
\lineto(307.90930883,665.18783798)
\lineto(306.05330883,669.92383798)
\lineto(305.79730883,669.92383798)
\curveto(304.30397549,668.04650464)(302.72530883,666.68117131)(301.06130883,665.82783798)
\curveto(299.39730883,664.97450464)(297.11464216,664.54783798)(294.21330883,664.54783798)
\curveto(291.09864216,664.54783798)(288.51730883,665.44383798)(286.46930883,667.23583798)
\curveto(284.42130883,669.02783798)(283.39730883,671.82250464)(283.39730883,675.61983798)
\curveto(283.39730883,679.33183798)(284.69864216,682.06250464)(287.30130883,683.81183798)
\curveto(289.90397549,685.56117131)(293.80797549,686.54250464)(299.01330883,686.75583798)
\lineto(305.09330883,686.94783798)
\lineto(305.09330883,688.48383798)
\curveto(305.09330883,690.31850464)(304.60264216,691.66250464)(303.62130883,692.51583798)
\curveto(302.68264216,693.36917131)(301.35997549,693.79583798)(299.65330883,693.79583798)
\curveto(297.94664216,693.79583798)(296.28264216,693.53983798)(294.66130883,693.02783798)
\curveto(293.03997549,692.55850464)(291.41864216,691.96117131)(289.79730883,691.23583798)
\lineto(286.66130883,697.69983798)
\curveto(288.49597549,698.63850464)(290.56530883,699.38517131)(292.86930883,699.93983798)
\curveto(295.17330883,700.53717131)(297.56264216,700.83583798)(300.03730883,700.83583798)
\closepath
\moveto(305.09330883,681.37983798)
\lineto(301.38130883,681.25183798)
\curveto(298.30930883,681.16650464)(296.17597549,680.61183798)(294.98130883,679.58783798)
\curveto(293.78664216,678.56383798)(293.18930883,677.21983798)(293.18930883,675.55583798)
\curveto(293.18930883,674.10517131)(293.61597549,673.05983798)(294.46930883,672.41983798)
\curveto(295.32264216,671.82250464)(296.43197549,671.52383798)(297.79730883,671.52383798)
\curveto(299.84530883,671.52383798)(301.57330883,672.12117131)(302.98130883,673.31583798)
\curveto(304.38930883,674.55317131)(305.09330883,676.28117131)(305.09330883,678.49983798)
\closepath
}
}
{
\newrgbcolor{curcolor}{0 0 0}
\pscustom[linestyle=none,fillstyle=solid,fillcolor=curcolor]
{
\newpath
\moveto(338.56531469,664.54783798)
\curveto(333.35998135,664.54783798)(329.32798135,665.97717131)(326.46931469,668.83583798)
\curveto(323.65331469,671.69450464)(322.24531469,676.23850464)(322.24531469,682.46783798)
\curveto(322.24531469,686.73450464)(322.97064802,690.21183798)(324.42131469,692.89983798)
\curveto(325.87198135,695.58783798)(327.87731469,697.57183798)(330.43731469,698.85183798)
\curveto(333.03998135,700.13183798)(336.02664802,700.77183798)(339.39731469,700.77183798)
\curveto(341.78664802,700.77183798)(343.85598135,700.53717131)(345.60531469,700.06783798)
\curveto(347.39731469,699.59850464)(348.95464802,699.04383798)(350.27731469,698.40383798)
\lineto(347.46131469,691.04383798)
\curveto(345.96798135,691.64117131)(344.55998135,692.13183798)(343.23731469,692.51583798)
\curveto(341.95731469,692.89983798)(340.67731469,693.09183798)(339.39731469,693.09183798)
\curveto(334.44798135,693.09183798)(331.97331469,689.57183798)(331.97331469,682.53183798)
\curveto(331.97331469,679.03317131)(332.61331469,676.45183798)(333.89331469,674.78783798)
\curveto(335.21598135,673.12383798)(337.05064802,672.29183798)(339.39731469,672.29183798)
\curveto(341.40264802,672.29183798)(343.17331469,672.54783798)(344.70931469,673.05983798)
\curveto(346.24531469,673.61450464)(347.73864802,674.36117131)(349.18931469,675.29983798)
\lineto(349.18931469,667.17183798)
\curveto(347.73864802,666.23317131)(346.20264802,665.57183798)(344.58131469,665.18783798)
\curveto(343.00264802,664.76117131)(340.99731469,664.54783798)(338.56531469,664.54783798)
\closepath
}
}
{
\newrgbcolor{curcolor}{0 0 0}
\pscustom[linestyle=none,fillstyle=solid,fillcolor=curcolor]
{
\newpath
\moveto(386.18131078,692.96383798)
\lineto(374.72531078,692.96383798)
\lineto(374.72531078,665.18783798)
\lineto(365.18931078,665.18783798)
\lineto(365.18931078,692.96383798)
\lineto(353.73331078,692.96383798)
\lineto(353.73331078,700.13183798)
\lineto(386.18131078,700.13183798)
\closepath
}
}
{
\newrgbcolor{curcolor}{0 0 0}
\pscustom[linestyle=none,fillstyle=solid,fillcolor=curcolor]
{
\newpath
\moveto(402.18127855,700.13183798)
\lineto(402.18127855,686.69183798)
\lineto(415.49327855,686.69183798)
\lineto(415.49327855,700.13183798)
\lineto(425.02927855,700.13183798)
\lineto(425.02927855,665.18783798)
\lineto(415.49327855,665.18783798)
\lineto(415.49327855,679.58783798)
\lineto(402.18127855,679.58783798)
\lineto(402.18127855,665.18783798)
\lineto(392.64527855,665.18783798)
\lineto(392.64527855,700.13183798)
\closepath
}
}
{
\newrgbcolor{curcolor}{0 0 0}
\pscustom[linestyle=none,fillstyle=solid,fillcolor=curcolor]
{
\newpath
\moveto(444.22932055,700.13183798)
\lineto(444.22932055,686.30783798)
\curveto(444.22932055,685.58250464)(444.18665388,684.68650464)(444.10132055,683.61983798)
\curveto(444.05865388,682.55317131)(443.99465388,681.46517131)(443.90932055,680.35583798)
\curveto(443.86665388,679.24650464)(443.80265388,678.24383798)(443.71732055,677.34783798)
\curveto(443.63198721,676.49450464)(443.56798721,675.91850464)(443.52532055,675.61983798)
\lineto(459.65332055,700.13183798)
\lineto(471.10932055,700.13183798)
\lineto(471.10932055,665.18783798)
\lineto(461.89332055,665.18783798)
\lineto(461.89332055,679.13983798)
\curveto(461.89332055,680.24917131)(461.93598721,681.50783798)(462.02132055,682.91583798)
\curveto(462.10665388,684.32383798)(462.19198721,685.62517131)(462.27732055,686.81983798)
\curveto(462.40532055,688.05717131)(462.49065388,688.99583798)(462.53332055,689.63583798)
\lineto(446.46932055,665.18783798)
\lineto(435.01332055,665.18783798)
\lineto(435.01332055,700.13183798)
\closepath
}
}
{
\newrgbcolor{curcolor}{0 0 0}
\pscustom[linestyle=none,fillstyle=solid,fillcolor=curcolor]
{
\newpath
\moveto(504.0052766,700.13183798)
\lineto(514.5012766,700.13183798)
\lineto(500.6772766,683.36383798)
\lineto(515.7172766,665.18783798)
\lineto(504.9012766,665.18783798)
\lineto(490.6292766,682.91583798)
\lineto(490.6292766,665.18783798)
\lineto(481.0932766,665.18783798)
\lineto(481.0932766,700.13183798)
\lineto(490.6292766,700.13183798)
\lineto(490.6292766,683.17183798)
\closepath
}
}
{
\newrgbcolor{curcolor}{0 0 0}
\pscustom[linestyle=none,fillstyle=solid,fillcolor=curcolor]
{
\newpath
\moveto(560.26125805,700.13183798)
\lineto(570.75725805,700.13183798)
\lineto(556.93325805,683.36383798)
\lineto(571.97325805,665.18783798)
\lineto(561.15725805,665.18783798)
\lineto(546.88525805,682.91583798)
\lineto(546.88525805,665.18783798)
\lineto(537.34925805,665.18783798)
\lineto(537.34925805,700.13183798)
\lineto(546.88525805,700.13183798)
\lineto(546.88525805,683.17183798)
\closepath
}
}
{
\newrgbcolor{curcolor}{0 0 0}
\pscustom[linestyle=none,fillstyle=solid,fillcolor=curcolor]
{
\newpath
\moveto(607.42919555,682.72383798)
\curveto(607.42919555,676.92117131)(605.89319555,672.44117131)(602.82119555,669.28383798)
\curveto(599.79186221,666.12650464)(595.65319555,664.54783798)(590.40519555,664.54783798)
\curveto(587.16252888,664.54783798)(584.26119555,665.25183798)(581.70119555,666.65983798)
\curveto(579.18386221,668.06783798)(577.19986221,670.11583798)(575.74919555,672.80383798)
\curveto(574.29852888,675.53450464)(573.57319555,678.84117131)(573.57319555,682.72383798)
\curveto(573.57319555,688.52650464)(575.08786221,692.98517131)(578.11719555,696.09983798)
\curveto(581.14652888,699.21450464)(585.30652888,700.77183798)(590.59719555,700.77183798)
\curveto(593.88252888,700.77183798)(596.78386221,700.06783798)(599.30119555,698.65983798)
\curveto(601.81852888,697.25183798)(603.80252888,695.20383798)(605.25319555,692.51583798)
\curveto(606.70386221,689.82783798)(607.42919555,686.56383798)(607.42919555,682.72383798)
\closepath
\moveto(583.30119555,682.72383798)
\curveto(583.30119555,679.26783798)(583.85586221,676.64383798)(584.96519555,674.85183798)
\curveto(586.11719555,673.10250464)(587.97319555,672.22783798)(590.53319555,672.22783798)
\curveto(593.05052888,672.22783798)(594.86386221,673.10250464)(595.97319555,674.85183798)
\curveto(597.12519555,676.64383798)(597.70119555,679.26783798)(597.70119555,682.72383798)
\curveto(597.70119555,686.17983798)(597.12519555,688.76117131)(595.97319555,690.46783798)
\curveto(594.86386221,692.21717131)(593.02919555,693.09183798)(590.46919555,693.09183798)
\curveto(587.95186221,693.09183798)(586.11719555,692.21717131)(584.96519555,690.46783798)
\curveto(583.85586221,688.76117131)(583.30119555,686.17983798)(583.30119555,682.72383798)
\closepath
}
}
{
\newrgbcolor{curcolor}{0 0 0}
\pscustom[linestyle=none,fillstyle=solid,fillcolor=curcolor]
{
\newpath
\moveto(659.33316234,700.13183798)
\lineto(659.33316234,665.18783798)
\lineto(650.43716234,665.18783798)
\lineto(650.43716234,682.33983798)
\curveto(650.43716234,684.04650464)(650.45849568,685.71050464)(650.50116234,687.33183798)
\curveto(650.58649568,688.95317131)(650.69316234,690.44650464)(650.82116234,691.81183798)
\lineto(650.62916234,691.81183798)
\lineto(640.96516234,665.18783798)
\lineto(633.79716234,665.18783798)
\lineto(624.00516234,691.87583798)
\lineto(623.74916234,691.87583798)
\curveto(623.91982901,690.46783798)(624.02649568,688.95317131)(624.06916234,687.33183798)
\curveto(624.15449568,685.75317131)(624.19716234,684.00383798)(624.19716234,682.08383798)
\lineto(624.19716234,665.18783798)
\lineto(615.30116234,665.18783798)
\lineto(615.30116234,700.13183798)
\lineto(628.80516234,700.13183798)
\lineto(637.50916234,676.45183798)
\lineto(646.34116234,700.13183798)
\closepath
}
}
{
\newrgbcolor{curcolor}{0 0 0}
\pscustom[linestyle=none,fillstyle=solid,fillcolor=curcolor]
{
\newpath
\moveto(678.85315355,700.13183798)
\lineto(678.85315355,686.69183798)
\lineto(692.16515355,686.69183798)
\lineto(692.16515355,700.13183798)
\lineto(701.70115355,700.13183798)
\lineto(701.70115355,665.18783798)
\lineto(692.16515355,665.18783798)
\lineto(692.16515355,679.58783798)
\lineto(678.85315355,679.58783798)
\lineto(678.85315355,665.18783798)
\lineto(669.31715355,665.18783798)
\lineto(669.31715355,700.13183798)
\closepath
}
}
{
\newrgbcolor{curcolor}{0 0 0}
\pscustom[linestyle=none,fillstyle=solid,fillcolor=curcolor]
{
\newpath
\moveto(726.02119555,700.83583798)
\curveto(730.71452888,700.83583798)(734.29852888,699.81183798)(736.77319555,697.76383798)
\curveto(739.29052888,695.75850464)(740.54919555,692.66517131)(740.54919555,688.48383798)
\lineto(740.54919555,665.18783798)
\lineto(733.89319555,665.18783798)
\lineto(732.03719555,669.92383798)
\lineto(731.78119555,669.92383798)
\curveto(730.28786221,668.04650464)(728.70919555,666.68117131)(727.04519555,665.82783798)
\curveto(725.38119555,664.97450464)(723.09852888,664.54783798)(720.19719555,664.54783798)
\curveto(717.08252888,664.54783798)(714.50119555,665.44383798)(712.45319555,667.23583798)
\curveto(710.40519555,669.02783798)(709.38119555,671.82250464)(709.38119555,675.61983798)
\curveto(709.38119555,679.33183798)(710.68252888,682.06250464)(713.28519555,683.81183798)
\curveto(715.88786221,685.56117131)(719.79186221,686.54250464)(724.99719555,686.75583798)
\lineto(731.07719555,686.94783798)
\lineto(731.07719555,688.48383798)
\curveto(731.07719555,690.31850464)(730.58652888,691.66250464)(729.60519555,692.51583798)
\curveto(728.66652888,693.36917131)(727.34386221,693.79583798)(725.63719555,693.79583798)
\curveto(723.93052888,693.79583798)(722.26652888,693.53983798)(720.64519555,693.02783798)
\curveto(719.02386221,692.55850464)(717.40252888,691.96117131)(715.78119555,691.23583798)
\lineto(712.64519555,697.69983798)
\curveto(714.47986221,698.63850464)(716.54919555,699.38517131)(718.85319555,699.93983798)
\curveto(721.15719555,700.53717131)(723.54652888,700.83583798)(726.02119555,700.83583798)
\closepath
\moveto(731.07719555,681.37983798)
\lineto(727.36519555,681.25183798)
\curveto(724.29319555,681.16650464)(722.15986221,680.61183798)(720.96519555,679.58783798)
\curveto(719.77052888,678.56383798)(719.17319555,677.21983798)(719.17319555,675.55583798)
\curveto(719.17319555,674.10517131)(719.59986221,673.05983798)(720.45319555,672.41983798)
\curveto(721.30652888,671.82250464)(722.41586221,671.52383798)(723.78119555,671.52383798)
\curveto(725.82919555,671.52383798)(727.55719555,672.12117131)(728.96519555,673.31583798)
\curveto(730.37319555,674.55317131)(731.07719555,676.28117131)(731.07719555,678.49983798)
\closepath
}
}
{
\newrgbcolor{curcolor}{0 0 0}
\pscustom[linestyle=none,fillstyle=solid,fillcolor=curcolor]
{
\newpath
\moveto(779.26920141,692.96383798)
\lineto(767.81320141,692.96383798)
\lineto(767.81320141,665.18783798)
\lineto(758.27720141,665.18783798)
\lineto(758.27720141,692.96383798)
\lineto(746.82120141,692.96383798)
\lineto(746.82120141,700.13183798)
\lineto(779.26920141,700.13183798)
\closepath
}
}
{
\newrgbcolor{curcolor}{0 0 0}
\pscustom[linestyle=none,fillstyle=solid,fillcolor=curcolor]
{
\newpath
\moveto(785.73316918,665.18783798)
\lineto(785.73316918,700.13183798)
\lineto(795.26916918,700.13183798)
\lineto(795.26916918,686.62783798)
\lineto(799.87716918,686.62783798)
\curveto(805.21050251,686.62783798)(809.15716918,685.77450464)(811.71716918,684.06783798)
\curveto(814.27716918,682.36117131)(815.55716918,679.77983798)(815.55716918,676.32383798)
\curveto(815.55716918,672.91050464)(814.36250251,670.20117131)(811.97316918,668.19583798)
\curveto(809.58383585,666.19050464)(805.65850251,665.18783798)(800.19716918,665.18783798)
\closepath
\moveto(820.61316918,665.18783798)
\lineto(820.61316918,700.13183798)
\lineto(830.14916918,700.13183798)
\lineto(830.14916918,665.18783798)
\closepath
\moveto(795.26916918,671.77983798)
\lineto(799.68516918,671.77983798)
\curveto(801.56250251,671.77983798)(803.07716918,672.09983798)(804.22916918,672.73983798)
\curveto(805.42383585,673.42250464)(806.02116918,674.57450464)(806.02116918,676.19583798)
\curveto(806.02116918,678.75583798)(803.86650251,680.03583798)(799.55716918,680.03583798)
\lineto(795.26916918,680.03583798)
\closepath
}
}
{
\newrgbcolor{curcolor}{0 0 0}
\pscustom[linestyle=none,fillstyle=solid,fillcolor=curcolor]
{
\newpath
\moveto(246.6050795,95.10467888)
\lineto(232.4610795,62.78467888)
\curveto(231.1810795,59.84067888)(229.81574617,57.32334554)(228.3650795,55.23267888)
\curveto(226.91441283,53.14201221)(225.07974617,51.54201221)(222.8610795,50.43267888)
\curveto(220.6850795,49.32334554)(217.8050795,48.76867888)(214.2210795,48.76867888)
\curveto(213.11174617,48.76867888)(211.89574617,48.85401221)(210.5730795,49.02467888)
\curveto(209.25041283,49.19534554)(208.03441283,49.43001221)(206.9250795,49.72867888)
\lineto(206.9250795,58.04867888)
\curveto(207.9490795,57.62201221)(209.05841283,57.32334554)(210.2530795,57.15267888)
\curveto(211.49041283,56.98201221)(212.66374617,56.89667888)(213.7730795,56.89667888)
\curveto(215.90641283,56.89667888)(217.44241283,57.40867888)(218.3810795,58.43267888)
\curveto(219.31974617,59.49934554)(220.06641283,60.77934554)(220.6210795,62.27267888)
\lineto(204.8130795,95.10467888)
\lineto(215.0530795,95.10467888)
\lineto(223.5650795,75.32867888)
\curveto(223.86374617,74.68867888)(224.2690795,73.77134554)(224.7810795,72.57667888)
\curveto(225.2930795,71.42467888)(225.6770795,70.44334554)(225.9330795,69.63267888)
\lineto(226.2530795,69.63267888)
\curveto(226.5090795,70.40067888)(226.87174617,71.40334554)(227.3410795,72.64067888)
\curveto(227.8530795,73.87801221)(228.3010795,74.96601221)(228.6850795,75.90467888)
\lineto(236.6210795,95.10467888)
\closepath
}
}
{
\newrgbcolor{curcolor}{0 0 0}
\pscustom[linestyle=none,fillstyle=solid,fillcolor=curcolor]
{
\newpath
\moveto(258.70105704,84.35267888)
\lineto(258.70105704,71.55267888)
\curveto(258.70105704,68.52334554)(260.10905704,67.00867888)(262.92505704,67.00867888)
\curveto(264.75972371,67.00867888)(266.46639037,67.20067888)(268.04505704,67.58467888)
\curveto(269.62372371,68.01134554)(271.20239037,68.56601221)(272.78105704,69.24867888)
\lineto(272.78105704,84.35267888)
\lineto(282.31705704,84.35267888)
\lineto(282.31705704,49.40867888)
\lineto(272.78105704,49.40867888)
\lineto(272.78105704,63.29667888)
\curveto(271.28772371,62.48601221)(269.58105704,61.73934554)(267.66105704,61.05667888)
\curveto(265.74105704,60.41667888)(263.56505704,60.09667888)(261.13305704,60.09667888)
\curveto(257.50639037,60.09667888)(254.60505704,61.01401221)(252.42905704,62.84867888)
\curveto(250.25305704,64.72601221)(249.16505704,67.56334554)(249.16505704,71.36067888)
\lineto(249.16505704,84.35267888)
\closepath
}
}
{
\newrgbcolor{curcolor}{0 0 0}
\pscustom[linestyle=none,fillstyle=solid,fillcolor=curcolor]
{
\newpath
\moveto(306.63706583,85.05667888)
\curveto(311.33039916,85.05667888)(314.91439916,84.03267888)(317.38906583,81.98467888)
\curveto(319.90639916,79.97934554)(321.16506583,76.88601221)(321.16506583,72.70467888)
\lineto(321.16506583,49.40867888)
\lineto(314.50906583,49.40867888)
\lineto(312.65306583,54.14467888)
\lineto(312.39706583,54.14467888)
\curveto(310.90373249,52.26734554)(309.32506583,50.90201221)(307.66106583,50.04867888)
\curveto(305.99706583,49.19534554)(303.71439916,48.76867888)(300.81306583,48.76867888)
\curveto(297.69839916,48.76867888)(295.11706583,49.66467888)(293.06906583,51.45667888)
\curveto(291.02106583,53.24867888)(289.99706583,56.04334554)(289.99706583,59.84067888)
\curveto(289.99706583,63.55267888)(291.29839916,66.28334554)(293.90106583,68.03267888)
\curveto(296.50373249,69.78201221)(300.40773249,70.76334554)(305.61306583,70.97667888)
\lineto(311.69306583,71.16867888)
\lineto(311.69306583,72.70467888)
\curveto(311.69306583,74.53934554)(311.20239916,75.88334554)(310.22106583,76.73667888)
\curveto(309.28239916,77.59001221)(307.95973249,78.01667888)(306.25306583,78.01667888)
\curveto(304.54639916,78.01667888)(302.88239916,77.76067888)(301.26106583,77.24867888)
\curveto(299.63973249,76.77934554)(298.01839916,76.18201221)(296.39706583,75.45667888)
\lineto(293.26106583,81.92067888)
\curveto(295.09573249,82.85934554)(297.16506583,83.60601221)(299.46906583,84.16067888)
\curveto(301.77306583,84.75801221)(304.16239916,85.05667888)(306.63706583,85.05667888)
\closepath
\moveto(311.69306583,65.60067888)
\lineto(307.98106583,65.47267888)
\curveto(304.90906583,65.38734554)(302.77573249,64.83267888)(301.58106583,63.80867888)
\curveto(300.38639916,62.78467888)(299.78906583,61.44067888)(299.78906583,59.77667888)
\curveto(299.78906583,58.32601221)(300.21573249,57.28067888)(301.06906583,56.64067888)
\curveto(301.92239916,56.04334554)(303.03173249,55.74467888)(304.39706583,55.74467888)
\curveto(306.44506583,55.74467888)(308.17306583,56.34201221)(309.58106583,57.53667888)
\curveto(310.98906583,58.77401221)(311.69306583,60.50201221)(311.69306583,62.72067888)
\closepath
}
}
{
\newrgbcolor{curcolor}{0 0 0}
\pscustom[linestyle=none,fillstyle=solid,fillcolor=curcolor]
{
\newpath
\moveto(345.16507169,48.76867888)
\curveto(339.95973835,48.76867888)(335.92773835,50.19801221)(333.06907169,53.05667888)
\curveto(330.25307169,55.91534554)(328.84507169,60.45934554)(328.84507169,66.68867888)
\curveto(328.84507169,70.95534554)(329.57040502,74.43267888)(331.02107169,77.12067888)
\curveto(332.47173835,79.80867888)(334.47707169,81.79267888)(337.03707169,83.07267888)
\curveto(339.63973835,84.35267888)(342.62640502,84.99267888)(345.99707169,84.99267888)
\curveto(348.38640502,84.99267888)(350.45573835,84.75801221)(352.20507169,84.28867888)
\curveto(353.99707169,83.81934554)(355.55440502,83.26467888)(356.87707169,82.62467888)
\lineto(354.06107169,75.26467888)
\curveto(352.56773835,75.86201221)(351.15973835,76.35267888)(349.83707169,76.73667888)
\curveto(348.55707169,77.12067888)(347.27707169,77.31267888)(345.99707169,77.31267888)
\curveto(341.04773835,77.31267888)(338.57307169,73.79267888)(338.57307169,66.75267888)
\curveto(338.57307169,63.25401221)(339.21307169,60.67267888)(340.49307169,59.00867888)
\curveto(341.81573835,57.34467888)(343.65040502,56.51267888)(345.99707169,56.51267888)
\curveto(348.00240502,56.51267888)(349.77307169,56.76867888)(351.30907169,57.28067888)
\curveto(352.84507169,57.83534554)(354.33840502,58.58201221)(355.78907169,59.52067888)
\lineto(355.78907169,51.39267888)
\curveto(354.33840502,50.45401221)(352.80240502,49.79267888)(351.18107169,49.40867888)
\curveto(349.60240502,48.98201221)(347.59707169,48.76867888)(345.16507169,48.76867888)
\closepath
}
}
{
\newrgbcolor{curcolor}{0 0 0}
\pscustom[linestyle=none,fillstyle=solid,fillcolor=curcolor]
{
\newpath
\moveto(392.78106778,77.18467888)
\lineto(381.32506778,77.18467888)
\lineto(381.32506778,49.40867888)
\lineto(371.78906778,49.40867888)
\lineto(371.78906778,77.18467888)
\lineto(360.33306778,77.18467888)
\lineto(360.33306778,84.35267888)
\lineto(392.78106778,84.35267888)
\closepath
}
}
{
\newrgbcolor{curcolor}{0 0 0}
\pscustom[linestyle=none,fillstyle=solid,fillcolor=curcolor]
{
\newpath
\moveto(408.78103555,84.35267888)
\lineto(408.78103555,70.91267888)
\lineto(422.09303555,70.91267888)
\lineto(422.09303555,84.35267888)
\lineto(431.62903555,84.35267888)
\lineto(431.62903555,49.40867888)
\lineto(422.09303555,49.40867888)
\lineto(422.09303555,63.80867888)
\lineto(408.78103555,63.80867888)
\lineto(408.78103555,49.40867888)
\lineto(399.24503555,49.40867888)
\lineto(399.24503555,84.35267888)
\closepath
}
}
{
\newrgbcolor{curcolor}{0 0 0}
\pscustom[linestyle=none,fillstyle=solid,fillcolor=curcolor]
{
\newpath
\moveto(450.82907755,84.35267888)
\lineto(450.82907755,70.52867888)
\curveto(450.82907755,69.80334554)(450.78641088,68.90734554)(450.70107755,67.84067888)
\curveto(450.65841088,66.77401221)(450.59441088,65.68601221)(450.50907755,64.57667888)
\curveto(450.46641088,63.46734554)(450.40241088,62.46467888)(450.31707755,61.56867888)
\curveto(450.23174421,60.71534554)(450.16774421,60.13934554)(450.12507755,59.84067888)
\lineto(466.25307755,84.35267888)
\lineto(477.70907755,84.35267888)
\lineto(477.70907755,49.40867888)
\lineto(468.49307755,49.40867888)
\lineto(468.49307755,63.36067888)
\curveto(468.49307755,64.47001221)(468.53574421,65.72867888)(468.62107755,67.13667888)
\curveto(468.70641088,68.54467888)(468.79174421,69.84601221)(468.87707755,71.04067888)
\curveto(469.00507755,72.27801221)(469.09041088,73.21667888)(469.13307755,73.85667888)
\lineto(453.06907755,49.40867888)
\lineto(441.61307755,49.40867888)
\lineto(441.61307755,84.35267888)
\closepath
}
}
{
\newrgbcolor{curcolor}{0 0 0}
\pscustom[linestyle=none,fillstyle=solid,fillcolor=curcolor]
{
\newpath
\moveto(510.6050336,84.35267888)
\lineto(521.1010336,84.35267888)
\lineto(507.2770336,67.58467888)
\lineto(522.3170336,49.40867888)
\lineto(511.5010336,49.40867888)
\lineto(497.2290336,67.13667888)
\lineto(497.2290336,49.40867888)
\lineto(487.6930336,49.40867888)
\lineto(487.6930336,84.35267888)
\lineto(497.2290336,84.35267888)
\lineto(497.2290336,67.39267888)
\closepath
}
}
{
\newrgbcolor{curcolor}{0 0 0}
\pscustom[linestyle=none,fillstyle=solid,fillcolor=curcolor]
{
\newpath
\moveto(566.86101505,84.35267888)
\lineto(577.35701505,84.35267888)
\lineto(563.53301505,67.58467888)
\lineto(578.57301505,49.40867888)
\lineto(567.75701505,49.40867888)
\lineto(553.48501505,67.13667888)
\lineto(553.48501505,49.40867888)
\lineto(543.94901505,49.40867888)
\lineto(543.94901505,84.35267888)
\lineto(553.48501505,84.35267888)
\lineto(553.48501505,67.39267888)
\closepath
}
}
{
\newrgbcolor{curcolor}{0 0 0}
\pscustom[linestyle=none,fillstyle=solid,fillcolor=curcolor]
{
\newpath
\moveto(614.02895255,66.94467888)
\curveto(614.02895255,61.14201221)(612.49295255,56.66201221)(609.42095255,53.50467888)
\curveto(606.39161921,50.34734554)(602.25295255,48.76867888)(597.00495255,48.76867888)
\curveto(593.76228588,48.76867888)(590.86095255,49.47267888)(588.30095255,50.88067888)
\curveto(585.78361921,52.28867888)(583.79961921,54.33667888)(582.34895255,57.02467888)
\curveto(580.89828588,59.75534554)(580.17295255,63.06201221)(580.17295255,66.94467888)
\curveto(580.17295255,72.74734554)(581.68761921,77.20601221)(584.71695255,80.32067888)
\curveto(587.74628588,83.43534554)(591.90628588,84.99267888)(597.19695255,84.99267888)
\curveto(600.48228588,84.99267888)(603.38361921,84.28867888)(605.90095255,82.88067888)
\curveto(608.41828588,81.47267888)(610.40228588,79.42467888)(611.85295255,76.73667888)
\curveto(613.30361921,74.04867888)(614.02895255,70.78467888)(614.02895255,66.94467888)
\closepath
\moveto(589.90095255,66.94467888)
\curveto(589.90095255,63.48867888)(590.45561921,60.86467888)(591.56495255,59.07267888)
\curveto(592.71695255,57.32334554)(594.57295255,56.44867888)(597.13295255,56.44867888)
\curveto(599.65028588,56.44867888)(601.46361921,57.32334554)(602.57295255,59.07267888)
\curveto(603.72495255,60.86467888)(604.30095255,63.48867888)(604.30095255,66.94467888)
\curveto(604.30095255,70.40067888)(603.72495255,72.98201221)(602.57295255,74.68867888)
\curveto(601.46361921,76.43801221)(599.62895255,77.31267888)(597.06895255,77.31267888)
\curveto(594.55161921,77.31267888)(592.71695255,76.43801221)(591.56495255,74.68867888)
\curveto(590.45561921,72.98201221)(589.90095255,70.40067888)(589.90095255,66.94467888)
\closepath
}
}
{
\newrgbcolor{curcolor}{0 0 0}
\pscustom[linestyle=none,fillstyle=solid,fillcolor=curcolor]
{
\newpath
\moveto(665.93291934,84.35267888)
\lineto(665.93291934,49.40867888)
\lineto(657.03691934,49.40867888)
\lineto(657.03691934,66.56067888)
\curveto(657.03691934,68.26734554)(657.05825268,69.93134554)(657.10091934,71.55267888)
\curveto(657.18625268,73.17401221)(657.29291934,74.66734554)(657.42091934,76.03267888)
\lineto(657.22891934,76.03267888)
\lineto(647.56491934,49.40867888)
\lineto(640.39691934,49.40867888)
\lineto(630.60491934,76.09667888)
\lineto(630.34891934,76.09667888)
\curveto(630.51958601,74.68867888)(630.62625268,73.17401221)(630.66891934,71.55267888)
\curveto(630.75425268,69.97401221)(630.79691934,68.22467888)(630.79691934,66.30467888)
\lineto(630.79691934,49.40867888)
\lineto(621.90091934,49.40867888)
\lineto(621.90091934,84.35267888)
\lineto(635.40491934,84.35267888)
\lineto(644.10891934,60.67267888)
\lineto(652.94091934,84.35267888)
\closepath
}
}
{
\newrgbcolor{curcolor}{0 0 0}
\pscustom[linestyle=none,fillstyle=solid,fillcolor=curcolor]
{
\newpath
\moveto(685.45291055,84.35267888)
\lineto(685.45291055,70.91267888)
\lineto(698.76491055,70.91267888)
\lineto(698.76491055,84.35267888)
\lineto(708.30091055,84.35267888)
\lineto(708.30091055,49.40867888)
\lineto(698.76491055,49.40867888)
\lineto(698.76491055,63.80867888)
\lineto(685.45291055,63.80867888)
\lineto(685.45291055,49.40867888)
\lineto(675.91691055,49.40867888)
\lineto(675.91691055,84.35267888)
\closepath
}
}
{
\newrgbcolor{curcolor}{0 0 0}
\pscustom[linestyle=none,fillstyle=solid,fillcolor=curcolor]
{
\newpath
\moveto(732.62095255,85.05667888)
\curveto(737.31428588,85.05667888)(740.89828588,84.03267888)(743.37295255,81.98467888)
\curveto(745.89028588,79.97934554)(747.14895255,76.88601221)(747.14895255,72.70467888)
\lineto(747.14895255,49.40867888)
\lineto(740.49295255,49.40867888)
\lineto(738.63695255,54.14467888)
\lineto(738.38095255,54.14467888)
\curveto(736.88761921,52.26734554)(735.30895255,50.90201221)(733.64495255,50.04867888)
\curveto(731.98095255,49.19534554)(729.69828588,48.76867888)(726.79695255,48.76867888)
\curveto(723.68228588,48.76867888)(721.10095255,49.66467888)(719.05295255,51.45667888)
\curveto(717.00495255,53.24867888)(715.98095255,56.04334554)(715.98095255,59.84067888)
\curveto(715.98095255,63.55267888)(717.28228588,66.28334554)(719.88495255,68.03267888)
\curveto(722.48761921,69.78201221)(726.39161921,70.76334554)(731.59695255,70.97667888)
\lineto(737.67695255,71.16867888)
\lineto(737.67695255,72.70467888)
\curveto(737.67695255,74.53934554)(737.18628588,75.88334554)(736.20495255,76.73667888)
\curveto(735.26628588,77.59001221)(733.94361921,78.01667888)(732.23695255,78.01667888)
\curveto(730.53028588,78.01667888)(728.86628588,77.76067888)(727.24495255,77.24867888)
\curveto(725.62361921,76.77934554)(724.00228588,76.18201221)(722.38095255,75.45667888)
\lineto(719.24495255,81.92067888)
\curveto(721.07961921,82.85934554)(723.14895255,83.60601221)(725.45295255,84.16067888)
\curveto(727.75695255,84.75801221)(730.14628588,85.05667888)(732.62095255,85.05667888)
\closepath
\moveto(737.67695255,65.60067888)
\lineto(733.96495255,65.47267888)
\curveto(730.89295255,65.38734554)(728.75961921,64.83267888)(727.56495255,63.80867888)
\curveto(726.37028588,62.78467888)(725.77295255,61.44067888)(725.77295255,59.77667888)
\curveto(725.77295255,58.32601221)(726.19961921,57.28067888)(727.05295255,56.64067888)
\curveto(727.90628588,56.04334554)(729.01561921,55.74467888)(730.38095255,55.74467888)
\curveto(732.42895255,55.74467888)(734.15695255,56.34201221)(735.56495255,57.53667888)
\curveto(736.97295255,58.77401221)(737.67695255,60.50201221)(737.67695255,62.72067888)
\closepath
}
}
{
\newrgbcolor{curcolor}{0 0 0}
\pscustom[linestyle=none,fillstyle=solid,fillcolor=curcolor]
{
\newpath
\moveto(785.86895841,77.18467888)
\lineto(774.41295841,77.18467888)
\lineto(774.41295841,49.40867888)
\lineto(764.87695841,49.40867888)
\lineto(764.87695841,77.18467888)
\lineto(753.42095841,77.18467888)
\lineto(753.42095841,84.35267888)
\lineto(785.86895841,84.35267888)
\closepath
}
}
{
\newrgbcolor{curcolor}{0 0 0}
\pscustom[linestyle=none,fillstyle=solid,fillcolor=curcolor]
{
\newpath
\moveto(792.33292618,49.40867888)
\lineto(792.33292618,84.35267888)
\lineto(801.86892618,84.35267888)
\lineto(801.86892618,70.84867888)
\lineto(806.47692618,70.84867888)
\curveto(811.81025951,70.84867888)(815.75692618,69.99534554)(818.31692618,68.28867888)
\curveto(820.87692618,66.58201221)(822.15692618,64.00067888)(822.15692618,60.54467888)
\curveto(822.15692618,57.13134554)(820.96225951,54.42201221)(818.57292618,52.41667888)
\curveto(816.18359285,50.41134554)(812.25825951,49.40867888)(806.79692618,49.40867888)
\closepath
\moveto(827.21292618,49.40867888)
\lineto(827.21292618,84.35267888)
\lineto(836.74892618,84.35267888)
\lineto(836.74892618,49.40867888)
\closepath
\moveto(801.86892618,56.00067888)
\lineto(806.28492618,56.00067888)
\curveto(808.16225951,56.00067888)(809.67692618,56.32067888)(810.82892618,56.96067888)
\curveto(812.02359285,57.64334554)(812.62092618,58.79534554)(812.62092618,60.41667888)
\curveto(812.62092618,62.97667888)(810.46625951,64.25667888)(806.15692618,64.25667888)
\lineto(801.86892618,64.25667888)
\closepath
}
}
{
\newrgbcolor{curcolor}{0 0 0}
\pscustom[linestyle=none,fillstyle=solid,fillcolor=curcolor]
{
\newpath
\moveto(206.37147262,398.8204877)
\lineto(220.57947262,398.8204877)
\curveto(226.63813929,398.8204877)(231.22480596,397.96715437)(234.33947263,396.2604877)
\curveto(237.49680596,394.55382104)(239.07547263,391.54582104)(239.07547263,387.2364877)
\curveto(239.07547263,384.63382104)(238.45680596,382.4364877)(237.21947263,380.6444877)
\curveto(236.02480596,378.8524877)(234.29680596,377.7644877)(232.03547263,377.3804877)
\lineto(232.03547263,377.0604877)
\curveto(233.52880596,376.71915437)(234.89413929,376.1644877)(236.13147263,375.3964877)
\curveto(237.41147263,374.67115437)(238.41413929,373.58315437)(239.13947262,372.1324877)
\curveto(239.86480596,370.68182104)(240.22747263,368.76182104)(240.22747262,366.3724877)
\curveto(240.22747262,362.23382104)(238.71280596,358.99115437)(235.68347263,356.6444877)
\curveto(232.69680596,354.29782104)(228.62213929,353.1244877)(223.45947263,353.1244877)
\lineto(206.37147262,353.1244877)
\closepath
\moveto(216.03547263,380.7084877)
\lineto(221.66747263,380.7084877)
\curveto(224.48347263,380.7084877)(226.42480596,381.13515437)(227.49147263,381.9884877)
\curveto(228.60080596,382.8844877)(229.15547263,384.20715437)(229.15547263,385.9564877)
\curveto(229.15547263,387.70582104)(228.51547263,388.9644877)(227.23547262,389.7324877)
\curveto(225.95547262,390.5004877)(223.92880596,390.8844877)(221.15547263,390.8844877)
\lineto(216.03547263,390.8844877)
\closepath
\moveto(216.03547263,373.0284877)
\lineto(216.03547263,361.1244877)
\lineto(222.37147262,361.1244877)
\curveto(225.27280596,361.1244877)(227.29947263,361.67915437)(228.45147263,362.7884877)
\curveto(229.60347262,363.9404877)(230.17947263,365.45515437)(230.17947263,367.3324877)
\curveto(230.17947263,369.03915437)(229.58213929,370.4044877)(228.38747263,371.4284877)
\curveto(227.23547263,372.49515437)(225.12347263,373.0284877)(222.05147263,373.0284877)
\closepath
}
}
{
\newrgbcolor{curcolor}{0 0 0}
\pscustom[linestyle=none,fillstyle=solid,fillcolor=curcolor]
{
\newpath
\moveto(279.90752927,353.1244877)
\lineto(270.37152927,353.1244877)
\lineto(270.37152927,380.9004877)
\lineto(261.60352927,380.9004877)
\curveto(261.0488626,374.07382104)(260.30219593,368.56982104)(259.36352927,364.3884877)
\curveto(258.46752927,360.24982104)(257.18752927,357.2204877)(255.52352927,355.3004877)
\curveto(253.90219593,353.42315437)(251.74752927,352.4844877)(249.05952927,352.4844877)
\curveto(246.8408626,352.4844877)(245.02752927,352.82582104)(243.61952927,353.5084877)
\lineto(243.61952927,361.1244877)
\curveto(244.6008626,360.69782104)(245.6248626,360.4844877)(246.69152927,360.4844877)
\curveto(247.45952927,360.4844877)(248.16352927,360.8684877)(248.80352927,361.6364877)
\curveto(249.44352927,362.4044877)(250.0408626,363.79115437)(250.59552927,365.7964877)
\curveto(251.1928626,367.80182104)(251.72619593,370.5964877)(252.19552927,374.1804877)
\curveto(252.6648626,377.80715437)(253.09152927,382.4364877)(253.47552927,388.0684877)
\lineto(279.90752927,388.0684877)
\closepath
}
}
{
\newrgbcolor{curcolor}{0 0 0}
\pscustom[linestyle=none,fillstyle=solid,fillcolor=curcolor]
{
\newpath
\moveto(304.22755856,388.7724877)
\curveto(308.9208919,388.7724877)(312.5048919,387.7484877)(314.97955856,385.7004877)
\curveto(317.4968919,383.69515437)(318.75555856,380.60182104)(318.75555856,376.4204877)
\lineto(318.75555856,353.1244877)
\lineto(312.09955856,353.1244877)
\lineto(310.24355856,357.8604877)
\lineto(309.98755856,357.8604877)
\curveto(308.49422523,355.98315437)(306.91555856,354.61782104)(305.25155856,353.7644877)
\curveto(303.58755856,352.91115437)(301.3048919,352.4844877)(298.40355856,352.4844877)
\curveto(295.2888919,352.4844877)(292.70755856,353.3804877)(290.65955856,355.1724877)
\curveto(288.61155856,356.9644877)(287.58755856,359.75915437)(287.58755856,363.5564877)
\curveto(287.58755856,367.2684877)(288.8888919,369.99915437)(291.49155856,371.7484877)
\curveto(294.09422523,373.49782104)(297.99822523,374.47915437)(303.20355856,374.6924877)
\lineto(309.28355856,374.8844877)
\lineto(309.28355856,376.4204877)
\curveto(309.28355856,378.25515437)(308.7928919,379.59915437)(307.81155856,380.4524877)
\curveto(306.8728919,381.30582104)(305.55022523,381.7324877)(303.84355856,381.7324877)
\curveto(302.1368919,381.7324877)(300.4728919,381.4764877)(298.85155856,380.9644877)
\curveto(297.23022523,380.49515437)(295.6088919,379.89782104)(293.98755856,379.1724877)
\lineto(290.85155856,385.6364877)
\curveto(292.68622523,386.57515437)(294.75555856,387.32182104)(297.05955856,387.8764877)
\curveto(299.36355856,388.47382104)(301.7528919,388.7724877)(304.22755856,388.7724877)
\closepath
\moveto(309.28355856,369.3164877)
\lineto(305.57155856,369.1884877)
\curveto(302.49955856,369.10315437)(300.36622523,368.5484877)(299.17155856,367.5244877)
\curveto(297.9768919,366.5004877)(297.37955856,365.1564877)(297.37955856,363.4924877)
\curveto(297.37955856,362.04182104)(297.80622523,360.9964877)(298.65955856,360.3564877)
\curveto(299.5128919,359.75915437)(300.62222523,359.4604877)(301.98755856,359.4604877)
\curveto(304.03555856,359.4604877)(305.76355856,360.05782104)(307.17155856,361.2524877)
\curveto(308.57955856,362.48982104)(309.28355856,364.21782104)(309.28355856,366.4364877)
\closepath
}
}
{
\newrgbcolor{curcolor}{0 0 0}
\pscustom[linestyle=none,fillstyle=solid,fillcolor=curcolor]
{
\newpath
\moveto(359.97156442,388.0684877)
\lineto(359.97156442,360.1004877)
\lineto(365.09156442,360.1004877)
\lineto(365.09156442,340.5804877)
\lineto(356.51556442,340.5804877)
\lineto(356.51556442,353.1244877)
\lineto(333.02756442,353.1244877)
\lineto(333.02756442,340.5804877)
\lineto(324.45156442,340.5804877)
\lineto(324.45156442,360.1004877)
\lineto(327.39556442,360.1004877)
\curveto(328.93156442,362.44715437)(330.23289776,365.11382104)(331.29956442,368.1004877)
\curveto(332.36623109,371.12982104)(333.21956442,374.32982104)(333.85956442,377.7004877)
\curveto(334.49956442,381.11382104)(334.96889776,384.56982104)(335.26756442,388.0684877)
\closepath
\moveto(350.43556442,380.9004877)
\lineto(343.26756442,380.9004877)
\curveto(342.75556442,377.01782104)(342.05156442,373.32715437)(341.15556442,369.8284877)
\curveto(340.25956442,366.3724877)(339.00089776,363.12982104)(337.37956442,360.1004877)
\lineto(350.43556442,360.1004877)
\closepath
}
}
{
\newrgbcolor{curcolor}{0 0 0}
\pscustom[linestyle=none,fillstyle=solid,fillcolor=curcolor]
{
\newpath
\moveto(385.4435361,388.7084877)
\curveto(390.26486943,388.7084877)(394.0835361,387.32182104)(396.8995361,384.5484877)
\curveto(399.7155361,381.81782104)(401.1235361,377.91382104)(401.1235361,372.8364877)
\lineto(401.1235361,368.2284877)
\lineto(378.5955361,368.2284877)
\curveto(378.68086943,365.5404877)(379.47020277,363.4284877)(380.9635361,361.8924877)
\curveto(382.4995361,360.3564877)(384.6115361,359.5884877)(387.2995361,359.5884877)
\curveto(389.51820277,359.5884877)(391.54486943,359.80182104)(393.3795361,360.2284877)
\curveto(395.25686943,360.69782104)(397.17686943,361.40182104)(399.1395361,362.3404877)
\lineto(399.1395361,354.9804877)
\curveto(397.39020277,354.12715437)(395.57686943,353.5084877)(393.6995361,353.1244877)
\curveto(391.82220277,352.69782104)(389.5395361,352.4844877)(386.8515361,352.4844877)
\curveto(383.35286943,352.4844877)(380.2595361,353.1244877)(377.5715361,354.4044877)
\curveto(374.8835361,355.72715437)(372.7715361,357.68982104)(371.2355361,360.2924877)
\curveto(369.6995361,362.93782104)(368.9315361,366.28715437)(368.9315361,370.3404877)
\curveto(368.9315361,374.39382104)(369.61420277,377.78582104)(370.9795361,380.5164877)
\curveto(372.3875361,383.24715437)(374.32886943,385.29515437)(376.8035361,386.6604877)
\curveto(379.27820277,388.02582104)(382.15820277,388.7084877)(385.4435361,388.7084877)
\closepath
\moveto(385.5075361,381.9244877)
\curveto(383.63020277,381.9244877)(382.09420277,381.32715437)(380.8995361,380.1324877)
\curveto(379.70486943,378.93782104)(379.00086943,377.08182104)(378.7875361,374.5644877)
\lineto(392.1635361,374.5644877)
\curveto(392.12086943,376.65515437)(391.54486943,378.4044877)(390.4355361,379.8124877)
\curveto(389.36886943,381.2204877)(387.72620277,381.9244877)(385.5075361,381.9244877)
\closepath
}
}
{
\newrgbcolor{curcolor}{0 0 0}
\pscustom[linestyle=none,fillstyle=solid,fillcolor=curcolor]
{
\newpath
\moveto(440.16351071,353.1244877)
\lineto(430.62751071,353.1244877)
\lineto(430.62751071,380.9004877)
\lineto(421.85951071,380.9004877)
\curveto(421.30484404,374.07382104)(420.55817738,368.56982104)(419.61951071,364.3884877)
\curveto(418.72351071,360.24982104)(417.44351071,357.2204877)(415.77951071,355.3004877)
\curveto(414.15817738,353.42315437)(412.00351071,352.4844877)(409.31551071,352.4844877)
\curveto(407.09684404,352.4844877)(405.28351071,352.82582104)(403.87551071,353.5084877)
\lineto(403.87551071,361.1244877)
\curveto(404.85684404,360.69782104)(405.88084404,360.4844877)(406.94751071,360.4844877)
\curveto(407.71551071,360.4844877)(408.41951071,360.8684877)(409.05951071,361.6364877)
\curveto(409.69951071,362.4044877)(410.29684404,363.79115437)(410.85151071,365.7964877)
\curveto(411.44884404,367.80182104)(411.98217738,370.5964877)(412.45151071,374.1804877)
\curveto(412.92084404,377.80715437)(413.34751071,382.4364877)(413.73151071,388.0684877)
\lineto(440.16351071,388.0684877)
\closepath
}
}
{
\newrgbcolor{curcolor}{0 0 0}
\pscustom[linestyle=none,fillstyle=solid,fillcolor=curcolor]
{
\newpath
\moveto(464.54754001,388.7084877)
\curveto(469.36887334,388.7084877)(473.18754001,387.32182104)(476.00354001,384.5484877)
\curveto(478.81954001,381.81782104)(480.22754001,377.91382104)(480.22754001,372.8364877)
\lineto(480.22754001,368.2284877)
\lineto(457.69954001,368.2284877)
\curveto(457.78487334,365.5404877)(458.57420667,363.4284877)(460.06754001,361.8924877)
\curveto(461.60354001,360.3564877)(463.71554001,359.5884877)(466.40354001,359.5884877)
\curveto(468.62220667,359.5884877)(470.64887334,359.80182104)(472.48354001,360.2284877)
\curveto(474.36087334,360.69782104)(476.28087334,361.40182104)(478.24354001,362.3404877)
\lineto(478.24354001,354.9804877)
\curveto(476.49420667,354.12715437)(474.68087334,353.5084877)(472.80354001,353.1244877)
\curveto(470.92620667,352.69782104)(468.64354001,352.4844877)(465.95554001,352.4844877)
\curveto(462.45687334,352.4844877)(459.36354001,353.1244877)(456.67554001,354.4044877)
\curveto(453.98754001,355.72715437)(451.87554001,357.68982104)(450.33954001,360.2924877)
\curveto(448.80354001,362.93782104)(448.03554001,366.28715437)(448.03554001,370.3404877)
\curveto(448.03554001,374.39382104)(448.71820667,377.78582104)(450.08354001,380.5164877)
\curveto(451.49154001,383.24715437)(453.43287334,385.29515437)(455.90754001,386.6604877)
\curveto(458.38220667,388.02582104)(461.26220667,388.7084877)(464.54754001,388.7084877)
\closepath
\moveto(464.61154001,381.9244877)
\curveto(462.73420667,381.9244877)(461.19820667,381.32715437)(460.00354001,380.1324877)
\curveto(458.80887334,378.93782104)(458.10487334,377.08182104)(457.89154001,374.5644877)
\lineto(471.26754001,374.5644877)
\curveto(471.22487334,376.65515437)(470.64887334,378.4044877)(469.53954001,379.8124877)
\curveto(468.47287334,381.2204877)(466.83020667,381.9244877)(464.61154001,381.9244877)
\closepath
}
}
{
\newrgbcolor{curcolor}{0 0 0}
\pscustom[linestyle=none,fillstyle=solid,fillcolor=curcolor]
{
\newpath
\moveto(526.11551462,340.5804877)
\lineto(517.53951462,340.5804877)
\lineto(517.53951462,353.1244877)
\lineto(487.97151462,353.1244877)
\lineto(487.97151462,388.0684877)
\lineto(497.50751462,388.0684877)
\lineto(497.50751462,360.2924877)
\lineto(511.45951462,360.2924877)
\lineto(511.45951462,388.0684877)
\lineto(520.99551462,388.0684877)
\lineto(520.99551462,360.1004877)
\lineto(526.11551462,360.1004877)
\closepath
}
}
{
\newrgbcolor{curcolor}{0 0 0}
\pscustom[linestyle=none,fillstyle=solid,fillcolor=curcolor]
{
\newpath
\moveto(571.55551169,388.0684877)
\lineto(582.05151169,388.0684877)
\lineto(568.22751169,371.3004877)
\lineto(583.26751169,353.1244877)
\lineto(572.45151169,353.1244877)
\lineto(558.17951169,370.8524877)
\lineto(558.17951169,353.1244877)
\lineto(548.64351169,353.1244877)
\lineto(548.64351169,388.0684877)
\lineto(558.17951169,388.0684877)
\lineto(558.17951169,371.1084877)
\closepath
}
}
{
\newrgbcolor{curcolor}{0 0 0}
\pscustom[linestyle=none,fillstyle=solid,fillcolor=curcolor]
{
\newpath
\moveto(618.72344919,370.6604877)
\curveto(618.72344919,364.85782104)(617.18744919,360.37782104)(614.11544919,357.2204877)
\curveto(611.08611585,354.06315437)(606.94744919,352.4844877)(601.69944919,352.4844877)
\curveto(598.45678252,352.4844877)(595.55544919,353.1884877)(592.99544919,354.5964877)
\curveto(590.47811585,356.0044877)(588.49411585,358.0524877)(587.04344919,360.7404877)
\curveto(585.59278252,363.47115437)(584.86744919,366.77782104)(584.86744919,370.6604877)
\curveto(584.86744919,376.46315437)(586.38211585,380.92182104)(589.41144919,384.0364877)
\curveto(592.44078252,387.15115437)(596.60078252,388.7084877)(601.89144919,388.7084877)
\curveto(605.17678252,388.7084877)(608.07811585,388.0044877)(610.59544919,386.5964877)
\curveto(613.11278252,385.1884877)(615.09678252,383.1404877)(616.54744919,380.4524877)
\curveto(617.99811585,377.7644877)(618.72344919,374.5004877)(618.72344919,370.6604877)
\closepath
\moveto(594.59544919,370.6604877)
\curveto(594.59544919,367.2044877)(595.15011585,364.5804877)(596.25944919,362.7884877)
\curveto(597.41144919,361.03915437)(599.26744919,360.1644877)(601.82744919,360.1644877)
\curveto(604.34478252,360.1644877)(606.15811585,361.03915437)(607.26744919,362.7884877)
\curveto(608.41944919,364.5804877)(608.99544919,367.2044877)(608.99544919,370.6604877)
\curveto(608.99544919,374.1164877)(608.41944919,376.69782104)(607.26744919,378.4044877)
\curveto(606.15811585,380.15382104)(604.32344919,381.0284877)(601.76344919,381.0284877)
\curveto(599.24611585,381.0284877)(597.41144919,380.15382104)(596.25944919,378.4044877)
\curveto(595.15011585,376.69782104)(594.59544919,374.1164877)(594.59544919,370.6604877)
\closepath
}
}
{
\newrgbcolor{curcolor}{0 0 0}
\pscustom[linestyle=none,fillstyle=solid,fillcolor=curcolor]
{
\newpath
\moveto(670.62741598,388.0684877)
\lineto(670.62741598,353.1244877)
\lineto(661.73141598,353.1244877)
\lineto(661.73141598,370.2764877)
\curveto(661.73141598,371.98315437)(661.75274932,373.64715437)(661.79541598,375.2684877)
\curveto(661.88074932,376.88982104)(661.98741598,378.38315437)(662.11541598,379.7484877)
\lineto(661.92341598,379.7484877)
\lineto(652.25941598,353.1244877)
\lineto(645.09141598,353.1244877)
\lineto(635.29941598,379.8124877)
\lineto(635.04341598,379.8124877)
\curveto(635.21408265,378.4044877)(635.32074932,376.88982104)(635.36341598,375.2684877)
\curveto(635.44874932,373.68982104)(635.49141598,371.9404877)(635.49141598,370.0204877)
\lineto(635.49141598,353.1244877)
\lineto(626.59541598,353.1244877)
\lineto(626.59541598,388.0684877)
\lineto(640.09941598,388.0684877)
\lineto(648.80341598,364.3884877)
\lineto(657.63541598,388.0684877)
\closepath
}
}
{
\newrgbcolor{curcolor}{0 0 0}
\pscustom[linestyle=none,fillstyle=solid,fillcolor=curcolor]
{
\newpath
\moveto(690.1474072,388.0684877)
\lineto(690.1474072,374.6284877)
\lineto(703.4594072,374.6284877)
\lineto(703.4594072,388.0684877)
\lineto(712.9954072,388.0684877)
\lineto(712.9954072,353.1244877)
\lineto(703.4594072,353.1244877)
\lineto(703.4594072,367.5244877)
\lineto(690.1474072,367.5244877)
\lineto(690.1474072,353.1244877)
\lineto(680.6114072,353.1244877)
\lineto(680.6114072,388.0684877)
\closepath
}
}
{
\newrgbcolor{curcolor}{0 0 0}
\pscustom[linestyle=none,fillstyle=solid,fillcolor=curcolor]
{
\newpath
\moveto(737.31544919,388.7724877)
\curveto(742.00878252,388.7724877)(745.59278252,387.7484877)(748.06744919,385.7004877)
\curveto(750.58478252,383.69515437)(751.84344919,380.60182104)(751.84344919,376.4204877)
\lineto(751.84344919,353.1244877)
\lineto(745.18744919,353.1244877)
\lineto(743.33144919,357.8604877)
\lineto(743.07544919,357.8604877)
\curveto(741.58211585,355.98315437)(740.00344919,354.61782104)(738.33944919,353.7644877)
\curveto(736.67544919,352.91115437)(734.39278252,352.4844877)(731.49144919,352.4844877)
\curveto(728.37678252,352.4844877)(725.79544919,353.3804877)(723.74744919,355.1724877)
\curveto(721.69944919,356.9644877)(720.67544919,359.75915437)(720.67544919,363.5564877)
\curveto(720.67544919,367.2684877)(721.97678252,369.99915437)(724.57944919,371.7484877)
\curveto(727.18211585,373.49782104)(731.08611585,374.47915437)(736.29144919,374.6924877)
\lineto(742.37144919,374.8844877)
\lineto(742.37144919,376.4204877)
\curveto(742.37144919,378.25515437)(741.88078252,379.59915437)(740.89944919,380.4524877)
\curveto(739.96078252,381.30582104)(738.63811585,381.7324877)(736.93144919,381.7324877)
\curveto(735.22478252,381.7324877)(733.56078252,381.4764877)(731.93944919,380.9644877)
\curveto(730.31811585,380.49515437)(728.69678252,379.89782104)(727.07544919,379.1724877)
\lineto(723.93944919,385.6364877)
\curveto(725.77411585,386.57515437)(727.84344919,387.32182104)(730.14744919,387.8764877)
\curveto(732.45144919,388.47382104)(734.84078252,388.7724877)(737.31544919,388.7724877)
\closepath
\moveto(742.37144919,369.3164877)
\lineto(738.65944919,369.1884877)
\curveto(735.58744919,369.10315437)(733.45411585,368.5484877)(732.25944919,367.5244877)
\curveto(731.06478252,366.5004877)(730.46744919,365.1564877)(730.46744919,363.4924877)
\curveto(730.46744919,362.04182104)(730.89411585,360.9964877)(731.74744919,360.3564877)
\curveto(732.60078252,359.75915437)(733.71011585,359.4604877)(735.07544919,359.4604877)
\curveto(737.12344919,359.4604877)(738.85144919,360.05782104)(740.25944919,361.2524877)
\curveto(741.66744919,362.48982104)(742.37144919,364.21782104)(742.37144919,366.4364877)
\closepath
}
}
{
\newrgbcolor{curcolor}{0 0 0}
\pscustom[linestyle=none,fillstyle=solid,fillcolor=curcolor]
{
\newpath
\moveto(790.56345505,380.9004877)
\lineto(779.10745505,380.9004877)
\lineto(779.10745505,353.1244877)
\lineto(769.57145505,353.1244877)
\lineto(769.57145505,380.9004877)
\lineto(758.11545505,380.9004877)
\lineto(758.11545505,388.0684877)
\lineto(790.56345505,388.0684877)
\closepath
}
}
{
\newrgbcolor{curcolor}{0 0 0}
\pscustom[linestyle=none,fillstyle=solid,fillcolor=curcolor]
{
\newpath
\moveto(797.02742282,353.1244877)
\lineto(797.02742282,388.0684877)
\lineto(806.56342282,388.0684877)
\lineto(806.56342282,374.5644877)
\lineto(811.17142282,374.5644877)
\curveto(816.50475615,374.5644877)(820.45142282,373.71115437)(823.01142282,372.0044877)
\curveto(825.57142282,370.29782104)(826.85142282,367.7164877)(826.85142282,364.2604877)
\curveto(826.85142282,360.84715437)(825.65675615,358.13782104)(823.26742282,356.1324877)
\curveto(820.87808949,354.12715437)(816.95275615,353.1244877)(811.49142282,353.1244877)
\closepath
\moveto(831.90742282,353.1244877)
\lineto(831.90742282,388.0684877)
\lineto(841.44342282,388.0684877)
\lineto(841.44342282,353.1244877)
\closepath
\moveto(806.56342282,359.7164877)
\lineto(810.97942282,359.7164877)
\curveto(812.85675615,359.7164877)(814.37142282,360.0364877)(815.52342282,360.6764877)
\curveto(816.71808949,361.35915437)(817.31542282,362.51115437)(817.31542282,364.1324877)
\curveto(817.31542282,366.6924877)(815.16075615,367.9724877)(810.85142282,367.9724877)
\lineto(806.56342282,367.9724877)
\closepath
}
}
{
\newrgbcolor{curcolor}{0 0 0}
\pscustom[linestyle=none,fillstyle=solid,fillcolor=curcolor]
{
\newpath
\moveto(90.6615468,565.99789197)
\curveto(90.6615468,564.2058935)(90.02688067,562.76233917)(88.75754842,561.667229)
\curveto(87.48821617,560.57211882)(85.87043977,559.88767496)(83.90421922,559.61389741)
\lineto(83.90421922,559.50189751)
\curveto(86.34332825,559.25300883)(88.20999333,558.56856497)(89.50421445,557.44856592)
\curveto(90.82332444,556.35345574)(91.48287943,554.92234585)(91.48287943,553.15523624)
\curveto(91.48287943,550.81568268)(90.52465803,548.89923987)(88.60821522,547.40590781)
\curveto(86.71666127,545.93746461)(83.92910809,545.20324301)(80.24555567,545.20324301)
\curveto(78.22955739,545.20324301)(76.43755891,545.32768735)(74.86956025,545.57657603)
\curveto(73.32645045,545.82546471)(71.98245159,546.18635329)(70.83756368,546.65924177)
\lineto(70.83756368,551.40057107)
\curveto(71.60911858,551.02723806)(72.48022895,550.70368278)(73.45089479,550.42990523)
\curveto(74.42156063,550.18101655)(75.4046709,549.98190561)(76.40022561,549.83257241)
\curveto(77.39578032,549.70812807)(78.31666842,549.6459059)(79.16288992,549.6459059)
\curveto(81.52733236,549.6459059)(83.23221979,549.98190561)(84.27755224,550.65390504)
\curveto(85.34777355,551.35079334)(85.8828842,552.32145918)(85.8828842,553.56590256)
\curveto(85.8828842,554.83523481)(85.1113293,555.75612292)(83.56821951,556.32856687)
\curveto(82.02510971,556.9258997)(79.94688926,557.22456611)(77.33355815,557.22456611)
\lineto(74.83222695,557.22456611)
\lineto(74.83222695,561.62989569)
\lineto(77.07222504,561.62989569)
\curveto(79.18777879,561.62989569)(80.83044406,561.76678447)(82.00022084,562.04056201)
\curveto(83.16999762,562.31433956)(83.99133026,562.71256144)(84.46421874,563.23522766)
\curveto(84.93710723,563.75789388)(85.17355147,564.38011557)(85.17355147,565.10189274)
\curveto(85.17355147,566.02278084)(84.76288516,566.74455801)(83.94155252,567.26722423)
\curveto(83.12021989,567.81477932)(81.91310981,568.08855686)(80.32022227,568.08855686)
\curveto(78.92644568,568.08855686)(77.63222456,567.88944592)(76.43755891,567.49122404)
\curveto(75.24289326,567.09300215)(74.12289422,566.57033593)(73.07756177,565.92322537)
\lineto(70.61356387,569.69388883)
\curveto(71.98245159,570.58988806)(73.50067252,571.29922079)(75.16822666,571.82188702)
\curveto(76.8357808,572.34455324)(78.81444578,572.60588635)(81.1042216,572.60588635)
\curveto(84.1157746,572.60588635)(86.45532816,571.98366466)(88.1228823,570.73922127)
\curveto(89.8153253,569.49477789)(90.6615468,567.91433479)(90.6615468,565.99789197)
\closepath
}
}
{
\newrgbcolor{curcolor}{0 0 0}
\pscustom[linestyle=none,fillstyle=solid,fillcolor=curcolor]
{
\newpath
\moveto(104.36289613,566.37122499)
\curveto(107.10067158,566.37122499)(109.19133646,565.77389217)(110.63489079,564.57922652)
\curveto(112.10333398,563.40944973)(112.83755558,561.60500683)(112.83755558,559.16589779)
\lineto(112.83755558,545.57657603)
\lineto(108.95489222,545.57657603)
\lineto(107.87222648,548.33924034)
\lineto(107.72289327,548.33924034)
\curveto(106.8517829,547.24413017)(105.9308948,546.4476864)(104.96022896,545.94990905)
\curveto(103.98956312,545.45213169)(102.65800869,545.20324301)(100.96556569,545.20324301)
\curveto(99.14867835,545.20324301)(97.64290185,545.72590924)(96.4482362,546.77124168)
\curveto(95.25357055,547.81657412)(94.65623773,549.44679496)(94.65623773,551.66190418)
\curveto(94.65623773,553.82723567)(95.41534819,555.4201232)(96.93356912,556.44056678)
\curveto(98.45179005,557.46101035)(100.72912145,558.03345431)(103.76556331,558.15789865)
\lineto(107.31222695,558.26989855)
\lineto(107.31222695,559.16589779)
\curveto(107.31222695,560.2361191)(107.02600497,561.02011844)(106.45356102,561.51789579)
\curveto(105.90600593,562.01567314)(105.13445103,562.26456182)(104.13889632,562.26456182)
\curveto(103.14334161,562.26456182)(102.17267577,562.11522861)(101.2268988,561.8165622)
\curveto(100.28112183,561.54278466)(99.33534486,561.19434051)(98.38956788,560.77122976)
\lineto(96.56023611,564.54189321)
\curveto(97.63045742,565.0894483)(98.8375675,565.52500349)(100.18156636,565.84855877)
\curveto(101.52556521,566.19700292)(102.9193418,566.37122499)(104.36289613,566.37122499)
\closepath
\moveto(107.31222695,555.02190132)
\lineto(105.14689546,554.94723472)
\curveto(103.35489699,554.89745698)(102.1104536,554.5739017)(101.41356531,553.97656888)
\curveto(100.71667701,553.37923605)(100.36823287,552.59523672)(100.36823287,551.62457088)
\curveto(100.36823287,550.77834938)(100.61712154,550.16857212)(101.1148989,549.7952391)
\curveto(101.61267625,549.44679496)(102.25978681,549.27257288)(103.05623058,549.27257288)
\curveto(104.25089623,549.27257288)(105.25889537,549.62101703)(106.080228,550.31790533)
\curveto(106.90156064,551.03968249)(107.31222695,552.04768163)(107.31222695,553.34190275)
\closepath
}
}
{
\newrgbcolor{curcolor}{0 0 0}
\pscustom[linestyle=none,fillstyle=solid,fillcolor=curcolor]
{
\newpath
\moveto(137.06687246,565.96055867)
\lineto(137.06687246,545.57657603)
\lineto(131.50421053,545.57657603)
\lineto(131.50421053,561.7792289)
\lineto(124.11221683,561.7792289)
\lineto(124.11221683,545.57657603)
\lineto(118.5495549,545.57657603)
\lineto(118.5495549,565.96055867)
\closepath
}
}
{
\newrgbcolor{curcolor}{0 0 0}
\pscustom[linestyle=none,fillstyle=solid,fillcolor=curcolor]
{
\newpath
\moveto(154.24021005,566.33389169)
\curveto(156.52998588,566.33389169)(158.38420652,565.43789245)(159.80287198,563.64589398)
\curveto(161.22153744,561.87878437)(161.93087017,559.26545326)(161.93087017,555.80590065)
\curveto(161.93087017,552.32145918)(161.19664857,549.6832392)(159.72820538,547.89124073)
\curveto(158.25976218,546.09924225)(156.38065267,545.20324301)(154.09087684,545.20324301)
\curveto(152.62243365,545.20324301)(151.45265687,545.46457613)(150.5815465,545.98724235)
\curveto(149.71043613,546.53479744)(149.0011034,547.14457469)(148.45354831,547.81657412)
\lineto(148.1548819,547.81657412)
\curveto(148.35399284,546.77124168)(148.45354831,545.77568697)(148.45354831,544.82991)
\lineto(148.45354831,536.61658366)
\lineto(142.89088638,536.61658366)
\lineto(142.89088638,565.96055867)
\lineto(147.40821587,565.96055867)
\lineto(148.1922152,563.30989426)
\lineto(148.45354831,563.30989426)
\curveto(149.0011034,564.1312269)(149.735325,564.84055963)(150.6562131,565.43789245)
\curveto(151.57710121,566.03522528)(152.77176686,566.33389169)(154.24021005,566.33389169)
\closepath
\moveto(152.44821158,561.8912288)
\curveto(151.00465725,561.8912288)(149.98421367,561.43078475)(149.38688085,560.50989665)
\curveto(148.78954802,559.61389741)(148.47843718,558.25745412)(148.45354831,556.44056678)
\lineto(148.45354831,555.84323395)
\curveto(148.45354831,553.87701341)(148.73977029,552.35879248)(149.31221425,551.28857117)
\curveto(149.90954707,550.24323872)(150.97976838,549.7205725)(152.52287818,549.7205725)
\curveto(153.79221043,549.7205725)(154.72554297,550.24323872)(155.3228758,551.28857117)
\curveto(155.94509749,552.35879248)(156.25620833,553.88945784)(156.25620833,555.88056726)
\curveto(156.25620833,559.88767496)(154.98687608,561.8912288)(152.44821158,561.8912288)
\closepath
}
}
{
\newrgbcolor{curcolor}{0 0 0}
\pscustom[linestyle=none,fillstyle=solid,fillcolor=curcolor]
{
\newpath
\moveto(185.04016195,555.80590065)
\curveto(185.04016195,552.42101465)(184.14416272,549.80768354)(182.35216424,547.96590733)
\curveto(180.58505464,546.12413112)(178.17083447,545.20324301)(175.10950374,545.20324301)
\curveto(173.2179498,545.20324301)(171.5255068,545.61390933)(170.03217473,546.43524197)
\curveto(168.56373154,547.2565746)(167.40639919,548.45124025)(166.56017769,550.01923891)
\curveto(165.71395619,551.61212645)(165.29084544,553.54101369)(165.29084544,555.80590065)
\curveto(165.29084544,559.19078666)(166.17440024,561.79167333)(167.94150985,563.60856068)
\curveto(169.70861945,565.42544802)(172.13528405,566.33389169)(175.22150365,566.33389169)
\curveto(177.13794646,566.33389169)(178.83038946,565.92322537)(180.29883266,565.10189274)
\curveto(181.76727585,564.2805601)(182.9246082,563.08589445)(183.7708297,561.51789579)
\curveto(184.6170512,559.94989712)(185.04016195,558.04589875)(185.04016195,555.80590065)
\closepath
\moveto(170.96550727,555.80590065)
\curveto(170.96550727,553.78990237)(171.28906255,552.25923701)(171.93617311,551.21390456)
\curveto(172.60817254,550.19346099)(173.69083829,549.6832392)(175.18417035,549.6832392)
\curveto(176.65261354,549.6832392)(177.71039042,550.19346099)(178.35750098,551.21390456)
\curveto(179.02950041,552.25923701)(179.36550012,553.78990237)(179.36550012,555.80590065)
\curveto(179.36550012,557.82189894)(179.02950041,559.32767543)(178.35750098,560.32323014)
\curveto(177.71039042,561.34367372)(176.64016911,561.8538955)(175.14683705,561.8538955)
\curveto(173.67839385,561.8538955)(172.60817254,561.34367372)(171.93617311,560.32323014)
\curveto(171.28906255,559.32767543)(170.96550727,557.82189894)(170.96550727,555.80590065)
\closepath
}
}
{
\newrgbcolor{curcolor}{0 0 0}
\pscustom[linestyle=none,fillstyle=solid,fillcolor=curcolor]
{
\newpath
\moveto(197.92012841,545.20324301)
\curveto(194.88368655,545.20324301)(192.53168855,546.03702008)(190.86413442,547.70457422)
\curveto(189.22146915,549.37212835)(188.40013651,552.02279276)(188.40013651,555.65656745)
\curveto(188.40013651,558.14545422)(188.82324727,560.17389693)(189.66946877,561.7418956)
\curveto(190.51569027,563.30989426)(191.68546705,564.46722661)(193.17879911,565.21389264)
\curveto(194.69702004,565.96055867)(196.43924078,566.33389169)(198.40546133,566.33389169)
\curveto(199.79923792,566.33389169)(201.006348,566.19700292)(202.02679158,565.92322537)
\curveto(203.07212402,565.64944783)(203.98056769,565.32589255)(204.75212259,564.95255953)
\lineto(203.10945732,560.65922985)
\curveto(202.23834695,561.007674)(201.41701432,561.29389598)(200.64545942,561.51789579)
\curveto(199.89879339,561.7418956)(199.15212736,561.8538955)(198.40546133,561.8538955)
\curveto(195.51835268,561.8538955)(194.07479835,559.80056392)(194.07479835,555.69390075)
\curveto(194.07479835,553.6530136)(194.44813136,552.1472371)(195.1947974,551.17657126)
\curveto(195.96635229,550.20590542)(197.0365736,549.7205725)(198.40546133,549.7205725)
\curveto(199.57523811,549.7205725)(200.60812612,549.86990571)(201.50412536,550.16857212)
\curveto(202.40012459,550.4921274)(203.27123496,550.92768258)(204.11745646,551.47523767)
\lineto(204.11745646,546.73390838)
\curveto(203.27123496,546.18635329)(202.37523573,545.80057584)(201.42945875,545.57657603)
\curveto(200.50857065,545.32768735)(199.33879387,545.20324301)(197.92012841,545.20324301)
\closepath
}
}
{
\newrgbcolor{curcolor}{0 0 0}
\pscustom[linestyle=none,fillstyle=solid,fillcolor=curcolor]
{
\newpath
\moveto(237.04542624,565.96055867)
\lineto(237.04542624,545.57657603)
\lineto(231.48276431,545.57657603)
\lineto(231.48276431,561.7792289)
\lineto(224.0907706,561.7792289)
\lineto(224.0907706,545.57657603)
\lineto(218.52810867,545.57657603)
\lineto(218.52810867,565.96055867)
\closepath
}
}
{
\newrgbcolor{curcolor}{0 0 0}
\pscustom[linestyle=none,fillstyle=solid,fillcolor=curcolor]
{
\newpath
\moveto(254.21877145,566.33389169)
\curveto(256.50854728,566.33389169)(258.36276792,565.43789245)(259.78143338,563.64589398)
\curveto(261.20009884,561.87878437)(261.90943157,559.26545326)(261.90943157,555.80590065)
\curveto(261.90943157,552.32145918)(261.17520997,549.6832392)(259.70676678,547.89124073)
\curveto(258.23832358,546.09924225)(256.35921407,545.20324301)(254.06943825,545.20324301)
\curveto(252.60099505,545.20324301)(251.43121827,545.46457613)(250.5601079,545.98724235)
\curveto(249.68899753,546.53479744)(248.9796648,547.14457469)(248.43210971,547.81657412)
\lineto(248.1334433,547.81657412)
\curveto(248.33255424,546.77124168)(248.43210971,545.77568697)(248.43210971,544.82991)
\lineto(248.43210971,536.61658366)
\lineto(242.86944778,536.61658366)
\lineto(242.86944778,565.96055867)
\lineto(247.38677727,565.96055867)
\lineto(248.1707766,563.30989426)
\lineto(248.43210971,563.30989426)
\curveto(248.9796648,564.1312269)(249.7138864,564.84055963)(250.6347745,565.43789245)
\curveto(251.55566261,566.03522528)(252.75032826,566.33389169)(254.21877145,566.33389169)
\closepath
\moveto(252.42677298,561.8912288)
\curveto(250.98321865,561.8912288)(249.96277508,561.43078475)(249.36544225,560.50989665)
\curveto(248.76810943,559.61389741)(248.45699858,558.25745412)(248.43210971,556.44056678)
\lineto(248.43210971,555.84323395)
\curveto(248.43210971,553.87701341)(248.71833169,552.35879248)(249.29077565,551.28857117)
\curveto(249.88810847,550.24323872)(250.95832978,549.7205725)(252.50143958,549.7205725)
\curveto(253.77077183,549.7205725)(254.70410437,550.24323872)(255.3014372,551.28857117)
\curveto(255.92365889,552.35879248)(256.23476973,553.88945784)(256.23476973,555.88056726)
\curveto(256.23476973,559.88767496)(254.96543748,561.8912288)(252.42677298,561.8912288)
\closepath
}
}
{
\newrgbcolor{curcolor}{0 0 0}
\pscustom[linestyle=none,fillstyle=solid,fillcolor=curcolor]
{
\newpath
\moveto(274.86406534,566.37122499)
\curveto(277.60184078,566.37122499)(279.69250567,565.77389217)(281.13606,564.57922652)
\curveto(282.60450319,563.40944973)(283.33872479,561.60500683)(283.33872479,559.16589779)
\lineto(283.33872479,545.57657603)
\lineto(279.45606143,545.57657603)
\lineto(278.37339568,548.33924034)
\lineto(278.22406247,548.33924034)
\curveto(277.35295211,547.24413017)(276.432064,546.4476864)(275.46139816,545.94990905)
\curveto(274.49073232,545.45213169)(273.1591779,545.20324301)(271.4667349,545.20324301)
\curveto(269.64984755,545.20324301)(268.14407106,545.72590924)(266.94940541,546.77124168)
\curveto(265.75473976,547.81657412)(265.15740693,549.44679496)(265.15740693,551.66190418)
\curveto(265.15740693,553.82723567)(265.9165174,555.4201232)(267.43473833,556.44056678)
\curveto(268.95295926,557.46101035)(271.23029065,558.03345431)(274.26673251,558.15789865)
\lineto(277.81339616,558.26989855)
\lineto(277.81339616,559.16589779)
\curveto(277.81339616,560.2361191)(277.52717418,561.02011844)(276.95473022,561.51789579)
\curveto(276.40717513,562.01567314)(275.63562023,562.26456182)(274.64006553,562.26456182)
\curveto(273.64451082,562.26456182)(272.67384498,562.11522861)(271.72806801,561.8165622)
\curveto(270.78229103,561.54278466)(269.83651406,561.19434051)(268.89073709,560.77122976)
\lineto(267.06140531,564.54189321)
\curveto(268.13162662,565.0894483)(269.33873671,565.52500349)(270.68273556,565.84855877)
\curveto(272.02673442,566.19700292)(273.42051101,566.37122499)(274.86406534,566.37122499)
\closepath
\moveto(277.81339616,555.02190132)
\lineto(275.64806467,554.94723472)
\curveto(273.85606619,554.89745698)(272.61162281,554.5739017)(271.91473451,553.97656888)
\curveto(271.21784622,553.37923605)(270.86940207,552.59523672)(270.86940207,551.62457088)
\curveto(270.86940207,550.77834938)(271.11829075,550.16857212)(271.6160681,549.7952391)
\curveto(272.11384546,549.44679496)(272.76095602,549.27257288)(273.55739978,549.27257288)
\curveto(274.75206543,549.27257288)(275.76006457,549.62101703)(276.58139721,550.31790533)
\curveto(277.40272984,551.03968249)(277.81339616,552.04768163)(277.81339616,553.34190275)
\closepath
}
}
{
\newrgbcolor{curcolor}{0 0 0}
\pscustom[linestyle=none,fillstyle=solid,fillcolor=curcolor]
{
\newpath
\moveto(307.12004014,560.62189655)
\curveto(307.12004014,559.52678637)(306.771596,558.59345383)(306.0747077,557.82189894)
\curveto(305.40270827,557.05034404)(304.39470913,556.55256668)(303.05071028,556.32856687)
\lineto(303.05071028,556.17923367)
\curveto(304.46937573,556.00501159)(305.60181921,555.50723424)(306.44804072,554.68590161)
\curveto(307.31915109,553.88945784)(307.75470627,552.8814587)(307.75470627,551.66190418)
\curveto(307.75470627,550.4921274)(307.44359542,549.44679496)(306.82137373,548.52590685)
\curveto(306.22404091,547.60501875)(305.2658195,546.88324158)(303.94670951,546.36057536)
\curveto(302.62759952,545.83790914)(300.89782322,545.57657603)(298.7573806,545.57657603)
\lineto(289.0507222,545.57657603)
\lineto(289.0507222,565.96055867)
\lineto(298.7573806,565.96055867)
\curveto(300.35026813,565.96055867)(301.76893359,565.7863366)(303.01337697,565.43789245)
\curveto(304.28270923,565.11433717)(305.27826393,564.55433765)(306.0000411,563.75789388)
\curveto(306.74670713,562.98633898)(307.12004014,561.94100654)(307.12004014,560.62189655)
\closepath
\moveto(301.48271161,560.17389693)
\curveto(301.48271161,561.41834032)(300.49960134,562.04056201)(298.53338079,562.04056201)
\lineto(294.61338413,562.04056201)
\lineto(294.61338413,558.00856544)
\lineto(297.89871466,558.00856544)
\curveto(299.06849144,558.00856544)(299.95204625,558.17034308)(300.54937907,558.49389836)
\curveto(301.17160076,558.84234251)(301.48271161,559.40234204)(301.48271161,560.17389693)
\closepath
\moveto(302.00537783,551.96057059)
\curveto(302.00537783,552.75701436)(301.68182255,553.32945832)(301.03471199,553.67790246)
\curveto(300.4124903,554.05123548)(299.49160219,554.23790199)(298.27204768,554.23790199)
\lineto(294.61338413,554.23790199)
\lineto(294.61338413,549.42190609)
\lineto(298.38404758,549.42190609)
\curveto(299.42938003,549.42190609)(300.28804596,549.6085726)(300.96004539,549.98190561)
\curveto(301.65693368,550.3801275)(302.00537783,551.03968249)(302.00537783,551.96057059)
\closepath
}
}
{
\newrgbcolor{curcolor}{0 0 0}
\pscustom[linestyle=none,fillstyle=solid,fillcolor=curcolor]
{
\newpath
\moveto(327.80268466,565.96055867)
\lineto(327.80268466,558.12056535)
\lineto(335.56801138,558.12056535)
\lineto(335.56801138,565.96055867)
\lineto(341.13067331,565.96055867)
\lineto(341.13067331,545.57657603)
\lineto(335.56801138,545.57657603)
\lineto(335.56801138,553.97656888)
\lineto(327.80268466,553.97656888)
\lineto(327.80268466,545.57657603)
\lineto(322.24002273,545.57657603)
\lineto(322.24002273,565.96055867)
\closepath
}
}
{
\newrgbcolor{curcolor}{0 0 0}
\pscustom[linestyle=none,fillstyle=solid,fillcolor=curcolor]
{
\newpath
\moveto(355.31734292,566.37122499)
\curveto(358.05511837,566.37122499)(360.14578326,565.77389217)(361.58933758,564.57922652)
\curveto(363.05778078,563.40944973)(363.79200237,561.60500683)(363.79200237,559.16589779)
\lineto(363.79200237,545.57657603)
\lineto(359.90933901,545.57657603)
\lineto(358.82667327,548.33924034)
\lineto(358.67734006,548.33924034)
\curveto(357.80622969,547.24413017)(356.88534159,546.4476864)(355.91467575,545.94990905)
\curveto(354.94400991,545.45213169)(353.61245549,545.20324301)(351.92001248,545.20324301)
\curveto(350.10312514,545.20324301)(348.59734865,545.72590924)(347.402683,546.77124168)
\curveto(346.20801735,547.81657412)(345.61068452,549.44679496)(345.61068452,551.66190418)
\curveto(345.61068452,553.82723567)(346.36979499,555.4201232)(347.88801592,556.44056678)
\curveto(349.40623685,557.46101035)(351.68356824,558.03345431)(354.7200101,558.15789865)
\lineto(358.26667375,558.26989855)
\lineto(358.26667375,559.16589779)
\curveto(358.26667375,560.2361191)(357.98045177,561.02011844)(357.40800781,561.51789579)
\curveto(356.86045272,562.01567314)(356.08889782,562.26456182)(355.09334311,562.26456182)
\curveto(354.09778841,562.26456182)(353.12712257,562.11522861)(352.18134559,561.8165622)
\curveto(351.23556862,561.54278466)(350.28979165,561.19434051)(349.34401468,560.77122976)
\lineto(347.5146829,564.54189321)
\curveto(348.58490421,565.0894483)(349.7920143,565.52500349)(351.13601315,565.84855877)
\curveto(352.48001201,566.19700292)(353.8737886,566.37122499)(355.31734292,566.37122499)
\closepath
\moveto(358.26667375,555.02190132)
\lineto(356.10134226,554.94723472)
\curveto(354.30934378,554.89745698)(353.0649004,554.5739017)(352.3680121,553.97656888)
\curveto(351.67112381,553.37923605)(351.32267966,552.59523672)(351.32267966,551.62457088)
\curveto(351.32267966,550.77834938)(351.57156834,550.16857212)(352.06934569,549.7952391)
\curveto(352.56712304,549.44679496)(353.2142336,549.27257288)(354.01067737,549.27257288)
\curveto(355.20534302,549.27257288)(356.21334216,549.62101703)(357.03467479,550.31790533)
\curveto(357.85600743,551.03968249)(358.26667375,552.04768163)(358.26667375,553.34190275)
\closepath
}
}
{
\newrgbcolor{curcolor}{0 0 0}
\pscustom[linestyle=none,fillstyle=solid,fillcolor=curcolor]
{
\newpath
\moveto(390.55998849,566.33389169)
\curveto(392.84976432,566.33389169)(394.70398496,565.43789245)(396.12265042,563.64589398)
\curveto(397.54131588,561.87878437)(398.25064861,559.26545326)(398.25064861,555.80590065)
\curveto(398.25064861,552.32145918)(397.51642701,549.6832392)(396.04798382,547.89124073)
\curveto(394.57954063,546.09924225)(392.70043111,545.20324301)(390.41065529,545.20324301)
\curveto(388.94221209,545.20324301)(387.77243531,545.46457613)(386.90132494,545.98724235)
\curveto(386.03021457,546.53479744)(385.32088184,547.14457469)(384.77332675,547.81657412)
\lineto(384.47466034,547.81657412)
\curveto(384.67377128,546.77124168)(384.77332675,545.77568697)(384.77332675,544.82991)
\lineto(384.77332675,536.61658366)
\lineto(379.21066482,536.61658366)
\lineto(379.21066482,565.96055867)
\lineto(383.72799431,565.96055867)
\lineto(384.51199364,563.30989426)
\lineto(384.77332675,563.30989426)
\curveto(385.32088184,564.1312269)(386.05510344,564.84055963)(386.97599154,565.43789245)
\curveto(387.89687965,566.03522528)(389.0915453,566.33389169)(390.55998849,566.33389169)
\closepath
\moveto(388.76799002,561.8912288)
\curveto(387.32443569,561.8912288)(386.30399212,561.43078475)(385.70665929,560.50989665)
\curveto(385.10932647,559.61389741)(384.79821562,558.25745412)(384.77332675,556.44056678)
\lineto(384.77332675,555.84323395)
\curveto(384.77332675,553.87701341)(385.05954873,552.35879248)(385.63199269,551.28857117)
\curveto(386.22932551,550.24323872)(387.29954682,549.7205725)(388.84265662,549.7205725)
\curveto(390.11198887,549.7205725)(391.04532141,550.24323872)(391.64265424,551.28857117)
\curveto(392.26487593,552.35879248)(392.57598678,553.88945784)(392.57598678,555.88056726)
\curveto(392.57598678,559.88767496)(391.30665452,561.8912288)(388.76799002,561.8912288)
\closepath
}
}
{
\newrgbcolor{curcolor}{0 0 0}
\pscustom[linestyle=none,fillstyle=solid,fillcolor=curcolor]
{
\newpath
\moveto(411.24261568,566.33389169)
\curveto(414.05505773,566.33389169)(416.28261139,565.52500349)(417.92527665,563.90722709)
\curveto(419.56794192,562.31433956)(420.38927456,560.03700816)(420.38927456,557.07523291)
\lineto(420.38927456,554.38723519)
\lineto(407.24795241,554.38723519)
\curveto(407.29773015,552.81923653)(407.7581742,551.58723758)(408.62928457,550.69123834)
\curveto(409.52528381,549.7952391)(410.75728276,549.34723949)(412.32528142,549.34723949)
\curveto(413.61950254,549.34723949)(414.80172376,549.47168382)(415.87194507,549.7205725)
\curveto(416.96705525,549.99435005)(418.08705429,550.40501636)(419.23194221,550.95257145)
\lineto(419.23194221,546.65924177)
\curveto(418.21149863,546.16146442)(417.15372176,545.80057584)(416.05861158,545.57657603)
\curveto(414.9635014,545.32768735)(413.63194698,545.20324301)(412.06394831,545.20324301)
\curveto(410.02306116,545.20324301)(408.21861825,545.57657603)(406.65061959,546.32324206)
\curveto(405.08262092,547.09479696)(403.85062197,548.23968487)(402.95462274,549.7579058)
\curveto(402.0586235,551.3010156)(401.61062388,553.25479171)(401.61062388,555.61923415)
\curveto(401.61062388,557.98367658)(402.00884576,559.96234156)(402.80528953,561.55522909)
\curveto(403.62662216,563.14811662)(404.75906564,564.34278227)(406.20261997,565.13922604)
\curveto(407.6461743,565.93566981)(409.32617287,566.33389169)(411.24261568,566.33389169)
\closepath
\moveto(411.27994898,562.37656172)
\curveto(410.1848388,562.37656172)(409.28883956,562.02811758)(408.59195127,561.33122928)
\curveto(407.89506297,560.63434099)(407.48439666,559.55167524)(407.35995232,558.08323205)
\lineto(415.16261234,558.08323205)
\curveto(415.13772347,559.30278656)(414.80172376,560.32323014)(414.1546132,561.14456277)
\curveto(413.53239151,561.96589541)(412.5741701,562.37656172)(411.27994898,562.37656172)
\closepath
}
}
{
\newrgbcolor{curcolor}{0 0 0}
\pscustom[linestyle=none,fillstyle=solid,fillcolor=curcolor]
{
\newpath
\moveto(443.23726925,565.96055867)
\lineto(443.23726925,549.6459059)
\lineto(446.22393337,549.6459059)
\lineto(446.22393337,538.25924893)
\lineto(441.22127097,538.25924893)
\lineto(441.22127097,545.57657603)
\lineto(427.5199493,545.57657603)
\lineto(427.5199493,538.25924893)
\lineto(422.51728689,538.25924893)
\lineto(422.51728689,549.6459059)
\lineto(424.23461876,549.6459059)
\curveto(425.130618,551.01479362)(425.88972847,552.57034785)(426.51195016,554.31256859)
\curveto(427.13417185,556.0796782)(427.63194921,557.94634327)(428.00528222,559.91256382)
\curveto(428.37861524,561.90367324)(428.65239278,563.91967152)(428.82661485,565.96055867)
\closepath
\moveto(437.67460732,561.7792289)
\lineto(433.49327755,561.7792289)
\curveto(433.19461114,559.51434194)(432.78394482,557.36145488)(432.2612786,555.32056773)
\curveto(431.73861237,553.30456945)(431.00439078,551.4130155)(430.05861381,549.6459059)
\lineto(437.67460732,549.6459059)
\closepath
}
}
{
\newrgbcolor{curcolor}{0 0 0}
\pscustom[linestyle=none,fillstyle=solid,fillcolor=curcolor]
{
\newpath
\moveto(458.05855386,566.37122499)
\curveto(460.79632931,566.37122499)(462.88699419,565.77389217)(464.33054852,564.57922652)
\curveto(465.79899171,563.40944973)(466.53321331,561.60500683)(466.53321331,559.16589779)
\lineto(466.53321331,545.57657603)
\lineto(462.65054995,545.57657603)
\lineto(461.56788421,548.33924034)
\lineto(461.418551,548.33924034)
\curveto(460.54744063,547.24413017)(459.62655253,546.4476864)(458.65588669,545.94990905)
\curveto(457.68522085,545.45213169)(456.35366642,545.20324301)(454.66122342,545.20324301)
\curveto(452.84433608,545.20324301)(451.33855958,545.72590924)(450.14389393,546.77124168)
\curveto(448.94922828,547.81657412)(448.35189546,549.44679496)(448.35189546,551.66190418)
\curveto(448.35189546,553.82723567)(449.11100592,555.4201232)(450.62922685,556.44056678)
\curveto(452.14744778,557.46101035)(454.42477918,558.03345431)(457.46122104,558.15789865)
\lineto(461.00788468,558.26989855)
\lineto(461.00788468,559.16589779)
\curveto(461.00788468,560.2361191)(460.7216627,561.02011844)(460.14921875,561.51789579)
\curveto(459.60166366,562.01567314)(458.83010876,562.26456182)(457.83455405,562.26456182)
\curveto(456.83899934,562.26456182)(455.8683335,562.11522861)(454.92255653,561.8165622)
\curveto(453.97677956,561.54278466)(453.03100259,561.19434051)(452.08522561,560.77122976)
\lineto(450.25589384,564.54189321)
\curveto(451.32611515,565.0894483)(452.53322523,565.52500349)(453.87722409,565.84855877)
\curveto(455.22122294,566.19700292)(456.61499953,566.37122499)(458.05855386,566.37122499)
\closepath
\moveto(461.00788468,555.02190132)
\lineto(458.84255319,554.94723472)
\curveto(457.05055472,554.89745698)(455.80611133,554.5739017)(455.10922304,553.97656888)
\curveto(454.41233474,553.37923605)(454.0638906,552.59523672)(454.0638906,551.62457088)
\curveto(454.0638906,550.77834938)(454.31277927,550.16857212)(454.81055663,549.7952391)
\curveto(455.30833398,549.44679496)(455.95544454,549.27257288)(456.75188831,549.27257288)
\curveto(457.94655396,549.27257288)(458.9545531,549.62101703)(459.77588573,550.31790533)
\curveto(460.59721837,551.03968249)(461.00788468,552.04768163)(461.00788468,553.34190275)
\closepath
}
}
{
\newrgbcolor{curcolor}{0 0 0}
\pscustom[linestyle=none,fillstyle=solid,fillcolor=curcolor]
{
\newpath
\moveto(485.61053267,565.96055867)
\lineto(491.73319413,565.96055867)
\lineto(483.66920099,556.17923367)
\lineto(492.44252686,545.57657603)
\lineto(486.1331989,545.57657603)
\lineto(477.80787265,555.91790056)
\lineto(477.80787265,545.57657603)
\lineto(472.24521072,545.57657603)
\lineto(472.24521072,565.96055867)
\lineto(477.80787265,565.96055867)
\lineto(477.80787265,556.06723376)
\closepath
}
}
{
\newrgbcolor{curcolor}{0 0 0}
\pscustom[linestyle=none,fillstyle=solid,fillcolor=curcolor]
{
\newpath
\moveto(512.22917698,561.7792289)
\lineto(505.54651601,561.7792289)
\lineto(505.54651601,545.57657603)
\lineto(499.98385408,545.57657603)
\lineto(499.98385408,561.7792289)
\lineto(493.3011931,561.7792289)
\lineto(493.3011931,565.96055867)
\lineto(512.22917698,565.96055867)
\closepath
}
}
{
\newrgbcolor{curcolor}{0 0 0}
\pscustom[linestyle=none,fillstyle=solid,fillcolor=curcolor]
{
\newpath
\moveto(521.3758143,565.96055867)
\lineto(521.3758143,557.89656554)
\curveto(521.3758143,557.47345479)(521.35092543,556.95078857)(521.3011477,556.32856687)
\curveto(521.27625883,555.70634518)(521.23892553,555.07167906)(521.18914779,554.4245685)
\curveto(521.16425892,553.77745794)(521.12692562,553.19256954)(521.07714789,552.66990332)
\curveto(521.02737015,552.17212597)(520.99003685,551.83612626)(520.96514798,551.66190418)
\lineto(530.37313997,565.96055867)
\lineto(537.05580095,565.96055867)
\lineto(537.05580095,545.57657603)
\lineto(531.67980552,545.57657603)
\lineto(531.67980552,553.71523577)
\curveto(531.67980552,554.36234633)(531.70469439,555.09656792)(531.75447213,555.91790056)
\curveto(531.80424986,556.73923319)(531.8540276,557.49834366)(531.90380533,558.19523195)
\curveto(531.97847194,558.91700911)(532.02824967,559.4645642)(532.05313854,559.83789722)
\lineto(522.68247985,545.57657603)
\lineto(515.99981888,545.57657603)
\lineto(515.99981888,565.96055867)
\closepath
}
}
{
\newrgbcolor{curcolor}{0 0 0}
\pscustom[linestyle=none,fillstyle=solid,fillcolor=curcolor]
{
\newpath
\moveto(554.22908639,566.33389169)
\curveto(556.51886222,566.33389169)(558.37308286,565.43789245)(559.79174832,563.64589398)
\curveto(561.21041378,561.87878437)(561.91974651,559.26545326)(561.91974651,555.80590065)
\curveto(561.91974651,552.32145918)(561.18552491,549.6832392)(559.71708172,547.89124073)
\curveto(558.24863853,546.09924225)(556.36952901,545.20324301)(554.07975319,545.20324301)
\curveto(552.61130999,545.20324301)(551.44153321,545.46457613)(550.57042284,545.98724235)
\curveto(549.69931247,546.53479744)(548.98997974,547.14457469)(548.44242465,547.81657412)
\lineto(548.14375824,547.81657412)
\curveto(548.34286918,546.77124168)(548.44242465,545.77568697)(548.44242465,544.82991)
\lineto(548.44242465,536.61658366)
\lineto(542.87976272,536.61658366)
\lineto(542.87976272,565.96055867)
\lineto(547.39709221,565.96055867)
\lineto(548.18109154,563.30989426)
\lineto(548.44242465,563.30989426)
\curveto(548.98997974,564.1312269)(549.72420134,564.84055963)(550.64508944,565.43789245)
\curveto(551.56597755,566.03522528)(552.7606432,566.33389169)(554.22908639,566.33389169)
\closepath
\moveto(552.43708792,561.8912288)
\curveto(550.99353359,561.8912288)(549.97309002,561.43078475)(549.37575719,560.50989665)
\curveto(548.77842437,559.61389741)(548.46731352,558.25745412)(548.44242465,556.44056678)
\lineto(548.44242465,555.84323395)
\curveto(548.44242465,553.87701341)(548.72864663,552.35879248)(549.30109059,551.28857117)
\curveto(549.89842341,550.24323872)(550.96864472,549.7205725)(552.51175452,549.7205725)
\curveto(553.78108677,549.7205725)(554.71441931,550.24323872)(555.31175214,551.28857117)
\curveto(555.93397383,552.35879248)(556.24508468,553.88945784)(556.24508468,555.88056726)
\curveto(556.24508468,559.88767496)(554.97575242,561.8912288)(552.43708792,561.8912288)
\closepath
}
}
{
\newrgbcolor{curcolor}{0 0 0}
\pscustom[linestyle=none,fillstyle=solid,fillcolor=curcolor]
{
\newpath
\moveto(585.0290383,555.80590065)
\curveto(585.0290383,552.42101465)(584.13303906,549.80768354)(582.34104059,547.96590733)
\curveto(580.57393098,546.12413112)(578.15971081,545.20324301)(575.09838009,545.20324301)
\curveto(573.20682614,545.20324301)(571.51438314,545.61390933)(570.02105108,546.43524197)
\curveto(568.55260788,547.2565746)(567.39527553,548.45124025)(566.54905403,550.01923891)
\curveto(565.70283253,551.61212645)(565.27972178,553.54101369)(565.27972178,555.80590065)
\curveto(565.27972178,559.19078666)(566.16327658,561.79167333)(567.93038619,563.60856068)
\curveto(569.6974958,565.42544802)(572.1241604,566.33389169)(575.21037999,566.33389169)
\curveto(577.1268228,566.33389169)(578.81926581,565.92322537)(580.287709,565.10189274)
\curveto(581.7561522,564.2805601)(582.91348454,563.08589445)(583.75970604,561.51789579)
\curveto(584.60592755,559.94989712)(585.0290383,558.04589875)(585.0290383,555.80590065)
\closepath
\moveto(570.95438362,555.80590065)
\curveto(570.95438362,553.78990237)(571.2779389,552.25923701)(571.92504946,551.21390456)
\curveto(572.59704888,550.19346099)(573.67971463,549.6832392)(575.17304669,549.6832392)
\curveto(576.64148988,549.6832392)(577.69926676,550.19346099)(578.34637732,551.21390456)
\curveto(579.01837675,552.25923701)(579.35437646,553.78990237)(579.35437646,555.80590065)
\curveto(579.35437646,557.82189894)(579.01837675,559.32767543)(578.34637732,560.32323014)
\curveto(577.69926676,561.34367372)(576.62904545,561.8538955)(575.13571339,561.8538955)
\curveto(573.66727019,561.8538955)(572.59704888,561.34367372)(571.92504946,560.32323014)
\curveto(571.2779389,559.32767543)(570.95438362,557.82189894)(570.95438362,555.80590065)
\closepath
}
}
{
\newrgbcolor{curcolor}{0 0 0}
\pscustom[linestyle=none,fillstyle=solid,fillcolor=curcolor]
{
\newpath
\moveto(607.69035264,560.62189655)
\curveto(607.69035264,559.52678637)(607.3419085,558.59345383)(606.6450202,557.82189894)
\curveto(605.97302077,557.05034404)(604.96502163,556.55256668)(603.62102278,556.32856687)
\lineto(603.62102278,556.17923367)
\curveto(605.03968823,556.00501159)(606.17213171,555.50723424)(607.01835322,554.68590161)
\curveto(607.88946359,553.88945784)(608.32501877,552.8814587)(608.32501877,551.66190418)
\curveto(608.32501877,550.4921274)(608.01390792,549.44679496)(607.39168623,548.52590685)
\curveto(606.79435341,547.60501875)(605.836132,546.88324158)(604.51702201,546.36057536)
\curveto(603.19791202,545.83790914)(601.46813572,545.57657603)(599.3276931,545.57657603)
\lineto(589.6210347,545.57657603)
\lineto(589.6210347,565.96055867)
\lineto(599.3276931,565.96055867)
\curveto(600.92058063,565.96055867)(602.33924609,565.7863366)(603.58368947,565.43789245)
\curveto(604.85302173,565.11433717)(605.84857643,564.55433765)(606.5703536,563.75789388)
\curveto(607.31701963,562.98633898)(607.69035264,561.94100654)(607.69035264,560.62189655)
\closepath
\moveto(602.05302411,560.17389693)
\curveto(602.05302411,561.41834032)(601.06991384,562.04056201)(599.10369329,562.04056201)
\lineto(595.18369663,562.04056201)
\lineto(595.18369663,558.00856544)
\lineto(598.46902716,558.00856544)
\curveto(599.63880394,558.00856544)(600.52235875,558.17034308)(601.11969157,558.49389836)
\curveto(601.74191326,558.84234251)(602.05302411,559.40234204)(602.05302411,560.17389693)
\closepath
\moveto(602.57569033,551.96057059)
\curveto(602.57569033,552.75701436)(602.25213505,553.32945832)(601.60502449,553.67790246)
\curveto(600.9828028,554.05123548)(600.06191469,554.23790199)(598.84236018,554.23790199)
\lineto(595.18369663,554.23790199)
\lineto(595.18369663,549.42190609)
\lineto(598.95436008,549.42190609)
\curveto(599.99969253,549.42190609)(600.85835846,549.6085726)(601.53035789,549.98190561)
\curveto(602.22724618,550.3801275)(602.57569033,551.03968249)(602.57569033,551.96057059)
\closepath
}
}
{
\newrgbcolor{curcolor}{0 0 0}
\pscustom[linestyle=none,fillstyle=solid,fillcolor=curcolor]
{
\newpath
\moveto(621.46629922,566.37122499)
\curveto(624.20407467,566.37122499)(626.29473956,565.77389217)(627.73829388,564.57922652)
\curveto(629.20673708,563.40944973)(629.94095867,561.60500683)(629.94095867,559.16589779)
\lineto(629.94095867,545.57657603)
\lineto(626.05829531,545.57657603)
\lineto(624.97562957,548.33924034)
\lineto(624.82629636,548.33924034)
\curveto(623.95518599,547.24413017)(623.03429789,546.4476864)(622.06363205,545.94990905)
\curveto(621.09296621,545.45213169)(619.76141179,545.20324301)(618.06896878,545.20324301)
\curveto(616.25208144,545.20324301)(614.74630494,545.72590924)(613.5516393,546.77124168)
\curveto(612.35697365,547.81657412)(611.75964082,549.44679496)(611.75964082,551.66190418)
\curveto(611.75964082,553.82723567)(612.51875129,555.4201232)(614.03697222,556.44056678)
\curveto(615.55519314,557.46101035)(617.83252454,558.03345431)(620.8689664,558.15789865)
\lineto(624.41563004,558.26989855)
\lineto(624.41563004,559.16589779)
\curveto(624.41563004,560.2361191)(624.12940807,561.02011844)(623.55696411,561.51789579)
\curveto(623.00940902,562.01567314)(622.23785412,562.26456182)(621.24229941,562.26456182)
\curveto(620.24674471,562.26456182)(619.27607887,562.11522861)(618.33030189,561.8165622)
\curveto(617.38452492,561.54278466)(616.43874795,561.19434051)(615.49297098,560.77122976)
\lineto(613.6636392,564.54189321)
\curveto(614.73386051,565.0894483)(615.94097059,565.52500349)(617.28496945,565.84855877)
\curveto(618.62896831,566.19700292)(620.0227449,566.37122499)(621.46629922,566.37122499)
\closepath
\moveto(624.41563004,555.02190132)
\lineto(622.25029856,554.94723472)
\curveto(620.45830008,554.89745698)(619.2138567,554.5739017)(618.5169684,553.97656888)
\curveto(617.82008011,553.37923605)(617.47163596,552.59523672)(617.47163596,551.62457088)
\curveto(617.47163596,550.77834938)(617.72052463,550.16857212)(618.21830199,549.7952391)
\curveto(618.71607934,549.44679496)(619.3631899,549.27257288)(620.15963367,549.27257288)
\curveto(621.35429932,549.27257288)(622.36229846,549.62101703)(623.18363109,550.31790533)
\curveto(624.00496373,551.03968249)(624.41563004,552.04768163)(624.41563004,553.34190275)
\closepath
}
}
{
\newrgbcolor{curcolor}{0 0 0}
\pscustom[linestyle=none,fillstyle=solid,fillcolor=curcolor]
{
\newpath
\moveto(641.21561801,565.96055867)
\lineto(641.21561801,558.12056535)
\lineto(648.98094473,558.12056535)
\lineto(648.98094473,565.96055867)
\lineto(654.54360666,565.96055867)
\lineto(654.54360666,545.57657603)
\lineto(648.98094473,545.57657603)
\lineto(648.98094473,553.97656888)
\lineto(641.21561801,553.97656888)
\lineto(641.21561801,545.57657603)
\lineto(635.65295608,545.57657603)
\lineto(635.65295608,565.96055867)
\closepath
}
}
{
\newrgbcolor{curcolor}{0 0 0}
\pscustom[linestyle=none,fillstyle=solid,fillcolor=curcolor]
{
\newpath
\moveto(665.74361215,565.96055867)
\lineto(665.74361215,557.89656554)
\curveto(665.74361215,557.47345479)(665.71872328,556.95078857)(665.66894555,556.32856687)
\curveto(665.64405668,555.70634518)(665.60672338,555.07167906)(665.55694564,554.4245685)
\curveto(665.53205677,553.77745794)(665.49472347,553.19256954)(665.44494574,552.66990332)
\curveto(665.395168,552.17212597)(665.3578347,551.83612626)(665.33294583,551.66190418)
\lineto(674.74093782,565.96055867)
\lineto(681.4235988,565.96055867)
\lineto(681.4235988,545.57657603)
\lineto(676.04760338,545.57657603)
\lineto(676.04760338,553.71523577)
\curveto(676.04760338,554.36234633)(676.07249224,555.09656792)(676.12226998,555.91790056)
\curveto(676.17204771,556.73923319)(676.22182545,557.49834366)(676.27160319,558.19523195)
\curveto(676.34626979,558.91700911)(676.39604752,559.4645642)(676.42093639,559.83789722)
\lineto(667.0502777,545.57657603)
\lineto(660.36761673,545.57657603)
\lineto(660.36761673,565.96055867)
\closepath
}
}
{
\newrgbcolor{curcolor}{0 0 0}
\pscustom[linestyle=none,fillstyle=solid,fillcolor=curcolor]
{
\newpath
\moveto(695.64755342,566.33389169)
\curveto(698.45999547,566.33389169)(700.68754913,565.52500349)(702.3302144,563.90722709)
\curveto(703.97287967,562.31433956)(704.7942123,560.03700816)(704.7942123,557.07523291)
\lineto(704.7942123,554.38723519)
\lineto(691.65289016,554.38723519)
\curveto(691.70266789,552.81923653)(692.16311195,551.58723758)(693.03422231,550.69123834)
\curveto(693.93022155,549.7952391)(695.1622205,549.34723949)(696.73021917,549.34723949)
\curveto(698.02444029,549.34723949)(699.2066615,549.47168382)(700.27688281,549.7205725)
\curveto(701.37199299,549.99435005)(702.49199204,550.40501636)(703.63687995,550.95257145)
\lineto(703.63687995,546.65924177)
\curveto(702.61643638,546.16146442)(701.5586595,545.80057584)(700.46354932,545.57657603)
\curveto(699.36843914,545.32768735)(698.03688472,545.20324301)(696.46888606,545.20324301)
\curveto(694.42799891,545.20324301)(692.623556,545.57657603)(691.05555733,546.32324206)
\curveto(689.48755867,547.09479696)(688.25555972,548.23968487)(687.35956048,549.7579058)
\curveto(686.46356124,551.3010156)(686.01556162,553.25479171)(686.01556162,555.61923415)
\curveto(686.01556162,557.98367658)(686.41378351,559.96234156)(687.21022727,561.55522909)
\curveto(688.03155991,563.14811662)(689.16400339,564.34278227)(690.60755771,565.13922604)
\curveto(692.05111204,565.93566981)(693.73111061,566.33389169)(695.64755342,566.33389169)
\closepath
\moveto(695.68488672,562.37656172)
\curveto(694.58977655,562.37656172)(693.69377731,562.02811758)(692.99688901,561.33122928)
\curveto(692.30000072,560.63434099)(691.8893344,559.55167524)(691.76489006,558.08323205)
\lineto(699.56755008,558.08323205)
\curveto(699.54266122,559.30278656)(699.2066615,560.32323014)(698.55955094,561.14456277)
\curveto(697.93732925,561.96589541)(696.97910784,562.37656172)(695.68488672,562.37656172)
\closepath
}
}
{
\newrgbcolor{curcolor}{0 0 0}
\pscustom[linestyle=none,fillstyle=solid,fillcolor=curcolor]
{
\newpath
\moveto(737.34884152,565.96055867)
\lineto(737.34884152,549.6459059)
\lineto(740.33550564,549.6459059)
\lineto(740.33550564,538.25924893)
\lineto(735.33284323,538.25924893)
\lineto(735.33284323,545.57657603)
\lineto(721.63152157,545.57657603)
\lineto(721.63152157,538.25924893)
\lineto(716.62885916,538.25924893)
\lineto(716.62885916,549.6459059)
\lineto(718.34619103,549.6459059)
\curveto(719.24219027,551.01479362)(720.00130073,552.57034785)(720.62352242,554.31256859)
\curveto(721.24574412,556.0796782)(721.74352147,557.94634327)(722.11685449,559.91256382)
\curveto(722.4901875,561.90367324)(722.76396505,563.91967152)(722.93818712,565.96055867)
\closepath
\moveto(731.78617959,561.7792289)
\lineto(727.60484981,561.7792289)
\curveto(727.3061834,559.51434194)(726.89551708,557.36145488)(726.37285086,555.32056773)
\curveto(725.85018464,553.30456945)(725.11596304,551.4130155)(724.17018607,549.6459059)
\lineto(731.78617959,549.6459059)
\closepath
}
}
{
\newrgbcolor{curcolor}{0 0 0}
\pscustom[linestyle=none,fillstyle=solid,fillcolor=curcolor]
{
\newpath
\moveto(762.32481466,555.80590065)
\curveto(762.32481466,552.42101465)(761.42881543,549.80768354)(759.63681695,547.96590733)
\curveto(757.86970735,546.12413112)(755.45548718,545.20324301)(752.39415645,545.20324301)
\curveto(750.50260251,545.20324301)(748.81015951,545.61390933)(747.31682744,546.43524197)
\curveto(745.84838425,547.2565746)(744.6910519,548.45124025)(743.8448304,550.01923891)
\curveto(742.9986089,551.61212645)(742.57549815,553.54101369)(742.57549815,555.80590065)
\curveto(742.57549815,559.19078666)(743.45905295,561.79167333)(745.22616256,563.60856068)
\curveto(746.99327216,565.42544802)(749.41993676,566.33389169)(752.50615636,566.33389169)
\curveto(754.42259917,566.33389169)(756.11504217,565.92322537)(757.58348537,565.10189274)
\curveto(759.05192856,564.2805601)(760.20926091,563.08589445)(761.05548241,561.51789579)
\curveto(761.90170391,559.94989712)(762.32481466,558.04589875)(762.32481466,555.80590065)
\closepath
\moveto(748.25015998,555.80590065)
\curveto(748.25015998,553.78990237)(748.57371526,552.25923701)(749.22082582,551.21390456)
\curveto(749.89282525,550.19346099)(750.975491,549.6832392)(752.46882306,549.6832392)
\curveto(753.93726625,549.6832392)(754.99504313,550.19346099)(755.64215369,551.21390456)
\curveto(756.31415312,552.25923701)(756.65015283,553.78990237)(756.65015283,555.80590065)
\curveto(756.65015283,557.82189894)(756.31415312,559.32767543)(755.64215369,560.32323014)
\curveto(754.99504313,561.34367372)(753.92482182,561.8538955)(752.43148976,561.8538955)
\curveto(750.96304656,561.8538955)(749.89282525,561.34367372)(749.22082582,560.32323014)
\curveto(748.57371526,559.32767543)(748.25015998,557.82189894)(748.25015998,555.80590065)
\closepath
}
}
{
\newrgbcolor{curcolor}{0 0 0}
\pscustom[linestyle=none,fillstyle=solid,fillcolor=curcolor]
{
\newpath
\moveto(775.20480401,545.20324301)
\curveto(772.16836215,545.20324301)(769.81636415,546.03702008)(768.14881001,547.70457422)
\curveto(766.50614475,549.37212835)(765.68481211,552.02279276)(765.68481211,555.65656745)
\curveto(765.68481211,558.14545422)(766.10792286,560.17389693)(766.95414436,561.7418956)
\curveto(767.80036587,563.30989426)(768.97014265,564.46722661)(770.46347471,565.21389264)
\curveto(771.98169564,565.96055867)(773.72391638,566.33389169)(775.69013693,566.33389169)
\curveto(777.08391352,566.33389169)(778.2910236,566.19700292)(779.31146718,565.92322537)
\curveto(780.35679962,565.64944783)(781.26524329,565.32589255)(782.03679819,564.95255953)
\lineto(780.39413292,560.65922985)
\curveto(779.52302255,561.007674)(778.70168992,561.29389598)(777.93013502,561.51789579)
\curveto(777.18346899,561.7418956)(776.43680296,561.8538955)(775.69013693,561.8538955)
\curveto(772.80302827,561.8538955)(771.35947395,559.80056392)(771.35947395,555.69390075)
\curveto(771.35947395,553.6530136)(771.73280696,552.1472371)(772.47947299,551.17657126)
\curveto(773.25102789,550.20590542)(774.3212492,549.7205725)(775.69013693,549.7205725)
\curveto(776.85991371,549.7205725)(777.89280172,549.86990571)(778.78880095,550.16857212)
\curveto(779.68480019,550.4921274)(780.55591056,550.92768258)(781.40213206,551.47523767)
\lineto(781.40213206,546.73390838)
\curveto(780.55591056,546.18635329)(779.65991132,545.80057584)(778.71413435,545.57657603)
\curveto(777.79324625,545.32768735)(776.62346946,545.20324301)(775.20480401,545.20324301)
\closepath
}
}
{
\newrgbcolor{curcolor}{0 0 0}
\pscustom[linestyle=none,fillstyle=solid,fillcolor=curcolor]
{
\newpath
\moveto(799.47146407,565.96055867)
\lineto(805.59412552,565.96055867)
\lineto(797.53013239,556.17923367)
\lineto(806.30345825,545.57657603)
\lineto(799.99413029,545.57657603)
\lineto(791.66880405,555.91790056)
\lineto(791.66880405,545.57657603)
\lineto(786.10614212,545.57657603)
\lineto(786.10614212,565.96055867)
\lineto(791.66880405,565.96055867)
\lineto(791.66880405,556.06723376)
\closepath
}
}
{
\newrgbcolor{curcolor}{0 0 0}
\pscustom[linestyle=none,fillstyle=solid,fillcolor=curcolor]
{
\newpath
\moveto(814.59139047,565.96055867)
\lineto(814.59139047,557.89656554)
\curveto(814.59139047,557.47345479)(814.5665016,556.95078857)(814.51672387,556.32856687)
\curveto(814.491835,555.70634518)(814.4545017,555.07167906)(814.40472396,554.4245685)
\curveto(814.37983509,553.77745794)(814.34250179,553.19256954)(814.29272406,552.66990332)
\curveto(814.24294632,552.17212597)(814.20561302,551.83612626)(814.18072415,551.66190418)
\lineto(823.58871614,565.96055867)
\lineto(830.27137712,565.96055867)
\lineto(830.27137712,545.57657603)
\lineto(824.8953817,545.57657603)
\lineto(824.8953817,553.71523577)
\curveto(824.8953817,554.36234633)(824.92027056,555.09656792)(824.9700483,555.91790056)
\curveto(825.01982603,556.73923319)(825.06960377,557.49834366)(825.11938151,558.19523195)
\curveto(825.19404811,558.91700911)(825.24382584,559.4645642)(825.26871471,559.83789722)
\lineto(815.89805602,545.57657603)
\lineto(809.21539505,545.57657603)
\lineto(809.21539505,565.96055867)
\closepath
}
}
{
\newrgbcolor{curcolor}{0 0 0}
\pscustom[linestyle=none,fillstyle=solid,fillcolor=curcolor]
{
\newpath
\moveto(842.89006745,565.96055867)
\lineto(848.9753956,565.96055867)
\lineto(852.82072566,554.4992351)
\curveto(853.0198366,553.92679114)(853.16916981,553.35434718)(853.26872528,552.78190323)
\curveto(853.36828075,552.20945927)(853.44294735,551.59968201)(853.49272509,550.95257145)
\lineto(853.60472499,550.95257145)
\curveto(853.6793916,551.59968201)(853.77894707,552.20945927)(853.9033914,552.78190323)
\curveto(854.02783574,553.35434718)(854.18961338,553.92679114)(854.38872432,554.4992351)
\lineto(858.15938778,565.96055867)
\lineto(864.13271603,565.96055867)
\lineto(855.50872337,542.96324492)
\curveto(854.7122796,540.84769117)(853.57983612,539.26724807)(852.11139293,538.22191563)
\curveto(850.64294974,537.15169431)(848.9380623,536.61658366)(846.99673062,536.61658366)
\curveto(846.34962006,536.61658366)(845.80206497,536.65391696)(845.35406535,536.72858356)
\curveto(844.90606573,536.7783613)(844.50784385,536.84058347)(844.1593997,536.91525007)
\lineto(844.1593997,541.32057965)
\curveto(844.40828838,541.27080192)(844.73184366,541.22102418)(845.13006554,541.17124645)
\curveto(845.52828743,541.12146871)(845.93895374,541.09657984)(846.36206449,541.09657984)
\curveto(847.53184127,541.09657984)(848.45272938,541.45746843)(849.12472881,542.17924559)
\curveto(849.79672823,542.87613388)(850.30695002,543.72235539)(850.65539417,544.71791009)
\lineto(850.99139388,545.72590924)
\closepath
}
}
{
\newrgbcolor{curcolor}{0 0 0}
\pscustom[linestyle=none,fillstyle=solid,fillcolor=curcolor]
{
\newpath
\moveto(89.87754747,513.95527058)
\curveto(89.87754747,512.8601604)(89.52910332,511.92682786)(88.83221502,511.15527296)
\curveto(88.1602156,510.38371806)(87.15221646,509.88594071)(85.8082176,509.6619409)
\lineto(85.8082176,509.51260769)
\curveto(87.22688306,509.33838562)(88.35932654,508.84060826)(89.20554804,508.01927563)
\curveto(90.07665841,507.22283186)(90.51221359,506.21483272)(90.51221359,504.9952782)
\curveto(90.51221359,503.82550142)(90.20110275,502.78016898)(89.57888106,501.85928088)
\curveto(88.98154823,500.93839277)(88.02332682,500.21661561)(86.70421684,499.69394939)
\curveto(85.38510685,499.17128316)(83.65533054,498.90995005)(81.51488792,498.90995005)
\lineto(71.80822952,498.90995005)
\lineto(71.80822952,519.2939327)
\lineto(81.51488792,519.2939327)
\curveto(83.10777545,519.2939327)(84.52644091,519.11971062)(85.7708843,518.77126647)
\curveto(87.04021655,518.44771119)(88.03577126,517.88771167)(88.75754842,517.09126791)
\curveto(89.50421445,516.31971301)(89.87754747,515.27438056)(89.87754747,513.95527058)
\closepath
\moveto(84.24021893,513.50727096)
\curveto(84.24021893,514.75171434)(83.25710866,515.37393603)(81.29088811,515.37393603)
\lineto(77.37089145,515.37393603)
\lineto(77.37089145,511.34193947)
\lineto(80.65622199,511.34193947)
\curveto(81.82599877,511.34193947)(82.70955357,511.50371711)(83.3068864,511.82727239)
\curveto(83.92910809,512.17571654)(84.24021893,512.73571606)(84.24021893,513.50727096)
\closepath
\moveto(84.76288516,505.29394462)
\curveto(84.76288516,506.09038838)(84.43932988,506.66283234)(83.79221932,507.01127649)
\curveto(83.16999762,507.3846095)(82.24910952,507.57127601)(81.029555,507.57127601)
\lineto(77.37089145,507.57127601)
\lineto(77.37089145,502.75528011)
\lineto(81.14155491,502.75528011)
\curveto(82.18688735,502.75528011)(83.04555329,502.94194662)(83.71755271,503.31527964)
\curveto(84.41444101,503.71350152)(84.76288516,504.37305651)(84.76288516,505.29394462)
\closepath
}
}
{
\newrgbcolor{curcolor}{0 0 0}
\pscustom[linestyle=none,fillstyle=solid,fillcolor=curcolor]
{
\newpath
\moveto(113.54684947,498.90995005)
\lineto(107.98418754,498.90995005)
\lineto(107.98418754,515.11260292)
\lineto(102.86952523,515.11260292)
\curveto(102.54596995,511.13038409)(102.11041477,507.91972016)(101.56285968,505.48061113)
\curveto(101.04019346,503.06639096)(100.29352743,501.29928135)(99.32286159,500.17928231)
\curveto(98.37708461,499.08417213)(97.12019679,498.53661704)(95.55219813,498.53661704)
\curveto(94.25797701,498.53661704)(93.20020013,498.73572798)(92.3788675,499.13394986)
\lineto(92.3788675,503.57661275)
\curveto(92.95131146,503.32772407)(93.54864428,503.20327973)(94.17086597,503.20327973)
\curveto(94.61886559,503.20327973)(95.02953191,503.42727954)(95.40286492,503.87527916)
\curveto(95.77619794,504.32327878)(96.12464209,505.13216698)(96.44819737,506.30194376)
\curveto(96.79664151,507.47172054)(97.10775236,509.10194137)(97.3815299,511.19260626)
\curveto(97.65530745,513.30816002)(97.90419613,516.00860216)(98.12819594,519.2939327)
\lineto(113.54684947,519.2939327)
\closepath
}
}
{
\newrgbcolor{curcolor}{0 0 0}
\pscustom[linestyle=none,fillstyle=solid,fillcolor=curcolor]
{
\newpath
\moveto(127.73351877,519.70459901)
\curveto(130.47129421,519.70459901)(132.5619591,519.10726619)(134.00551343,517.91260054)
\curveto(135.47395662,516.74282376)(136.20817822,514.93838085)(136.20817822,512.49927182)
\lineto(136.20817822,498.90995005)
\lineto(132.32551486,498.90995005)
\lineto(131.24284911,501.67261437)
\lineto(131.09351591,501.67261437)
\curveto(130.22240554,500.57750419)(129.30151743,499.78106042)(128.33085159,499.28328307)
\curveto(127.36018575,498.78550571)(126.02863133,498.53661704)(124.33618833,498.53661704)
\curveto(122.51930098,498.53661704)(121.01352449,499.05928326)(119.81885884,500.1046157)
\curveto(118.62419319,501.14994815)(118.02686036,502.78016898)(118.02686036,504.9952782)
\curveto(118.02686036,507.16060969)(118.78597083,508.75349723)(120.30419176,509.7739408)
\curveto(121.82241269,510.79438438)(124.09974408,511.36682834)(127.13618594,511.49127267)
\lineto(130.68284959,511.60327258)
\lineto(130.68284959,512.49927182)
\curveto(130.68284959,513.56949313)(130.39662761,514.35349246)(129.82418365,514.85126981)
\curveto(129.27662856,515.34904717)(128.50507366,515.59793584)(127.50951896,515.59793584)
\curveto(126.51396425,515.59793584)(125.54329841,515.44860264)(124.59752144,515.14993622)
\curveto(123.65174446,514.87615868)(122.70596749,514.52771453)(121.76019052,514.10460378)
\lineto(119.93085874,517.87526724)
\curveto(121.00108005,518.42282233)(122.20819014,518.85837751)(123.55218899,519.18193279)
\curveto(124.89618785,519.53037694)(126.28996444,519.70459901)(127.73351877,519.70459901)
\closepath
\moveto(130.68284959,508.35527534)
\lineto(128.5175181,508.28060874)
\curveto(126.72551962,508.23083101)(125.48107624,507.90727573)(124.78418794,507.3099429)
\curveto(124.08729965,506.71261008)(123.7388555,505.92861074)(123.7388555,504.9579449)
\curveto(123.7388555,504.1117234)(123.98774418,503.50194614)(124.48552153,503.12861313)
\curveto(124.98329889,502.78016898)(125.63040945,502.60594691)(126.42685321,502.60594691)
\curveto(127.62151886,502.60594691)(128.629518,502.95439105)(129.45085064,503.65127935)
\curveto(130.27218327,504.37305651)(130.68284959,505.38105565)(130.68284959,506.67527677)
\closepath
}
}
{
\newrgbcolor{curcolor}{0 0 0}
\pscustom[linestyle=none,fillstyle=solid,fillcolor=curcolor]
{
\newpath
\moveto(160.25083431,519.2939327)
\lineto(160.25083431,502.97927992)
\lineto(163.23749844,502.97927992)
\lineto(163.23749844,491.59262295)
\lineto(158.23483603,491.59262295)
\lineto(158.23483603,498.90995005)
\lineto(144.53351436,498.90995005)
\lineto(144.53351436,491.59262295)
\lineto(139.53085196,491.59262295)
\lineto(139.53085196,502.97927992)
\lineto(141.24818383,502.97927992)
\curveto(142.14418307,504.34816764)(142.90329353,505.90372188)(143.52551522,507.64594261)
\curveto(144.14773691,509.41305222)(144.64551427,511.2797173)(145.01884728,513.24593785)
\curveto(145.3921803,515.23704726)(145.66595784,517.25304555)(145.84017992,519.2939327)
\closepath
\moveto(154.68817238,515.11260292)
\lineto(150.50684261,515.11260292)
\curveto(150.2081762,512.84771596)(149.79750988,510.69482891)(149.27484366,508.65394176)
\curveto(148.75217744,506.63794347)(148.01795584,504.74638953)(147.07217887,502.97927992)
\lineto(154.68817238,502.97927992)
\closepath
}
}
{
\newrgbcolor{curcolor}{0 0 0}
\pscustom[linestyle=none,fillstyle=solid,fillcolor=curcolor]
{
\newpath
\moveto(175.10946749,519.66726571)
\curveto(177.92190953,519.66726571)(180.14946319,518.85837751)(181.79212846,517.24060111)
\curveto(183.43479373,515.64771358)(184.25612636,513.37038218)(184.25612636,510.40860693)
\lineto(184.25612636,507.72060922)
\lineto(171.11480422,507.72060922)
\curveto(171.16458196,506.15261055)(171.62502601,504.9206116)(172.49613638,504.02461236)
\curveto(173.39213561,503.12861313)(174.62413457,502.68061351)(176.19213323,502.68061351)
\curveto(177.48635435,502.68061351)(178.66857557,502.80505785)(179.73879688,503.05394652)
\curveto(180.83390706,503.32772407)(181.9539061,503.73839039)(183.09879402,504.28594548)
\lineto(183.09879402,499.9926158)
\curveto(182.07835044,499.49483844)(181.02057356,499.13394986)(179.92546338,498.90995005)
\curveto(178.83035321,498.66106138)(177.49879878,498.53661704)(175.93080012,498.53661704)
\curveto(173.88991297,498.53661704)(172.08547006,498.90995005)(170.5174714,499.65661608)
\curveto(168.94947273,500.42817098)(167.71747378,501.5730589)(166.82147454,503.09127983)
\curveto(165.92547531,504.63438962)(165.47747569,506.58816574)(165.47747569,508.95260817)
\curveto(165.47747569,511.3170506)(165.87569757,513.29571558)(166.67214134,514.88860311)
\curveto(167.49347397,516.48149065)(168.62591745,517.6761563)(170.06947178,518.47260006)
\curveto(171.5130261,519.26904383)(173.19302467,519.66726571)(175.10946749,519.66726571)
\closepath
\moveto(175.14680079,515.70993575)
\curveto(174.05169061,515.70993575)(173.15569137,515.3614916)(172.45880308,514.6646033)
\curveto(171.76191478,513.96771501)(171.35124846,512.88504926)(171.22680412,511.41660607)
\lineto(179.02946415,511.41660607)
\curveto(179.00457528,512.63616059)(178.66857557,513.65660416)(178.02146501,514.4779368)
\curveto(177.39924331,515.29926943)(176.44102191,515.70993575)(175.14680079,515.70993575)
\closepath
}
}
{
\newrgbcolor{curcolor}{0 0 0}
\pscustom[linestyle=none,fillstyle=solid,fillcolor=curcolor]
{
\newpath
\moveto(207.02942394,498.90995005)
\lineto(201.46676201,498.90995005)
\lineto(201.46676201,515.11260292)
\lineto(196.35209969,515.11260292)
\curveto(196.02854441,511.13038409)(195.59298923,507.91972016)(195.04543414,505.48061113)
\curveto(194.52276792,503.06639096)(193.77610189,501.29928135)(192.80543605,500.17928231)
\curveto(191.85965908,499.08417213)(190.60277126,498.53661704)(189.03477259,498.53661704)
\curveto(187.74055147,498.53661704)(186.68277459,498.73572798)(185.86144196,499.13394986)
\lineto(185.86144196,503.57661275)
\curveto(186.43388592,503.32772407)(187.03121874,503.20327973)(187.65344044,503.20327973)
\curveto(188.10144005,503.20327973)(188.51210637,503.42727954)(188.88543939,503.87527916)
\curveto(189.2587724,504.32327878)(189.60721655,505.13216698)(189.93077183,506.30194376)
\curveto(190.27921598,507.47172054)(190.59032682,509.10194137)(190.86410437,511.19260626)
\curveto(191.13788191,513.30816002)(191.38677059,516.00860216)(191.6107704,519.2939327)
\lineto(207.02942394,519.2939327)
\closepath
}
}
{
\newrgbcolor{curcolor}{0 0 0}
\pscustom[linestyle=none,fillstyle=solid,fillcolor=curcolor]
{
\newpath
\moveto(218.41610324,511.41660607)
\lineto(222.3360999,511.41660607)
\curveto(225.47209723,511.41660607)(227.78676193,510.91882872)(229.28009399,509.92327401)
\curveto(230.79831492,508.9277193)(231.55742539,507.42194281)(231.55742539,505.40594452)
\curveto(231.55742539,503.41483511)(230.86053709,501.83439201)(229.4667605,500.66461523)
\curveto(228.07298391,499.49483844)(225.77076365,498.90995005)(222.56009971,498.90995005)
\lineto(212.85344131,498.90995005)
\lineto(212.85344131,519.2939327)
\lineto(218.41610324,519.2939327)
\closepath
\moveto(225.99476346,505.33127792)
\curveto(225.99476346,506.82460998)(224.73787564,507.57127601)(222.2241,507.57127601)
\lineto(218.41610324,507.57127601)
\lineto(218.41610324,502.75528011)
\lineto(222.2987666,502.75528011)
\curveto(223.36898791,502.75528011)(224.25254272,502.94194662)(224.94943101,503.31527964)
\curveto(225.64631931,503.71350152)(225.99476346,504.38550095)(225.99476346,505.33127792)
\closepath
}
}
{
\newrgbcolor{curcolor}{0 0 0}
\pscustom[linestyle=none,fillstyle=solid,fillcolor=curcolor]
{
\newpath
\moveto(258.02674223,491.59262295)
\lineto(253.02407982,491.59262295)
\lineto(253.02407982,498.90995005)
\lineto(235.77609451,498.90995005)
\lineto(235.77609451,519.2939327)
\lineto(241.33875644,519.2939327)
\lineto(241.33875644,503.09127983)
\lineto(249.47741618,503.09127983)
\lineto(249.47741618,519.2939327)
\lineto(255.04007811,519.2939327)
\lineto(255.04007811,502.97927992)
\lineto(258.02674223,502.97927992)
\closepath
}
}
{
\newrgbcolor{curcolor}{0 0 0}
\pscustom[linestyle=none,fillstyle=solid,fillcolor=curcolor]
{
\newpath
\moveto(269.82405679,519.70459901)
\curveto(272.56183224,519.70459901)(274.65249712,519.10726619)(276.09605145,517.91260054)
\curveto(277.56449464,516.74282376)(278.29871624,514.93838085)(278.29871624,512.49927182)
\lineto(278.29871624,498.90995005)
\lineto(274.41605288,498.90995005)
\lineto(273.33338714,501.67261437)
\lineto(273.18405393,501.67261437)
\curveto(272.31294356,500.57750419)(271.39205546,499.78106042)(270.42138962,499.28328307)
\curveto(269.45072378,498.78550571)(268.11916935,498.53661704)(266.42672635,498.53661704)
\curveto(264.60983901,498.53661704)(263.10406251,499.05928326)(261.90939686,500.1046157)
\curveto(260.71473121,501.14994815)(260.11739839,502.78016898)(260.11739839,504.9952782)
\curveto(260.11739839,507.16060969)(260.87650885,508.75349723)(262.39472978,509.7739408)
\curveto(263.91295071,510.79438438)(266.19028211,511.36682834)(269.22672397,511.49127267)
\lineto(272.77338761,511.60327258)
\lineto(272.77338761,512.49927182)
\curveto(272.77338761,513.56949313)(272.48716563,514.35349246)(271.91472168,514.85126981)
\curveto(271.36716659,515.34904717)(270.59561169,515.59793584)(269.60005698,515.59793584)
\curveto(268.60450227,515.59793584)(267.63383643,515.44860264)(266.68805946,515.14993622)
\curveto(265.74228249,514.87615868)(264.79650552,514.52771453)(263.85072854,514.10460378)
\lineto(262.02139677,517.87526724)
\curveto(263.09161808,518.42282233)(264.29872816,518.85837751)(265.64272702,519.18193279)
\curveto(266.98672587,519.53037694)(268.38050246,519.70459901)(269.82405679,519.70459901)
\closepath
\moveto(272.77338761,508.35527534)
\lineto(270.60805612,508.28060874)
\curveto(268.81605765,508.23083101)(267.57161426,507.90727573)(266.87472597,507.3099429)
\curveto(266.17783767,506.71261008)(265.82939353,505.92861074)(265.82939353,504.9579449)
\curveto(265.82939353,504.1117234)(266.0782822,503.50194614)(266.57605956,503.12861313)
\curveto(267.07383691,502.78016898)(267.72094747,502.60594691)(268.51739124,502.60594691)
\curveto(269.71205689,502.60594691)(270.72005603,502.95439105)(271.54138866,503.65127935)
\curveto(272.3627213,504.37305651)(272.77338761,505.38105565)(272.77338761,506.67527677)
\closepath
}
}
{
\newrgbcolor{curcolor}{0 0 0}
\pscustom[linestyle=none,fillstyle=solid,fillcolor=curcolor]
{
\newpath
\moveto(307.08270064,519.2939327)
\lineto(313.2053621,519.2939327)
\lineto(305.14136896,509.51260769)
\lineto(313.91469483,498.90995005)
\lineto(307.60536686,498.90995005)
\lineto(299.28004062,509.25127458)
\lineto(299.28004062,498.90995005)
\lineto(293.71737869,498.90995005)
\lineto(293.71737869,519.2939327)
\lineto(299.28004062,519.2939327)
\lineto(299.28004062,509.40060779)
\closepath
}
}
{
\newrgbcolor{curcolor}{0 0 0}
\pscustom[linestyle=none,fillstyle=solid,fillcolor=curcolor]
{
\newpath
\moveto(334.59730612,509.13927468)
\curveto(334.59730612,505.75438867)(333.70130688,503.14105756)(331.90930841,501.29928135)
\curveto(330.1421988,499.45750514)(327.72797864,498.53661704)(324.66664791,498.53661704)
\curveto(322.77509396,498.53661704)(321.08265096,498.94728335)(319.5893189,499.76861599)
\curveto(318.1208757,500.58994862)(316.96354336,501.78461427)(316.11732186,503.35261294)
\curveto(315.27110035,504.94550047)(314.8479896,506.87438772)(314.8479896,509.13927468)
\curveto(314.8479896,512.52416068)(315.73154441,515.12504736)(317.49865401,516.9419347)
\curveto(319.26576362,518.75882204)(321.69242822,519.66726571)(324.77864781,519.66726571)
\curveto(326.69509063,519.66726571)(328.38753363,519.25659939)(329.85597682,518.43526676)
\curveto(331.32442002,517.61393413)(332.48175237,516.41926848)(333.32797387,514.85126981)
\curveto(334.17419537,513.28327115)(334.59730612,511.37927277)(334.59730612,509.13927468)
\closepath
\moveto(320.52265144,509.13927468)
\curveto(320.52265144,507.12327639)(320.84620672,505.59261103)(321.49331728,504.54727859)
\curveto(322.16531671,503.52683501)(323.24798245,503.01661322)(324.74131451,503.01661322)
\curveto(326.20975771,503.01661322)(327.26753458,503.52683501)(327.91464514,504.54727859)
\curveto(328.58664457,505.59261103)(328.92264428,507.12327639)(328.92264428,509.13927468)
\curveto(328.92264428,511.15527296)(328.58664457,512.66104946)(327.91464514,513.65660416)
\curveto(327.26753458,514.67704774)(326.19731327,515.18726953)(324.70398121,515.18726953)
\curveto(323.23553802,515.18726953)(322.16531671,514.67704774)(321.49331728,513.65660416)
\curveto(320.84620672,512.66104946)(320.52265144,511.15527296)(320.52265144,509.13927468)
\closepath
}
}
{
\newrgbcolor{curcolor}{0 0 0}
\pscustom[linestyle=none,fillstyle=solid,fillcolor=curcolor]
{
\newpath
\moveto(364.87461398,519.2939327)
\lineto(364.87461398,498.90995005)
\lineto(359.68528507,498.90995005)
\lineto(359.68528507,508.91527487)
\curveto(359.68528507,509.91082957)(359.6977295,510.88149542)(359.72261837,511.82727239)
\curveto(359.7723961,512.77304936)(359.83461827,513.64415973)(359.90928488,514.4406035)
\lineto(359.79728497,514.4406035)
\lineto(354.15995644,498.90995005)
\lineto(349.97862666,498.90995005)
\lineto(344.26663153,514.4779368)
\lineto(344.11729832,514.4779368)
\curveto(344.21685379,513.65660416)(344.27907596,512.77304936)(344.30396483,511.82727239)
\curveto(344.35374257,510.90638428)(344.37863143,509.88594071)(344.37863143,508.76594166)
\lineto(344.37863143,498.90995005)
\lineto(339.18930252,498.90995005)
\lineto(339.18930252,519.2939327)
\lineto(347.06662914,519.2939327)
\lineto(352.14395815,505.48061113)
\lineto(357.29595377,519.2939327)
\closepath
}
}
{
\newrgbcolor{curcolor}{0 0 0}
\pscustom[linestyle=none,fillstyle=solid,fillcolor=curcolor]
{
\newpath
\moveto(376.26124179,519.2939327)
\lineto(376.26124179,511.45393937)
\lineto(384.02656851,511.45393937)
\lineto(384.02656851,519.2939327)
\lineto(389.58923044,519.2939327)
\lineto(389.58923044,498.90995005)
\lineto(384.02656851,498.90995005)
\lineto(384.02656851,507.3099429)
\lineto(376.26124179,507.3099429)
\lineto(376.26124179,498.90995005)
\lineto(370.69857986,498.90995005)
\lineto(370.69857986,519.2939327)
\closepath
}
}
{
\newrgbcolor{curcolor}{0 0 0}
\pscustom[linestyle=none,fillstyle=solid,fillcolor=curcolor]
{
\newpath
\moveto(403.77593057,519.70459901)
\curveto(406.51370602,519.70459901)(408.6043709,519.10726619)(410.04792523,517.91260054)
\curveto(411.51636842,516.74282376)(412.25059002,514.93838085)(412.25059002,512.49927182)
\lineto(412.25059002,498.90995005)
\lineto(408.36792666,498.90995005)
\lineto(407.28526092,501.67261437)
\lineto(407.13592771,501.67261437)
\curveto(406.26481734,500.57750419)(405.34392924,499.78106042)(404.37326339,499.28328307)
\curveto(403.40259755,498.78550571)(402.07104313,498.53661704)(400.37860013,498.53661704)
\curveto(398.56171279,498.53661704)(397.05593629,499.05928326)(395.86127064,500.1046157)
\curveto(394.66660499,501.14994815)(394.06927217,502.78016898)(394.06927217,504.9952782)
\curveto(394.06927217,507.16060969)(394.82838263,508.75349723)(396.34660356,509.7739408)
\curveto(397.86482449,510.79438438)(400.14215589,511.36682834)(403.17859775,511.49127267)
\lineto(406.72526139,511.60327258)
\lineto(406.72526139,512.49927182)
\curveto(406.72526139,513.56949313)(406.43903941,514.35349246)(405.86659546,514.85126981)
\curveto(405.31904037,515.34904717)(404.54748547,515.59793584)(403.55193076,515.59793584)
\curveto(402.55637605,515.59793584)(401.58571021,515.44860264)(400.63993324,515.14993622)
\curveto(399.69415627,514.87615868)(398.7483793,514.52771453)(397.80260232,514.10460378)
\lineto(395.97327055,517.87526724)
\curveto(397.04349186,518.42282233)(398.25060194,518.85837751)(399.5946008,519.18193279)
\curveto(400.93859965,519.53037694)(402.33237624,519.70459901)(403.77593057,519.70459901)
\closepath
\moveto(406.72526139,508.35527534)
\lineto(404.5599299,508.28060874)
\curveto(402.76793143,508.23083101)(401.52348804,507.90727573)(400.82659975,507.3099429)
\curveto(400.12971145,506.71261008)(399.78126731,505.92861074)(399.78126731,504.9579449)
\curveto(399.78126731,504.1117234)(400.03015598,503.50194614)(400.52793334,503.12861313)
\curveto(401.02571069,502.78016898)(401.67282125,502.60594691)(402.46926502,502.60594691)
\curveto(403.66393067,502.60594691)(404.67192981,502.95439105)(405.49326244,503.65127935)
\curveto(406.31459508,504.37305651)(406.72526139,505.38105565)(406.72526139,506.67527677)
\closepath
}
}
{
\newrgbcolor{curcolor}{0 0 0}
\pscustom[linestyle=none,fillstyle=solid,fillcolor=curcolor]
{
\newpath
\moveto(434.83723973,515.11260292)
\lineto(428.15457875,515.11260292)
\lineto(428.15457875,498.90995005)
\lineto(422.59191682,498.90995005)
\lineto(422.59191682,515.11260292)
\lineto(415.90925585,515.11260292)
\lineto(415.90925585,519.2939327)
\lineto(434.83723973,519.2939327)
\closepath
}
}
{
\newrgbcolor{curcolor}{0 0 0}
\pscustom[linestyle=none,fillstyle=solid,fillcolor=curcolor]
{
\newpath
\moveto(438.60788162,498.90995005)
\lineto(438.60788162,519.2939327)
\lineto(444.17054355,519.2939327)
\lineto(444.17054355,511.41660607)
\lineto(446.85854126,511.41660607)
\curveto(449.96964972,511.41660607)(452.27186999,510.91882872)(453.76520205,509.92327401)
\curveto(455.25853411,508.9277193)(456.00520014,507.42194281)(456.00520014,505.40594452)
\curveto(456.00520014,503.41483511)(455.30831184,501.83439201)(453.91453525,500.66461523)
\curveto(452.52075866,499.49483844)(450.23098283,498.90995005)(447.04520777,498.90995005)
\closepath
\moveto(458.95453096,498.90995005)
\lineto(458.95453096,519.2939327)
\lineto(464.51719289,519.2939327)
\lineto(464.51719289,498.90995005)
\closepath
\moveto(444.17054355,502.75528011)
\lineto(446.74654136,502.75528011)
\curveto(447.84165154,502.75528011)(448.72520634,502.94194662)(449.39720577,503.31527964)
\curveto(450.09409406,503.71350152)(450.44253821,504.38550095)(450.44253821,505.33127792)
\curveto(450.44253821,506.82460998)(449.18565039,507.57127601)(446.67187475,507.57127601)
\lineto(444.17054355,507.57127601)
\closepath
}
}
{
\newrgbcolor{curcolor}{0 0 0}
\pscustom[linestyle=none,fillstyle=solid,fillcolor=curcolor]
{
\newpath
\moveto(80.76851029,266.24972483)
\curveto(84.20317403,266.24972483)(86.70450524,265.5030588)(88.2725039,264.00972674)
\curveto(89.86539143,262.54128355)(90.6618352,260.51284083)(90.6618352,257.92439859)
\curveto(90.6618352,256.35639992)(90.33827992,254.90040116)(89.69116936,253.55640231)
\curveto(89.0440588,252.21240345)(87.96139306,251.12973771)(86.44317213,250.30840507)
\curveto(84.94984006,249.48707244)(82.90895291,249.07640612)(80.32051067,249.07640612)
\lineto(77.89384607,249.07640612)
\lineto(77.89384607,239.59374753)
\lineto(72.25651754,239.59374753)
\lineto(72.25651754,266.24972483)
\closepath
\moveto(80.46984388,261.62039544)
\lineto(77.89384607,261.62039544)
\lineto(77.89384607,253.70573551)
\lineto(79.76051115,253.70573551)
\curveto(81.35339868,253.70573551)(82.6102865,254.01684636)(83.53117461,254.63906805)
\curveto(84.47695158,255.28617861)(84.94984006,256.31906662)(84.94984006,257.73773208)
\curveto(84.94984006,260.32617432)(83.456508,261.62039544)(80.46984388,261.62039544)
\closepath
}
}
{
\newrgbcolor{curcolor}{0 0 0}
\pscustom[linestyle=none,fillstyle=solid,fillcolor=curcolor]
{
\newpath
\moveto(103.61650141,260.38839649)
\curveto(106.35427686,260.38839649)(108.44494174,259.79106367)(109.88849607,258.59639802)
\curveto(111.35693926,257.42662123)(112.09116086,255.62217833)(112.09116086,253.18306929)
\lineto(112.09116086,239.59374753)
\lineto(108.2084975,239.59374753)
\lineto(107.12583176,242.35641184)
\lineto(106.97649855,242.35641184)
\curveto(106.10538818,241.26130167)(105.18450008,240.4648579)(104.21383424,239.96708055)
\curveto(103.24316839,239.46930319)(101.91161397,239.22041451)(100.21917097,239.22041451)
\curveto(98.40228363,239.22041451)(96.89650713,239.74308074)(95.70184148,240.78841318)
\curveto(94.50717583,241.83374562)(93.90984301,243.46396646)(93.90984301,245.67907568)
\curveto(93.90984301,247.84440717)(94.66895347,249.4372947)(96.1871744,250.45773828)
\curveto(97.70539533,251.47818185)(99.98272673,252.05062581)(103.01916859,252.17507015)
\lineto(106.56583223,252.28707005)
\lineto(106.56583223,253.18306929)
\curveto(106.56583223,254.2532906)(106.27961025,255.03728994)(105.7071663,255.53506729)
\curveto(105.15961121,256.03284464)(104.38805631,256.28173332)(103.3925016,256.28173332)
\curveto(102.39694689,256.28173332)(101.42628105,256.13240011)(100.48050408,255.8337337)
\curveto(99.53472711,255.55995616)(98.58895014,255.21151201)(97.64317316,254.78840126)
\lineto(95.81384139,258.55906471)
\curveto(96.8840627,259.1066198)(98.09117278,259.54217499)(99.43517164,259.86573027)
\curveto(100.77917049,260.21417442)(102.17294708,260.38839649)(103.61650141,260.38839649)
\closepath
\moveto(106.56583223,249.03907282)
\lineto(104.40050074,248.96440622)
\curveto(102.60850227,248.91462848)(101.36405888,248.5910732)(100.66717059,247.99374038)
\curveto(99.97028229,247.39640755)(99.62183815,246.61240822)(99.62183815,245.64174238)
\curveto(99.62183815,244.79552088)(99.87072682,244.18574362)(100.36850418,243.8124106)
\curveto(100.86628153,243.46396646)(101.51339209,243.28974438)(102.30983586,243.28974438)
\curveto(103.50450151,243.28974438)(104.51250065,243.63818853)(105.33383328,244.33507683)
\curveto(106.15516592,245.05685399)(106.56583223,246.06485313)(106.56583223,247.35907425)
\closepath
}
}
{
\newrgbcolor{curcolor}{0 0 0}
\pscustom[linestyle=none,fillstyle=solid,fillcolor=curcolor]
{
\newpath
\moveto(125.23248909,260.35106319)
\curveto(126.70093229,260.35106319)(128.06982001,260.15195225)(129.33915226,259.75373036)
\curveto(130.63337338,259.38039735)(131.66626139,258.79550896)(132.43781629,257.99906519)
\curveto(133.23426006,257.20262142)(133.63248194,256.18217785)(133.63248194,254.93773446)
\curveto(133.63248194,253.71817995)(133.25914892,252.74751411)(132.51248289,252.02573694)
\curveto(131.76581686,251.30395978)(130.78270659,250.78129356)(129.56315207,250.45773828)
\lineto(129.56315207,250.27107177)
\curveto(130.43426244,250.07196083)(131.21826177,249.78573885)(131.91515007,249.41240584)
\curveto(132.61203836,249.06396169)(133.15959345,248.57862877)(133.55781534,247.95640708)
\curveto(133.98092609,247.35907425)(134.19248146,246.55018605)(134.19248146,245.52974248)
\curveto(134.19248146,244.40974343)(133.83159288,243.36441099)(133.10981572,242.39374515)
\curveto(132.41292742,241.44796817)(131.31781724,240.67641327)(129.82448518,240.07908045)
\curveto(128.33115312,239.50663649)(126.43959918,239.22041451)(124.14982335,239.22041451)
\curveto(120.76493734,239.22041451)(118.15160623,239.64352527)(116.30983002,240.48974677)
\lineto(116.30983002,245.08174286)
\curveto(117.15605153,244.68352097)(118.18893953,244.32263239)(119.40849405,243.99907711)
\curveto(120.65293744,243.67552183)(121.97204742,243.51374419)(123.36582402,243.51374419)
\curveto(124.88404495,243.51374419)(126.16582163,243.68796627)(127.21115407,244.03641041)
\curveto(128.25648652,244.38485456)(128.77915274,244.99463182)(128.77915274,245.86574219)
\curveto(128.77915274,247.48351859)(126.85026549,248.29240679)(122.992491,248.29240679)
\lineto(120.82715951,248.29240679)
\lineto(120.82715951,252.13773685)
\lineto(122.8804911,252.13773685)
\curveto(124.72226731,252.13773685)(126.14093276,252.28707005)(127.13648747,252.58573647)
\curveto(128.15693105,252.88440288)(128.66715283,253.4444024)(128.66715283,254.26573504)
\curveto(128.66715283,254.9128456)(128.34359755,255.39817852)(127.69648699,255.7217338)
\curveto(127.04937643,256.07017794)(125.99159956,256.24440002)(124.52315636,256.24440002)
\curveto(123.55249052,256.24440002)(122.48226921,256.13240011)(121.31249243,255.9084003)
\curveto(120.16760452,255.6844005)(119.09738321,255.36084522)(118.1018285,254.93773446)
\lineto(116.45916323,258.82039783)
\curveto(117.62894001,259.26839744)(118.89827226,259.62928603)(120.26715999,259.90306357)
\curveto(121.66093658,260.20172998)(123.31604628,260.35106319)(125.23248909,260.35106319)
\closepath
}
}
{
\newrgbcolor{curcolor}{0 0 0}
\pscustom[linestyle=none,fillstyle=solid,fillcolor=curcolor]
{
\newpath
\moveto(149.87247935,260.35106319)
\curveto(152.16225517,260.35106319)(154.01647582,259.45506395)(155.43514128,257.66306548)
\curveto(156.85380673,255.89595587)(157.56313946,253.28262476)(157.56313946,249.82307215)
\curveto(157.56313946,246.33863068)(156.82891787,243.7004107)(155.36047467,241.90841223)
\curveto(153.89203148,240.11641375)(152.01292197,239.22041451)(149.72314614,239.22041451)
\curveto(148.25470295,239.22041451)(147.08492616,239.48174763)(146.21381579,240.00441385)
\curveto(145.34270543,240.55196894)(144.6333727,241.16174619)(144.08581761,241.83374562)
\lineto(143.78715119,241.83374562)
\curveto(143.98626214,240.78841318)(144.08581761,239.79285847)(144.08581761,238.8470815)
\lineto(144.08581761,230.63375516)
\lineto(138.52315568,230.63375516)
\lineto(138.52315568,259.97773017)
\lineto(143.04048516,259.97773017)
\lineto(143.8244845,257.32706576)
\lineto(144.08581761,257.32706576)
\curveto(144.6333727,258.1483984)(145.36759429,258.85773113)(146.2884824,259.45506395)
\curveto(147.2093705,260.05239678)(148.40403615,260.35106319)(149.87247935,260.35106319)
\closepath
\moveto(148.08048087,255.9084003)
\curveto(146.63692655,255.9084003)(145.61648297,255.44795625)(145.01915015,254.52706815)
\curveto(144.42181732,253.63106891)(144.11070647,252.27462562)(144.08581761,250.45773828)
\lineto(144.08581761,249.86040545)
\curveto(144.08581761,247.89418491)(144.37203959,246.37596398)(144.94448354,245.30574267)
\curveto(145.54181637,244.26041022)(146.61203768,243.737744)(148.15514748,243.737744)
\curveto(149.42447973,243.737744)(150.35781227,244.26041022)(150.95514509,245.30574267)
\curveto(151.57736678,246.37596398)(151.88847763,247.90662934)(151.88847763,249.89773876)
\curveto(151.88847763,253.90484646)(150.61914538,255.9084003)(148.08048087,255.9084003)
\closepath
}
}
{
\newrgbcolor{curcolor}{0 0 0}
\pscustom[linestyle=none,fillstyle=solid,fillcolor=curcolor]
{
\newpath
\moveto(170.55510653,260.35106319)
\curveto(173.36754858,260.35106319)(175.59510224,259.54217499)(177.23776751,257.92439859)
\curveto(178.88043278,256.33151106)(179.70176541,254.05417966)(179.70176541,251.09240441)
\lineto(179.70176541,248.40440669)
\lineto(166.56044327,248.40440669)
\curveto(166.610221,246.83640803)(167.07066505,245.60440908)(167.94177542,244.70840984)
\curveto(168.83777466,243.8124106)(170.06977361,243.36441099)(171.63777228,243.36441099)
\curveto(172.9319934,243.36441099)(174.11421461,243.48885532)(175.18443592,243.737744)
\curveto(176.2795461,244.01152155)(177.39954515,244.42218786)(178.54443306,244.96974295)
\lineto(178.54443306,240.67641327)
\curveto(177.52398949,240.17863592)(176.46621261,239.81774734)(175.37110243,239.59374753)
\curveto(174.27599225,239.34485885)(172.94443783,239.22041451)(171.37643917,239.22041451)
\curveto(169.33555201,239.22041451)(167.53110911,239.59374753)(165.96311044,240.34041356)
\curveto(164.39511178,241.11196846)(163.16311283,242.25685637)(162.26711359,243.7750773)
\curveto(161.37111435,245.3181871)(160.92311473,247.27196321)(160.92311473,249.63640565)
\curveto(160.92311473,252.00084808)(161.32133662,253.97951306)(162.11778038,255.57240059)
\curveto(162.93911302,257.16528812)(164.0715565,258.35995377)(165.51511082,259.15639754)
\curveto(166.95866515,259.95284131)(168.63866372,260.35106319)(170.55510653,260.35106319)
\closepath
\moveto(170.59243983,256.39373322)
\curveto(169.49732965,256.39373322)(168.60133042,256.04528908)(167.90444212,255.34840078)
\curveto(167.20755383,254.65151249)(166.79688751,253.56884674)(166.67244317,252.10040355)
\lineto(174.47510319,252.10040355)
\curveto(174.45021433,253.31995806)(174.11421461,254.34040164)(173.46710405,255.16173427)
\curveto(172.84488236,255.98306691)(171.88666095,256.39373322)(170.59243983,256.39373322)
\closepath
}
}
{
\newrgbcolor{curcolor}{0 0 0}
\pscustom[linestyle=none,fillstyle=solid,fillcolor=curcolor]
{
\newpath
\moveto(214.57105268,259.97773017)
\lineto(214.57105268,239.59374753)
\lineto(184.21907853,239.59374753)
\lineto(184.21907853,259.97773017)
\lineto(189.78174046,259.97773017)
\lineto(189.78174046,243.7750773)
\lineto(196.61373464,243.7750773)
\lineto(196.61373464,259.97773017)
\lineto(202.17639657,259.97773017)
\lineto(202.17639657,243.7750773)
\lineto(209.00839075,243.7750773)
\lineto(209.00839075,259.97773017)
\closepath
}
}
{
\newrgbcolor{curcolor}{0 0 0}
\pscustom[linestyle=none,fillstyle=solid,fillcolor=curcolor]
{
\newpath
\moveto(228.7950281,260.35106319)
\curveto(231.60747015,260.35106319)(233.83502381,259.54217499)(235.47768908,257.92439859)
\curveto(237.12035435,256.33151106)(237.94168698,254.05417966)(237.94168698,251.09240441)
\lineto(237.94168698,248.40440669)
\lineto(224.80036484,248.40440669)
\curveto(224.85014257,246.83640803)(225.31058662,245.60440908)(226.18169699,244.70840984)
\curveto(227.07769623,243.8124106)(228.30969518,243.36441099)(229.87769385,243.36441099)
\curveto(231.17191497,243.36441099)(232.35413618,243.48885532)(233.42435749,243.737744)
\curveto(234.51946767,244.01152155)(235.63946672,244.42218786)(236.78435463,244.96974295)
\lineto(236.78435463,240.67641327)
\curveto(235.76391106,240.17863592)(234.70613418,239.81774734)(233.611024,239.59374753)
\curveto(232.51591382,239.34485885)(231.1843594,239.22041451)(229.61636074,239.22041451)
\curveto(227.57547358,239.22041451)(225.77103068,239.59374753)(224.20303201,240.34041356)
\curveto(222.63503335,241.11196846)(221.4030344,242.25685637)(220.50703516,243.7750773)
\curveto(219.61103592,245.3181871)(219.1630363,247.27196321)(219.1630363,249.63640565)
\curveto(219.1630363,252.00084808)(219.56125819,253.97951306)(220.35770195,255.57240059)
\curveto(221.17903459,257.16528812)(222.31147807,258.35995377)(223.75503239,259.15639754)
\curveto(225.19858672,259.95284131)(226.87858529,260.35106319)(228.7950281,260.35106319)
\closepath
\moveto(228.8323614,256.39373322)
\curveto(227.73725122,256.39373322)(226.84125199,256.04528908)(226.14436369,255.34840078)
\curveto(225.4474754,254.65151249)(225.03680908,253.56884674)(224.91236474,252.10040355)
\lineto(232.71502476,252.10040355)
\curveto(232.6901359,253.31995806)(232.35413618,254.34040164)(231.70702562,255.16173427)
\curveto(231.08480393,255.98306691)(230.12658252,256.39373322)(228.8323614,256.39373322)
\closepath
}
}
{
\newrgbcolor{curcolor}{0 0 0}
\pscustom[linestyle=none,fillstyle=solid,fillcolor=curcolor]
{
\newpath
\moveto(248.02167729,259.97773017)
\lineto(248.02167729,252.13773685)
\lineto(255.78700401,252.13773685)
\lineto(255.78700401,259.97773017)
\lineto(261.34966594,259.97773017)
\lineto(261.34966594,239.59374753)
\lineto(255.78700401,239.59374753)
\lineto(255.78700401,247.99374038)
\lineto(248.02167729,247.99374038)
\lineto(248.02167729,239.59374753)
\lineto(242.45901536,239.59374753)
\lineto(242.45901536,259.97773017)
\closepath
}
}
{
\newrgbcolor{curcolor}{0 0 0}
\pscustom[linestyle=none,fillstyle=solid,fillcolor=curcolor]
{
\newpath
\moveto(272.54967142,259.97773017)
\lineto(272.54967142,251.91373704)
\curveto(272.54967142,251.49062629)(272.52478256,250.96796007)(272.47500482,250.34573837)
\curveto(272.45011595,249.72351668)(272.41278265,249.08885056)(272.36300492,248.44174)
\curveto(272.33811605,247.79462944)(272.30078275,247.20974104)(272.25100501,246.68707482)
\curveto(272.20122728,246.18929747)(272.16389397,245.85329776)(272.13900511,245.67907568)
\lineto(281.5469971,259.97773017)
\lineto(288.22965807,259.97773017)
\lineto(288.22965807,239.59374753)
\lineto(282.85366265,239.59374753)
\lineto(282.85366265,247.73240727)
\curveto(282.85366265,248.37951783)(282.87855152,249.11373942)(282.92832925,249.93507206)
\curveto(282.97810699,250.75640469)(283.02788472,251.51551516)(283.07766246,252.21240345)
\curveto(283.15232906,252.93418061)(283.2021068,253.4817357)(283.22699567,253.85506872)
\lineto(273.85633698,239.59374753)
\lineto(267.173676,239.59374753)
\lineto(267.173676,259.97773017)
\closepath
}
}
{
\newrgbcolor{curcolor}{0 0 0}
\pscustom[linestyle=none,fillstyle=solid,fillcolor=curcolor]
{
\newpath
\moveto(302.45362796,260.35106319)
\curveto(305.26607001,260.35106319)(307.49362366,259.54217499)(309.13628893,257.92439859)
\curveto(310.7789542,256.33151106)(311.60028683,254.05417966)(311.60028683,251.09240441)
\lineto(311.60028683,248.40440669)
\lineto(298.45896469,248.40440669)
\curveto(298.50874243,246.83640803)(298.96918648,245.60440908)(299.84029685,244.70840984)
\curveto(300.73629608,243.8124106)(301.96829504,243.36441099)(303.5362937,243.36441099)
\curveto(304.83051482,243.36441099)(306.01273604,243.48885532)(307.08295735,243.737744)
\curveto(308.17806753,244.01152155)(309.29806657,244.42218786)(310.44295449,244.96974295)
\lineto(310.44295449,240.67641327)
\curveto(309.42251091,240.17863592)(308.36473403,239.81774734)(307.26962385,239.59374753)
\curveto(306.17451368,239.34485885)(304.84295925,239.22041451)(303.27496059,239.22041451)
\curveto(301.23407344,239.22041451)(299.42963053,239.59374753)(297.86163187,240.34041356)
\curveto(296.2936332,241.11196846)(295.06163425,242.25685637)(294.16563501,243.7750773)
\curveto(293.26963578,245.3181871)(292.82163616,247.27196321)(292.82163616,249.63640565)
\curveto(292.82163616,252.00084808)(293.21985804,253.97951306)(294.01630181,255.57240059)
\curveto(294.83763444,257.16528812)(295.97007792,258.35995377)(297.41363225,259.15639754)
\curveto(298.85718657,259.95284131)(300.53718514,260.35106319)(302.45362796,260.35106319)
\closepath
\moveto(302.49096126,256.39373322)
\curveto(301.39585108,256.39373322)(300.49985184,256.04528908)(299.80296355,255.34840078)
\curveto(299.10607525,254.65151249)(298.69540893,253.56884674)(298.57096459,252.10040355)
\lineto(306.37362462,252.10040355)
\curveto(306.34873575,253.31995806)(306.01273604,254.34040164)(305.36562548,255.16173427)
\curveto(304.74340378,255.98306691)(303.78518238,256.39373322)(302.49096126,256.39373322)
\closepath
}
}
{
\newrgbcolor{curcolor}{0 0 0}
\pscustom[linestyle=none,fillstyle=solid,fillcolor=curcolor]
{
\newpath
\moveto(331.20026041,259.97773017)
\lineto(331.20026041,251.91373704)
\curveto(331.20026041,251.49062629)(331.17537155,250.96796007)(331.12559381,250.34573837)
\curveto(331.10070494,249.72351668)(331.06337164,249.08885056)(331.01359391,248.44174)
\curveto(330.98870504,247.79462944)(330.95137174,247.20974104)(330.901594,246.68707482)
\curveto(330.85181627,246.18929747)(330.81448296,245.85329776)(330.7895941,245.67907568)
\lineto(340.19758609,259.97773017)
\lineto(346.88024706,259.97773017)
\lineto(346.88024706,239.59374753)
\lineto(341.50425164,239.59374753)
\lineto(341.50425164,247.73240727)
\curveto(341.50425164,248.37951783)(341.52914051,249.11373942)(341.57891824,249.93507206)
\curveto(341.62869598,250.75640469)(341.67847371,251.51551516)(341.72825145,252.21240345)
\curveto(341.80291805,252.93418061)(341.85269579,253.4817357)(341.87758465,253.85506872)
\lineto(332.50692597,239.59374753)
\lineto(325.82426499,239.59374753)
\lineto(325.82426499,259.97773017)
\closepath
}
}
{
\newrgbcolor{curcolor}{0 0 0}
\pscustom[linestyle=none,fillstyle=solid,fillcolor=curcolor]
{
\newpath
\moveto(370.96019329,239.59374753)
\lineto(365.39753136,239.59374753)
\lineto(365.39753136,255.7964004)
\lineto(360.28286905,255.7964004)
\curveto(359.95931377,251.81418157)(359.52375859,248.60351764)(358.9762035,246.1644086)
\curveto(358.45353728,243.75018844)(357.70687125,241.98307883)(356.73620541,240.86307978)
\curveto(355.79042843,239.7679696)(354.53354061,239.22041451)(352.96554195,239.22041451)
\curveto(351.67132083,239.22041451)(350.61354395,239.41952546)(349.79221132,239.81774734)
\lineto(349.79221132,244.26041022)
\curveto(350.36465527,244.01152155)(350.9619881,243.88707721)(351.58420979,243.88707721)
\curveto(352.03220941,243.88707721)(352.44287573,244.11107702)(352.81620874,244.55907664)
\curveto(353.18954176,245.00707625)(353.53798591,245.81596445)(353.86154119,246.98574124)
\curveto(354.20998533,248.15551802)(354.52109618,249.78573885)(354.79487372,251.87640374)
\curveto(355.06865127,253.99195749)(355.31753995,256.69239964)(355.54153976,259.97773017)
\lineto(370.96019329,259.97773017)
\closepath
}
}
{
\newrgbcolor{curcolor}{0 0 0}
\pscustom[linestyle=none,fillstyle=solid,fillcolor=curcolor]
{
\newpath
\moveto(382.16022135,259.97773017)
\lineto(382.16022135,251.91373704)
\curveto(382.16022135,251.49062629)(382.13533248,250.96796007)(382.08555475,250.34573837)
\curveto(382.06066588,249.72351668)(382.02333258,249.08885056)(381.97355484,248.44174)
\curveto(381.94866598,247.79462944)(381.91133267,247.20974104)(381.86155494,246.68707482)
\curveto(381.8117772,246.18929747)(381.7744439,245.85329776)(381.74955503,245.67907568)
\lineto(391.15754702,259.97773017)
\lineto(397.840208,259.97773017)
\lineto(397.840208,239.59374753)
\lineto(392.46421258,239.59374753)
\lineto(392.46421258,247.73240727)
\curveto(392.46421258,248.37951783)(392.48910144,249.11373942)(392.53887918,249.93507206)
\curveto(392.58865692,250.75640469)(392.63843465,251.51551516)(392.68821239,252.21240345)
\curveto(392.76287899,252.93418061)(392.81265672,253.4817357)(392.83754559,253.85506872)
\lineto(383.4668869,239.59374753)
\lineto(376.78422593,239.59374753)
\lineto(376.78422593,259.97773017)
\closepath
}
}
{
\newrgbcolor{curcolor}{0 0 0}
\pscustom[linestyle=none,fillstyle=solid,fillcolor=curcolor]
{
\newpath
\moveto(431.88815238,249.82307215)
\curveto(431.88815238,246.43818615)(430.99215314,243.82485504)(429.20015467,241.98307883)
\curveto(427.43304506,240.14130262)(425.0188249,239.22041451)(421.95749417,239.22041451)
\curveto(420.06594023,239.22041451)(418.37349722,239.63108083)(416.88016516,240.45241347)
\curveto(415.41172197,241.2737461)(414.25438962,242.46841175)(413.40816812,244.03641041)
\curveto(412.56194661,245.62929795)(412.13883586,247.55818519)(412.13883586,249.82307215)
\curveto(412.13883586,253.20795816)(413.02239067,255.80884483)(414.78950027,257.62573218)
\curveto(416.55660988,259.44261952)(418.98327448,260.35106319)(422.06949407,260.35106319)
\curveto(423.98593689,260.35106319)(425.67837989,259.94039687)(427.14682308,259.11906424)
\curveto(428.61526628,258.2977316)(429.77259863,257.10306595)(430.61882013,255.53506729)
\curveto(431.46504163,253.96706862)(431.88815238,252.06307025)(431.88815238,249.82307215)
\closepath
\moveto(417.8134977,249.82307215)
\curveto(417.8134977,247.80707387)(418.13705298,246.27640851)(418.78416354,245.23107606)
\curveto(419.45616297,244.21063249)(420.53882871,243.7004107)(422.03216077,243.7004107)
\curveto(423.50060397,243.7004107)(424.55838084,244.21063249)(425.2054914,245.23107606)
\curveto(425.87749083,246.27640851)(426.21349055,247.80707387)(426.21349055,249.82307215)
\curveto(426.21349055,251.83907044)(425.87749083,253.34484693)(425.2054914,254.34040164)
\curveto(424.55838084,255.36084522)(423.48815953,255.871067)(421.99482747,255.871067)
\curveto(420.52638428,255.871067)(419.45616297,255.36084522)(418.78416354,254.34040164)
\curveto(418.13705298,253.34484693)(417.8134977,251.83907044)(417.8134977,249.82307215)
\closepath
}
}
{
\newrgbcolor{curcolor}{0 0 0}
\pscustom[linestyle=none,fillstyle=solid,fillcolor=curcolor]
{
\newpath
\moveto(452.98144903,255.7964004)
\lineto(446.29878805,255.7964004)
\lineto(446.29878805,239.59374753)
\lineto(440.73612612,239.59374753)
\lineto(440.73612612,255.7964004)
\lineto(434.05346514,255.7964004)
\lineto(434.05346514,259.97773017)
\lineto(452.98144903,259.97773017)
\closepath
}
}
{
\newrgbcolor{curcolor}{0 0 0}
\pscustom[linestyle=none,fillstyle=solid,fillcolor=curcolor]
{
\newpath
\moveto(470.11741287,259.97773017)
\lineto(476.24007432,259.97773017)
\lineto(468.17608119,250.19640517)
\lineto(476.94940705,239.59374753)
\lineto(470.64007909,239.59374753)
\lineto(462.31475285,249.93507206)
\lineto(462.31475285,239.59374753)
\lineto(456.75209092,239.59374753)
\lineto(456.75209092,259.97773017)
\lineto(462.31475285,259.97773017)
\lineto(462.31475285,250.08440526)
\closepath
}
}
{
\newrgbcolor{curcolor}{0 0 0}
\pscustom[linestyle=none,fillstyle=solid,fillcolor=curcolor]
{
\newpath
\moveto(488.22403391,260.38839649)
\curveto(490.96180936,260.38839649)(493.05247424,259.79106367)(494.49602857,258.59639802)
\curveto(495.96447177,257.42662123)(496.69869336,255.62217833)(496.69869336,253.18306929)
\lineto(496.69869336,239.59374753)
\lineto(492.81603,239.59374753)
\lineto(491.73336426,242.35641184)
\lineto(491.58403105,242.35641184)
\curveto(490.71292068,241.26130167)(489.79203258,240.4648579)(488.82136674,239.96708055)
\curveto(487.8507009,239.46930319)(486.51914647,239.22041451)(484.82670347,239.22041451)
\curveto(483.00981613,239.22041451)(481.50403963,239.74308074)(480.30937398,240.78841318)
\curveto(479.11470833,241.83374562)(478.51737551,243.46396646)(478.51737551,245.67907568)
\curveto(478.51737551,247.84440717)(479.27648597,249.4372947)(480.7947069,250.45773828)
\curveto(482.31292783,251.47818185)(484.59025923,252.05062581)(487.62670109,252.17507015)
\lineto(491.17336473,252.28707005)
\lineto(491.17336473,253.18306929)
\curveto(491.17336473,254.2532906)(490.88714276,255.03728994)(490.3146988,255.53506729)
\curveto(489.76714371,256.03284464)(488.99558881,256.28173332)(488.0000341,256.28173332)
\curveto(487.00447939,256.28173332)(486.03381355,256.13240011)(485.08803658,255.8337337)
\curveto(484.14225961,255.55995616)(483.19648264,255.21151201)(482.25070566,254.78840126)
\lineto(480.42137389,258.55906471)
\curveto(481.4915952,259.1066198)(482.69870528,259.54217499)(484.04270414,259.86573027)
\curveto(485.38670299,260.21417442)(486.78047959,260.38839649)(488.22403391,260.38839649)
\closepath
\moveto(491.17336473,249.03907282)
\lineto(489.00803324,248.96440622)
\curveto(487.21603477,248.91462848)(485.97159139,248.5910732)(485.27470309,247.99374038)
\curveto(484.57781479,247.39640755)(484.22937065,246.61240822)(484.22937065,245.64174238)
\curveto(484.22937065,244.79552088)(484.47825932,244.18574362)(484.97603668,243.8124106)
\curveto(485.47381403,243.46396646)(486.12092459,243.28974438)(486.91736836,243.28974438)
\curveto(488.11203401,243.28974438)(489.12003315,243.63818853)(489.94136578,244.33507683)
\curveto(490.76269842,245.05685399)(491.17336473,246.06485313)(491.17336473,247.35907425)
\closepath
}
}
{
\newrgbcolor{curcolor}{0 0 0}
\pscustom[linestyle=none,fillstyle=solid,fillcolor=curcolor]
{
\newpath
\moveto(509.84001778,260.35106319)
\curveto(511.30846097,260.35106319)(512.6773487,260.15195225)(513.94668095,259.75373036)
\curveto(515.24090207,259.38039735)(516.27379008,258.79550896)(517.04534498,257.99906519)
\curveto(517.84178874,257.20262142)(518.24001063,256.18217785)(518.24001063,254.93773446)
\curveto(518.24001063,253.71817995)(517.86667761,252.74751411)(517.12001158,252.02573694)
\curveto(516.37334555,251.30395978)(515.39023528,250.78129356)(514.17068076,250.45773828)
\lineto(514.17068076,250.27107177)
\curveto(515.04179113,250.07196083)(515.82579046,249.78573885)(516.52267876,249.41240584)
\curveto(517.21956705,249.06396169)(517.76712214,248.57862877)(518.16534402,247.95640708)
\curveto(518.58845477,247.35907425)(518.80001015,246.55018605)(518.80001015,245.52974248)
\curveto(518.80001015,244.40974343)(518.43912157,243.36441099)(517.71734441,242.39374515)
\curveto(517.02045611,241.44796817)(515.92534593,240.67641327)(514.43201387,240.07908045)
\curveto(512.93868181,239.50663649)(511.04712786,239.22041451)(508.75735203,239.22041451)
\curveto(505.37246603,239.22041451)(502.75913492,239.64352527)(500.91735871,240.48974677)
\lineto(500.91735871,245.08174286)
\curveto(501.76358021,244.68352097)(502.79646822,244.32263239)(504.01602274,243.99907711)
\curveto(505.26046612,243.67552183)(506.57957611,243.51374419)(507.9733527,243.51374419)
\curveto(509.49157363,243.51374419)(510.77335032,243.68796627)(511.81868276,244.03641041)
\curveto(512.8640152,244.38485456)(513.38668143,244.99463182)(513.38668143,245.86574219)
\curveto(513.38668143,247.48351859)(511.45779418,248.29240679)(507.60001969,248.29240679)
\lineto(505.4346882,248.29240679)
\lineto(505.4346882,252.13773685)
\lineto(507.48801978,252.13773685)
\curveto(509.32979599,252.13773685)(510.74846145,252.28707005)(511.74401616,252.58573647)
\curveto(512.76445973,252.88440288)(513.27468152,253.4444024)(513.27468152,254.26573504)
\curveto(513.27468152,254.9128456)(512.95112624,255.39817852)(512.30401568,255.7217338)
\curveto(511.65690512,256.07017794)(510.59912824,256.24440002)(509.13068505,256.24440002)
\curveto(508.16001921,256.24440002)(507.0897979,256.13240011)(505.92002112,255.9084003)
\curveto(504.7751332,255.6844005)(503.70491189,255.36084522)(502.70935718,254.93773446)
\lineto(501.06669192,258.82039783)
\curveto(502.2364687,259.26839744)(503.50580095,259.62928603)(504.87468867,259.90306357)
\curveto(506.26846526,260.20172998)(507.92357497,260.35106319)(509.84001778,260.35106319)
\closepath
}
}
{
\newrgbcolor{curcolor}{0 0 0}
\pscustom[linestyle=none,fillstyle=solid,fillcolor=curcolor]
{
\newpath
\moveto(529.92535951,259.97773017)
\lineto(536.01068766,259.97773017)
\lineto(539.85601772,248.5164066)
\curveto(540.05512866,247.94396264)(540.20446187,247.37151868)(540.30401734,246.79907473)
\curveto(540.40357281,246.22663077)(540.47823941,245.61685351)(540.52801715,244.96974295)
\lineto(540.64001705,244.96974295)
\curveto(540.71468366,245.61685351)(540.81423913,246.22663077)(540.93868347,246.79907473)
\curveto(541.06312781,247.37151868)(541.22490545,247.94396264)(541.42401639,248.5164066)
\lineto(545.19467984,259.97773017)
\lineto(551.16800809,259.97773017)
\lineto(542.54401543,236.98041642)
\curveto(541.74757167,234.86486267)(540.61512819,233.28441957)(539.14668499,232.23908713)
\curveto(537.6782418,231.16886581)(535.97335436,230.63375516)(534.03202268,230.63375516)
\curveto(533.38491212,230.63375516)(532.83735703,230.67108846)(532.38935741,230.74575506)
\curveto(531.94135779,230.7955328)(531.54313591,230.85775497)(531.19469176,230.93242157)
\lineto(531.19469176,235.33775115)
\curveto(531.44358044,235.28797342)(531.76713572,235.23819568)(532.1653576,235.18841795)
\curveto(532.56357949,235.13864021)(532.9742458,235.11375134)(533.39735655,235.11375134)
\curveto(534.56713334,235.11375134)(535.48802144,235.47463993)(536.16002087,236.19641709)
\curveto(536.8320203,236.89330538)(537.34224208,237.73952689)(537.69068623,238.73508159)
\lineto(538.02668595,239.74308074)
\closepath
}
}
{
\newrgbcolor{curcolor}{0 0 0}
\pscustom[linestyle=none,fillstyle=solid,fillcolor=curcolor]
{
\newpath
\moveto(558.9706648,259.97773017)
\lineto(558.9706648,252.51106986)
\curveto(558.9706648,250.74396026)(559.79199743,249.86040545)(561.4346627,249.86040545)
\curveto(562.50488401,249.86040545)(563.50043872,249.97240536)(564.42132682,250.19640517)
\curveto(565.34221493,250.44529385)(566.26310303,250.76884913)(567.18399114,251.16707101)
\lineto(567.18399114,259.97773017)
\lineto(572.74665307,259.97773017)
\lineto(572.74665307,239.59374753)
\lineto(567.18399114,239.59374753)
\lineto(567.18399114,247.69507396)
\curveto(566.31288077,247.22218548)(565.31732606,246.78663029)(564.19732701,246.38840841)
\curveto(563.07732797,246.0150754)(561.80799572,245.82840889)(560.38933026,245.82840889)
\curveto(558.2737765,245.82840889)(556.5813335,246.36351954)(555.31200125,247.43374085)
\curveto(554.04266899,248.52885103)(553.40800287,250.18396073)(553.40800287,252.39906996)
\lineto(553.40800287,259.97773017)
\closepath
}
}
{
\newrgbcolor{curcolor}{0 0 0}
\pscustom[linestyle=none,fillstyle=solid,fillcolor=curcolor]
{
\newpath
\moveto(586.93332346,260.38839649)
\curveto(589.67109891,260.38839649)(591.7617638,259.79106367)(593.20531812,258.59639802)
\curveto(594.67376132,257.42662123)(595.40798291,255.62217833)(595.40798291,253.18306929)
\lineto(595.40798291,239.59374753)
\lineto(591.52531955,239.59374753)
\lineto(590.44265381,242.35641184)
\lineto(590.2933206,242.35641184)
\curveto(589.42221023,241.26130167)(588.50132213,240.4648579)(587.53065629,239.96708055)
\curveto(586.55999045,239.46930319)(585.22843603,239.22041451)(583.53599302,239.22041451)
\curveto(581.71910568,239.22041451)(580.21332918,239.74308074)(579.01866354,240.78841318)
\curveto(577.82399789,241.83374562)(577.22666506,243.46396646)(577.22666506,245.67907568)
\curveto(577.22666506,247.84440717)(577.98577553,249.4372947)(579.50399646,250.45773828)
\curveto(581.02221738,251.47818185)(583.29954878,252.05062581)(586.33599064,252.17507015)
\lineto(589.88265428,252.28707005)
\lineto(589.88265428,253.18306929)
\curveto(589.88265428,254.2532906)(589.59643231,255.03728994)(589.02398835,255.53506729)
\curveto(588.47643326,256.03284464)(587.70487836,256.28173332)(586.70932365,256.28173332)
\curveto(585.71376895,256.28173332)(584.74310311,256.13240011)(583.79732613,255.8337337)
\curveto(582.85154916,255.55995616)(581.90577219,255.21151201)(580.95999522,254.78840126)
\lineto(579.13066344,258.55906471)
\curveto(580.20088475,259.1066198)(581.40799483,259.54217499)(582.75199369,259.86573027)
\curveto(584.09599255,260.21417442)(585.48976914,260.38839649)(586.93332346,260.38839649)
\closepath
\moveto(589.88265428,249.03907282)
\lineto(587.71732279,248.96440622)
\curveto(585.92532432,248.91462848)(584.68088094,248.5910732)(583.98399264,247.99374038)
\curveto(583.28710434,247.39640755)(582.9386602,246.61240822)(582.9386602,245.64174238)
\curveto(582.9386602,244.79552088)(583.18754887,244.18574362)(583.68532623,243.8124106)
\curveto(584.18310358,243.46396646)(584.83021414,243.28974438)(585.62665791,243.28974438)
\curveto(586.82132356,243.28974438)(587.8293227,243.63818853)(588.65065533,244.33507683)
\curveto(589.47198797,245.05685399)(589.88265428,246.06485313)(589.88265428,247.35907425)
\closepath
}
}
{
\newrgbcolor{curcolor}{0 0 0}
\pscustom[linestyle=none,fillstyle=solid,fillcolor=curcolor]
{
\newpath
\moveto(609.40797327,239.22041451)
\curveto(606.37153141,239.22041451)(604.01953341,240.05419158)(602.35197927,241.72174572)
\curveto(600.70931401,243.38929985)(599.88798137,246.03996426)(599.88798137,249.67373895)
\curveto(599.88798137,252.16262572)(600.31109212,254.19106843)(601.15731362,255.7590671)
\curveto(602.00353513,257.32706576)(603.17331191,258.48439811)(604.66664397,259.23106414)
\curveto(606.1848649,259.97773017)(607.92708564,260.35106319)(609.89330619,260.35106319)
\curveto(611.28708278,260.35106319)(612.49419286,260.21417442)(613.51463644,259.94039687)
\curveto(614.55996888,259.66661933)(615.46841255,259.34306405)(616.23996745,258.96973103)
\lineto(614.59730218,254.67640135)
\curveto(613.72619181,255.0248455)(612.90485918,255.31106748)(612.13330428,255.53506729)
\curveto(611.38663825,255.7590671)(610.63997222,255.871067)(609.89330619,255.871067)
\curveto(607.00619753,255.871067)(605.56264321,253.81773542)(605.56264321,249.71107225)
\curveto(605.56264321,247.6701851)(605.93597622,246.1644086)(606.68264225,245.19374276)
\curveto(607.45419715,244.22307692)(608.52441846,243.737744)(609.89330619,243.737744)
\curveto(611.06308297,243.737744)(612.09597098,243.88707721)(612.99197021,244.18574362)
\curveto(613.88796945,244.5092989)(614.75907982,244.94485408)(615.60530132,245.49240917)
\lineto(615.60530132,240.75107988)
\curveto(614.75907982,240.20352479)(613.86308058,239.81774734)(612.91730361,239.59374753)
\curveto(611.99641551,239.34485885)(610.82663872,239.22041451)(609.40797327,239.22041451)
\closepath
}
}
{
\newrgbcolor{curcolor}{0 0 0}
\pscustom[linestyle=none,fillstyle=solid,fillcolor=curcolor]
{
\newpath
\moveto(637.18396368,255.7964004)
\lineto(630.5013027,255.7964004)
\lineto(630.5013027,239.59374753)
\lineto(624.93864077,239.59374753)
\lineto(624.93864077,255.7964004)
\lineto(618.25597979,255.7964004)
\lineto(618.25597979,259.97773017)
\lineto(637.18396368,259.97773017)
\closepath
}
}
{
\newrgbcolor{curcolor}{0 0 0}
\pscustom[linestyle=none,fillstyle=solid,fillcolor=curcolor]
{
\newpath
\moveto(646.51723698,259.97773017)
\lineto(646.51723698,252.13773685)
\lineto(654.2825637,252.13773685)
\lineto(654.2825637,259.97773017)
\lineto(659.84522563,259.97773017)
\lineto(659.84522563,239.59374753)
\lineto(654.2825637,239.59374753)
\lineto(654.2825637,247.99374038)
\lineto(646.51723698,247.99374038)
\lineto(646.51723698,239.59374753)
\lineto(640.95457505,239.59374753)
\lineto(640.95457505,259.97773017)
\closepath
}
}
{
\newrgbcolor{curcolor}{0 0 0}
\pscustom[linestyle=none,fillstyle=solid,fillcolor=curcolor]
{
\newpath
\moveto(671.04523112,259.97773017)
\lineto(671.04523112,251.91373704)
\curveto(671.04523112,251.49062629)(671.02034225,250.96796007)(670.97056451,250.34573837)
\curveto(670.94567565,249.72351668)(670.90834234,249.08885056)(670.85856461,248.44174)
\curveto(670.83367574,247.79462944)(670.79634244,247.20974104)(670.7465647,246.68707482)
\curveto(670.69678697,246.18929747)(670.65945367,245.85329776)(670.6345648,245.67907568)
\lineto(680.04255679,259.97773017)
\lineto(686.72521776,259.97773017)
\lineto(686.72521776,239.59374753)
\lineto(681.34922234,239.59374753)
\lineto(681.34922234,247.73240727)
\curveto(681.34922234,248.37951783)(681.37411121,249.11373942)(681.42388895,249.93507206)
\curveto(681.47366668,250.75640469)(681.52344442,251.51551516)(681.57322215,252.21240345)
\curveto(681.64788875,252.93418061)(681.69766649,253.4817357)(681.72255536,253.85506872)
\lineto(672.35189667,239.59374753)
\lineto(665.66923569,239.59374753)
\lineto(665.66923569,259.97773017)
\closepath
}
}
{
\newrgbcolor{curcolor}{0 0 0}
\pscustom[linestyle=none,fillstyle=solid,fillcolor=curcolor]
{
\newpath
\moveto(705.91456253,259.97773017)
\lineto(712.03722398,259.97773017)
\lineto(703.97323085,250.19640517)
\lineto(712.74655671,239.59374753)
\lineto(706.43722875,239.59374753)
\lineto(698.11190251,249.93507206)
\lineto(698.11190251,239.59374753)
\lineto(692.54924058,239.59374753)
\lineto(692.54924058,259.97773017)
\lineto(698.11190251,259.97773017)
\lineto(698.11190251,250.08440526)
\closepath
}
}
{
\newrgbcolor{curcolor}{0 0 0}
\pscustom[linestyle=none,fillstyle=solid,fillcolor=curcolor]
{
\newpath
\moveto(712.74649599,259.97773017)
\lineto(718.83182414,259.97773017)
\lineto(722.6771542,248.5164066)
\curveto(722.87626514,247.94396264)(723.02559834,247.37151868)(723.12515382,246.79907473)
\curveto(723.22470929,246.22663077)(723.29937589,245.61685351)(723.34915362,244.96974295)
\lineto(723.46115353,244.96974295)
\curveto(723.53582013,245.61685351)(723.6353756,246.22663077)(723.75981994,246.79907473)
\curveto(723.88426428,247.37151868)(724.04604192,247.94396264)(724.24515286,248.5164066)
\lineto(728.01581632,259.97773017)
\lineto(733.98914456,259.97773017)
\lineto(725.36515191,236.98041642)
\curveto(724.56870814,234.86486267)(723.43626466,233.28441957)(721.96782147,232.23908713)
\curveto(720.49937827,231.16886581)(718.79449084,230.63375516)(716.85315916,230.63375516)
\curveto(716.2060486,230.63375516)(715.65849351,230.67108846)(715.21049389,230.74575506)
\curveto(714.76249427,230.7955328)(714.36427239,230.85775497)(714.01582824,230.93242157)
\lineto(714.01582824,235.33775115)
\curveto(714.26471692,235.28797342)(714.5882722,235.23819568)(714.98649408,235.18841795)
\curveto(715.38471596,235.13864021)(715.79538228,235.11375134)(716.21849303,235.11375134)
\curveto(717.38826981,235.11375134)(718.30915792,235.47463993)(718.98115734,236.19641709)
\curveto(719.65315677,236.89330538)(720.16337856,237.73952689)(720.51182271,238.73508159)
\lineto(720.84782242,239.74308074)
\closepath
}
}
{
\newrgbcolor{curcolor}{0 0 0}
\pscustom[linestyle=none,fillstyle=solid,fillcolor=curcolor]
{
\newpath
\moveto(752.17049626,259.97773017)
\lineto(752.17049626,252.13773685)
\lineto(759.93582298,252.13773685)
\lineto(759.93582298,259.97773017)
\lineto(765.49848491,259.97773017)
\lineto(765.49848491,239.59374753)
\lineto(759.93582298,239.59374753)
\lineto(759.93582298,247.99374038)
\lineto(752.17049626,247.99374038)
\lineto(752.17049626,239.59374753)
\lineto(746.60783433,239.59374753)
\lineto(746.60783433,259.97773017)
\closepath
}
}
{
\newrgbcolor{curcolor}{0 0 0}
\pscustom[linestyle=none,fillstyle=solid,fillcolor=curcolor]
{
\newpath
\moveto(779.68515452,260.38839649)
\curveto(782.42292996,260.38839649)(784.51359485,259.79106367)(785.95714918,258.59639802)
\curveto(787.42559237,257.42662123)(788.15981397,255.62217833)(788.15981397,253.18306929)
\lineto(788.15981397,239.59374753)
\lineto(784.27715061,239.59374753)
\lineto(783.19448486,242.35641184)
\lineto(783.04515166,242.35641184)
\curveto(782.17404129,241.26130167)(781.25315318,240.4648579)(780.28248734,239.96708055)
\curveto(779.3118215,239.46930319)(777.98026708,239.22041451)(776.28782408,239.22041451)
\curveto(774.47093673,239.22041451)(772.96516024,239.74308074)(771.77049459,240.78841318)
\curveto(770.57582894,241.83374562)(769.97849612,243.46396646)(769.97849612,245.67907568)
\curveto(769.97849612,247.84440717)(770.73760658,249.4372947)(772.25582751,250.45773828)
\curveto(773.77404844,251.47818185)(776.05137983,252.05062581)(779.08782169,252.17507015)
\lineto(782.63448534,252.28707005)
\lineto(782.63448534,253.18306929)
\curveto(782.63448534,254.2532906)(782.34826336,255.03728994)(781.7758194,255.53506729)
\curveto(781.22826431,256.03284464)(780.45670942,256.28173332)(779.46115471,256.28173332)
\curveto(778.4656,256.28173332)(777.49493416,256.13240011)(776.54915719,255.8337337)
\curveto(775.60338021,255.55995616)(774.65760324,255.21151201)(773.71182627,254.78840126)
\lineto(771.88249449,258.55906471)
\curveto(772.95271581,259.1066198)(774.15982589,259.54217499)(775.50382474,259.86573027)
\curveto(776.8478236,260.21417442)(778.24160019,260.38839649)(779.68515452,260.38839649)
\closepath
\moveto(782.63448534,249.03907282)
\lineto(780.46915385,248.96440622)
\curveto(778.67715538,248.91462848)(777.43271199,248.5910732)(776.7358237,247.99374038)
\curveto(776.0389354,247.39640755)(775.69049125,246.61240822)(775.69049125,245.64174238)
\curveto(775.69049125,244.79552088)(775.93937993,244.18574362)(776.43715728,243.8124106)
\curveto(776.93493464,243.46396646)(777.5820452,243.28974438)(778.37848896,243.28974438)
\curveto(779.57315461,243.28974438)(780.58115375,243.63818853)(781.40248639,244.33507683)
\curveto(782.22381902,245.05685399)(782.63448534,246.06485313)(782.63448534,247.35907425)
\closepath
}
}
{
\newrgbcolor{curcolor}{0 0 0}
\pscustom[linestyle=none,fillstyle=solid,fillcolor=curcolor]
{
\newpath
\moveto(90.32583549,213.3111042)
\lineto(90.32583549,192.92712155)
\lineto(84.76317356,192.92712155)
\lineto(84.76317356,209.12977442)
\lineto(77.37117985,209.12977442)
\lineto(77.37117985,192.92712155)
\lineto(71.80851792,192.92712155)
\lineto(71.80851792,213.3111042)
\closepath
}
}
{
\newrgbcolor{curcolor}{0 0 0}
\pscustom[linestyle=none,fillstyle=solid,fillcolor=curcolor]
{
\newpath
\moveto(114.66716315,203.15644618)
\curveto(114.66716315,199.77156017)(113.77116392,197.15822906)(111.97916544,195.31645285)
\curveto(110.21205584,193.47467664)(107.79783567,192.55378854)(104.73650494,192.55378854)
\curveto(102.844951,192.55378854)(101.15250799,192.96445485)(99.65917593,193.78578749)
\curveto(98.19073274,194.60712012)(97.03340039,195.80178577)(96.18717889,197.36978444)
\curveto(95.34095739,198.96267197)(94.91784664,200.89155922)(94.91784664,203.15644618)
\curveto(94.91784664,206.54133218)(95.80140144,209.14221886)(97.56851105,210.9591062)
\curveto(99.33562065,212.77599354)(101.76228525,213.68443721)(104.84850485,213.68443721)
\curveto(106.76494766,213.68443721)(108.45739066,213.27377089)(109.92583386,212.45243826)
\curveto(111.39427705,211.63110563)(112.5516094,210.43643998)(113.3978309,208.86844131)
\curveto(114.2440524,207.30044265)(114.66716315,205.39644427)(114.66716315,203.15644618)
\closepath
\moveto(100.59250847,203.15644618)
\curveto(100.59250847,201.14044789)(100.91606375,199.60978253)(101.56317431,198.56445009)
\curveto(102.23517374,197.54400651)(103.31783948,197.03378472)(104.81117155,197.03378472)
\curveto(106.27961474,197.03378472)(107.33739162,197.54400651)(107.98450218,198.56445009)
\curveto(108.6565016,199.60978253)(108.99250132,201.14044789)(108.99250132,203.15644618)
\curveto(108.99250132,205.17244446)(108.6565016,206.67822096)(107.98450218,207.67377566)
\curveto(107.33739162,208.69421924)(106.26717031,209.20444103)(104.77383824,209.20444103)
\curveto(103.30539505,209.20444103)(102.23517374,208.69421924)(101.56317431,207.67377566)
\curveto(100.91606375,206.67822096)(100.59250847,205.17244446)(100.59250847,203.15644618)
\closepath
}
}
{
\newrgbcolor{curcolor}{0 0 0}
\pscustom[linestyle=none,fillstyle=solid,fillcolor=curcolor]
{
\newpath
\moveto(137.51512493,192.92712155)
\lineto(131.952463,192.92712155)
\lineto(131.952463,209.12977442)
\lineto(126.83780069,209.12977442)
\curveto(126.51424541,205.14755559)(126.07869023,201.93689166)(125.53113514,199.49778263)
\curveto(125.00846892,197.08356246)(124.26180289,195.31645285)(123.29113705,194.19645381)
\curveto(122.34536007,193.10134363)(121.08847225,192.55378854)(119.52047359,192.55378854)
\curveto(118.22625247,192.55378854)(117.16847559,192.75289948)(116.34714296,193.15112136)
\lineto(116.34714296,197.59378425)
\curveto(116.91958692,197.34489557)(117.51691974,197.22045123)(118.13914143,197.22045123)
\curveto(118.58714105,197.22045123)(118.99780737,197.44445104)(119.37114038,197.89245066)
\curveto(119.7444734,198.34045028)(120.09291755,199.14933848)(120.41647283,200.31911526)
\curveto(120.76491697,201.48889204)(121.07602782,203.11911287)(121.34980537,205.20977776)
\curveto(121.62358291,207.32533152)(121.87247159,210.02577366)(122.0964714,213.3111042)
\lineto(137.51512493,213.3111042)
\closepath
}
}
{
\newrgbcolor{curcolor}{0 0 0}
\pscustom[linestyle=none,fillstyle=solid,fillcolor=curcolor]
{
\newpath
\moveto(140.42713716,213.3111042)
\lineto(146.51246531,213.3111042)
\lineto(150.35779537,201.84978062)
\curveto(150.55690631,201.27733667)(150.70623952,200.70489271)(150.80579499,200.13244875)
\curveto(150.90535046,199.56000479)(150.98001706,198.95022754)(151.0297948,198.30311698)
\lineto(151.1417947,198.30311698)
\curveto(151.21646131,198.95022754)(151.31601678,199.56000479)(151.44046112,200.13244875)
\curveto(151.56490545,200.70489271)(151.72668309,201.27733667)(151.92579404,201.84978062)
\lineto(155.69645749,213.3111042)
\lineto(161.66978574,213.3111042)
\lineto(153.04579308,190.31379045)
\curveto(152.24934932,188.19823669)(151.11690584,186.61779359)(149.64846264,185.57246115)
\curveto(148.18001945,184.50223984)(146.47513201,183.96712918)(144.53380033,183.96712918)
\curveto(143.88668977,183.96712918)(143.33913468,184.00446248)(142.89113506,184.07912909)
\curveto(142.44313544,184.12890682)(142.04491356,184.19112899)(141.69646941,184.2657956)
\lineto(141.69646941,188.67112518)
\curveto(141.94535809,188.62134744)(142.26891337,188.57156971)(142.66713525,188.52179197)
\curveto(143.06535714,188.47201424)(143.47602345,188.44712537)(143.8991342,188.44712537)
\curveto(145.06891099,188.44712537)(145.98979909,188.80801395)(146.66179852,189.52979111)
\curveto(147.33379795,190.22667941)(147.84401973,191.07290091)(148.19246388,192.06845562)
\lineto(148.5284636,193.07645476)
\closepath
}
}
{
\newrgbcolor{curcolor}{0 0 0}
\pscustom[linestyle=none,fillstyle=solid,fillcolor=curcolor]
{
\newpath
\moveto(169.47245771,213.3111042)
\lineto(169.47245771,205.84444389)
\curveto(169.47245771,204.07733428)(170.29379034,203.19377948)(171.93645561,203.19377948)
\curveto(173.00667692,203.19377948)(174.00223163,203.30577938)(174.92311973,203.52977919)
\curveto(175.84400784,203.77866787)(176.76489594,204.10222315)(177.68578405,204.50044503)
\lineto(177.68578405,213.3111042)
\lineto(183.24844598,213.3111042)
\lineto(183.24844598,192.92712155)
\lineto(177.68578405,192.92712155)
\lineto(177.68578405,201.02844799)
\curveto(176.81467368,200.5555595)(175.81911897,200.12000432)(174.69911992,199.72178243)
\curveto(173.57912088,199.34844942)(172.30978862,199.16178291)(170.89112316,199.16178291)
\curveto(168.77556941,199.16178291)(167.08312641,199.69689357)(165.81379415,200.76711488)
\curveto(164.5444619,201.86222506)(163.90979578,203.51733476)(163.90979578,205.73244398)
\lineto(163.90979578,213.3111042)
\closepath
}
}
{
\newrgbcolor{curcolor}{0 0 0}
\pscustom[linestyle=none,fillstyle=solid,fillcolor=curcolor]
{
\newpath
\moveto(197.47244204,213.68443721)
\curveto(200.28488409,213.68443721)(202.51243775,212.87554901)(204.15510302,211.25777261)
\curveto(205.79776829,209.66488508)(206.61910092,207.38755368)(206.61910092,204.42577843)
\lineto(206.61910092,201.73778072)
\lineto(193.47777878,201.73778072)
\curveto(193.52755651,200.16978205)(193.98800056,198.9377831)(194.85911093,198.04178386)
\curveto(195.75511017,197.14578463)(196.98710912,196.69778501)(198.55510779,196.69778501)
\curveto(199.84932891,196.69778501)(201.03155012,196.82222935)(202.10177143,197.07111802)
\curveto(203.19688161,197.34489557)(204.31688066,197.75556189)(205.46176857,198.30311698)
\lineto(205.46176857,194.0097873)
\curveto(204.441325,193.51200994)(203.38354812,193.15112136)(202.28843794,192.92712155)
\curveto(201.19332776,192.67823288)(199.86177334,192.55378854)(198.29377468,192.55378854)
\curveto(196.25288753,192.55378854)(194.44844462,192.92712155)(192.88044595,193.67378758)
\curveto(191.31244729,194.44534248)(190.08044834,195.5902304)(189.1844491,197.10845133)
\curveto(188.28844986,198.65156112)(187.84045024,200.60533724)(187.84045024,202.96977967)
\curveto(187.84045024,205.3342221)(188.23867213,207.31288708)(189.03511589,208.90577461)
\curveto(189.85644853,210.49866215)(190.98889201,211.6933278)(192.43244633,212.48977156)
\curveto(193.87600066,213.28621533)(195.55599923,213.68443721)(197.47244204,213.68443721)
\closepath
\moveto(197.50977534,209.72710725)
\curveto(196.41466517,209.72710725)(195.51866593,209.3786631)(194.82177763,208.6817748)
\curveto(194.12488934,207.98488651)(193.71422302,206.90222076)(193.58977868,205.43377757)
\lineto(201.3924387,205.43377757)
\curveto(201.36754984,206.65333209)(201.03155012,207.67377566)(200.38443956,208.4951083)
\curveto(199.76221787,209.31644093)(198.80399646,209.72710725)(197.50977534,209.72710725)
\closepath
}
}
{
\newrgbcolor{curcolor}{0 0 0}
\pscustom[linestyle=none,fillstyle=solid,fillcolor=curcolor]
{
\newpath
\moveto(216.69907597,213.3111042)
\lineto(216.69907597,205.47111087)
\lineto(224.46440269,205.47111087)
\lineto(224.46440269,213.3111042)
\lineto(230.02706462,213.3111042)
\lineto(230.02706462,192.92712155)
\lineto(224.46440269,192.92712155)
\lineto(224.46440269,201.3271144)
\lineto(216.69907597,201.3271144)
\lineto(216.69907597,192.92712155)
\lineto(211.13641404,192.92712155)
\lineto(211.13641404,213.3111042)
\closepath
}
}
{
\newrgbcolor{curcolor}{0 0 0}
\pscustom[linestyle=none,fillstyle=solid,fillcolor=curcolor]
{
\newpath
\moveto(241.22708536,213.3111042)
\lineto(241.22708536,205.24711106)
\curveto(241.22708536,204.82400031)(241.2021965,204.30133409)(241.15241876,203.6791124)
\curveto(241.12752989,203.05689071)(241.09019659,202.42222458)(241.04041886,201.77511402)
\curveto(241.01552999,201.12800346)(240.97819669,200.54311507)(240.92841895,200.02044885)
\curveto(240.87864122,199.52267149)(240.84130792,199.18667178)(240.81641905,199.0124497)
\lineto(250.22441104,213.3111042)
\lineto(256.90707201,213.3111042)
\lineto(256.90707201,192.92712155)
\lineto(251.53107659,192.92712155)
\lineto(251.53107659,201.06578129)
\curveto(251.53107659,201.71289185)(251.55596546,202.44711345)(251.60574319,203.26844608)
\curveto(251.65552093,204.08977871)(251.70529866,204.84888918)(251.7550764,205.54577748)
\curveto(251.829743,206.26755464)(251.87952074,206.81510973)(251.90440961,207.18844274)
\lineto(242.53375092,192.92712155)
\lineto(235.85108994,192.92712155)
\lineto(235.85108994,213.3111042)
\closepath
}
}
{
\newrgbcolor{curcolor}{0 0 0}
\pscustom[linestyle=none,fillstyle=solid,fillcolor=curcolor]
{
\newpath
\moveto(271.13102664,213.68443721)
\curveto(273.94346869,213.68443721)(276.17102235,212.87554901)(277.81368761,211.25777261)
\curveto(279.45635288,209.66488508)(280.27768552,207.38755368)(280.27768552,204.42577843)
\lineto(280.27768552,201.73778072)
\lineto(267.13636337,201.73778072)
\curveto(267.18614111,200.16978205)(267.64658516,198.9377831)(268.51769553,198.04178386)
\curveto(269.41369477,197.14578463)(270.64569372,196.69778501)(272.21369238,196.69778501)
\curveto(273.5079135,196.69778501)(274.69013472,196.82222935)(275.76035603,197.07111802)
\curveto(276.85546621,197.34489557)(277.97546525,197.75556189)(279.12035317,198.30311698)
\lineto(279.12035317,194.0097873)
\curveto(278.09990959,193.51200994)(277.04213271,193.15112136)(275.94702254,192.92712155)
\curveto(274.85191236,192.67823288)(273.52035794,192.55378854)(271.95235927,192.55378854)
\curveto(269.91147212,192.55378854)(268.10702921,192.92712155)(266.53903055,193.67378758)
\curveto(264.97103188,194.44534248)(263.73903293,195.5902304)(262.84303369,197.10845133)
\curveto(261.94703446,198.65156112)(261.49903484,200.60533724)(261.49903484,202.96977967)
\curveto(261.49903484,205.3342221)(261.89725672,207.31288708)(262.69370049,208.90577461)
\curveto(263.51503312,210.49866215)(264.6474766,211.6933278)(266.09103093,212.48977156)
\curveto(267.53458525,213.28621533)(269.21458382,213.68443721)(271.13102664,213.68443721)
\closepath
\moveto(271.16835994,209.72710725)
\curveto(270.07324976,209.72710725)(269.17725052,209.3786631)(268.48036223,208.6817748)
\curveto(267.78347393,207.98488651)(267.37280761,206.90222076)(267.24836328,205.43377757)
\lineto(275.0510233,205.43377757)
\curveto(275.02613443,206.65333209)(274.69013472,207.67377566)(274.04302416,208.4951083)
\curveto(273.42080247,209.31644093)(272.46258106,209.72710725)(271.16835994,209.72710725)
\closepath
}
}
{
\newrgbcolor{curcolor}{0 0 0}
\pscustom[linestyle=none,fillstyle=solid,fillcolor=curcolor]
{
\newpath
\moveto(313.01898124,213.3111042)
\lineto(313.01898124,192.92712155)
\lineto(307.45631931,192.92712155)
\lineto(307.45631931,209.12977442)
\lineto(300.0643256,209.12977442)
\lineto(300.0643256,192.92712155)
\lineto(294.50166367,192.92712155)
\lineto(294.50166367,213.3111042)
\closepath
}
}
{
\newrgbcolor{curcolor}{0 0 0}
\pscustom[linestyle=none,fillstyle=solid,fillcolor=curcolor]
{
\newpath
\moveto(330.19232645,213.68443721)
\curveto(332.48210228,213.68443721)(334.33632292,212.78843797)(335.75498838,210.9964395)
\curveto(337.17365384,209.22932989)(337.88298657,206.61599879)(337.88298657,203.15644618)
\curveto(337.88298657,199.6720047)(337.14876497,197.03378472)(335.68032178,195.24178625)
\curveto(334.21187859,193.44978777)(332.33276908,192.55378854)(330.04299325,192.55378854)
\curveto(328.57455005,192.55378854)(327.40477327,192.81512165)(326.5336629,193.33778787)
\curveto(325.66255253,193.88534296)(324.9532198,194.49512022)(324.40566471,195.16711965)
\lineto(324.1069983,195.16711965)
\curveto(324.30610924,194.1217872)(324.40566471,193.12623249)(324.40566471,192.18045552)
\lineto(324.40566471,183.96712918)
\lineto(318.84300278,183.96712918)
\lineto(318.84300278,213.3111042)
\lineto(323.36033227,213.3111042)
\lineto(324.1443316,210.66043979)
\lineto(324.40566471,210.66043979)
\curveto(324.9532198,211.48177242)(325.6874414,212.19110515)(326.6083295,212.78843797)
\curveto(327.52921761,213.3857708)(328.72388326,213.68443721)(330.19232645,213.68443721)
\closepath
\moveto(328.40032798,209.24177433)
\curveto(326.95677365,209.24177433)(325.93633008,208.78133028)(325.33899725,207.86044217)
\curveto(324.74166443,206.96444293)(324.43055358,205.60799964)(324.40566471,203.7911123)
\lineto(324.40566471,203.19377948)
\curveto(324.40566471,201.22755893)(324.69188669,199.709338)(325.26433065,198.63911669)
\curveto(325.86166347,197.59378425)(326.93188478,197.07111802)(328.47499458,197.07111802)
\curveto(329.74432683,197.07111802)(330.67765937,197.59378425)(331.2749922,198.63911669)
\curveto(331.89721389,199.709338)(332.20832474,201.24000336)(332.20832474,203.23111278)
\curveto(332.20832474,207.23822048)(330.93899248,209.24177433)(328.40032798,209.24177433)
\closepath
}
}
{
\newrgbcolor{curcolor}{0 0 0}
\pscustom[linestyle=none,fillstyle=solid,fillcolor=curcolor]
{
\newpath
\moveto(350.83762034,213.72177051)
\curveto(353.57539578,213.72177051)(355.66606067,213.12443769)(357.109615,211.92977204)
\curveto(358.57805819,210.75999526)(359.31227979,208.95555235)(359.31227979,206.51644332)
\lineto(359.31227979,192.92712155)
\lineto(355.42961643,192.92712155)
\lineto(354.34695068,195.68978587)
\lineto(354.19761748,195.68978587)
\curveto(353.32650711,194.59467569)(352.405619,193.79823192)(351.43495316,193.30045457)
\curveto(350.46428732,192.80267721)(349.1327329,192.55378854)(347.4402899,192.55378854)
\curveto(345.62340256,192.55378854)(344.11762606,193.07645476)(342.92296041,194.1217872)
\curveto(341.72829476,195.16711965)(341.13096194,196.79734048)(341.13096194,199.0124497)
\curveto(341.13096194,201.17778119)(341.8900724,202.77066873)(343.40829333,203.7911123)
\curveto(344.92651426,204.81155588)(347.20384565,205.38399984)(350.24028751,205.50844417)
\lineto(353.78695116,205.62044408)
\lineto(353.78695116,206.51644332)
\curveto(353.78695116,207.58666463)(353.50072918,208.37066396)(352.92828522,208.86844131)
\curveto(352.38073013,209.36621867)(351.60917524,209.61510734)(350.61362053,209.61510734)
\curveto(349.61806582,209.61510734)(348.64739998,209.46577414)(347.70162301,209.16710772)
\curveto(346.75584604,208.89333018)(345.81006906,208.54488603)(344.86429209,208.12177528)
\lineto(343.03496031,211.89243874)
\curveto(344.10518163,212.43999383)(345.31229171,212.87554901)(346.65629056,213.19910429)
\curveto(348.00028942,213.54754844)(349.39406601,213.72177051)(350.83762034,213.72177051)
\closepath
\moveto(353.78695116,202.37244684)
\lineto(351.62161967,202.29778024)
\curveto(349.8296212,202.24800251)(348.58517781,201.92444723)(347.88828952,201.3271144)
\curveto(347.19140122,200.72978158)(346.84295707,199.94578224)(346.84295707,198.9751164)
\curveto(346.84295707,198.1288949)(347.09184575,197.51911764)(347.5896231,197.14578463)
\curveto(348.08740046,196.79734048)(348.73451102,196.62311841)(349.53095478,196.62311841)
\curveto(350.72562043,196.62311841)(351.73361957,196.97156255)(352.55495221,197.66845085)
\curveto(353.37628484,198.39022801)(353.78695116,199.39822715)(353.78695116,200.69244827)
\closepath
}
}
{
\newrgbcolor{curcolor}{0 0 0}
\pscustom[linestyle=none,fillstyle=solid,fillcolor=curcolor]
{
\newpath
\moveto(383.09359515,207.97244208)
\curveto(383.09359515,206.8773319)(382.745151,205.94399936)(382.0482627,205.17244446)
\curveto(381.37626327,204.40088956)(380.36826413,203.90311221)(379.02426528,203.6791124)
\lineto(379.02426528,203.52977919)
\curveto(380.44293074,203.35555712)(381.57537422,202.85777976)(382.42159572,202.03644713)
\curveto(383.29270609,201.24000336)(383.72826127,200.23200422)(383.72826127,199.0124497)
\curveto(383.72826127,197.84267292)(383.41715043,196.79734048)(382.79492873,195.87645238)
\curveto(382.19759591,194.95556427)(381.2393745,194.23378711)(379.92026451,193.71112089)
\curveto(378.60115453,193.18845466)(376.87137822,192.92712155)(374.7309356,192.92712155)
\lineto(365.0242772,192.92712155)
\lineto(365.0242772,213.3111042)
\lineto(374.7309356,213.3111042)
\curveto(376.32382313,213.3111042)(377.74248859,213.13688212)(378.98693198,212.78843797)
\curveto(380.25626423,212.46488269)(381.25181894,211.90488317)(381.9735961,211.10843941)
\curveto(382.72026213,210.33688451)(383.09359515,209.29155206)(383.09359515,207.97244208)
\closepath
\moveto(377.45626661,207.52444246)
\curveto(377.45626661,208.76888584)(376.47315634,209.39110753)(374.50693579,209.39110753)
\lineto(370.58693913,209.39110753)
\lineto(370.58693913,205.35911097)
\lineto(373.87226966,205.35911097)
\curveto(375.04204645,205.35911097)(375.92560125,205.52088861)(376.52293407,205.84444389)
\curveto(377.14515577,206.19288804)(377.45626661,206.75288756)(377.45626661,207.52444246)
\closepath
\moveto(377.97893283,199.31111612)
\curveto(377.97893283,200.10755988)(377.65537755,200.68000384)(377.00826699,201.02844799)
\curveto(376.3860453,201.401781)(375.4651572,201.58844751)(374.24560268,201.58844751)
\lineto(370.58693913,201.58844751)
\lineto(370.58693913,196.77245161)
\lineto(374.35760258,196.77245161)
\curveto(375.40293503,196.77245161)(376.26160096,196.95911812)(376.93360039,197.33245114)
\curveto(377.63048869,197.73067302)(377.97893283,198.39022801)(377.97893283,199.31111612)
\closepath
}
}
{
\newrgbcolor{curcolor}{0 0 0}
\pscustom[linestyle=none,fillstyle=solid,fillcolor=curcolor]
{
\newpath
\moveto(409.5629014,213.68443721)
\curveto(411.85267723,213.68443721)(413.70689788,212.78843797)(415.12556333,210.9964395)
\curveto(416.54422879,209.22932989)(417.25356152,206.61599879)(417.25356152,203.15644618)
\curveto(417.25356152,199.6720047)(416.51933993,197.03378472)(415.05089673,195.24178625)
\curveto(413.58245354,193.44978777)(411.70334403,192.55378854)(409.4135682,192.55378854)
\curveto(407.945125,192.55378854)(406.77534822,192.81512165)(405.90423785,193.33778787)
\curveto(405.03312748,193.88534296)(404.32379475,194.49512022)(403.77623966,195.16711965)
\lineto(403.47757325,195.16711965)
\curveto(403.67668419,194.1217872)(403.77623966,193.12623249)(403.77623966,192.18045552)
\lineto(403.77623966,183.96712918)
\lineto(398.21357773,183.96712918)
\lineto(398.21357773,213.3111042)
\lineto(402.73090722,213.3111042)
\lineto(403.51490655,210.66043979)
\lineto(403.77623966,210.66043979)
\curveto(404.32379475,211.48177242)(405.05801635,212.19110515)(405.97890446,212.78843797)
\curveto(406.89979256,213.3857708)(408.09445821,213.68443721)(409.5629014,213.68443721)
\closepath
\moveto(407.77090293,209.24177433)
\curveto(406.3273486,209.24177433)(405.30690503,208.78133028)(404.7095722,207.86044217)
\curveto(404.11223938,206.96444293)(403.80112853,205.60799964)(403.77623966,203.7911123)
\lineto(403.77623966,203.19377948)
\curveto(403.77623966,201.22755893)(404.06246164,199.709338)(404.6349056,198.63911669)
\curveto(405.23223843,197.59378425)(406.30245974,197.07111802)(407.84556953,197.07111802)
\curveto(409.11490179,197.07111802)(410.04823432,197.59378425)(410.64556715,198.63911669)
\curveto(411.26778884,199.709338)(411.57889969,201.24000336)(411.57889969,203.23111278)
\curveto(411.57889969,207.23822048)(410.30956744,209.24177433)(407.77090293,209.24177433)
\closepath
}
}
{
\newrgbcolor{curcolor}{0 0 0}
\pscustom[linestyle=none,fillstyle=solid,fillcolor=curcolor]
{
\newpath
\moveto(430.24552859,213.68443721)
\curveto(433.05797064,213.68443721)(435.2855243,212.87554901)(436.92818957,211.25777261)
\curveto(438.57085483,209.66488508)(439.39218747,207.38755368)(439.39218747,204.42577843)
\lineto(439.39218747,201.73778072)
\lineto(426.25086532,201.73778072)
\curveto(426.30064306,200.16978205)(426.76108711,198.9377831)(427.63219748,198.04178386)
\curveto(428.52819672,197.14578463)(429.76019567,196.69778501)(431.32819433,196.69778501)
\curveto(432.62241546,196.69778501)(433.80463667,196.82222935)(434.87485798,197.07111802)
\curveto(435.96996816,197.34489557)(437.08996721,197.75556189)(438.23485512,198.30311698)
\lineto(438.23485512,194.0097873)
\curveto(437.21441155,193.51200994)(436.15663467,193.15112136)(435.06152449,192.92712155)
\curveto(433.96641431,192.67823288)(432.63485989,192.55378854)(431.06686122,192.55378854)
\curveto(429.02597407,192.55378854)(427.22153117,192.92712155)(425.6535325,193.67378758)
\curveto(424.08553384,194.44534248)(422.85353488,195.5902304)(421.95753565,197.10845133)
\curveto(421.06153641,198.65156112)(420.61353679,200.60533724)(420.61353679,202.96977967)
\curveto(420.61353679,205.3342221)(421.01175867,207.31288708)(421.80820244,208.90577461)
\curveto(422.62953508,210.49866215)(423.76197856,211.6933278)(425.20553288,212.48977156)
\curveto(426.64908721,213.28621533)(428.32908578,213.68443721)(430.24552859,213.68443721)
\closepath
\moveto(430.28286189,209.72710725)
\curveto(429.18775171,209.72710725)(428.29175248,209.3786631)(427.59486418,208.6817748)
\curveto(426.89797589,207.98488651)(426.48730957,206.90222076)(426.36286523,205.43377757)
\lineto(434.16552525,205.43377757)
\curveto(434.14063638,206.65333209)(433.80463667,207.67377566)(433.15752611,208.4951083)
\curveto(432.53530442,209.31644093)(431.57708301,209.72710725)(430.28286189,209.72710725)
\closepath
}
}
{
\newrgbcolor{curcolor}{0 0 0}
\pscustom[linestyle=none,fillstyle=solid,fillcolor=curcolor]
{
\newpath
\moveto(462.24015164,213.3111042)
\lineto(462.24015164,196.99645142)
\lineto(465.22681577,196.99645142)
\lineto(465.22681577,185.60979445)
\lineto(460.22415336,185.60979445)
\lineto(460.22415336,192.92712155)
\lineto(446.52283169,192.92712155)
\lineto(446.52283169,185.60979445)
\lineto(441.52016929,185.60979445)
\lineto(441.52016929,196.99645142)
\lineto(443.23750116,196.99645142)
\curveto(444.1335004,198.36533914)(444.89261086,199.92089338)(445.51483255,201.66311411)
\curveto(446.13705425,203.43022372)(446.6348316,205.2968888)(447.00816461,207.26310935)
\curveto(447.38149763,209.25421876)(447.65527517,211.27021705)(447.82949725,213.3111042)
\closepath
\moveto(456.67748971,209.12977442)
\lineto(452.49615994,209.12977442)
\curveto(452.19749353,206.86488746)(451.78682721,204.71200041)(451.26416099,202.67111326)
\curveto(450.74149477,200.65511497)(450.00727317,198.76356103)(449.0614962,196.99645142)
\lineto(456.67748971,196.99645142)
\closepath
}
}
{
\newrgbcolor{curcolor}{0 0 0}
\pscustom[linestyle=none,fillstyle=solid,fillcolor=curcolor]
{
\newpath
\moveto(477.06146677,213.72177051)
\curveto(479.79924222,213.72177051)(481.88990711,213.12443769)(483.33346143,211.92977204)
\curveto(484.80190463,210.75999526)(485.53612622,208.95555235)(485.53612622,206.51644332)
\lineto(485.53612622,192.92712155)
\lineto(481.65346286,192.92712155)
\lineto(480.57079712,195.68978587)
\lineto(480.42146391,195.68978587)
\curveto(479.55035354,194.59467569)(478.62946544,193.79823192)(477.6587996,193.30045457)
\curveto(476.68813376,192.80267721)(475.35657934,192.55378854)(473.66413633,192.55378854)
\curveto(471.84724899,192.55378854)(470.3414725,193.07645476)(469.14680685,194.1217872)
\curveto(467.9521412,195.16711965)(467.35480837,196.79734048)(467.35480837,199.0124497)
\curveto(467.35480837,201.17778119)(468.11391884,202.77066873)(469.63213977,203.7911123)
\curveto(471.1503607,204.81155588)(473.42769209,205.38399984)(476.46413395,205.50844417)
\lineto(480.01079759,205.62044408)
\lineto(480.01079759,206.51644332)
\curveto(480.01079759,207.58666463)(479.72457562,208.37066396)(479.15213166,208.86844131)
\curveto(478.60457657,209.36621867)(477.83302167,209.61510734)(476.83746696,209.61510734)
\curveto(475.84191226,209.61510734)(474.87124642,209.46577414)(473.92546944,209.16710772)
\curveto(472.97969247,208.89333018)(472.0339155,208.54488603)(471.08813853,208.12177528)
\lineto(469.25880675,211.89243874)
\curveto(470.32902806,212.43999383)(471.53613814,212.87554901)(472.880137,213.19910429)
\curveto(474.22413586,213.54754844)(475.61791245,213.72177051)(477.06146677,213.72177051)
\closepath
\moveto(480.01079759,202.37244684)
\lineto(477.84546611,202.29778024)
\curveto(476.05346763,202.24800251)(474.80902425,201.92444723)(474.11213595,201.3271144)
\curveto(473.41524766,200.72978158)(473.06680351,199.94578224)(473.06680351,198.9751164)
\curveto(473.06680351,198.1288949)(473.31569218,197.51911764)(473.81346954,197.14578463)
\curveto(474.31124689,196.79734048)(474.95835745,196.62311841)(475.75480122,196.62311841)
\curveto(476.94946687,196.62311841)(477.95746601,196.97156255)(478.77879864,197.66845085)
\curveto(479.60013128,198.39022801)(480.01079759,199.39822715)(480.01079759,200.69244827)
\closepath
}
}
{
\newrgbcolor{curcolor}{0 0 0}
\pscustom[linestyle=none,fillstyle=solid,fillcolor=curcolor]
{
\newpath
\moveto(504.61344559,213.3111042)
\lineto(510.73610704,213.3111042)
\lineto(502.67211391,203.52977919)
\lineto(511.44543977,192.92712155)
\lineto(505.13611181,192.92712155)
\lineto(496.81078556,203.26844608)
\lineto(496.81078556,192.92712155)
\lineto(491.24812363,192.92712155)
\lineto(491.24812363,213.3111042)
\lineto(496.81078556,213.3111042)
\lineto(496.81078556,203.41777929)
\closepath
}
}
{
\newrgbcolor{curcolor}{0 0 0}
\pscustom[linestyle=none,fillstyle=solid,fillcolor=curcolor]
{
\newpath
\moveto(531.23205938,209.12977442)
\lineto(524.5493984,209.12977442)
\lineto(524.5493984,192.92712155)
\lineto(518.98673647,192.92712155)
\lineto(518.98673647,209.12977442)
\lineto(512.3040755,209.12977442)
\lineto(512.3040755,213.3111042)
\lineto(531.23205938,213.3111042)
\closepath
}
}
{
\newrgbcolor{curcolor}{0 0 0}
\pscustom[linestyle=none,fillstyle=solid,fillcolor=curcolor]
{
\newpath
\moveto(540.37869669,213.3111042)
\lineto(540.37869669,205.24711106)
\curveto(540.37869669,204.82400031)(540.35380782,204.30133409)(540.30403009,203.6791124)
\curveto(540.27914122,203.05689071)(540.24180792,202.42222458)(540.19203018,201.77511402)
\curveto(540.16714132,201.12800346)(540.12980802,200.54311507)(540.08003028,200.02044885)
\curveto(540.03025254,199.52267149)(539.99291924,199.18667178)(539.96803038,199.0124497)
\lineto(549.37602236,213.3111042)
\lineto(556.05868334,213.3111042)
\lineto(556.05868334,192.92712155)
\lineto(550.68268792,192.92712155)
\lineto(550.68268792,201.06578129)
\curveto(550.68268792,201.71289185)(550.70757679,202.44711345)(550.75735452,203.26844608)
\curveto(550.80713226,204.08977871)(550.85690999,204.84888918)(550.90668773,205.54577748)
\curveto(550.98135433,206.26755464)(551.03113207,206.81510973)(551.05602093,207.18844274)
\lineto(541.68536225,192.92712155)
\lineto(535.00270127,192.92712155)
\lineto(535.00270127,213.3111042)
\closepath
}
}
{
\newrgbcolor{curcolor}{0 0 0}
\pscustom[linestyle=none,fillstyle=solid,fillcolor=curcolor]
{
\newpath
\moveto(573.23196879,213.68443721)
\curveto(575.52174462,213.68443721)(577.37596526,212.78843797)(578.79463072,210.9964395)
\curveto(580.21329618,209.22932989)(580.92262891,206.61599879)(580.92262891,203.15644618)
\curveto(580.92262891,199.6720047)(580.18840731,197.03378472)(578.71996411,195.24178625)
\curveto(577.25152092,193.44978777)(575.37241141,192.55378854)(573.08263558,192.55378854)
\curveto(571.61419239,192.55378854)(570.44441561,192.81512165)(569.57330524,193.33778787)
\curveto(568.70219487,193.88534296)(567.99286214,194.49512022)(567.44530705,195.16711965)
\lineto(567.14664064,195.16711965)
\curveto(567.34575158,194.1217872)(567.44530705,193.12623249)(567.44530705,192.18045552)
\lineto(567.44530705,183.96712918)
\lineto(561.88264512,183.96712918)
\lineto(561.88264512,213.3111042)
\lineto(566.3999746,213.3111042)
\lineto(567.18397394,210.66043979)
\lineto(567.44530705,210.66043979)
\curveto(567.99286214,211.48177242)(568.72708373,212.19110515)(569.64797184,212.78843797)
\curveto(570.56885994,213.3857708)(571.76352559,213.68443721)(573.23196879,213.68443721)
\closepath
\moveto(571.43997031,209.24177433)
\curveto(569.99641599,209.24177433)(568.97597241,208.78133028)(568.37863959,207.86044217)
\curveto(567.78130676,206.96444293)(567.47019592,205.60799964)(567.44530705,203.7911123)
\lineto(567.44530705,203.19377948)
\curveto(567.44530705,201.22755893)(567.73152903,199.709338)(568.30397298,198.63911669)
\curveto(568.90130581,197.59378425)(569.97152712,197.07111802)(571.51463692,197.07111802)
\curveto(572.78396917,197.07111802)(573.71730171,197.59378425)(574.31463453,198.63911669)
\curveto(574.93685622,199.709338)(575.24796707,201.24000336)(575.24796707,203.23111278)
\curveto(575.24796707,207.23822048)(573.97863482,209.24177433)(571.43997031,209.24177433)
\closepath
}
}
{
\newrgbcolor{curcolor}{0 0 0}
\pscustom[linestyle=none,fillstyle=solid,fillcolor=curcolor]
{
\newpath
\moveto(604.03195121,203.15644618)
\curveto(604.03195121,199.77156017)(603.13595197,197.15822906)(601.3439535,195.31645285)
\curveto(599.57684389,193.47467664)(597.16262372,192.55378854)(594.101293,192.55378854)
\curveto(592.20973905,192.55378854)(590.51729605,192.96445485)(589.02396399,193.78578749)
\curveto(587.55552079,194.60712012)(586.39818845,195.80178577)(585.55196694,197.36978444)
\curveto(584.70574544,198.96267197)(584.28263469,200.89155922)(584.28263469,203.15644618)
\curveto(584.28263469,206.54133218)(585.1661895,209.14221886)(586.9332991,210.9591062)
\curveto(588.70040871,212.77599354)(591.12707331,213.68443721)(594.2132929,213.68443721)
\curveto(596.12973572,213.68443721)(597.82217872,213.27377089)(599.29062191,212.45243826)
\curveto(600.75906511,211.63110563)(601.91639745,210.43643998)(602.76261896,208.86844131)
\curveto(603.60884046,207.30044265)(604.03195121,205.39644427)(604.03195121,203.15644618)
\closepath
\moveto(589.95729653,203.15644618)
\curveto(589.95729653,201.14044789)(590.28085181,199.60978253)(590.92796237,198.56445009)
\curveto(591.59996179,197.54400651)(592.68262754,197.03378472)(594.1759596,197.03378472)
\curveto(595.6444028,197.03378472)(596.70217967,197.54400651)(597.34929023,198.56445009)
\curveto(598.02128966,199.60978253)(598.35728937,201.14044789)(598.35728937,203.15644618)
\curveto(598.35728937,205.17244446)(598.02128966,206.67822096)(597.34929023,207.67377566)
\curveto(596.70217967,208.69421924)(595.63195836,209.20444103)(594.1386263,209.20444103)
\curveto(592.67018311,209.20444103)(591.59996179,208.69421924)(590.92796237,207.67377566)
\curveto(590.28085181,206.67822096)(589.95729653,205.17244446)(589.95729653,203.15644618)
\closepath
}
}
{
\newrgbcolor{curcolor}{0 0 0}
\pscustom[linestyle=none,fillstyle=solid,fillcolor=curcolor]
{
\newpath
\moveto(626.69320452,207.97244208)
\curveto(626.69320452,206.8773319)(626.34476037,205.94399936)(625.64787208,205.17244446)
\curveto(624.97587265,204.40088956)(623.96787351,203.90311221)(622.62387465,203.6791124)
\lineto(622.62387465,203.52977919)
\curveto(624.04254011,203.35555712)(625.17498359,202.85777976)(626.02120509,202.03644713)
\curveto(626.89231546,201.24000336)(627.32787065,200.23200422)(627.32787065,199.0124497)
\curveto(627.32787065,197.84267292)(627.0167598,196.79734048)(626.39453811,195.87645238)
\curveto(625.79720528,194.95556427)(624.83898388,194.23378711)(623.51987389,193.71112089)
\curveto(622.2007639,193.18845466)(620.4709876,192.92712155)(618.33054497,192.92712155)
\lineto(608.62388657,192.92712155)
\lineto(608.62388657,213.3111042)
\lineto(618.33054497,213.3111042)
\curveto(619.92343251,213.3111042)(621.34209797,213.13688212)(622.58654135,212.78843797)
\curveto(623.8558736,212.46488269)(624.85142831,211.90488317)(625.57320547,211.10843941)
\curveto(626.3198715,210.33688451)(626.69320452,209.29155206)(626.69320452,207.97244208)
\closepath
\moveto(621.05587599,207.52444246)
\curveto(621.05587599,208.76888584)(620.07276571,209.39110753)(618.10654517,209.39110753)
\lineto(614.1865485,209.39110753)
\lineto(614.1865485,205.35911097)
\lineto(617.47187904,205.35911097)
\curveto(618.64165582,205.35911097)(619.52521062,205.52088861)(620.12254345,205.84444389)
\curveto(620.74476514,206.19288804)(621.05587599,206.75288756)(621.05587599,207.52444246)
\closepath
\moveto(621.57854221,199.31111612)
\curveto(621.57854221,200.10755988)(621.25498693,200.68000384)(620.60787637,201.02844799)
\curveto(619.98565468,201.401781)(619.06476657,201.58844751)(617.84521205,201.58844751)
\lineto(614.1865485,201.58844751)
\lineto(614.1865485,196.77245161)
\lineto(617.95721196,196.77245161)
\curveto(619.0025444,196.77245161)(619.86121034,196.95911812)(620.53320977,197.33245114)
\curveto(621.23009806,197.73067302)(621.57854221,198.39022801)(621.57854221,199.31111612)
\closepath
}
}
{
\newrgbcolor{curcolor}{0 0 0}
\pscustom[linestyle=none,fillstyle=solid,fillcolor=curcolor]
{
\newpath
\moveto(640.46921213,213.72177051)
\curveto(643.20698758,213.72177051)(645.29765247,213.12443769)(646.74120679,211.92977204)
\curveto(648.20964999,210.75999526)(648.94387158,208.95555235)(648.94387158,206.51644332)
\lineto(648.94387158,192.92712155)
\lineto(645.06120822,192.92712155)
\lineto(643.97854248,195.68978587)
\lineto(643.82920927,195.68978587)
\curveto(642.9580989,194.59467569)(642.0372108,193.79823192)(641.06654496,193.30045457)
\curveto(640.09587912,192.80267721)(638.7643247,192.55378854)(637.07188169,192.55378854)
\curveto(635.25499435,192.55378854)(633.74921786,193.07645476)(632.55455221,194.1217872)
\curveto(631.35988656,195.16711965)(630.76255373,196.79734048)(630.76255373,199.0124497)
\curveto(630.76255373,201.17778119)(631.5216642,202.77066873)(633.03988513,203.7911123)
\curveto(634.55810606,204.81155588)(636.83543745,205.38399984)(639.87187931,205.50844417)
\lineto(643.41854296,205.62044408)
\lineto(643.41854296,206.51644332)
\curveto(643.41854296,207.58666463)(643.13232098,208.37066396)(642.55987702,208.86844131)
\curveto(642.01232193,209.36621867)(641.24076703,209.61510734)(640.24521233,209.61510734)
\curveto(639.24965762,209.61510734)(638.27899178,209.46577414)(637.3332148,209.16710772)
\curveto(636.38743783,208.89333018)(635.44166086,208.54488603)(634.49588389,208.12177528)
\lineto(632.66655211,211.89243874)
\curveto(633.73677342,212.43999383)(634.94388351,212.87554901)(636.28788236,213.19910429)
\curveto(637.63188122,213.54754844)(639.02565781,213.72177051)(640.46921213,213.72177051)
\closepath
\moveto(643.41854296,202.37244684)
\lineto(641.25321147,202.29778024)
\curveto(639.46121299,202.24800251)(638.21676961,201.92444723)(637.51988131,201.3271144)
\curveto(636.82299302,200.72978158)(636.47454887,199.94578224)(636.47454887,198.9751164)
\curveto(636.47454887,198.1288949)(636.72343755,197.51911764)(637.2212149,197.14578463)
\curveto(637.71899225,196.79734048)(638.36610281,196.62311841)(639.16254658,196.62311841)
\curveto(640.35721223,196.62311841)(641.36521137,196.97156255)(642.18654401,197.66845085)
\curveto(643.00787664,198.39022801)(643.41854296,199.39822715)(643.41854296,200.69244827)
\closepath
}
}
{
\newrgbcolor{curcolor}{0 0 0}
\pscustom[linestyle=none,fillstyle=solid,fillcolor=curcolor]
{
\newpath
\moveto(660.21853092,213.3111042)
\lineto(660.21853092,205.47111087)
\lineto(667.98385765,205.47111087)
\lineto(667.98385765,213.3111042)
\lineto(673.54651958,213.3111042)
\lineto(673.54651958,192.92712155)
\lineto(667.98385765,192.92712155)
\lineto(667.98385765,201.3271144)
\lineto(660.21853092,201.3271144)
\lineto(660.21853092,192.92712155)
\lineto(654.65586899,192.92712155)
\lineto(654.65586899,213.3111042)
\closepath
}
}
{
\newrgbcolor{curcolor}{0 0 0}
\pscustom[linestyle=none,fillstyle=solid,fillcolor=curcolor]
{
\newpath
\moveto(684.74652506,213.3111042)
\lineto(684.74652506,205.24711106)
\curveto(684.74652506,204.82400031)(684.72163619,204.30133409)(684.67185846,203.6791124)
\curveto(684.64696959,203.05689071)(684.60963629,202.42222458)(684.55985855,201.77511402)
\curveto(684.53496969,201.12800346)(684.49763638,200.54311507)(684.44785865,200.02044885)
\curveto(684.39808091,199.52267149)(684.36074761,199.18667178)(684.33585874,199.0124497)
\lineto(693.74385073,213.3111042)
\lineto(700.42651171,213.3111042)
\lineto(700.42651171,192.92712155)
\lineto(695.05051629,192.92712155)
\lineto(695.05051629,201.06578129)
\curveto(695.05051629,201.71289185)(695.07540516,202.44711345)(695.12518289,203.26844608)
\curveto(695.17496063,204.08977871)(695.22473836,204.84888918)(695.2745161,205.54577748)
\curveto(695.3491827,206.26755464)(695.39896044,206.81510973)(695.4238493,207.18844274)
\lineto(686.05319062,192.92712155)
\lineto(679.37052964,192.92712155)
\lineto(679.37052964,213.3111042)
\closepath
}
}
{
\newrgbcolor{curcolor}{0 0 0}
\pscustom[linestyle=none,fillstyle=solid,fillcolor=curcolor]
{
\newpath
\moveto(709.34913752,192.92712155)
\lineto(703.33847597,192.92712155)
\lineto(708.82647129,200.99111469)
\curveto(707.78113885,201.41422544)(706.84780631,202.0986693)(706.02647368,203.04444627)
\curveto(705.23002991,204.01511211)(704.83180803,205.3342221)(704.83180803,207.00177624)
\curveto(704.83180803,209.04266339)(705.60336293,210.59821762)(707.14647272,211.66843893)
\curveto(708.68958252,212.76354911)(710.6682475,213.3111042)(713.08246767,213.3111042)
\lineto(722.56512626,213.3111042)
\lineto(722.56512626,192.92712155)
\lineto(717.00246433,192.92712155)
\lineto(717.00246433,200.50578177)
\lineto(713.9411336,200.50578177)
\closepath
\moveto(710.28247005,206.96444293)
\curveto(710.28247005,206.11822143)(710.61846977,205.446222)(711.2904692,204.94844465)
\curveto(711.96246862,204.47555616)(712.83357899,204.23911192)(713.9038003,204.23911192)
\lineto(717.00246433,204.23911192)
\lineto(717.00246433,209.39110753)
\lineto(713.19446757,209.39110753)
\curveto(712.19891287,209.39110753)(711.46469127,209.14221886)(710.99180278,208.6444415)
\curveto(710.5189143,208.17155302)(710.28247005,207.61155349)(710.28247005,206.96444293)
\closepath
}
}
{
\newrgbcolor{curcolor}{0 0 0}
\pscustom[linestyle=none,fillstyle=solid,fillcolor=curcolor]
{
\newpath
\moveto(756.42646146,213.3111042)
\lineto(756.42646146,196.99645142)
\lineto(759.41312558,196.99645142)
\lineto(759.41312558,185.60979445)
\lineto(754.41046318,185.60979445)
\lineto(754.41046318,192.92712155)
\lineto(740.70914151,192.92712155)
\lineto(740.70914151,185.60979445)
\lineto(735.7064791,185.60979445)
\lineto(735.7064791,196.99645142)
\lineto(737.42381097,196.99645142)
\curveto(738.31981021,198.36533914)(739.07892067,199.92089338)(739.70114237,201.66311411)
\curveto(740.32336406,203.43022372)(740.82114141,205.2968888)(741.19447443,207.26310935)
\curveto(741.56780744,209.25421876)(741.84158499,211.27021705)(742.01580706,213.3111042)
\closepath
\moveto(750.86379953,209.12977442)
\lineto(746.68246976,209.12977442)
\curveto(746.38380334,206.86488746)(745.97313703,204.71200041)(745.45047081,202.67111326)
\curveto(744.92780458,200.65511497)(744.19358299,198.76356103)(743.24780601,196.99645142)
\lineto(750.86379953,196.99645142)
\closepath
}
}
{
\newrgbcolor{curcolor}{0 0 0}
\pscustom[linestyle=none,fillstyle=solid,fillcolor=curcolor]
{
\newpath
\moveto(781.40243461,203.15644618)
\curveto(781.40243461,199.77156017)(780.50643537,197.15822906)(778.7144369,195.31645285)
\curveto(776.94732729,193.47467664)(774.53310712,192.55378854)(771.4717764,192.55378854)
\curveto(769.58022245,192.55378854)(767.88777945,192.96445485)(766.39444739,193.78578749)
\curveto(764.92600419,194.60712012)(763.76867184,195.80178577)(762.92245034,197.36978444)
\curveto(762.07622884,198.96267197)(761.65311809,200.89155922)(761.65311809,203.15644618)
\curveto(761.65311809,206.54133218)(762.53667289,209.14221886)(764.3037825,210.9591062)
\curveto(766.07089211,212.77599354)(768.49755671,213.68443721)(771.5837763,213.68443721)
\curveto(773.50021911,213.68443721)(775.19266212,213.27377089)(776.66110531,212.45243826)
\curveto(778.12954851,211.63110563)(779.28688085,210.43643998)(780.13310235,208.86844131)
\curveto(780.97932386,207.30044265)(781.40243461,205.39644427)(781.40243461,203.15644618)
\closepath
\moveto(767.32777993,203.15644618)
\curveto(767.32777993,201.14044789)(767.65133521,199.60978253)(768.29844577,198.56445009)
\curveto(768.97044519,197.54400651)(770.05311094,197.03378472)(771.546443,197.03378472)
\curveto(773.01488619,197.03378472)(774.07266307,197.54400651)(774.71977363,198.56445009)
\curveto(775.39177306,199.60978253)(775.72777277,201.14044789)(775.72777277,203.15644618)
\curveto(775.72777277,205.17244446)(775.39177306,206.67822096)(774.71977363,207.67377566)
\curveto(774.07266307,208.69421924)(773.00244176,209.20444103)(771.5091097,209.20444103)
\curveto(770.0406665,209.20444103)(768.97044519,208.69421924)(768.29844577,207.67377566)
\curveto(767.65133521,206.67822096)(767.32777993,205.17244446)(767.32777993,203.15644618)
\closepath
}
}
{
\newrgbcolor{curcolor}{0 0 0}
\pscustom[linestyle=none,fillstyle=solid,fillcolor=curcolor]
{
\newpath
\moveto(794.28242395,192.55378854)
\curveto(791.24598209,192.55378854)(788.89398409,193.38756561)(787.22642996,195.05511974)
\curveto(785.58376469,196.72267388)(784.76243206,199.37333829)(784.76243206,203.00711297)
\curveto(784.76243206,205.49599974)(785.18554281,207.52444246)(786.03176431,209.09244112)
\curveto(786.87798581,210.66043979)(788.04776259,211.81777213)(789.54109465,212.56443817)
\curveto(791.05931558,213.3111042)(792.80153632,213.68443721)(794.76775687,213.68443721)
\curveto(796.16153346,213.68443721)(797.36864354,213.54754844)(798.38908712,213.27377089)
\curveto(799.43441956,212.99999335)(800.34286323,212.67643807)(801.11441813,212.30310505)
\lineto(799.47175286,208.00977538)
\curveto(798.60064249,208.35821952)(797.77930986,208.6444415)(797.00775496,208.86844131)
\curveto(796.26108893,209.09244112)(795.5144229,209.20444103)(794.76775687,209.20444103)
\curveto(791.88064822,209.20444103)(790.43709389,207.15110944)(790.43709389,203.04444627)
\curveto(790.43709389,201.00355912)(790.81042691,199.49778263)(791.55709294,198.52711678)
\curveto(792.32864784,197.55645094)(793.39886915,197.07111802)(794.76775687,197.07111802)
\curveto(795.93753365,197.07111802)(796.97042166,197.22045123)(797.8664209,197.51911764)
\curveto(798.76242013,197.84267292)(799.6335305,198.27822811)(800.47975201,198.8257832)
\lineto(800.47975201,194.0844539)
\curveto(799.6335305,193.53689881)(798.73753127,193.15112136)(797.79175429,192.92712155)
\curveto(796.87086619,192.67823288)(795.70108941,192.55378854)(794.28242395,192.55378854)
\closepath
}
}
{
\newrgbcolor{curcolor}{0 0 0}
\pscustom[linestyle=none,fillstyle=solid,fillcolor=curcolor]
{
\newpath
\moveto(818.54902298,213.3111042)
\lineto(824.67168443,213.3111042)
\lineto(816.6076913,203.52977919)
\lineto(825.38101716,192.92712155)
\lineto(819.0716892,192.92712155)
\lineto(810.74636296,203.26844608)
\lineto(810.74636296,192.92712155)
\lineto(805.18370103,192.92712155)
\lineto(805.18370103,213.3111042)
\lineto(810.74636296,213.3111042)
\lineto(810.74636296,203.41777929)
\closepath
}
}
{
\newrgbcolor{curcolor}{0 0 0}
\pscustom[linestyle=none,fillstyle=solid,fillcolor=curcolor]
{
\newpath
\moveto(833.66901041,213.3111042)
\lineto(833.66901041,205.24711106)
\curveto(833.66901041,204.82400031)(833.64412155,204.30133409)(833.59434381,203.6791124)
\curveto(833.56945494,203.05689071)(833.53212164,202.42222458)(833.48234391,201.77511402)
\curveto(833.45745504,201.12800346)(833.42012174,200.54311507)(833.370344,200.02044885)
\curveto(833.32056627,199.52267149)(833.28323296,199.18667178)(833.2583441,199.0124497)
\lineto(842.66633609,213.3111042)
\lineto(849.34899706,213.3111042)
\lineto(849.34899706,192.92712155)
\lineto(843.97300164,192.92712155)
\lineto(843.97300164,201.06578129)
\curveto(843.97300164,201.71289185)(843.99789051,202.44711345)(844.04766824,203.26844608)
\curveto(844.09744598,204.08977871)(844.14722371,204.84888918)(844.19700145,205.54577748)
\curveto(844.27166805,206.26755464)(844.32144579,206.81510973)(844.34633465,207.18844274)
\lineto(834.97567597,192.92712155)
\lineto(828.29301499,192.92712155)
\lineto(828.29301499,213.3111042)
\closepath
}
}
\end{pspicture}
}
		\caption{Ограниченная система прав}
		\label{fig:img2}
	\end{center}
\end{figure}
