В условиях пандемии учебные заведения ощутили нехватку программного обеспечения,
направленного на проведение онлайн-занятий. Обычно такие занятия проводятся с
применением системы программ: какой-либо мессенджер с чатом, поддержкой голосовой
связи и демонстрации экрана, и вместе с ним инструмент, удобный для проведения
занятия -- к примеру, интерактивная доска. Было бы намного удобнее, чтобы все это
располагалось в одной программе. Именно так возникла идея создания
проекта <<Geometry Room>>: создать единый сервис для проведения онлайн-занятий.
Так как объединить все инструменты для всех предметов, преподаваемых в
учебных заведениях, просто невозможно, было принято решение взять одну
из самых сложных сфер предметов с точки зрения онлайн-преподавания -- геометрию.