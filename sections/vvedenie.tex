В условиях пандемии и дистанционного обучения появилась необходимость в выборе подходящего
программного обеспечения для проведения онлайн-занятий. Это вызвало определенные трудности:
не для каждой дисциплины можно было подобное подобрать. К примеру, в общем доступе есть большое
множество сервисов с встроенной интерактивной доской, чатом, голосовой связью, функцией демонстрации экрана.
Большинство из них достаточно удобны для проведения онлайн-занятий по некоторым дисциплинам,
особенно если они лекционные, где, по большей части, не требуется индивидуальное взаимодействие
преподавателя с учеником. Неудобства начинаются на практических занятиях, в частности
по математическим дисциплинам: алгебра, теория чисел, планиметрия, стереометрия и т.д. На таких
занятиях, помимо всего прочего, требуется индивидуальное взаимодействие с каждым учеником.
Трудности заключаются в том, что для этих целей нет специализированного сервиса, объединяющего
все необходимое. Под необходимым подразумеваются в первую очередь средства общения (чат и голосовая связь)
и интерактивные инструменты для работы с математическими и геометрическими объектами. Популярные
сервисы не обладают такой комбинацией. К примеру, известный от компании Google продукт Jamboard, который,
по сути своей, является исключительно интеративной доской без каких-либо дополнительных функций.

\begin{figure}[h]
	\begin{center}
		\includegraphics[width=\linewidth]{vvedenie/img1.png}
		\caption{Jamboard}
		\label{vvedenie_img1}
	\end{center}
\end{figure}

Кроме этого, можно рассмотреть сочетание нескольких программ: мессенджера и специализированного инструмента.
В качестве такого сочетания можно взять сервис Discord в качестве средства общения и инструмент для решения
математических задач Mathway. 

\begin{figure}[h]
	\begin{center}
		\includegraphics[width=\linewidth]{vvedenie/img2.png}
		\caption{Инструмент для решения математических задач Mathway}
		\label{vvedenie_img2}
	\end{center}
\end{figure}

Связка получается достаточно удобной, но только для проведения лекционных занятий. У нее есть существенный
минус: в среде Mathway невозможно работать группой людей. К примеру, преподаватель попросил ученика
построить график функции. Для выполнения этого задания ученику нужно включить в сервисе Discord
демонстрацию экрана, после этого переключиться на Mathway и соответственно построить требуемый график функции.
Ситуация станет намного сложнее, если ученику придется использовать наработки преподавателя, т.е. появятся
дополнительные действия по их копированию.

Таким образом, вопрос поиска специализированного программного обеспечения остается открытым. Именно поэтому
возникла идея создания многопользовательского интерактивного сервиса, который бы объединил средство общения и
инструмент для работы с математическими и геометрическими объектами. Таким сервисом стал проект «Geometry Room».

\begin{figure}[h]
	\begin{center}
		\includegraphics[width=\linewidth]{vvedenie/img3.png}
		\caption{Страница входа в «Geometry Room>>}
		\label{vvedenie_img2}
	\end{center}
\end{figure}
